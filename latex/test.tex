\documentclass[a5paper, twoside]{book}


%%
% Book metadata
\title{Nissepedia ett urval}
\author{Nissepediaklubben}
%\publisher{Videnskabens Forlag}


%%
% For nicely typeset tabular material
\usepackage{booktabs}

\usepackage[swedish]{babel}
\usepackage[T1]{fontenc}
\usepackage[utf8]{inputenc}
\usepackage{color}

\definecolor{light-gray}{gray}{0.5}

%\usepackage{csquotes}
%\MakeOuterQuote{"}


% Inserts a blank page
\newcommand{\blankpage}{\newpage\hbox{}\thispagestyle{empty}\newpage}

\newcommand{\quotetext}[0]{%
    ``#1''
}

\usepackage{units}

% Typesets the font size, leading, and measure in the form of 10/12x26 pc.
\newcommand{\measure}[3]{#1/#2$\times$\unit[#3]{pc}}

\providecommand\phantomsection{}


% Generates the index
%\usepackage{makeidx}
%\usepackage{multind}
%\usepackage[columns=3,font=footnotesize]{idxlayout}

%\usepackage{imakeidx}
%\usepackage[splitindex]{imakeidx}[nonewpage]

\usepackage[nonewpage,splitindex,makeindex]{imakeidx}



%\makeindex[title=Det hele,columns=1]
%\makeindex[name=saker,title=Saker,columns=1]
%\makeindex[name=andrasaker,title=Andra Saker,columns=1]
%
%%AUTOGENERATED
%\makeindex[name=television, title=television, columns=1]
%\makeindex[name=dialekter, title=dialekter, columns=1]
%\makeindex[name=musik, title=musik, columns=1]
%\makeindex[name=litteratur, title=litteratur, columns=1]
%\makeindex[name=djurriket, title=djurriket, columns=1]
%\makeindex[name=politik_och_debatt, title=politik och debatt, columns=1]
%\makeindex[name=dialekter, title=dialekter, columns=1]
%\makeindex[name=musik, title=musik, columns=1]
%\makeindex[name=litteratur, title=litteratur, columns=1]
%\makeindex[name=bubologi, title=bubologi, columns=1]
%\makeindex[name=overlevnadsknep, title=overlevnadsknep, columns=1]
%\makeindex[name=konst_och_kultur, title=konst och kultur, columns=1]



%



\begin{document}
\small
\pagenumbering{arabic}
\raggedbottom

% Front matter
\frontmatter

% r.1 blank page
\blankpage


% r.3 full title page

% v.4 copyright page
\newpage

~\vfill
\thispagestyle{empty}
\setlength{\parindent}{0pt}
\setlength{\parskip}{\baselineskip}

\par\textsc{Utegiven av Videnskabens Forlag}

\par\textsc{www.nissepedia.org}

\par Här kan man skriva in licens eller typ sånna saker

\par\textit{Första utgåvan, Januari 2014}


% r.7 dedication
\cleardoublepage

Här kan man ha en dedikation

% r.9 introduction
\cleardoublepage
\chapter*{Introduktion}
\index{Introduktion}

Här kan man ha en introduktion
\index[saker]{Här markerar jag en sak}


\chapter*{Förord}
\index{Förord}
Här kommer det ett litet förord.


%%
% Start the main matter (normal chapters)
\mainmatter

\noindent
\pagestyle{myheadings}
\markboth{Nissepedia}{Ett urval}

\chapter*{Ett urval}
\newpage

\small{
\textbf{"göra-allt-för-att-slippa-lumpen"-historier}
\label{86f09c4b94bc85d3c0ffc70ae67b8361}
 Dessa historier handlar om män som verkligen, verkligen, verkligen inte velat göra lumpen. Historien utspelar sig på mönstringen och kan låta så här:

 \quotetext{Min kusins \textsc{(se kusin s.~\pageref{f7f20d5744925e2e72e5524035a162be})} polare rullade in sig i en matta och sa att han var en varmkorv för att få frisedel \textsc{(s.~\pageref{ec0e47809187866739cd1a19e3d1ed37})}.}

 Det finns också historier med en mer sedelärande tvist, t.ex. denna.

 \quotetext{Min kusins \textsc{(se kusin s.~\pageref{f7f20d5744925e2e72e5524035a162be})} polare sa att han var från en annan planet på mönstringen, så han förlorade körkortet.}

}

\small{
\textbf{...men det får man väl inte säga i det här jävla landet!}
\label{ff3b0ddb9108e1ec17e77bd33a02354e}
 ...MDFMVISIDHJL! är en bisats som uppstod samtidigt som kommentarsfält på webbtidningar. Foliehattsbärande timbroentusiaster samlade sig i dessa fält så snart de fick reda  på deras uppkomst. \textit{\quotetext{En plats där jag får spy ut hat och klaga på allt jag känner mig hotad av? utmärkt!}}, lär de ha tänkt, de arslena. I Internets mörkaste avkrok, Flashback, fann våghalsiga nätarkeologer ett manifest, författat av en anonym cyberfascist. Efter en noga granskning genomförd av etnologer och historiker vid Umeå Universitet har man sedemera fastslagit att nedanstående text är den första gången MDFMVISIDHJL! nämns i text.

 HEAD3: Manifestet
 Lyss opp, brunblåa kamrater! Den genuskulturmarxistiska kryptojudiskaochsamtidigtislamistiska maffians grepp om Sverige \textsc{(s.~\pageref{b1999637949ed135b2ca03f3a38460cc})} hårdnar. Sedan Gudrun Schyman, Kurdo Baksi och Mona Sahlin könsomskurit tre pursvenska femåriga flickor utan bedövning (för källa, se info14) har Sverige låsts i en full multikultinelson \textsc{(se multikulti s.~\pageref{25eea9148080d30d384ce1c1277ef126})}. Sedan Peter Wolodarski, Robert Aschberg och Göran Greider virat på sig turbaner och sedan gjort gemensam sak i att kväva det fria ordet genom att skita det i munnen. Sedan Liv Strömqvist, Teppas och Annika Lantz , höga på khat, dragit fram en snara och hängt åsiktsfriheten från en radiomast. Sedan de händelserna inträffat finns det en mängd saker man inte får säga i det här jävla landet. Vill du diskret påtala att familjen Bonnier är judar och att det är lite intressant hur judar alltid äger media? Får inte!
 Vill du strö ordet neger kring dig likt verbalt strössel på allas tillvarokaka? Får inte!
 Vill du skryta för dina polare om hur du tafsade på en brud bara på skoj? Får inte!
 Vill du klaga på hur turkar alltid tafsar på brudar? Får inte!
 Vill du ifrågasätta hur många judar som egentligen dog i förintelsen? Får inte!
 Vill du skämta om att tjejer är dummare än killar men mena det litegrann bara? Får inte!
 Vill du i förtroende skriva på din blogg att det är äckligt med bögar? Får inte!
 Vill du i smyg viska att han där den svartmusikga på jobbet luktar lite främmande? Får inte!

 Efter att alla dessa saker blivit bannlysta av den genuskulturmarxistiska kryptojudiskaochsamtidigtislamistiska maffian, vad i helvete ska man då tala om i det här jävla landet?!
 Får du säga att en tjej är tjock? Vem vet i det här jävla landet?
 Får du säga att en snubbe är ful? Vem vet i det här jävla landet?
 Får du be om att få mjölken skickad till dig? Vem vet i det här jävla landet?

 Res er, ni förtryckta medelålders vita män med rätt bra inkomst! Res er, ni svettiga våldtäktsbenägna innebandykillar! Och säg ifrån, alla ni Kamprads \textsc{(se Ingvar Kamprad s.~\pageref{c5f2e9ee9a39f83c39079dbcf01d8809})} därute! Våga säga vad man inte får säga i det här jävla landet!

}

\small{
\textbf{13}
\label{c51ce410c124a10e0db5e4b97fc2af39}
 är benämningen på den fasta nyckel \textsc{(se fasta nycklar s.~\pageref{ad577d76d7747bfd314d442197fc8587})} som används för att justera muttrar på alla maskiner med lite självrespekt. Kräver din maskin mindre nycklar än så har du med all säkerhet blivit lurad av en storfräsare \textsc{(s.~\pageref{4db17005692cd83e3e946a1311b81ed0})} som skrattar gott när hen tömmer sin hästhandlarplånbok \textsc{(s.~\pageref{2f8fbda5296f2f6cab04d88082ed9015})} framför en drägglande bankir. Ställ in skräpet \textsc{(se trivselskrot s.~\pageref{6235563333e8dc26c9fc54e9e70c85ed})} bland dom andra trasiga plastleksakerna i garaget och skaffa något rejält istället!

 HEAD2: Trivia
 Troligheten att just denna fasta nyckel \textsc{(se fasta nycklar s.~\pageref{ad577d76d7747bfd314d442197fc8587})} är nednött till obrukbarhet är 100\% högre än för de andra nycklarna i ditt set.

}

\small{
\textbf{2038}
\label{2557911c1bf75c2b643afb4ecbfc8ec2}
 HEAD2: Milleniumbuggen
 I slutet på 1900-talet började det tisslas och tasslas kring ett tekniskt problem som troligtvis skulle uppstå kring all digital teknik vi hade introducerat i våra liv. Det var om teknikprylen hade någon slags inbyggd tidsfunktion som problemet tänktes kunna uppstå. För det var när klockan slog över från 1999/12/31 - 23:59:59 till 2000/01/01 - 00:00:00 som teknik världen över skulle börja bete sig konstigt. Kanske skulle den digitala tidtavlan vid busstationen helt plötsligt visa 1900/01/01 - 00:00:00, märkliga tecken eller möjligtvis slockna helt? Vad kunde hända med känsligare utrustning? Innehåller pacemakers möjligen nån klocka som kanske slutar fungera och dräper stackars Signhild i nåt som skulle kunna likna hjärtattack orsakad av milleniumskiftets nyårsfyverkerier?

 När denna rädsla för den felande tekniken tog sitt starkaste uttryck talade man om att det USA finns en centraldator som hanterar alla missilsstationer världen över som kanske, kanske skulle råka ut för denna tekniska miss och förorsaka jordens undergång!
 Idén kring denna centraldator känns helt befängd - den har troligtvis hämtats från filmen War Games \textsc{(s.~\pageref{0b505d579ff1c7e9c1b25ccc867cfb62})}, men man vet aldrig - det finns siffror som pekar på att det spenderades 300 miljarder dollar på att förbereda sig för och reparera efter milleniumbuggen.

 HEAD2: 2038-buggen
 \textit{Den 19e januari 2038 klockan 03:14:07 kommer en liknande datorbugg att uppstå, men den här gången är det på riktigt.}

 I våra framtida \quotetext{smarta hem} kan det hända att den av en centraldator styrda bakgrundsmusiken spelas i Piff och Puff-tempo, att våra Hooverboards tar oss på en lodrätt färd och sedan släpper av oss när vi är ett par meter ovanför marken. Och framförallt, alla missiler kommer att avfyras mellanm kontinenterna och SVT kommer att avbryta sina nattliga sändningar av syskonkanalens SVT24s tysta textbaserade nyhetsändningar för att visa klimaxscenen från filmen Hardware \textsc{(s.~\pageref{3ca14c518d1bf901acc339e7c9cd6d7f})}, bara som av en slump, medans bomberna börjar slå ner världen världen över. Den tekniska förklaringen för varför det här kommer uppstå följer nedan

 Det är om ungefär det här scenariot som Discharge \textsc{(s.~\pageref{7084c38f1708430f138336428e4ac7cb})} försöker förmedla i de flesta av sina låtar.

 HEAD4: Den tekniska biten
 I väldigt många datorsystem lagras datum och tid som antalet sekunder som har passerat sedan Torsdag den 1 Januari 1970. Dagens datum (då artieln skrivs) 2012/01/06 12:24:00 representeras som 1325852640 sekunder sedan 1970/01/01. Problemet uppstår när man lagrar dom här sekunderna i datorn. Dom lagras som en integer på 32 bitar (ettor eller nollor). Det högsta talet som en Integer på 32 bitar, en serie på 32 ettor och nollor (t.ex. klassikern 00111011101010110010111110101111) kan anta är 2147483647 och det motsvarar den tidigare nämda tidpunkten 2038/01/19 03:14:07. Efter 2147483647 slår talet om till -2147483648, det vill säga 1901/12/13 20:45:52 och det är \textbf{mycket allvarligt!}

}

\small{
\textbf{35007}
\label{c7d7a54da0f35a8abde8961b6897abe1}
 (utläses uppochned, \quotetext{Loose}) var ett holländskt rymdrockband som bland annat gjorde två temaskivor: En om havet och en om matematik. De gjorde också en tia \textsc{(s.~\pageref{e7292d5ba58672ce7f6fc3c0b646ab63})} om månlandningen.
 HEAD2: Discografi (urval)
 \begin{itemize}
 \item Sea of Tranquility 10" (2001)
 \item Liquid (2002)
 \item Phase V (2005)
 \end{itemize}

}

\small{
\textbf{50 million piece of shit}
\label{20dea3d3f625b865ae7fd554c02c6936}
 Brittiska tidningars smeknamn på fotbollsspelaren Fernando Torres efter att denne sålts till Chelsea FC och bara gjorde två betydelselösa mål under första säsongen.

}

\small{
\textbf{80talet}
\label{fef6822ae867bc17fd5b4761ca145293}
 var den tid under sent 1900tal som många gillade allra bäst.
 Saker var STORA och komponenter fick kosta pengar.
 Elektronik, telefoner, bilar, högtalare, gräsklippare mm tillverkades för att hålla i många många år.

 Datorer på den här tiden var däremot inte så snabba som dagens, men det gjorde ingenting för på den här tiden hade man sällan bråttom.
 Uttryck som: fasen vilken tid det tar, var ännu inte uppfunna.

 Anledningen till att saker var stora var förmodligen att man verkligen skulle känna att man ägde nåt och att det märktes väl vad man ägde och inte, sedan bodde man i större hus och lägenheter på den här tiden så man hade mer plats till allt.

 Musiken som producerades på 80talet var förebilden till all dagens musik.

 104\% av Trance och Dancemusik innehåller snuttar och samplingar från kända låtar från 80talet.

 Vhs-spelaren (även kallad video) var en populär mediaspelare i var mans vardagsrum.

 Framför allt så var bensinpriset lägre under 80talet, det var därför man inte hade så bråttom att komma snabbare in på 2000talet då mycket blev sämre.

 Alla kvinnor och män såg bättre ut på 80talet, visserligen har mycket sexigare kläder kommit på senare tid men detta för att kompensera den allt fulare utvecklingen på det kvinnliga och manliga könet.

 HEAD3: Summering

 Har du sett 80talet så har du sett allt som är värt att se.

 Men har du sett 70talet \textsc{(s.~\pageref{5f334093f378400a9b73d7c365b899a5})} så behöver du inte se 80talet.

}

\small{
\textbf{A-hootin' and a-hollerin'}
\label{1928c39ea0f58992a3e5f53d143a23ff}
 är ett annat uttryck för \quotetext{tjo och tjim}. Uttrycket används som i följande meningar.
 \begin{itemize}
 \item Det var väldigt kul på krogen i fredags \textsc{(se fredag s.~\pageref{80d41f1e0b14eacb229eea9618632e88})}. Det var en väldig massa a-hootin' and a-hollerin'.
 \item Gunborg \textsc{(s.~\pageref{9e29dc34382963ae7d76a742e98637a4})} berättade om sin semester på festen, men jag kunde inte höra vad hon sa på grund av allt a-hootin' and a-hollerin'.
 \end{itemize}

}

\small{
\textbf{A-traktor}
\label{68eb4e0240edaaca3face5a1ee84e9ac}
 Innan man skaffar dieselbil med lastgaller \textsc{(s.~\pageref{73b1f975c67393304ff101482965163c})} kan man skaffa en a-traktor.
 Alternativt skaffar man senare i livet en A-traktorreggad Scania 112 för snöröjningens skull.
 Det är inte bara jävligt coolt med a-traktor, man får numera köra den redan som 15-åring och man kan då skjutsa sina kompisar till grannbyn när det vankas åkerdisco \textsc{(s.~\pageref{6248d2a43b234b98de8b2beb2fe95ffc})}. Skatten är även förmånlig.
 Man skulle lätt kunna tro att ett så här briljant fordon vore vanligast bland mer utvecklade folkslag, men statistiken visar att Västergötland är det A-traktortätaste länet med hela 1950 stycken inregistrerade.
 Det finns några eviga frågor \textsc{(s.~\pageref{3bd505f805d94787ec0cc431648a7826})} som gäller A-traktorer:
 \begin{itemize}
 \item Dubbla växellådor eller AGA-spärr?
 \item Flak eller lucka?
 \item Porrljus eller inte?
 \item Lägga pengar på stereon eller Jokkmokkstrim?
 \end{itemize}

 Den som finner svar på dessa frågor och är utrustad med svets och ett tålamod i klass med Dalai Llama kan framåt vårkanten rulla ut på vägarna.

}

\small{
\textbf{A-wop-bop-a-loo-bop-a-lop-bam-boom}
\label{fe31d426b36075592465b90605764340}
 är den fetaste raden i Little Richards låt Tutti frutti. Näst efter den kommer \quotetext{tutti frutti, oh Rudy}.

}

\small{
\textbf{Abdera}
\label{af6265e74ffc27627369e11195c4a675}
 är en ort i Grekland som i dagsläget har ungefär 4000 invånare. Redan under antiken \textsc{(se de gamla grekerna s.~\pageref{4a5fb3d6ce79b5ff43b33f8f7d843672})} beskrevs luften där som så dålig att man blev dum av att inandas den. Detta gav senare upphov till adjektivet \textit{abderitisk}, som är den etymologiska källan till det idag flitigt använda ”idiot”. Man förstår att det inte bor så många kvar där.
 (Den som vill investera i stenbaserad arkitektur med havsläge kontaktar Prof. Etienne's Aeolian Nights \textsc{(se Användare: prof. Etienne s.~\pageref{a9878d2280e5a39becac8f73d113df91})}.)

}

\small{
\textbf{Abu Garcia}
\label{ebb8e709f4430ad487471fd1acdf28e2}
 är ett företag som sysslar med försäljning av fiskeartiklar. Tidigare var det två företag, Abu och Garcia. Abu var svenskt och är en förkortning för Aktiebolaget Urmakarna. I begynnelsen pysslade man enbart med urmakeri, men arbetarnas gedigna kunskaper i finmekanik visade sig på 1930-talet även passa utmärkt till att knyta fast fiskelina på en bit klarlackad bambu. Garcia var amerikanskt och är inte förkortning för någonting. Varje fiskeredskapsaffär \textsc{(s.~\pageref{1b1aa77debacc344b3c1342e51abfb55})} av rang erbjuder produkter från Abu Garcia.
 HEAD2: andra användningar
 Abu Garcia! används ofta i Västerbotten \textsc{(s.~\pageref{d4b008c5143dcffb6b8c35f3876c2a19})} som \textit{iAy caramba!} används i Mexiko, alltså som en interjektion.

}

\small{
\textbf{AC/DC-gitarr}
\label{c688c3a81724e01058f2d15116b26aa9}
 En AC/DC-gitarr är en gitarr med vinröd kropp (ibland svart) och svart huvud som är extremt rå. Till formen påminner kroppen lite om eldflammor, vilket bara det är extremt rått. Dessutom är det ingen vanlig gitarr utan en elgitarr; också rått som fan. Och så figurerar den som mordvapen på omslaget till AC/DCs liveskiva \textsc{(se tolva s.~\pageref{75e2490604087d3d303b09a98803a16b})} \textit{If You Want Blood You've Got It}, så case closed. AC/DC-gitarren saluförs av den amerikanska tillverkaren Gibson, som även bygger oråa instrument såsom mandoliner och basgitarrer. Vill man höra äkta AC/DC-gitarr kan man med fördel lyssna på \textit{Manglar som ägg \textsc{(s.~\pageref{7b1e91fdfd952485ddd3bc6ef4e40b3c})}} eller \textit{Bonfire \textsc{(s.~\pageref{b0759e17c7cc70d7522a6b63a05c914e})}}.

 HEAD2: Personer som spelar på AC/DC-gitarrer

 \begin{itemize}
 \item Angus Young
 \item Mob 47-Åke \textsc{(s.~\pageref{486ee67ac39debabed3d92a7555dcebd})}
 \item Tommy Iommi
 \item Pete Townshend
 \item Nisse Hellberg
 \end{itemize}

}

\small{
\textbf{AC4}
\label{17380c1b3ba45cc445b955cc6e133b5d}
 Kristet AC/DC-coverband från Umeå.

 Category:Musik \textsc{(s.~\pageref{38cce583d2d3675d645425cb435aa2bb})}

}

\small{
\textbf{Acne}
\label{450e166c06161deddfc97749332c61cb}
 , en fruktad viral sjukdom och ett populärt svenskt klädesmärke. Sjukdomen yttrar sig ofta med maculopapulösa utslag, comedoner och pustler centrerad till ansikte och bål hos juveniler endast för att vidare sprida sig via de sensoriska nerverna, bl.a. trigeminus ganglierna, in till thalamus och hypothalamus för att störa de basala funktioner såsom sexualitet (ofta resulterande i överdriven sexualitet såsom \quotetext{premature ejaculation}) och aggression. Sjukdomen kallas i folkspråk för \quotetext{tonårsfrossa}. För närvarande finns det ingen behandling för Acne utan bör behandlas med \quotetext{Restriction of the social behavior and contacts}, kyshetsbälten och i de värsta fallen bör patienterna sättas i karantän på obestämd tid.

 Klädesmärket har tagit namnet då utbudet riktar sig mot de sjuka tonåringarna som har förlorat sin vett och därmed också klädessmak.

}

\small{
\textbf{Adak}
\label{2fc7bfc5c68257bcd1717bf2898fab16}
 är en ort i Malå \textsc{(s.~\pageref{41da4620e87888eaaeafcb3004a8d177})} Kommun där det bor ungefär 200 personer. En gång i tiden fanns den tillsynes outömliga Adakgruvan med tillhörande by inte så långt från orten. Ett militärt stenkast därifrån låg den lika majestätiska gruvan Rudtjebäcken, även den med tillhörande bebyggelse. På sjuttiotalet fylldes bägge gruvorna igen och bägge byarna förvandlades till spökstäder, då Boliden AB bestämde sig för att fokusera sin produktion i just Boliden. Idag finns där bara tall, gran och en minnessten som berättar om hur kapitalismen och staten svek glesbygden.

 Idag kretsar produktionen kring kultur och institutionen Sagabiografen som varje sommar har filmfestival. På vintern kan man åka på drive-in bio medelst skoter \textsc{(s.~\pageref{b1120baa83f380cd42a805a4e823cb1b})}.

}

\small{
\textbf{Adde Malmberg}
\label{1390facdddaee5ed00a964fbe93b30b9}
 är Sveriges rolighetsminister. Tillsammans med \quotetext{Babben} Larsson kan han få vem som helst att skratta på sig.

}

\small{
\textbf{ADHD-fläta}
\label{20707f66b5c8077e1008cd4698c46322}
 En ADHD-fläta eller råttsvans är en tofs eller fläta som föräldrar låter växa ut från deras låg- och mellanstadie-barns runda små huvudens baksida. Flätan, eller tofsen, finns där för att signalera att barnet är ett riktigt jobbigt barn. En lös teori är att barnet får flätan bortklippt när det slutar vara jobbigt.

}

\small{
\textbf{Adidas-Klas}
\label{a8bb38544b2c4f56034ab8536edb58e6}
 , eller Klabbarparn som han egentligen heter är Karl-Haldos storebror. Han ser väldigt farlig ut på grund av att han har en Adidas-logotyp tatuerad på hakan som löper ner till bröset, där logotypen på ett okonventionellt sätt möter ett drakskelett. Ett annat yttre attribut är Adidas-Klas knogar som bär bokstäverna F-O-R-D, i vad som tycks vara en hyllning till det amerikanska bilfabrikatet Ford (Ford har även europeiska fabriker, sedan många år tillbaka). Gamla människor kan uppfatta Adidas-Klas som varande en aningens farlig, men det beror enkom på fördomar och generationsklyftor. Adidas-Klas är en jättefin människa och dagisfröken.

 Adidas-Klas har spelat trummor i band som Zombiefied och Sista Civilisationens Död (i sistnämnda ingick även Skägg-Peter och en jazzentusiast), en uppgift han klarade med bravur. Vidare har han sett storsjöodjuret och tavelsjöodjuret.

 Mer än såhär behöver man inte veta om Adidas-Klas

}

\small{
\textbf{Adjektiv}
\label{67d02147cd8595eaf13c1a90aba99dcc}
 är inte okej i det postmoderna samhället.
 Allt är normalt \textsc{(s.~\pageref{5c455ca1c87070883ff0a4c13ae8937f})}. Således finns inga orättvisor, inga fattiga, inga tjocka, inga svarta, inget dyrt. Alla är individer \textsc{(se individ s.~\pageref{41beed76a0af9b4f550f7ebdecd3e700})} och allt är bara, bara bra. Kaffet är inte kallt, bussen är inte försenad, klockan går inte för sakta. Allt är bara, bara bra.
 Adjektiv är väldigt kränkande \textsc{(s.~\pageref{5311bb8220aa4c45c14a860bfaa3b0db})}.

}

\small{
\textbf{Aforismer}
\label{42ffbda16f05ff4f9e22b93031d09223}
 Många äro de munläderförsedda som ägnat tid åt att författa aforismer - Oscar Wilde \textsc{(s.~\pageref{379811323e22c1397246d12721dc9fc8})}, Nietzsche, Mark Twain och Platon är endast en handfull av dem. En aforism är en slagkraftig sentens som fångar upp och förmedlar en sanning om människan, varat eller något annat tidlöst och centralt.
 HEAD2: Exempel på aforismer
 \begin{itemize}
 \item Av sina fiender lär sig den vise mycket. - \textbf{Aristoteles}
 \item Nöden är geniets drivfjäder. - \textbf{Honoré de Balzac }
 \item Moralister är människor som kliar sig själva där det kliar på andra. - \textbf{Samuel Beckett}
 \item Mitt i den mörkaste vintern, lärde jag mig äntligen, att det inom mig, finns en oförgänglig sommar. \textbf{Albert Camus}
 \item Det är som min far brukade säga; \quotetext{Vi har kanske inte så mycket att leva av, men om pojken vill ha ett smörpapper så ska han fan i mig ha ett smörpapper!} - \textbf{Mark Frygell}
 \item Den främsta dygden är att lägga band på sig och hålla tungan rätt i mun \textsc{(se näbbmun  s.~\pageref{9e3395be14cf14f92e8cd1e93eb7599b})}. - \textbf{Geoffrey Chaucer}
 \item De flesta människor lever i ruinerna av sina vanor. - \textbf{Jean Cocteau}
 \item Hä ä bar å åk! - \textbf{Ingemar Stenmark \textsc{(s.~\pageref{989f4dba6b1fb2e5920e2c251fd693a2})}}
 \end{itemize}

}

\small{
\textbf{Aftonbladet}
\label{e9ebf180c01d806db2fefd7f53b7a146}
 är en svensk kvällstidning som bäst kan liknas med ett hav av skit i vars mitt en enda liten ö reser sig. På den ön är allas vår Åsa Linderborg kung. Hon är världens finaste och klokaste kung.

}

\small{
\textbf{Akademiker}
\label{05c87f09f25afaf9f129a84230ab1008}
 Att vara akademiker är att förstå det högtravande och pretentiösa språk som används på universitet och högre lärosäten. Det handlar om att till exempel säga \quotetext{esoterisk} istället för \quotetext{intern} eller \quotetext{kontext} istället för \quotetext{sammanhang}.

}

\small{
\textbf{Aktersegla}
\label{fd1077f333993f7f88b3b43533db9b98}
 Lämna någon eller något bakom sig. Ett klassiskt exempel är när Dia Psalma akterseglade Birdnest records och började ge ut sina skivor på egen hand.

}

\small{
\textbf{Albert II}
\label{8a80daf328e56d2b30df9fb6c782146d}
 , född 19 april 1942 i Florida, USA, död 14 juni 1949 i rymden, var den första apan i rymden. Albert II föddes i ett av NASA:s laboratorium där han spenderade större delen av sin levnad med att klättra i träd och delta i experiment. Han döptes efter apan Albert som var den första apan att nästan besöka rymden men som olyckligtvis dog innan han hann fram. Albert II överlevde färden upp i rymden men klarade inte tillbakaresan. Inga officiella statyer eller frimärken har någonsin skapats till Albert II:s minne.

}

\small{
\textbf{Albin}
\label{899dffe3a176a36256e0a00fb97698d3}
 är ett namn som nyblivna föräldrar ger till sitt barn med förhoppningen att det ska göra att barnet automatiskt blir lillgammalt \textsc{(se lillgammal s.~\pageref{d30762f704f0731fb3b08cd0846128d4})}. Albin betyder inte just något särskilt.

}

\small{
\textbf{Aleksandr Karelin}
\label{7db555630a4ad78feb3477db9b1ee464}
 Aleksandr Aleksandrovitj Karelin, även kallad \quotetext{det ryska experimentet} och \quotetext{lyftkranen från Sibirien}, född 19 september 1967 i Novosibirsk, Sovjet, är ingen kille man bråkar med. Till yrket brottare.


 HEAD2: Föda och beteenden
 För några år sedan visade Sportspegeln ett inslag från Karelins hem i Ryssland i vilket det tydligt framgick vilken fin människa Karelin är. Där visades också hur Karelin stod i en snödriva och grillade korv på en klotgrill medan han drack en klunk vodka då och då för att hålla värmen, en typisk ägmästarsituation \textsc{(se ägmästare s.~\pageref{8324518500d7e7ccd22ae364887d4476})}.

}

\small{
\textbf{Ales Stenar}
\label{2b28507979e217cfe15c9d6455eabd18}
 är en fornlämning från Sveriges \textsc{(se Sverige s.~\pageref{b1999637949ed135b2ca03f3a38460cc})} första lärosäte, långt innan Uppsala Universitet och folkskolereformen. Här fick man lära sig hur man gjorde sitt eget tråg \textsc{(s.~\pageref{1e0e0470206e0f2baad8e628ba8f770c})}, hur man gjorde för att plundra, varför det är viktigt att slänga gamlingar utför ättestupan och andra viktiga saker för att klara sig förr i tiden. Själva stenarna tros ha använts till stenkrig på rasterna, en lek som ansågs mycket pedagogisk enligt dåtidens läroplan.

}

\small{
\textbf{Alfons Åberg}
\label{3c49eba29ed964486f6392305ad63694}
 Storköpenhamns värsta hallick. Känd för att styra med järnhand

}

\small{
\textbf{Alg-Börje}
\label{623e6649678262697f465fe2cdb41679}
 är en karl (förmodligen) från Valbo utanför Gävle som klarar uppehället genom att kränga hygienprodukter baserade på alger.
 Algerna i Alg-Börjes produkter kommer från ishavet. Alg-Börje är till och med generalagent för isländska havsalger, vilket låter ganska mäktigt.

 Exempel på produkter är:

 \begin{itemize}
 \item Alg-Börjes Algschampoo
 \item Alg-Börjes Algkroppstvål
 \item Alg-Börjes Liniment
 \item Alg-Börjes Tångbad
 \item Alg-Börjes Algtabletter
 \item Alg-Börjes Algmjöl (finmald)
 \item Alg-Börjes Algmjöl (grovmald)
 \item Alg-Börjes Corallkalk
 \end{itemize}

 Källa: [http://www.Alg-Borje.se]

}

\small{
\textbf{Alg-Gutten}
\label{a0ec949e57eca216fde1bbd5dff07c19}
 är en man (förmodligen) från Finspång som säljer hundmat gjord på havsalger. Algbaserad hundmat sägs ge både blankare päls och ökad pigmentering. Hundmaten har flera slående likheter med Alg-Börjes \textsc{(se Alg-Börje s.~\pageref{623e6649678262697f465fe2cdb41679})} produkter för \textsc{(s.~\pageref{5a98c81c7b5b60a5777a92b943f53a41})} människor.

 Källa: [http://www.alggutten.se]

}

\small{
\textbf{Alice Tegnér}
\label{66a2a0b3aa1a42e1e5ae2d20dd1bdca6}
 är ledare för den ljusskygga organisation som i folkmun har kommit att kallas Alice Tegnér-sällskapet. Det har länge spekulerats i organisationens utbredning. Envisa urbana legender om vilda inträdesriter samt infiltration av samhällsuppehållande institutioner förekommer, men mycket är nog bara snack. Enligt avhoppare ska Tegnér kommunicera med organisationens medlemmar genom infernalisk musik, som organisationen sedan studerar i bokform likväl som genom framförandet av sånger som förts ned i leden från ledare till medlem. Gällande detta musikaliska inslag finns vissa påtagliga bevis, där boken \quotetext{Nu ska vi sjunga,} som kan studeras bland annat på Karolina Rediviva och Kungliga Biblioteket är det mest framträdande.

}

\small{
\textbf{Alienation}
\label{7c5c79d2842cce56e7e1f1d288b86f52}
 Så här är det. Människor gillar inte att göra samma sak hela tiden. De vill skapa och använda en massa delar av hjärnan dagligen. Kapitalister ogillar detta för att det är ineffektivt. Människor måste tyvärr arbeta för att få lön och kunna leva, och då måste de använda den där speciella delen av hjärnan som kapitalisterna vill att de ska använda. Genom att människor arbetar upprätthåller de också det kapitalistiska systemet genom att förmera kapitalet, och i förlängningen reproducerar de sitt eget förtryck. Tråkigt, men vad ska man göra åt det? Att gå med i ett fackförbund är en bra början, att beväpna sig en bättre.


 Funny fact: Alienation är också en jävligt bra punklåt av det engelska bandet Crisis [http://www.youtube.com/watch?v=2vlRg9uo_QI]

}

\small{
\textbf{Alkisschäfer}
\label{347febbc28041eae88556d2e7ced587b}
 En alkisschäfer är en hund som inte är riktigt  herrelös men heller inte har någon tydlig ägare. Vanligtvis saknar den någon av kroppens lemmar, typ ett ben eller ett öga. De flesta har lätt att vistas bland människor men vissa kan va lite folkilskna av sig och då måste man riva i. Alkisschäfern är vanligtvis kopplad, men det är inte alltid säkert att det är någon som håller i kopplet. Den är rätt skitig men glad. Typiska namn på en alkisschäfer är  Pudas \textsc{(se pudaslåda  s.~\pageref{6a56958e2057dd500650e2be8049e033})}, Kari, Sam och Torax. Som vanligt är det inte hunden det är fel på utan ägaren. Namnet till trots kan en hund från nästan vilken ras som helt bli en alkisschäfer.

}

\small{
\textbf{Alkohol}
\label{11c589cba1a208e0359048a78e6b88b8}
 är ett vanligt lösningsmedel. Det löser problem.

 Utan alkohol kan man inte ha roligt, och har man roligt utan alkohol vore det roligare med alkohol. Inte minst enligt Boris Jeltsin.

}

\small{
\textbf{Alkoläskfylla}
\label{8234165b965f2b1378f10acd340dc126}
 Sittandes på en stol vid ett köksbord med chipsskål. Någon spelar en Gessle-låt på gitarr. Vem är hon med flätorna? Reda ut vem som kom först, Niclas Strömstedt eller Mikael Rickfors. Ingen vet. Resa sig för att gå ut på balkongen, till synes för att röka men egentligen för att prutta lite. \textit{Vafalls!} Dörrposten gungar.

 \textit{Alkoläskfylla}.
 {{Yrsel}}

}

\small{
\textbf{Alla x i y}
\label{18d4689248d1c32d716dab95e7e57b17}
 är ett sorts versmått som används för att förolämpa någon genom att, mer eller mindre subtilt, påpeka att denne är dum i huvudet \textsc{(se huvud s.~\pageref{e906cd95a540df9b16d0460fb4cf0adc})}.
 Vanligtvis används meningen så här: \quotetext{Johanssons grabb har ju inte alla x i y}.

 Här följer en lista på förolämpningar i \quotetext{x i y}-form.

 \begin{itemize}
 \item Alla ägg \textsc{(s.~\pageref{128a5feb8e12d0aa622e0298a8332980})} i korgen
 \item Alla får i fållan
 \item Alla läsk i backen
 \item Alla hästar \textsc{(se häst s.~\pageref{b4c608370b339da095c5f8db7fab0945})} i hagen
 \item Alla indianer i kanoten
 \item Alla nitar i västen
 \item Alla sidor i boken
 \item Alla finnar \textsc{(se finland s.~\pageref{631d44eaa1254ff71a1e11ba021d1266})} i bastun
 \item Allt var i finnen
 \item Alla Oi! i Oi!kören \textsc{(se skinhead s.~\pageref{a54bc1b5d472b5afed8e84004b6441c4})}
 \item Alla kulor i påsen
 \item Alla fän i hönsgården \textsc{(se hönsgård s.~\pageref{3f284439cd46e0b187d34410aa79b2fb})}
 \item Alla pennor i skrinet
 \item Alla fjädrar i Chapeau de paysanen \textsc{(se Chapeau de paysan s.~\pageref{27aa75146d9ab723d1423168a2539d5d})}
 \item Alla knappar i västen
 \item Alla uvar \textsc{(se uv s.~\pageref{45210da832f9626829457a65e9e7c4d0})} i uvslaget
 \item Alla skrovmål i calzonen \textsc{(se calskrove s.~\pageref{84ff54e779ee49fdad21e17c20f14453})}
 \item Alla färger i regnbågen
 \item Alla chips i skålen
 \item Alla skivor \textsc{(se sjua s.~\pageref{e7bf63fa6d0d29bd89c23f833b979a15})} i fodralen
 \item Alla brugdar \textsc{(se brugd s.~\pageref{d6b6b68506b8f1daad3a2ddbfaf8d863})} i Skagerrak
 \item Alla hästar i kanoten
 \end{itemize}

}

\small{
\textbf{Allergi}
\label{23773a17729d8e7e24da798e97533aeb}
 Identitetsmarkör hos stadsbarn. Tidigare kallat högfärdshosta.

}

\small{
\textbf{Allting}
\label{2ea7603b8880ffdf729128008f5d252d}
 är på låtsas.


 Källa: Per-Olof Frimodigh

}

\small{
\textbf{Allväderstövlar}
\label{5bd78c98912d328dae77b1dff5148e45}
 Till skillnad från den av många älskade gummistöveln, de lite fräckare MC-stöveln och fuck-me-stöveln samt mer mystikomspunna varianter som såna där jättehöga stövlar som fiskegubbar har när de av nån anledning står mitt i en å och fiskar är allväderstöveln en stövel som skapats genom ett gediget allround-tänkande. Här har man satt funktionen först, men för den delen inte glömt att tänka på design, för att ta fram en stövel som är det självklara alternativet vecka efter vecka, året runt. Allväderstöveln värmer och skyddar mot väta, och tillåter således bäraren att kunna njuta av att delta i polarexpeditioner eller hoppa i vattenpölar utan att plågas av elementen. Dess onda tvilling är snowjoggern \textsc{(se snowjoggers s.~\pageref{ee340dd9a61f36aaa0f7581db6e3d374})}.
 HEAD2: Allväderstöveln i populärkulturen
 Allväderstöveln förknippas ofta med Christopher Tolkien, JRRs ohängde son.

}

\small{
\textbf{Aln}
\label{b31e552786a58ecaff9efcfe53231ed3}
 Från armbågen till långfingerspetsen.

}

\small{
\textbf{Alternativa namn på bakverk}
\label{95983a7c39a10b94946c312761bf3db6}
 Alla som hänger med i vad som diskuteras ute i Svea Rikes fikarum, mellan hyllraderna på din lokala konsumbutik \textsc{(s.~\pageref{70e4875f7c2c177596305006e46b7ca9})} eller på caféet vid torget kan omöjligen ha undgått diskussionen om att chokladbollen faktiskt, FAKTISKT!, heter negerboll. Alla som säger nåt annat är bara demokratihatare eller törra jävlar som borde slappna av och inte bli så provocerade bara för att någon lägger sig till med att säga N-ordet. Sen att man inte får säga vad man vill i det här jävla landet, det är en annan, om än tätt sammanflätad, fråga. Det har baskemig alltid hetat negerboll och det är alla etniska svenskar mitt i livets förbannade RÄTT att säga det. Nåväl, kan man tycka, men varför begränsa sig vid rasistiska tillmälen i samband med trefikat? Ta steg in i tvåtusentalet och välj ett lite mer intersektionellt \textsc{(se intersektionalitet s.~\pageref{6dc08633cdbf83eb418ea31ef0302c51})} spår när du beställer fikabröd.

 Ska du ta in en bit prinsesstårta för att fira att Jimmie Åkesson gått ner i vikt? Prinsesstårta!? Prenumererar du på BANG eller? En bit Horfittetårta tack! Kanske något matigare, en källarfranska? FEL! Josef Fritzl-franska heter det såklart! Slappna av för fan. Dags för mer sötebröd, kanske en Napoleonbakelse? App app app, Adolf Hitlerbakelse ska det va, eller vill du vara politiskt \quotetext{korrekt}? Tönt. Ett wienerbröd med lite extra mormorshosta, kanske? Sieg Heilbröd med äckligt gammalt var och pojksperma!

 Ska du prompt bete dig som ett kräk så ska det fanimig märkas ordentligt.

}

\small{
\textbf{Alvedon}
\label{cb0caf905b97cf4bb6a1cfec483cbf3a}
 Medicinskt piller som ordineras människor vars öron börjat få en allt för spetsig form på toppen.

}

\small{
\textbf{Alvparty}
\label{0f1970046158fc84b3227dd11015d1af}
 Ett alvparty är ett party med alvtema. Ofta är gästerna utklädda till alver - man äter mat inspirerad av J.R.R Tolkiens \textsc{(se J.R.R Tolkien s.~\pageref{3f0b7fcbd9fa7369ca314a46c280b67e})} böcker, och man har stövlar. Vissa gillar att bära en stav och att fästa latex på öronen så att de ser ut att vara spetsiga. Att memorera någon fras på \quotetext{alviska} ur någon rollspelsbok brukar vara uppskattat. Har man riktig tur får man ligga med någon, och inget säger att det nödvändigtvis måste vara med festens obligatoriska överviktiga gothare.

}

\small{
\textbf{Amazonskrake}
\label{c8b53d55fb445965eb6afbce9865c210}
 är ett djur i familjen marklevande ankor, en kvist på fåglarnas gren i det brokiga träd som symboliserar djurrikets fantastiska mångfald. Amazonskraken kan bli upp till en halvmeter hög och ha en vingspann på sjuttio centimeter. På grund av dess tubformade kropp flyger den, vilket den dock ogärna gör, på så vis att den pumpar sig upp och samtidigt fram för att sedan sjunka ner igen efter några sekunder, så att dess rörelse beskriver en båge. Efter att ha hämtat andan är det dags för ännu några sekunders frenetiskt flaxande, och så vidare. Den går heller inte särskilt bra eftersom dess ben är så korta att den endast kan ta mycket korta steg. Den har på grund av detta blivit ett eftertraktat bytesmål för reptiler, som på grund av dessas växelvärmesystem inte kan förfölja sina byten om de springer iväg. Bland stammar i områden där Amazonskraken lever har det blivit ett slags rituell sport för unga ogifta män att springa genom djungeln och sparka Amazonskraken så att den flyger i en båge som påminner om fågelns flykt. De unga män som sparkat flest fåglar får gifta sig \textsc{(s.~\pageref{fde23aab828cff50483a59fc662f8fa8})} med döttrar till socialt prominenta stammedlemmar. Amazonkraken är mycket, mycket utrotningshotad.

}

\small{
\textbf{Ambigram}
\label{5560a8970af5de25fb1923c975c0fbe0}
 Ett ambigram är ett ord eller fras skrivet med en symmetri som gör att det går att läsa även upp och ner eller spegelvänt. De tidigast kända ambigramen skapades i England av Peter Newell, illustratör åt Mark Twain och Lewis Carroll. Det är mycket svårare att göra ambigram än besläktade fenomen såsom palindromer och anagram. Det är även mycket psykedelisk.

}

\small{
\textbf{Amebix}
\label{06e1b1571c8367d12818923147516b0e}
 är inte världens bästa band \textsc{(se Gism s.~\pageref{025b6da73f168a6d2e766e79c9b2941a})}. De är dock enhälligt framröstade till titeln: Världens ballaste band. Amebix är också en kissekatt \textsc{(s.~\pageref{0fd9accd1d8c95e86a96f681b6805948})} som bor i Prag.

}

\small{
\textbf{An old man learn english}
\label{29861842ad25eae97282e45aeb96d437}
 [http://www.youtube.com/watch?v=DJJRy7OHn-Q]

 Detta stycke ofattbar video väcker många frågor. Vem är mannen? Var har han lärt sig dansa? Vad fan pysslar han egentligen med? Vad tycker de uppiffade damerna om gubbjäveln? Kan gitarristen solot i Detroit Rock City? Vem har skrivit denna massiva hit? Hur går texten förutom \quotetext{Cat. Hand some. Waiting. Farmer. Father. Mother. Dog. I don't know. One one. Two. One two one two. Best. Best.}

 Och framförallt. Hur fan får man tag i mer???

 EDIT: Den khmer-amerikanske rapartisten Prach Ly har nu kontaktats i hopp om att få några av frågetecknen utsuddade.

}

\small{
\textbf{Ana uvar i mossen}
\label{e715db369ab4e623631a8819632d78a0}
 \quotetext{Att ana ugglor i mossen} betyder att något lurt är på gång. \quotetext{Att ana uvar \textsc{(se uv s.~\pageref{45210da832f9626829457a65e9e7c4d0})} i mossen} är att jordens jävla undergång är på väg, men omständigheterna kring den är lite vaga. Lite som Susan Sontag menar när hon säger att \quotetext{It isn't Apocalypse now! any more. It's apocalypse from now on.}

}

\small{
\textbf{Andra sidan ån}
\label{a38d73b3d20d187492469b6b0c81fb84}
 Ungarna på andra sidan ån har lite jobbigare dialekt, de lägger en lite gnällig betoning på deras E:n när de pratar och alla låter likadant. Ungarna på andra sidan ån är alltid först med allting \textsc{(s.~\pageref{2ea7603b8880ffdf729128008f5d252d})}, vare sig det handlar om moped, bredbandsuppkoppling eller flickvän.

}

\small{
\textbf{Andörjan}
\label{dce7c6b6c4ef3e8cbc15edafbd1cc086}
 Västerbottnisk benämning på skidspår.
 Används oftast när en något snabbare skidåkare kommer ikapp en något långsammare dito och tycker att den sistnämnda borde ge plats för den snabbare. Kommandot som den snabbare skidåkaren utbrister i är -He \textsc{(s.~\pageref{6f96cfdfe5ccc627cadf24b41725caa4})} dej bortu Andörjan!! Fritt översatt: -Var vänlig och flytta dej ur spåret så att jag kan åka förbi! I södra Sverige används termen \quotetext{Ur spår!}.

}

\small{
\textbf{Angel of death}
\label{cb9b9541ba49042c65216f68f0b5d042}
 är en låt av det amerikanska thrash metalbandet Slayer. Låten skrevs i samarbete med Vägverket och syftar till att uppmärksamma trafikanter om vikten av att alltid ta hänsyn till den döda vinkeln vid omkörningar. I ett tidigt skede av låtskrivarprocessen var Dr. Alban påtänkt som gästartist, men denne skrev istället den egna låten \textit{10 små moppepojkar }där Kerry King \textsc{(s.~\pageref{d6b6ab73f6de8a63b008622c780b0ad5})} bidrar med gitarrpålägg.

}

\small{
\textbf{Anglosax}
\label{75591674b0deca83291ccfef6f4f557c}
 Modernt fiskeredskap som uppstod ur ett juridiskt kryphål i EU:s \textsc{(se EU s.~\pageref{4829322d03d1606fb09ae9af59a271d3})} fiskedeklaration om förbud mot angling och gäddsaxar.

}

\small{
\textbf{Ankeborgslagstiftning}
\label{8598b83afdaf4801d1875e348df3ea53}
 Man äger allt som hamnar på ens gård.

}

\small{
\textbf{Ankfot}
\label{e72e010c637946e14c6401dba9a47d00}
 är en benämning på det stadium av yttre dekadens då ens strumpa har halkat ner på foten och nu sticker ut en bit utanför tå-raden, så att foten påminner om en ands fot. Detta fenomen är vanligt bland ensamstående medelålders män och deras tonåriga gelikar, men förekommer också i andra samhällsgrupper, om än med betydligt mer sparsam frekvens. Ankfot har blivit vanligare i senare tid i takt med att kvalitén i strumpresåren stadigt blivit sämre, medan antalet par strumpor i storpackserbjudanden blivit fler. Detta i full harmoni med det nyliberala \textsc{(se nyliberalism s.~\pageref{a562ace16486d966be4513ea22aee287})} konsumtionssamhälle som tvingats fram på mer ordnade samhällsformers bekostnad.
 HEAD2: Åtgärder
 Den som drabbas av ankfot har, lite förenklat, två huvudsakliga åtgärder att välja mellan. Det ena är dra upp strumpan så att den sitter som den ska och det andra är att helt sonika dra av strumpan och kliva omkring barfota. I ett längre perspektiv kan strumpor med bättre kvalité (t.ex. Biltemas) vara nödvändigt att inhandla, men detta är i första hand möjligt för den förvärvsarbetande medelklassen, varför krav på statliga åtgärder framförts från vänsterkanten av det politiska spektrumet. Ett annat förslag, som framförts av handarbetande miljöpartister \textsc{(se miljöpartiet s.~\pageref{3e11b29518eeea19128b64869699f363})}, är att höja momsen på vanliga strumpor och reducera momsen för strumpor som är försedda med dragsko \textsc{(s.~\pageref{0d3beb9223700e39e09040e9bbd3644b})}.
 HEAD2: Ankfot i danskt mode
 I Danmark \textsc{(s.~\pageref{5331d7fd27772396f412a5b6d19bad44})} har ankfot samma funktion som rökrocken har i resten av västvärlden. När dansken är hemma och njuter av ledighet klär hen av sig sina kläder och går omkring i tubsockor med en cigg i mungipan, petar lite i brasan och mår bra i största allmänhet.

}

\small{
\textbf{Anki Blombergsson}
\label{72337d6503a1e80dbdcf23831dc1996b}
 är en svensk översättare som bland annat översatt senaste Bond-rullen.

 Det som gör att Blombergsson kommer att nämnas i historiens annaler är hennes kamp för att få in ordet färst \textsc{(s.~\pageref{5530a65addfadb2019a87b18347ab20b})} i svenska språket.

}

\small{
\textbf{Anna skipper}
\label{69070525b7d98c75ddeb9a53f582fbfb}


}

\small{
\textbf{Anna Skipper}
\label{69070525b7d98c75ddeb9a53f582fbfb}
 \textbf{Anna Magdalena Skipper}, född januari 1968 är en svensk programledare i Du är vad du äter i TV3. Vid sidan av TV-karriären är hon kommunalråd i Skänninge för Sverigedemokraterna.
 Rykten gör gällande att Anna Skipper under inspelningen av Du är vad du äter-avsnittet med Bert Karlsson bjöd ut sig sexuellt till Bert för fri entré på Skara sommarland i resten av sitt liv.
 Är också en jävel på vegan \textsc{(se veganer s.~\pageref{2a12d5d6ae91d2f4f7d9af3cef58e75c})}.

}

\small{
\textbf{Anna-Karin Hatt}
\label{5aae353dc0406c675ce0796616720df9}
 Ett ganska lajbans namn på en ganska olajbans politiker.

}

\small{
\textbf{Annie Lööf}
\label{6427a5c40ebf75b63298a9b8df8ca54e}
 Margaret Thatcher \textsc{(s.~\pageref{0bdaa2c5b2f4fb15d678c3e54c10d347})} och Ayn Rands kärleksbarn. Obekräftade källor säger också att hon har en affär med Joseph Kony[http://sv.wikipedia.org/wiki/Joseph_Kony]

}

\small{
\textbf{Annika}
\label{fe3be36bccbe5ea96bfba2e631fda48f}
 är granne till Pippi Långstrump och syster till Tommy. Liksom honom är hon i jämförelse med den betydligt mer framfusiga och företagsamme Pippi lite av en mes.

}

\small{
\textbf{Annons}
\label{816231c1c88ab4c0f93ca97a88ee541f}
 En annons är en text på webben eller i tidningen som beskriver en vara, tjänst eller annat som man ofta vill sälja byta eller köpa.
 Dags för några historier ur det verkliga livet.
 HEAD2: Fallstudie
 ''
 Jag har valt att beskriva en annons där man säljer.''

 Som tex om jag vill sälja min trötta rostiga och slitna gamla Volvo 240 från -83 så skriver jag detta i annonsens rubrik:

 \textbf{\quotetext{Väldigt fin och lättstartad Volvo 240 från 83.}}

 Sedan kommer man till finessen, i en annonsbeskrivning så gäller det att ljuga så mycket som möjligt för att lura en eventuell köpare att tro att detta är det enda exemplaret i världen och till detta låga pris.
 \textlessblockquote\textgreater
 Bilen har varit endast tjejkörd, aldrig vinterkörd och alltid stått i garage.
 Den drar inte olja och knappt någon bensin.
 Har rullat 34000Mil i skrivandets stund så den är knappt inkörd.
 Den har aldrig krånglat så nu har jag ledsnat på att äga detta underbara fordon.
 Tyvärr så är det inte min bil på bilden men en exakt likadan fast Röd.

 Pris vid snabb affär 13500Kr.
 Ring Jocke.
 \textless/blockquote\textgreater

 Att man skriver pris vid snabb affär samt säljes i befintligt skick är mest för att det låter ballt, ingen vill köpa något i obefintligt skick och långsam affär är inte säljaren intresserad av, han/hon vill ju få sina pengar och byta telefonnummer \textsc{(s.~\pageref{0978f3303660fc9c74d08f85b89ba974})} så fort som möjligt.

}

\small{
\textbf{Antal klick till Hitler}
\label{2784fd9ed4c944d86073d12811cf4a06}
 Att på Wikipedia trycka \quotetext{slumpartikel} och därefter, via de blå länkarna i texten, klicka sig mot artikeln om Adolf Hitler. Den som klickar minst antal gånger vinner.

 Viggos rekord är ett klick. Slumpsida -\textgreater Berlin -\textgreater Adolf Hitler

 Nissepedias \textsc{(se Nissepedia s.~\pageref{62400dadecd90cb5cd39062abe5a3e4a})} motsvarighet till denna tävlingsform kallas i folkmun \quotetext{Antal klick till uv \textsc{(s.~\pageref{45210da832f9626829457a65e9e7c4d0})}}.
 Ett närbesläktat spel är det som går ut på att klicka på \quotetext{slumpartikel} på Wikipedia och räkna antalet gånger det tar tills man slutligen kommer till artikeln om ugglor.

}

\small{
\textbf{Anti-speciesism}
\label{f520aed28c98e2d9f59ce26d3c8bc523}
 Att ogilla när folk säger \quotetext{Ditt jävla svin!}, \quotetext{Fan jag måste gå och räva! \textsc{(se räva s.~\pageref{c13f687883c1eb0be3be218fff63e6b8})}} eller \quotetext{Du är ju lika oattraktiv som en flodhäst \textsc{(se Hippopotamus s.~\pageref{9b4609b17fea63f3f3f067fc2f465c6e})}}. Speciesism är strängt förbjudet på Folkkök \textsc{(s.~\pageref{15983d1934522d4d08e766108357201b})} i Umeå \textsc{(s.~\pageref{bd1e37dc477bb704c667ed1a4606df71})}, och detta uttrycks \textit{med emfas} på en handskriven och vackert illustrerad lapp, så där får man hålla tand för tunga!

}

\small{
\textbf{Antikvärde}
\label{1444201042a389f85dbb353fdb47ac4e}
 är den mystiska kraft som håller landets loppmarknader igång.


 Några handfasta exempel:
 \begin{itemize}
 \item En Pelle Karlsson-lp \textsc{(se Pelle Karlsson s.~\pageref{1a8c873ff230698396c324f14c02b7fa})} som i praktiken har ett värde under noll ökar genom antikvärde till över nypris.
 \item Chartersouvenirer från 70-talets Mallorca får ett värde i klass med huppdiamanten.
 \item Kopparbyttor
 \item Shabbychic
 \item Trälådor som det förvarats margarin eller socker i för ett halvt sekel sen.
 \end{itemize}

}

\small{
\textbf{Antilop}
\label{b9ca018be08f1eb67852a2df5b6730b0}
 en (\textit{Taurotragus oryx}) är ett enigmatiskt och mytomspunnet djur som vi vet mycket lite om men som ofta förekommer i folksägner och sagor. Den sägs inge en närmast spirituell känsla där den svävar på sina långa vingar över isbrytaren som enträget stävar fram genom Nordvästpassagen. Bristen på kunskap om djuret och de många myter som omgärdar det gör det svårt att med säkerhet uttala sig om dess naturliga beteendemönster. En stark hypotes är dock att den lever på slemm och annat snusk och att den under den mörkaste årstiden lägger ett litet, litet ägg i en liten, liten korg som den försiktigt tar i näbben innan den beger sig iväg ut över de vita vidderna.

}

\small{
\textbf{Antirep}
\label{eb508922f79a60f76b0278edfea25a8c}
 Otroligt provokativt band bestående av
 \begin{itemize}
 \item Anton (Dead ones)
 \item Spetan (The triffs)
 \item Erik (Pöbeln)
 \item Svartmetallare (nåt hertsömetallband)
 \end{itemize}

 Deras mesta hit \quotetext{Slå Arvo \textsc{(s.~\pageref{f9ba22fb1ff25fd8f83d48f361aea0a4})} i huvet med en påk} gick varm på kassettbandare på gulan i slutet av 90-talet.

}

\small{
\textbf{Anton Abele}
\label{0906f6e1d290c547e1fb93c6ff6a0b44}
 är en moderat \textsc{(s.~\pageref{c4564b188cb670841733a3ff923c2fb0})} riksdagspolitiker (sveriges yngsta!) vars hjärtefråga i valtid var att han var emot våld \textsc{(s.~\pageref{c01df500e07826fb356183119ff0d07c})}. Denna valkampanj sköttes på internet och när den avslutades hade Abele bytt hjärtefråga. Nu är han emot näthat. I hans ansikte sitter också Sveriges största haka \textsc{(se Haka (vanlig) s.~\pageref{3b8edf3dc8968e6b2805dc512af3b68c})}. Om dessa två stycken fakta har med varandra att göra låter vi vara osagt.

 När Anton flyttade hemifrån höll det på att resultera i katastrof. Man hade helt enkelt förträngt hakan vid lägenhetsanskaffandet och flytten till hans första lägenhet, en etta, blev ytterst traumatisk. Som den äkta moderat \textsc{(s.~\pageref{c4564b188cb670841733a3ff923c2fb0})} han är så lyckades han dock köpa sig förtur i bostadskön och fick förstahandskontrakt på en 1,5a på Kungsholmen på rekordtid.

}

\small{
\textbf{Anton Lavey}
\label{869cf213daf268853824c26db9960ab7}
 40-talistgenerationens Alexander Bard. Hittade på en egen religion där han var bäst, nämligen \quotetext{den moderna} satanismen, som i princip är en plankning av vilket nyliberalt partiprogram som helst + cape \textsc{(s.~\pageref{8b04f4091aa625f56b3f7da315a1e231})}.

}

\small{
\textbf{Antonio Gramsci}
\label{d4d0da57d321555b3550f1d7cffa3249}
 (uttalas /gramʃi/) var en snäll kommunist \textsc{(s.~\pageref{fd9bf7896d396992b29d542a0200b800})} som tillfångatogs av de elaka fascisterna och spärrades in i ett torn som nådde ända upp till himlen. Han blev förtrollad av fascisternas konung, Musselini \textsc{(s.~\pageref{6b19e50717b8e033d311707b4fa69f9d})}, så att hans ryggrad blev för kort. Gramsci led mycket av detta och allt han hade som tröst var sin marxism och sina små anteckningsböcker i vilken han skrev och skrev, dagarna i ända. En dag kom en riddare till tornet och bankade på dess port. Riddaren hade en rustning av skinande metall och en lång lans och red en stor vit häst vid namn TB \textsc{(se Torbjörn s.~\pageref{c3e6fb6fb2b655457597f063bd9392e8})} Rolandsson. Vakterna sprang förskräckta till kungen och berättade att en ståtlig riddare väntade utanför porten. Den sluga kungen blev lika förskräckt och trodde att sina dagar var räknade. Han grät som en italiensk fotbollsspelare och svärjde italienska svordomar \textsc{(s.~\pageref{3a490d7017c99929180c9d80e52e5926})}. Men så kom en krum gubbe släpandes fram till tronen och viskade i kungens öra \textsc{(s.~\pageref{c4774ec92abe06f5664e18f44446d7e7})}, och genast blev Mussolini på bättre humör och gned sina händer. Han lät tillkalla vakterna och beordrade dem att tala om för riddaren att om han kunde svara på tre enkla frågor skulle han genast bli insläppt.
 \begin{itemize}
 \item Den första frågan löd: Varför har inte arbetarna i Italien gjort revolution, då de förutsättningar som Marx har beskrivit finns tillgängliga?
 \item Den andra frågan löd: Varför har vissa idéer, hur folkfientliga de än är, större genomslag än andra?
 \item Den tredje frågan löd: Vilken roll har och bör de intellektuella ha i samhället?
 \end{itemize}
 Riddaren kliade sig i skägget och fann inget svar på någon av frågorna. Men Gramsci, slug som han var, hade fäst några av sina anteckningar på en pil som han sköt ner från tornet så att den fastnade i riddarens sköld. Riddaren läste Gramscis små lappar och kunde genast svara på kungens frågor. Kungen blev nu utom sig av ilska och lät slå ihjäl den gamla gubben. I tumultet som utbröt öppnades porten av vakterna, som trodde att kungen var en rättsälskande kung, och riddaren kunde rida in på TB Rolandsson och döda alla i slottet förutom Gramsci, som förtrollades till en kentaur och red iväg på regnbågen. Kvar fanns bara Gramscis \textit{Prison Notebooks} som än idag skänker glädje åt vänsterintellektuella i hela kungariket och även utanför det.

}

\small{
\textbf{Apor vi minns}
\label{ae663c0ea498c2a641a8f701dc30da0a}
 Apor vi med glädje och ett visst vemod minns.

 HEAD3: Herr Nilsson
 En något diffus liten apa iförd kofta. Nilsson hölls fången av en rödhårig riskkapitalist och tvingades bevittna ungdomlig baluns och dekadens. Koftan hintar om att Nilsson trots allt hade högre mål i livet men det är oklart var Nilsson tog vägen då hans matmor flyttade sin väska med guldpengar till nåt skatteparadis.


 HEAD3:  Apan Ola
 Apan Ola blev tv-kändis när han iförd blöja levde city-life i Stockholm. Han åkte sedan som många andra ungdomar på semester i Thailand. I likhet med dessa blev han sedan kvar, dock inte på Bangkok Hilton efter att ha försökt bära en väska med \quotetext{örter} genom tullen. Olas öde blev istället Bangkok Zoo. När detta uppdagats steg Olas popularitet till nära Noshörningen Nelson \textsc{(se noshörningen nelson s.~\pageref{e439707db1c491d30a2ac06e71632fe6})}.
 Apan Ola har besjungits av Arsedestroyer.


 HEAD3: Albert II
 Albert \textsc{(se Albert II s.~\pageref{8a80daf328e56d2b30df9fb6c782146d})} var den första apan i rymden men vad hade han för det?


 HEAD3: Nicke Nyfiken
 Många barns första möte med en tecknad apa. Väldigt intetsägande personlighet och har förmodligen haft det jävligt tungt sedan strålkastarljuset slocknade.
 Nicke Nyfiken har aldrig besjungits av Arsedestroyer.

}

\small{
\textbf{App}
\label{d2a57dc1d883fd21fb9951699df71cc7}
 En App är en makapär som gör att din telefon kan användas på andra sätt än att bara ringa med. Som att se hur full du är, vad som visas på tv \textsc{(se television s.~\pageref{79464212afb7fd6c38699d0617eaedeb})} eller roliga historier om Chuck Norris.

}

\small{
\textbf{Arbetarblåsa}
\label{ad723511b102fd84dca91aca133ffef8}
 En riktig arbetare pissar var tredje timme oavsett behov.

}

\small{
\textbf{Arbetarklassrock}
\label{3268200a49db708660491e54e53c05c3}
 i finbyxor \textsc{(se min bästa byxa s.~\pageref{d713d68db15d469d6e39abacefefb3ab})} är ett begrepp som betonar kvalité och äkthet hos rockmusik. Just därför utger sig många rockband för att spela arbetarklassrock medan de i själva verket inte alls spelar arbetarklassrock. Riktig hederlig arbetarklassrock kan låta lite hur som helst och återfinns över hela rockens spektrum; i Oi!, metal, gubbrock \textsc{(s.~\pageref{017518791bbb6d1db3fca1e31b678b4d})}, sydstatsrock och så vidare. Gemensamt för många arbetarklassrockband är att de har minst en låt om hur mycket det suger att fira jul med släkten. Ett exempel på en sådan låt är The Warriors - \textit{Holiday Songs} [http://youtu.be/rzE_GALTzTE].
 HEAD2: Exempel på arbetarklassrock
 AC/DC
 Thin Lizzy
 The Last Resort
 Oxblood [http://youtu.be/00N4JAPuVlI]
 Bruce Springsteen
 The Replacements
 CCR
 Grateful Dead \textsc{(se deadhead s.~\pageref{30563e7c77afbd00a3aafa07829c95d3})}
 Entombed
 Rose Tattoo
 Taste
 Blue Cheer
 Köttgrottorna
 Motörhead
 Lynyrd Skynyrd
 Hurriganes
 KSMB
 Manowar
 Snoddas \textsc{(se Gösta \quotetext{Snoddas} Nordgren s.~\pageref{5cb1aa19b3f60a517978ebea69456dcf})}
 Slade [http://youtu.be/VXkxNei8aZ0]
 Quiet Riot [http://youtu.be/w0HpQplM7NE]
 The Oppressed [http://youtu.be/uXVAR1you00]
 4-skins [http://youtu.be/Eon_VN96tO4]
 The Crack [http://youtu.be/JZx5-LEot74]
 One Way System [http://youtu.be/ACC3Nexh1lo]
 Edguy [http://youtu.be/2yxvk5KZq40]
 Troublemakers
 Kicki Danielsson \textsc{(s.~\pageref{b4646e392dda635159575835254d4ef1})}
 Wo Fat

}

\small{
\textbf{Arbetslinjen}
\label{a7d4c1873c9542a1c6a48a1e52bdb823}
 är en tavla av den svenske skäggförsedde konstnären, sexköparen, examensfuskaren, låtsasforskaren och moderate \textsc{(se moderat s.~\pageref{c4564b188cb670841733a3ff923c2fb0})} politikern Sven-Otto \textsc{(se Svotto s.~\pageref{a54b74d16960ccfdc5c60c57fb0fe954})} Littorin \textsc{(se urin s.~\pageref{524fd7acb94f9c2d879b5c1cf8335669})} (1966-). Tavlan har också av dess skapare försetts med ett slags tagline, som följer: \quotetext{Arbetslinjen: Det som skiljer det inrutade, kontrollerade från det fria, från egenmakten.} Tavlan föreställer ett Piet Mondrian-inspirerat men avsevärt mycket fulare rutmönster som med en tydlig diagonal linje (arbetslinjen) skiljs av från en vacker, solstrålande sommarhimmel. Det rutiga fältet i tavlans undre del representerar alltså det inrutade och kontrollerade som årtionden av arbetsplatskamp, fackliga aktiviteter och vänsterpolitik skapat, medan den övre delen representerar nyliberal politik och en avreglerad arbetsmarknad. Det är oklart om Littorin också vill peka mot den himmel från vilken hans idoler inom NATO fäller bomber över fattiga araber.

}

\small{
\textbf{Arbetsplatskamp}
\label{76633c675b579e234a7705f5bba99244}
 Betydde förr att ge förmän och chefer kompanistryk. Betyder sen Saltsjöbadsavtalet \textsc{(s.~\pageref{8fb9ced0a7fc25125895e5496f9e95b8})} att sätta upp en \quotetext{Livsfarlig ledning}-skylt på chefens dörr.

}

\small{
\textbf{Arggissa}
\label{daf9816e5a975817604c64c6c515f098}
 Ett fenomen som seglat upp på tapeten de senaste åren är fenomenet att \quotetext{killgissa}. Detta innebär att en man tack vare sitt patriarkalkulturellt uppbackade snubbpatos tillåter sig att i alla lägen uttala sig tvärsäkert om allt mellan mullvadshål och vad rymden \textsc{(s.~\pageref{6d5ad1e8996d7ec9d8ac6058649290c0})} egentligen består av.

 Ett besläktat fenomen är att arggissa, vilket är en könlös företeelse. Arggissning är när en person ombeds att uttala sig om något den hatar. Låt säga att en person deltar i en dokumentärfilm om det lokala punkhuset. I intervjun arggissar personen att punkhuset aldrig fått en spänn av kommunen, eftersom personen har en negativ förförståelse av kommunal kulturverksamhet. Vad som är problematiskt med arggissningar är att de ibland inte stämmer överens med verkligheten. Till exempel kan det visa sig att kommunen i det hypotetiska exemplet ger ett bidrag på nästan 250 000 kronor om året till det lokala punkhuset. I såna fall är det lätt hänt att arggissaren känner sig  som ett klantarsle \textsc{(se Praktarsle (negativ) s.~\pageref{97a372c56edf8bade3fc4bdc4456f303})}.

}

\small{
\textbf{Arkivskita}
\label{6ef28e856f8167ccceff91f7bf5ce507}
 Att arkivskita är att gå ner i arkivet på universitetsbiblioteket \textsc{(s.~\pageref{e69cf3b6f7c7f3872fe561600a7e9aa7})} i Umeå och träcka på den fräschaste toaletten på campus. Det är alltid värt en omväg att gå till detta marint tematiserade avträde och vilskita \textsc{(s.~\pageref{d8991eedd83b1eb75ae7c2cf9daaad92})}. Bara några meter från hemlighuset ligger dessutom barnboksavdelningen så ny förströelse finns alltid att tillgå. Om man lite snyggt vill meddela kollektivet att man måste besöka denna anrättning kan en säga \quotetext{Jag ska gå och arkivera en grej}.

}

\small{
\textbf{Armlängd}
\label{515759efdfc5b2db913ca56e7ddc5a2d}
 Svårdefinierad måttenhet som är högst subjektiv då den alltid är lika lång som mätarens utsträckta arm. Således är en armlängd för dvärgen i Willow lika giltig som en armlängd för Boris Karloff. Kloakdjursvurmaren \textsc{(se Kloakdjur s.~\pageref{592254da9a0f26310a65ff83f6d73c9e})} tillika vetenskapsmannen Carl von Linné \textsc{(s.~\pageref{5e8380bf6b7ce99678e6752b6d9e709e})} jobbade länge för att standardisera armlängden, där en armlängd för en man skulle vara lika lång som hans egen arm och en armlängd för kvinnor skulle vara lika lång som drottning Marie Antoinettes arm. Detta förslag gick aldrig igenom eftersom bakåtsträvaren Ludvig XVI vägrade låta Linné utföra den noggranna, avklädda fysiska undersökning han menade att uppdraget krävde. I Sverige motsvarar dock en armlängd 1,5 alnar \textsc{(se aln s.~\pageref{b31e552786a58ecaff9efcfe53231ed3})} på en icke-vanskapt fullvuxen människa - det vill säga, en människa som inte har t-rexarmar \textsc{(s.~\pageref{0b2dbf0eb2888d887370538902e974d4})}.

 HEAD2: Armlängd inom sociologin
 Att hålla en person på en armlängds avstånd är väldigt stigmatiserande \textsc{(se stigma s.~\pageref{98410ec61c6964eac5c923a594841696})}, och detta beror oftast på kön, klass eller etnicitet. På senare tid beror det på alla \textit{samtidigt} \textsc{(se intersektionalitet s.~\pageref{6dc08633cdbf83eb418ea31ef0302c51})}.

}

\small{
\textbf{Arne Domnéus}
\label{86c80adfcac4ca926721e77e17204876}
 Arne \quotetext{Dompan} Domnérus (1924-2008) var en svensk klarinettist och altsaxofonist som förutom att vara professionell live- och skivartist medverkat i filmen \textit{Kvinnan som försvann} (1949). Ironiskt nog försvann också Arne Domnérus under ett trolleritrick utfört av Sveriges ärke-juggalo Carl-Einar Häckner på Gröna Lunds lilla scen i Augusti 2008. Domnérus förklarades död då han ännu inte dykt upp igen i Augusti 2010 och är nu struken ur kyrkoboken.

}

\small{
\textbf{Aron Jonason}
\label{7a5b98f8cc113bea7b6031de86fa7bf6}
 (1838 - 1914) var göteborgare \textsc{(se göteborg s.~\pageref{0e9b11e435dd9f73e87e868667e1d6f0})} nåt så jävulskt, och även Oscar II:s hovfotograf. Under en av upplagorna till brugd-racet \textsc{(se brugd s.~\pageref{d6b6b68506b8f1daad3a2ddbfaf8d863})} Orust Runt där Jonason jobbade med att fota kungen i fördelaktiga vinklar råkade han uppfinna den moderna ordvitsen. På den här tiden var fotografiet beroende av magnesiumblixt för att få verkligheten att fastna på film, och dess sken blir mycket intensivt. Kungen störde sig på detta och utbrast \quotetext{Det var hemskt vad Jonason blixtrar mycket!} och Jonason replikerade \quotetext{Ja, blixtrar den ene så åskar den andre.} Detta blev också hans sista ord innan han slukades av en brugd, men då Jonason trots allt var människa och inte plankton klättrade han ut relativt oskadd. Hans riktiga sista ord var i själva verket \quotetext{Lite kaliumcyanid har ingen dött av!}

}

\small{
\textbf{Arselhaka}
\label{9e4937897b8431412a9d8eb7561f5ec6}
 Genetisk betingelse som är få förunnat men mångas dröm.

}

\small{
\textbf{Artur Hazelius}
\label{cfe3ab83bbf192ab78a5b06cdd7cbf9f}
 Artur Immanuel Hazelius (1833 - 1901) är en av de mest produktiva skojarna i Svea Rikes brokiga historia, men mycket av hans repertoar av jävelskap \textsc{(s.~\pageref{46845591177f16920cd586a5baf5a625})} har fallit i glömska på grund av svenskens oförmåga att ta till sig sin historia. Bland de som fortfarande har Hazelius färskt i minnet, och av dessa har samtliga lärt om honom postumt, vet man att berätta att Hazelius sällan eller aldrig sågs utan sin käpp, vilken ofta spelade stor roll i hans streck och skojarfasoner.

 HEAD2: Barndom
 Redan i barndomen uppvisade Hazelius anlag för skojeri och båg. Redan som femåring ska han ha stulit en gardinstång från fattigstugan på Gärdet i Stockholm \textsc{(s.~\pageref{edcd259e0a03c7ab70feb186bae19f13})} och använt denna för att knacka på fönster på andra våningen runt om på Östermalm och på så vis skrämt slag på intet ont anande damer och herrar. Enligt ett brev från Hazelius till sin vän Anders Abraham Grafström (1790-1870) ska denna gardinstång ha legat honom så varm om hjärtat att han lät göra en promenadkäpp av den - och denna käpp var han aldrig utan i resten av sitt liv. Det berättas att han som tioåring ska ha använt käppen för att slå ner skatbon som han sedan placerade i skorstenar runt om i Gamla stan så att byggnader blev rökfyllda, ett spratt som i dagspressen omtalades som \quotetext{eld-och-svavel-eftermiddagen,} en referens till den bibliska berättelsen om Sodom och Gomorra vilket i sin tur medförde associationer till det faktum att Gamla stan på denna tid var centrum för prostitution och syndfullt leverne.

 HEAD2: Ungdomsår
 Under tonåren blev Hazelius utskickad i riket av sin fader, officiellt för att besöka släktingar och lära sig om landets historia, seder och bruk, inofficiellt för att inte betunga sina arma föräldrar med mer socialt uppseendeväckande tilltag än han redan gjort. Hazelius tog detta tillfälle i akt och satte igång en spiral av skojeri och spratt. Redan på tåget till Borlänge ska han ha använt käppen sin för att slå av hatten på folk som stod på perronger utmed resans väg när tåget startade efter att ha släppt på och av passagerare. Enligt lokala tidningar ska dessa spratt ha åtföljts av tillrop i stil med \quotetext{Länge leve den danske kongen} och \quotetext{Klang i Storkyrkans klockor!} inifrån kupén. Trovärdigheten bakom dessa  rapporterade citeringar bör ses på med viss skepsis, men antyder den infantila brist på koherens som Hazelius upptåg i ungdomsåren forfarande karaktäriserades av. Väl i Borlänge, sägs det, ska Hazelius med hjälp av sin käpp ha avlägsnat tuppen från Stora Tunas Kyrkas klocktorn och istället lagt dit en skjuten och avfjädrad fasan som han stulit från en av släktens vänner, Jägmästare Efrahim Andersson. Densamme utsattes för ännu ett spratt när Hazelius maskerad och iklädd militäruniform beordrade Andersson att stå vakt vid gärdsgårdsgrinden då armén enligt den förklädde Hazelius misstänkte en närstående anstormning av dansken och var underbemannad tills dess att man fick understöd från Stockholm. Andersson, som ömmade för sin byggd, ska ha intagit försvarsposition vid gärdsgårdsgrinden och stannat på sin post långt in på småtimmarna då han bara med ytterst stora mödor av socknens präst kunde övertalas att acceptera att han blivit utsatt för ett spratt och att den väntande anstormningen var ett påhitt, samt att prästen var just en präst och inte en dansk medlöpare.

 HEAD2: I vuxen ålder
 Det är dock i vuxen ålder som Artur Hazelius skojerier når sin fulla komplexitet, för vilket han trots allt gått till historien. Hazelius blev nu lektor vid Stockholm Högre Lärareseminarium där Hazelius allt som oftast for med osanning och lade krokben \textsc{(s.~\pageref{85c111491df5c9adeb8d907f3203238f})} för sina elever och kollegor medelst sin älskade käpp. Han författade här sin \textit{\quotetext{Swenska folkdräkter och lokala seder}}, ett hopkok av påhitt och lögner som man fotfarande in på nittonhundratalet använde som grundbok i etnologi och folkhistoria vid Uppsala och Lunds Universitet. Vidare ska han ha slagit sig ihop med en annan ökänd skojare, nämligen Gustav \quotetext{Frippe} Fredriksson, för att på Kungliga Dramatiska Teatern sätta upp en pjäs som annonserades som en odyssé genom svensk landsbygdshistoria. Vid premiären visade det sig dock att Hazelius och Fredriksson fyllt Dramaten med hönsfåglar och uvar \textsc{(se uv s.~\pageref{45210da832f9626829457a65e9e7c4d0})} som satte skräck i Stockholmsfolket. Åter igen kunde Hazelius få sig ett gott skratt på sina medmänniskors bekostnad.

 HEAD2: Ålderns höst och Hazelius död
 Hazelius skojerier fortgick fram till 60årsåldern, då han ska ha svindlat prinsessan Margareta i ett av Sveriges \textsc{(se Sverige s.~\pageref{b1999637949ed135b2ca03f3a38460cc})} första telefonsamtal. Hazelius utgav sig då för att vara Emir av kalifatet och tillstod att han hade vägarna förbi Sveriges huvudstad på sin resa till S:t Petersburg. Han förslog att prinsessan skulle möta upp med pompa och ståt vid Skeppsholmens angöringsplats, men då prinsessan och hennes ekipage anlände fanns där bara Hazelius som ska ha dunkat till henne i ryggen med sin käpp, vilket påstås ha orsakat njurproblem som i sin tur föranledde prinsessans gulaktiga hy. Till slut fick man nog av Hazelius i Stockholm och Sveriges alla andra hörn som besökts av denne skojare och skurk. Han infångades, fråntogs käppen och avlivades genom att han slängdes i en djup brunn. Hazelius kropp fördes till Skansen, som Hazelius för övrigt grundade (liksom Nordiska Museet ett stenkast bort) [http://www.sub.su.se/national/ttorb3.htm], och begravdes där under en hög med skräp \textsc{(s.~\pageref{75f1a5320951ea0dd9aa3c0eaba2c2c7})} i björngrottan.

}

\small{
\textbf{Arvo}
\label{f9ba22fb1ff25fd8f83d48f361aea0a4}
 Gnällig kvartersbyling som vevat batong mot tonåringar i åratal innan han så att säga sket i den korvbröda och fick ett kvastskaft i skallen.
 Efter det är han svårt traumatiserad och gråter ut i tidningarna varje gång försäkringskassan ifrågasätter hans långtidssjukskrivning.
 Konstigt nog vaknar ingen av de han misshandlat under åren upp skrikande på nätterna, vilket Arvo själv tydligen gör.

 Historien om kvastskaftet är ett gott exempel på poetisk rättvisa \textsc{(s.~\pageref{72fd78f6f034d86b7e7dc10501a38bfc})} vilken besjunges av Antirep \textsc{(s.~\pageref{eb508922f79a60f76b0278edfea25a8c})}.

 Ketegori: Fantastiska levnadsöden‏ \textsc{(s.~\pageref{47fabeb901b839a0658f0fb6722545fa})}

}

\small{
\textbf{Asbest}
\label{b11501ddf43dd8d60056510777981c68}
 är en stad i den ryska oblasten Sverdlovsk, belägen strax intill Uralbergen. Med en befolkning på ungefär 70.000 människor motsvarar det en medelstor svensk stad eller årsförbrukningen arbetare i Apples kinesiska fabriker. Staden har också gett namn till en mineral som är bra till i princip allt: billig, beständig, brandsäker och estetiskt tilltalande. Med ett hus klätt i eternitplattor försäkrar du dig om en släktgård som kommer stå stadigt längre än akropolis, vilket förmodligen inte gäller för den ryska staden.

}

\small{
\textbf{ASEA}
\label{61585ca988d62029b6ee6adab6066c34}
 är ett svenskt företag med över hundra år på nacken som är verksamt inom industrisektorn. Från början riktade företaget in sig på att enbart erbjuda sina tjänster till fascistiska mönsterstater. Den besatta längtan efter mer profit gjorde dock att företaget började tumma på denna regel och man har idag verksamhet över hela världen. Den nya målsättningen är att skapa maskiner som stjäl jobben av så många arbetare som möjligt på hänsynslösast möjliga sätt. Den största framgången hittills uppnåddes med Percy Barnevik vid rodret när han lade ner den sista masugnen i Norberg och gjorde typ en fjärdedel av ortens befolkning arbetslösa. För detta belönades han med Svenska Arbetsgivareföreningens (nuvarande Svenskt Näringsliv) pris \textit{Guldsvartfoten}.

 Företaget har numera gått ihop med ett Schweiziskt institut \textsc{(se Institut och tankesmedjor s.~\pageref{c276b5997d5af80504f79b30d121cf62})} och bytt namn till ABB, som en hyllning till Anders Behring Breivik.
 ASEA har även gett upphov till ASEA-grönt \textsc{(s.~\pageref{e5ce0e93ee9c54094b4ec1c2027272ca})}.

}

\small{
\textbf{ASEA-grönt}
\label{e5ce0e93ee9c54094b4ec1c2027272ca}
 Har du någon gång varit inne på kontoret hos en slips \textsc{(s.~\pageref{61c7bd51d579af09c10142f4b55c848c})} har du säkert noterat att stolstyget, skrivbordsunderlägget, kassaskåpet och skrivmaskinen går i samma pacifiserande nyans. Denna stavas NCS 6020 G10Y.

 Se även: färgskala \textsc{(se Färgskala:ASEA-grönt s.~\pageref{bfd16917b4c5964997f496f424382446})}.

}

\small{
\textbf{Asko}
\label{758389c936c1768837f762ea06b0c55b}
 är en pigg och rask man som luktar tabasco. Han går klädd i boots och polisonger och har en pung stor som en familjepåse chips.

}

\small{
\textbf{Aslång}
\label{0143cb521f10c31af0632adbb114999b}
 Om något är aslång är det väldigt mycket längre än förväntat. Ett exempel på aslånga grejer är Matrix-filmerna och Joels lår.

}

\small{
\textbf{Assar}
\label{86e0b4e52475becda19c1c3d18b10347}
 Alla barnen läste om styckmordet utom Assar för han låg i plastkassar.

}

\small{
\textbf{Astor}
\label{75fa7246cdfdcf9b16439d6296e7e979}
 är ett mansnamn som betyder \quotetext{gigantisk}. Vanligtvis syftar detta på skäggväxt, mankhöjd eller kagge.

}

\small{
\textbf{Astronomi}
\label{2a1623f7fff522d7b310d7c44ef207df}
 (ej att förväxla med pseudo-vetenskapen astrologi) är en av världens äldsta vetenskaper och har under årtusenden utvecklats till en extremt komplex sådan, då det finns en bisarr mängd stjärnor och luriga rymdfenomen att hålla reda på.
 I vardagssammanhang är astronomi mest användbart när man ska få folk att komma ner på jorden under högtravande diskussioner. Detta görs enklast genom att alltid ha med sig en karta över vintergatan som man kan ta fram när en väns resonemang börjar spåra ur och bestämt påpeka för henom att \quotetext{vi är såhär små, alltså såhär små...}. Det brukar fungera, ibland.

 Ett annat vardagssammanhang astronomin lämpar sig i, är när man vill dricka öl och lyssna på Pink Floyds skiva Wish you were here. Att sitta och pimpla bira själv och youtubea rymden med lite Floyd i bakgrunden går inte av för hackor.

}

\small{
\textbf{Atlantica}
\label{6a64780e5925164861e14768e0b04b1f}
 \textit{Atlantica} är namnet på en flera volymer lång rapport från ett kontroversiellt forskningsarbete av Olof Rudbeck publicerat under slutet av 1600-talet. Rudbeck argumenterar i verket för att Sverige \textsc{(s.~\pageref{b1999637949ed135b2ca03f3a38460cc})} är det kulturellt högtstående Atlantis som enligt många folksägner förstördes under förhistorisk tid. Bland annat påstår Rudbeck att den grekiska mytologins Herkules i själva verket var svensk och att namnet kom från svenskans \textit{Här} + \textit{kulle}. Även om \textit{Atlantica} då den publicerades mottogs med entusiasm, speciellt inom sverige, och fick bifall från svenska likväl som utländska skiftkunniga (vilka den alltid ödmjuke Rudbeck lät samla i ett band med titeln \textit{Testemonia}) tvivlar vissa forskare idag på Rudbecks slutsatser. Denna pessimistiska attityd hos humanister och samhällsvetare visar än en gång på behovet av en handlingskraftig och bestämd minister som Jan Björklund \textsc{(s.~\pageref{0b9b757044804b9be0e218acdad358cc})}, som kan vrida tillbaka utvecklingen till 1600-talet och återskapa det mer visionära klimat som rådde vid landets läroverk på Rudbecks tid.

}

\small{
\textbf{Atomångest}
\label{16df915e34a40562db7fab59c10ff5d9}
 var ett punkband från Fagersta som tyckte att folk skulle kasta gatsten \textsc{(se irländsk konfetti s.~\pageref{149459a4b475b90c8513551228efc472})}. De sa aldrig på vad.

}

\small{
\textbf{Att psykedelisera sin vardag}
\label{ac0a9b23e06116650a85505c85f16fd2}
 är ett häfte utgivet av Prof. Etiennes \textsc{(se Användare: Prof. Etienne s.~\pageref{a9878d2280e5a39becac8f73d113df91})} far Zarathustra Etienne under 60-talets allra gladaste hippiedagar \textsc{(se hippie s.~\pageref{14fd61fa8edcb67c5c7886f11af8431e})}. Skriften syftade dels till att öppna människors sinnen för \quotetext{softa prylar \textsc{(se smutt s.~\pageref{d9114ffee4f2dcee302ae2b19ce5eea9})}} och dels till att låta Etienne D.Ä. titulera sig författare så han kunde åka på uppläsningsturnéer med fri sprit och groupies. Häftet är slutsålt sedan länge och hela exemplar utan vin- och meskalinfläckar betingar idag höga summor på Ebay och Flashback. I Danmark \textsc{(s.~\pageref{5331d7fd27772396f412a5b6d19bad44})} fick texten sådant genomslag att anhängarna till Etienne D.Ä:s tankar länge hade status som minoritetsfolk. Statusen slopades 1996 när gruppen snarare ansågs utgöra majoritet.

 HEAD2: Etienne D.Ä:s 10 hetaste tips till en mer psykedelisk vardag
 \begin{itemize}
 \item \textbf{Gå passgång}. Även om det så bara är från soffan till klo \textsc{(s.~\pageref{642b921dce80b8405b7f32f9974c5a40})} så låt inte konventionerna hindra dig från att spacea \textsc{(se rymden s.~\pageref{6d5ad1e8996d7ec9d8ac6058649290c0})} till din resa.
 \end{itemize}

 \begin{itemize}
 \item \textbf{Tilltala allt och alla med Herr Hermelin}. Tidigare generationer kan inte styra dina tankar. Vill du att allt ska heta Herr Hermelin så heter det så.
 \end{itemize}

 \begin{itemize}
 \item \textbf{Använd tejp istället för kniv och gaffel}. Be en kamrat vira lite gaffa runt dina händer med den häftande sidan utåt och frukosten blir sig aldrig lik.
 \end{itemize}

 \begin{itemize}
 \item \textbf{Sov i vatten}. Återgå till rötterna då vi alla var mikroorganismer som bara flöt runt och varken kunde tänka eller agera. OBS! Kom ihåg flytvästen \textsc{(se flytväst s.~\pageref{ad12637b03f6d664b47cff669471387f})}.
 \end{itemize}

 \begin{itemize}
 \item \textbf{Ha fransar av matrialet mocka}. Alla ytor på din kropp är potentiella fästpunkter, liksom alla ytor på ditt eventuella fordon.
 \end{itemize}

 \begin{itemize}
 \item \textbf{Klä ut dig till ett djur \textsc{(se Klä ut sig till ett djur s.~\pageref{78663eff2fe898143e822b7f9d4851f7})}}. Djuren tänker inte som oss andra. Dom gör som dom vill. Kura ihop dig bakom soffan med tigersmink i ansiktet och tänk att utrymmet är ett gryt \textsc{(se fördelar med att bo i gryt s.~\pageref{87f4e00149377054658d1b0851c4718b})}.
 \end{itemize}

 \begin{itemize}
 \item \textbf{Klä ut dig till Ray Jones IV \textsc{(s.~\pageref{4e526c431120b692a2dc2fc9aa8612be})}}. Han är lugn och samtidigt är han förmodligen kung. Res i dig själv genom att vara någon annan på samma gång.
 \end{itemize}

 \begin{itemize}
 \item \textbf{Använd stora mängder knark \textsc{(s.~\pageref{bebc5e7342ca2f076b3d32ed6c557398})}}. Gå till jobbet påtänd \textsc{(se stenad s.~\pageref{dec4a3a91f0f2bf8dcf033a8cfeaa554})}. Du har aldrig sett något liknande och det har inte dina arbetskamrater heller, vilket gör att revolutionen sprider sig direkt.
 \end{itemize}

 \begin{itemize}
 \item \textbf{Bli ett deadhead \textsc{(s.~\pageref{30563e7c77afbd00a3aafa07829c95d3})}}. Varje deadhead är ett mikrokosmos i det makro som stavas Greatful dead. Du är en del av ett annat universum men som lever här på jorden.
 \end{itemize}

 \begin{itemize}
 \item \textbf{Bli helt orimligt besatt av Kosta Boda-konstnären Monica Backströms glaskonst}. Din levnadsyta kan aldrig bli för full av lampetter, vaser, skålar och paraplystativ av glas - formade som svampar.
 \end{itemize}

 \begin{itemize}
 \item \textbf{Fantisera}. I fantasin bestämmer du allting. Vad som helst kan hända men vill du inte kan du bestämma att det inte ska göra det. Fuckin' groovy, man.
 \end{itemize}

 \begin{itemize}
 \item \textbf{Sitt \textsc{(se sitta s.~\pageref{123c3e95c62201513a344526a2fec502})} i saccosäck}. Säcken formar sig efter din kropp fastän den egentligen är en stol. Ungefär som att du skulle hoppa runt ett hörn.
 \end{itemize}

 \begin{itemize}
 \item \textbf{Tillverka ett par glasögon av två kalejdoskop}. Denna sorts glasögon kallas ofta för danska standardglasögon och är för vardagliga ändamål fullkomligt värdelösa. Men du är inte ute efter det vardagliga och med dessa glasögon har du alltid ett ständigt föränderligt mish-mash framför dig.
 \end{itemize}

 \begin{itemize}
 \item \textbf{Rulla upp en kanelbulle och använd den som skärp}. Mest för att det är ett ovanligt att göra ett skärp av kanelbullemateriel. Men också för att kanelbullen sällan används som skärp.
 \end{itemize}

 \begin{itemize}
 \item \textbf{Spela kristen folk-psych}. Det finns gott om religiöst folk som inte vill annat än att resa land och rike runt och sjunga om flummiga saker. De har sällan eller aldrig några invändningar när du föreslår att folk-psych kan vara den mest lämpliga formen för att föra ut ert (deras) budskap. Och att älskog är gudomlig njutning.
 \end{itemize}

 Som synes utgörs Etiennes 10 tips egentligen av 15 punkter men eftersom Jesus \textsc{(s.~\pageref{110d46fcd978c24f306cd7fa23464d73})} bara hade 10 budord ville han också ha det. Den som inte är psykedelisk nog att förstå denna förklaring är inte heller redo för frälsningen.

}

\small{
\textbf{Att vara så hög på psykedelisk svamp att man entusiastiskt diskuterar med en knotig masurbjörk}
\label{d7ae40e0f7cb2c530987cac136fd8428}
 är en upplevelse som är lika fascinerande som olaglig. Den inger perspektiv på den för många så diskvattenfärgade vardagen. Vissa påstår att denna upplevelse har samma effekt på livet som valet att koppla över från en repris av en av Colin Nutleys \textsc{(se Colin Nutley s.~\pageref{b7e4eb146052f2edb273b55e35f4f078})} föraktbara filmer till första Stjärnornas Krig-filmen har för TV-tittandet.
 För att uppleva denna fantastiska erfarenhet, fullfölj nedanstående uppgifter:

 \begin{itemize}
 \item Res till Mexiko
 \item Tala med \textit{Eduardo \textsc{(s.~\pageref{6d6354ece40846bf7fca65dfabd5d9d4})}}
 \item Betala honom
 \item Kräma utan omsvep upp den lilla påsen \textit{Eduardo} ger dig i ändtarmen
 \item Res hem med tillmötesgående uppsyn
 \item Vid ankomst till Sverige: \textsc{(se Sverige s.~\pageref{b1999637949ed135b2ca03f3a38460cc})} Berätta inte för tullpersonalen att du har olaglikt knark uppstuvat i arslet. Vi i nissepedias \textsc{(se nissepedia s.~\pageref{62400dadecd90cb5cd39062abe5a3e4a})} obetalda arbetslag kan inte nog betona vikten av just detta steg i processen.
 \item Uppsök en solig skogsglänta och ät samtliga svampar
 \end{itemize}

 OBS!!: Nissepedia uppmanar på inget vis sina läsare till att vara så höga på psykedelisk svamp att de entusiastiskt diskuterar med knotiga masurbjörkar

}

\small{
\textbf{Attackspya}
\label{0a9f7d0a04593d9aade146f54000cdd8}
 En attackspya tar man till när man är i ett väldigt trängt läge. Kanske när man vill fly undan herr konstapel. Man liksom slungar sin kropp fram och stöter upp rakt i ansiktet på sin angripare. Detta är ju sjukt äckligt så man undkommer typ jämt.

}

\small{
\textbf{Augustibuller}
\label{6072997c8cf781008ad5748003565c8a}
 Sketen punkfestival som existerade mellan 1996 - 2007. 1996-2000 var det gratis och de som var där var mest där för att kolla på när sina polare gjorde bort sig på scen, förutom år 2000 när två killar kom för att heila på Charta 77-spelningen.
 2001-2007 kostade det stålar att gå in, vilket gjorde att alla träskpunkare \textsc{(s.~\pageref{484838b3db1adb135ea74d6fc61e44c0})} plankade och de flesta som betalade in sig sket i att kolla på banden, vilket var rätt skönt för de få procent som ville se ett band.

}

\small{
\textbf{Aurora}
\label{99c8ef576f385bc322564d5694df6fc2}
 är ett varumärkesskyddat namn på punksaft.

}

\small{
\textbf{Australien}
\label{e727d8d1b3162a732c7f706d55de64f3}
 är ett land i Australien. Ingen annan världsdel ville ha det så det fick bli en egen. Det utmärks främst för sin stora grad av konstighet och var för brittiska imperiet ungefär vad Danmark \textsc{(s.~\pageref{5331d7fd27772396f412a5b6d19bad44})} är för Norden.


 HEAD2: Historia
 Ursprungligen befolkades Australien av ett storband som hette \textit{Aboriginerna} där alla medlemmar spelade psykedeliska dronetoner på didgeridoo, sittandes i en klippskreva i Ayers Rock. Ett vanligt rep \textsc{(se repet s.~\pageref{0714379932aa997070168553fe416a96})} tog ungefär en vecka att genomföra och en konsert upp emot en  månad. Dom va fetnöjda med sin tillvaro men plötsligt kom några engelska skepp seglandes på jakt efter en lämplig plats att dumpa alla jobbiga skottar \textsc{(s.~\pageref{c2e5f84c76d823ea9482387bfb950791})} som bara satt hemma och lirade drone på sina säckpipor. Dronemusik var nämligen jävligt ute i England. Så där satt man mest hela dagarna, tutande i sina lurar och njutande av långa släpande dronetoner, tills någon kom på känguruboxningen. Då började alla med det istället.


 HEAD2: Ekonomi
 Australiens ekonomi är relativt god eftersom man har en skitstor ö med nästan alla typer av naturresurser och det inte finns något annat land på ön som kan kriga om det. I städerna jobbar de flesta människor med att köra trucks fulla med får och på landsbygden jobbar majoriteten med att skaffa sig en redig dagsfylla \textsc{(s.~\pageref{e79459471993abd0ccde4df08bafdb22})}. Den vanligaste valutan \textsc{(se valuta s.~\pageref{cf1e2a0af4955aa7539b6e12e9d282e6})} i Australien är AC/DC-plektrum, ett från Malcolm Young är värt 10 och ett från Angus 20. Wombats accepteras också på de flesta ställen och dom är värda 100. Exporten består främst av Vegemite, ett pålägg som görs av ett jästextrakt som uppstår som en slaggprodukt vid tillverkning av öl.


 HEAD2: Geografi, klimat och miljö
 Alla typer av klimatzoner finns i Australien men det mesta är öken. För att det inte ska se så tråkigt ut ställer man ut en massa får där, det är det man har sina trucks till. Man har också små populationer av en massa konstiga djur som är till för djurprogrammen i TV. Landet är indelat i sex delstater utan att någon riktigt vet varför.


 HEAD2: Fauna
 Australiens fauna karakteriseras av tre saker; farlighet, konstighet och svaghet. På Australien bor nämligen typ 90\% av alla världens giftormar, skorpioner, spindlar, havsanemoner och en massa djur som också finns på andra ställen men finns på Australien i en giftig variant. Här finns även världens enda giftiga däggdjur, nämligen näbbdjuret. Denna krabat representeras också den konstiga delen av faunan tillsammans med euchidnan, kängurun, koalan, tasmanska djävlar, emuer och kockaburrafåglar. Dessa djur utmärker sig med att de bara är konstiga, men precis som alla andra djur försöker de bara överleva och fortplanta sig, så låt dem va! Denna på gift och konstighet byggda fauna är dock lika stabilt som ett korthus i orkan. Kommer en annan art in i den australiensiska faunan så rämnar hela skiten. I resten av världen ofarliga djur som kaniner och oxgrodor har i Australien ätit upp grödor så att de hamnat på gränsen till svält. Det här beror på att de inte har några naturliga fiender på ön, och det här kan tyckas lite märkligt då alla av de jävulskt giftiga djuren borde kunna ta kål på en gullig kanin. Men icket.

 HEAD2: Kultur
 Musiken i Australien består mest av \textit{AC/DC} men \textit{Midnight Oil} finns också. På idrottssidan har man uppfunnit en massa egna sporter för att man ligger så långt bort och inte orkar lära sig reglerna till dom som resten av världen tävlar i, inget land orkar ändå åka så långt bort för en landskamp. Landhockey till exempel håller nog ingen annan nation på med seriöst. Film finns inget att berätta om för det har dom inte kommit på där än. Detta är också anledningen till varför skådespelare som Russell Crowe, Nicole Kidman och allas vår Mel Gibson tvingats söka sig till andra delar av världen för att få ihop till lite bröd och mjölk att ställa på bordet.

 HEAD2: Kokkonst
 I Australien grillar de inte räkor (shrimp) utan den lite större sorten som på engelska kallas prawns. De dricker inte heller ölet Foster's, utan dricker uteslutande Victoria Bitter.

 Vill man veta mer om Australien kan man fråga  Jons polare Nicke \textsc{(se Användare:Jons polare Nicke  s.~\pageref{2de727011f24f1e63e85612e859a47d8})} för han har varit där.


 Källa: Utrikesdepartementet

}

\small{
\textbf{Avancerade glasögon}
\label{af3e24904a76537ca385e42fe952fa6c}
 Folk som bär avancerade glasögon bör bemötas med misstro.

 Om bäraren är man och dessutom doftar herrparfym \textsc{(se luktagott s.~\pageref{f9613f1654fe61d6a5c0787c85daeeaf})}, handhälsar och har blekta tänder-skriv inte på något!


 Om bäraren är kvinna och jobbar på någon slags myndighet och gillar multikulti \textsc{(s.~\pageref{25eea9148080d30d384ce1c1277ef126})} är det bara att backa.


 HEAD2:  Se även
 \begin{itemize}
 \item Kommunistglasögon \textsc{(s.~\pageref{1bc58f6f6a934c05a63add653dbeadf0})}
 \end{itemize}

}

\small{
\textbf{Avbolagisering}
\label{58b3f1fd3e8dc4bd01e05c07628b3409}
 Återbördande av folkets stulna resurser.

}

\small{
\textbf{Avgå}
\label{8e534fae0a7fa17a231d2791b5c89e75}
 Att avgå är att träda ner från en post man blivit tilldelad eller intagit på eget bevåg, ofta av politisk natur. Som uppmaning är det det snyggaste sättet att avsluta en insändare eller ett samtal till Ring P1 \textsc{(s.~\pageref{4db71b60775c55748348514df36a155d})} på.

 Följande personer borde avgå:
 \begin{itemize}
 \item Alla
 \end{itemize}

}

\small{
\textbf{Axe}
\label{c89ee4a2c9dcae6581ca53bff9aa4ee5}
 är en kroppsdeodorant speciellt framtagen för män. Den finns i utförandena unlimited, touch, Africa, pulse och phoenix. Vilken som är den bästa axe-sorten är inte tydligt och en gång för alla klarlagt. Alla måste göra sig sin egen uppfattning. Axe leverans i stryktålig aluminiumburk och sprutas på och i kroppsdelar som luktar illa.
 HEAD2: Försäljningsstrategi
 Axes målgrupp är i första hand hockeyspelare \textsc{(se hockey s.~\pageref{df0349ce110b69f03b4def8012ae4970})} och pojkar som går på högstadiet. Vad dessa har gemensamt är att de aldrig får ligga med frivilliga. Därför låter man från företagets håll bedyra att den som använder axe genast blir ett glödhett sexmonster som ingen kvinna kan emotstå. Än är det inte vetenskapligt bevisat att detta är sant, så ord står mot ord.

 HEAD2: Trivia
 Axe betyder yxa \textsc{(s.~\pageref{bd74f429522c7c1481fbba07187efc6b})} på engelska, men i England heter deodoranten Lynx, vilket betyder lodjur \textsc{(se Hur man ritar ett snyggt lodjurshuvud s.~\pageref{85f12831da9d7403326be028c34be8a9})}.

}

\small{
\textbf{Axess tv}
\label{c260511560d4554345f8670625b9ffb8}
 Tv-kanalernas Rusta.
 Borgerlig propaganda så odiskret att den får DDR-tv att framstå som saklig och objektiv.
 Gratis bara för att ingen skulle betala för eländet.

}

\small{
\textbf{Axtorpet}
\label{11f14a5854300c512ed501986f38a609}
 är vad Umeå-stadsdelen Berghems mest centrumliknande del kallas i folkmun, men är egentligen namnet på den spel-, porr \textsc{(se pörr s.~\pageref{5faa435e2f0af7617816f0cade262581})} och tobaksaffär (billigaste stocken i stan!) som ligger där. Brevid spel-, porrtidnings- och tobaksaffären ligger en kinesresturang med en kapacitet för uppskattningsvis fyra pilsnergubbar åt gången, men på sommaren ställer man ut ett vitt platsbord så då får det plats sju gubbar (bordet står mot väggen). Bredvid kinesen ligger Vänsterpartiets lokal och ett äldreboende. Lite längre bort har vi två loppmarknader varav den ena säljer nazi-memorabilia och den andra hutlöst dyra (troligtvis stulna) cyklar.

}

\small{
\textbf{Ayn Rand}
\label{08bc6fbe57f6ad5c154a3d93954fbb9b}
 Galen kärring som tyckte att folk skulle få göra som de ville förutom att betala skatt eller bry sig om andra människor på något som helst sätt. Hon ville t.ex. röka nåt så kopiöst och gjorde därför det, fick lungcancer och dog tack vare att hon stod utan allmän sjukvård. Poetisk rättvisa \textsc{(s.~\pageref{72fd78f6f034d86b7e7dc10501a38bfc})}.

 Anton Lavey \textsc{(s.~\pageref{869cf213daf268853824c26db9960ab7})} sällade sig, tillsammans med bland andra Margaret Thatcher \textsc{(s.~\pageref{0bdaa2c5b2f4fb15d678c3e54c10d347})}, till hennes mest namnkunniga beundrare. Frågan vem av de två som är ondast är en svårt nöt att knäcka. Även gamle världsbankschefen Alan Greenspan är ett fan.

}

\small{
\textbf{Baam, whaam, thank you ma'am}
\label{8ccc8c5370e85cd0f19db2d56b3c1cab}
 är den fetaste raden i MC5s låt Rocket Reducer No. 62.
 Näst efter den kommer \quotetext{rama lama fa fa fa!}

}

\small{
\textbf{Babs Kramer}
\label{a22cc8726ce747ca28a4b080acd0ecdb}
 är mor till karaktären Cosmo Kramer i TV-serien Sienfeld. Hon arbetar som toalettvärd när hon först dyker upp i serien, men slutar efter påtryckningar från sin son. Hon har också en sexuell affär med huvudpersonen Jerrys antagonist Newman.

}

\small{
\textbf{Backa med släp}
\label{2e9a3db5d49f5a13e74bbfa2a3105c3d}
 Att backa med släp är en modern initiationsrit in i mandomen. Tricket är att släphelvetet svänger åt motsatt håll som bilen \textsc{(se bil s.~\pageref{b3188f47d2eac7efc3f1258dc673a9fe})} gör när man backar \textsc{(se backa om s.~\pageref{1f2b57b534649cc5f703f428ee6fe16c})}, som är motsatt håll när man kör framåt, och tvärtom, beroende på perspektiv \textsc{(s.~\pageref{1606dd19366985367d677f7b6de46e52})}.

}

\small{
\textbf{Backa om}
\label{1f2b57b534649cc5f703f428ee6fe16c}
 är att ta en till portion av något man just ätit. Vanligtvis backar man om när man är hemma hos någon annan för att signalera att det smakade gott. Det är väldigt ovanligt att man backar om på lyxrestaurang för då blir det dubbelt så dyrt, och så mycket pengar är det inte många som har. Äter man däremot på rekorderlig lunchrestaurang där gästerna har byxor med mycket verktyg; är det närmast praxis att alltid backa om.
 Är man hemma hos mormor eller farmor måste man backa om minst en gång, helst två. Att backa om har ingenting med bilar att göra, även om det förstås vore väldigt coolt.

}

\small{
\textbf{Backpatch}
\label{f33ef45f7ffbb637aaaf00c3a2073c1e}
 En backpatch är ett oftast rektangelformat tygstycke med tryck eller broderi som för tankarna till ett visst band eller skiva. Den ses vanligen fäst vid en jeansväst inuti vilken man allt som oftast hittar en hårdrockare \textsc{(se hårdrock s.~\pageref{a4566a810e7ad85a57ddc84083a8139b})}. Vissa, mycket få, backpatches uttrycker en åsikt, uppmaning (fuck the world) eller bärarens gängtillhörighet (Mölndal Hogs MC). De bästa backpatchesen innehåller orden \quotetext{Manowar \textsc{(s.~\pageref{ac62eaec6dc3e81da86dfbb5252c0ffc})}}, \quotetext{Saxon}, \quotetext{AC/DC \textsc{(se australien s.~\pageref{e727d8d1b3162a732c7f706d55de64f3})}} eller liknande och till yttermera visso en fantasieggande bild av en barbar med svärd och lättklädda ungmöer som klänger sig fast kring dennes bepansrade fötter. Kanske flyger han fram på en Harley Davidson med eldhjul? Endast fantasin sätter gränser.
 HEAD2: Tillverka en egen backpatch
 Köp ett kilo bintje \textsc{(s.~\pageref{f21f4f64cb0df1775b5c2a7dc0d83c6c})} på ett välsorterat varuhus och hala ner kommunflaggan utanför kommunhuset. Skär försiktigt ut följande bokstäver ur potatisen: A, N, T, H, R, ett till A och slutligen ett X. Måla potatisen och tryck bokstäverna i ett hörn på flaggan, som du sedan klipper ut och syr fast på din jeansväst.

}

\small{
\textbf{Bacon}
\label{7813258ef8c6b632dde8cc80f6bda62f}
 är rökt och rimmat sidfläsk som vanligen steks i tunna \textsc{(s.~\pageref{00f1b109163e2c7e424e60cda2354c55})} skivor. Det förekommer i frukostsammanhang, framför allt i England och USA (då ofta tillsammans med stekta ägg eller äggröra), på pizza, kebab, tacos och i hamburgare.

 Börjar man stekningen i sval panna får man mjukstekt bacon, men lägger man i skivorna först sedan pannan blivit ordentligt het, får man knaperstekt bacon. Om brandlarmet går vid stekning vet man att man lyckats.


 HEAD2: 140 g
 Det har länge dryftats om varför bacon säljs i förpackingar om just 140 gram.

 En tänkbar förklaring är att det är nära ett tredjedels (svenskt) skålpund = 141,67 gram.

 Det kan också vara för att 140g blir 7 skivor. 20 g styck. Det faktum att sju är ett primtal gör det ytterst svårt att dela förpackningens innehåll rättvist. Det är alltså ett sätt att tvinga konsumenten att köpa fler förpackningar än ett för att kunna dela rättvist om man inte är en eller sju personer i hushållet.

 5 oz är ungefär 140 g. Bacon är något vi förknippar med den engelsktalande världen, inte minst USA. Därför är det troligt att vi även adopterat amerikanska mått vad det gäller förpackningen.

}

\small{
\textbf{Baddaren}
\label{30b7bf427e64d4c140b3fc394b601589}
 är ett av de märken som utfärdas av simförbundet efter godkänt prov. Baddaren finns i två olika utföranden, blå respektive gul. Vissa tycker att det är väl enkelt att ta baddaren, varför många elitsimmare inte alls bryr sig om att ta märket.
 HEAD2: Baddaren blå
 För att ha gjort sig förtjänt av baddaren blå ska märkestagaren ha genomfört godkänt:
 \begin{itemize}
 \item \textbf{Dopp-prov}.
 \end{itemize}
 Märkestagaren skall doppa huvudet under vattnet fem gånger.

 \begin{itemize}
 \item \textbf{Bubbelprov}.
 \end{itemize}
 Märkestagaren skall andas in, hålla
 andan och blåsa ut under vattnet.
 Detta skall upprepas fem gånger.

 \begin{itemize}
 \item \textbf{Glidprov}.
 \end{itemize}
 Märkestagaren skall här \quotetext{glida} i vattnet 5 sekunder med framsträckta och raka armar, ansiktet skall vara
 under vattenytan. Frånskjut får göras och provet kan utföras där märkestagaren bottnar. Handstöd med platta eller simdyna får
 användas. Momentet upprepas fem gånger.

 HEAD2: Baddaren gul
 Nu har vi kommit till baddaren gul och det hela blir genast svårare.
 För att få baddaren gul måste märkestagaren ha utfört godkänt:
 \begin{itemize}
 \item \textbf{Hopp-prov}.
 \end{itemize}
 Märkestagaren skall hoppa i vattnet
 fem gånger.

 \begin{itemize}
 \item \textbf{Bubbelprov}.
 \end{itemize}
 Märkestagaren skall andas in, hålla
 andan och blåsa ut under vattnet.
 Detta skall upprepas fem gånger.

 \begin{itemize}
 \item \textbf{Glidprov}.
 \end{itemize}
 Märkestagaren skall glida i vattnet
 5 sekunder med framsträckta och
 raka armar, ansiktet skall vara
 under vattenytan. Frånskjut får
 göras från kanten. Handstöd med
 platta eller simdyna får användas.
 Momentet upprepas fem gånger.
 Samtliga prov utförs där
 märkestagaren inte bottnar!

}

\small{
\textbf{Bade}
\label{2732b87a2aaf51c364aad093ed6a67b6}
 Att bade är att bada genom att en yngre man slingrar omkring naken på en äldre mans rygg samtidigt som denne simmar som vanligt.

 Exempel: \quotetext{Nä du Emil, nu går vi och bader!} [http://www.youtube.com/watch?v=srPputONMt4]

}

\small{
\textbf{Bag-in-box}
\label{1fdd5e1bb07154385669cd70e53bd354}
 , också känd som bagina, bib och lådvin. Utan denna kreation hade ingen druckit vin \textsc{(s.~\pageref{62911ad86d6181442022683afb480067})} idag.

}

\small{
\textbf{Bailando}
\label{cc749d093e7adb1e0213c7b9af296bc2}
 är en belgisk ringsignal som satte halva Europa ur spel 1996 och ett litet tag därefter - nämligen den del av Europa som hellre grillar körv än botaniserar i musik. Det är mycket möjligt att den kommer att bli sjukt poppis bland hipsters inom en snar framtid så vill man invistera har man chansen nu.

}

\small{
\textbf{Bajsalåkta}
\label{63d23d12d5a32dde5649d2f7cc5a76a3}
 är Sveriges \textsc{(se Sverige s.~\pageref{b1999637949ed135b2ca03f3a38460cc})} högst belägna mulltoa på 1226 m ö h. Den är belägen på Låktatjokka fjällstation nära Björkliden, Norrbotten. Ett besök på denna är ofta en schysst omväxling från att bajsa bland stenrösen, en vanlig praktik i det stiglösa land som omger Låktatjokka. Dock ska det sägas att det även här drar lite om skinkorna. Som alla mulltoer finns det förbehåll och dessa är framförda på rim (typ).

 \quotetext{\textlessi\textgreaterRespektera den naturliga processen
 släng endast papper i toaletten\textless/i\textgreater}

}

\small{
\textbf{Bajskorvar i nacken}
\label{f2714801a00494f2ef070519d2ffd3df}
 Crustnacke

}

\small{
\textbf{Bakficka}
\label{d259b5ebe8541b74129f0c78a82335b7}
 En bakficka är en ficka på baksidan av ett par byxor. Byxor med bakficka är vanligtvis av typen jeans eller chinos. Vad som gör bakfickan speciell mot andra fickor är dess förträfflighet som förvaringsutrymme under längre perioder. Medan andra fickor vanligtvis förvarar ett föremål under några timmar eller dagar kan bakfickan enkelt lagra saker i flera månader utan att det stör dess bärare.


 Typiska föremål lämpade att förvaras i bakfickan för \textsc{(s.~\pageref{5a98c81c7b5b60a5777a92b943f53a41})} att alltid finnas till hands kan vara: viktiga papper \textsc{(s.~\pageref{810193ff4e7ae05223a81e960d806ddf})}, fiskelina, en pennstump, plånbok \textsc{(se hästhandlarplånbok s.~\pageref{2f8fbda5296f2f6cab04d88082ed9015})}, snusdosa och isskrapa.


 För den som vantrivs i byxor rekommenderas i stället magväska för att enkelt kunna bära med sig ovan listade nödvändigheter.

}

\small{
\textbf{Bakisångest}
\label{85c5f4a59ce183d8b1d4ac4ffaa43764}
 Ahh, bakisångest. En grå slöja som läggs över ens ansikte dagen efter man haft en spritfylla \textsc{(s.~\pageref{0668c687b51995118ec27cbf25061118})}. Allt man ser går i ett skimmer av menlöshet och det känns som att hjärtat halkar neråt i bröstkorgen, som en loska på en betongvägg. Man kan försöka trösta sig med att det bara är kemiskt (om man inte gjort något askefft), men det hjälper sällan. Ofta sitter bakisångesten i någon dag eller två efter att den fysiska bakfyllan upphört. Det finns inget knep för att undfly bakfylleångest. Så försök inte ens, din tönt.

}

\small{
\textbf{Ballerina}
\label{c775dd5f303797621d8e2a8369546b27}
 är en köpeskaka som man inte får så ofta, men som samtidigt inte är så långt ur räckhåll (som t.ex dammsugare el. sk. punchrullar) att man inte i alla fall kan hoppas på att den som gjort ens matsäck har valt att bre på lite extra. Ballerina, som sin syster, Singoallan, består av två kakdelar med fyllning mellan. Ballerinans fyllning är gjord av ett slags chokladsmet. Den ena av ballerinans kakdelar är ljus och har formen av en tjock cirkel, medan den andra är mörk och erbjuder hela kompositionen en rejäl bas. När man äter ballerina är det viktigt att man först frigör den övre delen av kakan genom att liksom skruva av den från fyllningen, som sedan antingen skopas upp medelst den övre kakdelen, eller helt sonika skrapas av med framtänderna. Vissa, mycket löjliga, människor envisas med att efter att ha ätit fyllningen klappa ihop de två kakdelarna och äta dessa tillsammans, likt en smörgås av kakdeg, men detta är som antytts bara fånigt.
 \textbf{Kuriosa}: Trallpunkbandet Dr Herman ifrån Piteå besjöng denna kaka, bland annat i textraden: \quotetext{\textit{Vissa gillar knark, andra alkohol men jag, jag tar en ballerina}}

}

\small{
\textbf{Balticgruppen}
\label{9cf3b30a8e655c9893ae8d71505e72ea}
 är en landburen gren av de Somaliska sjörövare som satt adenviken i skräck de senaste åren. Till skillnad från dessa somaliska släktingar har Balticgruppen Västerbotten \textsc{(s.~\pageref{d4b008c5143dcffb6b8c35f3876c2a19})} och Umeå som huvudområde och istället för sjöröveri ägnar man sig åt entreprenad \textsc{(s.~\pageref{2d3b60492ed3cebe0a3cf341bc5b20b5})}. Annars är likheterna slående: Likt de desperata unga män från Afrikas horn vilka ser pirateri som en väg ur svälten har man tagit 111.000 Umebor som gisslan och kapar med denna förhandlingsfördel till sig byggkontrakt och gåvor i form av rökelse och myrra framburna av Umeå universitets ledning. Balticgruppens tentakler når ända in i den politiska maktens toppskikt, det vill säga kommunalrådet Lennart Holmlunds \textsc{(se Lennart Holmlund s.~\pageref{26d063a59c90487b11c8f5b4fa9af348})} bastu, i vilken lösensummorna kamratligt diskuteras medan khat-bladen går runt bland de inblandade och får dem att slutligen omfamna varandra under galna gapskratt. Stråtrövarbandets ledare är en enigmatisk figur som enligt sägnen går under namnet Krister \quotetext{Svartskägg} Olsson, som sägs rida fram på en vit häst med brinnande ögon. Och helvetet följer honom.

}

\small{
\textbf{Balutägg}
\label{8561722057ba8ec26075477dab5ef4de}
 är en österländsk delikatess. Den görs av ett inseminerat, men inte färdigruvat ägg. Kort och gott ett fågelfoster. Traditionellt används ank- eller hönsägg, men vi på Nissepedia \textsc{(s.~\pageref{62400dadecd90cb5cd39062abe5a3e4a})} ser ingen anledning att inte göra balut av uv- \textsc{(se uv s.~\pageref{45210da832f9626829457a65e9e7c4d0})}, svan- \textsc{(se svan s.~\pageref{f80f1875ab3ebccf935723ba83b6da63})} och strutsägg.

}

\small{
\textbf{Banan}
\label{aec7bd708ed2ad3435b9a9883ac7f45c}
 är en sorts frukt \textsc{(s.~\pageref{7b0faed51fc6c55d2431ed677d0989ad})}. Dess gula skal har kommit att bli en internationell symbol för slapstick-humor och har som sådan hyllats i kända pop-konstverk, skivomslag och tatueringar. De godaste bananerna kommer från sopcontainrar \textsc{(se sopletare s.~\pageref{a9729f0b2c1c1ec17cec4dc9fdb10007})} enligt Orgamastron Andersson \textsc{(s.~\pageref{84def8cf16b128f0d5db5f0c9c0ad3ca})}.

 Bananen hittade ursprungligen in på den internationella handelsmarknaden som en restprodukt av den geniala uppfinningen banankartongen.

}

\small{
\textbf{Bananas}
\label{ec121ff80513ae58ed478d5c5787075b}
 När man går bananas betyder det att man tappar sin skit och ballar/freakar ur. Man går bananas \textsc{(se banan s.~\pageref{aec7bd708ed2ad3435b9a9883ac7f45c})} när man blir extremt glad så man inte kan hejda sig utan i ett tillstånd av total eufori tappar koncepten och struntar i sociala normer. Ett lysande exempel på en man som gått bananas var Tom Cruise när han gästade Oprah Winfrey och hoppade upp och ner i hennes soffa för att han var så exalterad.
 Inte bara människor kan gå bananas. Även händelser kan vara helt bananas. Exempel på en sån händelse kan vara en fest där alla har jättekul och dansar väldigt mycket. Om det kan man säga: \textit{\quotetext{Det var heeelt bananas i fredags!}}
 Då man struntar i sociala normer när man går bananas kan det ibland kopplas till sinnesförvirring och vansinne (samt hasch \textsc{(se stenad s.~\pageref{dec4a3a91f0f2bf8dcf033a8cfeaa554})}. Vissa individer kan befinna sig i ett konstant bananastillstånd, varpå det upphör vara något roligt eller positivt och blir ensidigt obehagligt. När det inträffar blir handlingen att kalla någon bananas något nedsättande och ett sätt att ifrågasätta en persons mentala status. Om t.ex. forskare X i sin avhandling påstår att månen är gjord av ost och att fåglar är förklädda råttor kan forskare Y lätt avfärda dennes tes genom att till kollega Z säga något i stil med: \textit{\quotetext{Ähhh den där liraren är helt bananas. Jag har hört att han driver en kombinationsaffär \textsc{(se kombinationsaffärer s.~\pageref{0a2777bf1366a8a9a5b8eab9ca1496a1})}. Jo, det är sant! Hälften skivnasare, hälften fiskmånglare! Inget han \textsc{(se randolfo s.~\pageref{b8f0a32f840f1db27a2c12e17b640fb2})} säger kan vara sant.}}

}

\small{
\textbf{Bandsallad}
\label{ed5a9449af2203ba49daa6004b9b8af5}


}

\small{
\textbf{Bar överkropp}
\label{019d47c3788050e7f22df1d7ce5ea8fe}
 , alt. barre eller bar över, är att vara klädd omvänd Kalle anka \textsc{(s.~\pageref{64db68f686a0ca4d9d641061cb3fdf13})}, dvs Mimmi Pigg \textsc{(s.~\pageref{47a20f7432f125f29ac8d0101be60ad7})}. Det betyder att man har byxor men ingen tröja och är något som är förbehållet män, då det heter \quotetext{topless} för kvinnor. Det här är populärt främst bland män i femtioårsåldern som inte riktigt fattat att deras kropp inte åldrats särskilt väl - här är premissen ju fulare desto troligare. Det är också ganska vanligt bland yngre män som är berusade \textsc{(se bärsfylla s.~\pageref{9380b60f9ee744b9acf978fe6f1a9545})} och bara känner att det vore lite \quotetext{skönt} att ta av sig tröjan. Det är nästan aldrig ok att ha bar överkropp bland folk. Är du osäker kan du gärna konsultera listan nedan.

 HEAD2: Platser och tillfällen där det är socialt accepterat att ha bar överkropp
 \begin{itemize}
 \item Inomhus i sitt eget hem
 \item På sin egen tomt
 \item På stranden eller kring poolen
 \item När man rider på en häst, om man är Rysslands president, Putin.
 \item När ingen ser
 \item När man lirar gura i High on Fire.
 \item När man grovarbetar utomhus och är fackansluten.
 \end{itemize}

}

\small{
\textbf{Barbados}
\label{8c7c9b1149e9de712b9e5abab45d700c}
 Den mixade doft av klor och urin som uppstår när man sitter flera timmar och super i en bubbelpool.

}

\small{
\textbf{Barberare}
\label{f5683c848bc74039df5513e734636898}
 är som en frisör fast för skägget och mustaschen och tar löjligt mycket betalt för att i minst en ½ timme vifta med en kniv under hakan på en.

}

\small{
\textbf{Barn}
\label{5dfcc0aab2f3db925b2d51ba73e48946}
 (För information om hur barn blir till, se erotik) \textsc{(se erotik s.~\pageref{972f097461d1eab1c1ff104757bad922})}
 Vad barn egentligen är satte länge griller i huvudet \textsc{(se huvud s.~\pageref{e906cd95a540df9b16d0460fb4cf0adc})} på glädjevetenskapen \textsc{(se glädjevetenskaper  s.~\pageref{7e4eadb905a6345ef2a6ce2b5b179847})}, men nu är det allmänt känt att barn är som vuxna fast mindre och yngre. Barn är alltså små, små personer. De skiljer sig från dvärgar i det att de ofta växer till sig. Många barn har svårt att ta ansvar och brukar därför normalt inte förses med samma rättigheter och skyldigheter som andra. Än så länge får barn till exempel inte förvärvsarbeta, något som frihetsälskande moderater \textsc{(se moderat s.~\pageref{c4564b188cb670841733a3ff923c2fb0})} dock har för avsikt att ändra på å det snaraste. I linje med Fas 3 åläggs de dock inte sällan enklare sysslor, så som att måla toalettrullar och klistra fast fjädrar på påskris \textsc{(s.~\pageref{a9d744074ec3fda67c8e7b52801e5178})}.
 HEAD2: Kännetecken
 Barn är ofta småvuxna och har normalt nedsatt förmåga vad gäller bordsskick och de enklaste motoriska rörelser. Många barn uppvisar också stora svårigheter med att uttrycka sig i tal såväl som skrift. Barn saknar också ofta skäggväxt. På plussidan kan sägas att de har en betydligt lägre tyngdpunkt än andra människor, vilket är bra när det blåser kraftigt.
 HEAD2: Barn i mytologin
 Jesusbarnet känner många till. Jesus \textsc{(s.~\pageref{110d46fcd978c24f306cd7fa23464d73})} föddes uti ett stall och påstås ha legat i ett tråg \textsc{(s.~\pageref{1e0e0470206e0f2baad8e628ba8f770c})} avsett för utfodring av åsnor. Också i Buddhismens kosmologi återfinns barnet, närmare bestämt i berättelsen om Buddhas födelse. Före sin upplysning ska Buddha, som egentligen inte alls är ett namn utan en beteckning på en upplyst människa, ha kallats Siddharta Guatama. Siddhartas mor ska enligt sägnen ha stått upp då hon födde sin son. Sonen föll således till marken, varpå han ska ha rest sig upp, tagit ett steg mot norr, ett mot söder, ett mot väst och ett mot öst och sedan sagt \quotetext{detta är sista gången jag föds,} vilket kan förstås som en kommentar föranledd av hans irritation över att ha fallit från moderns sköte ner på marken så att han blev vimmelkantig.

 HEAD2: Barn i kulturen
 Popbandet Hanson består uteslutande av barn.

}

\small{
\textbf{Barnaga}
\label{e792f1686861319e3e87f1d348a105d5}
 Slår du dina barn med en kulhammare under fötterna så får de inga blåmärken att visa upp för polisen \textsc{(se polis s.~\pageref{fa296149fa58bfd4408e407cc3fd3be5})}.

}

\small{
\textbf{Barnagans förträffliga pedagogik}
\label{9e0723018fdd5ac13da751c48083a4e3}
 \textit{Barnagans förträffliga pedagogik} är en bok av Prof. Etienne \textsc{(se Användare: Prof. Etienne s.~\pageref{a9878d2280e5a39becac8f73d113df91})} som utgavs av bokförlaget Rabén \& Sjögren 1998. Redan då boken enbart existerade i manusformat blev den omtalad då det kommit till allmänhetens (kulturelitens) kännedom att boken alls inte skulle ha med barnaga att göra utan istället var en lätt omarbetad översättning av Kenneth Grahames klassiker \textit{The Wind in the Willows} (Sv. Det susar i säven). Detta visade sig dock inte stämma. Boken är precis som den utger sig för att vara en handbok i barnaga och har ett långt appendix i vilket författaren argumenterar för ett utbrett bruk av denna traditionstyngda uppfostringsmetod.

}

\small{
\textbf{Barnamord}
\label{abb9d8f6fb223dbc9b3db79dea136d08}
 Belgiens \textsc{(se Belgien s.~\pageref{f79ffe9e826a19f9f6a446c90e21c4e3})} huvudexport.

}

\small{
\textbf{Barndom}
\label{f0e8bb93975a5bb66cfa2e0cf784ed92}
 en är en tid som den västerländske människan bör förtränga med alla till buds stående medel.


 Det finns en mängd olika strategier för detta, såsom:


 Att sluta umgås med sina föräldrar

 Att sluta träffa personer man lärde känna före levnadsår 14

 Att flytta långt från uppväxtsorten

 Sprit

}

\small{
\textbf{Barnuppfostran}
\label{c8913b063e7230468c016f8ec38bc13b}
 är en aktivitet där den vuxna gemenskapen undviker att ge ett barn uppmärksamhet förutom när den anses göra något fel. Det kan till exempel handla om att skälla högt med en sträng röst för att barnet ägnat sig åt osedligheter som att: stå upp när den äter chips, äta en bullbit som hamnat på golvet, upprepa namnet Sean Banan ett femtiotal gånger, försöka öppna ryggsäcken själv men råka gå på fel fack och så vidare och så vidare. Tanken med denna aktivitet är att skydda barnet från det alltid gäckande utanförskapet, en löst definierad samhällelig stämpel som gör det otroligt ansträngande och i många fall omöjligt för det senare vuxna barnet att bli en högpresterande producent med någon som helst möjlighet att garantera sig själv, sina nära och eventuella barn en dräglig materiell standard.

 HEAD3: Se även:

 Barnaga \textsc{(s.~\pageref{e792f1686861319e3e87f1d348a105d5})}

 Prof. Etiennes mästerverk om barnuppfostran Din avkomma och du \textsc{(s.~\pageref{a94d2a0e2987fb13963d974bf02db4a8})}

}

\small{
\textbf{Bartolomaios}
\label{f3ded3b72322e9e517b9292195323030}
 är liksom Phillipos en av de mest anonyma av Jesu \textsc{(se jesus s.~\pageref{110d46fcd978c24f306cd7fa23464d73})} apostlar och förtjänar således en artikel på Nissepedia \textsc{(s.~\pageref{62400dadecd90cb5cd39062abe5a3e4a})}, denna outtömliga källa av svåråtkomlig kunskap. Efter att Jesus blev haffad av romarna ska Bartolomaios ha dragit till Indien för att missionera och, får man anta, vila ut sig lite efter ett par händelserika år. Att han valde att åka på semester till just Indien säger något om vilken anspråkslös och tillbakalutad person Bartolomaios var.  Tyvärr blev han skinnflådd på vägen hem från Indien och detta blir inte mindre makabert av att han är skyddshelgon för bokbindare, garvare, handskmakare, läderarbetare och skomakare.

}

\small{
\textbf{Basist i refused}
\label{b0e4cc440138a041acde8c43207797e4}
 \textsc{(s.~\pageref{723634aa8cde73188d4661bb3fe81ce4})} är det tredje vanligaste jobbet i Umeå kommun, efter städare på NUS och lektor vid universitetet.

}

\small{
\textbf{Basklarinett}
\label{34c56a45635881b3d3ad006192dd39ce}
 Dubbelt så stor och fyra gånger så dyr som en vanlig klarinett.

}

\small{
\textbf{Basshunter}
\label{010d34dfb3da37fb53769b52a4153fda}
 kallas en fiskare som är speciellt inriktad på abborre.

}

\small{
\textbf{Batmanvägen}
\label{a2658b2780a7b92223895bc82b4ebf11}
 är en liten skogsväg i Kärrgruvan, precis bredvid där Hawaii-kråkans föräldrar bor. Vägen går in i skogen och kommer ut igen något hundratal meter senare, varför den för många kan förefalla ganska onödig. Ortnamnsforskare har dock funnit en legend som säger att Batman brukar bränna runt där med batmobilen, och Batman är ju så långt ifrån onödig man kan komma. Just nu ligger ett omkullblåst träd rakt \textsc{(s.~\pageref{92be9c2f6a2fa0abd7fbcbebc76531ea})} över Batmanvägen, ungefär på mitten, vilket kan tänkas förklara varför det var så länge sen batmobilen syntes till där. För så stor som den antagligen är kan den inte va lätt att backa.

}

\small{
\textbf{Batyskaf}
\label{fdf7f9c549a401ddd5ef7ddb5b80a738}
 (från grekiskans βαθύς bathys, \quotetext{djup}, och σκάφη skaphē, \quotetext{båt}) är en undervattensfarkost som uppfanns 1939 av Auguste Piccard och är en föregångare till moderna ubåtar. Den saknar maskineri för att kunna färdas åt sidorna och klarar bara av att åka ner och upp en gång innan den måste tillbaka till hamn för service, så den är inte så särdeles praktisk. Det som är bra med batyskafen är att den reglerar sin dyknivå med ballast i form av järnkulor och fotogen istället för komprimerad luft som i moderna ubåtar. Det gör att man kan dyka mycket djupare än med en ubåt, och 1960 gjorde en batyskaf det hittills enda bemannade besöket på jordens djupaste punkt, Marianergraven, 10911 meter under havsytan. Det fanns inte så mycket att se där så ingen har åkt dit igen eller brytt sig om att tanka på en batyskaf med nytt fotogen och järnkulor för att dyka ner och upp en gång sen dess.

}

\small{
\textbf{Beakta}
\label{5fb8066c875cfced859cf8968e991628}
 Ett knasigt ord jag \textsc{(se User: Ålidhemacademy s.~\pageref{7340dc3e8c6d6572e8bc03f6878f0050})} ofta förväxlar med betrakta.

 Category:I språkets periferi \textsc{(s.~\pageref{bf30f792212df1e2b73341279ff614c6})}

}

\small{
\textbf{Beat}
\label{ed6966981beb595bd0a50e4371805dec}
 Lika delar sodomi, frijazz och opiater.

 [http://www.youtube.com/watch?v=OQ4VBNoJ9Fc]

}

\small{
\textbf{Beivra}
\label{01ba763d5296a13e14342078351bb3ea}
 Ett ord som verkar ha en negativ innebörd, men som lätt kan tolkas som positiv. \quotetext{Det gjorde mig beivrad} låter ju onekligen som något bra, och \quotetext{Överträdelse beivras} låter som att man ivrigt vill att någon ska göra överträdelser. Förvirrande, onekligen.

}

\small{
\textbf{Belgien}
\label{f79ffe9e826a19f9f6a446c90e21c4e3}
 är ett land i Europa som är känt för att vara världens sämsta land, något som gång på gång styrkts i empiriska studier.

 HEAD2: Belgiens historia
 Belgien är Europas direkta motsvarighet till Afghanistan. Landet var i grunden inte relevant utan skapades enbart som en buffertzon mellan dåvarande stormakterna Frankrike och Nederländerna. Det är många som idag ångrar beslutet.

 HEAD2: Belgiens politiska läge
 Belgien beskrivs dagligen av framstående statsvetare som ett politiskt moras. För er som inte vet vad det ordet betyder så är det synonymt med sump- eller träskmark (artikelns författare var tvungen att öppna en osedvanligt dammig ordbok för att begripa detta ord). Grejen med politiken i Belgien är att det egentligen består av två länder, nämligen Flandern där de flesta pratar förflackad holländska och Vallonien där alla pratar franska. Det finns också en liten minoritet av tysktalande i östra Belgien, men de har inget att säga till om. Inte nog med att dessa länder skiljer sig i språk, de skiljer sig också i ekonomiska tillgångar, där Vallonien är fattigt och Flandern är rikt. När då dessa två länder ska försöka enas och tillsammans skapa en federal regering för att förvalta landet så går det rakt åt helvete varje gång. Belgarna (alltså flamländarna och vallonerna) bråkar något så jävulskt om allt mellan himmel och jord och kommer aldrig fram till något. Det kan vara så att det mesta beror på att de helt enkelt pratar olika språk, men man får hoppas att de i alla fall har tolk. I slutet av dessa skrikmatcher går alla hem och är skitsura inför nästa möte och så där håller det på. Det är inte helt omöjligt att vi i framtiden kommer att förpassa landet Belgien till historiens annaler, precis som Leopolds Kongo \textsc{(s.~\pageref{9a135c36b0d8d599c2777b54c10ceb6d})}, och istället prata om Flandern och Vallonien.

 EUs \textsc{(se EU s.~\pageref{4829322d03d1606fb09ae9af59a271d3})} \quotetext{huvudstad} är dock densamma som Belgiens, nämligen Bryssel. Här möts de flesta av EUs institutioner för att fatta beslut, utom en av dem som träffas i Strasbourg.

 HEAD2: Belgisk kultur
 Belgarna uppfann pommes frites, trots att fransmännen idogt hävdar att det var dom. Men Frankrike har bidragit med så mycket annat till världshistorien att de gott borde kunna unna belgarna detta. Tack vare belgarnas koloniala erövringar så blev de också hejare på att tillverka choklad \textsc{(s.~\pageref{6d4129e58eba1497a3f2fc9128cc3b23})}. På det mer finkulturella planet så har Belgien levererat väldigt många kvalitativa tecknade serier, där Hergés Tintin kanske är den mest kända. Bland de kulturella avarterna bör det nämns att barnamordsfrekvensen \textsc{(se barnamord s.~\pageref{abb9d8f6fb223dbc9b3db79dea136d08})} är helt oproportionerligt hög.

 HEAD2: Belgien som idrottsnation
 De är kassa på de flesta sporter, men landsvägscykling är de ena riktiga hejare på. Eddy Merckx \textsc{(s.~\pageref{c91a3f6993ed23dd05cee5ff8e52c938})}, historiens genom tiderna bästa cyklist, är ju som bekant belgare.

 HEAD2: Infrastruktur
 I Belgien finns det inte mindre än 7 kärnkraftverk. Varav fyra särskilt avsedda för att tillgodose energibehovet vid konsumtion av internetporr \textsc{(se pörr s.~\pageref{5faa435e2f0af7617816f0cade262581})}.

}

\small{
\textbf{Belgisk jättekanin}
\label{4683df87b389ab00a45e8287521a73f6}
 är världens största kaninras och kallas därför också, kort och gott, Belgisk \textsc{(se Belgien s.~\pageref{f79ffe9e826a19f9f6a446c90e21c4e3})} jätte. Arten avlades fram under mellankrigstiden av den Belgiska armén genom korsning av Bayersk vädur och den franska rasen Fauve de Bourgogne. Detta blev känt av den brittiska underrättelsetjänsten som under trettiotalet lyckades smuggla ut en individ \textsc{(s.~\pageref{41beed76a0af9b4f550f7ebdecd3e700})} av rasen från en anläggning i Antwerpen. Dessvärre dog detta exemplar av stress \textsc{(s.~\pageref{e10a36f1a5231e597daf8f42dc1ab55a})} under resan över engelska kanalen. Kroppen obducerades för att eventuella ledtrådar till kaninens militära användning skulle kunna sökas, trots att dess indresserade beteendemönster gått förlorat, men till föga resultat. Efter andra världskrigets slut infångade Belgiska och Holländska bönder \textsc{(s.~\pageref{30a6fc00c9102680b8196b1b79935ec4})} ett antal exemplar som hade släppts ut inför anstående Nazitysk belägring och minst två exemplar hamnade så småningom i MI6s laboratorium, men inte heller dessa gav något svar på rasens gåta. Däremot blev ättlingar till de infångade individerna populära husdjur i Belgien \textsc{(s.~\pageref{f79ffe9e826a19f9f6a446c90e21c4e3})} såväl som i mer civiliserade länder runtom i Europa.

}

\small{
\textbf{Belgisk öl}
\label{3420b042bc838a4c8a53db2875070a92}
 Bortsett från blask såsom Stella Artois så är Belgisk öl god och ofta rysligt stark. Det är så på grund av Belgares vana att utöva sin kultur \textsc{(se barnamord s.~\pageref{abb9d8f6fb223dbc9b3db79dea136d08})}, detta skapar ångest och som alla vet botar man ångest med en bärsfylla \textsc{(s.~\pageref{9380b60f9ee744b9acf978fe6f1a9545})}. Man kan även grubbla över varför så många Belgiska ölsorter tillverkas i pedofilreservat; kallade \quotetext{kloster}.

}

\small{
\textbf{Belka}
\label{9f148033690e82549848dea862a1a9ee}
 var den andra hunden i omloppsbana i rymden.

}

\small{
\textbf{Bellman}
\label{b6b9660f4f754e67face0b2633b39aa6}
 Carl Mikael Bellman (1740-1795) var en svensk skald och föreståndare för det kungliga lotteriet. Han var bosatt i Stockholm \textsc{(s.~\pageref{edcd259e0a03c7ab70feb186bae19f13})} och ska ha umgåtts en del med en viss tysk och en ryss, och utöver det även med en viss Frkn. Ulla Vinblad. Enligt utsago ska Bellman vid ett tillfälle ha diskuterat den överlägsna hastigheten hos riket Sveriges \textsc{(se Sverige s.~\pageref{b1999637949ed135b2ca03f3a38460cc})} tåg med sina tyska och ryska vänner och då ha överdrivit något.

}

\small{
\textbf{Bengt}
\label{2be19239ba225a12dbd56ea424a66672}
 heter min farfar. Han har arbetat bland annat som:
 \begin{itemize}
 \item Grävmaskinist
 \item Vaktmästare
 \item Asfaltsläggare
 \item Bryggbyggare
 \end{itemize}
 HEAD2: Idellt arbete
 Han har också:
 \begin{itemize}
 \item Varit valfunktionär
 \item Ordförande i vägföreningen
 \item Ledamot i skytteföreningen
 \item Lett ungdomsluftgevärskyttet
 \item Rått om hembygdsgården
 \item Klippt gamla tanters gräsmattor
 \item Hjälp folk med trasiga bilar
 \item spelat bandy och fotboll
 \end{itemize}

}

\small{
\textbf{Benny Bus}
\label{a8289efd495ef49dbe0225de89f7f019}
 är en träskpunkare \textsc{(s.~\pageref{484838b3db1adb135ea74d6fc61e44c0})} från Avesta i dalarnas län. Han tycker om att dricka mäsk och vara arbetslös. \textlesss\textgreaterDet är oklart vad Benny gör idag\textless/s\textgreater

 Senaste nytt: Benny har flyttat till Skåne och äger en stormhatt.

}

\small{
\textbf{Bensträckare}
\label{b7f58d514c506dc96a77284fcc03039a}
 Slang för att ta en paus. Bensträckaren går ofta ut på att man som tjänsteman går till fikarummet, häller upp en cacao creme \textsc{(s.~\pageref{ab4762d6b262c568ca4511942bd68bf1})}, sen går tillbaka till kontoret och spelar ms röj i fyrtiofem minuter eller så.

}

\small{
\textbf{Berghem}
\label{a6b1df39fa9b1b94dc92200594a8ccd6}
 är en stadsdel i Umeå som består av två delar, norra och södra. Södra Berghem består av korridorer och studentlägenheter med intressanta gatnamn som Skidspåret och Rågången. Norra Berghem är ett villaområde som räknas som Umeås gräddhylla där det bor en massa rikingar. Berghems centrum, som på intet vis ligger i centrum av stadsdelen, är Axtorpet \textsc{(s.~\pageref{11f14a5854300c512ed501986f38a609})}.

}

\small{
\textbf{Berghem HC}
\label{72c5e1ef562098496277726ca12aa149}
 är ett fotbollslag hemmavarande på Berghem, Umeå. Det består av ett okänt antal mer eller mindre talangfulla, men framför allt spelsugna, män och kvinnor i åldersspannet 20-30 år, men laget är också öppet för förslag vad gäller spelare i andra åldrar.

 HEAD2: Hemmaplan
 Lagets hemmaplan är Berghemsskolans grusplan, i folkmun även kallad Berghemsvallen. Lagets klubbstuga heter Rött \textsc{(s.~\pageref{dacd03b85a85d8c8b67c702e1872c498})} och dess sekretäriat tillika kansli heter Hallonvägen 2.

 HEAD2: Vunna titlar
 I Berghem HC vinner alla, alltid. Vill man vara petig har laget dock gripit följande åtrovärda pokaler:

 \begin{itemize}
 \item Silvermedalj i Frihetliga fotbollscupen 2013
 \end{itemize}

 HEAD2: Lagets filosofi
 Lagets filosofi är att \quotetext{passar du så passar du,} ett finurligt motto och ordlek där ordet passas båda betydelser nyttjas. Så sent som för en timme sedan blev fem av lagets spelare berömda för sin trevliga och tillmötesgående attityd av en kille med vit mössa som inbjöds att vara den sjätte spelaren på plan (dels för att han verkade sugen på att lira fotboll \textsc{(s.~\pageref{961bd74d34872ff94a4df3a16119096e})}, men inte minst för att många i laget dagen till ära fotbollsvägrade) \textsc{(se fotbollsvägra s.~\pageref{0edd1cff3088f6b1fbc12c3be289aaed})}. Laget utövar strikt nolltolerans mot sexism, rasism och småborgerligt beteende. Omotiverat gap och skrik beivras \textsc{(se beivra s.~\pageref{01ba763d5296a13e14342078351bb3ea})}.

 HEAD2: Sticking it to the man
 Laget är förföljt och förtryckt av bylingen \textsc{(s.~\pageref{b54c92e1b1671e982dc24eefae2edce1})} som inte låter laget klippa upp lås som använts för att låsa ihop målen på Berghemsskolans grusplan. Laget gör detta ändå, naturligtvis.

 HEAD2: Support och tifo
 Laget supportas helhjärtat av det legendariska Oi!-bandet Skinned Alive \textsc{(s.~\pageref{88b01b69ece92f30d83711e8a65fd542})}, MC-outlawsen i Rainbow Riders \textsc{(s.~\pageref{54b5b4739e6bc150148c5019e1793413})} och av lagmedlemmar spridda i diaspora.
 HEAD2: Andra betydelser
 Skinned Alive \textsc{(s.~\pageref{88b01b69ece92f30d83711e8a65fd542})} har en låt som heter Berghem HC på sjuan \textsc{(se sjua s.~\pageref{e7bf63fa6d0d29bd89c23f833b979a15})} \textit{Axtorpet \textsc{(s.~\pageref{11f14a5854300c512ed501986f38a609})} Bootboys}.

}

\small{
\textbf{Berghems Resturang}
\label{604ffa4dabe275c615bc9f8960b68d84}
 är, namnet till trots, en pizzeria. Den är belägen vid Axtorpet \textsc{(s.~\pageref{11f14a5854300c512ed501986f38a609})}, Berghem, Umeå. De serverar samma mat som alla andra pizzerior, men de är unika i det att de knådar ut sin deg på en brits från ett bårhus. Anledningen till detta något oortodoxa bakbord är att samtliga av pizzerians bagare jobbade som läkare innan de kom till Sverige, men vägrar lägga sina gamla liv bakom sig. Det här är också anledningen till deras ibland korta och burdusa stuk när man beställer en falafeltallrik \textsc{(se falafel s.~\pageref{b2d6ec45472467c836f253bd170182c7})}.

}

\small{
\textbf{Bergslagen}
\label{2844a3a8251745f1f093b6f86f909183}
 är ett område i Sverige \textsc{(s.~\pageref{b1999637949ed135b2ca03f3a38460cc})} tidigare helt finskspråkigt.

}

\small{
\textbf{Berguv}
\label{e382b890bd1bfb294f22cb145833881e}
 ( latin: \textit{Bubo bubo} ) är enligt många den förnämaste av fåglar och är lika mytomspunnen som den är älskad och fruktad. Berguven är också den största uven \textsc{(se uv s.~\pageref{45210da832f9626829457a65e9e7c4d0})} och kan bli upp till två meter i diameter.
 HEAD2: Föda
 Berguven kan äta allt som andra djur kan äta, och mer därtill. Ofta har den setts flyga in på indiska restauranger och flyga iväg med en Tandoori eller Tikka Massala, men oftast äter den sådant som den kan hitta i naturen: Pinnar, grus och små djur står många gånger på menyn för denna fantastiska uv \textsc{(s.~\pageref{45210da832f9626829457a65e9e7c4d0})}.
 HEAD2: Rede
 Berguvens traditionella rede är som en stor plattform uppe på något högt och otillgängligt berg. Där ruvar den på sina ägg \textsc{(s.~\pageref{128a5feb8e12d0aa622e0298a8332980})} och ser ner över mänskligheten som hukar under dess penetrerade blick.
 HEAD2: Häckning
 Berguven häckar en gång vart tredje år och får då tre ungar som den låter stanna i boet i tre månader. Sen ska de ut, så är det bara!
 HEAD2: Berguven i populärkulturen
 Rockgruppen Sleep ägnade hela andra skivan, \textit{Holy Mountain}, åt berguvar och berguvsrelaterade teman. Ett exempel är textraderna: \quotetext{Ride the Eurasian Eagle-owl toward the crimson eye/ Flap your wings under Mars' red sky} från låten \textit{Bubonaut \textsc{(s.~\pageref{c09400b667b41bb7bd0de4f9aa2d71ce})}}.

 HEAD2: Berguven i sportens värld
 En berguv kallad Bubi förhalade ett EM-kval i fotboll mellan Finland \textsc{(se Finländare s.~\pageref{fc472090d678bd6f029cd80792f4a36d})} och Belgien \textsc{(s.~\pageref{f79ffe9e826a19f9f6a446c90e21c4e3})} med hela sex \textsc{(se sexa s.~\pageref{4b1fabe53857b0a2ace6ae22008fe13e})} minuter. Detta genom att bara landa och sprida skräck, såsom uvar gör.

 HEAD2: Berguven i asfaltverket
 Berguven häckar högst upp i asfaltverket. - Vi skulle upp och göra något i elskåpet, berättar Andreas Johansson om hur äggen upptäcktes. När de tre äggen kläckts kan ungarna komma att flyttas så att asfaltverket kan köra i gång årets produktion.
 Berguven häckar gärna på en klipphylla eller kanske tar den över ett gammalt rovfågelbo.
 Men högt uppe på ett asfaltverk - 15 meter ovan mark - längs vägen till Fågelmyra \textsc{(s.~\pageref{044cc1889af3f8f726910a98218d6e2d})} tycks också gå bra.
 Här har honan lagt tre ägg och i går hade ett av dem kläckts.

 Asfaltverkets personal följer nu intresserat uvarnas rörelser. Asfaltverket ska enligt planerna köra i gång den 2 maj för att under året producera 40 000 ton asfalt.
 Innan det trycks på startknappen ska sannolikt ungarna, när alla kläckts, ha flyttats till en annan plats uppe på asfaltverket. Det är i alla fall tanken.
 Där får de ligga lite mer i lugn och ro - även om det kommer att låta en del på platsen, så ska det inte störa berguvarna. För att en flytt ska kunna ske krävs tillstånd från länsstyrelsen.
 Om äggen skulle flyttas innan de kläckts är risken större att de överges, än om ungarna flyttas.

 - Det här sinkar oss lite. Vi kan inte serva anläggningen som planerat. Men vi måste ta hänsyn till naturen, säger produktionschef Ulf Modén.

 Han berättar att asfaltverket inte kan starta så länge berguven häckar på nuvarande plats. Förhoppningen är att berguvens val av boplats ska kunna få en lösning som inte påverkar uvarna \textsc{(se uv s.~\pageref{45210da832f9626829457a65e9e7c4d0})} negativt.
 En kamera på asfaltverket har nu riktats om. Inne på kontoret kan personal följa utvecklingen i uvens \textsc{(se uv s.~\pageref{45210da832f9626829457a65e9e7c4d0})} bo.
 - Hanen syns också till här ibland, berättar Andreas Johansson.

 En död råtta hittades en bit upp på asfaltverket, troligen ett byte som uven tappat och inte lyckats få upp.
 Det är hanen som väljer boplats, och honan ruvar på äggen i drygt en månads tid.

 Kommunbiolog Sören Nyström säger att berguven häckat i området kring Fågelmyra i 20 år. Berguvparet håller sig till sitt revir.
 - Ofta har den bytt ställe där den lagt boet, konstaterar Sören Nyström.

 HEAD2: Berguven inom friluftsutrustningsbranschen
 Berguven var en tillverkare av ryggsäckar och tält, vars främsta egenskap är att de är ganska fula men håller i ungefär en aeon.

}

\small{
\textbf{Bert Karlssons telefonnummer}
\label{958ccc32173aa9f4086ac4a314f4909e}
 är 0705932999.

}

\small{
\textbf{Bertolt Brecht}
\label{5796d03aaabd9acc8eacbbe2195b60af}
 (10 Februari 1898 – 14 Augusti 1956) var en stor tysk dramatiker och poet. Han förnyade teatern genom att med sin livslånga förbindelse med marxistiskt tänkande utveckla den moderna teatern till ett forum för politisk debatt och analys och vad han kom att kalla dialektisk \textsc{(se dialektik s.~\pageref{5c0ded4e9796ad82ecd11d1a0010bf6b})} estetik. Brecht arbetade under långa perioder vid Berliner Enseble och turnerade också med denna inflytelserika institution för att sprida sina visioner för den moderna teatern. Brecht var en framträdande figur i den tyska vänstern innan det andra världskriget \textsc{(se det stora fosterländska kriget s.~\pageref{8e55572fc7b7490da402e43a822eb3da})} och umgicks i kretsar där sådana centrala gestalter som Adorno, Horkheimer och Walter Benjamin ingick. I sina mest kända verk, så som \textit{Den kaukasiska kritcirkeln} och \textit{Skräck och misär i det tredje riket} uttryckte Brecht sin avsky för fascismen, som han tvingades undfly den tyska mustigheten \textsc{(s.~\pageref{682ccd5fdc3aff0c97e8845c3d6b6ca8})} efter att Hitler \textsc{(se Rudolf Hitler s.~\pageref{c9f9ecff9e5071300a593974776e5085})} tagit makten i landet 1933. Han flydde till Danmark, Sverige \textsc{(s.~\pageref{b1999637949ed135b2ca03f3a38460cc})} och (av oklara skäl) Finland \textsc{(s.~\pageref{631d44eaa1254ff71a1e11ba021d1266})} och senare till USA \textsc{(se United States of America s.~\pageref{ade6b3bd5e720abb20ed8a9a4c6b9ae8})}, men återvände till Tyskland efter kriget. Han dog slutligen i DDR i Augusti 1956.

}

\small{
\textbf{Berätta}
\label{4f84e02a70b3bbb57fa83da31bf7a16f}
 Att \quotetext{berätta} är ett annat ord för att skriva deckare. Deckarförfattare har ett medfött behov av att \quotetext{berätta}. De måste \quotetext{berätta,} annars stängs deras \quotetext{berättelser} in. Då kan de inte längre leva, utan dör. När de har börjat skriva (dvs berätta) tar texten över och författarna \textit{kan} inte längre styra den, säger de i alla eventuella intervjuer. I sina memoarer berättar de om deras fascination av \quotetext{ordet} som utvecklades redan i tidig ålder \textsc{(s.~\pageref{d7a7467d6b0b94f50c209220eab58dd1})}. Medan andra barn var ute och lekte läste författaren/berättaren i faderns bibliotek och \quotetext{orden} var hans eller hennes vänner, inte de andra idiotiska rackarungarna som ägnade sig åt att åka kälke.

}

\small{
\textbf{Beskinnad}
\label{ec57618f74a0080007691bea04795906}
 Beskinningen är ett extra hårt skinn.
 Exempel saker som kan vara beskinnade.
 Druvor
 Korvar
 Näsor

}

\small{
\textbf{Besvikelse}
\label{a3cdc7d1b33db6959c1d3a78b1f47012}
 uppstår när något man hoppats på inte inträffar. Det kan till exempel vara så att man sitter på en uteservering i en sydsvensk stad, med ett gäng softa tjejer och en trevlig killkompis. Under kvällens gång pratar man lite extra med en av de softa tjejerna och kommer fram till att: \textit{\quotetext{det här verkar vara en tjej som inte skulle banga på att röka lite mary jane och dela ett sexpack med mig vid ett senare skede}}. Senare under kvällen visar det sig att hon inte alls bangar på något sådant, med undantaget att det inte är dig hon vill göra det med, utan din polare.
 I många fall, särskilt i samhällen präglade av en luthersk anda, leder ofta besvikelser till skam \textsc{(s.~\pageref{e7d275bbd2f3522805002be76a53ccd8})}, då man känner sig som en idiot som någonsin vågat hoppas på något. I särskilt svårartade fall, eller vid en lång serie av mindre besvikelser, kan även bitterhet \textsc{(s.~\pageref{054dc5418cb9804455d8c5eeec6be8fd})} uppstå.

}

\small{
\textbf{Beteendevetarhuset}
\label{7d706dee7b04df6ec921433510e41f63}
 är tillsammans med Lindellhallen den finaste delen av Umeå Universitet. Där har det inte sparats in på några utgifter minsann! Golvet är av svart och vit polerad marmor och inte ruttnande linoleum från sjuttiotalet som i Humhuset. I fiket säljer en alltid leende personal komplicerade bakverk till unga kvinnor med komplicerade frisyrer och blekta tänder. Där går det att hitta rätt. Där är akustiken genomtänkt. Där finns ståtliga krukväxter. Där finns allt.

}

\small{
\textbf{Bibeln}
\label{7de7d2a7d608c9a2044f50688bc63e27}
 är en späckad faktabok om fårskötsel. Det haglar tips om hur herden bäst tar hand om sina lamm till vardags, men även hur han bör handla i specifika situationer såsom om en buske börjar brinna, man faller ner i en brunn eller om det regnar jättemycket. Boken är skriven av gud \textsc{(s.~\pageref{91e49146121c992aab11a19c77e26cf0})} och har sålts i fler exemplar än till och med \textit{Fass}. I och med försäljningssuccén \textsc{(se braksuccé s.~\pageref{678371d35369d3d29afceb1445630833})} har boken senare översatts till andra format, till exempel sånghäftet \textit{Psalmboken} och landet Australien \textsc{(s.~\pageref{e727d8d1b3162a732c7f706d55de64f3})}.

}

\small{
\textbf{Bil}
\label{b3188f47d2eac7efc3f1258dc673a9fe}
 En bil är ett transportfordon som vanligtvis rullar på fyra \textsc{(s.~\pageref{7bdb5385ce8e0b1cbc7c15b1d71e8e7d})} hjul, drivs av en bensinmotor och har plats för en förare och fyra \textsc{(s.~\pageref{7bdb5385ce8e0b1cbc7c15b1d71e8e7d})} medpassagerare. Bilen uppfanns för ungefär 100 år sedan och dess popularitet växer fortfarande stadigt. Det finns väldigt många olika modeller av bilar och den snyggaste är Volvo 240.

 Bil bör inte förväxlas med buss \textsc{(s.~\pageref{e57167c19ed4b7c62a6527f85687cfab})}, ett annat motorfordon som körs på hjul.

}

\small{
\textbf{Bilbatteri}
\label{3894bc588cff2547bbfa28aaf455313a}
 är ett fiskeredskap som används av yrkesfiskare för att komma åt större mängder av havets läckerheter. Bilbatteriet slängs helt sonika i vattnet och efter någon vecka är det bara att ro ut och håva in allt som flutit upp till ytan. För större vattendrag med stor genomströmning, såsom älvar, kan det vara värt att slänga i en kvicksilvertermometer också. För att få fiska med bilbatteri krävs tillstånd från länsstyrelsen, men det är idag mest en formalitet som lever kvar från sovjettiden när man inte var riktigt säker på vad som fanns i importerade bilbatterier från östblocket.

}

\small{
\textbf{Bildekal}
\label{7bdee158c028cef0e4bcd04631250180}
 er är ett klassiskt verktyg för enkelriktad masskommunikation. Fram till internet \textsc{(se World Wide Web s.~\pageref{3b7d657e8b7bf25a9d524b60d9bb17df})} intåg var bildekalen det överlägset billigast alternativet för att nå ut till tusentals människor. Avsändaren dikterar helt sonika sitt budskap på en plastremsa med klisterförsedd baksida och applicerar remsan på bilens bakre kofångare. Alla bakomliggande trafikanter kommer därefter att läsa, och med säkerhet också begrunda, den korta parlören. Sedan internet blev en del av var människas hushåll är det många som numera väljer att vädra sina åsikter där istället. Den sista stora bildekalsvågen i Sverige skedde 1994 efter folkomröstningen om EU och löd \quotetext{Skyll inte på mig - jag röstade NEJ}. Man kan fortfarande se en och annan \quotetext{Sänk dieselskatten!} och \quotetext{Skjut alla vargar - För djurens skull}, men det är inte alls lika vanligt. Föregångare till bildekalen var den skitiga bakrutan och före det standaret.

}

\small{
\textbf{Bilmaskin}
\label{96bb88b1798075c970052ea4805880fa}
 När folk snackar om att de har bilat en hel eftermiddag så har de inte varit ute och cruisat med bilen. Bila gör man när man använder en bilmaskin. En bilmaskin är ett verktyg som faktiskt låter som att det borde tillhöra verktygslådan i garaget, men det gör det inte. Använd inte en bilmaskin när du mekar med bilen, för det kommer du att ångra. Har man bilat ett tag så känns det som att händerna har blivit ungefär dubbelt så stora, då kan man testa på hur osmidigt det känns att göra pilliga saker om man är basketspelare. Vill man uppnå effekten av att ens händer är hälften så stora kan man använda en hyvel \textsc{(s.~\pageref{1668e2298ba60f14922e2cca6aa96538})}.

}

\small{
\textbf{Bilprovningen}
\label{9516542b6862983521f399807cf913da}
 är  en institution som syftar till att skydda bilägande medborgare från sitt fordon.
 Alla måste besiktiga vare sig det behövs eller ej, vanliga pantade knegare har inte vett att byta vindrutetorkare eller medföra varningstriangel, därför måste en betrodd kontrollant se åt så detta blir gjort.
 Det finaste man kan få på bilprovningen är ett blankt protokoll \textsc{(s.~\pageref{4219b486a772f32467674f4516155f9e})} och en mugg ljummet kaffe \textsc{(s.~\pageref{a51a0cac0ce374a853d2359417debc28})}. Det sämsta är körförbud.

 HEAD2:  Bilprovningen i internationella sammanhang

 Bilprovningen är något som typ bara finns i Sverige \textsc{(s.~\pageref{b1999637949ed135b2ca03f3a38460cc})}, därför kan folk från andra länder bli ytterst förvirrade när dom plötsligt får körförbud eller böter, i regel ringer de då till trafikverkets telefonsupport \textsc{(se telefon s.~\pageref{15a957eec81cff8df3172257b813e2d3})} och skäller på telefonisterna. När detta inte hjälper drar man till med ett \quotetext{\textit{får jag prata med din chef}}. Mycket vanligt bland utbytesstudenter \textsc{(s.~\pageref{397699f3732b0c22f3c532a111697539})}.

}

\small{
\textbf{Bilsupa}
\label{31d543e92baaa02b9ef3287a75f2e172}
 Få saker är väl så fina som att sitta ned i en bekväm soffa med en bira \textsc{(se djungelkasse s.~\pageref{74611749734f87b8dca27bcd78a9cc0c})} i näven och AC/DC \textsc{(se bonfire s.~\pageref{b0759e17c7cc70d7522a6b63a05c914e})} på stereon. Det skulle väl i så fall vara att susa fram på en slingrande landsväg med en kvarting \textsc{(se åka vikingaskepp s.~\pageref{59819f07de01e025b0ca8c53b6481ac6})} innanför västen och Lemmy \textsc{(s.~\pageref{6cc2f8758343439728f308f08a4a8fad})} dånande ur stereon. Att bilsupa kombinerar dessa två aktiviteter på det mest angenäma sätt.

 Bilsuparen tar plats i baksätet på en svensk- \textsc{(se Volvo 240-serien s.~\pageref{9a8db892f6b42596f2abce57f62a6399})}, tysk- eller amerikansksnickrad bil där benutrymmet är väl tilltaget. Någon tvingas sitta shot gun där gemenskapen inte är lika stor, men detta kompenceras i och med närheten till stereon. För att bilsupningen ska fungera krävs också att en chauför engageras för att ratta fordonet och lagen förbjuder strängeligen att denna person låter sig berusas. Styrfylla är nämligen farligt och i princip bara okej om man trampar trehjuling och trallar på \textit{Min kära gamla soppeskål \textsc{(s.~\pageref{42ce7b9516a04ae0a46ccb0de720f3c3})}} iförd cykelhjälm \textsc{(s.~\pageref{eccbf5255f85f74dc67bffbcae78793b})}.

 Mycket nöje!

}

\small{
\textbf{Biltelefon}
\label{27e67ebb2cd0325b83606cd0f369f44a}
 en är ett redskap som mycket har underlättat kommunikation mellan bilburna människor som hela tiden måste vara tillgängliga, på grund av sina yrken eller sin snabba livsstil. Biltelefonen monteras där kassettspelaren normalt sitter i bilen och drivs med hjälp av en kontakt som pluggas in i cigaretttänderhålet som finns däromkring. Tyvärr sprakar det så mycket i biltelefonen att det är hart när omöjligt att begripa varandra, men lyckas man identifiera den andres röst kan man anta att han eller hon vill en något, och eftersom man ju redan sitter i bilen kan man då med fördel styra den till den plats där man tror att ens interlokatör med störst säkerhet befinner sig.

}

\small{
\textbf{Bintje}
\label{f21f4f64cb0df1775b5c2a7dc0d83c6c}
 (latinskt namn \textit{poop-orbis})är en potatissort som togs fram 1905 av den holländske potatiskonnässören Kornelis Lieuwes de Vries. Sorten är döpt efter de Vries kompis Bintje Jansma, vilket gör den till en av två sorters potatis som är döpta efter mansnamn (det andra är King Edward) \textsc{(se King Edward s.~\pageref{a081b9ae5423fc12a8439e33b2af8bed})}. Bintje smakar nåt rent förjävligt om man inte dränker i ketchup. Dess mycket hårda konsistens gör dock sorten mycket lämplig till potatistryck \textsc{(s.~\pageref{a658ef6f51769cd542118c30fddd3bf1})}, vilket också är dess främsta användningsområde. Bintje är skattefritt eftersom det är så äckligt.

}

\small{
\textbf{Biologer}
\label{f45ce3d5a42f66b950a62e19ce1b44d1}
 jobbar med vetenskapen biologi, det vill säga läran om allt levande. De kan det mesta om djur och växter och vet precis var olika utrotningshotade djur \textsc{(s.~\pageref{24a427a5537c2c8918cfa213ae099a74})} och växter lever. Kunskap är som många vet synonymt med makt och i biologernas fall är detta tydligare än mycket. Hur då undrar ni? Tänk er att en ny motorväg ska byggas och när alla beslut fattats, budget gjorts i ordning och alla maskiner just börjat hålla på. En biolog träder då fram och skriker: \quotetext{STOOOPP!}. Det visar sig att på just den platsen häckar den sista Brunfläckiga Skedstorken \textsc{(se Skedstork s.~\pageref{76648d90910c2fd6fcd81b3f3f28d9ea})} i Europa. Bara att dra om vägen, ingen vill väl rubba den biologiska mångfalden? Tänkte väl inte det.

}

\small{
\textbf{Birgitta Höijer}
\label{9c402d608d4c87384133bd5f8b522574}
 är professor och tillsammans med Olle Findahl \textsc{(s.~\pageref{433e1dc6d01073b9b2b4a5a6294d0597})} författare till boken \textit{Text och innehållsanalys – En översikt av några analystraditioner}.
 Se även: Kvantitativ innehållsanalys \textsc{(s.~\pageref{8cbd40215a0453bdd47cd6ef47c53ec2})}.

}

\small{
\textbf{Bitsk}
\label{30e98ae3eef65db46b53a7ebe393124e}
 Ungersk nödvaluta på 1950-talet. Trycktes på skrubbat bibelpapper och kallades i folkmun för Mumin, vilket är ungerska för Mammon. Är fortfarande inte värt något, inte ens för samlare.

}

\small{
\textbf{Bitterhet}
\label{054dc5418cb9804455d8c5eeec6be8fd}
 är en rätt dryg känsla. Om man, för att ta ett slumpmässigt exempel, har suttit på en uteservering i en sydsvensk stad och kuttrat lite med en soft tjej, sen blivit skamsen \textsc{(se skam s.~\pageref{e7d275bbd2f3522805002be76a53ccd8})} och besviken \textsc{(se besvikelse s.~\pageref{a3cdc7d1b33db6959c1d3a78b1f47012})} när hon drar med din polare, är det inte omöjligt att bitterhet uppstår. Vissa blir bittra på kvinnosläktet vid såna händelser, men rent teoretiskt är det inte omöjligt att man i det här fallet blir bitter på sin polare istället och ba: \textit{\quotetext{Helvete! Han är både snygg, trevlig och spelar i ett band som är mer populärt än mitt och har en massa balla skivor jag vill ha... satan! Jag ska ta de där jävla skivorna han bad mig ta med till distrot i tyskland och slänga i en flod... as!!! Han beter sig fan arrogant... åhhh den jäveln!!!!}}. Den sortens bitterhet är väldigt barnslig, då ens polare knappast kan hjälpa att den är snygg, trevlig och spelar i ett band som är mer populärt än ditt. I många fall går bitterhet över efter ett tag, men en del personer, särskilt skitgubbar, har lätt för att göra bitterhet till sitt mest framträdande karaktärsdrag. Sådana bittra skitgubbar är rätt roliga på håll, men egentligen djupt tragiska, då bitterhet är en känsla som förhindrar en från att leva i nuet.

}

\small{
\textbf{Bjuddosa}
\label{bfc161fdb69b8da0af744f6649ecf2bc}
 En bjuddosa är en dosa med redan använda snusar i. Den används mest i nödlägen eller av människor som inte kan kontrollera snåltarmen \textsc{(s.~\pageref{ae50b82f824beabd2246d5aa9c7ac61e})}.
 \textbar

}

\small{
\textbf{Bjudsprit}
\label{3a68804bcc7740bc3fd426c893757a06}
 är en flaska som står längst in i barskåpet och åker fram när det kommer gäster. Gemensamt för alla bjudspritsorter är att de smakar vidrigt, t.ex. risbrännvin eller polsk whiskey. Smaken påminner gärna om lysfotogen.

}

\small{
\textbf{Bjudtermos}
\label{6ea19ffffbb3c85b7fed9d6574df646f}
 En bjudtermos är en kaffetermos med kopp i båda ändarna. Perfekt om man gillar sällskap.

}

\small{
\textbf{Bjuv}
\label{32e5337e416a21e84d2e980c4d4e5297}
 En bjuv är en uv \textsc{(s.~\pageref{45210da832f9626829457a65e9e7c4d0})} som man ger bort till nån.

}

\small{
\textbf{Björn}
\label{d009f007520b1ebbccd1b55ff5d072e4}
 kan vara
 \begin{itemize}
 \item En tubaspelare \textsc{(se dragbasun s.~\pageref{0315aaaabb57a67312aa3316fd2006e1})} som bor på Sjöfartsgatan.
 \item Ett brunt djur som har förståndet att sova hela vintern. \textsc{(se Björn (djur) s.~\pageref{f144d7aeb6983afe1e65027901c6d05f})}
 \item En förståndshandikappad författare på Österlen.
 \item En förbjuden dansare \textsc{(se dans s.~\pageref{ef86916bc6f9f2f6866df100a192161f})}
 \end{itemize}

}

\small{
\textbf{Björn (djur)}
\label{f144d7aeb6983afe1e65027901c6d05f}
 Björnen är ett däggdjur \textsc{(se ägg s.~\pageref{128a5feb8e12d0aa622e0298a8332980})} som finns över nästan hela världen i två olika varianter: vanlig björn och cirkusbjörn. Ibland är dom bruna och finns i skogen, ibland vita på tundran, ibland uppstoppade bredvid eldstaden. Nästan all tecknad film gestaltar björnar som snälla och gulliga men det är helt fel. Går du fram till en björn kommer den garranterat att nita dig. Föreställ dig att gå runt på alla fyra \textsc{(s.~\pageref{7bdb5385ce8e0b1cbc7c15b1d71e8e7d})} i skogen en hel sommar och bara leva på kottar och bär så skulle nog inte du heller vara så glad. Eller att hela dagarna cykla runt på enhjuling jonglerandes med tre träskbabianer, ganska överskattat i längden.

 När vintern kommer drar cirkusbjörnarna på sig grillorna och [http://youtu.be/8PEl9PJCC48 lirar hockey inför extatiska ryssar]. Dom andra björnarna tycker vintern suger så dom sveper en dunk glykol och gräver ner sig i snön i ett halvår istället. Nu kan ni allt ni behöver veta om björnar.

}

\small{
\textbf{Björn Ranelid}
\label{b374a5d86cf98cd5ba2a0ff96d5a9e97}
 är alltid så brun. Och blond. Enligt honom själv varken solar han solarium eller blonderar sig. Enligt Linda Norrman Skugge \textsc{(s.~\pageref{141f8115f3f69da45ddb845f4575ac21})} använder han läppglans och rakar armarna. Björn tycker det är fint att skriva snusk.
 Björn kan inte formulera sig begripligt, det beror inte på att han är skåning som man lätt skulle kunna tro.

}

\small{
\textbf{Black flag}
\label{2d16f78ff2b4f2f55709bc84725d97c7}
 är ett av punkgenrens viktigaste band. Bandet startades i Kalifornien, av småföretagaren tillika gräsrökaren Greg Ginn, år 1976. Under sin karriär hann bandet byta medlemmar ca. 40 gånger och utveckla sin musik, från garageig hardcorepunk, via black sabbath-dyrkande haschrock, till poesiuppläsning uppbackad av frijazz. Så här såg en tidig variant av bandet ut.


 Från vänster till höger står Greg Ginn, Chuck Dukowski, Robo (farbrorn med keps) och Dez Cadena (killen med otroligt ansikte).
 Sen den tiden har mycket förändrats för alla som var med i bandet. Henry Rollins, som är bandets kanske mest kände sångare, är nu en småsur gubbig tv-personlighet som går att jämföra med Kjell Bergkvist eller Leif GW Persson. Greg Ginn har fortsatt röka kopiösa mängder weed och har sedan slutet av 80-talet gett ut en stadigt värdelös flora band på sitt skivbolag SST Records (fram tills dess släppte SST bara bra grejer, nästan). Robo och Dez har spelat i Misfits (minus Danzig). Med den orkestern har de bland annat framträtt på en finlandsfärja.


 HEAD3: Återföreningar
 Black Flag har återförenats för några enstaka konserter sedan sitt uppbrott. Första gången var 2003, för att samla in pengar till en organisation som hjälpte katter. Ginn berättar om sitt intresse för katter i en intervju från den perioden: \textit{I got into it gradually and just got more and more involved to the point where I now have probably 80 cats.} Jepp mina damer och herrar, 80 kattjävlar. Mycket weed som sagt.

 Andra återföreningar, mellan 2003 och 2010 har inte alls varit lika ideologiskt drivna, utan handlat mest om 50-årskalas och musikgalor. Ungefär som när Dead Kennedys spelade i Avesta runt 2002. Men allt det kom att ändras 2013, då Black Flag tillkännagav att de skulle släppa sin första skiva på en jädrans massa år. Så här ser Black Flag ut 2013. Från höger till vänster står Gregory AMoore (trummor), Greg Ginn (gitarr), Ron Reyes (sång) och Dave Klein (bas).


 Trots att de ser ut som ett band din styvfarsa hyr in för att spela på ditt 40-årskalas, är faktiskt den första låten de släppte inte helt förjävlig. Det enda problemet är att den är släppt under namnet Black Flag. Starta ett nytt band, liksom. Samtidigt turnerar ett annat band med dekade rockers under namnet Flag. Flag består av Black Flags första sångare, Keith Morris, Dez Cadena på gitarr, Chuck Dukowski på bas, Bill Stevenson på trummor och Steven Egerton på gitarr. Tillsammans turnérar de runt i Belgien \textsc{(s.~\pageref{f79ffe9e826a19f9f6a446c90e21c4e3})} och Tyskland \textsc{(s.~\pageref{b1b58da783b6d5fa090f3015f1889869})} och mjölkar 100-åriga nostalgiker och borttappade 20-åringar som önskar att de nånsin fått uppleva något \quotetext{äkta}, på sina surt förvärvade småpengar.

 Slutet gott, allting gott.

}

\small{
\textbf{Blandfylla}
\label{ce03a6e2dd2abedfad0fceead180fec4}
 En blandfylla uppnås genom att inmundiga rusdrycker utan att ta hänsyn till dess karaktär. Den som vill uppnå en blandfylla sveper starköl, torr cider, Riesling, brännvin, mellanöl, likör, Fernet Branca, rakvatten och rosévin utan hejd. Blandfyllan är ett tveeggat svärd, antingen spårar man ur och spenderar dagen efter med att spy tills man får näsblod, eller så händer inget spektakulärt alls. Var aktsamma!

}

\small{
\textbf{Blandsvulst}
\label{9b3b156680385daf3f07ee17ac7588f4}
 en är en svulst vars innehåll består av naglar och hår från ett foster som aldrig utvecklades till fullo. Blandsvulsten uppträder således på barn som skulle ha blivit tvillingar men som istället, i livmodern, absorberat det andra barnet. Tvillingbarn som absorberar sina tilltänkta syskon absorberar dess styrka och blir dubbelt så mäktiga som vanliga människor.

}

\small{
\textbf{Blankt protokoll}
\label{4219b486a772f32467674f4516155f9e}
 Ett blankt protokoll är det finaste ett fordon kan föräras hos bilprovningen \textsc{(s.~\pageref{9516542b6862983521f399807cf913da})}.

 
 Dito anses dock ej fint som betyg från realskolan.

}

\small{
\textbf{Blauer Schwede}
\label{0ce4c1a6d9ef5d592e8030f2b6d8a8d6}
 är det tyska namnet på potatissorten som svenskar odlat i åkrarna sedan 1930-talet under namnet Blå Kongo. Färgen på skalet har gett den dess blåa namn men var Kongo och Schweden kommer in i bilden är inte helt klarlagt.

}

\small{
\textbf{Blaze Baylika}
\label{eb554f160fccb43920b1904dca6135c3}
 är ett kvinnonamn som ges särskilt partyglada och samtidigt råa flickor. För vissa slår det runt helt, men det kan ju hända alla människor oavsett vad man heter.

 \textbf{Kända personer med detta namn:}
 Agneta Blaze Baylika Paulo Coelho Sjödin
 Vivienne Patricia Blaze Baylica \quotetext{Patti} Scialfa
 Carola Blaze Baylika Maria Varg Vikernes Häggkvist
 Pippilotta Viktualia Rullgardina Krusmynta Blaze Baylika \quotetext{Pippi} Efraimsdotter Långstrump

}

\small{
\textbf{Blekinge}
\label{973711e4fb6d107e75da77ff9e4aedb7}
 är landets mest tätbefolkade landskap, sägs det. Vem bor då i Blekinge, kan man med all rätt fråga sig? Jo, majoriteten av Sveriges \textsc{(se Sverige s.~\pageref{b1999637949ed135b2ca03f3a38460cc})} nynazister, that's who! Av ett märkligt sammanträffande \textsc{(se märkliga sammanträffanden s.~\pageref{f46282d99158f351a81b9deaff157b4e})} kommer också Sverigedemokraternas \textit{übersturmführer} Jimmie Åkesson från detta germanska paradis. Många av bygdens söner och döttrar har sedan publicerandet av Olof Rudbecks \textit{Atlantica \textsc{(s.~\pageref{6a64780e5925164861e14768e0b04b1f})}} gått i polemik med denna Uppsala-son och hävdat att någon av öarna i landskapets vackra skärgård är den ariska rasens svunna Atlantis och alls inte Uppsalabygden som Rudbeck fräckt nog lät påskina. Som landskapsvapen har man ett ståtligt släktträd som heraldiskt representerar det Blekingska folkets genetiska utveckling. På grund av den edeniska atmosfär som råder har man problem med slaviska båtflyktingar som med alla till buds stående medel försöker ta sig till Blekinge och hotar att anhopa sig i kåkstäder utanför Karlskrona. Ett av de mest uppmärksammade exemplen härpå är den Sovjetiska U-båt som turligt nog gick på grund utanför landskapets kust och alltså inte hann lossa sin last av bleksiktiga karelska ryssar och andra svaga människor. Båten kunde sedemera avvisas å det bestämdaste av det lokala regementet där för övrigt merparten av Blekinges alla nynazister arbetar.

}

\small{
\textbf{Blogga}
\label{e1038f7b8db3fda54de45813a4878c1a}
 Som att klossa \textsc{(s.~\pageref{0e57164ee7dd893f0f88aac39c7806b7})}, fast den flyter. Flytetyget kallas för blogg.

}

\small{
\textbf{Blomkålsöra}
\label{abc81463a2d11b31c192a0fce03510a8}
 är ett muterat öra \textsc{(s.~\pageref{c4774ec92abe06f5664e18f44446d7e7})} som är vanligt hos brottare. Utövare av grekisk-romersk brottning får ofta huvudet nedtryckt i mattan, vilket gör att brosket i örat förstörs. Detta gör i sin tur att örat blir missbildat och lite påminner om blomkål, och \textit{voila!} - blomkålsöra!
 HEAD2: Kända personligheter med blomkålsöra
 Sveriges mest kända blomkålsöra sitter på sidan av Pelle Svenssons \textsc{(se Pelle svensson  s.~\pageref{26d88b383fd38f349c7741ca7051904e})} huvud.

}

\small{
\textbf{Blåmes}
\label{f5240c42dbdc261fb6e3fb8155edaa6b}
 En blåmes är någon som sympatiserar med eller har för vana att rösta på något av de borgliga partierna. Efter valet 2011 upptäcktes en under-art till blåmesen, nämligen brunblåmesen.

}

\small{
\textbf{Blåval}
\label{df5ac0401fcf8ea926b70eedfc67e82d}
 är katastrofer som inträffar med jämna mellanrum på grund av människors historielöshet. Allt går åt helvete då. Det senaste svenska blåvalet inträffade 19 september 2010.

}

\small{
\textbf{Blåvitt}
\label{0f52aa49c4f8ad1ffc94f831701fc119}
 Det fanns en tid då livet var mer än att konsumera glättiga saker. En tid då konsumtion bara var ett nödvändigt ont för att hålla sig vid liv. En limpa var en limpa. Toapapper var toapapper, utan några jävla lamm som skuttade runt.

}

\small{
\textbf{Bo Landin}
\label{b823a53e6e9ca5640e7f236e2f5cfc98}
 var en värmländsk biolog och journalist som under 1990-talet utmanade Arne Weises monopolställning som ciceron för tv-program om gulliga djur genom att på TV4 lansera den uppkäftiga uppstickaren \textit{Naturen}.

 HEAD2: Uppgång och fall
 I sitt program tog Bo tittarna närmare gulliga djur än någon svensk journalist tidigare gjort och slösade ingen tid på sedelärande historier om vikten av sopsortering, tjuvjakt och annat stämningssänkande. Här var det trivsel som gällde och om något rådjurskid ibland råkade fastna i en ölburk eller en pelikan svalde ett klockbatteri så klipptes det raskt bort. Allt har dock ett slut och efter tio år bestämde sig TV4 för att skrota programmet och satsa på ännu en dokusåpa istället. Avskedet ska ha varit smärtsamt för Bo och efter några tunga år runt millenieskiftet började han sakta etablera sig på den amerikanska marknaden som actiondjurfilmare, främst inom hundslagsmål.

 HEAD2: Okända sidor
 Som den renässansman han var hann Bo Landin även med att år 2004 producera och regissera en norsk-svensk filmatisering av Macbeth. Konstverket spelades in på ishotellet i Jukkasjärvi och finns att köpa på DVD hos flera nätbutiker för runt en femtiolapp.

 HEAD2: Slutet
 År 2007 befann sig Bo på de arktiska vidderna för att spela in ett inslag om konsten att jaga isbjörn. Tyvärr ville det sig inte bättre än att tältet han sov i misstogs för en präktig hona av en brunstig valross och Bo krossades på några sekunder under dess vikt. Filmeteamet begravde honom på plats och havsviken där händelsen inträffade heter numera \textit{Bosse bay}.

}

\small{
\textbf{Bob Dylan}
\label{827efb37d941b6390f6a9217719e95df}
 föddes 1941 i Duluth, Minnesota som Robert Zimmerman och blev, sedan han spelat in sin nasalgnolande stämma ackompanjerad av ackegura, en förgrundsgestalt för proteströrelser i USA under 60-talet, då  hans texter ofta innehöll skarp samhällskritik. Hans mest kända låt är The times they are a-changin' som handlar om att världen är i förändring och att den gamla generationen ska stiga åt sidan om de inte vill hjälpa förändringen.

 HEAD3: Akademisk influens
 Efter att de ungdomar som lyssnade på Dylan back in the day växt upp och blivit akademiker har Dylan blivit den mest citerade tänkaren inom den lärda världen, strax före flintbrillot Michel Foucault. Det är hart när omöjligt att slå upp en bok lånad från ett universitetsbibliotek \textsc{(se universitetsbiblioteket s.~\pageref{e69cf3b6f7c7f3872fe561600a7e9aa7})} utan läsa meningen \quotetext{Som Bob Dylan skrev 1963, 'the times they are a-changin'}. Låten citeras ofta för att den 50+ farbror (det är utan undantag farbröder som gillar brun sprit och tjeckiskt öl (och som tvångsmässigt måste påpeka hur mycket de gillar brun sprit och tjeckiskt öl så fort någon sorts dryck förs på tal) som lyssnar på Bob Dylan) som skrivit boken ska ge en hint om att han kommer från en rebellisk bakgrund, för att vinna läsarens respekt och beundran. Det som undgår författaren i det läget är att Bob Dylan är en av världens största artister någonsin och att försöka hävda att man är ball för att man lyssnat på Dylan är ungefär som att försöka hävda att man är cineast för att man sett Lejonkungen.

 HEAD3: Judas och grisbluesens genesis
 Efter att Dylan spelat akustisk gitarr i några år uppfanns elektriciteten av Graham Bell varpå Dylan tänkte att han skulle \quotetext{plugga in} och testa det här med att lira elgura. Detta väckte starka reaktioner hos de horder av hippies \textsc{(s.~\pageref{4dc77d6258fd18e7c0dd5eece5c7c81c})} han influerat, då de tyckte att inpluggandet var att sälja ut sin konstnärliga integritet. Vid ett tillfälle, 1966 i Manchester, England \textsc{(s.~\pageref{f48861ca24e26a23a923ca68657079f4})}, blev en snubbe i publiken så sur på den elektriska musiken att han ropade \quotetext{Judas!} till Bob Dylan, vilket gjorde Bob Dylan väldigt sur. Med sådana fans är det inte svårt att förstå varför Dylan idag är en av världens mest publikfrånvända live-artister. I dagsläget beskrivs hans musik som grisblues och han har alltid ryggen vänd mot publiken när han sitter vid sin orgel och spelar de minst populära låtarna från hans senaste pisskivor, självklart alltid inför slutsålda arenor fyllda till bredden med farbröder som envist hävdar att de gillar framträdandet. Huruvida Dylan bara existerar som artist idag för att testa hur långt han kan gå med att skita sina anhängare i ansiktet utan att de märker det är oklart, men hans julskiva borde ge en fingervisning.

 HEAD3: Dylan och Sverige
 Sverige \textsc{(s.~\pageref{b1999637949ed135b2ca03f3a38460cc})} har som bekant en av de högsta nivåerna av akademiker per capita i hela världen och därmed också en väldigt stor Dylansk anhängarskara. Detta har förorsakat att en oöverskådlig mängd böcker om svenska farbröders förhållande till Dylan getts ut, alla har blivit bestsellers. Sveriges finaste Dylan-fanboy är Mikael Wiehe som är så gullig att han kommer undan med att gilla Bob Dylan. Kolla bara [http://www.youtube.com/watch?v=feXjFU4zmzg] här!!

}

\small{
\textbf{Bockskäggsmetal}
\label{a41cf0247e1e8fe6d074fd8d3c8d1c5f}
 är en musikstil som är rätt trist.

}

\small{
\textbf{Boden}
\label{7c5dfb91b1d55bff98ec6d4faf83976b}
 betyder botten på tyska och det är en rätt målande beskrivning av denna mänsklighetens bakgård.
 Om svensk film \textsc{(s.~\pageref{9a059a165228aea794882a2677ac50c3})} behöver ett stycke miserabel betong väljer man Boden som inspelningsplats (sant!).
 Man hade en gång ett sjukhus, 10 000 bassar och en travbana, nu är blott travbanan kvar och bara tills nån kommer på att man kan flytta den till Luleå \textsc{(s.~\pageref{3cefb5ac35187749592f1ebb25472b99})}. Det största som hänt i Boden på modern tid är att Lenin bytte tåg där 1917. Sen åkte han till Ryssland och gjorde historia \textsc{(se mänsklighetens historia s.~\pageref{5d87ba4132f8bdfa8c6294514c570c3f})}. Som alla kommuner på dekis har Boden självklart ett äventyrsbad \textsc{(s.~\pageref{8e36481b72c8061bb9ff74c1df3b0b66})}, därtill en vilda västernstad där man kan dricka sig redlös och hamna i slagsmål, gärna efter travet.

 HEAD2:  Bodens \quotetext{centrum} har många sevärdheter
 På Bodens bibliotek kan man möta många intellektuella från afrikas horn som köper guld, samt med begränsad framgång försöker köpa sex av ensamma tjejer.
 Utanför biblioteket kan man träffa infödingar som köper tjack \textsc{(se tjackad s.~\pageref{b12ceb5f265e6ab9afcd2c662715e0b5})} och gillar att åka moped. Längre bort i byn ligger Kafe Kaf som har helt okej mackor. Ännu längre bort ligger Rönnbäcks gatukök vilket saluför sin berömda flottbomb, vi talar alltså 2x250g, en kartong strippe samt all gegga du vill. Det är ett monster!
 HEAD2: Bodens landsbygd
 Ända fram till den nesliga kommunsammanslagningen 1968 ingick den nordöstra delen av kommunen i Råneå landskommun och den västliga delen var självstyrande, Edefors landskommun. När det nya storboden skapades gick det raskt utför med dessa trevliga småorter.

}

\small{
\textbf{Boetius de Dacia}
\label{ad665ff4eb3f48ad8f41fc7fc9c246c8}
 var en kristen filosof som levde under senare hälften av 1200-talet. Boetius är en latinsk version av det nordiska namnet Bo och de Dacia betyder \quotetext{från Danmark \textsc{(s.~\pageref{5331d7fd27772396f412a5b6d19bad44})}}. Informellt, nordiska kristna filosofer emellan, kallades han Dannebosse.

 Boetius dog i skam, efter att ha blivit utsparkad ur Rom. Hans exil berodde på att han utöver att vara kristen var naturvetare.  Således sökte han tvångsmässigt logiska lösningar på allt möjligt. Eftersom logik och religion sällan går hand i hand uttryckte Boetius en del tankar som inte var råpoppis hos storfräsare \textsc{(s.~\pageref{4db17005692cd83e3e946a1311b81ed0})} som påven. Typ åsikten att döda inte nånsin kan leva igen var inte superpopulär i 1200-talets Rom.

 Sedan Boetius de Dacias död har logiken varit tabu i konungariket Danmark. Skolbarn med nyfikna, utforskande sinnen som funnit logiska samband i sin vardag har i århundraden fått sig en dansk skalle av ansvarstagande vuxna, ackompanjerat av det danska ordstävet: \quotetext{\textit{Nej du raske lille dreng, som taenke sig sandheden se - hold du din kaeft eller gå ud i exil som han Dannebosse!}}

}

\small{
\textbf{Bohuslän}
\label{bb5862965ae625fab43df483726de9d5}
 finns inte längre, upphörde att existera 1998. Det finns ett landskap kvar som heter Bohuslän men det är en annan femma \textsc{(s.~\pageref{d974e0811fe7a4d49a9062d33b66a88d})}.

}

\small{
\textbf{Bokbål}
\label{95ba942fddfa43299693158f961bfa99}
 är en praktik när man kastar böcker på en eld, men så fort man bränner böcker så kommer man också snart att bränna människor. Så sluta upp med den skiten och gör nåt vettigt på fritiden, typ rundpingis \textsc{(s.~\pageref{921d37e7abc34a9b324440904981aabf})} eller nåt.

 Enligt vetenskapliga studier går surdegspappor \textsc{(s.~\pageref{617533958bf226f7259a890bb6c15822})} häpnadsväckande ofta över till bokbål; om dom inte blir bög \textsc{(s.~\pageref{a1bd23bf21b5add8ef4fceba9c763237})} förstås.

}

\small{
\textbf{Bonad}
\label{1401e262dbae3ccdfdf165d7a1700595}
 En bonad är ett stycke broderad väv som hängs upp i hemmet, ofta ovanför dörrar, för att inge de boende och eventuella besökare en känsla av frid och lågintensiv glädje, eller en känsla av att i universums perspektiv vara blott ett futtigt sandkorn. Den typiska bonaden är rektangulär och kantad med ett blommigt mönster, vari en klok uppmaning, sanning eller betraktelse står att läsa i snirkligt utförda bokstäver. Likt en rustik fax \textsc{(s.~\pageref{236c3b7f761221f195b428aca2f06c4b})} sänder alltså bonaden ut ett meddelande från brodör till mottagare.
 HEAD2: Typiska klokheter som kommunicerats via bonader
 \begin{itemize}
 \item Livet är för kort för att vara dammfritt
 \item Gör ej det idag, som kan skjutas upp till morgondagen
 \item Små små ord av kärlek / sagda varje dag / ger åt livet lycka / åt hemmet sitt behag
 \item Man ska inte sörja det man saknar, utan glädjas åt det man har
 \item Försumma inte det lilla du förmår för det stora du inte mäktar
 \item Livet är inte de dagar som gått, utan de dagar man minns
 \item Käng och sprit, gubbjävel!
 \end{itemize}

}

\small{
\textbf{Bonfire}
\label{b0759e17c7cc70d7522a6b63a05c914e}
 är en samlingsbox av det australiensiska \textsc{(se australien s.~\pageref{e727d8d1b3162a732c7f706d55de64f3})} rockbandet AC/DC. På över fem \textsc{(se femma s.~\pageref{d974e0811fe7a4d49a9062d33b66a88d})} skivor får man ta del av bandets musik både live och i studio, utgivet och tidigare outgivet material. Boxen är en hyllning till bandets avlidne sångare Bon Scott, som enligt Malcolm Young sa: \quotetext{when he's a fucking big-shot, he wants his solo album to be called 'Bonfire'}. Förutom musiken innehåller boxen också en poster, nyckelring, backstagepass och en bok full med häftiga bilder på Bon och bandet. Eftersom boxar är ganska dyra grejer är det inte alla som har råd att köpa såna, särskilt inte AC/DCs fans som alla är vanliga hederliga knegare. För att komma runt detta problem startade AC/DC den digitala musiktjänsten \textit{Spotify} där man kan lyssna på hela Bonfire gratis.

}

\small{
\textbf{Bonnseg}
\label{dec68c64fff51615cb6156b75333870c}
 (Adjektiv) beskriver någon som till det visuella mest ser senig ut men som i själva verket är senig och exceptionellt stark. Ordet kan ha uppkommit pga att bönder \textsc{(s.~\pageref{30a6fc00c9102680b8196b1b79935ec4})} förr i tiden inte hade resurser nog att äta sig tjocka trots att de kroppsarbetade hela dagarna. Personer av rang som är eller har varit bonnsega:


 \begin{itemize}
 \item Kåra-Henrik, hovslagare.
 \end{itemize}

 \begin{itemize}
 \item Lisbeth Salander, mördare och datasnille.
 \end{itemize}

 \begin{itemize}
 \item Nestor Machno, en av arkitekterna bakom den anarkistiska rådsrepublik som uppstod kring 1920 i östra Ukraina genom handlingens propaganda, företrädelsevis till häst \textsc{(s.~\pageref{b4c608370b339da095c5f8db7fab0945})}.
 \end{itemize}

 se även indianmuskler \textsc{(s.~\pageref{a0e24bd0dfe9431f72896e16614e79c0})}

}

\small{
\textbf{Boris Jeltsin}
\label{0d1b5bf57840dff017d7bbcd4c196dcb}
 är namnet på en extremt prisvärd vodka man kan köpa i Tyskland.

}

\small{
\textbf{Bortamatch}
\label{cd6d0633ccd83f4f6c53b718ff787942}
 Man vaknar och funderar på var man är. Man letar ihop sina kläder, men glömmer minst ett (1) plagg. Man möts av okända människor i hallen och slås av en oerhörd känsla av skam. Man letar sig ut ur huset och möts av en aldrig så irriterande sol. Man frågar en förbipasserande vad klockan är och vilken dag det är. Man använder sina sista vakna hjärnceller till att hitta hem. Man möts av ens kombos som kommer med gliringar och jobbiga frågor.

 \textit{Bortamatch}

}

\small{
\textbf{Bortstoppad uv}
\label{86574b11bb49a6f8e32d9f716676236a}
 En bortstoppad uv är en uv \textsc{(s.~\pageref{45210da832f9626829457a65e9e7c4d0})} som bedömts opassande att ha framme när man t.ex. får gäster och således stuvats in i ett skåp eller upp på vinden. Den bortstoppade uven sitter sen i detta mörka utrymme och funderar på om det är dag eller natt, men kan omöjligen bli säker på varesig det ena eller det andra. Det här är mycket förvirrande för uven. Sen gråter den uvtårar \textsc{(s.~\pageref{8f041258aba5f2fb5aca5d11b2c5f1b0})} och undrar vad den gjort för att förtjäna en sån här behandling. \quotetext{Var det nåt jag sa?} kanske den tänker, men den kommer sen fort på att uvar inte kan prata och fortsätter fundera. Uven börjar känna sig förtvivlad till slut, men dessa känslor omvandlas fort till den känsla som ligger uvar närmast. Hämnd.

}

\small{
\textbf{Bosporen}
\label{6010d16af1ae60a54bd546875577f7c4}
 är ett sund i Turkiet som delar Istanbul i två delar. Utan Bosporen skulle det inte finnas någon sjöförbindelse mellan Svarta havet och Medelhavet, så ni kan tro det är en viktig farled. Fast den är bara 700 meter bred på vissa ställen så det blir riktigt trångt när alla båtar ska kajka fram och tillbaka. Antagligen har en massa olika grupper genom historien krigat om vem som kontrollerar Bosporen men det var för jobbigt att leta fram fakta om. Men förr eller senare brukar det ju bli krig om det mesta [http://www.youtube.com/watch?v=0L0YVJC1E-A].

}

\small{
\textbf{Bot-fan}
\label{aabae9c99305dd2cf1f7cb4e8bc22be2}
 är ett djur som tillhör grenen digitala virus-liknande saker. Den kommer hit till Nissepedia \textsc{(s.~\pageref{62400dadecd90cb5cd39062abe5a3e4a})} och ställer till med jävelskap \textsc{(s.~\pageref{46845591177f16920cd586a5baf5a625})} då och då, till allas förtret.


 HEAD2:  Se även

 \begin{itemize}
 \item De stora bot-attackerna 2011 \textsc{(s.~\pageref{c7dd995dad0d892085806b68800cca79})}
 \end{itemize}

}

\small{
\textbf{Botte}
\label{59d505d6448b03a0331e6fc09a69d3b9}
 är ett namn som ges till exceptionellt trevliga människor som har ett brett socialt nätverk. Det är inte så många som heter Botte heller så det är jättelätt att presentera sig. Säger man att man är Bottes pajk så är allt lugnt. Botte är en försvenskning av engelskans \textit{bottle} (butelj).

}

\small{
\textbf{Brak}
\label{18144a52e0ab0417932d87bf084491ae}
 är ett annat ord för stor, vilket kan exemplifieras av orden braksuccé \textsc{(s.~\pageref{678371d35369d3d29afceb1445630833})} och brakare \textsc{(s.~\pageref{da8590943fa645cfceaa235a83d1d797})}.

}

\small{
\textbf{Brakare}
\label{da8590943fa645cfceaa235a83d1d797}
 En brakare är ett annat ord för en sedel med valören 1000 kronor och synonymt med tunka, lakan, lök, släng, tuss, och ett K.

}

\small{
\textbf{Brakflopp}
\label{9bf3e16604e00c99d3a6b9b93e4d68c4}
 Någonting som går åt skogen.


 HEAD2: Kända Brakfloppar

 Carolas inhopp som sångerska i Marduk

 Birgit Friggebo försöker \quotetext{få ner} stämningen under det infekterade PR-mötet i Rinkeby 1992, genom att mana till allsång av \quotetext{We Shall Overcome}.

 Karl XII invaderar Ryssland

 Microsofts lansering av Digerdöden 1340

 Åka från Tilburg till Den Bosch för att köpa knark

 Minitel

 Nickes samtliga bilköp.

}

\small{
\textbf{Braksuccé}
\label{678371d35369d3d29afceb1445630833}
 är ett annat ord för en stor succé, alltså något väldigt lyckat tilltag, så som uppfinningarna Korv med bröd \textsc{(s.~\pageref{8ed6a229bd465c6f2a0a73f65534056b})} och pudaslåda \textsc{(s.~\pageref{6a56958e2057dd500650e2be8049e033})}.

 På grund av Braksuccé med Braksuccé så har lobbyistorganet BrakfloppInternational instiftats med syfte att stimulera användandet av termen Brakflopp \textsc{(s.~\pageref{9bf3e16604e00c99d3a6b9b93e4d68c4})}. Desvärre, som namnet antyder, så blev detta ingen Braksuccé utan något utav en Brakflopp \textsc{(s.~\pageref{9bf3e16604e00c99d3a6b9b93e4d68c4})}.

 ‏

}

\small{
\textbf{Brandklipparen}
\label{e8aaa0dc22fb08e055f0f48b2f25e35d}
 var Karl XIIs häst \textsc{(s.~\pageref{b4c608370b339da095c5f8db7fab0945})}. Den ligger begravd med minnessten och allt utanför Ängsö slott där den dog efter att ha sprungit rätt in i en stenmur. På stenen står: \quotetext{Konung Carl XII:s siste häst stört anno 1740}. Klippare (är ett onomatopoeiskt ord och) betyder ungefär klappra, precis som engelskans \quotetext{clipper} och latinets \quotetext{clipperus}. Var hästen dessutom rödbrun till färgen var namnet givet. Han härstammade från Småland och fick inga kända avkommor.




 En hyllningsdikt till Brandklipparen, skaldad av Carl Snoilsky:
 \textlessi\textgreaterBrandklipparn frågade ej: varthän?
 Och ännu mindre: varför?
 Men travade friskt bland snön i Ukrän
 Som på sin äng vid Kungsör.

 En häst, en häst att rädda vår kung,
 Ur hängbår krossad och stjälpt!
 Brandklipparn kom över rökhöljd ljung
 På honom vart kungen hjälpt.

 En sakta gnäggning han från sig gav,
 Det låg en klagan däri:
 Jag mäktar ej mera - herre, sitt av!
 Jag tror, att det är förbi.

 Kungen han dödde, han trillade av,
 Men havremoppen travade på.
 Av majestätet blev det bara skit kvar,
 Men pollen var glad ändå!
 \textless/i\textgreater

}

\small{
\textbf{Bredvid}
\label{f28dfb27edfaf756fc3b10514f0aeab2}
 tar upp onödigt mycket plats, då det hade räckt med antingen bred eller vid.

}

\small{
\textbf{Brevlåda}
\label{82763e379777595d3c8c8f7b38e51bbd}
 En brevlåda är en behållare av plåt som används för att förmedla budskap, korta och långa, över hela världen. Sändaren skriver sitt meddelande på ett papper, eller annat lämpligt material med en plan yta, och anger även en position för vart i världen den vill att meddelandet ska föras. En person i blå kläder tömmer varje dag behållaren och för sedan meddelandet till den angivna punkten och placerar den i närmsta postlåda \textsc{(s.~\pageref{8134b1e3ad53642c3d3373e6ea72ed35})}. Snabbt, enkelt, bekvämt!

}

\small{
\textbf{Brian Epstein}
\label{73a1efc86272f861b10c61c49671bc1f}
 (1934-1967) var en brittisk skivhandlare och manager. Han upptäcktes av Pelle Karlsson \textsc{(s.~\pageref{1a8c873ff230698396c324f14c02b7fa})} och deras samarbete ledde bland annat till att Pelle blev kontrakterad av Hemmets Härold \textsc{(s.~\pageref{812e1c9a5a14e8d21ece7bfbdba893c4})}. Vis av de erfarenheter han fått av Pelle gick Epstein sedan vidare till att sköta affärerna åt The Beatles och Garry \& the Pacemakers.

}

\small{
\textbf{Bricka}
\label{0072a4ca9825dbec7dfa6e6cb9b23022}
 En bricka är ett slags plan skiva av trä, metall eller plast varpå saker ställs och transporteras. Brickan kan ha en kant som förhindrar att godset som transporteras på den faller av och därmed också att genanta situationer uppstår. De bästa och mest omtyckta brickorna är prydda med mönster, så som kurbits, eller bilder, som kan föreställa söta djur, barn \textsc{(s.~\pageref{5dfcc0aab2f3db925b2d51ba73e48946})} som pussas eller en vacker vy av äng och sjö. De sämsta brickorna är tillverkade på träslöjden och skänks bort som julklapp. Dessa brickor saknar ofta både bilder och mönster. Istället har skaparen av okänd anledning bränt in \quotetext{AIK} eller \quotetext{häst är bäst} i brickan med glödpenna.

 HEAD2:  Olika sorters brickor

 \begin{itemize}
 \item Ostbricka \textsc{(s.~\pageref{bf06c995c523e159eb93017810ee8f44})}
 \item Mäklarbricka \textsc{(s.~\pageref{818c8a106ef37f756992c962950fd74b})}
 \end{itemize}

}

\small{
\textbf{Brillkommenister}
\label{baa057b925aa9c528ab62b48fb8cdc05}
 Tycker man måste engagera sig, gärna genom att samla namnunderskrifter \quotetext{Mot nedskärningarna} eller \quotetext{Mot Rasismen}.
 Brillkommenister gillar definitivt multikulti \textsc{(s.~\pageref{25eea9148080d30d384ce1c1277ef126})} och bär troligen avancerade glasögon \textsc{(s.~\pageref{af3e24904a76537ca385e42fe952fa6c})}. Man tycker borgerligheten är för hemsk men gillar detta med ROT och RUT så man kan höja värdet på sin bostadsrätt.

}

\small{
\textbf{Brinner för att sälja}
\label{4397dcfa1c80db06a775fb49f5171806}
 Den som brinner för att sälja brinner också någon annanstans, vad det lider.

}

\small{
\textbf{Brismonstret}
\label{81d5803194a4a25fe4c75e9f9857d980}
 är den ideella organisationen Barnens \textsc{(se Barn s.~\pageref{5dfcc0aab2f3db925b2d51ba73e48946})} Rätt I Samhällets[http://www.bris.se] maskot. Ingen vet varför BRIS valde ett monster som maskot, men en möjlig förklaring är att reklambyrån Fultjack \& Co bjuder sina kunder på spacecake när det är dags att välja bland deras förslag. Brismonstret har inget namn, dock ser det lite ut att heta något i stil med Reidar den Röde.
 


 ]]

}

\small{
\textbf{Britts mode}
\label{4222116edbe095681ea4a4513b21bd44}
 är en anrik kvinnoklädsaffär \textsc{(se kvinnokläder s.~\pageref{d09ebf3842ad7452891cf646bf47b3a0})} i byn Rimbo \textsc{(s.~\pageref{d93bb016d5aea9c6faa22c5b544a4fdb})} i Norrtälje \textsc{(s.~\pageref{7527f7dad9445013a559dc7e2a91f3b3})} kommun. Britts mode tillhandahåller byxdresser, kjolar och blusar, samt klänningar till damer i den övre medelåldern som härstammar från glesbygden snarare än den mer mode-intensiva storstaden. I området finns också en sjundedagsadventistisk skola \textsc{(s.~\pageref{e80411442bbb22f9ae7ed44d5780cfc2})} och förmodligen utgör dess kvinnliga gymnasieungdomar också en potentiell kundkrets, eftersom dessa förkastar samtida mode och, om man ska vara helt ärlig, samtiden generellt. Många Rimbo-kvinnor \textsc{(s.~\pageref{fee636b63bc1a08e245dc0aaf820b974})} handlar dock ändå sina kläder från etablerade klädkedjor eller i någon av Britts bittra konkurrenters klädbodar, av en anledning vilken inte enkelt låter sig förklaras.

}

\small{
\textbf{Brobo}
\label{81277c9a75a38cea387057fe30fe3204}
 En leva som brobo innebär att man bor med sin bro'. Detta levnadsskick föregår ofta vad som ofta kallas foebo.

}

\small{
\textbf{Brokiga Blad}
\label{379db04f3b9eac09c39343f0592c224a}
 var signaturen luffaren och konstnären Herman Sixtus Andersson (1873 i Norrköping - död 1954 i Norrköping) ibland använde vid signeringen av sina tavlor. Anderssons mål med sitt konstnärskap var, förutom överlevnad, att måla en miljon tavlor så att alla i Sverige skulle ha varsin. Enligt Hönsalottas luffarmuseum hann han dock bara måla mellan 200 och 300.000 tavlor, samtliga med landskapsmotiv. För att hinna med att måla alla brukade Andersson måla ungefär 20 stycken åt gången. På så sätt kunde han grönmåla alla först för att få gräs och när den sista var klar hade den första torkat så att han kunde börja måla blå himmel. När djuren skulle in i bilden använde han schabloner och många tavlor kom därför att se ganska lika ut. Vinkeln på djuren kunde dock skilja sig åt för ju fullare han var, ju snedare lär dom ha blivit. Tavlorna kostade vanligtvis 2.50 styck. Istället för färg använde Andersson sockerlösningen Dextrin, strösocker, vatten och potatismjöl. Det kokades ihop till en sörja som han sedan blandade kulörpulver i för att få rätt färg. Strösockret hade han i då det gjorde att färgen torkade snabbare. Under sitt kringflackande liv lär Andersson ha mött både Anders Zorn och prins Eugen, men det är oklart om något utbyte i inspiration och teknik skedde.

}

\small{
\textbf{Brugd}
\label{d6b6b68506b8f1daad3a2ddbfaf8d863}
 en är världens näst största fisk.

 HEAD2: Taxonomi
 Brugden döptes av Carl von Linné \textsc{(s.~\pageref{5e8380bf6b7ce99678e6752b6d9e709e})} till \textit{Cetorhinus maximus} efter det latinska översättningen av den gamla KSMB-dängan \textit{Glappkäft}. Den är ensam medlem i familjen \textit{Cetorhinus} och är jätteledsen över det här. Ofta simmar de runt och gråter, men i havet kan ingen se dess tårar.

 HEAD2: Föda
 Brugden är enligt alla definitioner en haj, men om hajsläktet liknas vid olika subkulturer där vithajen är hårdrockare, hammarhajen skinhead och tigerhajen punkare så är brugden hippie. När dess artfränder simmar omkring och äter andra fiskar och ställer till med jävelskap \textsc{(s.~\pageref{46845591177f16920cd586a5baf5a625})} så simmar brugden omkring och gapar och äter plankton. Det kan också vara så att brugden fascineras asmycket över havet och helt enkelt simmar runt och gapar över dess storslagenhet, råkar på köpet svälja en massa plankton och har genom detta ödets nyck inte slagits ut av det naturliga urvalet. Där har den moderna vetenskapen, återigen, misslyckats.

 HEAD2: Fortplantning
 När brugdar ska \quotetext{få till det} så simmar en av dem uppochner var på den andra liksom dockar sina könsorgan med den andras. Det är helt omöjligt för den andra att veta om det var skönt eller inte då brugden ändå gapar hela tiden. Att brugdar inte har utvecklat något språk är också en bidragande orsak. När brugdarna kopulerat färdigt skiljs de åt lite tafatt sådär och ses inte föränn äggen kläckts. Brugdhanar är asdåliga farsor.

 HEAD2: Brugden som vara
 Brugden är extremt sällan en del i det kapitalistiska systemet, men enligt sägnen ska fiskmånglaren Randolfo \textsc{(s.~\pageref{b8f0a32f840f1db27a2c12e17b640fb2})} en gång ha köpt in en brugd som han försökt kränga i sin fisk/skivaffär \textsc{(se kombinationsaffärer s.~\pageref{0a2777bf1366a8a9a5b8eab9ca1496a1})}. Den dog i hans föräldrars badkar och såldes senare som fyllning i Humlans \textsc{(se Humlan s.~\pageref{113017afa0a9549cccc931300ba2edb3})} tonfiskbaguetter. Dock är det inget fel på själva smaken av brugd, något som den kände ymparen Nick Holdone tagit vara på i sin kreation brugd-sviskon \textsc{(s.~\pageref{a6a1c3bbd109173fc773aec1dc6754e0})}.

 HEAD2: Brugden inom sociologin
 Socialfilosofen Anthony Giddens har liknat moderniteten vid en brugd, en kraft som ingen kan kontrollera utan bara går på utan en tanke på vad som komma skall. Liknelsen är pissdålig då brugden är betydligt softare \textsc{(se stenad s.~\pageref{dec4a3a91f0f2bf8dcf033a8cfeaa554})} än moderniteten.

 HEAD2: Brugden inom sporten
 I Mexiko - tuppfäktning, på Sveriges \textsc{(se Sverige s.~\pageref{b1999637949ed135b2ca03f3a38460cc})} västkust - brugdrace. Tävlingen går ut på att två brugdar fångas in och spänns framför varsin luftmadrass. Vidare placeras en bohuslänning som inte bryr sig om denne lever eller dör på madrassen och racet är igång. Det största brugdracet är Orust Runt där även hinder i form av flytminor placerats ut. Trots att djurrättsaktivister försöker stoppa tävlingen varje år har den hållit på sedan den uttalade brugdrace-entusiasten Oskar II:s tid.

}

\small{
\textbf{Brugd-sviskon}
\label{a6a1c3bbd109173fc773aec1dc6754e0}
 Sviskon med en distinkt smak utav brugd \textsc{(s.~\pageref{d6b6b68506b8f1daad3a2ddbfaf8d863})}.

}

\small{
\textbf{Brugdguldet}
\label{99e8882b92612b0d60d660ddd1233587}
 delas årligen ut till en person eller grupp som åstadkommit en stark prestation relaterad till näringsintag, eller någon annan extraordinär prestation som för tankarna till brugden \textsc{(se brugd s.~\pageref{d6b6b68506b8f1daad3a2ddbfaf8d863})}, denna världens näst största fisk.

}

\small{
\textbf{Bruksgök}
\label{3b67db58a20f207d4ebb61801133e5f4}
 En bruksgök är en tam eller halvtam gök som hålls för vissa bruksändamål. Vanliga sådana ändamål är:

 \begin{itemize}
 \item Tidtagning
 \item Rensning av avlopp
 \item Underhållning
 \end{itemize}
 HEAD2: Pris och prestanda
 En bra bruksgök med god stamtavla pensioneras vanligtvis efter ca sju år, men i vissa fall har bruksgökar varit arbetsaktiva så länge som tio år. Priset på en vaccinerad bruksgök med veterinärbesiktning låg i Augusti 2011 på mellan 100 och 150 kr, men sjönk under slutet av 2011 och början av 2012 som en effekt av oroligheten på de europeiska finansmarknaderna. Inte ens denna näringsgren går opåverkat ur oroligheterna på världens börser alltså.
 HEAD2: Skötsel
 Används bruksgöken för att rensa avlopp bör den spolas av ordentligt efter varje brukstillfälle. Annars räcker det med en gång i månaden. Kasta åt den lite cornflakes en gång om dagen så blir det bra. Vatten kan också vara nödvändigt, men den behöver inget fint bordsvatten, utan vanligt krandito fungerar utmärkt.
 HEAD2: Vanliga namn på bruksgökar
 Fartyg döps ofta efter kvinnor, brukgökar ofta efter arenarockband och isländska skalder som:
 \begin{itemize}
 \item Egil Skalagrimsson
 \item U2
 \end{itemize}
 HEAD2: Synonym
 Bruksgök är även en arkaisk fras, synonym med fenomenet vi idag kallar produktionsknull \textsc{(s.~\pageref{c89065c54da84e9bc9dc59992a4dffa8})}.

}

\small{
\textbf{Bruksortskäng}
\label{288a48443155dc360f3fda6ad75b3c0d}
 är en speciell sorts kängpunk, geografiskt specifik för mellansverige. Än mer specifikt kan genrens kulturella center preciseras till vad som inom akademiska kretsar kallas bruksortskängtriangeln. Bruksortskängtriangelns noder är Hedemora i norr, Örebro i sydväst och Eskilstuna i sydost. Inom denna triangel har all relevant bruksortskäng någonsin producerats. Meanwhile, Asocial, No Security, Svart Parad, Crude SS, Uncurbed m.fl. har alla haft sin bas i den mellansvenska myllan.

 Vad som skiljer bruksortskängen från vanlig jävla käng är svårdefinierat. Rent generellt handlar det givetvis  om d-takt och risig produktion - business as usual så att säga. Kanske är det den stora finska diasporan inom triangeln som bidrar med ett mystifierande \textit{je ne sais quoi} till evergreens som Offer för ett mord av Svart Parad? Eller bör man söka sig bortom kultur och gå till materia och ekonomiska betingelser, som Marx \textsc{(se varför Marx hade rätt s.~\pageref{809ed1a081f14fb42e23fee9b229c44f})} hade påbjudit? I såna fall finner vi anledningen till bruksortkängens överlägsenhet, i att denna triangel är den del av Sverige som blivit mest brutaliserad av modernitetens framfart (näst de stackars Baggböleborna, men det resulterade inte i nån bra käng). Lika snabbt som löften om guld och gröna skogar kom, lika kvickt ryktes de undan av rikskapitalister som flyttade sina affärer till utvecklingsländer på 80-talet. Den besvikelse och desperation som präglar bruksortskängtriangeln, har onekligen färgat musiken som kommer därifrån.

}

\small{
\textbf{Brumma}
\label{e199eb7ebcb529a43335ce5690788e66}
 är slang för skita och härstammar från den oförglömliga badrumsbombscenen i dödligt vapen 2 där Mel Gibson frågar Danny Glover, som sitter på toa, vad han håller på med. Danny svarar då, lite nonchalant, ”jag brummar”.

}

\small{
\textbf{Brunei}
\label{0be0d89a7660a0f90bc223b97924a527}
 Ett fint ställe.

}

\small{
\textbf{Brunka}
\label{e9b217feddf7c7c0f64400fe683de947}
 (verb, obestämd form singular; bestämd form brunk) är fenomenet att masturbera efter avklarad avförings-session, gärna av ren och skär uttråkning. Några få individer har berättelser om hur de utfört dessa handlingar samtidigt, men det har ännu inte styrkts av SIFO.

}

\small{
\textbf{Brunt}
\label{21da6b54ea17e869ea6d48557107b6fa}
 Slang för lågbudgetvarianter av chokladkräm typ Nutella.
 En impopulär lågbudgetvariant heter \quotetext{Choco Nussa} och finns att inlanda i Lidl-butiker runt om i Europa.
 Används även inom uttryck som \quotetext{Jag skulle inte göra det för allt Brunt i Husqvarna} eller inbjudningsfraser som \quotetext{Ska du med hem på en kopp kaffe och lite Brunt?}

 bild: http://fddb.info/static/db/108/schokoladencreme-choco-nussa.jpg



 Category: Mat \textsc{(s.~\pageref{f1f39d486a77189ba935eee9ef98cd3f})}

}

\small{
\textbf{Bruvd}
\label{7bb713571f15e8d3940fb38184b81991}
 en är ett fasansfullt kreatur som är en korsning mellan en brugd \textsc{(s.~\pageref{d6b6b68506b8f1daad3a2ddbfaf8d863})} och en uv \textsc{(s.~\pageref{45210da832f9626829457a65e9e7c4d0})}. Brugdens gigantiska gap och kroppstorlek har kombinerats med uvens förmåga att flyga och även deras onda sinnelag. Ser du en bruvd är det kört. Du kan försöka springa, men enda skillnaden är att du dör trött.

}

\small{
\textbf{Brygguv}
\label{5894c643203150034da4bc9347e32354}
 Det finns en typ av uvar \textsc{(se uv s.~\pageref{45210da832f9626829457a65e9e7c4d0})} som bara har en uppgift: Att sitta och se snygga ut.

 Förekommer ibland i smärre flockar, men även en och en.

}

\small{
\textbf{Brännvin}
\label{ff49ececa32cff978496a39635496f46}
 är ett urtida bedövningsmedel. Det är även ett mycket bra lösningsmedel, löser fett utan problem. Det är också bra på att lösa problem, i mindre utsträckning att skapa dem. C2H5OH är det som bör förtäras, andra varianter kan ge vissa oönskade bieffekter, som t.ex. CH3OH som kan leda till en formalinindränkt död. Introducerades i Sverige \textsc{(s.~\pageref{b1999637949ed135b2ca03f3a38460cc})} av Eva Ekeblad \textsc{(s.~\pageref{de66bb2a3f5c71b15a204f8e773ea925})}.

 HEAD2:  Externa länkar

 [http://www.gitsolsson.se/kronvodka.htm Lite sprithistoria]

}

\small{
\textbf{Bröka}
\label{60862d3b986c7bbedc86064c842c5a6c}
 (verb, obestämd form singular; bestämd form \textit{brök}) är en multipel handling där utövaren njuter ett rökdon samtidigt som den träckar. Fenomenet har blivit mindre vanligt på senare tid i och med utedassens dalande popularitet och rökförbud på krogen.

}

\small{
\textbf{Bröstarkt}
\label{9d9073d6a3c73a175f38e959d256cbff}
 är ett ord som beskriver något som innehåller mycket bröd. En macka kan till exempel upplevas som bröstark om man har för lite smör, ost och skinka på den. En pizza \textsc{(s.~\pageref{7cf2db5ec261a0fa27a502d3196a6f60})} kan också vara bröstark om man har för lite tomatsås, ost och skinka på den.

}

\small{
\textbf{Bubogram}
\label{f982d96c5736941c5c32a279cf570719}
 Ett bubogram är när man får post medelst uv \textsc{(s.~\pageref{45210da832f9626829457a65e9e7c4d0})}. Det kan i vissa särfall vara en uv som har med sig en inslagen uv.

}

\small{
\textbf{Bubologi}
\label{dbb8269622d6bcc3580ec1c2d8d43bd9}
 Nissepedias \textsc{(se Nissepedia s.~\pageref{62400dadecd90cb5cd39062abe5a3e4a})} huvudämne.

}

\small{
\textbf{Bubonaut}
\label{c09400b667b41bb7bd0de4f9aa2d71ce}
 är det klassik-grekiska ordet för en person som flyger på ryggen av en uv \textsc{(s.~\pageref{45210da832f9626829457a65e9e7c4d0})}. Jämför med argonaut, en person i det mytologiska skeppet Argos besättning.

}

\small{
\textbf{Buggar}
\label{52a576ce956a42fabf1b77a87518bd19}
 är söderslang för ben, som i det som fötterna sitter fast i.

}

\small{
\textbf{Bukfylla}
\label{f904f4abb175812d2f7938d05cd94459}
 är mat som ger valuta för pengarna utan särskilt mycket glädje. Bukfylla är gärna brödstark \textsc{(se bröstarkt s.~\pageref{9d9073d6a3c73a175f38e959d256cbff})} och inköpt i stora kvantiteter.


 HEAD2:  Se även

 Bärsfylla \textsc{(s.~\pageref{9380b60f9ee744b9acf978fe6f1a9545})}

}

\small{
\textbf{Bull}
\label{c4ceb152db108935c71875ae7eaeaaec}
 \quotetext{Fan va bull!} kan man höra Malmöiter utbrista. Det betyder att något är tråkigt och/eller dåligt. Hos andra folkslag är \quotetext{bull} en förkortning av engelskans \quotetext{bullshit}, som betyder ungefär 'snickelidingsnack', men den betydelsen är naturligtvis högst perifer.

}

\small{
\textbf{Burre}
\label{6e54c504971bbe1f8d46e006550af1ca}
 är en karaktär i tidningen Bamse av Rune Andreasson. Burre är en humanoid hund med randig tröja och lila byxor med dragsko \textsc{(s.~\pageref{0d3beb9223700e39e09040e9bbd3644b})}. Han är lite av en bråkstake och strulputte \textsc{(s.~\pageref{21651c95306d1b1e281443f8620910da})}. Han är lätt missanpassad och kolerisk och har fattiga föräldrar som antyds ha alkoholproblem. Samtidigt som Burre har en aggressiv framtoning och ibland uppvisar ett oacceptabelt beteende kan man ana en komplex djup-psykologi hos honom. Det närmsta en vän han har är Nalle-Maja, med vilken han har ett slags motsägelsefullt och distansierat förhållande. Han har också uppvisat ett genuint medmänskligt beteende vid flera tillfällen. Denna komplicerade och ambivalenta natur är kanske det som skapar Burres magnetiska kraft och han fortsätter att fascinera nya generationer av läsare. Bland de som tagit Burre till sitt hjärta återfinns nissepedias \textsc{(se nissepedia s.~\pageref{62400dadecd90cb5cd39062abe5a3e4a})} egen Ronny \textsc{(se Användare: Ronny s.~\pageref{c7fc87f27db026e1c60a6ac2cb1fd820})}.

}

\small{
\textbf{Buss}
\label{e57167c19ed4b7c62a6527f85687cfab}
 En boll av snus, älskad bland bandyspelare.
 HEAD2: Andra användningar
 Ett slags korvformad bil för kommunal transport av bandyspelare.

}

\small{
\textbf{Buss 85}
\label{c7f78d6a64e6921e84be4513166cdade}
 var den sista av länstrafikens bussar \textsc{(se buss s.~\pageref{e57167c19ed4b7c62a6527f85687cfab})}, och all annan allmän kommunikation, som trafikerade Kärrgruvan. Turen utgick från Lilla Heden, svängde in på Linnévägen och sedan vidare mot Norberg, Fagersta och resten av världen. Numer finns ingen tidtabell ut ur Kärrgruvan. Kommunikationen är bruten och allt återgår så sakteliga till vad det var för 200 år sedan.

}

\small{
\textbf{Busskruven}
\label{8b9bd2ad3ddb8115f57d29ab3150bd7f}
 är ett sätt för bussresenärer \textsc{(se buss s.~\pageref{e57167c19ed4b7c62a6527f85687cfab})} att signalera till övriga passagerare att hon/han ska stiga av vid nästa hållplats. Den går ut på att man plockar ihop sina tillhörigheter, sätter eventuell väska i knät, sträcker ut kroppen och tar ett stort andetag, tittar sig omkring samt antar ett lätt nervöst uttryck i ansiktet. Den genomsnittliga busskruvningen sker ungefär 40 sekunder före avstigning, eller utifall resenären sitter inklämd mot fönstret ca 2-3 minuter innan. Busskruven står för ungefär 85\% av all icke-verbal kommunikation i Sverige \textsc{(s.~\pageref{b1999637949ed135b2ca03f3a38460cc})}.

 Category:Trafik \textsc{(s.~\pageref{8a2f75cb2fdbbd1b67833430f8bc0f33})}
 Category:Psykologi och beteenden \textsc{(s.~\pageref{a15cb4e60394f5be56df613817a9efb2})}

}

\small{
\textbf{Butiker som bara säljer skoter-kepsar}
\label{1104d57d523c5abf0a8273fff6b5fdd7}
 är speciella affärer som enbart fokuserar på att saluföra en vara, nämligen skoter-kepsar. Butikerna ligger vanligtvis lite avsides till och har hemmasnickrade fasadskyltar, eller bara några handtextade plakat fästa på lastpallar. Inuti en vanlig skoter-kepsbutik ser det lite solkigt ut; \textsc{(se semikolon s.~\pageref{a6e5810f9ad5798914f30165eba44dcb})} kanske ett uppstoppat älghuvud som har sett bättre dagar, kanske en oljig \textsc{(se olja s.~\pageref{59505758215c74f673dd94e519ad459c})} overall som ligger slängd på golvet, kanske en kaffekopp som senast blev diskad när Erlander var statsminister, alldeles säkert flera överviktiga medelålders män i flanellskjortor som sen ska gå hem för att slå sina fruar. Ibland innehåller affärens utbud flera kepsar med olika företagslogotyper \textsc{(s.~\pageref{6a414633590fd4cd6d6ac64798d14c14})}, och då har man hittat en riktig guldgruva. Allra helst vill man ha en keps med \textit{Lynx} gamla logotyp som är ett riktigt bra exempel på hur man inte ritar ett snyggt lodjurshuvud \textsc{(se hur man ritar ett snyggt lodjurshuvud s.~\pageref{85f12831da9d7403326be028c34be8a9})}.

}

\small{
\textbf{Bylingen}
\label{b54c92e1b1671e982dc24eefae2edce1}
 Byling är ett ord för polis och kommer från knoparmoj. Man ska passa sig för bylingen om man är en småskurk \textsc{(s.~\pageref{c25031c5d78d9ad6fae8ab8f08d5e9dd})} men om man är en storskurk är det lugnt.

}

\small{
\textbf{Byssare}
\label{99317503481e8bdd90e670c6c43f6fdf}
 Nån som inte är plassare \textsc{(s.~\pageref{12950d5d65bc221b46c02ba5d3a89bcf})}. Personen kan komma från förslagsvis Laisvall, Jutis \textsc{(s.~\pageref{e22cbc6c7bf2a278ba4374732fc1a6ae})} eller Galtisjaur.

}

\small{
\textbf{Bältdjur}
\label{e57c0e34724e888178ffeff956101271}
 (\textit{Dasypodidae}), eller bältor som de också kallas, är ett däggdjur som har ett slags hårt skal, vilket gör det meningslöst att försöka skjuta det. Kulorna studsar bara iväg. Bältdjuret tillhör familjen trögdjur, vilket märks på dess anmärkningsvärt höga viskositet \textsc{(s.~\pageref{17328a3aa2e9e596e033ccebf7995cc1})}. Det återfinns på båda de amerikanska kontinenterna, från Argentina i söder till the United States of America \textsc{(s.~\pageref{ade6b3bd5e720abb20ed8a9a4c6b9ae8})} i norr. Det äter små ryggradslösa djur så som insekter med sin långa sträva tunga och de små, för ändamålet specialiserade tänderna. Bältdjuret är det djur som sprider bältros varän det går så om man ser ett bältdjur är det klokt att vända på klacken och lägga benen på ryggen. Det finns tack och lov inga bältdjur i Holmsunds tropikhus \textsc{(s.~\pageref{5b087d935637ad4d1823cf48036e9be6})}, vare sig levande eller uppstoppade. Det hade isåfall stått på deras hemsida.

}

\small{
\textbf{Bärsele}
\label{0e7bfe60f3b24c02416a4aa67a7aa28f}
 En bärsele är inte, som man lätt kan luras tro av namnet, en sele flätad av lingon- och blåbärsris. Istället är den ofta gjord av läder eller ett kraftigt tyg. Selens funktion är att rädda bergsklättrare som trampat snett får att rasa i backen och därmed slå ihjäl sig. Selen fästs i ett rep som förankrats i berget med hjälp av en ishacka \textsc{(s.~\pageref{542fa66b98a928c7d702a666c97ce418})} eller annat vasst föremål. När bestigaren väl trampar i klaveret gör annordningen att denna bara ramlar någon meter. Det låter kanske lite mesigt men fråga Göran \textsc{(s.~\pageref{798906d6f87c98cb6c72c306560e30f4})} Kropp.

}

\small{
\textbf{Bärsfylla}
\label{9380b60f9ee744b9acf978fe6f1a9545}
 En riktig bärsfylla går till på följande sätt: Efter att ha kommit hem från systembolaget på fredag \textsc{(s.~\pageref{80d41f1e0b14eacb229eea9618632e88})} eftermiddag (om du pluggar gick du hem tidigt för att du kan, om du jobbar på fabrik låter du din vän stämpla ut dig när hen går hem) knäpper du en öl. Den första är med fördel en mindre, lätt variant, typ en Andersson \textsc{(se Orgasmatron Andersson s.~\pageref{992f857a2415202c7eb4b9f973ea11a0})}. Du kollar på någon rolig serie och fnissar, alternativt sitter på verandan och finljuger \textsc{(se finljuga s.~\pageref{4eee5e7eab6f049c4084d3a5161016f9})} till tonerna av CCR (eller CCM \textsc{(s.~\pageref{b42f1990c0cee8758b64584877d69b93})} om du är kristen). Middagsdags nalkas och du knäcker nummer två; krokodilen, Benny Bus \textsc{(s.~\pageref{a8289efd495ef49dbe0225de89f7f019})} favoritöl. Krokodilen avnjuts samtidigt som du fixar en flottig paj \textsc{(s.~\pageref{0b438dd454bc6a17de239ebf0a46b91b})} och lyssnar på Deep Purple på helgvolym \textsc{(s.~\pageref{3539fdeb41a5b216f614b6ced9ff5cff})} i köket. Orgelsolot i speed king har aldrig låtit bättre. Efter att ha svullat din paj och druckit minst en illeröl (5.2 \textsc{(se tvåa s.~\pageref{84fcc0494ecf9f5af79fcd9bed184a9a})} i folkmun) kommer ditt kompisgäng över. Ni dricker, röker cigg och pratar, högljutt. Med fördel spelas Anti-Cimex - Scandinavian Jawbreaker på maxvolym. Efter att ha suttit och låtit alkoholen stiga er till huvudet \textsc{(se huvud s.~\pageref{e906cd95a540df9b16d0460fb4cf0adc})} stressar ni ut på lokal.
 På lokal dansar ni till skämmig musik, köper minst två stora stark, skriker högljutt och kanske ber någon att dra åt helvete. I den här fasen når bärsfyllan sin peak. Någon som är bärsfull delar under klimax många symptom med en person som är spritfull \textsc{(se spritfylla s.~\pageref{0668c687b51995118ec27cbf25061118})}, men den bärsfulle beter sig någorlunda rationellt, vilket den spritfulle sällan gör.
 När du har tagit taxi hem med dina polare (efter ett snabbt och oerhört kostsamt stopp för folköl \textsc{(se Folle s.~\pageref{fe938fafef93bca3ba46995c6d409807})} på Statoil Öbacka), följer en extremt lam, men seglivad efterfest. Du däckar klockan sju och dagen efter har din avföring karaktär av jäst, malt och humle.
 {{Yrsel}}

}

\small{
\textbf{Bög}
\label{a1bd23bf21b5add8ef4fceba9c763237}
 är man enligt vissa glesbygdsprofiler om man har mindre än tre skotrar \textsc{(se skoter s.~\pageref{b1120baa83f380cd42a805a4e823cb1b})} på går'n.




 HEAD2:  Källor:

 [http://youtu.be/ruZQhRzV05o?t=3m55s Vittne]

}

\small{
\textbf{Bögvälsignelse}
\label{56f89fc16c0100929af5ef3e6fb97c58}
 Bögvälsingelsen lyder som följande:


 \textit{\quotetext{Det är okej att du är gay}}

 Den kommer ofta från kukenkillar \textsc{(s.~\pageref{3cf0284428a6f396e261986d14927a1b})} vilka själva är noga med att markera att dom absolut inte är såna, bara så att alla vet. Syftet med välsignelsen är att den som utger den tillåter objektet att göra som denne vill, vilket såklart kan te sig något konstigt.

}

\small{
\textbf{Bönder}
\label{30a6fc00c9102680b8196b1b79935ec4}
 kläcks ur de stora vita ägg \textsc{(s.~\pageref{128a5feb8e12d0aa622e0298a8332980})} som brukar ligga på olika svenska fält under slutet av sommaren. Alla hanar kläcks iklädda jeansoveraller \textsc{(se kanadensisk frack s.~\pageref{365eda0bac157feb7ee22ceb7f07e17d})} men alfahannarna får med sig en EPA som de kan köra direkt.

 Trots intensivt letande har honor aldrig påträffats.

}

\small{
\textbf{Cacao creme}
\label{ab4762d6b262c568ca4511942bd68bf1}
 är den finaste drycken man kan få ut ur en kaffeautomat av märket Wittenborg. Kaffet i en sådan automat är alltid äckligt, tevattnet ljummet och wiener mélangen skrattretande. Cacao cremen är vad som räddar tiefikat, lunchen och trefikat samt alla eventuella bensträckare \textsc{(s.~\pageref{b7f58d514c506dc96a77284fcc03039a})} däremellan.
 Cacao creme består av mjölk, chokladpulver och vatten, tillsatt i den ordningen. Om man är ett riktigt proffs drar man undan sitt dryckeskärl från automaten under de sista två sekunderna av upphällning då det under de sekunderna är vattnet som tillsätts. Vattnet undviker man för att ge en fylligare, mustigare upplevelse. Detta knep tas ofta till av de riktigt sura gubbarna på jobbet som egentligen vill dricka kaffe men blir för risiga i kistan av det och därför skiter i allt och bara vill hinka i sig chokladsörja hela dagarna.

}

\small{
\textbf{Cadet Laverne Hooks}
\label{06459846fd12518dbca0d2bc94cb4011}
 är en karaktär i de populära polisskolan-filmerna. Cdt. Leverne Hooks utmärker sig bland de andra kadetterna genom sin mycket ljusa och pipiga röst. Hon är också svart, vilket av någon anledning är roligt i the United States of America \textsc{(s.~\pageref{ade6b3bd5e720abb20ed8a9a4c6b9ae8})}.

}

\small{
\textbf{Café Lindell}
\label{787907d0d010e020b84fd2bbebefdaf2}
 är ett fik på universitet i Umeå. Här kan den hungrige och/eller törstige studenten köpa ekologiskt och rättvisemärkt till skyhöga priser. De har också en lunch som du betalar för efter vikt. Konstigt nog blir det nio gånger av tio 68kr, oavsätt vilka rätter och hur mycket man tar.

 Alla coola som fikar här sitter i fåtöljerna som vätter mot fontänsidan av Lindellhallen. De som är lite coola sitter vid de postmoderna borden på motsatt sida. De som helt saknar kulturellt kapital sitter vi skolbespisningsborden på resterande sidor.

 De har också alla veganer \textsc{(s.~\pageref{2a12d5d6ae91d2f4f7d9af3cef58e75c})} på universitetets basföda: Falafelmackan. Detta är en helt vanlig macka med kall falafel \textsc{(s.~\pageref{b2d6ec45472467c836f253bd170182c7})} intryckt i en baguette. Till detta en smaklös tomatskiva, en gurksnärt, sladdrig sallad och en fadd pepperoni. Ovanpå allt detta den ovanliga, i förhållande till falafeln, såsen Söt-sur. Man känner sig lite mätt men själsligt tom efter denna måltid.

 Här återfinns Universitets i särklass mäktigaste gofika: Lyx-biskvin. Detta smörkrämsfyllda bakverk är stort som ett kaffefat och säljs till det ohemult \textsc{(se ohemul s.~\pageref{91b8873590abd15ec344c2ba93d015cd})} höga priset av 36 riksdaler. Endast för storfräsare \textsc{(s.~\pageref{4db17005692cd83e3e946a1311b81ed0})}.

}

\small{
\textbf{Café Tornet}
\label{6b7041691b72072b93ad2184f2bae495}
 Här hänger alla juridik- och ekonomistudenter. Är man något sitter man i ett av båsen. Är man inte det sitter man någon annanstans. Café Tornet stoltserar med en veganrätt som verkligen får en att tänka på skolbespisningsmat \textsc{(s.~\pageref{0620e2bdb64059d3b73e2215e741d052})}.

}

\small{
\textbf{Calbuffe}
\label{159d92e35deca095577877b18b694291}
 En fantastisk utveckling av calskrove-konceptet \textsc{(se calskrove s.~\pageref{84ff54e779ee49fdad21e17c20f14453})}. Vissa certifierade pizzarestauranger ( i Skellefteå ) får sälja calzonepizzor med dragkedja, där man kan öppna Slash \textsc{(s.~\pageref{9fbbaa4cc515bc46e0c12e82a31df736})} stänga pizzan precis som man känner och fylla den med allsköns bråte från buffébordet.

}

\small{
\textbf{Calskrove}
\label{84ff54e779ee49fdad21e17c20f14453}
 Ett mellanmål i Skellefteå. Nära besläktat med svanskrove \textsc{(s.~\pageref{e543ead268a283bfdb5ea638d6cca4a2})} och transkrove \textsc{(s.~\pageref{1188281a09fb681b922e45663e5ffc4b})} såväl som uvsvane \textsc{(s.~\pageref{c5081b14cdeb1ff42b655213e80c9d51})} och serveras ofta som en vardaglig variant av dessa mer högtidliga rätter.

}

\small{
\textbf{Calzona}
\label{42bf60ead842afe1df27d41324e41a02}
 (Verb, infinitiv) är att klippa upp en dubbelvikt pizza, en s.k. Calzone, i ena änden och sedan trä den över en annan persons huvud, likt en ostfylld dumstrut. Detta är en populär bestraffning mot personer som inte är från Västmanland och flörtar med lokalbefolkningen på Green Man \textsc{(s.~\pageref{39c63ddb96a31b9610cd976b896ad4f0})} i Fagersta.

}

\small{
\textbf{Canned Heat}
\label{3bb6b8968422acde6198e55092b48235}
 är ett amerikanskt bluesband som skänkt fans av tungt \textsc{(s.~\pageref{2e9d5a15d21acbc3bd0704ea0cbdd67e})} gung oerhört mycket glädje genom åren. De gillade spånken, gräs och att åka på semester och skrev mest låtar som behandlar dessa ämnen. Kändast av deras låtar är \quotetext{Going up the country} som visar upp både så oerhört ljus sång att det har debatterats flitigt på bluesforum världen över huruvida sångaren Alan \quotetext{Blind Owl \textsc{(se Uv s.~\pageref{45210da832f9626829457a65e9e7c4d0})}} Wilson var kastratsångare eller inte, samt flöjt. Flöjt är som alla vet ett instrument med magiska egenskaper och just denna flöjt förtrollar lyssnaren till att vissla med i den medryckande melodin. Om ni tyckte att Wilsons smeknamn \quotetext{Blind Owl} var häftigt så kan vi på Nissepedia som andra internetbaserade uppslagsverk berätta att en klar majoritet av Canned Heats medlemmar, och det vill vi lova är en diger skara, hade precis lika balla smeknamn. Här följer ett axplock:

 \begin{itemize}
 \item Harvey \quotetext{The Snake} Mandel
 \item Mike \quotetext{Hollywood Fats} Mann
 \item Bob \quotetext{The Bear} Hite
 \item Larry \quotetext{The Mole} Taylor
 \item Stanley \quotetext{The Baron} Behrens
 \end{itemize}

 Och så vidare, och så vidare...
 Tänk er en reunion i himlen med bandets alla gamla medlemmar och en oerhört amerikansk presentatör rabblar de här namnen efter varandra, vilken fest!
 HEAD2: Namnet \quotetext{Canned Heat}
 Nu har ni läst ganska mycket om Canned Heat utan att få reda på varför de har detta märkliga namn, och det kan vi, fortfarande som andra internetbaserade uppslagsverk informera allmänheten om. Canned Heat är för den inte-så-blues-men-väl-friluftslivsintresserade amerikanen synonymt ett spritkök på konserv som man river upp och tänder på för att få sig lite värme en kylig dag. Fryser man ända in i själen, så som mången bluesmusiker gör, så kan man dricka rödspriten som finns inuti burken. Här bör dock poängteras att det är fan så mycket trevligare att lägga på Canned Heats LP \quotetext{Livin' the blues} från 1968, luta sig tillbaka i finfåtöljen och bara njuuuta. Som om inte glädjen kunde bli större så är det en dubbel.

}

\small{
\textbf{Cape}
\label{8b04f4091aa625f56b3f7da315a1e231}
 n är ett klädesplagg som är helt funktionslöst, det mest bara hänger över ryggen och fladdrar när det blåser. Yrkeskårer som är tillåtna att ha cape:

 \begin{itemize}
 \item Trollkarlar
 \item Advokater
 \item Vampyrer
 \item Uppvisningsmotorcykelförare
 \end{itemize}

}

\small{
\textbf{Carl von Linné}
\label{5e8380bf6b7ce99678e6752b6d9e709e}
 Blom- och djurkillen, (före adlandet 1757 Carl Linnæus, Carolus Linnæus), föddes i Råshult, Småland.

 HEAD2: Tidig barndom
 Som barn beskrevs Carl som oerhört vetgirig. Tidigt började han dissekera smådjur med sin första pennkniv, som han kallade \textit{Gram}, efter det svärd Sigurd använde för att dräpa draken Fafner i Völsungasagan. I historieskrivning om Linné har detta väckt debatt. Var Linné en psykopat, i klass med John Wayne Gacy? En akademiker som tog Linné i försvar, var Prof. Etienne \textsc{(s.~\pageref{56957a267e57df32753cf7f3b8a603d8})}. Särskilt i hans essäsamling \quotetext{\textit{Morsan! Farsan! Digga kniven! - en stridsskrift för barns rätt till att plåga djur utan att bli stigmatiserade av samhället}}. I övrigt var Linné ett väldigt glatt barn, som ofta sprang runt på ängarna kring Råshult och lekte med tygdrakar och frisbees av bark.

 HEAD2: Skolgång
 Linnés fallenhet för anatomi blev tydlig i hans skolgång, där han fick högsta betyg i alla naturämnen. Även i svenska och historia excellerade den unge Linné. Hans enda akilleshäl var gymnastiska övningar, något han vid elva års ålder skrev en dikt om:

 ''\quotetext{Hjärnan vara min starkaste muskel
 \textit{Min lekamen dock, likna mest ett ruckel}
 \textit{Spik och plank, huller om buller}
 \textit{Ett aber av kött, som jag var dag fördömer}}

 Den självhatande lille gossen vantrivdes också i Småland, där han kände att det klassiska räksalladsfrossande småländska råskinnet hyllades, samtidigt som hans mer introverta nördstuk inte till fullo uppskattades av omgivningen. Således bestämde han sig för att direkt efter att han  gått ut grundskolan flytta till en plats som genom historien alltid varit befolkat av introverta nördar med antisociala psykopattendenser - Uppsala \textsc{(s.~\pageref{1db4e388df1df7057b8f3d984c65ee88})}.

 HEAD2: Uppsala och systematik
 I Uppsala \textsc{(s.~\pageref{1db4e388df1df7057b8f3d984c65ee88})} stormtrivdes Linné. Han hade ett helt gäng kompisar som ofta spanade på brudar, handlade damasker på stan och drack ganska mycket bärs och absint blandat i soppskål. En kväll, efter en redig bärsfylla \textsc{(s.~\pageref{9380b60f9ee744b9acf978fe6f1a9545})}, stod Linné och spydde på biologiska fakulteten. Mitt i sin kaskad kom en tanke till Linné, om hur alla blommor är lika, men ändå lite olika. Utifrån den tanken kom hans tvångsmässiga behov att under resten av sitt liv systematisera alla blommor i sin omgivning.

 HEAD2: Rocken, torken \& resorna
 Genom att börja inordna allt i system blev Linné en rockstar i paritet med Paul Stanley, eftersom ingen tänkt på att allt kunde höra samman innan (precis som ingen kommit att tänka på riffet till The Deuce innan Paul Stanley). Han blev riktigt fet och grisig, ständigt bärandes på ett vattenskinn fyllt med baconfett.

 Hans mentor, biskopen och lurendrejaren Olof Rudbeckius d.y. såg till att skriva in Linné på dåtidens svar på Betty Ford-kliniken, hotell lappland i Gällivare. Väl där kom Linné till klarhet om en massa grejer. Han insåg att inte bara blommor kunde systematiseras, utan även samerna som masserade hans frontallob tills han somnade in i sin renskinnsfäll.

 Efter det tyckte Linné att det var rätt smutt \textsc{(s.~\pageref{d9114ffee4f2dcee302ae2b19ce5eea9})} att checka in sig själv på olika behandlingshem, främst i Sverige, men han gjorde också en avstickare till Nederländerna för att spana in några nya spännande plantor. Efter varje incheckning skrev Linné en bok, för att hålla sin ekonomi flytande. Han spelade mycket på hästar, men la ner efter att Brandklipparen \textsc{(s.~\pageref{e8aaa0dc22fb08e055f0f48b2f25e35d})} lubbat in i en mur och dött. Efter Brandklipparens frånfälle ska han ha sagt att hästsporten inte var sig lik och hans humör mulnade märkbart.

 HEAD2: Missbruk och död
 Trots all sin rehabilitering fortsatte Linné att missbruka flott och öl i stora doser. Han bar inte längre eleganta damasker, utan grå mjukisbyxor och filttofflor. Han lekte inte längre med tygdrakar, utan stegrande mängder griskött och alkohol. Vid sin död hade han blivit adlad, men också ett jävla vrak.

 Hans begravningståg följdes av tusentals människor, längst med Uppsalas gator. Idag vilar hans groteskt feta lik i Uppsala domkyrka.

}

\small{
\textbf{Carlsberg}
\label{b2f2c838b291ba6d244333a791a9b79f}
 är ett ölmärke från Danmark \textsc{(s.~\pageref{5331d7fd27772396f412a5b6d19bad44})}.

 HEAD2: Arbetarkamp

 Tidigare fick arbetare tre öl om dagen men då bestämde sig kapitalistpamparna för att djävlas med dom på golvet och sa att en öl, på lunchen, får fan räcka.
 Men det gjorde det inte!
 Arbetarna gick ut i strejk. Hur det gick för dom vet jag faktiskt inte då jag endast hittade nyheter om att dom gick ut i strejk för sina \quotetext{hävdvunna rättigheter}.

 HEAD2: Fasciststat

 Otroligt nog är det så att i fasciststaten Danmark \textsc{(s.~\pageref{5331d7fd27772396f412a5b6d19bad44})} tillåter bara sju procent av alla företag att deras anställda dricker alkohol på arbetstid!(sic!)

 -Det är bara sju procent av alla danska företag som tillåter sina anställda att dricka alkohol på arbetstid, säger Carlsbergs kommunikationsdirektör Jens Bekke.

}

\small{
\textbf{Carlshöjdare}
\label{c40367b35819e1f104e5495d049ac199}
 En Carlshöjdare är en extra prominent person som bor på stadsdelen Carlshem i Umeå.  Carlshöjdaren kan ofta ses vagga fram och tillbaka längs med Kieselstråket, där hen titt som tätt stannar för att lyssna till förbipasserande gubbar och gummors klagomål. Om det står i Carlshöjdarens makt hjälper hen gärna till, men aldrig utan att kräva en gentjänst. De flesta Carlshöjdarna är barn till kirurger, psykologer, radiopratare, datatekniker och journalister. Därmed kan livet för den som inte böjer sig efter Carlshöjdarens vilja te sig riktigt surt. Små gliringar utdelade i Filosofiska rummet i P1, en laptop-reparation som drar ut i evigheter, en kontaktannons insatt i Blänkaren som kungör att man söker en våldsamt flatulent partner, att man tvingas sitta i väntrummet hos psykologen i 40 min innan man får komma in, med bara postmodern poesi att läsa, eller att man efter ett rutinbesök på vårdcentral vaknar upp tre dagar senare i busskuren på Marmorvägen, med ett grishjärta inopererat i sin bröstkorg.

 =Förmåner=
 Det hedersdrivna kotteriet på Carlshem åtnjuter en serie privilegier som den vanlige Carlshemsbon bara kan drömma om.
 \begin{itemize}
 \item Carlshöjdaren har obegränsad tillgång till den exklusiva replokalen intill Carlskyrkan.
 \item Carlshöjdaren har 10 \% rabatt på pizzeria Atlanta och får alltid en gratis Trocadero på köpet.
 \item Carlshöjdaren får klättra upp på Carlshemsskolans tak och ingen törs säga åt den att komma ner omedelbart.
 \item Carlshöjdaren har förtur på alla balla grejer som kommer in på Gimonäs återvinningsstation, typ NES, sprayburkar och tuffa skateboards.
 \end{itemize}

}

\small{
\textbf{Cashewnöt}
\label{de0870a68a39e796868dcb3189b98e04}


}

\small{
\textbf{CCM}
\label{b42f1990c0cee8758b64584877d69b93}
 är en förkortning och står för Contemporary Christian Music, alltså Samtida Kristen Musik. CCM avgränsar mot traditionell kristen musik så som psalmer och körstycken. Till skillnad från de senare används i CCM vanliga rock/pop-instrument så som elgitarr \textsc{(s.~\pageref{a08bf8420208934bc59c7ed7385d4308})}, elbas och trummor. Inom CCM finns så många underkategorier att den som vill veta mer om denna spännande musikstil måste utforska den själv, men några olika stilar är kristen rap \textsc{(se Hip-hop s.~\pageref{66c22415908267e727d3fa4a63c16672})}, dansmusik, folkrock, kristen ful-sludge, punkrock, och till och med \quotetext{heavy \textsc{(s.~\pageref{7cfe64ea44dc3bbeb63b29ff3039a481})} metal.}

}

\small{
\textbf{Centerpartiet}
\label{e331dec360e356adc1e2db36fe9a9f3f}
 var först ett parti för vanliga bönder \textsc{(s.~\pageref{30a6fc00c9102680b8196b1b79935ec4})}, sen för nyliberala kulaker \textsc{(s.~\pageref{c17322f1f8b87ec8fc35538dbe1e9668})}.

 De är också fega för att de kör på förkortningen (c) och inte (cp \textsc{(se CP s.~\pageref{9c95319bf274672d6eae7eb97c3dfda5})} som vore brukligt. [http://sverigesradio.se/api/radio/radio.aspx?type=db\&id=2955463\&codingformat=.m4a\&metafile=asx]

 HEAD2:  Trivia:
 (Cp) är troligen det parti med störst andel harmynta anhängare.

 HEAD3: Slogans:
 \quotetext{För egocentriker}
 \quotetext{Förslava bävrarna \textsc{(se ivriga små bävrar s.~\pageref{6d10ab1ba7bd378ba7cc1629ddf2bbde})}}
 \quotetext{Det är vi som suger ut Sverige \textsc{(s.~\pageref{b1999637949ed135b2ca03f3a38460cc})}}
 \quotetext{Blod på min jord ger gåslever på mitt bord}
 \quotetext{\textit{Libro e moschetto — fascista perfetto}}

}

\small{
\textbf{Centralsydslaviska}
\label{6b4bcc89dc3f7b6fc554e5717d1f990a}
 är en inte särskilt populär beteckning på den språkfamilj som tidigare kallades serbokroatiska. Idag är inte serbokroatiska heller särskilt populär som beteckning, så då hajar du hur impopulär centralsydslaviskan är. Enligt ett konkurrerande internetuppslagsverk finns det bara två artiklar på hela internet som tar upp centralsydslaviska, men det är fel för det finns faktiskt tre.

}

\small{
\textbf{Champis}
\label{ce7011d454a0f4377acffb4751e18a88}
 Fattigmans-Pommac.

}

\small{
\textbf{Chapeau de paysan}
\label{27aa75146d9ab723d1423168a2539d5d}
 (bondhatt) är ett brunt läderartat plagg skapat för att skydda hårt arbetande bönder från solen. Bärs med fördel som en hatt.
 Under medeltiden användes plagget för att förnedra uppnosiga studenter på universitet. Studenten tvingades av en magister att ta på sig bondmössan för att påminna denne om dess rang i klassrummet. Detta förekom vid alla universitet under medeltiden.

 Thomas av Aquino var en mästare på att tvinga på studenter mössan för att få dem att lugna sig. Thomas av Aquino uppfann seden att tvinga studenten att härma olika djur på en bondgård när den väl var iförd mössan. Han uppfann även slangtermen att \quotetext{chappa} någon, som en förkortning av det betydligt krångligare \quotetext{chapeau de paysana} någon.
 Idag används inte längre chappning som en bestraffningsmetod, då det på dagens universitet är moderiktigt att se ut som en lurk.



 [http://1.bp.blogspot.com/_98hGiGFAYec/SaRDhm4wZhI/AAAAAAAACJA/iGH9MnParcA/s320/Dunderklumpen.gif Bild föreställandes individ iklädd en Chapeau de paysan]

}

\small{
\textbf{Charles Manson}
\label{068614c686e26d839098938aa4b847d2}
 Challe är den sjätte medlemmen av surf-rockbandet Beach Boys.

}

\small{
\textbf{Charles manson}
\label{068614c686e26d839098938aa4b847d2}
 , född som Charles Milles Maddox den 12 november 1934 i Cincinnati, Ohio, är en amerikansk mångsysslare. På hans CV återfinns bland annat mördare, karaktär i South Park, musiker, låtskrivare och sektledare.

}

\small{
\textbf{Charlotte}
\label{a6008231fa64ff879ecb286a657b6a99}
 är ett kvinnonamn från Frankrike och betyder kaskelott.

}

\small{
\textbf{Cheeseburga}
\label{831f648816a114e1ab52b3760f328e5a}
 Att cheeseburga någon är att försiktigt dela på en cheeseburgare så att man får två delar, en i vardera handen, och sedan klappa ihop dessa på var sida om någons huvud. Detta är en effektiv anfallsmetod som står till hands för den gatuköksbesökare som finner sig indragen i en dispyt och inte ser någon annan utväg än att gripa till ytterligheter.

}

\small{
\textbf{Chef}
\label{cbb4581ba3ada1ddef9b431eef2660ce}
 En chef är någon som parasiterar på andra. Chefer finns nästan över allt så det är svår att skydda sig, men en bra början är att studera Marx och CH Hermansson. En chef kan aldrig bli ens vän.

}

\small{
\textbf{Chipslåda}
\label{026a379e5dd81cf695341d3417515c2c}
 är en maträtt bestående av chips, ost och majonäs. Man går tillväga enligt följande:

 1. Strö chips i ett lager i en form.
 2. Häll på majonäs.
 3. Repetera tills du fyllt formen.
 4. Riv över så mycket ost att inget annat syns.
 5. Skjuts in i ugnen!
 6. Ta en mobiltelefon, slå in 11.
 7. Ät.

}

\small{
\textbf{Choklad}
\label{6d4129e58eba1497a3f2fc9128cc3b23}
 Belgiens \textsc{(se Belgien s.~\pageref{f79ffe9e826a19f9f6a446c90e21c4e3})} nästa största export. För populära användningsområden se artikeln om barnamord \textsc{(s.~\pageref{abb9d8f6fb223dbc9b3db79dea136d08})}.

}

\small{
\textbf{Christer Sandelin}
\label{84676026ecbbc0b0417cd2e74125c766}
 är en svensk artist som seglade upp på pop-himlen med banden Freestyle och Style under 80-talet. Han medverkade också med Freestyles andra medlemmar i filmen \textit{G som i gemenskap} (1983). Efter detta har Sandelin arbetat som soloartist och släppte så sent som 2004 albumet \textit{I stereo} tillsammans med artistkollegan Tommy Ekman. Ordvitsen i skivans titel, som syftar på att albumet skapats av två artister (alltså en duo), gjorde dock att Sandelin efter skivsläppet såg sig nödd att gå under jorden för att undvika repressalier. Svenska myndigheter utfärdade redan 2005 en begäran att Sandelin eftersöks både nationellt och internationellt för att ställas till svars för det han gjort. Den som har information om var Sandelin kan befinna sig eller vilka människor han har kontakt med bör höra av sig till polis, säkerhetspolis eller kustbevakning och kan kompenseras ekonomiskt.

}

\small{
\textbf{Christian Information Service Homepage}
\label{41f6a92690af566fe26cfeb327f82eb5}
 [http://www.endtime.net/kinfos.htm] (eng. Ung. Kristen information tjänst hemsida) är ett norskt kristet initiativ som går ut på att tillhandahålla viktig information till allmänheten. Informationen är av lite olika slag, men viktigast är att upplysa om att, som man skriver, \quotetext{det pågår en omfattande sammansvärjning för att ödelägga självständiga nationer. En europeisk och global revolution är på gång. Både de ekonomiska, politiska och religiösa krafterna i världen är i färd med att samla sig. Nationella lagar och regler får ge vika för internationella lagar.} Vad är det då som dessa onda krafter vill åstadkomma, undrar den frågvise nissepediabesökaren \textsc{(se nissepedia s.~\pageref{62400dadecd90cb5cd39062abe5a3e4a})}. Jo, man vill skapa anti-krists rike på jorden. Det kan man tydligt se i Uppenbarelseboken och genom att studera den katolska kyrkan, som leds av påven (dvs. anti-krist). Att man  inte ska fira söndagen \textsc{(se söndag s.~\pageref{85b2e5c3758394a24221d1abac79191a})} utan lördagen \textsc{(se lördag s.~\pageref{8d203c09d6ebbc3a0d797e14178798a0})}, är en annan tung punkt på dagordningen. Det talas också en del om judar och Israel.
 HEAD2: Fackföreningar - Djävulens påfund?
 Man får också veta att \quotetext{Fackföreningar och -förbund binder människor samman i buntar, och vi måste frigöra oss snarast möjligt, ty de har en kontrollerande makt över folk, och dessutom säger Bibeln att vänskap med världen är fiendskap mot Gud (Jak. 4:4). De stora fackföreningarna går även in för ett europeiskt och globalt centraliserat, internationellt mönster. De pressar fram sina påstådda rättigheter _ ofta på ett själviskt och ofint sätt - och binder medlemmarna till att följa deras i lag bestämda förordningar.}

}

\small{
\textbf{Christianiacykel}
\label{671a1992db86e328dc9c068647d57d6b}
 danskar på plundringståg med sin christianiacykel.]]
 En christianiacykel är en damcykel \textsc{(s.~\pageref{2f0f41314b4e4edb773a7ae87addc913})} som kraschats in i en kundvagn med sådan kraft att de två tingen absorberat varandra och blivit en ny helhet. Eftersom det är en dansk \textsc{(se danmark s.~\pageref{5331d7fd27772396f412a5b6d19bad44})} uppfinning saknar fordonet växlar (det var för svårt att konstruera) och hjulen är av helgjutet gummi (det var för svårt att göra ett hål inne i slangen utan att det blev punktering på utsidan). Bromsen består av en träpinne som hänger i ett snöre på styret och sticks in mellan ekrarna. Christianiacyklar används främst till att göra plundringståg och frakta alkisschäfrar \textsc{(se Alkisschäfer s.~\pageref{347febbc28041eae88556d2e7ced587b})}.

}

\small{
\textbf{Christina Husmark-Persson}
\label{26162477ef162a13e8321aeecdc8259c}
 Hemsk människa. Moderat \textsc{(s.~\pageref{c4564b188cb670841733a3ff923c2fb0})}. Fick sparken, men inget blev bättre.

}

\small{
\textbf{Christina Husmark-Persson.}
\label{21573103bb40870a28a8f162fbc31b9d}
 \textsc{(s.~\pageref{26162477ef162a13e8321aeecdc8259c})}

}

\small{
\textbf{Ciderhusreglerna}
\label{3b5c896ef75b6c0b39a600f671944bfe}
 ser ut som följande. Vissa lokala varianter kan förekomma:

 \begin{itemize}
 \item Inget spritvev \textsc{(s.~\pageref{982050892345f509daf436946af24dda})} inomhus
 \item Mata inte djuren
 \item Hälsa inte på pensionärer
 \item Skjut, gräv, tig
 \item Max tre (3) per person
 \item Det är äckligt med spenat
 \item Duscha med badbyxorna på
 \item Byt inte låt mitt i
 \item Ta med din egen bintje \textsc{(s.~\pageref{f21f4f64cb0df1775b5c2a7dc0d83c6c})}
 \item Ställ ut soporna eller elda upp dom
 \end{itemize}

}

\small{
\textbf{Cigg}
\label{2bcc66e1261fa5199a4f4decf2720ef5}
 är slang för cigaretter. Cigaretter är malen och torkad tobak uppblandat med allsköns skräp \textsc{(s.~\pageref{75f1a5320951ea0dd9aa3c0eaba2c2c7})} som placerats i en pappershylsa och ofta med ett gult filter längst ned. Änden med filtret placeras i munnen \textsc{(se mun s.~\pageref{6585f290ce92c3de5ff339920330e26f})} och den andra tänds på med hjälp av en tändare eller tändstickor. Sedan drar rökaren in röken i munnen (om denne är 14 år) eller i lungorna. Det som då händer är att personen blir cool på ett genuint sätt. Det kommer att vara coolt att röka så länge som det är farligt.

 Vi på Nissepedia \textsc{(s.~\pageref{62400dadecd90cb5cd39062abe5a3e4a})} har väl ändå någon form av samhällsansvar och bör påpeka att det är hälsosamt och ekonomiskt fördelaktigt att röka, och när du väl känt smaken av Virginia blandat med Burley så vet du att det inte spelar någon roll.
 HEAD3: Andra benämningar
 \begin{itemize}
 \item Tagg
 \item Puff
 \item Cancerpinne
 \item Lungtorpeder utan stötdämpare
 \item Rök
 \item Smoke
 \item Ciggaluring
 \item Ciggaretts
 \item Giftpinne
 \end{itemize}
 HEAD3: Tuffa Cigg
 \begin{itemize}
 \item John Silver
 \item Glenn
 \item Commerce
 \item Chesterfield utan filter
 \end{itemize}
 HEAD3: Töntiga Cigg
 \begin{itemize}
 \item Blend
 \item Prince
 \end{itemize}
 HEAD2: Externa länkar
 Cigarettspeida[http://www.Cigarettespedia.com]

}

\small{
\textbf{Cirkusdirektör}
\label{921ae2aafb89db4bd8217c99df4eba5f}
 är ett klassiskt yrke med hög status bland entreprenörer \textsc{(se entreprenad s.~\pageref{2d3b60492ed3cebe0a3cf341bc5b20b5})}. Jobbet går ut på att skriva livstidskontrakt med arma krakar på att dessa, mot undermålig betalning, visar upp sig i märkliga situationer inför publik. De arma krakarna får bo i gamla hästtransporter och för att Cirkusmyndigheten inte ska fatta misstanke tar man ofta med några riktiga djur för att verka trovärdig. När en föreställning är slut dödas de svagaste djuren och tillreds till punkgryta \textsc{(s.~\pageref{bf12834e2173c29a65420ed558b53c6b})} som resten av ensemblen äter. För att markera sin särställning slipper cirkusdirektören bära byxor som ger upphov till cirkuspung \textsc{(s.~\pageref{c2c41b1921dcdcab22f7d32b62d2d17a})}, vilket står i kontraktet att alla andra måste ha. Danmark \textsc{(s.~\pageref{5331d7fd27772396f412a5b6d19bad44})} har hittills inte haft en enda statsminister som inte tidigare varit cirkusdirektör.

}

\small{
\textbf{Cirkuspung}
\label{c2c41b1921dcdcab22f7d32b62d2d17a}
 är ett uråldrigt fenomen som uppstår när män bär sina byxor lite för högt uppdragna så att de skär in i skrevet. Genitalierna får därmed ingen naturligt plats att vistas på och hamnar lite över allt på ett osymmetriskt sätt. I det victorianska England var cirkuspung högsta mode, troligtvis på grund av att det endast var de högre stånden som hade råd att klä sig i något annat än taskigt åtsittande jutesäckar. Namnet kommer sig av att det inom cirkuskulturen alltid funnits en förkärlek till spandex, ett åtsittande material som cirkuspungar frodas i. Vill man undvika cirkuspung bör man välja byxor sydda av Avesta jet-tex \textsc{(se Min bästa byxa s.~\pageref{d713d68db15d469d6e39abacefefb3ab})}, ett material som har \quotetext{suverän passform}.

 Se även: Kristerstånd \textsc{(s.~\pageref{eacb7b3161f8220a0858dcc056a2fc8c})}.

}

\small{
\textbf{Civilpolis}
\label{3db747cc02328b12807d93c629bbf10c}
 Att avslöja en civilpolis är mycket lätt. Leta efter:

 \begin{itemize}
 \item Öronsnäcka, mest uppseendeväckande om de går i grupp. Kanske lyssnar de alla på Scooter?
 \item Ful keps/rakat huvud/polisfrisyr
 \item Snabba skor
 \item Osympatiskt utseende
 \item Och framförallt, Jeansen som Gud glömde. Lite baggy, lite slitningar, några fräcka detaljer och en snutröv \textsc{(s.~\pageref{f9a35b35e7ef367d07be9bd1e9357f83})} som försöker rymmas
 \end{itemize}

}

\small{
\textbf{Cock sparrer - platinum blonde}
\label{898438beb9b39308aa4508451aad67f5}
 Det definitiva beviset för att även solen har fläckar.

}

\small{
\textbf{Colin Nutley}
\label{b7e4eb146052f2edb273b55e35f4f078}
 Mannen som ger nazisterna rätt i påståendet:\quotetext{\textit{invandrarna kostar så mycket pengar}} för Colin Nutley har sugit många miljoner ur svenska filminstitutet.
 I korthet går hans film ut på att hans hustru ligger med nån bonnig typ och så blir det burleska förvecklingar och lite tårar. Eftersom alla filmer följer samma dramaturgi och innehåller samma skådespelare (Nutleys fru) \textsc{(se Nutleys fru s.~\pageref{0a6824de53984f1d29c42ca39c6eb180})} är det bortkastad tid att se mer än en. Den upptagne gör allra bäst i att nöja sig med att läsa texten på baksidan av en VHS-kassett.

}

\small{
\textbf{Congo}
\label{443fd8c93d17446bad49472af0e22dc3}
 är en rafflande actionfilm från 1995 som handlar om en flock apor som tar till vapen och börjar döda människor. Bara det känns ju ganska otippat och mäktigt, men ballast är nog att de även har laserögon \textsc{(s.~\pageref{7d642f9221f16b36fa9d731166ba3416})}. Varför aporna tar till vapen och har laserögon och vad människorna gör djupt inne i Kongos oändliga djungel, framgår inte riktigt av storyn. Den som vill veta kan konsultera videofodralets baksida för några vaga ledtrådar.

 HEAD2: Publikt mottagande
 Congo fick över lag väldigt dåligt betyg av etablerade förståsigpåare \textsc{(s.~\pageref{ff91afb86ce86124b6a517f3eb37bc18})}.

 Congo var länge den mest uthyrda filmen på Vega Video i Norberg.

}

\small{
\textbf{Conny}
\label{3af31637b5328eb0e6a050e23a064681}
 är en dansk kortform av det grekiska Konstantin. Namnet betyder skäggig gubbe. Conny har namnsdag 21 maj.

}

\small{
\textbf{Coop}
\label{0b5cb0ec5f538ad96aec1269bec93c9c}
\begin{enumerate}
\item Redirect Konsumbutik \textsc{(s.~\pageref{70e4875f7c2c177596305006e46b7ca9})}
\end{enumerate}

}

\small{
\textbf{Corporate social responsibility}
\label{14eb18288303a7408a099b83b5af7f08}
 Ännu dummare än det låter. Namnet är på engelska för att förvirra de som drabbas.

 Ett exempel på Corporate social responsibility är när BP:s styrelseordförande Carl-Henrik Svahnberg säger \quotetext{I care about the small people} efter att en oljeplattform försökt sig på en Eskimåvändning \textsc{(s.~\pageref{7ca6450f5cefc61523fade427738fff3})} och mer eller mindre tömt hela jordens innandöme på olja i mexikanska golfen.

}

\small{
\textbf{Cowboyhatt}
\label{229b4cd9f3e94e4794ffba4ed0bea704}
 Ett säkert tecken på att nåt inte står rätt till under hatten.

}

\small{
\textbf{CP}
\label{9c95319bf274672d6eae7eb97c3dfda5}
 är som alla vet ett paraplybegrepp, avlett från namnet på en obeskrivligt smärtsam nerv- och muskelskada, för folk som på olika sätt beter sig och är uppseendeväckande. CP beskrivs bäst som en personlighetsegenskap, den kan inte vara medfödd utan snarare inövad frivilligt eller ofrivilligt. Man kan vara CP på olika sätt, det finns ingen egenskap som utmärker CP mer än någon annan. Den som är CP/ är ett CP är dock på något sätt objektivt dum i huvudet.

 Att utrycka att någon är CP när de betett sig som ett CP är ett mycket smidigt sätt att markera att man för stunden reagerar på personen. Användandet av termen täcker in många komplexa och abstrakta attribut och egenskaper som annars är för krångliga och tar för lång tid att tänka ut i stunden, då en person är ett CP.

 Att vara ett CP är inte samma som att ha CP, eller att ha bärs \textsc{(s.~\pageref{a74b297c15834437ac2e49095492133c})} för den delen.
 HEAD2: Konkret exempel
 \begin{itemize}
 \item Anna Kinberg Batra, moderat ledamot i finansutskottet mandatperioden 2010-2014 (och fru till David Batra).
 \end{itemize}

}

\small{
\textbf{CQD}
\label{cf4d3265d3d4e019e3ed1010958698d0}
 Ute på havet finns det som bekant ett oändligt antal faror. Stormar, grund, sjörövare, Bermudatriangeln, skörbjugg, spökskepp, myteri, sjömonster, för att bara nämna några. Med uppfinnandet av den trådlösa telegrafen kunde sjöfarare plötsligt kommunicera mellan båtar och därmed undsätta varandra vid en nödsituation. Alla var jätteglada att få hjälp och slippa bli bitna av späckhuggare \textsc{(s.~\pageref{8cb9605d553e2ecca26af024d2fcc220})} och jättebläckfiskar. Från början använde telegrafisterna det, i morsekod, något krångliga HELP för att begära assistans. Det var svårt att skicka och missuppfattades ofta, varför man istället övergick till snabbkommandot CQD (come quick danger). Detta fungerade alldeles utmärkt och under de här åren var det nästan ingen som fick träben. Ända till 1906 när det hölls internationell radiotelegrafkonferens i Berlin och någon föreslog att man skulle ändra till det mer spirituella \textsc{(se hippie s.~\pageref{14fd61fa8edcb67c5c7886f11af8431e})} SOS (save our souls). Ända sedan dess har sjökaptener och radiotelegrafister tvingats kämpa med att memorera den nya signalen. Ingen vet hur många liv som gått till spillo på grund av att mottagaren ännu inte lärt sig denna nya särskeförkortning \textsc{(se särske s.~\pageref{552a5aad891937bf760fb193900ea140})}.

 HEAD2: Koderna i fråga

 CQD: en lång en kort en lång en kort, två långa en kort en lång, en lång två korta
 SOS: tre korta, tre långa, tre korta

}

\small{
\textbf{Crass}
\label{df658ae34e7649f4e9f73f9341991f50}
 En samling akademiska britter som hade ett kollektiv.
 Lite som nissepedia \textsc{(s.~\pageref{62400dadecd90cb5cd39062abe5a3e4a})}, fast i musikalisk form: alla får vara med även om man är eljest \textsc{(s.~\pageref{2ded84b3f6092a191088df8b0e9d0a57})}.

}

\small{
\textbf{Creddiga}
\label{95574174323e4f0cbd25b65f6c811f26}
 Nån har hört att det är inne i Stockholm \textsc{(s.~\pageref{edcd259e0a03c7ab70feb186bae19f13})}.

}

\small{
\textbf{Critters 3}
\label{835986186e5e81facd99684c86e29d41}
 är en skräckfilm från the United States of America \textsc{(s.~\pageref{ade6b3bd5e720abb20ed8a9a4c6b9ae8})}. I centrum står en grupp människor som bor i ett hyreshus och som är i färd med att bli vräkta av den onda hyresvärden. Gruppen har en intrikat sammansättning och i den ingår en manhaftig kvinna \textsc{(s.~\pageref{9a7760b2521c3471c47cd5d789a2d324})}, en lite frånvarande ensamstående far och hans två barn \textsc{(s.~\pageref{5dfcc0aab2f3db925b2d51ba73e48946})}, en tjock tant och ett äldre par. Värdens styvson, som spelas av Leonardo DiCaprio, dyker också upp efter ett tag. Huset invaderas av ludna bollar från yttre rymden som vill äta upp gruppens medlemmar, men som tur är äter de bara upp den onda hyresvärden och vaktmästaren.

}

\small{
\textbf{Crust as fuck existence}
\label{bd0b07abcc2f4c2a4e1aafdfed1f0e73}
 är en skiva av crustbandet Warcollapse \textsc{(s.~\pageref{d39e006de7daeb1166d6b6c7990582dd})}. Skivan är lätt deras tyngsta och därmed bästa. Låtarna är jättelånga och som tidigare sagt tunga som fan. På omslaget som syns nedan är det en tant som har huckle och kramar en döskalle och lipar. Det är crust det.

}

\small{
\textbf{Crustare}
\label{58197b69d4be0c18c21c6e8d3f950270}
 En crustare är någon som identifierar sig med subgenren crust. Crusten härstammar från kängpunken, men har ofta lite mer metalinfluenser, burkigare inspelningskvalité och texter om politiska grejer.

 HEAD3: Stil
 Stilmässigt karakteriseras crustaren av så kallade crustflätor (dreadlocks, typ) i nacken, en väst prydd av många tygmärken med bands namn på (ex. Doom, Dystopia, Nausea, Hellbastard, Amebix, Misantropic osv), en keps med tygmärke på, en magväska (som de kallar crustväska för att det ska vara lite ballare), trasiga byxor, en loppbiten schäfer \textsc{(se alkisschäfer s.~\pageref{347febbc28041eae88556d2e7ced587b})} och friluftsskor.
 HEAD3: Ideologi
 Crustaren utövar ofta dumpsterdiving \textsc{(se sopletare s.~\pageref{a9729f0b2c1c1ec17cec4dc9fdb10007})} (äter sopor), liftar frekvent med lastbilschaffisar och super. Vad som skiljer en crustare från en träskpunkare \textsc{(s.~\pageref{484838b3db1adb135ea74d6fc61e44c0})} är allt som oftast politiska idéer. Crustaren äter sopor som konsumtionskritik, inte för att det är gratis. Crustaren bor på gatan för att uppmärksamma bostadsbristen, inte för att den blev utkastad från morsan för att den var för äcklig. Crustaren liftar inte för att det är gratis och det kanske finns baconchips på golvet i lastbilshytten, utan för att den inte vill bidra till den miljöförstörande tåg- eller flygtrafiken.

 Trots att crustaren sällan har några andra ägodelar än det den har på/i kroppen, lyckas den likförbannat alltid ha något störigt fanzine med sig, i vilket alla små individuella bushandlingar \textsc{(se jävelskap s.~\pageref{46845591177f16920cd586a5baf5a625})} crustaren utför bekräftas som extremt revolutionära och potentitellt samhällskullkastande. På så sätt har crustscenen påfallande mycket gemensamt med vegan straight edge-scenen, men det talas det inte så ofta om.

}

\small{
\textbf{Crustknytning}
\label{f9f5d7389b273a02e992d998eaafcece}
 , AKA fladdermöss, är ett sätt att knyta skorna som framförallt brukas av träskpunkare \textsc{(s.~\pageref{484838b3db1adb135ea74d6fc61e44c0})} och andra slan \textsc{(s.~\pageref{caaad522de864ab45ed679c4a16edd8d})}. Istället för att använda den klassiska rosetten, åttan \textsc{(se åtta s.~\pageref{6fa68b0d02ec525fa72a51c13e5e3ed1})} eller råbandsknopen föredrar träskpunkaren att dra ut skosnörena ur de övre hålen och sedan knyta en rejäl dubbelknut i varje ände. Snörningen är nu helt omöjlig att dra åt men snörena åker inte ur, vilket är just den effekt som eftersöks eftersom det är jobbigt att knyta skorna. Plösen hänger nu slappt som tungan på en golden retriever \textsc{(s.~\pageref{2b3d7c01f0a8a57d8fa2a18b54993a6b})} och det fullkomligt rasar in grus i kängorna. Men det ger träsktrollet blanka fan i eftersom det är crust as fuck existence \textsc{(s.~\pageref{bd0b07abcc2f4c2a4e1aafdfed1f0e73})} och hen, om sanningen ska fram, aldrig lärt sig knyta en riktig knut eftersom större delen av skoltiden spenderades i rökrutan.

}

\small{
\textbf{Crustpippi}
\label{bed4fd9fed3852a30c590ce2152ff863}
 djuplodande research.]]
 Alla crustband av någorlunda rang har en egen crustpippi. Det är siluetten av en fågel och återfinns någonstans på omslaget. Kanske ska pippin symbolisera crustarnas längtan efter frihet eller så är det som vanligt bara en homage till Discharge, som populariserade pippikonceptet på sin skiva \textit{Never again} med en spetsad fredsduva. De flesta crustpippis är annars kråkor eftersom kråkorna lever ett crustigare liv \textsc{(se crust as fuck existence s.~\pageref{bd0b07abcc2f4c2a4e1aafdfed1f0e73})}. Stadsduvor är ju egentligen ännu crustigare men det är för svårt att skilja dom från fredsduvorna på ett svartvitt omslag täckt av vapen, nitar och misär.

}

\small{
\textbf{Curry-curry}
\label{8048f1110beaeff78aab6b8ad65e28d4}
 betyder i princip \quotetext{varning, se upp, det är något skumt på gång}. Något verkar curry-curry. Inte tillförlitligt. Kanske roligt men inte att lita på.  Att bygga sitt hus på sand \textsc{(s.~\pageref{88336b5bb2a1cc21bac7cf33fd451270})} är lite curry-curry tex.

}

\small{
\textbf{Currykondom}
\label{cd987f145bcbad879ca394d2d22e8ae6}
 En currykondom är en kondom full med curry. Som preventivmedel är den värdelös.

}

\small{
\textbf{Cykelhjälm}
\label{eccbf5255f85f74dc67bffbcae78793b}
 är en anordning för att skydda kraniet från företrädesvis trubbigt våld. Ungefär som en suspensoar, fast för huvudet.

 Barn är ofta ivriga användare av cykelhjälm, då de i sitt ständiga testande av gränser även stundtals försöker utmana gravitationen. När dessutom samhället har en nedlåtande syn på minderårigas bruk av folköl så är det inte lätt för de små liven att sänka sin tyngdpunkt på samma sätt som är brukligt för en medelålders man med pondus \textsc{(se kalaskula s.~\pageref{e889c1a4915c4b4aad08d49192e79276})} eller en stigmatiserad bärshagga. Således får de dras med en frigolitbit på huvudet, och drömma om den dag de är stora nog att känna fartvinden blåsa i håret när de rullar ner för en kurvig grusväg på en rostig Crescent med trasig fotbroms.

 HEAD3: Inverkan på sexualiteten

 Att skaffa sig en tillfällig sexuell kontakt bärandes cykelhjälm står när det gäller svårighetsgrad i paritet med att försöka lösa Goldbachs hypotes under en redig spritfylla \textsc{(s.~\pageref{0668c687b51995118ec27cbf25061118})}.

 HEAD3: Cykelhjälm på vuxna

 Hos vuxna är cykelhjälm ofta en säker signal på att bäraren har någon form av handikapp som inverkar på förståndet. Om bäraren av cykelhjälmen stirrar på en fast punkt i gatan tre meter bort, eller lallar lite för sig själv kan man nästan vara helt säker på att så är fallet. Observera att om bäraren är en äldre dam kan det vara en lågstadielärarinna som föregår med gott exempel på vägen till sin sadomasochistiska swingerträff. Man bör alltså ta sig en funderare eller två innan man i onödan ringer länsman och meddelar att ett av inhysehjonen har rymt.

 HEAD3: Cykelhjälm i storstaden

 I Stockholm, där man ju gillar att vara lite normbrytande, är fullvuxna, arbetsföra män med cykelhjälm en icke ovanlig syn. Fan vet varför, men kanske lever de i villfarelsen att de på något vis är unika och att världen på något vis skulle bli en fattigare plats om deras främre pannlob sipprade ner i rännstenen på Rådmansgatan.

 Vi andra kan lugnt trampa på, väl medvetna om den Lovecraftianska futtigheten i vår existens.

}

\small{
\textbf{D-takt}
\label{6ad6b7303fd9c7170886b11040e69994}
 Guds \textsc{(se Gud s.~\pageref{91e49146121c992aab11a19c77e26cf0})} hjärtrytm. Bupp u dupp u du.

 Ett väl beprövat knep för att få en punkare i säng är att dra raggningsrepliken: När jag ser på dig slår mitt hjärta i D-takt

 Category: Religion \textsc{(s.~\pageref{a9495ca51f21bcf94413d3a77bac9301})}
 Category: Musik \textsc{(s.~\pageref{c7347c7448ee86744c688e69d49926eb})}

}

\small{
\textbf{Da Ya Think I'm Sexy?}
\label{a0cdb5ccdbb006538694edf2f5ba7fb1}
 \textit{Da Ya Think I'm Sexy?} är en låt av den brittiske artisten Rod Stewart. Den gavs ut 1978 och alla intäckter från singeln gick till UNICEF. På vilket sätt temat i låten passar till en insamling för världens barn är lite av en gåta. Kanske var det ett slugt försök av Rod att försöka skaffa FN-mandat på att han är guds gåva till mänskligheten. En djuplodande analys av TV-programmet Mythbusters kom dock fram till att Rod aldrig någonsin varit särskilt snygg. Rod Stewart lever fortfarande och inte har han blivit nå snyggare med åren precis.

 Låten själv kom på plats 301 i tidningen Rolling Stones omröstning över världens 500 bästa låtar.

}

\small{
\textbf{Dackefejden}
\label{0aa3c03182b8c3edaf72541cd88161ff}
 Jag är inte helt säker på vad Dackefejden gick ut på men jag ska försöka förklara så gott jag kan: Fejden utkämpades under 1500-talet och tog formen av ett uppror som startades av missnöjda småländska bönder ledda av en man som hette Nisse, precis som nissepedias \textsc{(se nissepedia s.~\pageref{62400dadecd90cb5cd39062abe5a3e4a})} mytologiske grundare. I efternamn hette han Dacke. Upproret riktades mot Gustav Vasa. Å ena sidan arga bönder \textsc{(s.~\pageref{30a6fc00c9102680b8196b1b79935ec4})}, alltså, å andra sidan kungen. Ser man kronan som den tidens länsstyrelse kan Dackefejden med lite fantasi ses som Sveriges \textsc{(se Sverige s.~\pageref{b1999637949ed135b2ca03f3a38460cc})} första process mot länsstyrelsen \textsc{(se processa mot länsstyrelsen s.~\pageref{0ae3fdeda52fe82800b04c624330139c})}. Dessvärre slutade fejden med att bönderna förlorade, varpå allmän utrensning och massakrer följde. Ur ruinerna av den småländska glesbygden steg föregångaren till centerpartiet \textsc{(s.~\pageref{e331dec360e356adc1e2db36fe9a9f3f})} upp som en fågeln Fenix och resten, det vill säga alla högextrema och smyg-fascistiska fraktioner som knoppats av från detta parti, är historia.

}

\small{
\textbf{Dagblad}
\label{d074687e28275e17cdd4f778e9cf96c9}
 Du är för fin för att läsa avis. Du läser dagblad.

}

\small{
\textbf{Dagens Nyheter}
\label{b159d08de8d21d8a6d79374b02793693}
 , eller DN som den förkortas, är Sveriges \textsc{(se Sverige s.~\pageref{b1999637949ed135b2ca03f3a38460cc})} största, men inte blåaste, morgontidning. Missförstå inte, den är jävligt blå, men ibland skriver typ Kajsa Ekis Ekman nåt som man tokhåller med om. Sen skriver Peter Wolodarski nåt om att marknaden ska vara fri och man ba suckar och läser på flingpaketet istället.

}

\small{
\textbf{Dagsedel}
\label{7a59187e5d4c12ded69d05197d099485}
 En dagsedel är en välförtjänt örfil som man får när man missköter sig i sällskapslivet. Kanske har man kommenterat utseendet hos världparets barn då man är på besök nästgårds och sitter till bords. Kanske har man utpekat skavanker hos sitt sällskap under en kväll på lokal, eller så har man gått och förargat en bonde genom att sova i dennes potatisåker. Hursomhelst har någon blivit vred och vill genom att ge dig en dagsedel uppmana dig att genast upphöra med detta beteende.
 HEAD2: Folk som ska ha en dagsedel
 \begin{itemize}
 \item Carl Bildt
 \item Karl-Gustav XVI
 \item Pär Ström
 \item Adde Malmberg \textsc{(s.~\pageref{1390facdddaee5ed00a964fbe93b30b9})}
 \item Länsstyrelsen
 \end{itemize}

}

\small{
\textbf{Dagsfylla}
\label{e79459471993abd0ccde4df08bafdb22}
 är när man blir full på dagen. Det är en både väldigt rolig och intressant upplevelse. Att gå på renmarkstorget i Umeå, lullig och fin, mitt på en torsdag är en upplevelse alla borde prova på. Man känner sig helt frånkopplad den ordinarie verkligheten, samtidig som den aldrig känts mer påträngande. En dagsfylla uppnås med fördel på ett vernissage \textsc{(s.~\pageref{ce23171c27a2528bf43af71778ad0046})}, då det dels är gratis och man dessutom får grunda i magen med OLW-hjärtan.


 HEAD2:  Se även
 Bukfylla \textsc{(s.~\pageref{f904f4abb175812d2f7938d05cd94459})}

}

\small{
\textbf{Dagvill}
\label{481b5b179b30ebb00ec5c098747d9ade}
 Ett tillstånd som går ut på att en individ glömmer bort vilken dag det är. Den moderna vetenskapen har lyckats finna två anledningar till detta tillstånd. Den ena beror på arbetslöshet och/eller ledighet och gör att vilken dag det är blir oviktigt för att man ändå inte har några tider att passa. Den andra beror på att man helt enkelt har för mycket att göra och minnet sviker omedvetet för att dämpa ångest över allt som ska hinnas med, det som läkare kallar stress.

}

\small{
\textbf{Dajmgöke}
\label{d1553660d19e1d851e670e47eb3fcedb}
 är en delikatess som ofta serveras som förrätt i anslutning till mer högtidliga middagsfester. Delikatessens huvudingrediens är en vanlig gök som dött av förvåning med näbben på vid gavel. Sedan har en helt vanlig dubbeldajm resolut pressats ned i gökens gap som efter detta tejpas igen medelst frystejp. Alltihop vändsteks tillsammans med vitlök och timjan. Tillredningen avslutas med en skvätt citron som pressas över den nu gylldene rätten just före servering.

 Dajmgöke äts normalt på den lilla förrättstallriken och med en tretandad gaffel. Till rätten serveras vanligtvis pucko men ibland serveras ett torrt Bourdeaux-vin om en låda sådant råkat trilla av ett lastbilsflak i närheten.

}

\small{
\textbf{Damcykel}
\label{2f0f41314b4e4edb773a7ae87addc913}
 En stor seger för feminismen. Lite tuffare än en Herrcykel \textsc{(s.~\pageref{8da016d4ced142cc447d520eaa04c33e})}. Är med dagens avtagande kjolmode helt onödig.

 Intersektionalister \textsc{(se Intersektionalitet s.~\pageref{6dc08633cdbf83eb418ea31ef0302c51})} hävdar att man även måste ta i beaktande \textsc{(se beakta s.~\pageref{5fb8066c875cfced859cf8968e991628})} antalet växlar på cykeln. Se Flerväxlad cykel \textsc{(s.~\pageref{cd75a1ec5d4b7caabeaaaf25edee0250})}. Detta bidrar till att cyklar och genus har ett mycket inflammerat förhållande.

 Category:Trafik \textsc{(s.~\pageref{8a2f75cb2fdbbd1b67833430f8bc0f33})}
 Category:Konst och kultur \textsc{(s.~\pageref{5b7893b41d66acf6fe68114ec71b6743})}

}

\small{
\textbf{Dank}
\label{eee1edb16ac8987af66023852db6c513}
 beskriver hur läget är. Det kan betyda både segt och lugnt. Det tidígare betonar danks negativa kvalitéer, det vill säga ett läge där inget händer, vilket man tycker är jobbigt. Den senare betydelsen innebär att man tycker det är rätt skönt att det inte händer nåt. Hur man använder uttrycket beror på vilken inställning man har till livet. Softa personer man gärna blir vän med tycker det är nice att ha det dankt.

 Dank är också en sorts kula, avsedd för hasardspel. Den mäktigaste danken är en järndank som är stor, tung och ständigt i dominant ställning på nötta gatuhörn världen över.

}

\small{
\textbf{Danmark}
\label{5331d7fd27772396f412a5b6d19bad44}
 , även kallat Legoland \textsc{(se Lego s.~\pageref{3a22c9ea9a3039d180e0a514a5a3b619})}, är en fånig landmassa av kalksand där alla är rödbrusiga, heter Preben eller Margarete och röstar på nazisterna eller högerpartiet Venstre.
 HEAD2: Hur Danmark blev till
 I bibeln \textsc{(s.~\pageref{7de7d2a7d608c9a2044f50688bc63e27})} står att läsa om hur herren skapte Norden. På den första dagen skapade han Svea Rike, den andra Norge, sen Finland och sist Island. På den femte dagen tog Gud sig en redig blecka och vaknade på den sjätte dagen i Skåne och var riktigt risig i kistan. Med ena foten i Kattegatt och den andra i Östersjön satte sig vår herre och ut kom det som vi idag känner som Danmark.
 HEAD2: Dansk kultur
 Dansk lättöl \textsc{(s.~\pageref{3981afb990a974a3b4613a470e51e747})}, sirapskokt potatis och brunt bröd med rostbiff och majonäs, röd korv \textsc{(se rød pølse s.~\pageref{dd9f0cd7c204300945924c7de9eb5649})}, legoporr och Kim Larsen.
 HEAD2: Danmarks historia
 Som bekant konstaterade redan Hamlet \textsc{(s.~\pageref{ea3596139530b2abe7089082ab57ecbd})} att \textit{\quotetext{There is something rotten in the state of Denmark}} och inte har det blivit bättre precis. De enda årtalen att hålla reda på är 1984 och 1985 när Metallica spelade in \textit{Ride the lightning} och \textit{Master of puppets} i Köpenhamn.

 HEAD2: Folkdräkt
 Propellerkeps \textsc{(s.~\pageref{34087753e20a67ca90f6c51bcae4528e})} och därutöver kalle anka \textsc{(s.~\pageref{64db68f686a0ca4d9d641061cb3fdf13})}. Följer du dessa enkla tumregler kan du \textsc{(se Användare: prof. Etienne s.~\pageref{a9878d2280e5a39becac8f73d113df91})} fly från arga barnbokskonsumenter från vårt hemland, genom den outhärdliga danska landstungan, till friheten i Andorra.

 HEAD2: Danmarks framtid
 Danskarna kommer drivas bort från sina kolonier Grönland och Färöarna. Alla danskar på fastlandet får flytta till Fyn \textsc{(s.~\pageref{9854682d71fdb60a819d9188a846f42d})} medan resten av konungariket tillfaller dvärgarna. Dvärgar har i alla tider varit förföljda och hånade så dom förtjänar sannerligen ett eget land.
 HEAD2: Danmarks befolkning
 Danmark befolkas framförallt av danskar. De flesta av dessa arbetar med att arrangera Roskildefestivalen, men andra populära sysselsättningar inkluderar även att hissa Dannebrogar \textsc{(se Dannebrogen s.~\pageref{d8c97891c74597fa443ed507c4191fe0})} och cykla christianiacykel \textsc{(s.~\pageref{671a1992db86e328dc9c068647d57d6b})}. Ett stort problem med danskar är att dom typ inte fattar någonting. Det genererar till exempel problem när en dansk ska ringa utomlands. Man ba: \textit{Hallå?}, och dansken på en gång ba: \textit{Jæ gødæ, må dü spiese din kartoffel?}. Man fattar ingenting och ba: \textit{Ursäkta, vad snackar du om?}, men dansken ba skiter i frågan och babblar på: \textit{Jæ dæt ær gøtt. Jæg sætt hær pæ mïn chrïstïænïæcykel ok fjærsade lit grisæporr}. Man fattar ännu mindre och ba: \textit{Va fan snackar du om?}, men han ger sig inte utan bara kör på: \textit{Jæ, då koem Cærsten træskænde anders and \textsc{(se Kalle anka s.~\pageref{64db68f686a0ca4d9d641061cb3fdf13})} ok fræga om han må låna mïn Danebrøge. Mæn de er mïn Danebrøge! Mïn! Mïn! MÏN!}. Sen lägger han på.

 HEAD2: Dansk vetenskap
 I Danmark omöjliggörs all form av naturvetenskap tack vare att deras räknesystem är helt omöjligt att lära ut, och därmed finns inte matematik \textit{as we know it}. Samhällsvetenskaperna lyser också med sin frånvaro. Detta beror på var deras politiska sympatier ligger - se ovan. Inga upptäckter har gjorts av danskar sedan Tycho Brahes tid. Länge trodde man att Sören Kirkegaard var något på spåren, men det visade sig senare att så inte var fallet - han hade bara ångest, och så var det med det. Samma sak med Niels Bohr som trodde sig ha listat ut hur elektroner kretsade runt atomkärnan, men visade sig ha åt helvete fel. Hans modell stämmer typ på väteatomen men är en grov förenkling.
 HEAD2: Sex och samlevnad i Danmark
 I Danmark är det väldigt populärt med sån därade pörr \textsc{(s.~\pageref{5faa435e2f0af7617816f0cade262581})}, en av få saker som lockar turister till landet. I övrigt föredrar danskarna att ligga med varandras djur, till en billig penning (en femhunka för ett hästskjut är standard).
 HEAD2: Sport i Danmark
 Rundpingis, prutta högst \textsc{(se Prutta högljutt s.~\pageref{8fa3e05871b2747109018026471a935a})}, dricka 20 bärs \textsc{(se ha bärs s.~\pageref{a74b297c15834437ac2e49095492133c})} och sen försöka slå ner domaren.

}

\small{
\textbf{Dannebrogen}
\label{d8c97891c74597fa443ed507c4191fe0}
 är Danmarks \textsc{(se Danmark s.~\pageref{5331d7fd27772396f412a5b6d19bad44})} flagga och den äldsta nu officiellt använda flaggan i världen. Den sägs enligt legenden ha fallit från skyarna till de danska trupperna under ett slag i Estland på 1200-talet. Flaggan föreställer två korslagda smuggelcigg \textsc{(s.~\pageref{2bcc66e1261fa5199a4f4decf2720ef5})} mot bakgrund av en flottig röd bordsduk.

}

\small{
\textbf{Dans}
\label{ef86916bc6f9f2f6866df100a192161f}
 , en primitiv akt; ett primitivt sätt att skapa kontakt.


 HEAD2:  Källa
 [http://www.hiphoptexter.com/text.asp?textid=13494 Nynazistisk trubadur] efter andra skivan
 \textsc{(se första skivanalibi s.~\pageref{0fd7abea0db14d19df73202811130364})}

}

\small{
\textbf{Dansk advent}
\label{dc3610baedc341bc7fecde12589b848b}
 I Danmark \textsc{(s.~\pageref{5331d7fd27772396f412a5b6d19bad44})}, liksom i Sverige \textsc{(s.~\pageref{b1999637949ed135b2ca03f3a38460cc})}, firas sedan århundraden advent i väntan på julen, men i vårt lilla grannland i sydväst har man valt att fira på ett för oss annorlunda och ofta obegripligt sätt. Som alla andra dagar samlas man och dricker Tuborg och lyssnar på AC/DC, men till advent placerar man fyra cigaretter i tomflaskorna i den utsmyckade ölbacken. Sedan rullar man helt enkelt fram gamla farmor och tänder på när hon släpper väder. På så vis tänds \quotetext{ljusen,} som symboliserar danskens andaktsfulla vaka inför Jesu födelsedag \textsc{(se dansk jul s.~\pageref{a798972c27193a9a3d71b3c2f6e83eb4})}. Därefter kastar man sig utan omsvep på varandra i ett naket rallarslagsmål.

 På den danska statstelevisionen sänds en tillställning som är en mer grandios version av adventsfirandet i det danska folkhemmet. Denna årliga ceremoni inbegriper en hel del orgiastiska förnöjelser som alla är totalförbjudna i Sverige. De kulminerar i och med att Lars Krogh \textsc{(s.~\pageref{7a5a84f5d8e84be6ff55aa9709c3dacd})}, bakfull och stenad \textsc{(s.~\pageref{dec4a3a91f0f2bf8dcf033a8cfeaa554})} som en feministisk bloggare i Teheran, släpar sig fram till en enorm fyrbåk och lätt framåtlutad öppnar sitt anus, varpå Brøndbyernes IFs gamla fotbollsmålis Peter Schmeichel antänder den legendariske skivbolagsmannens gaser med sin pinup-zippotändare. Detta är startsignalen för flera dagar långa upplopp mellan ungdomar och poliskår och under hela tiden står varje soptunna i varje bebyggt hörn av landet i fyr, som en påminnelse om att ljuset återvänder och att det är så \quotetext{jaevlig dejlig} med grisfylla \textsc{(se Grisfull s.~\pageref{80fc21ba5a45f2d0cd24855d78fa7246})}.

}

\small{
\textbf{Dansk forskning}
\label{cab0bd13aa31ff615c472d4473f20709}
 Sedan 1951 går alla forskningsanslag i Danmark \textsc{(s.~\pageref{5331d7fd27772396f412a5b6d19bad44})} enkom till studier rörande avståndet mellan mannens penis och anus. Alltsedan Mogens Palleprotogen Hans Jörgen Jacobsen råkade visa sin ändalykt i \quotetext{Der må være en sengekant} \textsc{(se Sengekantsfilm s.~\pageref{36f1eab94ebc00e11292cfaa67acafa0})} så har detta relativa avstånd också varit centrum för kulturdebatten i Danmark. Forskningen är rörande överens om att ovan nämnda pikanta variabel är ständigt sjunkande. En sorts jakt mot mellangårdens krympande lebensraum. Vid nollpunkten kommer Danmark och övriga världen, som vi känner dem, att haverera. Videnskaben!

}

\small{
\textbf{Dansk jul}
\label{a798972c27193a9a3d71b3c2f6e83eb4}
 En äkta dansk jul firas tillsammans med sina nära och kära. Alltså de tältgrannar från Roskildefestivalen som man har kvar telefonnumret till. Den lilla gruppen samlas 3-4 dagar innan dopparedagen för att hinna bli varma i kläderna och öva upp levern inför det verkliga firandet. Barnen \textsc{(se Barn s.~\pageref{5dfcc0aab2f3db925b2d51ba73e48946})} stängs in i källaren för att ingen ska komma till skada, och om någon av dem ifrågasätter detta berättar man den gamla folksagan om Boetius de Dacia \textsc{(s.~\pageref{ad665ff4eb3f48ad8f41fc7fc9c246c8})}. Barnen blir då livrädda och allt är frid och fröjd. Juldagsmorgonen börjar med att gänget sätter sig på sina christianiacyklar \textsc{(se christianiacykel s.~\pageref{671a1992db86e328dc9c068647d57d6b})} och trampar iväg till Tivoli \textsc{(s.~\pageref{5926d4ffe73f106a0bc0929068981515})} för att sätta eld på julgranen där. De som är för bakfulla stannar hemma och släpper ut barnen ur källaren. Därefter följer den högtidliga julmiddagen där traditionella läckerheter som skagenröra \textsc{(se skagen s.~\pageref{d88f07528fa07f7be9318ece7656fd0b})}, balutägg \textsc{(s.~\pageref{8561722057ba8ec26075477dab5ef4de})}, uvsvane \textsc{(s.~\pageref{c5081b14cdeb1ff42b655213e80c9d51})}, rød pølse \textsc{(s.~\pageref{dd9f0cd7c204300945924c7de9eb5649})}, gamle Ole \textsc{(s.~\pageref{1092ad25ed7624715bba1a2a30eb8307})} och fläsksvålar \textsc{(s.~\pageref{a8f15feead71638a458fbbb29141931b})} dukas fram. Alla har svårt att vänta med klappöppningen till kvällen så för att dämpa nyfikenheten ger alla varandra en inslagen dunk snaps. När man druckit sig mätta samlas man runt brasan av läxor som ungarna gjort upp och läser grukar \textsc{(se gruk s.~\pageref{78233c5ad0b90efdffd147b849201ce4})}. Vid det här laget är alla så fulla att man glömt bort att det är julafton och det kommer därför som en total överraskning när Lars Krogh \textsc{(s.~\pageref{7a5a84f5d8e84be6ff55aa9709c3dacd})}, iklädd nerspytt lösskägg, sparkar in dörren och börjar kasta paket på allt som rör sig. Alla har önskat sig sengekantsfilm \textsc{(s.~\pageref{36f1eab94ebc00e11292cfaa67acafa0})}, vilket man också får och därmed är kvällen officiellt till ända. Ungarna schasas tillbaka till källaren och de vuxna som ännu är vakna hälsar Jesusbarnet \textsc{(se Jesus s.~\pageref{110d46fcd978c24f306cd7fa23464d73})} välkommen till världen med ett entusiastiskt gruppeknoll.

}

\small{
\textbf{Dansk kanot}
\label{884458cb1a5fba0e5e027e7d1377bbc8}
 Danmark \textsc{(s.~\pageref{5331d7fd27772396f412a5b6d19bad44})} är som bekant ett örike men landmassor utspridda värre än camparna på Roskildefestivalen. Något vår kung Karl X Gustav exempelvis fick erfara bittert när han släpade ut kavaleriet på öresunds bräckliga isar för att än en gång ge Preben på tassen och ta förråden av Tuborg \textsc{(s.~\pageref{49bb0f04b9993881c9d9c5b115cc35f0})} och Gamle Ole \textsc{(s.~\pageref{1092ad25ed7624715bba1a2a30eb8307})} i krigsbyte.  Simningens ädla konst har aldrig varit allmän kunskap hos befolkningen utan har mer betraktats som magi eftersom vatten är förknippat med tabubelagda ämnen som tvagning och alkoholfritt. Tappra försök med att pumpa christianiacykelns \textsc{(se christianiacykel s.~\pageref{671a1992db86e328dc9c068647d57d6b})} däck med helium och fylla dannebrogen \textsc{(s.~\pageref{d8c97891c74597fa443ed507c4191fe0})} med varmluft över en brinnande soptunna har allt som oftast slutat i katastrof när dansken önskat resa mellan två öar. Ett pionjärprojekt med att gå på havsbottnen mellan Fyn \textsc{(s.~\pageref{9854682d71fdb60a819d9188a846f42d})} och Samsø och andas i en trädgårdsslang var lyckat till en början men avbröts när tanten som ägde slangen ville ha tillbaka den.

 År 1970 bestämde sig den pensionerade pølsenauten \textsc{(se pølsenautologi  s.~\pageref{765f084d8da82e07e5e9acbbadd0d3f2})} Mogens Sandfær för att lösa detta problem en gång för alla. Hans mål var att skapa en farkost som kunde härma egenskaperna hos de sälkadaver han sett färdas nästan i nivå med vattenytan. Med hjälp av ett par vadarstövlar och en lång pinne lyckades han samla in tillräckligt med döda sälar som han knöt ihop med ett hamparep och fick på så sätt en konstruktion stark nog att bära honom. Den första prototypen var svår att manövrera men när han kompletterade uppfinningen med att använda en spade till att skotta bort vattnet i fören märkte han att konstruktionen faktiskt började backa. Det danska patentverker, Pax! \textsc{(se paxa s.~\pageref{0e00979a45d6f4083485e9c9fb01f590})}, beviljade patent och Sandfær har allt sedan dess varit en förmögen man med monopol på inhemsk sjöfart.

}

\small{
\textbf{Dansk kortlek}
\label{a020818cbeef7383cd1e77d36cd9a2db}
 En dansk kortlek skiljer sig från en traditionell anglosaxisk \textsc{(se anglosax s.~\pageref{75591674b0deca83291ccfef6f4f557c})} kortlek i det att den har numren utskrivna i alla fyra hörnen. Det klassiska, och betydligt snyggare är att man enbart har numren utskrivna uppe i vänstra hörnet och nere i högra. Visst, det blir lite krångligare att läsa men kortspel handlar ju till stor det om stil; att med eftertryck studera sin perfekta solfjäder och ödmjukt gratulera motståndaren till vinsten i en vänskaplig omgång Chicago. Stil är dock som bekant ett begrepp som inte existerar i Danmark \textsc{(s.~\pageref{5331d7fd27772396f412a5b6d19bad44})}, där man sopat in siffrorna på alla tillgängliga hörn för att så fort som möjligt kunna utropa sig till vinnare och hälla öl över motståndarna i ett parti Finns i sjön eller Svälta räv \textsc{(s.~\pageref{f5ba1e0ca45e2d553c6282cb290878dd})}.

}

\small{
\textbf{Dansk kostcirkel}
\label{02cded90af4097cde19fa2bebceafe6e}
 Den danska kostcirkeln skiljer sig från den internationellt erkända kostcirkeln genom att den saknar sex av dennas sju komponenter, men samtidigt har två helt egna kategorier, nämligen Tuborg \textsc{(s.~\pageref{49bb0f04b9993881c9d9c5b115cc35f0})} och lettisk smuggelcigg. Den näringskategori som de danska och internationella cirklarna har gemensamt är naturligtvis rotfrukterna, en kategori som i Danmark \textsc{(s.~\pageref{5331d7fd27772396f412a5b6d19bad44})} helt domineras av kartoffel.

}

\small{
\textbf{Dansk lättöl}
\label{3981afb990a974a3b4613a470e51e747}
 En Dansk lättöl tillverkas genom att slå lite vatten i en starköl. Det gör dansken när han behöver köra bil men prompt ska dricka starköl.

}

\small{
\textbf{Dansk midsommar}
\label{efeb6deed755bebe19738c9fb892b3b8}
 Trots att Danmark \textsc{(s.~\pageref{5331d7fd27772396f412a5b6d19bad44})} är platt som Jan Björklund \textsc{(s.~\pageref{0b9b757044804b9be0e218acdad358cc})} i en partiledardebatt är den ändå helt omöjligt att hitta en plan yta i trädgården för att duka upp till långbord. I mitten att grönytan ligger vanligtvis komposten för att enkelt kunna nås när Preben är sugen på sega maskar men inte orkar gå till Haribo-affären. Någonstans framför garaget ligger den kamouflerade björngropen som lille Cærsten byggt så där vill man inte vara. På uteplatsen ligger pantberget och eftersom hälften av flaskorna är krossade blir det för jobbigt att städa bort. Framför rabatten är inte heller någon idé för där sitter Bærsa-Lottas alkisschäfer \textsc{(s.~\pageref{347febbc28041eae88556d2e7ced587b})} fastkedjad vid det omkullblåsta vindkraftverket.

 Så där ser det ut på i princip alla danska gårdar så när midsommarafton infinner sig lastar man iställer christianiacykeln \textsc{(se christianiacykel s.~\pageref{671a1992db86e328dc9c068647d57d6b})} med allt som behövs och trampar ut på motorvägen där marken är platt och stadig. Pigena dukar upp bordet i ytterfilen där bilarna inte kör lika fort medan drængarna fäller en telefonstolpe att bygga midsommarstång av. Barnen plockar blommor som de dekorerar molotov cocktailsen \textsc{(se molotov cocktail s.~\pageref{7d215b4a5cc923f86bb4c229af6e2d61})} med som ska användas på natten när det är dags att städa bort kalaset.

 Som traditionen bjuder inleder bestefar firandet med ett osammanhängande tal som avslutas med att han spelar luftgitarr \textsc{(s.~\pageref{0e2415e86edc316f5338964c6ef145b5})} och skriker en låt av AC/DC. När någon tröttnat på att bara sitta still och supa klär den av sig Kalle Anka \textsc{(se kalle anka s.~\pageref{64db68f686a0ca4d9d641061cb3fdf13})} och dansar en sväng \textit{Små grodorna} på bordet. De andra brukar snabbt haka på och det tar inte lång tid innan dansen urartat i en ormgrop nere i diket. På väg hem från festen samlar de ogifta sju sorters pantburkar som de stoppar i kudden eftersom det mesta av dunet gick åt när rektorn skulle tjäras och fjädras på skolavslutningen.

 \textit{Dansk midsommar}

}

\small{
\textbf{Dansk onsdag}
\label{6b791a2c8244d0f85acaf3aad968442d}
 Som vanligt träffas man, lyssnar på AC/DC och dricker bärs \textsc{(se bärsfylla s.~\pageref{9380b60f9ee744b9acf978fe6f1a9545})}. \textit{Bron} står på på TV \textsc{(se television s.~\pageref{79464212afb7fd6c38699d0617eaedeb})} men ingen kollar så noga för den överröstas ändå av den lite nedgrisade australiensiska \textsc{(se australien s.~\pageref{e727d8d1b3162a732c7f706d55de64f3})} badboyen som snurrar på skivspelaren. Efter en back eller så ska Preben, umgängeskretsens lustigkurre, prompt ha en skämttatuering, som vanligt. Mot 21:34-tiden blir det sengekantsfilm \textsc{(s.~\pageref{36f1eab94ebc00e11292cfaa67acafa0})} och högljudda och samstämmiga samtal om böglandet Sverige \textsc{(s.~\pageref{b1999637949ed135b2ca03f3a38460cc})} där man varken får slakta åsnor när man gör film i Trollhättan eller köra bil hem från krogen, trots att man är för full att ta sig hem för egen maskin.

}

\small{
\textbf{Dansk påsk}
\label{a4f2d20941af07497959c5f11db47ac2}
 Inga konstigheter här. Man samlas, lyssnar på AC/DC och dricker sig full som vanligt. Kanske kollar man på TV.

 HEAD2: Traditioner
 Traditionsenligt knäcker dansken försiktigt ett ägg i båda ändarna, blåser ut innanmätet på garageuppfarten och
 fyller ägget med något som inte ska vara därinne. Sedan tappar man bort det för att man är lite på lyran och
 skäller ut någon för att ha snott det. På långfredagen mäter man varandra och den som är längst vinner en
 grenkabel att ha i förrådet när man använder borrmaskinen.

}

\small{
\textbf{Dansk sax}
\label{7d6cec14a694d6dbdcfa38bea3184a68}
 Att ta en kniv i varje hand och med hjälp av dessa klippa av något. Eftersom det saknas någon ledig hand som kan hålla i det som ska klippas itu brukar det sällan bli särskilt rakt. Det är dessutom mer regel än undantag att det stänker en del blod när man klipper med dansk sax. Men eftersom allt i Danmark \textsc{(s.~\pageref{5331d7fd27772396f412a5b6d19bad44})} är batikfärgat så gör det inget.

}

\small{
\textbf{Dansk semester}
\label{d23078bd5d0733edeb88fccd11d196e1}
 Sjukskriva sig och hänvisa till plötslig magsjuka.

}

\small{
\textbf{Dansk skalle}
\label{597900602d20c1206183c6e7bbcdcb91}
 En dansk skalle är en teknik vid handgemäng som går ut på att med ett kraftigt ryck med kroppen
 slunga sin panna i riktning mot sin motståndares huvud eller torso. Det gör ont.

}

\small{
\textbf{Dansk skrock}
\label{d7d9385790cae36226070ef73df98013}
 Även haschrökande \textsc{(se hasch s.~\pageref{1e93612a55f48e5fd9cbce22d0e71944})} nakennihilistister drabbas ibland av föreställningar som saknar logisk mening men ändå framstår för individen som av naturen givna. En vardaglig handling som att snatta mat till middagen kan hos en skrockfull dansk upplevas ge närmast övernaturliga konsekvenser om den inte utförs på rätt sätt. Om fenomenet ska ses som en folklig form av religion eller bara är resultatet av utbredd tillgång till kemiska droger bråkar den danska forskarvärlden fortfarande om (LSD framställs ju som bekant i laboratorier så forskarvärlden har mycket annat att ta i tu med först). Skrock är dock vanligare hos äldre, vilket talar för att det hänger ihop med långt gången delirium. Om du mot förmodan skulle befinna dig i Danmark \textsc{(s.~\pageref{5331d7fd27772396f412a5b6d19bad44})} kan det vara bra att känna till att nedanstående fenomen inte är något att oroa sig för. Det är inge dåre som gör detta, det är bara en vanlig skrockfull dansk:

 \begin{itemize}
 \item Lägg inte nycklarna på bordet. Då kan det hända att ägaren tar tillbaka dom.
 \item Om du ser en svart katt gå över vägen får du sju års bakfylla om du inte kastar katten över din högra axel.
 \item Om du hör en låt med Kim Larsen måste du dricka fem bärs.
 \item Knarka alltid tre gånger innan du går in genom kyrkporten.
 \item Om du krossar en spegel måste du krossa en till så det inte blir ojämt.
 \item Går du under en stege är du tvungen att gå passgång ända tills du somnar av utmattning.
 \end{itemize}

}

\small{
\textbf{Dansk tubkikare}
\label{958512f85ec4573e83995a4f6f95e137}
 Kika ner i Tuborgbuteljen \textsc{(se Tuborg s.~\pageref{49bb0f04b9993881c9d9c5b115cc35f0})} \textit{and Bob's your uncle}, som engelsmännen säger.

}

\small{
\textbf{Dansk ölkällare}
\label{ce1b72fead99074ab78d67282a29ff40}
 Källarrum i danskt hus. Ofta placerat under ölköket, öltoaletten, ölsovrummet och/eller Tuborg-vardagsrummet.

}

\small{
\textbf{Danska hedersbetygelser}
\label{799941a6e98a1446da72ff5483c6503d}
 Som alla andra nationer som vill framstå som civiliserade har Danmark \textsc{(s.~\pageref{5331d7fd27772396f412a5b6d19bad44})} hederstitlar som kan tilldelas särskilt framstående medborgare. Detta för att ge en stärkt känsla av den nationella gemenskap ingen dansk brydde sig om förens man började skylla alla problem på invandrarna. Utmärkelserna delas ut av De Kungelige Videnskabernas Sammensværjning som sammanträder varje lördag på Længelands nudistcamping. Ledamötena utses av Folketinget \textsc{(s.~\pageref{3e67fc9ae7590cbd0067b5015308da46})}. Problemet är som så ofta annars i Danmark att det sällan är någon som lyssnar på vad överheten \textsc{(se driva med överheten s.~\pageref{d2a6f08601a27a11e9a798bee876ee06})} har att säga så många är dubbade utan att veta om det. Torben Ulrich gick till exempel runt i flera år på Strøget utan att veta om att han belönats med hedersbetygelsen \textit{Sængrøgere} och det tillhörande Arne Jacobsen-stipendiet på 10.000 kapsyler. Titlarna är personliga och kan inte överlåtas på någon annan eller ärvas av avkommor.

 HEAD2: Danmarks finaste hederstitlar
 .]]
 \begin{itemize}
 \item \textit{Sængrøgere av guds nåde} - För mer än 40 år i frihetens tjänst (högst oklart vad som faktiskt menas med detta).
 \item \textit{Pilsnerdræng}/\textit{Fæbodjænte} - För nit och särskild redlighet när Roskildefestivalens bajamajor ska tömmas i havet.
 \item \textit{Olsen brother}/\textit{sister} - Tilldelas den som lyckats dricka mer än 5 000 bärs under ett kalenderår.
 \item \textit{Jesus av Læsarett} - Tilldelas den som lyckats dricka mer än 10 000 bärs under en kalendermånad.
 \item \textit{Brillerøv} - Föräras den som gått till bibblan och läst en hel bok utan bilder.
 \item \textit{Videnskabens bedste ven} - Nedsträcks till den dansk som kommit på något smart som gör att det känns socialt accepterat att glida omkring planlöst med en starköl och en halvbrunnen John Silver i ena näven och skjuta en barnvagn i den andra, utan att behöva ta ansvar.
 \item \textit{Tubsokker fremstillet af guld} - För att du kan sjunga texten till AC/DC-dängan \quotetext{Sink the Pink} samtidigt som du kör trampbåt själv, full.
 \item \textit{Legokong in absurdum} - Tillskänkes endast basister i D.A.D som gör något fantastiskt för landets femte stad - Esbjerg.
 \item \textit{Bestefar} - Eftersom endast en liten minoritet av danskarna har en känd fader belönas varje år av okänd anledning en framstående fritidsledare med ett trepack tubsockor.
 \item \textit{Simborgarmärket} - Alltsedan 1600-talets häxförföljelser åtnjuter simkunniga danskar stor respekt och vördnad.
 \item \textit{Hederslegionella} - Tilldelas var femte år en framstående utlänning. Mestadels har titeln tillfallit holländare som ådragit sig Syfilis \textsc{(s.~\pageref{6ef63d9f4b2e7a8686a4900dbb206a54})} och Gonorré under Roskildefestivalens dasstömmning.
 \item \textit{Rasmus Klump-priset \textsc{(se Rasmus Klump s.~\pageref{eac88a6def9b9f47888e7e3b62719cf1})}} - Ges den som på ett tydligt sätt uppvisar de egentskaper som serien förespråkar. Alltså piprökning och att gömma saker i munnen.
 \item \textit{Legoglands Mæsterdræng} - Tilldelas män som sköter sig när de besöker Legoland \textsc{(se Lego s.~\pageref{3a22c9ea9a3039d180e0a514a5a3b619})}.
 \item \textit{Professor/professora Flatologus} - Utdelas årligen av flatologiska sällskapet vid Köpenhamns universitet till den dansk som visat stor initiativkraft inom praktisk flatologi \textsc{(s.~\pageref{f56f2b86c16ace6216b4652c62b7afdc})}.
 \end{itemize}

}

\small{
\textbf{Danskt band}
\label{9b1945158e68bd4f89d0cec9ec05da46}
 Va fan är det för nåt? Är det vräkiga rockband som D.A.D? Eller är det stökiga, nihilistiska punkare som Hjertestop? Eller är det finstämda och sentimentala orkestrar, som Gasolin? Nej, nej, nej, det är betydligt tråkigare än så. Danskt \textsc{(se Danmark s.~\pageref{5331d7fd27772396f412a5b6d19bad44})} band är ett sätt att binda böcker som är typ som en häftad bok fast den har flikar på insidan av pärmen. Varför det heter danskt band är väl antagligen för att det är så danskar binder böcker. Konstigare än så är det inte.

}

\small{
\textbf{Danskt penicillin}
\label{2541daef615330e8fd4d7e5ac150c955}
 är en potent cocktail bestående av smørrebrød, brännvin och prostituerade. Precis som det penicillin som används i resten av världen uppfanns det danska penicillinet av en slump. Till skillnad från vanligt penicillin inträffade dock inte slumpen i ett laboratorium, utan i ett skjul utanför Palads København, där en stor orgie tog plats för att fira premiären av \textit{Olsen-banden på spanden} (1969). Niels Bohr, som deltog i firandet, skrev i sin dagbok morgonen därpå: \textit{Det vaer meg en ufaddebar kraft sem stremmede rättopp lekamen min... jeg kende meg likt Anders And om han pulet sine andfingrer oppi ett blixthul} (danska för eluttag)\textit{... ufaddbaert kraeftful!}

}

\small{
\textbf{Dating på nissepedia}
\label{3a0ebc282c39b9cf8572aeeda410d1d3}
 Det har kommit mycket reklam för datingsidor på Nissepedia den sista tiden. Det hela kan skapa en del irritation, men framförallt känns det oerhört missriktat och fullkomligt meningslöst. Har de inte förstått att Nissepedia redan är en datingsida, och inte nog med det - det är datingsidornas datingsida!

}

\small{
\textbf{Datingsidor på nätet}
\label{583f818c0bafa27297cd8f410f95adc3}
 För den som har retats med sin föredetta partner så till den milda grad att denna packat sin kappsäck och bosatt sig i storstan har altruister skapat datingsidor på nätet. Sådana sidor fungerar ungefär som kontaktannonser. Där kan man framställa sig själv som en blandning av den man vill vara och den man dessvärre är, minus dålig hållning, tveksamma sexuella preferenser och koleriskt morgonhumör. Lägg märke till att det är viktigt att liksom \quotetext{sälja in} sig hos de som besöker sidan. På så vis skapar man intresse hos det kön man önskar ligga med. Tidigare sköttes datinglivet medelst massfax \textsc{(s.~\pageref{236c3b7f761221f195b428aca2f06c4b})}, men nu används hemsidor eftersom det är mer miljövänligt.

 HEAD2: Inspiration
 För den som önskar skriva en annons på en dejtingsida på nätet följer här några användbara fraser som det går bra att klippa och klistra in lite hur som helst, men var noga med att fraserna du väljer inte motsäger varandra:

 \begin{itemize}
 \item Partiledare och föredetta nazisympatisör söker fruntimmer att para sig med
 \item Jag är en glad och pigg tjej \textsc{(s.~\pageref{8ad47065f969f6c49b3a91a6dc8daf6f})} på 55 jordsnurr..
 \item Skadeglad försäkringtjänsteman söker skadeglad försäkringstjänstekvinna
 \item Se på mig! Jag har numetal-byxa med dragsko! \textsc{(se dragsko s.~\pageref{0d3beb9223700e39e09040e9bbd3644b})}
 \item Jag jobbar på konsum \textsc{(se konsumbutik s.~\pageref{70e4875f7c2c177596305006e46b7ca9})} och älskar att läsa Margit Sandemos \textsc{(se Margit Sandemo s.~\pageref{d4a62753375ff2e975534b9ca740fd28})} böcker om isfolket.
 \item Var är min själsfrände? undrar innebandyspelare med fetor ex ore \textsc{(s.~\pageref{d3b96d618fb972d12fb0cdfdeaf13a98})}
 \item Man med lovikavantar önskar träffa en kvinna \textsc{(s.~\pageref{9a7760b2521c3471c47cd5d789a2d324})} - jag har inga krav, vilken som helst går bra
 \item Jag älskar att gå långa skogspromenader och att sitta framför brasan och sippa på ett glas fin päroncognac, med dig!
 \item Giktdrabbat burdust charmtroll till karl söker kona som vill ha något rejält att hålla i när det blåser upp till storm.
 \item Stigmatiserad bärshagga söker vad som helst med tillgång till Kir \textsc{(s.~\pageref{002e1a6e54da86cabc77fbb474c2df49})}. Endast seriösa svar.
 \item Harmynt bibliotekarie söker överviktig hålkortsoperatör \textsc{(se hålkort s.~\pageref{e361854e951ea42015b029ab30581844})} att resa till Aarhus med
 \item Smugglar du placebo-ryssfemmor genom Sápmi? Utklädd?
 \item Jag är en välhängd artonåring Warhammer-champion som söker kul umgänge utan förpliktelser
 \item Farbror med glasögon önskar träffa en ofattbart stor tant
 \item Chips Kisbye-fantast söker Chips Kisbye att älska i nöd och lust
 \item Före detta manisk \textsc{(se mani s.~\pageref{07cd55c7b42715ec44c133a6a165e8d2})} myntsamlare söker annan nyfrälst jukeboxentusiast
 \item Gillar du att andaktsfullt betrakta stjärnorna en molnfri \textsc{(se moln s.~\pageref{9da1014bea9aa67f9cae12e619d34aae})} natt? Det gör inte jag. Jag avskyr att andaktsfullt betrakta stjärnorna en molnfri \textsc{(se moln s.~\pageref{9da1014bea9aa67f9cae12e619d34aae})} natt. Ring mig.
 \end{itemize}

 HEAD2: Fraser att använda vid svar på annonser
 \begin{itemize}
 \item Jag är en oförbätterlig återfallsförbrytare och jag undrar om du doftar av blomster
 \item Jag är kanske inte så snygg (145cm/98kg) men jag har en murad enplansvilla med varmgarage
 \item Har du tänkt på hur konstigt det är att en åtta \textsc{(s.~\pageref{6fa68b0d02ec525fa72a51c13e5e3ed1})} är en sorts knut? Det tänker jag på jämt!
 \item Om du bodde i mitt grannskap skulle jag titta in genom ditt fönster varje natt, för du är så snygg.
 \item Ja, jag klär ut mig och smugglar ryssfemmor!
 \item Jag är så snål att jag alltid har persiennerna nere för att tapeten inte ska bli solblekt. Jag utlovar inga som helst överraskningar.
 \item Vill du mata mig som vore jag en fågel? Svar till: Carlos \textsc{(se Användare: Prof. Etienne s.~\pageref{a9878d2280e5a39becac8f73d113df91})}
 \item At some point or another I want to stop and get hold of a child, to tutor it in the ways of righteousness and procure some uncontaminated urine.
 \end{itemize}

}

\small{
\textbf{De gamla grekerna}
\label{4a5fb3d6ce79b5ff43b33f8f7d843672}
 När man talar om De gamla grekerna avser man sällan Stavros, Papadopoulos och Amos som sitter och lirar backgammon utanför den lokala kiosken. Snarare avses ett vilt gäng gubbar som härjade runt medelhavet några hundra år före kristus \textsc{(se jesus s.~\pageref{110d46fcd978c24f306cd7fa23464d73})}. Platon, Sokrates och Aristoteles är de man mest snackar om när det handlar om gamla greker. Platon var jätteelitistisk och snackade hela tiden om att allt var fake i den fysiska världen och att allt var mycket mer true i vad han kallade \quotetext{den äkta verkligheten}, dit ingen fick komma utom han och kanske Sokrates. Sokrates är mest känd för att han klagade på ungdomar innan Elvis hade juckat sitt första juck och för att han tog livet av sig genom att svepa en giftbägare, vilket man måste medge är extremt rått. Aristoteles ville katalogisera allt i hela världen. Han uppfann autismen. Aristoteles benämndes under medeltiden sällan med namn utan oftast kort och gott som Filosofen, alltid med stort F. Detta, får man anta, skulle ha gjort honom mycket, mycket nöjd och antagligen helt outhärdlig att umgås med. Detta skulle dock ha straffat sig på den tiden då de gamla grekerna även uppfann hybris och att man dör om man får sådan.
 Det finns många fler saker som de gamla grekerna var först med.

 \begin{itemize}
 \item Poeten Sapfo uppfann hela subkulturen \quotetext{svår poetlesbian} på sin ö, Lesbos.
 \item Archimedes uppfann att det bubblar om man fiser i badet. Han var även den första människan som drömde om att använda robotklor och värmestrålar som vapen.
 \item Diogenes var den första hippiecrustaren. Trots att han kom från en förmögen bakgrund (farsan var bankir, vettu) valde han att bo i en gammal tunna som han squattade. Han uppfann Kynismen (idag cynismen), som senare skulle komma att utvecklas till den flummigt sköna idétraditionen Stoicismen som typ går ut på att man ska ta det lugnt. Under sin levnad rackade han ner på etablissemanget (på den tiden Alexander den store) och Platon. Hur han dog är omdiskuterat, men enligt wikipedia ska han ha dött antingen av att ha käkat rutten bläckfisk (antagligen dumpstrad), ha blivit biten av en äcklig hund eller helt enkelt hållit andan i protest tills han dog. Crust as fuck existence \textsc{(s.~\pageref{bd0b07abcc2f4c2a4e1aafdfed1f0e73})}.
 \item Demokritos uppfann sitt namn till trots inte demokratin. Det var det en annan gammal grek som gjorde. Demokritos uppfann tanken om att allt består av små små bitar tills det tillslut inte går längre. Då har man den minsta biten. Atomen, alltså. På senare dar har det visat sig att det vi först kallade atomen inte var odelbar (som ordets ursprungliga innebörd, odelbar, antyder). Men det gör inget, då redan de gamla grekerna hade fel.
 \item De gamla grekerna uppfann även misogynin, då man tidigt hävdade att brudar var missbildade lägre stående varianter av män som borde låsas in. Därom tvistade sällan de lärde hos de gamla grekerna. Tankarna om att brudar sög var ofta vad som enade filosofer som Platon och Aristoteles som annars ofta tyckte att det den andra tänkte var helt bananas \textsc{(s.~\pageref{ec121ff80513ae58ed478d5c5787075b})}. Man bör notera att detta, precis som idag, inte hindrade grekgubbarna från att ligga med brudar. Det tyckte de var nice.
 \item Författaren Homeros skrev en bok som heter Odysséen som är den första kända källan där korv omnämns.
 \item Aischylos uppfann både tragiken och slapstickhumorn, båda i och med sin död. Aischylos dog genom att en \quotetext{örn} (antagligen en uv) \textsc{(se uv s.~\pageref{45210da832f9626829457a65e9e7c4d0})} släppte en sköldpadda i huvudet på honom från hundratals meters höjd. Eftersom han var en väldigt älskvärd grek resulterade hans död i tragikens uppkomst. Innan sin död skrev han tråkiga pjäser om vikten av försoning.
 \item Epikuros kom på idén med att undvika att göra jobbiga och svåra saker och istället hänga med polarna och fixa sig en sjysst dagsfylla \textsc{(s.~\pageref{e79459471993abd0ccde4df08bafdb22})}.
 \end{itemize}

}

\small{
\textbf{De stora bot-attackerna 2011}
\label{c7dd995dad0d892085806b68800cca79}
 Det fria ordets sista utpost, Nissepedia \textsc{(s.~\pageref{62400dadecd90cb5cd39062abe5a3e4a})}, attackerades hösten 2011 av botar \textsc{(se bot-fan s.~\pageref{aabae9c99305dd2cf1f7cb4e8bc22be2})} med syfte att sätta munkavle på detta frihetens uppslagsverk, genom att fylla det med skit om datingsidor på nätet \textsc{(s.~\pageref{583f818c0bafa27297cd8f410f95adc3})}. Detta motverkas genom ingrepp från Nissepedias huvudsakliga torpeder, Rainbow Riders \textsc{(s.~\pageref{54b5b4739e6bc150148c5019e1793413})}.

}

\small{
\textbf{Deadhead}
\label{30563e7c77afbd00a3aafa07829c95d3}
 kallas den person som är ett extra stort fan av den amerikanska rockgruppen Greatful Dead. Deadheads finns i tusental över hela världen och reser tillsammans runt för att se Greatful Dead framträda live. Många åker traktor eller cyklar till konsertera, eftersom det känns mest \quotetext{psychedelic}. Men är det jättelångt händer det att deadheadsen tar tåg.

 Det finns ingen formell antagningsprocess så vem som helst kan bli ett deadhead när helst den vill. Men om man kallar sig det utan att kunna jättemånga låttexter utantill och ha sett Greatful Dead live jättemånga gånger lär man få svårt att få tillträde till den inre kretsen. Det finns nämligen flera hårdföra subgrupper inom deadheadsen. Wharf Rats är deadheads som hjälper varandra att hålla sig drogfria under Greatful Dead-konserter, det är nämligen inte alltid så lätt. Deras raka motsats kallas Wookiees, som är deadheads som lever ungefär som träskpunkare \textsc{(s.~\pageref{484838b3db1adb135ea74d6fc61e44c0})}, vilket är väldigt lätt. En rapport från FN-organet WHO visar att det om några år beräknas finnas fler deadheads än blonda personer i världen.

 HEAD2: Kända deadheads
 Följande personer har själva eller av media deklarerats vara fanatiska deadheads:
 \begin{itemize}
 \item Tony Blair
 \item Bill Clinton
 \item Walter Cronkite
 \item Whoopi Goldberg
 \item Greg Ginn
 \end{itemize}

}

\small{
\textbf{Delfin}
\label{a62b1fca6b53d6670a84aa2c7b373b27}
 Överskattat djur som ingår i djurfamiljen valar. Har en blågrå färg och töntigt utseende. Det finns dokumenterade fall där delfiner har räddat människor från hajar, vilket känns extremt taskigt mot hajarna, som precis som alla andra djur måste få äta sig mätta. Delfiner har också ett konstigt läte och begynnande flint. Deras könsvätska är också basen för vegansvullet tofu \textsc{(s.~\pageref{5df7f1701b778d03d57456afea567922})}.

 Det är inte ovanligt att nyhippies, som fått för sig att det skulle vara en underbar och perspektivvidgande upplevelse att bada naken i månskenet med delfiner, blir brutalt gruppvåldtagna av de små gynnarna. Att delfiner enligt samma grupp av människor är den näst smartaste livsformen på vår planet gör bara övergreppen otäckare.

 Delfiner bör inte förväxlas med den mycket snyggare släktingen Helfinen \textsc{(s.~\pageref{3098a79b785d7db5e8f9e3c099db4d2c})}.

}

\small{
\textbf{Delfinapa}
\label{c87800dbc123aa38ce894fa9296afa5e}
 En delfinapa är en apa med en delfins \textsc{(se delfin s.~\pageref{a62b1fca6b53d6670a84aa2c7b373b27})} hjärna. Den är stark, smidig, snabb och farligt intelligent. Liksom den globala virus-pandemi som epidemologer länge fruktat ska bryta ut i vår globaliserade värld, men som ännu inte gjort det, har vi hittills, tack och lov, varit förskonad från delfinapan. Den förblir än så länge en evolutionär farhåga, men bör tas på stort allvar. Delfinapan skulle nämligen snabbt bli ett säkerhetsproblem på grund av sin kroppskapacitet och sitt instrumentella tänkande.
 HEAD2: Delfinapor i kulturen
 Nicholas Cage är lite som en delfinapa, fast inte lika smart.

}

\small{
\textbf{Den ambitiösa studenten}
\label{02257ef6d6da8e0f0721e2758eec3c71}
 tänker sig en framtida akademisk karriär och ägnar sig, till skillnad från hans eller hennes klasskompisar, åt att studera flitigt istället för att ränna omkring på nationer och dylikt. Den ambitiösa studenten läser Adorno, Foucault och Butler och imponerar på klassen och läraren med att säga sådant som \quotetext{men Foucault skriver att...} och så (trots att mannen i fråga varit död i snart trettio år). Den ambitiösa studenten har inget till övers för sexism och rasism. Identitetspolitik är den ambitiösa studentens shit. Den ambitiösa studenten lyssnar på samisk jazz och kvinnliga singer/songwriters och sitter uppkrupen som en katt \textsc{(se \quotetext{krypa ihop i soffan som en katt} s.~\pageref{4d1d36fa3c68844e34049d8b7db95af2})} i kuddhögen på folkkök \textsc{(s.~\pageref{15983d1934522d4d08e766108357201b})} och värmer händerna på en kopp chai-té. Att förakta facebook \textsc{(s.~\pageref{26cae7718c32180a7a0f8e19d6d40a59})} är självklart för henne/honom, men han/hon har ändå en facebooksida för att kunna få nyheter om Ship to Gaza och djurens rätt. Den ambitiösa studenten är så trött på alla gubbar inom akademin och vill vara med och förändra. Den ambitiösa studenten borde sättas på ett tåg till Sibirien, men då skulla han/hon bara se det som ett spännande tillfälle att lära sig mer om vår stora granne i öst eller som ett sätt att hitta material till exjobbet.


 HEAD2:  Se även:
 Den oambitiösa studenten \textsc{(s.~\pageref{773fb9013bfd8af98ed84fe0abc8748e})}
 Overallstudenter \textsc{(s.~\pageref{09a5062cf884d746996bf5a9f3669d1b})}
 Utbytesstudenter \textsc{(s.~\pageref{397699f3732b0c22f3c532a111697539})}

}

\small{
\textbf{Den arga groggen}
\label{f33988faebdb8ba973fc6da28d05ac62}
 , också känd som busgrogg eller rallartoddy, består av starköl som groggvirke \textsc{(s.~\pageref{ba264d4eb820b4066de4c8723a08f824})} blandad med brännvin \textsc{(s.~\pageref{ff49ececa32cff978496a39635496f46})} av gott märke. En lär bli på väldigt dåligt humör av denna.

}

\small{
\textbf{Den gråtande gitarren}
\label{1b7844b391737a2875d206c396f15f0f}
 är en ballad skriven av den spanske smörsångaren Erique Gunnarzalez. Musiken går i moll och präglas av typiska spanska tongångar. Texten handlar om Eriques förtvivlan över att hans kära mor bor på andra sidan av ett berg. Varje morgon vaknar han och önskar att han var vid hennes sida, men allt han har är sin sorg och sin gråtande gitarr \textsc{(s.~\pageref{a08bf8420208934bc59c7ed7385d4308})}. Låten kom på sista plats i 1978 års Eurovision song contest.

}

\small{
\textbf{Den lilla boken}
\label{ed0a870040c3e0e34442133e5ca9cb29}
 I internationell fotboll \textsc{(s.~\pageref{961bd74d34872ff94a4df3a16119096e})} på toppnivå händer det vid misslyckade frisparkar eller långskott att skytten tittar på sina lagkamrater och gör en gest med handflatorna som att  spelaren öppnar en liten pixibok och visar den för laget. Genom att göra så vill spelaren signalera att han är missnöjd med sitt skott och hade bett om ursäkt till kamraterna om det hade varit så att det på något vis var hans fel (detta är endast förekommande i herrfotboll) och inte en olycklig kosmisk slump. Kopplingen mellan situationen och den lilla boken är oklar men har något med medeltida katolsk skolastisk lärdom att göra.

}

\small{
\textbf{Den nordiska smuggeltriangeln}
\label{293ad8d2485f81edf32412f50c88c7b3}
 är namnet på det ekonomiska ekorrhjul som håller de svarta marknaderna i Norge, Sverige och Danmark snurrande.

 HEAD2: Kort historielektion
 Som alla vet har allt roligt alltid varit förbjudet i Norge. Förutom en halvtimme Fleksnes på lördagskvällarna finns det inget som inbjuder till skratt eller njutning. I grannlandet Danmark \textsc{(s.~\pageref{5331d7fd27772396f412a5b6d19bad44})}, å andra sidan, har det alltid varit lördag \textsc{(s.~\pageref{8d203c09d6ebbc3a0d797e14178798a0})} hela veckan och alla har varit fulla och tittat på sengekantsfilm \textsc{(s.~\pageref{36f1eab94ebc00e11292cfaa67acafa0})} så länge att man glömt bort vad ordet tråkigt ens betyder. Dessa förhållanden gav givetvis upphov till en aldrig sinande avundsjuka. Norrmännen ville så klart också supa och gruppeknalla, och i Danmark blängde man missunsamt på lusekoftorna av varm fårull (eftersom hela Danmark partajas sönder till en leråker en gång om året är grisar numera de enda boskapsdjur man klarar av att hushålla. Och eftersom dansk ekonomi till stor del bygger på export av pörrfilm \textsc{(se pörr s.~\pageref{5faa435e2f0af7617816f0cade262581})} med djur i är får och getter hett eftertraktade kritter). Så där höll man på och glodde missunsamt på varandra fram till 2009 när någon förståsigpåare \textsc{(s.~\pageref{ff91afb86ce86124b6a517f3eb37bc18})} med napoleonkomplex i Bryssel fick för sig att förbjuda riktiga glödlampor.

 HEAD2: Nutid
 Sverige har alltid hållit sig utanför det dansk-norska ställningskriget och istället fokuserat sin kraft på att försöka återta Finland \textsc{(s.~\pageref{631d44eaa1254ff71a1e11ba021d1266})}. I och med glödlampsförbudet kom dock denna historiska oförrätt att hamna i skymundan (no pun intended) och Sverige letade istället desperat efter ett sätt att lösa den stundande krisen med svinjobbiga lågenergilampor som tar tio minuter att tända och genererar huvudvärk. Om man ska säga något positivt om Norge så är det att dom inte är med i EU \textsc{(s.~\pageref{4829322d03d1606fb09ae9af59a271d3})} så där kunde man lugnt fortsätta att läsa sina biblar i skenet av glödande grafit (i Danmark sket man fullständigt i denna fråga eftersom allt ljus där, utom solens, kommer från brinnande soptunnor). Ur denna kris växte dock en givande handelsmarknad fram där de tre länderna på ett solidariskt sätt täckte varandras behov. Norge fick dansk fulsprit att spy upp på Karl Johan, Sverige fick norska glödlampor att sätta i lampan ovanför kökssoffan \textsc{(s.~\pageref{d1c2d6488fde9b41b5c6b2a03c5fd79c})}, och Danmark fick svenska får till sina porrfilmer.

}

\small{
\textbf{Den nya sångaren}
\label{2e55dbe6a48745ced354e0dd04dd4b80}
 heter Brian Johnson och spelar i AC/DC (tidigare Geordie). Han tog över efter Bon Scott 1980 då den senare gått och dött av att spy sig själv i munnen \textsc{(se mun s.~\pageref{6585f290ce92c3de5ff339920330e26f})}. Den nya sångarens kunde först höras på albumet \textit{Back in Black} (1980), men är inte lika bra som Bon Scott. Den nya sångaren bör ses på med 90\% skepsis och 10\% överseende.

}

\small{
\textbf{Den oambitiösa studenten}
\label{773fb9013bfd8af98ed84fe0abc8748e}
 är till skillnad från sin kusin, den ambitiösa studenten \textsc{(s.~\pageref{02257ef6d6da8e0f0721e2758eec3c71})}, inte motiverad. Studenten kan inte be bothered. Han/hon har just flygit ut ur sitt medelklasshem och vill egentligen hellre syssla med sin musik än att gå och lyssna på nån gubbe eller kärring som står och pratar om Egyptens koptiska minoritet eller dackefejden \textsc{(s.~\pageref{0aa3c03182b8c3edaf72541cd88161ff})} eller vad som nu står på schemat. Men ännu har DJ-karriären inte riktigt tagit fart. När studenten ändå tar sig till dessa tråkiga föreläsningar orkar han eller hon inte lyssna på alla tråkiga saker som lärs ut på den självvalda kurs han eller hon går utan använder sin iPhone4 till att kommentera Kalle \textsc{(se Kalle anka s.~\pageref{64db68f686a0ca4d9d641061cb3fdf13})} och Stinas facebookuppdateringar. När tentan kommer skickar den oambitiösa studenten ett mejl till läraren och förklarar att han/hon har haft så mycket att göra, så trassligt på relationsfronten och så oerhört svårt att få tag på kurslitteraturen att han/hon undrar om han/hon inte kan få göra tentan en annan gång. Studenten skriver otaliga brev och sitter i telefonsamtal med studentkåren för att på något vis tvinga institutionen där han eller hon pluggar att ge honom/henne ett betyg även om han/hon kanske \textit{tekniskt} sett inte har gjort \textit{alla} uppgifter som står i kursbeskrivningen. Det är ju trots allt inte studentens fel att föreläsningarna inte är roligare och att han eller hon har så svårt att komma ihåg information om så oviktiga och tråkiga saker! Om studenten inte har lärt sig något är det väl lärarens fel, för det är ju ändå lärarens jobb att lära ut saker, eller hur? Vidare kan det väl inte rimligen vara studentens ansvar att själv skaffa sig böcker som redan är utlånade på universitetsbiblioteket? Studenten funderar stundtals på att åka till London och arbeta i krogbranschen, skaffa sig erfarenhet och kanske lite kontakter inför den framtida musik-karriären. Studenten känner sig färdig med Sverige \textsc{(s.~\pageref{b1999637949ed135b2ca03f3a38460cc})}, som inte längre har något att erbjuda. Här är alla så \textit{square}.
 HEAD2:  Se även:
 Den ambitiösa studenten \textsc{(s.~\pageref{02257ef6d6da8e0f0721e2758eec3c71})}
 Overallstudenter \textsc{(s.~\pageref{09a5062cf884d746996bf5a9f3669d1b})}
 Utbytesstudenter \textsc{(s.~\pageref{397699f3732b0c22f3c532a111697539})}

}

\small{
\textbf{Den sjunde vågen}
\label{e52c70c809c8a01e091e35ffee8a1934}
 är lite större än alla andra vågor. Det finns de bakåtsträvare som insisterar på att saker och ting är mer komplicerade än så därute i verkligheten. Men det har de något bevis för det?

}

\small{
\textbf{Den tyska mustigheten}
\label{682ccd5fdc3aff0c97e8845c3d6b6ca8}
 har på oräkneliga vis påverkat Europas kulturhistoria och är dessutom föremål för mycket skratt och hån utanför Tysklands \textsc{(se Tyskland s.~\pageref{b1b58da783b6d5fa090f3015f1889869})} gränser, framförallt i Skandinavien. Den tyska mustigheten har lett till två världskrig och består i filosofiskt grubbleri av sådana som Kant, Leibniz, Nietzsche, Marx och Hegel \textsc{(se Friedrich Hegel s.~\pageref{eadd964c0c4d2479bfe20e49c8921e77})}, sådana ansiktsfrisyrer som dessa uppvisa, byggnadsverk så som Branderburger Tor i Berlin, lederhosen, apfelstrudel, Freikörperkultur \textsc{(s.~\pageref{40cdc17a157b501b2c84835ce6204f9c})}, Wagner, marschmusik, Caspar David Friedrich och mycket mer. Hos den enskilde individen kännetecknas den tyska mustigheten av ett slags koleriskt storhetsvansinne \textsc{(s.~\pageref{2f9c0ea6231e1de87c97eab41410c795})}, märkliga klädval och användandet av oväntade assessoarer, samt skägg.

}

\small{
\textbf{Den vedervärdige mannen från Säffle}
\label{e6c829ebc03d7696483c60996b81e40b}
 Svensk kriminalroman \textsc{(s.~\pageref{7651f0db40825f508a14f8111a05711e})} som handlar om samma saker som alla andra kriminalromaner.

}

\small{
\textbf{Dennis}
\label{7daacea5f373b4c1c054158b126d317f}
 är ett namn som främst ges till busfrön och fetton. Ibland ges det även till korv.
 Även namnet på en av Saida Anderssons \textsc{(se Saida Andersson s.~\pageref{b415cf75bbb474ceaed1e38d2d637939})} söner.

}

\small{
\textbf{Der er et yndigt land}
\label{418be869b674227e9dc5121e261e68db}
 \textit{Der er et yndigt land} är Danmarks \textsc{(se Danmark s.~\pageref{5331d7fd27772396f412a5b6d19bad44})} nationalsång. Den första texten skrevs redan 1819 men har reviderats flera gånger när man glömt bort hur den gick. Spåkforskare tror att titeln från början var \quotetext{Der er ett Syndigt land} och att ett S föll bort när någon slarvade, men det är svårt att veta säkert eftersom alla danska arkiv eldas upp på valborgsmässoafton varje år. Från början var det 12 verser men det fattar ju vem som helst att ingen dansk orkar sjunga så långt. Här är den vanligast förekommande texten:

 Der er et yndigt land,
 det står med brede bøge
 nær salten østerstrand

 Det bugter sig i bygsans dal,
 den reser sig før Danmark
 og det er Frejas sal

 Den sad i fordums tid
 med højerhænden kæmper,
 en rødlætt pølse gå i strid

 Så drog de frem til fjenders mén,
 nu hvile deres bene
 bag højens bautasten

 Det land endnu er skønt,
 ollon blå sig søen bælter,
 og løvet smager så grønt

 Og ædle kvinder, skønne mø'r
 og mænd og raske svende
 runke på de danskes øer

 Hil kong Christian og fædreland!
 Hil hver en dannebroger \textsc{(se dannebrogen s.~\pageref{d8c97891c74597fa443ed507c4191fe0})},
 Alle er vi Anders And!

 Vort gamle Danmark skal bestå,
 så længe Gasolin spejler
 sin Lille du i bølgen blå



 Som läsaren märker består texten mest av mer eller mindre tydliga referenser till onani och droger, vilka båda är kärnämnen i dansk grundskola.

}

\small{
\textbf{Det Bättre Partiet}
\label{a7d2ad8a2039054d6b1010fcdd5ec989}
 har två saker på dagordningen.

 1. Höj Kebnekaises sydtopp så att den blir högre än Galdhöpiggen (Norges högsta berg)

 2. Tillåt PVC i cykelslangar.

}

\small{
\textbf{Det gamla Silvio Berlusconi-knepet}
\label{1774dae882395a1c00e5fbbfc281d7c3}
 är ett retoriskt grepp som används av slipade makthavare för att ge sken av en politisk nystart. På julafton iklär sig makthavaren (knepet används bara i länder där makthaverskor inte existerar på denna nivå) ett par vita y-front och lägger sin i ett åsnetråg \textsc{(s.~\pageref{1e0e0470206e0f2baad8e628ba8f770c})} och sprattlar och jollrar glatt likt ett spädbarn. Proceduren pågår tills alla blivit glada. Detta är en symbolhandling för att visa att makthavaren lagt det gamla bakom sig och är redo att börja om på nytt. Jozef Ratzinger använde framgångsrikt detta knep när illvilliga medier började anklaga katolska \textsc{(se katolik s.~\pageref{75d0665472f571956570c00f3fccfbc2})} kyrkan för att ha osunda åsikter. I Sverige är det främst renässansmannen Fadde Darwich som använt knepet för att markera en ny inriktning i livet, till exempel när han skulle bli ståuppkomiker \textsc{(s.~\pageref{3d6d423564dc06ac53646ac45691566f})} [http://www.youtube.com/watch?v=gagVmuW7iRM].

}

\small{
\textbf{Det gamla Thore Skogman-knepet}
\label{66ed380a23aff7c21f84f3bf78effb16}
 är en metod för att dölja fisar om man vistas i grupp. Metoden uppfanns av Thore Skogman när denne var magsjuk och hade blivit inbokad till ett direktsänt framträdande i \textit{Hylands hörna}. Thore märkte under intervjun i sändingen att han hade en fis i kroppen som var mycket angelägen att komma ut. För att inte göras till åtlöje inför hela svenska folket \textsc{(s.~\pageref{3ed7e29e1fe8e143fdb81bf836742e4b})} bröt Thore utan förvarning ut i sin älskade hitlåt \textit{Surströmmingspolka} och passade i refrängen på att lätta på gasen. Hyland märkte inget utan trodde bara att Thore glömt borsta tänderna och fått en släng av fetor ex ore \textsc{(s.~\pageref{d3b96d618fb972d12fb0cdfdeaf13a98})}.

}

\small{
\textbf{Det löser sig}
\label{63c0bfde8db77768a5100dc10ee77528}
 sa hon som sket på sig i vattnet.
 [http://www.youtube.com/watch?v=I9gejRo9ew0]

}

\small{
\textbf{Det omedvetna}
\label{d653b6e53612e79237853e2e4dfaf4a4}
 är en term inom psykologins historia, som idag är arkaisk men som har haft ett visst betydande för det kontinentala tänkandet om människans inre värld. I en liten bok med titeln \textit{Die Mench und der unbewusstheit} (1922) framlade den Österrikiske psykoanalytikern Hermann Taschenmesser en teori om människans psykologiska dimension i vilken han i det stora hela understödde Freuds teorier om överjaget, jaget och det undermedvetna, men lade till det omedvetna, som enligt Taschenmesser förklarade människors ignorans och generella dumhet. Överjagets kontrollerande funktion och det undermedvetnas funktion som lagerrum för undantryckta minnen tycktes inte bevisa att människor enligt Taschenmesser inte i tillräkligt stor utsträckning köpte och läste hans första bok, den självbiografiska \textit{Herr Taschenmesser - Ein schrecklich netter Mensch} (1916). Därför krävdes ännu en nivå i teorin om människans inre värld som kunde förklara detta för Taschenmesser obegripliga faktum. Det omedvetna förklaras som en kall och mörk plats i vårt inre där dumhet och oförstånd är förlagt. Det är enligt Taschenmesser ofta det omedvetna som ligger bakom sådant som att många människor inte köper och läser en fantastisk bok om den så ligger mitt framför dem och bjuds till försäljning för ett mycket generöst nedsatt pris.

}

\small{
\textbf{Det politiska i C.C.Rs texter}
\label{ccc008a82515a9d6022c2c205b15df5a}
 \textit{Det politiska i C.C.Rs texter} är en bok av författaren, nervvraket, den ensamme trebarnsfadern och geniet \textsc{(se storhetsvansinne s.~\pageref{2f9c0ea6231e1de87c97eab41410c795})} John Andersson. Det är en djupläsning av Creedence Clearwater Revivals texter, med fokus på det samhällskritiska. Man tänker lätt att detta genuina \quotetext{feelgood}-band hellre skrev om att ha kul, men allvaret stod ibland för dörren och då var inte The Fog \textsc{(se The fog s.~\pageref{576875ef0042ff21c04f5f1b9377d4e7})} sen att uppmärksamma detta.

 HEAD2: Fortunate Son
 En klassanalys av vietnamkriget. Foggan tyckte att de som vurmade mest för att starta kriget var de som inte skulle åka dit och kriga, det vill säga överklassen. Han tyckte att flaggviftande rikemansungar (Fortunate sons) med silversked i skitan gått kunde hålla käften, eller hänga på sig en M-16 och åka dit och \quotetext{göra skillnad}.

 HEAD2: Ramble Tamble
 Denna låt är bara samhällskritisk i den utsträckning som bluesmusik är det generellt. Inga pengar, ingen mat, lån att betala, räkningar att betala, då tar man sitt pick och pack och \textit{Ramble Tamblar} därifrån, en praktik som är vanlig i samhällen med en låg dekommododifierngsgrad. En intressant detalj är att Foggan redan 1971 förutspådde Ronald Reagans tillträde som president med textraden \quotetext{actors in the white house}. Om det amerikanska folket bara hade lyssnat.

 HEAD2: Wrote a song for everyone
 Här möter vi The Fog \textsc{(s.~\pageref{576875ef0042ff21c04f5f1b9377d4e7})} i hans mest cyniska, men kanske också sannaste, stund. Textraderna som det bör läggas vikt vid lyder: \quotetext{Saw the people standin' thousand years in chains. Someone  says it's diff'rent now, look, it's just the same}. Vilka kedjor som åsyftas är något oklart. Antingen är det svarta i U.S.A, slaveriet och kopplingarna till rasismen i dagens amerikanska samhälle, eller de kedjor som de härskande klasserna slagit på arbetare genom alla tider.

 HEAD2: Om boken
 \textit{Det politiska i C.C.Rs texter} är fyra sidor lång och kan laddas ner som pdf från Malå \textsc{(s.~\pageref{41da4620e87888eaaeafcb3004a8d177})} Sågverks \textsc{(se Sågverk s.~\pageref{39a99a78876fd85985cc06fa0baa3c1a})} hemsida.

}

\small{
\textbf{Det ryska babusjka-stuket}
\label{82b3ee065c37c75fdee61cdf1edd9705}
 är ett sätt att klä sig som härstammar från vår store granne i öst, Ryssland. Babusjka-stuket är vanligt bland den kvinnliga delen av den ryska befolkningen, särskilt bland äldre generationer och på landsbygden, men har också exporterats till andra öststatsländer och till lite \quotetext{jordnära} lesbiska kvinnor i Västerbotten \textsc{(s.~\pageref{d4b008c5143dcffb6b8c35f3876c2a19})} som tidigare ofta frekventerat hardcore-spelningar. Babusjka-stuket kännetecknas av attribut som:
 \begin{itemize}
 \item Sjalett runt huvudet \textsc{(se huvud s.~\pageref{e906cd95a540df9b16d0460fb4cf0adc})}.
 \item Lång rustik klänning.
 \item Förkläde.
 \item Träskor eller grova arbetsskor.
 \item Uppkavlade ärmar \textsc{(s.~\pageref{a86a08e7e2aca91a21350bc184e05367})}, om så krävs.
 \end{itemize}

}

\small{
\textbf{Det senmoderna samhället}
\label{1a4c3b1112bd2b510a8c47eff69397b8}
 Undergångens första fas, eller sista beroende på perspektiv \textsc{(s.~\pageref{1606dd19366985367d677f7b6de46e52})}.

}

\small{
\textbf{Det stora fosterländska kriget}
\label{8e55572fc7b7490da402e43a822eb3da}
 är den ryska benämningen på vad vi andra kallar det andra världskriget. Under detta krig räddade Sovjetunionen Europa från den tyska mustigheten \textsc{(s.~\pageref{682ccd5fdc3aff0c97e8845c3d6b6ca8})} och efter dess slut lät ryssarna bygga vad man i öst kallar \quotetext{den anti-fascistiska skyddsvallen} och i väst \quotetext{Berlinmuren.}

}

\small{
\textbf{Det stora vadet 2013}
\label{bc0acf1a58eb7de5b4401bd3bfa3a7f2}
 handlade om huruvida Marko varit i Finland \textsc{(s.~\pageref{631d44eaa1254ff71a1e11ba021d1266})} eller inte. Vad som började som en skämtsam diskussion mellan två vänner kom under kvällen att urarta i en skyttegravsträta där de tvistande parterna hela tiden höjde sina insatser. Den ena parten hävdade bestämt att Marko varit där, vilket enkelt kunde bevisas genom fotodokumentation från den finlandsresa de tre inblandade gjort tillsammans. Motparten å sin sida menade att något sådant fotobevis \textsc{(se Fotografering s.~\pageref{176551844874f34f5bb9a9d0ac93f99a})} inte existerade, eftersom Marko aldrig var med. Frågan kan för en utomstående tyckas enkel; det är väl bara att fråga Marko. Ett rimligt antagande i vanliga fall, men när insatsen blivit så hög som i detta fall måste alla yttre parter synas noga i sömmarna innan de kan tillåtas avlägga vittnesmål. Någon kommer trots allt bli ruinerad när korten ligger på bordet. Sista ordet är ännu inte sagt i frågan men följande står på spel:


 1 platta bärs
 10 timmar hemhjälp
 En prydlig slips
 En ståtlig hatt
 En FIAT Panda till ett marknadsvärde av max 3.000 kronor
 Två fiskedrag av märket Rapala
 En guidad tur (i naturen)
 Ett besök i valfri stad
 1 liter spolarvätska (koncentrerad)
 5 liter bensin i reservdunk
 Ronnys \textsc{(se Användare: Ronny s.~\pageref{c7fc87f27db026e1c60a6ac2cb1fd820})} gröna tennisshorts
 En demokassett med No Fun At All ELLER första sjuan \textsc{(s.~\pageref{7b7c558fc3f8d8557ba30b082e644ea1})} med Atomångest

}

\small{
\textbf{Det susar i Säfve}
\label{703137131c7827fbaef28e85f3cd8b43}
 \textit{Det susar i Säfve} är en älskad barnbok av den brittiske författaren Kenneth \textsc{(se Kennet s.~\pageref{eb251e3745d960e2100c5435a32764c4})} Greheme (Org.titel \textit{The Wind in William Carlos Williams}). Boken handlar om gasbildningar och olika sorters baksug i den norrbottniske \textsc{(se norrbotten s.~\pageref{0e8c003b75982032cde152609ee94154})} författaren och debattören Torbjörn Säfves matsmältningssystem. Boken har givit upphov till en hel massa adapteringar i form av teateruppsättningar, ballettföreställningar, stop-motionfilmer och till och med ett dataspel, i vilket spelaren försöker släppa på trycket i tarmsystemet genom att leda ut gaser genom magmunnen \textsc{(se Övre magmunnen s.~\pageref{b0fbb0780611129ae5fc27c88d23d8f3})} och andra kaviteter, så att Säfve inte plågas av knip utan kan koncentrera sig på sitt författarskap och sin anarko-stalinism.

}

\small{
\textbf{Det tar sig}
\label{aad4b9ff77d05fcee27cdd7be15a76a7}
 sa mordbrännaren.

}

\small{
\textbf{Det var bättre förr}
\label{c7d3f908ea2aaab4cab2730336769b70}
 \begin{itemize}
 \item Kiss sminkade sig
 \item Fackföreningar bedrev facklig verksamhet
 \item Chef var ett skällsord
 \item Musik avnjöts på vinyl
 \item Även de mest medelmåttiga nissepediainläggen höll hög klass
 \item Barnprogram gjordes medelst dockor istället för 3D-animationer
 \item Discharge gjorde asfeta skivor
 \item Umeås kommunalråd hette... visst fan... Lennart Holmlund redan då. Nåväl, kanske inte allt var bättre förr
 \end{itemize}

}

\small{
\textbf{Diagram}
\label{d08cc3195cba7dd812ab0652a68bdeda}
 . X-axeln representerar diagram och y-axeln antal. Notera att diagrammet inte tar hänsyn till antalet sheik och drottning-diagram \textsc{(se Diagram med sheiker och drottningar istället för staplar s.~\pageref{2fa3b0a486162c5aac1497b6c58f43f8})} på nissepedia]]
 Diagram är ett slags bild som gör att det blir relativt svårt att förstå något relativt tråkigt, till exempel antalet personer i ett medelstort svenskt län som är över femtio bast och kör Merzedes-Benz. \textit{Dia} betyder \quotetext{över} eller \quotetext{ut} och har språkhistoriskt att göra med geometriska figurer som ritas med hjälp av linjer. \textit{Gram} är ett vanligt viktmått när man köper hasch \textsc{(s.~\pageref{1e93612a55f48e5fd9cbce22d0e71944})}.

}

\small{
\textbf{Diagram med sheiker och drottningar istället för staplar}
\label{2fa3b0a486162c5aac1497b6c58f43f8}
 är ett sorts diagram som är väldigt ovanligt. Det känns igen på att de istället för staplar har bilder av sheiker och drottningar vars längd utvisar den data som kan avläsas av diagrammet.

}

\small{
\textbf{Dialekter}
\label{1e35ddb6700b133ed0f9f078815422b8}
\begin{enumerate}
\item REDIRECT
\end{enumerate}

}

\small{
\textbf{Dialektik}
\label{5c0ded4e9796ad82ecd11d1a0010bf6b}
 är ett logiskt förfarande som används inom retorik, filosofi, naturvetenskap och politisk teori. Framförallt ligger det till grund för Marx historiematerielistiska teori om historiens utveckling, klasskampen och kapitalets tillkommande. Enkelt uttryckt består dialektiken i en progression som är resultatet av bildandet av en syntes av en tes och en antites. Följande exempel illustrerar en sådan dialektisk progression.
 Två polare, vi kan kalla dem Karl och Friedrich, diskuterar vad de ska äta till kvällsfika. Karl föreslår varma mackor (tes), medan Friedrich prompt vill ha glass (antites). Denna motsättning får som resultat att Karl och Friedrich i slutändan äter varsin klementin (syntes).

}

\small{
\textbf{Dic Ajlic}
\label{cdbbe8b881084d842bcc94e1f60465da}
 Granne i Malmö. När Dic får brev till sig med mer myndig ton tituleras han Ajlic Dic

}

\small{
\textbf{Dieselbil med lastgaller}
\label{73b1f975c67393304ff101482965163c}
 köres företrädesvis av ekonomiskt sinnade människor.
 LB-reggad Merca är inte bara snyggt, skatten blir mycket billigare och man kan lägga pengar på annat-
 till exempel röd diesel från Finland. Storswänsken \textsc{(se Storswänsk s.~\pageref{716f41dcabef6599bcf08334a8a6ae27})} påstår sig förlora väldigt mycket pengar på de glesbygdsbor som kör Lb-reggat på finndiesel. Säkert en promille av vad man förlorar på att tina bort snön från gatorna med elström istället för att skotta bort den med spade, som mer utvecklade folkslag gör.

}

\small{
\textbf{Dimma}
\label{b63ad17940b78107e72e63b7d637d91b}
 Ett meterologiskt fenomen. Det är typ vatten som inte är regn \textsc{(s.~\pageref{03456beeae643b4c33b17500a17d1d1e})} eller sånt man dricker i glas utan bara en rätt cool ånga som ofta infinner sig i London och de skotska högländerna. Dimma är dock mest känt för att Dag Vag skrev sin enda bra låt om dimma. Den heter Dimma och går att åtnjuta på skivan \textit{\quotetext{Scenbuddism}} från 1979.

}

\small{
\textbf{Din avkomma och du}
\label{a94d2a0e2987fb13963d974bf02db4a8}
 \textit{Din avkomma och du} är Prof. Etiennes mästerverk om barnuppfostran med undertiteln \quotetext{Så här ska du göra om du råkat hänga in den en stund för länge} och har knappt sålts i tre exemplar världen över. Den bör ses som ett direkt komplement till \textit{Barnagans förträffliga pedagogik \textsc{(s.~\pageref{9e0723018fdd5ac13da751c48083a4e3})}}. Vill du ha hela baletten får du köpa den direkt via förlaget, men här kommer ett urval av den gode professorns handfasta tips och råd.

 \begin{itemize}
 \item \textbf{Du är större än ditt barn}. Använd detta till din fördel så ofta som du kan, vare sig det gäller brottning eller i argumentationer med din fru om vem som ska ha mest att äta till middag
 \end{itemize}

 \begin{itemize}
 \item \textbf{Barn är helt jävla dumma i huvet}. Barn vet ingenting om någonting och det är din och bara din uppgift att lära dom hur det ligger till. Fritänkande är en enkelbiljett till rännstenen.
 \end{itemize}

 \begin{itemize}
 \item \textbf{Var en så liten del av ditt barns liv som möjligt}. På så vis kommer barnet uppskatta ditt sällskap mer, skäm inte bort barnet med närvaro och kravlös kärlek. Trams.
 \end{itemize}

 \begin{itemize}
 \item \textbf{Sanningen hör man av barn och fyllon}. Se till att vara helt jävla grisfull \textsc{(s.~\pageref{80fc21ba5a45f2d0cd24855d78fa7246})} så ofta det går, för att förenkla kommunikationen med ditt barn.
 \end{itemize}

}

\small{
\textbf{Direktörn}
\label{307279f4fa4053e1e1aebc1649eea8b8}
 Småländsk godsherre verkandes i Västerbotten \textsc{(s.~\pageref{d4b008c5143dcffb6b8c35f3876c2a19})}. När han inte ägnar sig åt vegan \textsc{(se veganer s.~\pageref{2a12d5d6ae91d2f4f7d9af3cef58e75c})} anställer han skvadronvis med indonesiska låglönearbetare att, på beställning, skriva nissepediaartiklar \textsc{(se nissepedia s.~\pageref{62400dadecd90cb5cd39062abe5a3e4a})} i ämnen han finner lustiga. Direktörns B12-brist har för mången år sedan försatt hans hjärna i ett så kokt \textsc{(s.~\pageref{19637e73db4270e8d526bb79b56852b7})} tillstånd att han ej själv har förmågan att författa.

 Direktörn har inte alltid stoltserat med titeln direktör. Nej, ack nej. I tidig ålder var han bara en simpel man av folket som bar namnet Rikard Ström. Rikard är ett mansnamn som har sitt ursprung i det fornhögtyska namnet Richart, som är sammansatt av Rik- som betyder 'härskare' och -hart som betyder 'hård'. Ström, i enstaka fall stavat Ströhm, är ett svenskt soldatnamn. Så det kanske faller sig lika naturligt som kontinentalplattorna att denna man skulle komma att bli Direktör Stöm.

 På Direktörens förfrågan kommer en rättelse / uppdatering.\quotetext{Jag har aldrig haft b12-brist. Däremot levde jag två år på pommes och var en av de veganer som introducerade fetsåsen (när grädden kom till Sverige) \textsc{(se Sverige s.~\pageref{b1999637949ed135b2ca03f3a38460cc})}}

 Recept på Direktörens mjuka sås:
 1 grädde, 50gram smör, 1 buljongtärning och en mängd örtkryddor.

}

\small{
\textbf{Discharge}
\label{7084c38f1708430f138336428e4ac7cb}
 Evolutionens första steg.

 Innan fanns ingenting.
 Efter fanns musik \textsc{(se lista över dis-namn s.~\pageref{466d0ad02bac3d76b9aecb05ac68c5ca})}.

}

\small{
\textbf{Diskett}
\label{160d5f536ad7ae443e05e9687170d4f4}
 En diskett är ett elegant litet föremål avsett för lagring av information. Disketten förs varsamt in i en datamaskin, varpå informationen (må det vara en bild eller en txt.-fil) kan föras över på den lilla disketten som sedan kan petas ut och lagras i ett diskett-etui, skickas per brev till avsedd mottagare eller helt sonika säljas till högstbjudande. Disketten finns i många olika utföranden. Den i särklass mest populära är 3.1/2"-disketten, som bland annat saluförs i paket om tio på välsorterade datavaruhus. Konsumenten kan få sina disketter i önskad färg och etiketter finns att tillgå för den ekonomiskt välbärgade \textsc{(se storfräsare s.~\pageref{4db17005692cd83e3e946a1311b81ed0})}, så att användaren kan nedteckna på disketten vad den innehåller. På så vis undviks förvirring i diskett-arkiv som med tiden växt sig så stora att de annars blir svårhanterliga.

}

\small{
\textbf{Divan}
\label{a01d1d03f174a0e65c9d2e21afccb478}
 En divan är som en korsning mellan en säng, en soffa och en stol. Man ligger, men det finns ett ryggstöd så att man kan hasa upp sig och ligga och läsa. På divaner ligger Sultaner \textsc{(se Sultan s.~\pageref{9af82031d374b97c9e73132a413cbdf5})}.

}

\small{
\textbf{Dixie Dave Collins}
\label{45f69e6babd6fbc829abbced2c492cc6}
 [http://www.youtube.com/watch?v=sLDHLbXfCXk] är en väldigt konjektural \textsc{(s.~\pageref{97b105b94d1adc2125ccd7409f18beda})} man som är bassist och sångare i sludge-bandet \textsc{(se sludge s.~\pageref{2ccd23d1cd0f95dc6984215a1f1b31ca})} Weedeater som huserar i Wellington, North Carolina. Han var också medlem i stilsättande och ökända Buzzov•en. I januari 2010 var olyckan framme för Dixie Dave, som av misstag sköt av sig ena stortån med sitt \quotetext{favorit-hagelgevär,} som det uttrycktes i pressreleasen om händelsen, vilket gjorde att inspelningen av Weedeaters mycket emotsedda fjärde album, \textit{Jason... the Dragon}, fick skjutas (get it?) upp ett år. Förlusten av stortån är så mycket mer allvarligt än man först kan tro eftersom Dixie Daves signum är att, medan han spelar på sin bas, lyfta ena benet utåt och sedan stampa ner det i marken. När så \textit{Jason... the Dragon} väl kom ut den första Mars 2011 fanns där som sista spår en instrumental banjolåt med tillhörande vattenljud som Dixie Dave spelade in medan han var hemma och repade sig efter skadeskjutningen, tungt \textsc{(se heavy  s.~\pageref{7cfe64ea44dc3bbeb63b29ff3039a481})} ner-medicinerad med smärtstillande preparat. Denna låt, som speglar en lätt efterbliven människas inre värld, kan höras här [http://www.youtube.com/watch?v=O85fzNgfLIM].  Den som bara vill höra Dixie Dave prata kan kika här [http://www.youtube.com/watch?v=IUr_dosw-jw\&feature=relmfu], och den som vill se Dixie Dave sjunga karaoke kan kolla här [http://youtu.be/RB-9kzoy-Ik].
 HEAD2: Filmskapare
 Dixie är inte bara musiker, utan har även skrivit manus till en musikvideo till en av hans eget bands låtar. Videon innehåller \quotetext{pretty much a Weedeater pep rally with zombie cheerleaders and grown-folks playing dodgeball and beating the shit out of each other.}
 HEAD2: Sagt av Dixie
 \quotetext{Your dumb-ass outlook is what makes me me}

}

\small{
\textbf{DJ Chris}
\label{f4578ba9c629ce66b09f06516b9771ab}
 var en legendarisk Gävle DJ under det sena 80-talet med regelbundna spelningar på ungdomsgården Fridebo i Hagaström. Under dessa gig var det inte ovanligt att DJ-Chris lämnade DJ-båset efter att ha lagt på Robin Becks cocacola-hit \quotetext{First time} för att själv bjuda upp någon lycklig Hagabrud på tryckare. Andra flitigt spelade låtar under dessa set var Italodisco-duon Scotch \quotetext{Disco band} och Alice Coppers \quotetext{He´s back (the man behind the mask)}. Mycket lite är känt om privatpersonen DJ Chris även om de flesta bedömare är överens om att han under sin storhetstid var student på Nynässkolan i Gävle. Den sista kända spelningen var på Fridebos påsklovsdisco 1989 och därefter slutar alla spår. Dock finns det rykten som säger att DJ Chris fortfarande är verksam och gig på så skilda ställen som London, Brisbane och Bejing har rapporterats, även om ingen av dessa med säkerhet kunnat bekräftas.

}

\small{
\textbf{Djungelkasse}
\label{74611749734f87b8dca27bcd78a9cc0c}
\begin{enumerate}
\item REDIRECT Åka på safari \textsc{(s.~\pageref{81c7409524ed7506f454be7fb17d4c38})}
\end{enumerate}

}

\small{
\textbf{Djur}
\label{c5acdb95a7feeab7475c8df4cd4c20c5}
 Det har nyligen kommit till artikelförfattarens kännedom att naturen ska vara full av olika sorters djur så som talgoxar \textsc{(se talgoxe s.~\pageref{4e9d46e4dca35138132c2977b1fcab12})} och brokigt färgade fiskar, vilket djupt har chockat honom. Djur ska tydligen vara varelser som saknar förmåga att tala. Vissa är stora, medan andra tvärtom är mycket små.

}

\small{
\textbf{Djurens nobelpris}
\label{476e59de9782d8a0733007286c14c656}
 är en utmärkelse som instiftats för att uppmärksamma särskilt smarta individer i djurens värld. Den första att få priset var apan Koko, för sin uppfinning att man kan använda en pinne som redskap. Eftersom det inte finns jättemånga smarta djur är det inte säkert att priset delas ut varje år. Ibland kan juryn avstå från att dela ut det och ibland kan till synes konstiga saker premieras. En förklaring till det är att djuren som sitter i juryn ibland är av samma ras som pristagaren. Priset är på hundra djurdollar och kan användas på världens alla zoo. Trots att djurens nobelpris funnits i mer än hundra år är det färre än tio tjejdjur som fått det.

 HEAD2: Vinnare av djurens nobelpris

 1977 Den snefotade ultrapelikanen \textsc{(se Snefotad ultrapelikan s.~\pageref{0def09852ec31cb5af0c38180b411782})} Klas för sin upptäckt att gamla chips smakar precis lika bra som fisk.
 1955 Uven \textsc{(se Uv s.~\pageref{45210da832f9626829457a65e9e7c4d0})} Flaxis för sin lansering av teorin om att det bara är ett eko man hör när man hoar, och inte signaler från ett parallelt uvuniversum.
 1954 Rodriguesflyghunden \textsc{(se Rodriguesflyghund s.~\pageref{19cc9824f1fa5834a866261fe69352ea})} för sin strävan att ena djur och människor.
 1942 Apan Albert II \textsc{(s.~\pageref{8a80daf328e56d2b30df9fb6c782146d})} för sin upptäckt av rymden.
 1918 Paradisfågeln Gunborg \textsc{(s.~\pageref{9e29dc34382963ae7d76a742e98637a4})} för sin uppfinning att bygga vackra bon av gammalt skräp \textsc{(s.~\pageref{75f1a5320951ea0dd9aa3c0eaba2c2c7})}.
 1917 Hästen Natasha för att ha ridit med Lenin på ryggen in i Vinterpalatset.
 1907 Brugden \textsc{(se Brugd s.~\pageref{d6b6b68506b8f1daad3a2ddbfaf8d863})} Sune för sin upptäckt att havet är djupt.
 1902 Hunden Rex för sin uppfinning att tigga vid matbordet.
 1901 Apan Koko för sin uppfinning pinnen.

}

\small{
\textbf{Djurvänner}
\label{5431567d639bf992d892343a9c5ba394}
 är över lag ganska enerverande individer, som tror dom är förmer bara för att dom tycker synd om noshörningen nelson \textsc{(s.~\pageref{e439707db1c491d30a2ac06e71632fe6})} och blir upprörda när dom får se bilder inifrån slakterier. De flesta djurvänner bor i storstäderna och varannan har pälsallergi \textsc{(s.~\pageref{23773a17729d8e7e24da798e97533aeb})}.

}

\small{
\textbf{Djävulens triangel}
\label{71652dc5bf1e6a9cc2d22b2f246569fd}
 Ett öde om en vattenfylld grav
 hemska kval i en annan sfär
 det är allt man kan tyda av skriften som härmed bränns
 stilla havets yta, en fasad, var på din vakt
 Innerst inne vet både du och jag

 Tusen plågade själars makt
 här gives ingen respit
 Du kommer dras med, du kommer dras ned

 Djävulens triangel
 Dödens tid
 Djävulens triangel
 Betala dina synder och inse

 Djävulens triangel, en virvel av tid
 bödeln från djupet

 När seglen fattar vind
 en riktning som är förutbestämd
 en gest av Satan, du befinner dig på en plats du inte tänkt
 när vågorna slutat gå
 och himlen blir mörk
 Ett förebud om din död, det är profetians tid att slå in

 din död... dess bröd... din död... jamen kom igen då för fan

 Åskmoln bildas i norr
 en blixt klyver din mast
 ett dovt ljud hörs från underjorden

 Djävulens triangel
 Dödens tid
 Djävulens triangel
 Betala dina synder och inse

 Djävulens triangel, en virvel av tid

 Ett ögonblick senare
 så öppnar sig havets käft
 Du kantrar, du sjunker, försvinner
 ner till Helvetets Port

 Djävulens triangel
 Dödens tid
 Djävulens triangel
 Betala dina synder och inse
 Djävulens triangel
 en virvel av tid
 Djävulens triangel
 tog även mig

}

\small{
\textbf{DN på stan}
\label{5a22ad78e75653c891a4ab0a4a94df7b}
 är en bilaga till dagstidningen Dagens Nyheter \textsc{(s.~\pageref{b159d08de8d21d8a6d79374b02793693})} och utkommer en gång i veckan, vanligtvis på torsdagar \textsc{(s.~\pageref{42daf19d5e9b792612b2038788e7ded1})}. Bilagan bevakar stockholmsprofiler \textsc{(se stockholmsprofil s.~\pageref{daaee4666c210c7a40537c2399f01556})} som är lite för fina för Facebook \textsc{(s.~\pageref{26cae7718c32180a7a0f8e19d6d40a59})}. \textbf{DN på stan} är nämligen en slags Facebook i pappersform \textsc{(s.~\pageref{37edcb2e533bd9c3e51f475c598b8671})}.

}

\small{
\textbf{Dojo}
\label{9bea15890f18ef35a12767fef5d234b8}
 Varje villaägare av rang som är vid sina sinnens fulla bruk låter inreda en dojo i källaren. I dojon bör det finnas en matta av något mjukt materiel som tillåter en rulla på golvet när man tränar utfall. Det bör också finnas ett ställ med katanas, bambustavar och olika fantasirikt utformade vapen som beställts från hobbex eller någon liknande välsorterad postorderkatalog. Till yttermera visso kan det finnas anledning att skaffa en ninjaformad punchbag som man kan träna kaststjärnekastning på.

}

\small{
\textbf{Doktorand}
\label{308932f67f983fbb70157f1a481f51ea}
 är världens smartaste anka.

}

\small{
\textbf{Dokumentärhora}
\label{a77ce2e0b66b8ce1b7b4d977777d0d7c}
 En dokumentärhora är en person som offrar sin personliga integritet för att sitta i dokumentärer och hypa folk/band/företeelser i utbyte mot exponering i dokumentären de deltar i och lite stålar. Den största dokumentärhoran är utan tvekan Scott Ian \textsc{(s.~\pageref{5cc33aadd6cf4b7006f5ef0c79b5fe5f})}, gitarrist i bandet Anthrax. Utan skam sitter dokumentärhoran i billiga VH1-studios och hypar metallica, Lemmy \textsc{(s.~\pageref{6cc2f8758343439728f308f08a4a8fad})} eller vem fan som helst, utan förbehåll.
 Man ska gärna framhålla hur valfri grupp/individ revolutionerat sin genre och är helt otrolig som människa. Leif GW Persson har föreslagits som dokumentärhora, vilket dock är helt felaktigt. Han är proffstyckare och anställs således för att tycka lite vad som helst om vad som helst, samtidigt som dokumentärhoran bara får säga positiva saker om dokumentärens ämne.

}

\small{
\textbf{Donk}
\label{3ac3836092d2f169445dfcffe720774b}
 Put a DONK on it

}

\small{
\textbf{Doom}
\label{b4f945433ea4c369c12741f62a23ccc0}
 kan syfta på två saker, vilka har det gemensamt att de i huvudsak uppskattas av män.

 HEAD2: Musik
 Doom är en förgrening från hårdrock \textsc{(s.~\pageref{a4566a810e7ad85a57ddc84083a8139b})} på musikens brokiga släktträd. Doom är till skillnad från vanlig hårdrock tyngre, går saktare och är i regel mer monotont. Doom kan i sig förgrenas i lite olika underkategorier, så som traditionell doom (Cathedral, Candlemass, Grand Magus, the Obsessed, Saint Vitus, osv), ny-doom (Electric Wizard, Pelican (första EPn), Floor), Kabbalah doom (Om), funeral doom (Skepticism, Corrupted), kosmisk doom (YOB, Ufomammut) och finsk doom (Reverend Bizarre). Av dessa är kosmisk, kabbalah och ny doom de bästa kategorierna. Det finns de som inte knarkar \textsc{(se knark s.~\pageref{bebc5e7342ca2f076b3d32ed6c557398})} men ändå lyssnar på eller rent av spelar doom, men detta är ett fenomen som den moderna vetenskapen ännu varken kan eller vill förklara.

 HEAD2: Spel
 Doom är ett spel av ID software som de gjorde 1993. Det var grymt och i sällan skådad 3D. Efteråt kom det sex stycken uppföljare som var tråkiga.

}

\small{
\textbf{Doom-excelarket}
\label{f60569a6f23dada8517bd066807a410e}
 är ett mytologiskt excelark som sägs innehålla en lista över doom-låtar \textsc{(se doom s.~\pageref{b4f945433ea4c369c12741f62a23ccc0})} och länkar till relaterade youtube-klipp. Ingen vet om detta ark faktiskt existerar, men spridningen av muntliga berättelser om detta blad, som nedtecknats på så skilda platser som Umeå, Stockholm \textsc{(s.~\pageref{edcd259e0a03c7ab70feb186bae19f13})} och Kamtjata (Ry. Камчатка), ger viss fingervisning om att en faktisk förlaga till legenden någon gång kan ha existerat. Andra menar att nedtraderade kopior fortfarande existerar, vilket ihärdigt tillbakavisas av diverse förståsigpåare \textsc{(s.~\pageref{ff91afb86ce86124b6a517f3eb37bc18})} och storfräsare \textsc{(s.~\pageref{4db17005692cd83e3e946a1311b81ed0})}. En mer positiv skriftkunnig, Prof. Etienne \textsc{(s.~\pageref{56957a267e57df32753cf7f3b8a603d8})}, har i boken \textit{Hur man bär iväg med femton kilo kopparskrot och kommer undan med det} (Timbro, 2008) framlagt en teori om att en etruskisk bokrulle som påfanns i utgrävningar i norditalien på 50-talet och som tros vara en lista på namn på get-ägande nihilister kan vara förlagan till denna osannolika legend. Inom judiska och kristna fundamentalistiska grupper i nordamerika \textsc{(se United States of America s.~\pageref{ade6b3bd5e720abb20ed8a9a4c6b9ae8})} har doom-excelarket kopplas samman med arken som enligt den bibliska \textsc{(se bibeln s.~\pageref{7de7d2a7d608c9a2044f50688bc63e27})} berättelsetraditionen hyste guds \textsc{(se gud s.~\pageref{91e49146121c992aab11a19c77e26cf0})} budord. Enligt en annan vida spridd berättelse ska detta ark finnas i en stängd, klandestin grupp på facebook \textsc{(s.~\pageref{26cae7718c32180a7a0f8e19d6d40a59})}, men anhängarna av denna teori ska enligt experter uppgå till en handfull halvlyckade skojare som till vardags lyssnar på hippie-musik.

}

\small{
\textbf{Dopesmoker}
\label{cebc8a343bbfefbfac0078fcd926a0e0}
 är en låt av den amerikanska rockorkestern Sleep och var 2009 års sommarplåga.
 Låten är otroligt hyllad i knarkarkretsar.

 Vill man vara en lustigkurre kan man ringa till nåt önskeprogram i radio och önska den.

}

\small{
\textbf{Dr. Alban}
\label{9756163bb9005234a901bcd148a44700}
 Svensk tandläkare som utselöts ur Svenska tandläkareföreningen efter att i en duett tillsammans med Kicki Danielsson \textsc{(s.~\pageref{b4646e392dda635159575835254d4ef1})} ha uppmanat till omåttlig konsumtion av söt frukt.

}

\small{
\textbf{Dragbasun}
\label{0315aaaabb57a67312aa3316fd2006e1}
 är ett musikinstrument inom familjen bleckblåsingar som uppfanns för att ersätta den betydligt enformigare fanfartrumpeten. Vad som gör dragbasunen unik jämfört med andra basuner är att man drar i den istället för att trycka på knappar. Ju mer man drar, desto bättre låter det. Kända svenskar som kan spela dragbasun är exempelvis Janne \quotetext{Loffe} Carlsson \textsc{(se skita i det blå skåpet s.~\pageref{97d35803d0f77bf90f90cd3c83dc323d})} och Jan Guillou \textsc{(s.~\pageref{63f2c8aba9686bc92efeb7eb21e35156})}.

}

\small{
\textbf{Dragsko}
\label{0d3beb9223700e39e09040e9bbd3644b}
 är den lilla hålgången längst upp på mysbyxor, till exempel, där det finns ett snöre som alltid åker in och som man inte kan få ut igen, om man nu har några. Dragsko nämns ofta tillsammans med uttrycket \quotetext{en byxa} - typ som \quotetext{en bekväm byxa med dragsko.} Dragskon är inte speciellt populär för tillfället, förutom i skid- och friluftskläder \textsc{(s.~\pageref{9739f8f4c55a7e8abb041c736f99aee7})} där det är mäkta populärt.

}

\small{
\textbf{Dragspel}
\label{f68d96fd163bec2d7dddd811909426e9}
 et har decennium efter decennium varit diskriminerat av fiolspelmän runt om i Sverige. De har ansett att dragspelet inte kunde begära få sin plats inom allmogekulturen, men ingenting kan vara mer felaktigt. Dragspelet har tillsammans med fiolen och i viss mån klarinetten i vissa landskap - t.ex. Västmanland - varit ett av instrumenten som funnits med vid dansbanor och logar och som trakterats av drängar i kammaren, då de fått någon ledig stund över.

 Nu är dragspelet helt accepterat som allmogeinstrument och det är därför i hög grad representerat vid svenska spelmansstämmor. Dragspelsstämman i Hjo näst sista veckan i augusti kan därför ses som en hyllning till dragspelet som instrument och uttolkare av dagens och gårdagens melodier.

 Källa: Åke Mokvist - \textit{Svenska folkfester}

}

\small{
\textbf{Dragspelsmuskeln}
\label{4265ffc7068c10706460aa133c2918bf}
 är en muskel i människokroppen utan riktigt bestämd placering. Troligen sitter den någonstans i överarmen eller på torson. Det är vanligt att idrottare skyller dåliga tävlingsresultat på sträckt dragspelsmuskel.

}

\small{
\textbf{Dressmann}
\label{02ee8e32b89869fffd11aceb4f2e1c10}
 är en norsk herrkläd-kedja. \quotetext{Dressmann} är ett anagram av \quotetext{Der SS-Mann}. Man får dra sina slutsatser själv.

}

\small{
\textbf{Dricka sitt eget badvatten}
\label{9e0465cd5a1cee0842af3328e77b89d5}
 Att dricka sitt eget badvatten är en gammal hederlig hälsokur som motverkar elakartade sjukdomar. Den uppfanns av Thomas av Aquino, mannen som även gjorde chapeau de paysanen \textsc{(se chapeau de paysan s.~\pageref{27aa75146d9ab723d1423168a2539d5d})} känd. Det började med att han läste fel i en sida i bibeln (Jesaja 36:12) \textsc{(se Jesaja 36:12 s.~\pageref{cddcbdb1e8a5df591e5efa642a584350})}, men när han testat kuren en gång kändes det så fett att han fortsatte med det. Orkar man inte dricka allt badvatten på en gång kan man spara en del i baljan, det blir bara bättre med tiden.

}

\small{
\textbf{Driva med överheten}
\label{d2a6f08601a27a11e9a798bee876ee06}
 Klassiskt grepp för att klä sin grådaskiga vardag i en lite ljusare nyans. Man kan till exempel påpeka att påvens hatt är för jävla fånig eller att ostron ser ut som snor. Eller göra som Idi Amin Dada vid ett statsbesök i England 1971 där han bad att få besöka Skottland, Irland och Wales för att träffa \quotetext{revolutionärer som kämpar mot ert imperialistiska förtryck}.

}

\small{
\textbf{Dropbox}
\label{2f33c46fab1f5e54cc9f97fe8571a4a6}
 är ett slags låda på internet som man kan lägga ner saker i. Sen kan man plocka upp dem igen när man behöver dem, lite som en digital version av Skalmans skal.

}

\small{
\textbf{Dubbelsovla}
\label{4a58428516d8ba930242406ad6073922}
 Att dubbelsovla (i mer sydsvenskt glossolali \quotetext{tvesovla}) är att ha två sorters pålägg på mackan, t.ex. både ost och skinka. Det här ses som slösaktigt och väldigt ofint. Vad som inbegrips i sovel, det vill säga, om t.ex. smöret räknas som sovel och således gör en vanlig ostmacka \textsc{(s.~\pageref{2e2a02f9cf463d37a5ab2cae4e0bed2a})} till en dubbelsovlad styggelse, är en diskussion som kan behöva tas i vissa delar av konungariket Sverige \textsc{(s.~\pageref{b1999637949ed135b2ca03f3a38460cc})}.

 HEAD2: Empiriska studier i ämnet
 I vissa skånska släkter kallas en dubbelsovlad smörgås för \quotetext{Perbengsare} efter mågen \textsc{(se måg s.~\pageref{55752d6920060b54fd689faee4ed037b})} Per Bengtsson. Denne (då blivande) måg \textsc{(s.~\pageref{55752d6920060b54fd689faee4ed037b})} dubbelsovlade under en frukost när han uppvaktade en flicka i släkten, något som inte uppskattades av svärföräldrarna. \quotetext{Karin ente ska du gefta dig me han, så slösaktig som han är!} lär gubben ha sagt. Därefter lär päronen \textsc{(se päronhalva  s.~\pageref{cc9c1bfa2ec4eaed89ca86a1b63e3a45})} ha insett vilken storfräsare \textsc{(s.~\pageref{4db17005692cd83e3e946a1311b81ed0})} han var och sedan välkomnat honom i släkten.

}

\small{
\textbf{Durkmojs}
\label{f920619d1c1bf264bcb086452f90fb6a}
 är sjömansspråk för \quotetext{sjömansspråk}.

}

\small{
\textbf{Durkslag}
\label{c48b23fc8215397e022152f51c8933aa}
 När fler än två personer står i en båt och slåss.

}

\small{
\textbf{Dussin}
\label{e4616326552f9ff9435bf747d0495940}
 Tolv stycken.

}

\small{
\textbf{Duvgubbar}
\label{90e175e73f5b2d30726d54bbdbf54538}
 är en benämning på människor som ägnar alldeles för stor del av sin fritid till att föda upp duvor och tävla med dessa makalösa fjäderfän. Tydliga kännetecken är att gubben i fråga har ett flertal duvslag och ungefär ett två tum tjockt lager duvskit på tomten. Det är då förstås inga vanliga stadsduvor, eller \quotetext{flygande råttor} som duvgubbarna själva kallar dem utan det finns två sorter. Den ena sorten är brevduvor som används i sporten duvracing, där uppfödarna släpper ut ett gäng duvor var och ser vem som hittar hem snabbast. De här duvorna orienterar sig genom rent jäva trolleri och hur långt från boet man än släpper ut en duva så hittar de hem. Det närmsta ett svar som den moderna vetenskapen lyckats producera är att duvorna är \quotetext{präglade av hemmet}, va fan nu det ska betyda. Den andra sortens duva är tumlarduvor som används i tävlingar där deltagarna får sin duva att flyga så högt som möjligt. Riktigt värdelöst.

 Duvrace är även big business, i alla fall i skitlandet Belgien \textsc{(s.~\pageref{f79ffe9e826a19f9f6a446c90e21c4e3})}, och den som plockar hem titeln Belgian Master får också en solkig attachéväska fylld med den ohemula prissumman 100 000 euro. Vad som är än mer upprörande är att duvor som vunnit många tävlingar penisoneras, som travhästar, till avel och rekordet för vad en sån här duva bringat hem på en auktion är den om möjligt än mer ohemula summan 2,7 miljoner kronor (!).

 Som ni förstår gör detta att duvgubbarna är beredda att ta till drastiska åtgärder för att skydda sina älsklingar och det är alls inte ovanligt att de förgiftar utrotningshotade rovfåglar för att undvika att en duva slutar som duvfärs i nåt falkbo. Det här leder i sin till att duvgubbar löper fyrtiotusen miljarder \textsc{(s.~\pageref{c2160bffc9c5ca88e77204672e62e489})} gånger större risk än gemene man att bli indragna i långa processer med länsstyrelsen \textsc{(se processa mot länsstyrelsen s.~\pageref{0ae3fdeda52fe82800b04c624330139c})}.

 Tack och lov finns det inga uvgubbar \textsc{(se uv s.~\pageref{45210da832f9626829457a65e9e7c4d0})}, och när det gör det är det fan bara att söka skydd.


 HEAD2: Duvgubbar i populärkulturen
 Duvgubbar figurerar i TV-serien \textit{The Wire} likväl som i förrädaren Elia Kazans film \textit{Storstadshamn} från 1954. I Jim Jarmuschs film \textit{ Ghost Dog } spelar Forrest Withaker en yrkesmördande duvgubbe som lystrar till samurajernas hederskodex \textsc{(s.~\pageref{dec840b19d3e79e3b3ce89b1995bafd9})}.

}

\small{
\textbf{Dvärgdvärguv}
\label{096add7bb1f62abf3939f922074e843d}
 en (Bubo Puttenutti) är så liten att man tror den är en plastleksak. När den börjar röra på sig blir man alldeles förskräckt. Dvärgdvärguven är framavlad enbart för att det ska gå att göra uvioli. Väl inbakad i pastakudden kan dvärgdvärguven leva flera veckor, för den andas så försiktigt. För att inte dra till sig uppmärksamhet.

}

\small{
\textbf{Dvärgpungsovare}
\label{d751d87e9373bbed2540bb46dba30e5c}
 är ett litet pungdjur som ser ut ungefär som en vanlig mus. Naturligtvis återfinns det som världens alla andra konstigaste djur i Australien \textsc{(se australien s.~\pageref{e727d8d1b3162a732c7f706d55de64f3})}. Arten är sjukt utrotningshotad och återfinns idag bara inom ett område på ungefär tio kvadratkilometer i en del av New South Wales som ofta är täckt av snö. Där har någon driftig australiensare självklart slagit upp en stor jävla skidanläggning med flera pister.

}

\small{
\textbf{Dyckert}
\label{4e5643f8df0b0729ba0c40470cc43d69}
 är en typ av spik som i princip saknar huvud \textsc{(s.~\pageref{e906cd95a540df9b16d0460fb4cf0adc})}. Det är ibland pyttelite bredare upptill på den men knappt så det syns. Dyckert används främst inomhus när man ska spika lister och liknande och inte vill att spiken ska synas.

 HEAD2: Trivia
 Inom byggsvängen kallas ibland tjänstemännen elakt för dyckertar av byggarbetarna.

}

\small{
\textbf{Dying victim of the city}
\label{3a99e128d1ee7ff795d228c1e6aa6315}
 You're walking to the \textsc{(s.~\pageref{6f96cfdfe5ccc627cadf24b41725caa4})} city
 a worthless move to drink
 You see the made-up faces
 that helped you make your choice
 hello girls, what's up tonight
 there's no reaction
 why don't you come and sit with me
 -and no attraction

 Dying vicitm of the city
 and we will pray for the
 dying victim of the city
 you have to fight for the
 dying victim of the city
 and we will pray for the
 dying victim of the city
 you have to fight for you life

 You're walking from the city
 nothing new under the moon
 I look at you with pity
 I was there once before
 So, do you really want to know
 what you should do
 You'd have to either change your ways
 Or get the hell out of here

}

\small{
\textbf{Däcka}
\label{0edc511d1358b2dba59c4e427036eab9}
 Att däcka är att somna på ett plötsligt och dramatiskt vis, och på så vis inrätta sig så att man ligger horisontellt mot golvet \textsc{(se on the floor s.~\pageref{71521aaa8b27f4d00d5b020276b3b0e4})}. Uttrycket kommer sig av att just detta är vanligt bland personer som arbetar inom sjöfarten, på grund av det mer eller mindre obligatoriska, mycket generöst tilltagna intaget av rusdrycker. Däcket är ju som bekant ett av golven, nämligen det översta, på en skuta, och därav uttrycket.

}

\small{
\textbf{Dävert}
\label{f30c023eff2afc2046f0627bb2a5398b}
 En dävert är ett klassiskt redskap som i mer än 100 år hjälpt fartyg i nöd att sjösätta sina livbåtar. Ofta består däverten av en vertikal stolpe med krökt topp i en stor båge, eftersom denna form anses vara den bästa att hänga upp en livbåt i. För större livbåtar behövs normalt två dävertar som håller varsin ände, för tillsammans är man som bekant starkare. Världens mest kända dävertar är förmodligen Axel Welins kvadrantdävertar, som producerades i Kolsva, Västmanland, och installerades på Titanic.

 Dävertar är enligt uppgifter på Internet också ett populärt motiv för vrakfotografer. Enligt andra, helt obekräftade, uppgifter har dävertar också gett upphov till hobbyfenomenet dävertspotting \textsc{(s.~\pageref{5597f81ec300fd8ff2834c47b79fcc2c})}.

}

\small{
\textbf{Dävertspotting}
\label{5597f81ec300fd8ff2834c47b79fcc2c}
 \textit{Dävertspotting} är en roman av den store svenske författaren Prof. Etienne \textsc{(se användare: Prof. Etienne s.~\pageref{a9878d2280e5a39becac8f73d113df91})}. Romanen utspelar sig i Sveriges \textsc{(se Sverige s.~\pageref{b1999637949ed135b2ca03f3a38460cc})} Edingburgh, Södertälje, och handlar om en grupp unga män som ägnar sig åt att besöka skepp för att studera dess dävertar \textsc{(se dävert s.~\pageref{f30c023eff2afc2046f0627bb2a5398b})}, samt att injicera stora mängder heroin medan de är i farten. Romanen utsågs snabbt till en av Sveriges genom tiderna mest mediokra verk vilket gjorde att den blev halvkänd och lästes av ett lagom stort antal personer, av vilka många tyckte att den var okej.

}

\small{
\textbf{Dårhusklippkortinförskaffarföranledande}
\label{dbe61c128c0e2ed1563fd1772ac9d00a}
 Ett adjektiv eller adverb som syftar till att en aktvitiet eller ett ting kommer att leda till att en individ \textsc{(s.~\pageref{41beed76a0af9b4f550f7ebdecd3e700})} kommer att frekventera psykvården.

}

\small{
\textbf{Döden}
\label{6f3c270eb5b4d979c777b4ec26dd106f}
 Ett fenomen i sig.

}

\small{
\textbf{E=mc2}
\label{3ca02d4b17b37dc481c95df2eacf1fd7}
 Slanguttryck i Köpenhamn med omnejd för att ta syntetiska droger och köra två motorcyklar samtidigt. Hojarna måste vara vända åt varsitt håll för att en ensam person enkelt ska kunna komma åt båda gashandtagen så färden går mest i cirklar.

}

\small{
\textbf{Eau-de-uv}
\label{3e46e685ae689c4f8e0560be2e7d303f}
 En parfym \textsc{(se luktagott s.~\pageref{f9613f1654fe61d6a5c0787c85daeeaf})} som doftar uv \textsc{(s.~\pageref{45210da832f9626829457a65e9e7c4d0})}. Doften beskrivs som förryckande. Förryckande, och lite svagt minnande om urin \textsc{(s.~\pageref{524fd7acb94f9c2d879b5c1cf8335669})}.

}

\small{
\textbf{Ebbe}
\label{a7cb5eddea4d149b3850d4089ba8f9d8}
 är ett palindrom, vilket bara det är lite suspekt. Dessutom träffar man sällan personer med detta namn, vilket kan få en att i all rätt begrunda vad alla som heter Ebbe gör och var de befinner sig.

}

\small{
\textbf{Eckös moppe}
\label{de545d489953e8740fd6677ce1e6798e}
 är ett åk med fransar, uppochnedvända kors och allt du kan önska dig, förutom fungerande kopplingshandtag. När du åker på Eckös moppe rejs-sparkar du in växlarna, vilket skapar en fartkänsla som endast kan liknas vid den fartkänsla det svenska folket upplevde när Lars Adaktussons kärriär gick i stöpet efter att han gick från SVT till TV8 eller vad fan det var.

}

\small{
\textbf{Ed Hardy}
\label{e3b7de0302ffdd8ec9ad544dba1d5b3d}
 Du är färgblind, din polare har tatuerat en örn på halsen.
 Ed Hardy är det givna valet.
 Nu syns det inte när du spytt ner dig på fyllan heller, spyan smälter in mot det heltäckande trycket.

}

\small{
\textbf{Eddy Merckx}
\label{c91a3f6993ed23dd05cee5ff8e52c938}
 är en belgisk \textsc{(se Belgien s.~\pageref{f79ffe9e826a19f9f6a446c90e21c4e3})} cyklist som vunnit Touren flera gånger. När en reporter en gång frågade hur han kunde cykla så fort svarade han (typ) \quotetext{När en trampa kommer upp så trycker jag ner den så fort det bara går}.

}

\small{
\textbf{Edmund}
\label{d3f8a05c465714e2c09d214af0e88897}
 känns spontant som ett lite festligt namn som är lätt att associera med den brittiska bildade borgarklassen: \quotetext{Edmund, pray do you take sugar with your tea?} \quotetext{Oh, I couldn't impose Mrs. Doppelworth!}

 Kända bärare av namnet är den nya zeeländske alpinisten och filantropen Edmund Hillary \textsc{(s.~\pageref{8c30ada4e29fa9820d3f6850dc843b0c})}.

}

\small{
\textbf{Edmund Hillary}
\label{8c30ada4e29fa9820d3f6850dc843b0c}
 Sir Edmund Hillary var en Nya Zeeländsk \textsc{(se Nya Zeeland s.~\pageref{cc538f38b19598eab98f434ece99de60})} alpinist och var en av de första att bestiga planeten jordens högsta berg, Mount Everest \textsc{(s.~\pageref{b5b5d890ef4ff008c7821da350799545})}. Bestigningen gjordes tillsammans med sherpan Tenzing Norgay \textsc{(s.~\pageref{5064d0d0c513aec890ffa0ef5d7577ac})} och när dessa två äventyrare nått toppen uppstod ett dilemma. Vem ska fota på toppen? Lotten föll på Hillary då nepalesen Norgay aldrig sett en kamera förut och, som Hillary träffsäkert lär ha uttryckt det: \quotetext{Toppen av Mount Everest är knappast en bra plats att förklara hur en kamera fungerar}. Således borde Norgay vara den förste mannen att beträda toppen, men detta har debatterats (se artikeln om Norgay \textsc{(se Tenzing Norgay s.~\pageref{5064d0d0c513aec890ffa0ef5d7577ac})}.

}

\small{
\textbf{Eduardo}
\label{6d6354ece40846bf7fca65dfabd5d9d4}
 Bor i skrubben hos Bastarden \textsc{(se Användare:Bastarden s.~\pageref{3dc9453755d73d04e64947ec1d7e1002})} och Pukan. I alla fall halva året då det är snöfritt utanför.
 Ägde en katt vid namn Morungo som kidnappades av ett katt-ufo (eller slutade han sina dagar på bordet hos kinesiska utbytesstudenter?) \textsc{(se utbytesstudenter s.~\pageref{397699f3732b0c22f3c532a111697539})}

}

\small{
\textbf{Edvin}
\label{31fbf11311684a7ecb580b8188f52df8}
 är ett så sällsynt namn att om man mot förmodan träffar någon som lystrar till namnet Edvin bör man genast fotografera sig med sin mobilkamera brevid den personen. Man kan då hålla upp ett ironiskt handtecken som man hämtat från någon Hip-Hopskiva och bara har ett hum om vad det kan betyda.

}

\small{
\textbf{Efterrätt}
\label{5fff6e8d7fdf5598341319db050f14c3}
 är den finaste och mest emotsedda delen av en måltid. Man får inte alltid efterrätt, men när man får det vet man att njuta av maten. Detta kan man göra antingen genom att äta sakta och verkligen känna smaken mot gompaletten eller genom att äta snabbt så man hinner ta flera gånger. Man får ofta efterrätt när man är på besök hos någon. Om man inte får efterrätt är det logiskt att ta sig en funderare på om man verkligen ska göra sig besvär med att komma tillbaka. Exempel på efterrätt är päronhalva \textsc{(s.~\pageref{cc9c1bfa2ec4eaed89ca86a1b63e3a45})} med After Eight.

}

\small{
\textbf{Egendom}
\label{6dcb2d477af2e9313da3a5a243555f6b}
 är stöld.

}

\small{
\textbf{Ekfors Kraft}
\label{6ff19e36814c0aacb845a631b83a3406}
 var ett elbolag som ägdes av Mikael Styrman. Styrman hade en lika ondskefull som genial affärsidé som han satte till verket 2007. Ekfors Kraft fick i uppdrag av Överkalix och Haparandas kommuner att tillhandahålla gatubelysning, något som som Ekfors Kraft levererade utan bekymmer till en början. Här kommer Styrmans plan in: När kommunerna väl blivit beroende av hans bolag så såg han helt enkelt till att höja priset till ungefär fyrtiotusen miljarder \textsc{(s.~\pageref{c2160bffc9c5ca88e77204672e62e489})} kronor/kwh, så att det lönar sig liksom. Tjänstemännen och politikerna fick såklart spel, men Styrman vek sig inte en tum. För er läsare som inte varit i Norrbotten \textsc{(s.~\pageref{0e8c003b75982032cde152609ee94154})} på vintern kan vi på Nissepedia \textsc{(s.~\pageref{62400dadecd90cb5cd39062abe5a3e4a})} informera att det är mörkt. Jävligt mörkt. Därför är gatubelysning mycket trevligt, så kommunerna var inte direkt i någon position att vänta ut Styrman och det hela gick till rättegång. Kommunerna vann och Ekfors Kraft sattes, som den kvicktänkte redan listat ut via den inledande meningens tempus, i konkurs. Styrman var missnöjd över detta.

}

\small{
\textbf{Ekorre}
\label{884a147b04857e85118450cc1fffc24f}
 n (\textit{Sciurus vulgaris}) tillhör familjen gnagare \textsc{(s.~\pageref{b5c3a0f14d5f76de604f5d8e4cc068ff})} och kan ofta skådas sittandes i granen, där de gärna skalar och förtär kottar. Ekorrar har mycket bra hörsel och när de får höra \textsc{(s.~\pageref{c4774ec92abe06f5664e18f44446d7e7})} barnen komma får de i naturliga fall väldigt bråttom. Då hoppar de vanligtvis till en tallegren. Tyvärr har vi under det senaste decenniet sett många fall av att ekorren då stöter sitt lilla ben, och den långa, ludna svansen.

 Källa: Alice Tegnér \textsc{(s.~\pageref{66a2a0b3aa1a42e1e5ae2d20dd1bdca6})}

}

\small{
\textbf{Elaka Anna}
\label{38a9b9dc0dd371d94f4b89384f3510a0}
 var alla småpunkares skräck. Hon bodde i ett crustpunkarkollektiv i en radhusförort till Stockholm \textsc{(s.~\pageref{edcd259e0a03c7ab70feb186bae19f13})}. Kollektivet benämndes av grannarna \quotetext{knarkarna på 75an}.

 Elaka Anna hatade när Grove Love Hallings osnutna punkkompisar från Skåne kom på besök och hon visade detta öppet, genom att dels vara jättesur och otrevlig och dels genom att högt beklaga sig och fråga sina kollektivkamrater \quotetext{hur länge ska de stanna} när besökarna befann sig i rummet intill.

 Det sägs att Elaka Anna faktiskt varit snäll en gång och skänkt bort ett soprumsfynd till en storväxt punkare från Kalmar.

}

\small{
\textbf{Eldräven}
\label{2870b08e55128c20596edff255d2e1c3}
 (även kallad \quotetext{Firefox} utanför Svea Rike \textsc{(se Sverige s.~\pageref{b1999637949ed135b2ca03f3a38460cc})} är en s.k. internetbläddrare som generellt är tuffare än Björn Skifs i blockbustern \textit{Strul} och på alla sätt smidigare än konkurrenten Internet Explorer. I de senare versionerna kan man t.ex. bara skriva in \quotetext{nissepedia} i adressfältet, trycka på enter och vips letar den fram http://www.nissepedia.com alldeles av sig själv och skänker några sekunder till ditt liv. Då Eldräven är baserad på öppen källkod så brukar den ibland tilldra sig skrattretande kritik från illvilliga slipskillar \textsc{(se storfräsare s.~\pageref{4db17005692cd83e3e946a1311b81ed0})} och annat löst folk. Om någon person i eller från Kista ber en att \quotetext{Eldräva} något specefikt så menar denne att du bör söka information på internet om detta något.

}

\small{
\textbf{Eljest}
\label{2ded84b3f6092a191088df8b0e9d0a57}
 Går inte att sätta fingret på.

}

\small{
\textbf{Elva}
\label{788bd84addbcf8f1767869759d4a2ad9}
 Längdskidsterm för beskrivning av kraftigt rinnande snor.


 Se även: Etta \textsc{(s.~\pageref{ba48f6c4097b7fc25ca11f1e544842d7})}, Tvåa \textsc{(s.~\pageref{84fcc0494ecf9f5af79fcd9bed184a9a})}, Trea \textsc{(s.~\pageref{6f94fdf535ab2e21147ea40ea920ca75})}, Fyra \textsc{(s.~\pageref{7bdb5385ce8e0b1cbc7c15b1d71e8e7d})}, Femma \textsc{(s.~\pageref{d974e0811fe7a4d49a9062d33b66a88d})}, Sexa \textsc{(s.~\pageref{4b1fabe53857b0a2ace6ae22008fe13e})}, Sjua \textsc{(s.~\pageref{e7bf63fa6d0d29bd89c23f833b979a15})}, Åtta \textsc{(s.~\pageref{6fa68b0d02ec525fa72a51c13e5e3ed1})}, Nia \textsc{(s.~\pageref{04a481486dd84d7c8bfdfc89d38136a6})}, Tia \textsc{(s.~\pageref{e7292d5ba58672ce7f6fc3c0b646ab63})}.

}

\small{
\textbf{Emetofobi}
\label{afcfb287e0c9a3f9fa7a3e6e748afdcf}
 Kräkskräck. Som alla andra fobier \textsc{(se fobi s.~\pageref{deaf5f7387941b1c8f557f135d4c370a})} botas den bäst med att man utsätter sig för skiten. Grants \textsc{(s.~\pageref{74dc2f36b83c605847a3519729a18d11})} kan funka.

}

\small{
\textbf{Emma-maria}
\label{e60d0e3b986eebe7b200e18d7afe90ab}
 hade stora ambitioner, tjejjouren som hon (eller jaja...) skapat var mycket viktig och angelägen. På frågan om den Miss Wet t-shirttävling hon deltagit i på Cleo några månader tidigare svarade hon helt enkelt \quotetext{Jag var väl ung och dum}. När nu karriären inom föreningslivet havererade satsade Emma-maria på en yrkeskarriär som sminkförsäljare \textsc{(se brinner för att sälja s.~\pageref{4397dcfa1c80db06a775fb49f5171806})}. Det gick väl bra i flera månader, men småstadstristessen smög sig på och som nån slags politisk markering ställde hon upp i Big Brother där hon begick otukt \textsc{(se pörr s.~\pageref{5faa435e2f0af7617816f0cade262581})} och kämpade så tappert. Efter det havererade sminkkarriären. Var hon tog vägen sen kan man spekulera i.


 Detta har gett upphov till frasen \quotetext{Att dra en Emma-Maria} vilket innebär att blixtsnabbt göra ett lappkast \textsc{(s.~\pageref{9dd6698c53a9d42abffb80092f739ae2})} beroende på vars vinden blåser.

}

\small{
\textbf{Emotofobi}
\label{ac9a644beefadd5cb140c93a9ec35123}
 Emetofobi \textsc{(s.~\pageref{afcfb287e0c9a3f9fa7a3e6e748afdcf})}

}

\small{
\textbf{England}
\label{f48861ca24e26a23a923ca68657079f4}
 är ett land som ligger på den största av de brittiska öarna och är en del av det förenade kungariket. Det har varit centrum i världens största imperium om man inte räknar det amerikanska nykoloniala imperiet. Men, let's face it, England är inte lägre det där storslagna landet det ibland talas om. Maten får en att spy bara av att tänka på den, husen är dåligt byggda, de kör på fel sida, super som svin, förtrycker sin arbetarklass, som för övrigt är livsfarlig och de har världens mest invecklade och därtill dåliga parlamentariska system. Det finns inte ett levande träd så långt ögat når, inga djur, bara gräs och grus. Städerna ser ut som skit och man ska vara glad om man tar sig därifrån oskadd.
 HEAD2: Ägarförhållanden
 England tillhör, enligt säkra källor, Colin McFaull.

}

\small{
\textbf{Enkelbekasin}
\label{92790a7d98a9c95e665e46ff0cc91f00}
 (Fronsv. \textit{enkel} ung. \quotetext{okomplicerad} Forndan. \textit{bekasin} ung. \quotetext{bensin}), eller enkelbeckasin som det egentligen stavas, är en vadarfågel i familjen snäppor. Enkelbeckasinen är brunspräcklig, stor som ett mjölkpaket ungefär och har en lång, rak näbb som den använder för att äta med. Näbben ser ungefär ut som en blyertspenna.

}

\small{
\textbf{Enkido}
\label{136a214496f9cc60bf72924a2424c413}
 är Gilgamesh polare i Gilgamesheposet, som skrevs nån gång för sisådär 3000 år sedan i Mesopotamien. Enkido och Gilgamesh drar i eposet omkring och jävlas \textsc{(se jävelskap s.~\pageref{46845591177f16920cd586a5baf5a625})} med jättar och gudomliga oxar, men sen mular Enkido och det är början på en gripande skildring av sorg och saknad efter en död \quotetext{vän.}

}

\small{
\textbf{Entreprenad}
\label{2d3b60492ed3cebe0a3cf341bc5b20b5}
 som skor sig på en medborgare.]]
 Omfördelning av det allmänna till enskilda.

}

\small{
\textbf{Entreprenör}
\label{3e3bed0ad880c109dd739947b6041ea8}
 Person som säljer sommarkatter på våren och kräftbete \textsc{(s.~\pageref{76499bc9cc050bed2beb8e36dd601066})} på hösten.

}

\small{
\textbf{Epikurism}
\label{198603b3bd87cb821515314304b24181}
 är en filosofisk åskådning namngiven efter Epikuros, en grekisk man som levde på 300talet före kristi \textsc{(se jesus s.~\pageref{110d46fcd978c24f306cd7fa23464d73})} födelse. I korthet går läran ut på att människans främsta uppgift är att eftersträva ett ganska sjysst, lugnt och tillbakalutat liv med mycket njutning i glada vänners lag. Människan ska undvika sådant som orsakar själslig och kroppslig smärta i henne själv och i andra. Är man törstig ska man dricka. Är man hungrig ska man käka. Är man taggad på att lyssna på The Who's \textit{Live at Leeds} på helgvolym \textsc{(s.~\pageref{3539fdeb41a5b216f614b6ced9ff5cff})} ska man, får man anta, göra det.
 Detta låter ju ganska bra kan vissa tycka, och framförallt har epikurismen efterlämnat sig ett lika underanvänt som tokbra adjektiv, nämligen \textit{epikurisk}.
 \textlessi\textgreater\quotetext{TB. Det är den där epikuriska mannen som bor på Berghem och bär gröna småbyxor.}

}

\small{
\textbf{Epizootilagen}
\label{bb3fa656326993784ac864edc12a2373}
 gäller sådana allmänfarliga djursjukdomar som kan spridas genom smitta bland djur eller från djur till människa.

 Lagen i sin helhet: [http://www.riksdagen.se/webbnav/index.aspx?nid=3911\&bet=1999:657]

}

\small{
\textbf{Eric Erfors}
\label{77b4a42490103982dba1f4bafed8a276}
 Polare till Nisse Schwarz \textsc{(s.~\pageref{5af492cc4c5ae8563564d2a80c2d4f56})}.

}

\small{
\textbf{Erik Hamren}
\label{eb9339f62d0339481125f7bcbd4f1e78}
 Erik Hamrén är en svensk man \textsc{(s.~\pageref{39c63ddb96a31b9610cd976b896ad4f0})} som gjort det uppseendeväckande valet att alltid gå klädd i väst. Därför har han blivit utvald att träna och leda Sveriges \textsc{(se Sverige s.~\pageref{b1999637949ed135b2ca03f3a38460cc})} landslag i fotboll \textsc{(s.~\pageref{961bd74d34872ff94a4df3a16119096e})}, vilket väl går sådär, om man ska vara helt ärlig. Många fotbollsentusiaster reagerade med resignation då SIFO tydligt visade att Hamrén lagt endast 39\% av sin arbetstid på att träna fotbollslaget och hela 120\% på sin egenproducerade operett, \textit{How the väst was won}, där han också spelar huvudroll.

}

\small{
\textbf{Erik Homburger Erikson}
\label{8fa15943b80847431017070c13ad957d}
 [[File:Erikson4.jpg\textbarthumb\textbaralt=Puzzle globe logo\textbarKolla på den självgode djäveln.]]

 Erik Homburger Erikson var en psykopat eller psykoanalytiker, kommer inte riktigt ihåg vilket men riktigt dumma djävla idéer hade han. I stort sett stal han Sigmund Freuds tankar och påstod glatt att det minsann var hans egna.
 Vetenskapliga bevis var tydligen inte ett krav på såna som höll på med vetenskap förr i tiden, man kunde bara komma på nåt och sen skriva jättemycket om det och vips så var man vetenskapsman!

 Av någon outgrundlig anledning så studeras hans verk fortfarande, detta trots att det finns nya, minst lika dumma tankar om människan och hennes natur.



 Category:Fantastiska levnadsöden \textsc{(s.~\pageref{0653a199a6857e4fb086e6107d442af8})}

}

\small{
\textbf{Erna}
\label{035b3c6377652bd7d49b5d2e9a53ef40}
 är ett kvinnonamn som kom till Sverige redan med visigoterna. Dess betydelse är omtvistad men den vanligaste tolkningen är \quotetext{den som inte bangar på att rulla hatt \textsc{(s.~\pageref{7c7afc9fb7bb52962f954c0cb548c10c})}}. Den maskulina formen av Erna är Arne, som man får om man skriver namnet baklänges. Den keltiska versionen av namnet är Enya.

}

\small{
\textbf{Ernst Billgren}
\label{788717210e9662a9365efdbc9094f936}
 är en av Sveriges \textsc{(se Sverige s.~\pageref{b1999637949ed135b2ca03f3a38460cc})} mest välkända och uppskattade konstnärer av två anledningar:
\begin{enumerate}
\item Han målar vagt humanoida ankor. Detta har många lätt att relaterat till.
\item Han trasslar inte till det med politik eller nåt slags djupare tanke med sin konst.
\end{enumerate}

}

\small{
\textbf{Ernst Haeckel}
\label{7e70f6916ca04431c182f0113b832ae3}
 (16 februari 1834 – 9 August 1919) var en tysk \textsc{(se tyskland s.~\pageref{b1b58da783b6d5fa090f3015f1889869})} som 20 september 1914 blev den som myntade begreppet ”första världskriget”. Haeckel var dock mycket mer än en mustig \textsc{(se den tyska mustigheten s.~\pageref{682ccd5fdc3aff0c97e8845c3d6b6ca8})} tysk som satt i en gigantist ekfotölj och funderade ut domedagsmättade termer. Han var framförallt en världsledande biolog med smak för att psykedelisera sin vardag \textsc{(s.~\pageref{ac0a9b23e06116650a85505c85f16fd2})}, vilket illustrationen till höger som han har ritat runt 1904 visar med all tydlighet.

}

\small{
\textbf{Erotik}
\label{972f097461d1eab1c1ff104757bad922}
 \textit{In medias res:} Hon hade aldrig lagt sina blekfeta fyllda tubsocksliknande fingrar kring en okvistad arbetarkuk förr. Grevinnans tjocka kinder rodnade. Tjänstefolket var ursinnigt, kallskänkorna knäckte nötter med sina skälvande sköten. Mest förryckt var Traudl, den tyska kammarjungfrun från Heidelberg.

}

\small{
\textbf{Eskimåpuss}
\label{38ff3378f460aef8ae0f9d4dc91f96b7}
 Det primära användningsområdet för näsan är odiskutabelt att snappa upp dofter i omgivningen, tätt följt av att den utgör en port där syre och luftföroreningar kan tränga in.
 Eskimåpussen kniper dock tredjeplatsen, i jämn kamp med näsans funktion som glasögonhållare \textsc{(se glasögon s.~\pageref{bb2f2cb84c42a821763d572f86b1e3c9})} och knarkredskap \textsc{(se knark s.~\pageref{bebc5e7342ca2f076b3d32ed6c557398})}.

 För att utföra en eskimåpuss för du varsamt din egen näsa mot en annan individs näsa.
 Vid kontakt för du näsan ömsom till vänster, ömsom till höger, i ett slags gnuggande rörelse, och känner snart hur du och den ljuvliga varelsen mittemot dig blir ett.

 Det är möjligt att du i vissa miljöer mitt i allt detta upplever en spritindränkt andedräkt, men var lugn, det är bara doftsinnet som försöker tränga sig in i rampljuset igen.
 Fokusera istället på känslan av fullständig lycka eskimåpussen bringar och du återvänder till den lilla bubbla som bildats kring er två.

 Eskimåpussen är ett effektivt sätt att visa uppskattning och/eller intresse.
 I lägen där du är lite småblyg eller en traditionell kyss på grund av andra omständigheter faller ur ekvationen är eskimåpussen det självklara valet.

 När du är svältfödd på närhet kommer en eskimåpuss ge en försmak på hur himmelriket känns.

}

\small{
\textbf{Eskimåspov}
\label{7d88e5ee3ad2479d7c1dfe662396fa1a}
 (\textit{Numenius borealis})är en jätteovanlig spov i familjen fåglar. Det kan till och med hända att den är så ovanlig att den blivit utdöd, ingen har i alla fall säkert sett någon sen 1980-talet. Om den finns kvar så bor den i Kanada och Alaska på sommaren och på Pampas slätter \textsc{(se slätt s.~\pageref{a9cde01124ca41f23d6044b3ba27b979})} på vintern. Den kvittrar klara visslande toner och har en lätt nedåtkrökt näbb som den kör ned i jorden för att hitta mask och annat smarrigt. Den är tillräckligt liten för att få plats i en normalstor cigarrlåda.

}

\small{
\textbf{Eskimåvändning}
\label{7ca6450f5cefc61523fade427738fff3}
 En eskimåvändning är att snurra runt 360 grader med en vattenfarkost. Detta rekommenderas främst med kajaker, även om det har rapporterats om lyckade förfaranden med betydligt större båtar. [http://www.youtube.com/watch?v=SQ8pG9ymIu4]
 Eskimåvänding är nära besläktat med kovändning \textsc{(s.~\pageref{f5443b8a36759ce46480e6a7992cc4f2})} där man dock roterar runt y-axeln istället för z-axeln. För den som önskar förändra sin situation rekommenderas starkt att välja kovändningen eftersom man efter avslutad eskimåvändning befinner sig i exakt samma position som innan. Tämligen meningslöst. Ännu en relaterad aktivitet är lappkastet \textsc{(se lappkast s.~\pageref{9dd6698c53a9d42abffb80092f739ae2})}, populärt bland fritidsskidåkare.

}

\small{
\textbf{Espresso}
\label{a2cc65ac1f8ba2f1f8b194c79a9d675f}
 , uttalat \quotetext{äksprässå} av vanligt hederligt folk och \quotetext{eespreeesooo} av kälkborgare \textsc{(s.~\pageref{0f34b469a48952e93688861083ace75a})}, är sjukt överskattat kaffe. Den mörka färgen kan vilseleda, det är inte normalstarkt kokkaffe utan surt bryggkaffe i koppar mindre än ett snapsglas. Eftersom Espressofenomenet stammar från Italien uppfattas det som fint och exklusivt av moderna människor. Prissättningen av Espresso följer en slags inverterad skala, ju mindre kopp desto dyrare. Särskilda maskiner används för att framställa denna dryck, de förekommer ofta på kaffestugor som vill vara förmer men är även en tydlig klassmarkör \textsc{(s.~\pageref{6a9c0c6836a0777442468f821837e795})} i privata hem. Har man \quotetext{kaffemaskin} röstar man borgerligt, har alldeles för stor köksbänk och saknar smaklökar.
 Normaliserandet av Espresso har gått så långt att Nissepediaskribenter ägnade en halv förmiddag åt att finna drickbart kaffe under en forskningsresa i Stockholm \textsc{(s.~\pageref{edcd259e0a03c7ab70feb186bae19f13})}.

}

\small{
\textbf{Ett kille}
\label{be4312be5658f9e6a9aea666c5c60b17}
 är en ung man som objektifierats av en eller flera kvinnor, kanske ett tjejgäng som varit ute och druckit vin och skrattat tillsammans. Killet blir sett inte som den person det egentligen är, utan som ett ting, en kropp för kvinnor att konsumera. Sven-Otto Litturin, även känd som \quotetext{Dangerzone2010}, har länge närt en önskan om att bli ett kille och har lagt ner både tid, pengar och karriär på detta, men har än så länge inte varit framgångsrik i sin strävan.

}

\small{
\textbf{Etta}
\label{ba48f6c4097b7fc25ca11f1e544842d7}
 n är ett svensk snus som finns i varianterna lös, portion och vit portion. Receptet har varit det samma sedan pangea.


 Se även: Tvåa \textsc{(s.~\pageref{84fcc0494ecf9f5af79fcd9bed184a9a})}, Trea \textsc{(s.~\pageref{6f94fdf535ab2e21147ea40ea920ca75})}, Fyra \textsc{(s.~\pageref{7bdb5385ce8e0b1cbc7c15b1d71e8e7d})}, Femma \textsc{(s.~\pageref{d974e0811fe7a4d49a9062d33b66a88d})}, Sexa \textsc{(s.~\pageref{4b1fabe53857b0a2ace6ae22008fe13e})}, Sjua \textsc{(s.~\pageref{e7bf63fa6d0d29bd89c23f833b979a15})}, Åtta \textsc{(s.~\pageref{6fa68b0d02ec525fa72a51c13e5e3ed1})}, Nia \textsc{(s.~\pageref{04a481486dd84d7c8bfdfc89d38136a6})}.

}

\small{
\textbf{EU}
\label{4829322d03d1606fb09ae9af59a271d3}
 Villaägarnas politiska organ.

}

\small{
\textbf{Eulalia II}
\label{6d54724bb4a27cf3c07517daa76deadb}
 är Åsa-Nisses bil \textsc{(s.~\pageref{b3188f47d2eac7efc3f1258dc673a9fe})}. Modellen är en 1932 års Ford B och skänker Åsa-Nisse stor glädje men ibland också bekymmer. Man vet aldrig riktigt vad som kommer hända när man åker med Eulalia II.

}

\small{
\textbf{Eva Ekeblad}
\label{de66bb2a3f5c71b15a204f8e773ea925}
 \textbf{Eva Ekeblad } (född 10 juli 1724 och död 15 maj 1786) var en svensk vetenskapskvinna, som idag främst ihågkoms för att ha informerat almogen hur man kokar brännvin av potatis.

}

\small{
\textbf{Everlife}
\label{235344472d00f90174cd1c9e10b20b5e}
 är ett kristet pop-band (se CCM) \textsc{(se CCM s.~\pageref{b42f1990c0cee8758b64584877d69b93})} från Indiana, Pennsylvania, USA \textsc{(se United States of America s.~\pageref{ade6b3bd5e720abb20ed8a9a4c6b9ae8})} som består av systrarna Amber, Sarah, and Julia Ross. Bandet länge haft ett nära förhållande till Buena Vista och  Disney \textsc{(se Kalle anka  s.~\pageref{64db68f686a0ca4d9d641061cb3fdf13})} som bidragit mycket till gruppens framgång i den kristna rockvärlden. Inte minst har gruppen medverkat på ett stort antal soundtracks för Disneyproduktioner. I USA har mottagningen av Everlifes vackra popmusik och saliga budskap varit eld och lågor, så att säga, men i Europa har man haft lägre försäljningssiffror. Detta kan antagligen förklaras med den ateistiska rockens dominans och apatin för Guds och Jesu ord på denna sida av Atlanten.
 HEAD2: Discografi
 Album:
 \begin{itemize}
 \item Everlife
 \item What's Beutiful
 \end{itemize}
 Singlar:
 \begin{itemize}
 \item \quotetext{Look Through My Eyes} (Disneymania 4)
 \item \quotetext{Find Yourself In You} (Hannah Montana OST)
 \item \quotetext{Real Wild Child} (The Wild soundtrack)
 \item \quotetext{I Could Get Used to This}
 \item \quotetext{Goodbye} (Disneymania 5 Wal-Mart Exclusive)
 \item \quotetext{Static} (promo only)
 \end{itemize}

 \quotetext{Surfa} gärna in på Everlifes hemsida och lär dig mer om bandet och dess budskap: [http://everlife.squarespace.com/]

}

\small{
\textbf{Evert Taube}
\label{add15aedc14973d5fe0f3982e46e40b8}
 (1890-1970) är en ärkesexist som tillhör sveriges mest älskade musiker och författare. Håkan Hellström är ett uttalat fan av denna man.
 Den enda grupp i det svenska samhället som verkligen inte ogillar Taube är straffade pedofiler, som tycker att han sätter saken i ett visst perspektiv..
 HEAD2: Exempel ur Evert Taubes konstnärliga verksamhet
 \begin{itemize}
 \item Flickan i Havanna
 \item Fiorella från Carmella
 \item Rosa på bal
 \item Flyg till Pampas
 \item I najdernas gränd
 \item Mina damer och herrar
 \end{itemize}

 HEAD2: Evert Taubes arv till eftervärlden
 \begin{itemize}
 \item Innevånarna i Roslagen, de mytomspunna \quotetext{rospiggarna,} har låtit anlägga en nöjespark i nationalskaldens ära, den kallas Evert Taubes värld \textsc{(s.~\pageref{9cab574b97cf87df67da43f2aa5b3c33})}
 \item Myntade i senare tonåren uttrycket lurmus \textsc{(s.~\pageref{23f18296e8df765844117b713fb4613f})} efter att både använt sig av Feminist-knepet \textsc{(s.~\pageref{0630009ea0dc2268ee6b7159e651aa62})} och bjudit på Päronhalva \textsc{(s.~\pageref{cc9c1bfa2ec4eaed89ca86a1b63e3a45})} men ändå inte fått ligga. Evert Taube skall icke förväxlas med Snälla killar som aldrig får ligga \textsc{(s.~\pageref{630d0607c17e587ef244461bbafe9b4b})}
 \end{itemize}

}

\small{
\textbf{Evert Taubes värld}
\label{9cab574b97cf87df67da43f2aa5b3c33}
 är en nöjespark i Roslagens famn. Här kan barn i alla åldrar dansa vals med Calle Schewen, eller bara se Rönnerdahl virvla sina lurviga ben om man hellre vill det. Efter det kan man ta en åktur med karusellen \quotetext{Briggen Blue Bird från Hull} och sedan gå in till Den Glade Bagaren i San Remo för att äta det bästa bröd som fås, och det är bäst just för att han är så glad. Skulle inte dessa aktiviteter vara nog så kan man bara ligga på Sjösala äng och titta på alla vackra blommor som slagit ut, nämligen gullviva, mandelblom, kattfot och blå viol.

}

\small{
\textbf{Eviga frågor}
\label{3bd505f805d94787ec0cc431648a7826}
 \begin{itemize}
 \item Blir man skinhead \textsc{(s.~\pageref{a54bc1b5d472b5afed8e84004b6441c4})} för att man är tjock \textsc{(s.~\pageref{638fc266ae8a51a3eeb87a4cab84e057})} eller blir man tjock för att man är skinhead?
 \item Hur långt är ett snöre? \textsc{(se snöre s.~\pageref{30b7be64f820e5ec00397848f6f8d1c8})}
 \item Patrik Sjöberg \textsc{(s.~\pageref{77703d875078935741a7e0904cd69fa4})} eller Johnny Ekström?
 \item Elvis eller Jesus? \textsc{(se Jesus s.~\pageref{110d46fcd978c24f306cd7fa23464d73})}
 \item Vad driver dom?
 \end{itemize}

}

\small{
\textbf{Ewan dobson}
\label{2bb5f3a1c89d7685fedec5e4984d1a81}
 spelar gitarr \textsc{(s.~\pageref{a08bf8420208934bc59c7ed7385d4308})} i ett kanske för en del udda val av kläder, men kamoflage i glada färger med ett skaderisk bakrund är helt rätt.

}

\small{
\textbf{Exegetik}
\label{8fb112a8a1a96830126c82d370087677}
 Krångligt ord men om man slår upp det så betyder det läran om tolkande av heliga skrifter. De flesta som håller på med detta dividerar förmodligen om det ska vara punkt eller semikolon \textsc{(s.~\pageref{a6e5810f9ad5798914f30165eba44dcb})} i slutet av någon bibelsida \textsc{(se bibeln s.~\pageref{7de7d2a7d608c9a2044f50688bc63e27})}. På Nissepedia \textsc{(s.~\pageref{62400dadecd90cb5cd39062abe5a3e4a})} gör sig dock exegetiken främst påmind i HratvinnFlygurs \textsc{(se Användare: HratvinnFlygur s.~\pageref{26c5d96dca8dfce84752fa1d4095fdb0})} utredning om det politiska i C.C.Rs texter \textsc{(s.~\pageref{ccc008a82515a9d6022c2c205b15df5a})}.

}

\small{
\textbf{Extremister}
\label{90c376545043c24bec238ce0a3bff5ad}
 Friterat smör.

}

\small{
\textbf{Facebook}
\label{26cae7718c32180a7a0f8e19d6d40a59}
 När folk talar om sociala medier på nätet är det med största säkerhet facebook de menar. Att förakta facebook är något väldigt fint som berättigar hataren till avsevärd social status. Att hata facebook är 2010talets svar på den elisabetanska tidens bleka hud och komplicerade håruppsättningar och visar liksom dessa statusmarkörer att man inte delar andras behov (i det elisabetanska fallet att kunna kroppsarbeta och i facebookfallet social interaktion). Hataren kan mycket väl ändå inneha ett facebookkonto men \quotetext{kommer aldrig ihåg att kolla det} eller \quotetext{förstår sig inte på hur det fungerar}. Han eller hon kan inte förstå varför man skulle vilja veta eller berätta att man druckit morgonkaffe eller gått ut med hunden. Han eller hon vill inte veta av sina gamla mellanstadiepolare och blir bedrövad när han eller hon ser dem på facebook. Han eller hon misstänker att riktiga möten människor mellan allt mer ersätts av virtuell interaktion, som aldrig kan ersätta riktiga möten ansikte mot ansikte och det komplicerade minspel som, påpekar den som föraktar facebook, utgör omkring 80\% av all kommunikation.

 Att förakta facebook är att jämföra med att vara less på julen och alla krav och all kommersialism som den medför.

}

\small{
\textbf{Fagersta}
\label{008e08fd02751800f729d6fa6f75a857}
 Sveriges Ruhrområde befolkat huvudsakligen av amfetaminister, nazister och en icke förringansvärd andel finnar \textsc{(se Finland s.~\pageref{631d44eaa1254ff71a1e11ba021d1266})}. Fagersta har förutom en \quotetext{festival} befolkad av dessa tre folkgrupper, ett bandylag samt en fabrik för bake-offbröd.

}

\small{
\textbf{Fagersta-Posten}
\label{e879bdcb386d850bb5606058db7464d4}
 Lokalblaskan för Fagersta, Norberg och Skinnskatteberg. Kravet för att få anställning är att kunna rimma på Ingrid och Sture. Trots sitt skrala innehåll har tidningen ändå ett stort antal prenumeranter vilket antas bero på att många människor i norra Västmanland eldar med ved och behöver tändningspapper på morgonen.
 Hösten 2009 tog Fagersta-Posten steget in i 2000-talet och lanserade en hemsida på World Wide Web \textsc{(s.~\pageref{3b7d657e8b7bf25a9d524b60d9bb17df})}.

}

\small{
\textbf{Fakta}
\label{fce663ae73dc87a727148bc3b94d1ffa}
 är en världsberömd musiker och Fagerstas svar på Brian Wilson. Lika stor som Wilson gjorde surfrocken för 50 år sen, lika stor gjorde Fakta surfpunken på 90-talet. Fakta heter Fakta för att han jobbade en tid på en livsmedelsbutik med samma namn. Han kommer få Polarpriset vilket år som helst. Han tycker om morötter och att spela schack.

}

\small{
\textbf{Faktoid}
\label{c2928b1ac8de5e1a9ad34a8ee7236cfa}
 Sanningar som förnekas av förståsigpåare \textsc{(s.~\pageref{ff91afb86ce86124b6a517f3eb37bc18})}, kulaker \textsc{(s.~\pageref{c17322f1f8b87ec8fc35538dbe1e9668})} och andra besserwissers. I folkrepubliker där man insett sagans förmåga att ingjuta kraft i en tappert kämpande befolkning är faktoidbegreppet sedan länge avskaffat och istället gläds man gemensamt åt Idi Amin Dadas världsrekord på 100 meter frisim och Den Käre Ledarens prickfria golfrunda.

 I Sverige representeras faktoidbegreppet främst av Prof. Etiennes \textsc{(se Användare: Prof. Etienne s.~\pageref{a9878d2280e5a39becac8f73d113df91})} självbiografier och Svottos \textsc{(se Svotto s.~\pageref{a54b74d16960ccfdc5c60c57fb0fe954})} akademiska kometkarriär. Visst kan man ha invändningar men haters always gonna hate.

}

\small{
\textbf{Falafel}
\label{b2d6ec45472467c836f253bd170182c7}
 Hoppressat sågspån stekt i olja. Anses vara en maträtt i orienten.

}

\small{
\textbf{False metal}
\label{813434993d4a8dd61692a7e8c09de959}
 är metal som är falsk, till skillnad från till exempel true Norwegian black metal. Kanske har metallen spelats på radio?

 Enligt en enig frikår så stavas true: \quotetext{No synths, no pedals, no wimps - Just Power, Metal and Might.} \textsc{(se Manowar s.~\pageref{ac62eaec6dc3e81da86dfbb5252c0ffc})}

}

\small{
\textbf{Fariséer}
\label{4503817c4c5816b5f30fffc91f66ac28}
 Robert är farisèernas konung. Den som vill veta mer om detta folkslag kan lämpligen läsa Gamla Testamentet istället för creddiga \textsc{(s.~\pageref{95574174323e4f0cbd25b65f6c811f26})} superhjälteserier \textsc{(s.~\pageref{c684b269a6e2112d19fe6ea6b203689d})}.

}

\small{
\textbf{Fasta nycklar}
\label{ad577d76d7747bfd314d442197fc8587}
 det är riktiga grejer det. Inget jävla larv med med skiftnyckel \textsc{(s.~\pageref{4eeef21c5cf8813ae82d2882f54c8e28})} som sliter ner gängorna och måste ställas in på nytt inför varje vridmoment. Nej, fasta nycklar ska det vara.

}

\small{
\textbf{Fattiga riddare}
\label{c53ee9c63bb93f36773e3c72dcccb306}
 Ett slags riddarmat bestående av mackor som man hällt pannkakssmet över och sedan stekt i en panna.

}

\small{
\textbf{Fax}
\label{236c3b7f761221f195b428aca2f06c4b}
 en är ett redskap för telekommunikation. Den möjliggör kommunikation i realtid mellan stora avstånd. Ett viktigt faxmeddelande som skickas från London kan, om inget går fel, printas ut ur en faxmaskin i Australien \textsc{(s.~\pageref{e727d8d1b3162a732c7f706d55de64f3})} redan en minut senare. Många olika faxmaskiner har uppfunnits, några prototyper så tidigt som på 1880talet, men den fax som vi idag har svårt att tänka oss att leva utan kommersialiserades under 70- och 80-talen. Den blev snabbt en populär telekommunikationsteknik bland unga urbana människor som är \quotetext{on the go,} som vill \quotetext{vara med} och som värdesätter ett brett informationsnät. Faxen användes till en början mest av större företag men, som står att läsa på wikipedia, \quotetext{numera är fax mellan privatpersoner inte alls ovanligt.} Telefaxen har jämte den minst lika populära digitala kommunikationen bidragit till att skapat en global kultur där information utbyts på bråkdelen av ett par sekunder, vilket inte minst blir märkbart då katastrofer inträffar i avlägsna delar av världen. Då kan nyhetsbyråer få ett fax med lite information om katastrofen och kanske även en bild eller två inom bara några minuter efter att katastrofen inträffat (förutsatt att man inte måste byta papper och sånt).

}

\small{
\textbf{Fejkspons}
\label{6cabd90a6c4b977b30a2f6cce8afb7ce}
 är ett fenomen nära besläktat med dubbel- och trippelmerching och är vanligt bland människor som är lite väl intresserade av märkeskläder. Professionella idrottsutövare, musiker och andra yrkesgrupper som inbegriper att individer blir till varumärken får ofta spons av olika företag. De får då betalt för att gå runt i kläder av ett visst märke så att andra ska köpa likadana kläder. De som gör det kan då råka ut för att råka fejksponsa ett företag, som inte är detsamma som att ha t.ex. en Nike t-shirt, utan att verkligen gå till överdrift. Säg att du gillar märket Haglöfs av någon anledning. Säg att du köper ryggsäck, sovsäck, byxor, fleecetröja, keps \textsc{(se kepsar med olika företagslogotyper s.~\pageref{6a414633590fd4cd6d6ac64798d14c14})}, shorts, plånbok \textsc{(se hästhandlarplånbok s.~\pageref{2f8fbda5296f2f6cab04d88082ed9015})}, axelremsväska samt jacka med Haglöfs karakteristiska logotyp, ja, då är du fejksponsad.

}

\small{
\textbf{Feliks Dzerzjinskij}
\label{6285a286adc7b538013c9a9430c7f252}
 Även känd som Tjekans tjusigaste skägg, grundade densamma i december 1917. Levde rövare \textsc{(se ryska rövare  s.~\pageref{196e0458db510192146b2f885a9a3fee})} hos borgare, kreti och pleti fram till 1926 då han plötsligt dog i en hjärtattack efter att ha hållit ett två timmar långt tal för partiets centralkommitté. Som den professionella kåkfarare han var under Tsartiden fick han tokstryk av plitar och andra lakejer i diverse fängelser och läger, pga detta var hans käke permanent ur led under den senare tiden av hans liv, vilket föranledde en redig fetor ex ore \textsc{(s.~\pageref{d3b96d618fb972d12fb0cdfdeaf13a98})} som var så tilltagen att dess jämlike sällan eller aldrig har skådats.

 Fakta för den konspiratoriskt lagde: född den 11e september, av alla datum, år 1877 i en halv-judisk familj i dåvarande polska konungadömet i nuvarande Vitryssland.

}

\small{
\textbf{Feminism}
\label{44a20d8673cfd6258002acb74ec2f83e}
 är ett slags politisk inriktning som grundar sig på idén att även kvinnor är människor. Tanken uppfanns av sådana som Mary Wollstonecraft och Emma Goldman men möttes med viss skepsis av gubbar som till utseendet påminde om tomten. Sedemera har dock fler och fler anslutit sig till idén om att det vore ganska bra om alla kvinnor i hela världen inte var förtryckta.

 I Kristdemokraterna \textsc{(s.~\pageref{18a843e4776b5003d411ce0148bab148})} håller man inte riktigt med. Ånej, män \textsc{(se man s.~\pageref{39c63ddb96a31b9610cd976b896ad4f0})} är män och kvinnor \textsc{(se kvinna s.~\pageref{9a7760b2521c3471c47cd5d789a2d324})} är kvinnor! ryter man på partikansliet så att det ekar i sakristian och med en sådan inlevelse att Jungfru Maria-figuren, som förärats en plats strax brevid (snett \textsc{(s.~\pageref{64b2eafa388ee50a226adc9013644f08})} bakom) bysten av Josef, tycks gråta av lycka. Eller så är det kanske bara det dunkla ljuset som löper in från medeltidskapellets blyinfattade fönster som spelar en ett spratt \textsc{(se jävelskap s.~\pageref{46845591177f16920cd586a5baf5a625})}.

 Se också: Liberalfeminism \textsc{(s.~\pageref{f3f6938f916c732c921fbeee9d82f818})}.

}

\small{
\textbf{Feminist-knepet}
\label{0630009ea0dc2268ee6b7159e651aa62}
 Feministknepet \textsc{(s.~\pageref{b5a2deaae58d913dc69ee852f19bcb17})}

}

\small{
\textbf{Feministknepet}
\label{b5a2deaae58d913dc69ee852f19bcb17}
 är att som kille bedyra att man minsann är feminist, med avsikt att få ligga. En A-kurs i genusvetenskap kan vara behövligt för att backa upp påståendet. Kanske en F!-knapp, alternativt en venussymbol.

 HEAD2: Berömda exempel
 Socialdemokraten Göran Persson gjorde detta knep för att försöka ligga med hela svenska folket. Han lyckades med Anitra Steen i alla fall. Halva den manliga delen av Umeås punkscen är ett annat exempel. Manliga SSUare har inte mycket annat än detta knep att förlita sig till, vilket är en förklaring till den dåliga tillväxten i partiet.

 Se även: Kvinnlig författare-knepet \textsc{(s.~\pageref{2df7cf3cc32dd55b7c833e6220d42c4a})}

}

\small{
\textbf{Femma}
\label{d974e0811fe7a4d49a9062d33b66a88d}
 En femma är ett vanligt viktmått när man köper hasch.


 Se även: Etta \textsc{(s.~\pageref{ba48f6c4097b7fc25ca11f1e544842d7})}, Tvåa \textsc{(s.~\pageref{84fcc0494ecf9f5af79fcd9bed184a9a})}, Trea \textsc{(s.~\pageref{6f94fdf535ab2e21147ea40ea920ca75})}, Fyra \textsc{(s.~\pageref{7bdb5385ce8e0b1cbc7c15b1d71e8e7d})}, Sexa \textsc{(s.~\pageref{4b1fabe53857b0a2ace6ae22008fe13e})}, Sjua \textsc{(s.~\pageref{e7bf63fa6d0d29bd89c23f833b979a15})}, Åtta \textsc{(s.~\pageref{6fa68b0d02ec525fa72a51c13e5e3ed1})}, Nia \textsc{(s.~\pageref{04a481486dd84d7c8bfdfc89d38136a6})}.

}

\small{
\textbf{Femtusen invånare-regeln}
\label{a7b03b6f44d69e9ad2ef37c307ef75a7}
 Alla som bor i ett samhälle med exakt femtusen invånare eller mindre måste heja på varandra, oavsett om de är tjenis eller ej.

}

\small{
\textbf{Ferry svan}
\label{54e7a6e83983cd24baa74644fd9b6376}
 är son till Gunde Svan \textsc{(s.~\pageref{f80f1875ab3ebccf935723ba83b6da63})}.

}

\small{
\textbf{Fet och grisig mat döpt efter lyxiga ställen/personer}
\label{f4e4d985528ce8d2da975e2a5cca4146}
 är ett klart uttryck för klassförakt. Tegaren \textsc{(se tegare s.~\pageref{61a9e94d20a0e011579891609fa7d765})} i Umeå är en korv i bröd med alla dressingar gatuköksinnehavaren kan uppbåda på, döpt efter stadens finare kvarter. Parisaren \textsc{(se parisare s.~\pageref{5aca28013b9a7e4088e7fb228f3e4827})} i Skellefteå, en svettig korvslant döpt efter den mest fashionabla staden i världen. Wallenbergaren är en svensk klassiker. En pannbiff gjord i stort sett uteslutande på grädde, döpt efter en förmögen häradshövding. Plassarn \textsc{(se plassare s.~\pageref{12950d5d65bc221b46c02ba5d3a89bcf})} är en i Arjeplog förekommande maträtt bestående av underdelen av ett hamburgerbröd, en stor parisare \textsc{(s.~\pageref{5aca28013b9a7e4088e7fb228f3e4827})}, två mosklickar samt räksallad. Däremot finns ingen byssare \textsc{(s.~\pageref{99317503481e8bdd90e670c6c43f6fdf})}, men det torde vara en luffarmacka \textsc{(s.~\pageref{0f7c04cb797f372ed3f219808e0af5b3})}.

}

\small{
\textbf{Fetma}
\label{2af6f57e1ea093fac100010c5c67af6e}
 kan orsakas av åtminstone två av de sju dödssynderna. Fetma blir allt vanligare i världen. I somliga kulturer hålls fetman för något åtråvärt. Nordkorea valde exempelvis nyligen en fet ung man till högste kamrat på just denna merit. I takt med att fetman blir allt vanligare så blir bristen på en tydlig terminologi tydlig. Att enbart säga: \quotetext{Du vet han den tjocka} är inte längre tillfylles. Ett anspråkslöst förslag således:

 \textbf{Barnfetma} är vanligast hos barn \textsc{(s.~\pageref{5dfcc0aab2f3db925b2d51ba73e48946})}. Den svåraste skiljelinjen är här mot mobbarfetman men vanligtvis så brukar ett par slappa handleder avslöja innehavaren.

 \textbf{Blekfetma} grasserar på de brittiska öarna, Sydafrika, Förenta staterna och Norge. Helt okänt öster om Åland. Utan fordon förväxlas ofta en blekfet man med en mopedfet dito.

 \textbf{Mopedfet} äro den ståtlige man som kombinerar sin fetma med ett offentligt liv på en moped. Mopedens storlek och modell är av stor betydelse för mopedfetmans gränser. Onslow i \quotetext{Keeping up appearances} skulle exempelvis vara mopedfet på en elegant Crescent kompakt, i fetmans limbo på en svart Zündapp och endast blekfet på en trimmad DT.

 \textbf{Bögfetman} ligger i huvudsak i betraktarens ögon. Växer ofta med antalet akademiska poäng. Ofta som barnfetman fast hos vuxna män.

 \textbf{Oi!fetton} hittas normalt i polarens inrökta etta \textsc{(s.~\pageref{ba48f6c4097b7fc25ca11f1e544842d7})} mellan alkisschäfern \textsc{(se alkisschäfer s.~\pageref{347febbc28041eae88556d2e7ced587b})} och traven med pizzakartonger. Avgörande är kammobyxorna och huvtröja med pitbull/nazi/antinazi/ACAB/fotbollstryck.

 \textbf{Veganfetma} återfinns nästa alltid hos veganer. Denna typ av fetman har en mycket distinkt karaktär – en tanig kroppshydda kombinerat med en ordentlig bukfetma bestående av 95\% transfett.  Veganer lever i villfarelsen att allt ”veganskt” är nyttigt och äter därför kopiösa mängde Tofuline, chokladbollar och andra onyttigheter som för stunden anses vara en passande diet inom sekten/subkulturen.


 Mopedfetman liksom bögfetman är de sista hundraprocentiga manligt homosociala sfärerna efter det att Liston beklagligt fått sparken från VSK. Hälften av allt fett som finns är kvinns.

}

\small{
\textbf{Fetor ex ore}
\label{d3b96d618fb972d12fb0cdfdeaf13a98}
 \textit{Fetor ex ore} är den latinska termen för dålig andedräkt. Detta symptom orsakas av sådant som lungsjukdomar, gasbildning vid nedbrytning av matrester, en övre magmun \textsc{(se Övre magmunnen s.~\pageref{b0fbb0780611129ae5fc27c88d23d8f3})} som står på vid gavel, skadad tunga eller trasiga tänder, samt rubbad matsmältning föranlett av leverskador. I Sverige \textsc{(s.~\pageref{b1999637949ed135b2ca03f3a38460cc})} är \textit{fetor ex ore} främst associerat med Lars Leijonborg som uppvisar samtliga av ovanstående åkommor, och mer därtill.

}

\small{
\textbf{Ffff}
\label{ece926d8c0356205276a45266d361161}
 Fjärdingsman, flaskan, flickan, fogden.
 Slutligen Fan själv.

 Detta skall man passa sig för.

 Källa:\textit{Ung lantmans tänkebok}

}

\small{
\textbf{FFSSB}
\label{c708e7f1f6d118408fc77e3517417d69}
 Föreningen för storspovens bevarande.
 Arrangerande tidigare fredagsyran i Åkerby.
 På senare år har föreningens aktivitet avtagit.
 Föreningens logotyp visar en Storspov \textsc{(s.~\pageref{9a741bd370edc42a6ff0daff656e4267})} i profil

}

\small{
\textbf{Fi-Sci}
\label{2fd0567d3f772dc8d3971dda9eef1f54}
 är en förkortning av begreppet Fictive Science. I de flesta fall avses en skönlitterär inriktning sprungen ur lika delar science fiction och Illustrerad Vetenskap. Och i viss mån ur pyramidmystik. Nån smart snubbe insåg att det huvudsakliga läsvärdet i både science fiction och populärvetenskap inte låg i om saker var sanna, och kanske till och med användbara. Utan i om själva formen för det skrivna hade en naturvetenskaplig resonemangsstruktur och om innehållet var spännande. Inte så noga om en dinosaurie hade fyra ben eller åtta, eller om den egentligen någonsin funnits eller inte. Bara den var hemsk, farlig och hade upptäckts i form av (STORA!) benknotor i en grop, uppgrävda av universitetsutbildade paleontologer.

 Utifrån denna insikt började den smarte snubben att skriva låtsasvetenskapliga böcker och artiklar som sålde som smör. Fler hakade på, och sommaren 2018 formligen exploderade genren. Särskilt boken Myrornas Hemliga Sexliv av Jonathan Bergom, till stora delar inspirerad av markis DeSade, toppade alla bestsellerlistor i USA och sydostasien. Till de skriverier som i efterhand klassats som tidig Fi-Sci räknas bland annat oljelobbyns vetenskapliga bevis för att växthuseffekten inte fanns. Något som vi nu alla, på 2200-talet, vet var helt fel, eftersom vi alla fått gälar.

 Ett av de stilistiska kraven På Fi-Sci är att det ska framställas som om det vore dagsens vetenskapliga sanning, och att det ingenstans i det publicerade ska ens antydas att det är hittepå. Fotnötter och referenslistor får gärna finnas. En av klassikerna rent stilistiskt är American History Revised av Elina Cardigan, där författaren hackade bl.a. USA:s Library of Congress databas så att alla påhittade referenser såg helt legitima ut när någon försökte kolla. Hon fick så småningom sex månaders fängelse för detta, men å andra sidan tjänade hon fjorton miljoner dollar på boken, deltagande i nätshower med mera. En tumregel för om ett verk ska räknas till högkvalitativ Fi-Sci i stället för skräpdito är att minst 50 \% av läsarna ska börja tro åtminstone litegrann på svamlet. I fallet AMH trodde under en period nästan 70 \% av den amerikanska befolkningen att George W Bush i hemlighet var transvestit och dessutom avlönad av Saudiarabien. Att det senare under Saudiarabiens demokratiska revolution 2064, vid offentliggörandet av säkerhetstjänstens arkiv, visade sig ha varit sant kommenterades av den då nittiofyraåriga Cardigan så här \quotetext{I knew there was something odd about him.}

}

\small{
\textbf{Fiacre}
\label{2b3b64fc8d4f2a049a089081d0fe96ec}
 är taxichaufförernas skyddshelgon och föddes på Irland på 600-talet. Detta är faktiskt, i motsats till mycket annat här på nissepedia \textsc{(s.~\pageref{62400dadecd90cb5cd39062abe5a3e4a})}, sant.

}

\small{
\textbf{Fikum}
\label{c37bb9b6e46bd149b59d007f9b8b88fb}
 är det kufiska men charmiga namnet på fiket som ligger i Umeå universitets svettigaste hus, teknikhuset. Fiket drivs av finniga studenter med gravitationshämmad hållning (Gargamelrygg) som säljer för varmt kaffe i för tunna \textsc{(s.~\pageref{00f1b109163e2c7e424e60cda2354c55})} behållare. Fikum är det enda fiket på universitetet som tillhandahåller Joltcola. På markplan i teknikhuset, just nedanför Fikum, finns en massa stolar och bord där folk kan sitta \textsc{(s.~\pageref{123c3e95c62201513a344526a2fec502})} och fika och ha intima tête-á-têtes. Olyckligtvis gillar inte folk som går på teknikhuset intimitet, utan föredrar att sitta och slurpa i sig Jolt vid datorer och kommunicera med sina vänner genom dessa. På grund av denna speciella omständighet sitter det aldrig någon vid dessa bord och ingen handlar på Fikum. Den mänskliga verksamheten i fiket för i dagsläget en tynande tillvaro i skuggan av varumaskinerna runt hörnet, vilka på grund av dess maskinella egenskaper aldrig behöver stänga eller gå och bröka \textsc{(s.~\pageref{60862d3b986c7bbedc86064c842c5a6c})}.

 HEAD2: Globetrotter XL


 Nere vid de aldrig använda borden står en annan sak som skulle kunna vara centrum för vilket vettigt hus som helst, men som står bortglömd i ett hörn i teknikhuset - Ett Volvo lastbilschassi av modellen Globetrotter XL. Man ser aldrig någon sitta och fika i chassit, eller ta lustiga posebilder framför det.

}

\small{
\textbf{Filateli}
\label{809d75ab9fac299d14f4518f0f53ceee}
 Den renaste formen av kärlek.

}

\small{
\textbf{Filipinsk apörn}
\label{33d996c9277047f9a54e50c2222031cb}
 Den filipinska apörnen är en blandning mellan örn och apa och ställer till förtret i Filipinerna genom att segla runt i skyn och in i hus och bodar och skrika för full hals. Den filipinska apörnen är enligt  auktoriteterna på ämnet \textsc{(se Användare:Hawaii-kråkan  s.~\pageref{9f4dab78598c396bc8790d9043576dab})} utrotningshotad. Konstigt nog.

}

\small{
\textbf{Filur}
\label{e308f4e2553faf188385f17ebda05242}
 är en förkortning av Flying Innovative Low-observable Unmanned Research vehicle. Filur är ett litet enmotorigt och obemannat flygplan tillverkat av SAAB. Filur är föregångare till Neuron.

 Saxat från slangopedia: en kille/snubbe som upplevs som lustig i både bemärkelsen som rolig och konstig.

 Synonymer: skojare, spjuver, rackarunge, spelevink, kanalje, krabat, räv, skälm, luv, luver, skalk, galgfågel, spefågel, gök, kurre, lurifax, lurendrejare, rackare

 Vad gör han? Han är verkligen en riktig filur...

}

\small{
\textbf{Finland}
\label{631d44eaa1254ff71a1e11ba021d1266}
 I Finland finns mycket som är typiskt för just detta landskap, så som det finska språket, finländare \textsc{(s.~\pageref{fc472090d678bd6f029cd80792f4a36d})}, läckerheter så som olika sorters rotsakslådor och korvsås, och ett samhälle som i mycket påminner om hur livet såg ut i resten av norden under 50-talet. Inte en mörkhyad står att finna så långt ögat når och män som har någon slags frisyr (dvs inte rakat huvud eller stubb) blir förr eller senare anklagade för att inte vara hetrosexuella, utdragna på gatan och dödade av arga folkmassor. Det finns naturligtvis en förklaring till detta: \textsc{(se etta s.~\pageref{ba48f6c4097b7fc25ca11f1e544842d7})} I det senaste finska valet röstade ca 25\% av finnarna på det högerextrema \quotetext{Sannfinländar}-partiet och under andra världskriget var Finland det land i världen som per capita bidrog med i särklass flest SS-frivilliga till det tredje riket.

}

\small{
\textbf{Finljuga}
\label{4eee5e7eab6f049c4084d3a5161016f9}
 är att samtala under lugna former med en viss grad av fria spekulationer. Samtalet får gärna röra sig obehindrat mellan ämnen också. Med fördel finljuger man sittandes på en bänk i solen. Det kan låta ungefär så här:

 -Troru he bi nå lingon i år?
 -Njaaaa.
 -Nä, prästn' troddent dä häller.
 -Jaså?
 -Jo han hade så fullt opp me annat.
 -Di säger du?
 -Gården å mor sin å påsken \textsc{(se påsk s.~\pageref{f8f0dd13b69a5c8ce56498e750551d3e})}.
 -Ja du hör.
 -Va?
 -Mmmm.

}

\small{
\textbf{Finländare}
\label{fc472090d678bd6f029cd80792f4a36d}
 Akta dig för dem för de är nästan alltid berusade och du kan ge dig fan på att de bär kniv. Men dom pratar som mumintroll och det är skoj. Var femte finländare, mot var tjugonde svensk, är rasist, har det visat sig vid de senaste valen i repektive länder.

 [[File:Meanwhileinfinland.jpg\textbarright\textbarthumb\textbarTypisk finländare]]

}

\small{
\textbf{Finsk inställning till rock}
\label{4ed4fae98186537624898984b82478b4}
 är ett psykologiskt fenomen och en politisk åskådning som är vanlig hos österlänningar. Den finska inställningen till rock är svårdefinierad, men tar sig i uttryck då män med mustasch sliter ner musiker från scenen i festrummet på färjor som trafikerar färjeleden Åbo-Stockholm. Rock, för den finske mannen, är inte detsamma som vi här i Sverige \textsc{(s.~\pageref{b1999637949ed135b2ca03f3a38460cc})} tänker oss det. Rock för den finske mannen är ett slags homosocial sfär vid sidan av vardagens alla krav och bekymmer, i vilken man tillåts bära bandana och supa så mycket att man bajsar på sig. Denna inställning har visats driva många manliga musikorkestrar till att enbart spela covers av klassiska Anthrax-låtar, medan andra reagerat genom att helt hänge sig åt musik som handlar om elaka kvinnor som inte vill gifta sig med finska pappersbruksarbetare \textsc{(se Finsk pappersbruksarbetarkraut s.~\pageref{c8378f1d21b173ee8ef7fa2dc3d7dd6d})} med blond mustasch \textsc{(s.~\pageref{78fe8e02985abb5090cb3f33ac2842d4})}, utan envisas om att vilja till Örnsköldsvik för att där skapa sig ett nytt liv bland stadens \textit{parnasse}. Den finska inställningen till rock har också vissa konsekvenser för män som härstammar från Sverige, och som besöker detta pungformade område, Finland \textsc{(s.~\pageref{631d44eaa1254ff71a1e11ba021d1266})}. Bland dessa konsekvenser utmärker sig att bli utskrattad, knäad i skrevet och därefter mördad.

}

\small{
\textbf{Finsk pappersbruksarbetarkraut}
\label{c8378f1d21b173ee8ef7fa2dc3d7dd6d}
 är en underavdelning till Krautrocken och framförs i första hand av arbetare vid finska \textsc{(se finland s.~\pageref{631d44eaa1254ff71a1e11ba021d1266})} pappersbruk. Genrens två största band är Circle och Pharaoh Overlörd, som har samma medlemmar.


 Se också: finsk inställning till rock \textsc{(s.~\pageref{4ed4fae98186537624898984b82478b4})}

}

\small{
\textbf{Finsk sommarsoppa}
\label{dccef5e6457380ac650792931d845edb}
 3 dl Kosken i en blommig djuptallrik.

}

\small{
\textbf{Finskt sämskskinn}
\label{ecdf6b5129df6ebb83a9b381b4b33553}
 Det sämskaste som finsk.


 {{Utmärkt}}

}

\small{
\textbf{Fischergasse}
\label{43190195ffa61ca2457acd2d7d1267a5}
 är en gata i Wien. Den mest intressanta byggnaden är Fischergasse 1 som är ett bostads och butikskomplex.

}

\small{
\textbf{Fischergasse 1}
\label{1f4408bb948106e325ce16152c2ac131}
 är ett bostads- och butikskomplex i centrala Wien. Byggnaden i sig ägs av en Judisk äldre man som inte är särskilt intresserad av renovering och som Åke skulle säga ser det ut som ett jävla mangel. Lägenheterna är aviga och dragiga. Dörrarna/väggarna är täckta av klotter från de lokala klotter-gängen. I byggnaden kan man hitta olika personer och rörelser. T.ex:

 \begin{itemize}
 \item En leksaks-butik för vuxna som också säljer modeller till krigs-dioramor. (Dock ej Warhammer el. dylikt. De har även ett sortiment av prydnadsvapen som inte går att använda men som ser bra ut.)
 \end{itemize}

 \begin{itemize}
 \item En äkta vapenbutik som säljer knivar + svärd för allt mellan matlagning och mord.
 \end{itemize}

 \begin{itemize}
 \item En bar som heter \quotetext{Ladies in the city}
 \end{itemize}

 \begin{itemize}
 \item Ett kollektiv för papperslösa ungdomar. (Vissa i huset påstår att dessa knarkar men det kvarstår som lösa spekulationer. En person har dock påstått sig sett en sk. \quotetext{avsugning} bakom en hiss i trapphuset mellan två män, men vittnet påpekar att hon \quotetext{inte var helt säker}.)
 \end{itemize}

 \begin{itemize}
 \item En konstnärsbutik som aldrig är öppen men verkar sälja någon typ av målerimaterial.
 \end{itemize}

 \begin{itemize}
 \item En hippie-tant som ibland får besök av en man med onaturligt tät mustasch \textsc{(s.~\pageref{78fe8e02985abb5090cb3f33ac2842d4})}. (Hennes hem fungerar även som ett \quotetext{alternativt} bibliotek som teoretiskt sett är öppet för allmänheten men inte praktiskt eftersom ingen vet om det.)
 \end{itemize}

 \begin{itemize}
 \item Ett kollektiv med tre män/pojkar och en hund. (Dessa studerar till konstnär, advokat, historiker. De har alla sportcyklar och innovativ inredning. Det gör dem till de mest högfärdiga och störiga i huset.)
 \end{itemize}

}

\small{
\textbf{Fiska kräfta med ficklampa}
\label{af8eb656aa88a4dcb43d2acec8ec16a9}
 Det är olagligt att fiska kräfta \textsc{(s.~\pageref{31d4f9ec82e212d1a52dc283f7335710})} med ficklampa, men vad är å andra sidan inte olagligt nu för tiden? Man får varken besikta utan ljuddämpare eller ta bilen hem från Kickis vägkrog. Att fiska kräfta med ficklampa går ut på att man tar roddbåten ut på sjön där det är grunt och lyser med ficklampan på kräftorna på botten. De blir då paralyserade av ljuset och ba FTW?! och då plockar man upp dem.

}

\small{
\textbf{Fiskeredskapsaffär}
\label{1b1aa77debacc344b3c1342e51abfb55}
 är affärer som tillhandahåller redskap användbara vid fiske och efterföljande tillredningsprocesser, så som rensning och rökning. Fiskeredskapsaffären drivs i nittionio fall av hundra av en manlig föreståndare vid namn Kjell-Åke. Fiskeredskapsaffärs-näringen bidrar med cirka fyra \textsc{(s.~\pageref{7bdb5385ce8e0b1cbc7c15b1d71e8e7d})} och en halv procent av Sveriges \textsc{(se Sverige s.~\pageref{b1999637949ed135b2ca03f3a38460cc})} totala bruttonationalprodukt, vilket ungefärligt motsvarar den skattefinansierade infrastrukturella uppbyggnaden i den norra halvan av landet (inräknat anläggandet av tågbanor, vägar, broar etc). En av de många anledningar som föreligger detta näringsområdes höga omsättning är att konsumenter ges incitament \textsc{(s.~\pageref{f9896a922c4b9345ceebc37009eaf545})} till relativt stora investeringar i spön, agn och tillbehör genom att en Abu Garcia-keps \textsc{(se Abu Garcia s.~\pageref{ebb8e709f4430ad487471fd1acdf28e2})} utan avgift byter hand om konsumenten handlar för mer än en viss, ganska hög, summa. Då Abu Garcia-kepsar är ett av de mest eftertraktade bytena inom det moderna sportfisket ökar denna försäljningsstrategi inte sällan omsättningen med över fyrtiotusen miljarder \textsc{(s.~\pageref{c2160bffc9c5ca88e77204672e62e489})} procent.

 se även Kepsar med olika företagslogotyper \textsc{(s.~\pageref{6a414633590fd4cd6d6ac64798d14c14})} och Abu Garcia \textsc{(s.~\pageref{ebb8e709f4430ad487471fd1acdf28e2})}

}

\small{
\textbf{Fisljud}
\label{08e848734dd882d0d6fd88a39f2e1543}
 Att göra fisljud med händerna var en mycket vanlig sysselsättning bland pubertala glopar \textsc{(se glop s.~\pageref{aae22e6d62a99e31db1de383aa15e538})} på landsbygden; före internet \textsc{(s.~\pageref{c3581516868fb3b71746931cac66390e})} och spelkonsoller ledde ungdomen bort från brännvin \textsc{(s.~\pageref{ff49ececa32cff978496a39635496f46})} och a-traktor \textsc{(s.~\pageref{68eb4e0240edaaca3face5a1ee84e9ac})}.

}

\small{
\textbf{Fistula brothers}
\label{a5171987d9e6dfab3b2657e4f34b1079}
 var en svensk popmusikgrupp som skapades och leddes av den mytomspunne entreprenören Prof. Etienne \textsc{(se Användare: Prof. Etienne s.~\pageref{a9878d2280e5a39becac8f73d113df91})}. Gruppen låg sommaren 1998 i topp på trackslistan med hitten \textit{Human Golf Course} och gjorde bejublade framträdanden bland annat på Vatten- och Arvikafestivalen, men försvann från rampljuset efter att Aftonbladet publicerade komprometterande bilder på en högröd och saliverande Prof. Etienne i famnen på Deepak Chopra, vilket så förargade Agneta Sjödin att hon utförde en fatwa över vår hjälte. Likt Salman Rushdie gick Etienne under jorden och flyttade från den ena medelstora svenska staden till den andra. Under de år han höll sig gömd skrev han sin hittills mest bejublade bok, \textit{Ett lönnfack av kött}.

}

\small{
\textbf{Fixed gear metal}
\label{27ce18b6dbf6db493c219c44d832c913}
 , eller hipstermetall (numer även Stockholmsprofilsmetal) \textsc{(se Stockholmsprofil s.~\pageref{daaee4666c210c7a40537c2399f01556})}, bör undvikas i den mån det är möjligt. Fixed gear metal spelas av sådana band som hipsters, som ofta cyklar på fixed gear bikes, lyssnar på vilket gör det till en förvirrande och delad upplevelse att själv lyssna på dessa band. Hipstern lyssnar inte på någon annan metal, utan väljer ett eller två band som han/hon sedan påstår sig ha lyssnat på sedan Moses låg i vassen. I resten av skivhyllan förvarar han/hon dubstep, postrock och hip-hop \textsc{(s.~\pageref{66c22415908267e727d3fa4a63c16672})} etc. Exempel på fixed gear metal är The Sword och SunnO))).

}

\small{
\textbf{Fjäriln vingad syns på Haga}
\label{9d56dfc3badeeecc2cfdab9095057706}
 \textit{Fjäriln vingad syns på Haga} är en sång, älskad i de breda folklagren, som handlar om motorcykelgänget Rainbow Riders \textsc{(s.~\pageref{54b5b4739e6bc150148c5019e1793413})} som håller till i stadsdelen Haga i Umeå. \quotetext{Fjärilarna} som omnämns i titel såväl som åtskilliga gånger i själva texten är en metafor för de motorburna unga männen i MC-gänget. Låten sjungs ofta i samband med skolavslutningar och folkparksevenemang runtom i landet.

}

\small{
\textbf{Fjärlin vingad syns på Haga}
\label{1c39dd6292f5d6d880ad08c30fea03a9}
 Fjäriln vingad syns på Haga \textsc{(s.~\pageref{9d56dfc3badeeecc2cfdab9095057706})}

}

\small{
\textbf{Fjärrjörgen}
\label{e0e87f2c00b4e598c023f7141f13ec3e}
 I en tid då det är viktigt att kalla saker vid sitt rätta namn, så vill vi härmed råda bot på missuppfattningen att fjärrkontroll, fjärrdosa, dosa eller fjärr skulle vara rätt benämning på den manick man använder för att styra något på distans. Det heter fjärrjörgen och inget annat.

}

\small{
\textbf{Fjång}
\label{d058207f6f6c2dd16de4a817c918185a}
 När ett djur blir upphetsat fylls dasen med blod. Dasen ställer sig då rakt ut (ibland blir den dock inte fullt så hård och reser sig bara halvdant). Detta för att penetrera en hona (eller vad som helst egentligen). Denna hårdhet av dase kallas fjång, andra ord för samma fenomen är bånge och erektion.

}

\small{
\textbf{Flaka}
\label{00d4da353ae52741c858f2fcfa77c8ba}
 Att flaka är att snatta nånting på en affär med hjälp av en kundvagn. Man placerar det man vill snatta, låt oss säga en back lättöl, under kundvagnen. Sen handlar man som vanligt förutom att man \quotetext{glömmer} att påpeka artikeln under kundvagnen vid kassan. Gamla människor gör det här utan att veta om det.

}

\small{
\textbf{Flanera}
\label{79666248bf957d16323b5dd4cc625548}
 Att flanera innebär att man utan någon direkt agenda går runt. Ofta tillsammans med någon man har ett amoröst/erotiskt \textsc{(se erotik s.~\pageref{972f097461d1eab1c1ff104757bad922})} intresse av. Flanören omgärdas av ett romantiskt skimmer då begreppet förknippas med ungdomar. Om man ser ett par äldre personer gå runt utan mål och syfte säger man snarare att de stryker omkring alternativt lurpassar än att de är ute och flanerar.
 Än så länge har inget etymologiskt samband mellan flanera och materialet flanell kunnat påvisas, men evighetsakademikern Prof. Etienne \textsc{(se Användare: Prof. Etienne s.~\pageref{a9878d2280e5a39becac8f73d113df91})} kungjorde i en intervju i Femina (Nr. 3 2009) att han inte skulle sova en blund innan han publicerat ett paper som grundligt redovisade kopplingen orden emellan. Vad som dock är klart är att det är nära besläktat med att tomgå \textsc{(s.~\pageref{736a242ce397e36da48e96a1015406cd})}.

}

\small{
\textbf{Flatologi}
\label{f56f2b86c16ace6216b4652c62b7afdc}
 (lat. \textit{flatus} ung. \quotetext{vind}, \quotetext{vindpust}) är den gren av den moderna medicinen som studerar flatulens, det vill säga pruttar. Många tycker att flatologi och flatologer mest är något att skratta åt, men man sätter lätt skrattet i halsen då man ser de ledsna ögonen hos en ung vegan \textsc{(s.~\pageref{792fec82e3a0dcea1817fd9ebfaf1533})} som inte kan ta det minsta steg utan att det slinker ut en liten rackare. Förutom att dag och natt arbeta för lindring hos denna krisgrupp är flatologins vita val och det världens mest skärpta flatologer arbetar för att lösa det välbekanta problem som uppstår då två människor som står inför möjligheten att inleda en relation tittar på film tillsammans. Liksom mången gång inom astrofysiken har man vid ett flertal tillfällen trott sig ha löst problemet, bara för att när den första uppståndelsen lagt sig inse att man missat någon viktig detalj. Redan Linné \textsc{(se Carl von Linné s.~\pageref{5e8380bf6b7ce99678e6752b6d9e709e})} påstod sig till exempel i sin \textit{Flatologia Laponika} (1742) ha kommit på en lösning, men denna avfärdades ganska snart av mustiga tyska vetenskapsmän med skägg. Under kalla kriget talades det om en kapplöpning mellan vetenskapsmän i öst respektive väst om vem som först skulle lösa problemet och det har spekulerats om att man i Sovjet faktiskt kom på en lösning som sedan hölls hemlig och som föll i glömska i och med kalla krigets slut. Som vanliget ligger sanningen bakom ogenomträngliga rökridåer. Tills vidare är man i både öst och väst hänvisade till det gamla beprövade knepet att börja röka så man kan gå ut på balkongen och prutta i smyg. Bäst läge har som så ofta dansken, för i vårt tomtenisseformade grannland \textsc{(se danmark s.~\pageref{5331d7fd27772396f412a5b6d19bad44})} i sydväst anses det artigt att ligga i soffan och prutta lite slappt medan man kollar med ett halvt öga på \textit{Dr. Dolittle} (1998) med Eddie Murphy och käkar wienerbröd.

}

\small{
\textbf{Flerväxlad cykel}
\label{cd75a1ec5d4b7caabeaaaf25edee0250}
 Ett väldigt åtråvärt fordon som under nittiotalet gav upphov till en kapprustning där cyklister tävlade i vem som hade flest växlar. Den gamla innernavsväxlade varianten tillät upp till femton, det senare systemet med utvändiga kedjeväxlar såg betydligt tuffare ut och tillät betydligt högre antal växlar. På en del flerväxlade cyklar måste man sluta trampa när man ska växla, medans en del är precis tvärtemot.

 HEAD4: Erfarenheter av turer på en blå tjugoettväxlad mountainbike
 \begin{itemize}
 \item Cyklar man på ettans växel är det lätt att cykla på bakhjulet, men man ser ut som ett slan \textsc{(s.~\pageref{caaad522de864ab45ed679c4a16edd8d})} för man måste trampa så otroligt snabbt.
 \item På tjugoettans växel så kan man komma upp i skrämmande höga hastigheter, men hamnar man i en uppförsbacke får man växla ner för annars finns en överhängade risk att man blir ansträngd.
 \item Nånstans kring elfte växeln är bäst om man bara vill glida runt.
 \end{itemize}

 Micke på Brånet hade flest växlar. Han sa att han hade en femtioväxlad cykel, men alla visste att det inte låg ett uns \textsc{(s.~\pageref{07525332b1617934911c9fbadb3a304e})} sanning i detta. För det var egentligen en femväxlad cykel som han ritat dit nollor efter varje siffra på.

}

\small{
\textbf{Flickorna på TV2}
\label{4e47673a893d7126c5518a132d1fb71b}
 är en sång av världsartisten Per Gessle. Den handlar om att han tycker de kvinnliga programpresentatörerna (populärt kallat \quotetext{hallåor}) på Sveriges andra TV-kanal är jävligt snygga. SVT tycker låten är så dålig att man nu bestämt sig för att avskaffa sina programpresentatörer för att förstöra Per Gessles liv.

}

\small{
\textbf{Flisbil}
\label{89900467e74c1de354e483c90b816b0e}
 är en lastbil som transporterar flis. Är man riktig snål och vill ta sig mellan Malå \textsc{(s.~\pageref{41da4620e87888eaaeafcb3004a8d177})} och Piteå \textsc{(s.~\pageref{db694f60fd74ffa986e086d8e29f73dd})} kan man alltid åka med en sån.

}

\small{
\textbf{Flodkanin}
\label{5b2fd3512fae865f843dfe95c778fa07}
 (\textit{Bunolagus monticularis}) är en kaninart som bara finns i Sydafrika. Den är väldigt utrotningshotad, typ bara 200 individer kvar, vilket antagligen beror på att det är en dålig kombination att vara kanin och bo i en flod i ett land där det finns vithajar \textsc{(se haj s.~\pageref{00ed0f1d6fa0f51775d9fd969adb4e3b})}.

}

\small{
\textbf{Flundra}
\label{0895eb8daa86f82a68575359cb48cfc6}
 n är en fisk som till skillnad från alla andra fiskar bor i halsen på Ernst Billgren \textsc{(s.~\pageref{788717210e9662a9365efdbc9094f936})}.

}

\small{
\textbf{Flytväst}
\label{ad12637b03f6d664b47cff669471387f}
 En \textbf{flytväst} fungerar ungefär som ett bilbälte, fast på vatten.

}

\small{
\textbf{Fläderflundra}
\label{a7bcb065c7d65d219825c737ce6d18fa}
 En fläderflundra är en flundra med en distinkt smak utav fläder. Antingen har den odlats i en fiskodling som vars vatten spetts ut med flädersaft för att ge fisken önskad smak eller så har den skördats från en fläderbuske där det ympats in flundra. Fläderflundra anses vara en delikatess på många håll i nordeuropa och har traditionellt varit en bröllopsrätt i Litauen.

}

\small{
\textbf{Fläsksvålar}
\label{a8f15feead71638a458fbbb29141931b}
 , eller \textit{ flæsgæsvælær } som det heter på danska \textsc{(se Danmark s.~\pageref{5331d7fd27772396f412a5b6d19bad44})} är höjden av dansk kokkonst. Fläsksvålens historia börjar hos bonden Preben Madsen som hade fått höra talas om den nya flugan om pommes frites och bestämde sig för att själv tillverka denna delikatess. Prebens snedsteg/genidrag att istället för att bruka alla skandinaviers favoritknöl potatis så använde han den överblivna och orakade svålen från gårdagens rimmade sidfläsk och stoppa detta i fritösen. Stay classy, Preben. Sedan dess har danskar från Jylland till Fyn \textsc{(s.~\pageref{9854682d71fdb60a819d9188a846f42d})} blivit alldeles till sig när en påse fläsksvålar öppnas och sprider sin karakteristiska doft av hundmat i rummet.

 Fläsksvålar är det livsmedel som stoltserar med överlägset högst andel transfetter om man inte räknar ren ister och kokosfett. Personer med svagt hjärta bör hållas långt borta från denna charkuteri, och detta kan vi på Nissepedia \textsc{(s.~\pageref{62400dadecd90cb5cd39062abe5a3e4a})} inte understryka nog.

}

\small{
\textbf{Flöjt}
\label{15912990d2886bb707954c9a4e933bc0}
 är som en liten visselpipa, men med en massa hål i. Den ligger ofta i ett flöjtfodral \textsc{(s.~\pageref{017a228af6c68112b4b309ac9a4a87dc})}.
 Tuffa hårdrocksband som Hawkwind kan balansera upp sin massiva hårdhet genom ett sju minuter långt flöjtparti. Detta är förbehållet riktigt hårda band,annars blir det lätt musikhögskolemusik \textsc{(s.~\pageref{28e79281eebe4f13098c88a3e51b6b8e})}.

}

\small{
\textbf{Flöjtfodral}
\label{017a228af6c68112b4b309ac9a4a87dc}
 är en det hölje eller den skida vari flöjtisten för ner sin flöjt \textsc{(s.~\pageref{15912990d2886bb707954c9a4e933bc0})} då han eller hon vilar och inte spelar på sitt lilla instrument. Flöjtfodralet uppfanns av den svenske författar'n, assäist'n, krönikör'n och debattör'n Oskar Sverre Lucien Henri Jan Guillou \textsc{(s.~\pageref{63f2c8aba9686bc92efeb7eb21e35156})}.

}

\small{
\textbf{Fnysning}
\label{27e6670d7ee5854c2acaa2b7a49d4975}
 Folkpartistisk \textsc{(se folkpartiet s.~\pageref{b692fa6a23fd557940474dc94909d80f})} nysning. Sällsynt tråkig, och rensar inte näsan alls.

}

\small{
\textbf{Fobi}
\label{deaf5f7387941b1c8f557f135d4c370a}
 Som att oroa sig i onödan, fast på anabola. Fobier kan ofta kopplas till traumatiska upplevelser. Man kan ha fobier för nästan vad som helst, till exempel Calskrove \textsc{(s.~\pageref{84ff54e779ee49fdad21e17c20f14453})} eller Brugd \textsc{(s.~\pageref{d6b6b68506b8f1daad3a2ddbfaf8d863})}.

 Motsatsen till fobi är Ibof \textsc{(s.~\pageref{6d7d119f17cba7cc9edb4448587cf1f5})}.

}

\small{
\textbf{Folk födda före 1970}
\label{40153cfa2d7f1d2926c3daa1b27705ed}
 Denna kategori människor har alla gemensamt att de passerat 40-strecket. En gemensam egenskap som särskiljer dem från yngre människor är att de talar in röstmeddelanden när någon de ringer inte svarar på mobilen. Vad detta beror på kan den moderna vetenskapen inte svara på, men den har gett oss så mycket annat så ingen ska börja gnälla bara därför! Det finns såklart en nedre gräns för denna praktik och det är folk som aldrig brytt sig om att lära sig hur en mobiltelefon fungerar för att de ändå har en fot i graven.

}

\small{
\textbf{Folke Pudas}
\label{d9f740fcf09e3cbdcaba15cf92783958}
 är en stor man

}

\small{
\textbf{Folketinget}
\label{3e67fc9ae7590cbd0067b5015308da46}
 är Danmarks \textsc{(se Danmark s.~\pageref{5331d7fd27772396f412a5b6d19bad44})} parlament. Där bestäms vilka lagar som ska gälla i Danmark, om man ska försöka erövra Skåne en gång till, vilka artister som ska bokas till Roskildefestivalen och andra, för danskar, viktiga frågor.

 Vilka som ska sitta i folketinget bestäms genom allmänna val där alla danskar skriver sin kandidat på en papperslapp som läggs i en Tuborg-flaska \textsc{(se Tuborg s.~\pageref{49bb0f04b9993881c9d9c5b115cc35f0})} och slängs i närmsta glasigloo. Jens Spendrup vinner varje gång men han är upptagen med att tappa upp öl så han avsäger sig alltid uppdraget.

 Namnet folketinget kommer sig av att vaktmästaren som släcker och stänger där varje kväll heter Folke.

}

\small{
\textbf{Folkhjälte}
\label{a7f0eb3e0e8ca199c745f5c9ea550404}
 En folkhjälte är någon som gjort något storslaget för ett folk, gärna mot det folkets förtryckare och utsugare.

 Exempel:

 Engelbrekt Engelbrektsson
 Folke Pudas \textsc{(se Pudaslåda s.~\pageref{6a56958e2057dd500650e2be8049e033})}
 Sigvard Thurneman \textsc{(s.~\pageref{f9661f47746535d9b19e7f86bbf41dbd})}
 Pelle svensson \textsc{(s.~\pageref{26d88b383fd38f349c7741ca7051904e})}
 Anckarström
 Eva Ekeblad \textsc{(s.~\pageref{de66bb2a3f5c71b15a204f8e773ea925})}
 Frantz Fanon
 Jurij Gagarin \textsc{(s.~\pageref{65a8533cd59bc537e760118da0751c50})}
 Pelle Fosshaug \textsc{(s.~\pageref{b6caa53a9a50eb546517552a5503e323})}
 Ivar Bryntse \textsc{(s.~\pageref{a5e922bb2ad7c32a2419b5ba3afcdc99})}

}

\small{
\textbf{Folkkök}
\label{15983d1934522d4d08e766108357201b}
 Namnet till trots är det inte ett kök för folk utan ett ställe där veganer \textsc{(s.~\pageref{2a12d5d6ae91d2f4f7d9af3cef58e75c})} träffas för att få magsjuka och dra politiskt korrekta vitsar.

}

\small{
\textbf{Folkpartiet}
\label{b692fa6a23fd557940474dc94909d80f}
 är ett parti för förbittrade marxister som tappat tron på socialismen. Många av dess medlemmar är egentligen godhjärtade men orkar inte längre engagera sig utan vill ha snabba totalitära lösningar. Vissa medlemmar kommer dock från annat håll. Jan Björklund \textsc{(s.~\pageref{0b9b757044804b9be0e218acdad358cc})} är exempelvis fortfarande uttalad nazist.

}

\small{
\textbf{Folkölsfylla}
\label{3cebf480c2ec39df0180688bf7c727ea}
 Ett sexpack trefemmor eller två sexpack tvååttor är tillräckligt. Se till att du har toapapper hemma, det kommer att kissas.


 Öppna en vid klockan två (folkölsfylla är ofta synonymt med dagsfylla) \textsc{(se dagsfylla s.~\pageref{e79459471993abd0ccde4df08bafdb22})}, du kanske diskar och lyssnar på p1. Blir lite arg över nån korkad grej men tar en klunk och tänker, jaja c'est la vie antar jag. Den första tar rätt lång tid och ljumnar lite, men det gör inte så mycket.


 Öppna nästa ca 40 minuter efter den första. Nu är du lite törstigare. Känner dig rolig nog för att skriva en artikel på nissepedia \textsc{(s.~\pageref{62400dadecd90cb5cd39062abe5a3e4a})} och börjar fundera på middag. Det blir nog fiskpinnar och spaghetti igen. Hmmm, kanske med remouladsås. Ja, det är gott, tänker du och halsar den sista fjärdedelen.


 Den tredje ölen behöver du snacks till. Vinägerchips småäts samtidigt som du spelar quizkampen i din mobiltelefon, eller dota 2 på din dator. Det går i alla fall skit i båda spelen. Du blir lite sur, men tänker, \quotetext{Life just be that way, I guess.} \textsc{(s.~\pageref{11f000e9e3f96eae83688170fc343ec7})} och sveper vad du har kvar i den tredje burken.


 Nu är det dags att gå på affärn! Du tar med den fjärde ölen och smådricker på vägen. Beroende på hur glad/ledsen du är, känner du dig som en stigmatiserad bärshagga \textsc{(se kalaskula s.~\pageref{e889c1a4915c4b4aad08d49192e79276})}, men nu kan du inte sluta dricka, så det är skit samma. När du kommer fram till din kvartersbutik ställer du din öl på elskåpet utanför och glider självsäkert in. Väl inne förser du dig med fiskpinnar och remouladsås och två folköl till. Dumt att chansa, liksom.


 När du kommer ut har du redan glömt ölen du lämnade på elskåpet. Men vad gör det? Du har ju fler. Nu går det undan. Du klunkar girigt i dig nummer fem innan du kommit hem. Nam nam nam!


 Det råddas en del vid spisen. Troligtvis lyssnar du på Armored Saint eller Tank på helgvolym \textsc{(s.~\pageref{3539fdeb41a5b216f614b6ced9ff5cff})}, trots att det bara är tisdag! Spaghettin kokar fast i kastrullens väggar och fiskpinnarna är rätt svarta när du är klar, men när du står och hojtar [https://www.youtube.com/watch?v=iybhCAznb4I HEAVY ARTILLERY!!!] med en spaghettistump hängandes ur mungipan bryr du dig inte så jävla hårt.


 Nu är du folkölsfull på riktigt. Klockan är typ sex och dina polare kommer snart över. Antingen gnäller du till dig att nån av dem handlar ett sexpack till dig, eller så går ni på lokal och blir fulla på riktigt. Vad som sker från den här punkten spelar i alla fall inte så stor roll - du är framme.

}

\small{
\textbf{Folkölsförädling}
\label{ad8fa31a77abc3e541a73f1ccc7afc78}
 Om man slänger in en folköl i frysen en timme och sen tar ut och dricker jättefort får man samma effekt som av en starköl.

}

\small{
\textbf{Folkölspriser}
\label{88a1df48cd2dd24a91bc440e98851393}
 En av de viktigaste frågorna i samhället som stenhårt ignoreras av politiker  i alla läger.
 Förr fick man ett 16-pack för 70 kr, idag anses samma summa för ett 6-pack helt \quotetext{normalt \textsc{(s.~\pageref{5c455ca1c87070883ff0a4c13ae8937f})}}.

}

\small{
\textbf{Folle}
\label{fe938fafef93bca3ba46995c6d409807}
 \textbf{Folköl} eller den mer populära benämningen Folle, är det man införskaffar sig då man inte  har bärs \textsc{(se Ha bärs s.~\pageref{a74b297c15834437ac2e49095492133c})} och bolaget har stängt. Trots att folle betraktas som öl kan det nästan oavsett mängd som inmundigas vara svårt att åstadkomma en rejäl bärsfylla \textsc{(se Bärsfylla s.~\pageref{9380b60f9ee744b9acf978fe6f1a9545})}.

}

\small{
\textbf{Fondue}
\label{98254ae1bc17c73df8f3d6a47beb333f}
 är för mat på 70-talet vad päronhalva \textsc{(s.~\pageref{cc9c1bfa2ec4eaed89ca86a1b63e3a45})} var för efterrätter under första halvan av 90-talet. Lyx!

}

\small{
\textbf{Forskningsinstitut i Schweiz}
\label{1686d12779e6c4574c24716d01189820}
 verkar bedriva ohemult \textsc{(se ohemul s.~\pageref{91b8873590abd15ec344c2ba93d015cd})} mycket forskning om karies, tandkräm och andra områden relaterade till dentalhygien. Ingen har dock någonsin lyckats lokalisera ett sådant institut, något som tillgängligheten till Google \textsc{(s.~\pageref{c822c1b63853ed273b89687ac505f9fa})} Earth förhoppningsvis kommer att ändra på.

}

\small{
\textbf{Fotboll}
\label{961bd74d34872ff94a4df3a16119096e}
 Tjugotvå män/kvinnor som alla will \textsc{(s.~\pageref{18218139eec55d83cf82679934e5cd75})} sparka på en boll. På gräs eller grus.

 Fotboll är världens största sport sett till antalet utövare, som är så stort att det är något alldeles makalöst, och en uppbygglig fritidssysselsättning. Månghövdad är den skara som tycker om att se på fotboll på TV liksom i verkliga livet och bland de som föredrar det sistnämnda alternativet finns de som gärna passar på att ge motståndralagets ivriga påhejare en dagsedel \textsc{(s.~\pageref{7a59187e5d4c12ded69d05197d099485})}, medelst slagträ eller, helt sonika, en knuten näve.

 Fotboll går till som så att två lag om tio eller, för småglin, sex spelare plus målis med fötter eller andra kroppsdelar föser bollen mellan sig och sedan, helst, in i motståndarnas mål. Man får inte vidröra bollen med händerna såvida inte man är målis, domare eller bollkalle, för då kallas det handboll och det är mycket mindre populärt. Varje gång bollen kommer in i motståndarlagets mål får man en poäng och man tar av sig tröjan och kramas. Sedan blåser domaren i sin lilla flöjt \textsc{(s.~\pageref{15912990d2886bb707954c9a4e933bc0})} och då är det dags att köra igång igen. Så fortskrider matchen i två gånger fyrtiofem minuter och sen sätter sig domaren ner och räknar ihop alla poäng som gjorts och utser ett vinnarlag. Om inget vinnarlag kan utses blir det oavgjort eller så måste man fortsätta att spela eller lägga straffar.

 Exempel på framgångsrika fotbollspelare är George \quotetext{Georgie} Best (bla Manchester United FC), Glenn Hysén (bla Liverpool FC och IFK Göteborg) och, sägs det, Julio Iglesias.

}

\small{
\textbf{Fotbollspundare}
\label{224ef8f932cf1be33943b7503e19b8ef}
 n är ständigt på jakt efter en fix. Fixen i detta fallet är en plan. Är fotbollspundaren riktigt sugen kan planen se ut lite hur som helst: lerig, asfalterad, ojämn och/eller i en sluttning. Är läget mindre desperat är siktet inställt på en fotbollsplan som faktiskt har någon form av mål.

 Fotbollspundarna använder olika tekniker för att få tag på nästa plan. Det rör sig främst om spanare, kontakter eller det kommunala bokningssystemet. Bokningsystemet används dock inte som hederliga medborgare använder det utan för att finna tomma planer. På samma sätt som tjuvar ringer innan de bryter sig in för att se om någon är hemma.

 Fotbollspundarna är inte direkt en utsatt grupp i samhället utan betår i regel av män och något färre kvinnor ur medelklassen och den universitetsutbildade arbetarklassen. De anser sig dock lite för fina och speciella för att vara med i en vanlig förening som faktiskt betalar och bokar tider hos kommunen.

 Missbruket kommer alltid förr eller senare att leda till kriminalitet. \quotetext{Brytfotboll} (alt. \quotetext{brytträning}) är den vanligaste formen av brottslighet. Ett gäng fotbollspundare bryter sig då olovligt in och tar över en plan för att spela. De tar sig in genom att klättra över grinden eller klippa upp låset. I Umeå är Minervaskolans konstgränsplan särskilt utsatt.

 Snabbaste vägen ut ur beroendet är en ordentlig korsbandsskada.

}

\small{
\textbf{Fotbollstid}
\label{3d900fac062e1ecff848efad8875cb66}
 Tid speciellt avsatt för fotbollsspelande, eller den tid som fotbollsspelare avtalat om att mötas för att spela fotboll \textsc{(s.~\pageref{961bd74d34872ff94a4df3a16119096e})}.

}

\small{
\textbf{Fotbollsvägra}
\label{0edd1cff3088f6b1fbc12c3be289aaed}
 Att fotbollsvägra är att inte dyka upp när det är dags att spela fotboll \textsc{(s.~\pageref{961bd74d34872ff94a4df3a16119096e})} fast man borde det (av moraliska, hälso- eller hedersrelaterade skäl). Man kanske har lovat att man ska dyka upp och så gör man rätt och slätt \textsc{(s.~\pageref{a9cde01124ca41f23d6044b3ba27b979})} inte det. Eller så lämnar man landet för att smygträna med gamla yrkesmilitärer på balkan, utan att finna lämpliga ersättare först. Det är inte att rekommendera att fotbollsvägra, speciellt om man spelar i Berghem HC \textsc{(s.~\pageref{72c5e1ef562098496277726ca12aa149})}, eftersom det medför svår social stigmatisering och mycket väl kan innebära slutet på ens karriär, såväl som ens försök att skaffa sig bestående socialt kapial.
 HEAD2: Undantag
 Rent teoretiskt kan det vara godtagbart att utebli från Berghem HCs fotbollstisdag om anledningen härtill är att man ska till Överklinten på surströmmingskalas.

}

\small{
\textbf{Fotografering}
\label{176551844874f34f5bb9a9d0ac93f99a}
 Har du en slumrande kreativ potential som endast får komma till utlopp då det ska sättas upp påskris \textsc{(s.~\pageref{a9d744074ec3fda67c8e7b52801e5178})} och julstrumpor? Då kan fotografering vara något för dig. Med dagens teknik behövs inga dyra framkallningsvätskor och mörkrum och du kan ta med dig din kamera ut i den friska luften. Det första du behöver tänka på är vad ditt motiv ska vara, och som vanligt kommer nissepedia \textsc{(s.~\pageref{62400dadecd90cb5cd39062abe5a3e4a})} till undsättning:
 HEAD2: förslag på motiv
 \begin{itemize}
 \item \textbf{Nedgångna industrilandskap}. Många gånger kan ett förfallet industrilandskap vara minst lika vackert som naturen. Patinan hos ett nedlagt bruk \textsc{(se Nedlagda industribyggnader s.~\pageref{cf7af2374abf8d8d6ccc035c2e0eb3be})} i Bergslagen kan vara ett nog så suggestivt motiv men det krävs att hitta precis rätt bild. Arbeta med kontraster mellan skugga och ljus, och mellan olika ytor, till exempel en betongvägg och rostiga rör.
 \item \textbf{Dig själv}. Ta bilder på dig själv i svartvitt. Du kanske sitter på golvet med huvudet lite på sned och armarna kring benen: Du fäster din lite sorgsna blick på en punkt strax bakom och ovanför kameran. Du kanske är lite i oskärpa. Skapa en blogg med svart bakgrund, The Smiths-citat och enigmatiska uttalanden om dig själv och publicera dina bilder där. Stäng av kommentarsfunktionen för att förhindra att kommentarerna ökar lavinartat.
 \item \textbf{Graffiti}. Ta gärna färgfoton på ytor täckta av graffiti. Genom att göra det blir du automatiskt en gatukonstnär med känsla för storstadens puls. Du är så ett med staden att den uttrycker dina urbant odefinierade tankar och perspektiv på samtiden. Du ställer ut på caféer där det säljs komplicerade, kalla kaffedrinkar och får ligga med attraktiva tjejer som är betydligt yngre än du själv om du är en kille eller flata, eller äldre män om du är tjej eller bög.
 \item \textbf{Dina tokiga punkarkompisar}. Dina kompisar är så spektakulära att de utgör ett outsinligt material. De har saker på huvudet som egentligen är till för något annat. De sitter uppflugna på offentliga skulpturer. Någon har somnat och råkat ut för ett spratt som alla dock kan skratta åt tillsammans i slutändan. Framkalla bilderna och klistra ihop dem till ett collage som liksom får stå för er oupplösliga vänskap utanför samhällets normer och krav.
 \item \textbf{Blad som flyter på vatten}. Bladen representerar oss människor, vattnet livet.
 \end{itemize}

}

\small{
\textbf{Foucaultfingret}
\label{6e91acdb69ab7aaa8df3ad46b3da6f5a}
 uppfanns av den franske filosofen, historikern, sociologen tillika flintbrillot Michel Foucault. Fingret används när man är på väg att lura in en meningsmotståndare i ett intellektuellt bakhåll.

 När du märker att den du debatterar mot är på väg att göra ett utlåtande som du krossat tusen gånger om i din hemmasnickrade akademiska dojo \textsc{(s.~\pageref{9bea15890f18ef35a12767fef5d234b8})}, är det lätt att för tidigt avlossa en salva shurikenvassa motargument. Det bästa i det läget är att istället tålmodigt vänta på att ens motståndare helt fullföljt sitt resonemang innan du skiljer dennes huvud från sin kropp, med hjälp av ditt katanaskarpa intellekt. Annars finns risken att din motståndare undkommer sargad från striden, med möjligheten att återvända senare med sina roninpolare \textit{en masse}.

 Men hur ska du hindra dig själv från att öppna flabben och avlossa alla pilar i din verbala armborstpistol för tidigt? Foucault hade lösningen. När du hör din motståndare börja fälla provocerande korkade utlåtanden, för ditt pekfinger in i din mun och alternera mellan att gnaga och suga på det tills tiden är mogen för att fullkomligt pulverisera din motståndare, utan att ge hen en möjlighet att undfly med själslivet i behåll. Banzai!

}

\small{
\textbf{Framgångsteologi}
\label{0c3b399eedaaef3220f1e59f6d138055}
 är en religiös åskådning inom protestantisk kristen tro och har sitt ursprung i the United States of America, men finns idag också representerad i den svenska frikyrkliga floran. Idéströmningen vilar på två ben, nämligen idén att ett lydigt beaktande av religiösa föreskrifter och normer leder till själslig, men framförallt ekonomisk framgång för individen, samt en stark tro på auktoritär församlingsstruktur enligt devisen \quotetext{one good shepherd.} I Sverige finns ett antal församlingar som bekänner sig till och predikar framgångsteologi. Bland dessa är Livets Ord i Uppsala den mest tongivande och inflytelserika församlingen.

}

\small{
\textbf{Framstjärt}
\label{843a33cdc4d90488cea3030f8b941e08}
 Den enda acceptabla benämningen på det kvinnliga könsorganet. [http://www.svenskaspraket.nu]

 Även herrar/damer som orerar väldigt mycket må ha en ovannämnd framstjärt

 Eg. \quotetext{Klappa igen framstjärten och sluta prata skit} - Olof Palme \textsc{(s.~\pageref{702b78623785546fb9c9890222376178})}

 Göteborg \textsc{(s.~\pageref{0e9b11e435dd9f73e87e868667e1d6f0})} kallas ofta skämtsamt för Sveriges \textsc{(se Sverige s.~\pageref{b1999637949ed135b2ca03f3a38460cc})} framstjärt, mest för att att göteborgarna kallar Stockholm sveriges baksida.

}

\small{
\textbf{Framtiden}
\label{0fcbd779fa61395ff504786a93a70440}
 bär saker i sitt sköte, likt en känguru.

}

\small{
\textbf{Frankrike}
\label{8a28b520a53cd68763ebf19b5599412b}


}

\small{
\textbf{Fransk gubbstoner}
\label{8646993da65aa67720f0c7865e5b6a53}
 Det krävs avancerade instrument för att skilja fransk stoner från gubbrock \textsc{(s.~\pageref{017518791bbb6d1db3fca1e31b678b4d})}. De två genrerna är svåra för en lekman att skilja av ett flertal anledningar.
 1. Som en i publiken övervägder man spontant att handla något man inte vill ha från merchbordet, som ett sätt att säga 'Tro på er själva! Låt ingen säga till er att ni inte kan'.2. Franska stonerband har ofta med sig egen rökmaskin.
 3. De har också allt som oftast tryckt upp turnétshirts med tourdates på ryggen. 4. Up close är det inte ovanligt att sångaren är till förvillelse lik Dominique Pinon. 5. Emperi visar att franska stonerband inte reflexmässigt skyggar tillbaka för användandet av sliderör.

}

\small{
\textbf{Franska svordomar}
\label{b842b5b380758ed7e123daede577617b}
 Frankrike \textsc{(s.~\pageref{8a28b520a53cd68763ebf19b5599412b})} har världens lamaste svordomar. Den vanligaste är \quotetext{sacré bleu} som betyder \quotetext{heliga blå}, vad fan är det liksom? Önskar man att det ska börja regna på sin fiende, eller att hon/han/det ska ha slut på varmvatten när det är dags att bada, eller vad är grejen? En annan fransk klassiker är \quotetext{baguette moisis}, som betyder ungefär \quotetext{din mögliga baguette}. Gud va jobbigt att bli kallad det... inte...

 Vill man lära sig att svära på ett annat språk vänder man sig istället med fördel till de italienska svordomarna \textsc{(se italienska svordomar s.~\pageref{3a490d7017c99929180c9d80e52e5926})}.

}

\small{
\textbf{Frasses}
\label{971e198d8fef127906319ec98ff657ce}
 är en norrländsk hamburgerkedja utan några större ambitioner. Inte heller personalen på Frasses har några större ambitioner. De är ofta finniga och flottiga också. De är å andra sidan inte hjärntvättade, åtminstone inte vad gäller hamburgare, vilket hamburgerkedjepersonalspersoner \textsc{(s.~\pageref{95a22fdc49f91291ae743f1372bca323})} normalt är, till exempel på den andra norrländska hamburgerkedjan, den med ambitioner. Frasses känns okapitalistiskt på något sätt, eller åtminstone bara lite. Det är rätt skönt.

 Urfrasses låg på Storgatan i Luleå, strax nedanför kyrkan och inte allt för långt från Hermelinsskolan (gymnasium). Där serverades på sjuttiotalet skrovmålen i plåtlådor. Extra dressing fanns bara en sort, och absolut inte i små knapsu plastburkar. Frasses har än idag den godaste orginaldressingen, vilket inte alla håller med om.

}

\small{
\textbf{Fredag}
\label{80d41f1e0b14eacb229eea9618632e88}
 är den dag i veckan när man går hem från arbetet/studierna och skriker \quotetext{BLACK SABBATH!} [http://www.youtube.com/watch?v=dRSGDgyfHTs] och helgen börjar.
 HEAD2: Fredagen inom judendomen
 Ett lustigt sammanträffande \textsc{(se Märkliga sammanträffanden s.~\pageref{f46282d99158f351a81b9deaff157b4e})} är att sabbaten firas just under fredagen inom judendomen.
 HEAD2: Fredag i nyliberalismens folkhem
 I nyliberalismens folkhem utmärks fredagen som den dag då man ägnar sig åt sk. \quotetext{fredagsmys \textsc{(s.~\pageref{bba247fdca80768603373605cbe7a934})},} vilket innebär att barnfamiljen vräker i sig chips och coca-cola och att nyförälskade par äter avokadohalvor med  räkor \textsc{(se räka  s.~\pageref{2e1cdd6fa81f4968c8c527854e0c629b})} på.

}

\small{
\textbf{Fredagslyx}
\label{2508e5844cd43a2aa6b385d72100ca2c}
 En kasse starköl och en falukorvsring.

}

\small{
\textbf{Fredagsmys}
\label{bba247fdca80768603373605cbe7a934}
 är ett påhitt av detaljhandeln i samarbete med nyliberala institut och tankesmedjor \textsc{(s.~\pageref{c276b5997d5af80504f79b30d121cf62})} och går ut på att barnfamiljer sitter i en soffgrupp, kollar på störd TV och smäller i sig chips och läskeblask som det icke fanns någon morgondag. Vissa påpekar dock att detta bara är en kommersiell variant av firandet av slutet av arbetsveckan, vilket har pågått sedan kristendomen institutionaliserades under antiken. Inkluderas även dessa sätt att ägna sig åt fredagsmys finns det många olika sätt på vilka man kan koppla av under fredagskvällen. Några av dessa är att:

 \begin{itemize}
 \item Rulla hatt \textsc{(s.~\pageref{7c7afc9fb7bb52962f954c0cb548c10c})}
 \item Ta av sig byxorna \textsc{(se sans pants s.~\pageref{e690d08a3200d783d98b198f0354bc85})}
 \item Lyssna på rymdrock och röka gräs \textsc{(se stenad s.~\pageref{dec4a3a91f0f2bf8dcf033a8cfeaa554})}
 \item Slåss utanför dansbanan
 \item Ägna sig åt en hobby
 \item Fira den judiska sabbaten.
 \item Spela spel
 \item Laga mat med polarna
 \item Leka charader
 \item Kolla på fotboll \textsc{(s.~\pageref{961bd74d34872ff94a4df3a16119096e})}
 \item Spela på travet
 \item Måla små, små tennfigurer och sedan leka att de krigar mot varandra (Warhammer)
 \item Dricka läsk och lyssna på Minor Threat
 \item Åka skateboard och ställa till med jävelskap \textsc{(s.~\pageref{46845591177f16920cd586a5baf5a625})}
 \item I sällskap med sina ungar äta sig spymätt på onyttigheter och kolla på Jokkis
 \item Lyssna på Sabbath och byta kammrem på en skrutt-Ford
 \end{itemize}
 

 HEAD2: Se även
 \quotetext{krypa upp i soffan som en brugd} \textsc{(s.~\pageref{ea7efaf2898e75a3e2cd2bb76fc18568})}

}

\small{
\textbf{Fredrik Boltes}
\label{ebd94af396b974ad4860d8863c52e9af}
 I nuläget finns inget viktigt att veta om denna person.

}

\small{
\textbf{Fredrik Reinfeldt}
\label{0c16c01849fc86b54e9e0e815490f747}
 är ett moderat[[skinhead]] \textsc{(se moderat s.~\pageref{c4564b188cb670841733a3ff923c2fb0})} som för närvarande är sveriges statsminister. Han tillåter inte ens sju procent av den del av Sveriges \textsc{(se Sverige s.~\pageref{b1999637949ed135b2ca03f3a38460cc})} befolkning som gär rätt för sig genom att arbetat att samtidigt dricka bira, vilket till och med fascisterna i Danmark \textsc{(s.~\pageref{5331d7fd27772396f412a5b6d19bad44})} gör. Reinfelt är också ansiktet utåt för de onda i J.R.R Tolkiens \textsc{(se J.R.R Tolkien s.~\pageref{3f0b7fcbd9fa7369ca314a46c280b67e})} Sagan om Ringen och mentor åt Crang i Teenage Mutant Ninja Turtles \textsc{(s.~\pageref{fd9ccf7b23fd53b8c3bb91065ab585ee})}. Enligt rykten ska Fredriks Reinfeldts farfar ha tillhört zulufolket och och ha praktiserat kannibalism [http://www.aftonbladet.se/nyheter/article441739.ab]. Sonsonen har dock övergett denna tradition för gamal hederlig utsugning.

 Se även: Randall Finefield \textsc{(s.~\pageref{42cdb7e2ced62d34c4f447c47eae332a})}

}

\small{
\textbf{Fredrik reinfeldt}
\label{0c16c01849fc86b54e9e0e815490f747}


}

\small{
\textbf{Freikörperkultur}
\label{40cdc17a157b501b2c84835ce6204f9c}
 är tyska och är en sammansättning av orden för \textit{frihet}, \textit{kropp} och \textit{kultur}. Som du säkert redan befarat rör det sig alltså om tysk  nudism \textsc{(se nudist  s.~\pageref{6283bcc4698e84866e85e43a77418abe})}, vilket innebär att man vid alla tillfällen som givs sliter av sig sitt bomullsfängelse och går omkring naken, att man uppmanar andra att ta efter detta gåtfulla beteende och att man tvingas sitta i koncentrationsläger när ens landsmän kommit på att de ska rösta in en galen massmördare från Österrike som rikskansler. Freikörperkultur är omåttligt populärt i Tyskland, speciellt i forna DDR, och är, paradoxalt nog, en mångmiljonindustri. Riktigt var i konceptet att gå runt naken alla dessa pengar ligger krävs det antaligen en \quotetext{frei} tysk att reda ut, men å andra sidan kan nog de flesta tänka sig att fortsätta förbli okunniga på detta område. Freikörperkultur är ett klockrent exempel på den tyska mustigheten \textsc{(s.~\pageref{682ccd5fdc3aff0c97e8845c3d6b6ca8})} och är besläktat med Kalle anka-konceptet \textsc{(se Kalle anka s.~\pageref{64db68f686a0ca4d9d641061cb3fdf13})}.

}

\small{
\textbf{Fri rörlighet}
\label{4360ec2152e06fe380908bc2b75f2cb3}
 Ett modernare ord för slavhandel. Används ofta av  tokliberaler \textsc{(se tokliberal s.~\pageref{531cb70b602e3f3c32d40bac64400830})} för att förgylla Litauiska svartarbetare.

}

\small{
\textbf{Fridhem}
\label{9d89d7b8df2ed94b0bc4ad84717e9517}
 är ett bostadsområde i Gävle känt för sina välklippta gräsmattor, välskötta rabatter och som \quotetext{stadsdelen där alla har ett leende på läpparna}. Området byggdes som en exakt kopia på Walt Disneys bostadsområde Celbration, Florida och har marknadsförts  som ett boende där trygghet och familjen står i fokus. Trots den trygghet och den livskvalitet som invånarna där känner så är området inte fritt från kritik. Dels har kritik framförts mot själva idén av att en välbesitten medelklass isolerar sig från omvärlden och därmed också stänger ute allt som inte passar enligt deras snäva definition av normalitet men på senare år har också en kritik framfört över vad som blivit av de nu vuxna personerna som fått växa upp i denna artificiella såpbubbla. Det har visat sig att flera av dessa har haft svårt att anpassa sig till världen utanför, som för dessa personer ter sig skrämmande och hotfull. Det är inte ovanligt att personer uppvuxna här ofta försöker hävda sig inom matchokulturer som t.ex att hävda att de är bra på kampsporter eller helt enkelt bra på att slåss, men där sanningen uteslutande ligger i att de endast \quotetext{slåss} i skyddade miljöer så som anonyma skribenter på internet eller i rollspelet World of Warcraft.

}

\small{
\textbf{Friedrich Hegel}
\label{eadd964c0c4d2479bfe20e49c8921e77}
 (1770 - 1831) var en tysk filosof som funderade fram dialektiken \textsc{(se dialektik s.~\pageref{5c0ded4e9796ad82ecd11d1a0010bf6b})}. Det går inte nog att understryka hur tysk denne man var då han led något kopiöst av den tyska mustigheten \textsc{(s.~\pageref{682ccd5fdc3aff0c97e8845c3d6b6ca8})} och det storhetsvansinne \textsc{(s.~\pageref{2f9c0ea6231e1de87c97eab41410c795})} denna medför. Det fick honom att hålla en föreläsning  om filosofins historia där han började med antiken \textsc{(se de gamla grekerna s.~\pageref{4a5fb3d6ce79b5ff43b33f8f7d843672})} och tesen Platon mot anti-tesen Aristoteles och jobbade sig dialektiskt genom filosofins historia fram till den slutgiltiga syntesen, nämligen Hegel själv. Hans böcker kallas, skämtsamt, Hegelstenar.

}

\small{
\textbf{Frihet}
\label{0ee37cea60c9b45e40dbc83c0c665085}
 är bara ett annat ord för att inte ha nåt kvar att förlora.

 Frågar man en tokliberal \textsc{(s.~\pageref{531cb70b602e3f3c32d40bac64400830})} vad frihet är så handlar det om något som har med pengar att göra, eller typ att få behandla folk som arbetar som boskap. Nåt i den stilen.

}

\small{
\textbf{Frikyrkligt lycklig}
\label{a5fd3ffe85fd75f8164baf2ca90152b9}
 Vi vanliga syndare kan endast drömma om hur det är att vara frikyrkligt lycklig, för om vi inte först börjar knarka \textsc{(se knark s.~\pageref{bebc5e7342ca2f076b3d32ed6c557398})} och leva satan och sedan blir pånyttfödda i herren kommer vi aldrig att få möjlighet att vara det. Att vara frikyrkligt lycklig är nämligen endast förunnat medlemmar av frikyrkorörelsen, som har utomhusgudtjänst, spelar gitarr \textsc{(s.~\pageref{a08bf8420208934bc59c7ed7385d4308})} och sjunger lovsånger med ett slags extatiskt leende över anletet. Detta leende följer med baptisten eller pingstvännen in i konsumbutiken \textsc{(se konsumbutik s.~\pageref{70e4875f7c2c177596305006e46b7ca9})}, biblioteket, skolan, kontoret eller vad det vara må. Den frikyrklige gläds åt skapelsens prakt och all den kärlek som utgå från den förlåtande fadern. Vi andra, människorna i allmänhet, ser inte detta utan oroar oss över arbetslivets små motgångar, sexuella relationer och politik. Vi går i strejk, grälar på partnern, bygger ut huset och går klädda i obekväma kläder, medan den lyckliga - ja näst intill euforiska - frikyrkomedlemmen ser i den eleganta böjningen av ett grässtrå herrens skapelsekraft och ser att detta är gott.

}

\small{
\textbf{Frilans}
\label{135b18ae63a1b2579331663f22dc3082}
 Ett finare ord för ganska arbetslös, särskilt inom kreativa yrken.

}

\small{
\textbf{Friluftskläder}
\label{9739f8f4c55a7e8abb041c736f99aee7}
 Så här är det: Människokroppen är genom evolutionen skapad att fungera perfekt vid 27 grader celsius, vindstilla och solsken. Då kan människan ägna sig åt Freikörperkultur \textsc{(s.~\pageref{40cdc17a157b501b2c84835ce6204f9c})} och slippa sina klädesfängelser. När vädret avviker från dessa optimala förhållanden bör kläder nyttjas. I vardagslivet är detta inga problem och inte något som tar upp särskilt stor tankeverksamhet. För människor vars främsta fritidsintresse är friluftsliv, dvs att vara utomhus och göra olika aktiviteter, har detta utvecklats till en hel vetenskap, ivrigt påhejade av den moderna kapitalismen \textsc{(se mänsklighetens historia s.~\pageref{5d87ba4132f8bdfa8c6294514c570c3f})}. Här följer därför en guide till denna snåriga djungel.

 
 HEAD2: Mikro- och makroklimat
 Makroklimatet, alltså rådande väderlek i det område du befinner dig är hart när omöjligt (om man inte är en superskurk). Vad du däremot kan göra något åt är mikroklimatet, alltså väderleken närmast kroppen eller hur man ska säga. En viktig tumregel är att hålla sig så torr som möjligt och detta är mycket mer aktuellt på vintern än på sommaren. På sommaren torkar svett och annat vatten om det är varmt och dess kylande effekt kan rentutsagt vara ganska ljuvlig \textsc{(se ljuv s.~\pageref{632bf5372f37d760ebb25b34ab411f71})}. På vintern har dock vatten den lilla egenheten att frysa till is och kyla ner kroppen på tok för mycket, i värsta fall tills den kallnar för gott \textsc{(se döden s.~\pageref{6f3c270eb5b4d979c777b4ec26dd106f})}. Här kommer då det första dilemmat: Hur ska man stänga ute regn utan att stänga inne svett? Lugn, svaret kommer snart.

 HEAD2: Lager på lager-principen
 Varje lager har olika funktioner och vad som gör principen så fiffig är att om en funktion inte behövs så kan det åtgärdas genom att det plagget stoppas i ryggsäcken. Så genialt, men ändå så enkelt. Hur personer som inte klär sig efter lager på lager-principen gör vettefan, och det finns inget intresse att ta reda på det.
 HEAD4: Lager 1
 Lagret närmast kroppen. Det får under som inga som helst omständigheter vara av bomull. Bomull är världens kanske sämsta material. Det andas (en term här betyder transporterar fukt/vattenånga) skitdåligt, kyler när den är blöt och är inte slitstark för fem öre. Skit i bomullen. Vad som däremot är grejen är olika funktionsmaterial, t.ex. syntetmaterialen polyamid och polyester. Andra är ull och den lite finare varianten merinoull. Syntetmaterialen har fördelen att de är snabbtorkande och transporterar fukt bra, däremot har de den dåliga egenskapen att lätt börjar lukta dåligt. Ulltyperna torkar inte lika fort, men värmer och andra sidan trots att de är blöta, på nåt lite halvmagiskt sätt. Välj därför ett underställ i något av dessa material om det är lite kallare. Är det varmare ute rekommenderas en t-shirt eller skjorta i något av dessa material eller bambu. Är det riktigt kallt räcker inte understället för att hålla värmen utan bör kompletteras med:
 HEAD4: Lager 2
 Lagret utanpå understället vars funktion är isolering, men det måste fortfarande andas. ANDAS! Viktigt. Rekommenderade material är fleece,  tjocklek beroende på temperatur. Det finns en uppsjö olika fleecemodeller och fabrikat. Den största och antagligen bästa fleecetillverkaren är Polartec det är också deras fleecetyper som den här artikelns författare är mest bevandrad i. Microfleece är den billigaste och mest basic fleecen. Den är ganska tunn men värmer bra utan att vara bylsig eller väga så mycket. En dyrare men bekvämare och ännu mindre bylsig variant är powerstretchfleece. Den är hur bekväm som helst men andas sämre än microfleece, inget för krävande aktiviteter då man svettas mycket. Sen finns också klassikern fiberpäls, samma material som i Hälleforskavajen \textsc{(se Hällefors s.~\pageref{e144fd5ba5ee4d7c395f18c9b1a4cd1f})}. Den är riktigt varm och andas ganska bra.
 HEAD4: Lager 3
 Det slutliga lagret, skyddet mellan dig och vädrets makter. Detta lager bör minst vara vind- och vatten\textit{avvisande} men helst vind- och vatten\textit{tätt}. En kortare tur i skogen kräver kanske inte täthet men en tur på kalfjället kräver i 99 fall av 100 täta kläder. Är det vinter så kommer det garanterat inte att regna, men det brukar gå att bli blöt ändå. Bra avvisande material är alla t.ex. Fjällrävens klassiska G-1000, en bomulls och polyesterblandning som har impregnerats för bästa funktion. Andra varianter med samma sammansättning (65 polyester, 35 bomull) går lika bra. Bra täta material är friluftsmaterialens cadillac, Gore-Tex \textsc{(s.~\pageref{cc1e3b66bcda33218427995652e4e31a})}. Detta fantastiska membranmaterial har den nästintill gudomliga egenskapen att det är 100\% vind- och vattentätt, men, MEN, andas. Det finns olika varianter av Gore-tex och den bästa är Performance Shell. Slitstarkt, tätt och andas bra. Softshell är ett bättre alternativ på vintern då det andas ännu bättre än Performance Shell, vattenpelaren (ett mått som anger hur blött ett plagg kan bli innan det blir blött igenom) är visserligen lägre, men som sagt, det regnar sällan på vintern. Softshell är så slitstarkt att det praktiskt taget är oförstörbart. Otroligt, men sant.

}

\small{
\textbf{Frisedel}
\label{ec0e47809187866739cd1a19e3d1ed37}
 Den 1 Juli 2010 fick hela svenska folket frisedel. Det sorgligaste med detta är att utbudet av vansinniga \quotetext{göra-allt-för-att-slippa-lumpen}-historier \textsc{(s.~\pageref{86f09c4b94bc85d3c0ffc70ae67b8361})} kommer att minska exponentiellt över tid. Det näst sorligaste är att vi nu har en yrkesarmé och att dödande av fattiga bönder \textsc{(s.~\pageref{30a6fc00c9102680b8196b1b79935ec4})} i andra delar av världen har blivit ett legitimt yrke.

}

\small{
\textbf{Frispel}
\label{dba619b015f18238c48bb8f495b50915}
 Arkaiskt uttryck för att gå bananas \textsc{(s.~\pageref{ec121ff80513ae58ed478d5c5787075b})}.

}

\small{
\textbf{Front row banger}
\label{bd4740b18eb6aca9232b48ea476f7547}
 En front row banger är en person som står längst fram på en hårdrockskonsert och headbangar \textsc{(se headbanga s.~\pageref{2846d09194c713ef6abd3c0c7eadbf5a})}. Uttrycket populariserades i och med [https://www.youtube.com/watch?v=crxsXWYWvN8 bröderna nifelheims medverkan i SVT.] Att stå längst fram vid kravallstaketet och headbanga visar att man är extremt hängiven till artisten man tittar på, på samma sätt som att göra ledingreppet \textsc{(s.~\pageref{528126abac2b649bb4fdf7bd2764726f})} på en Tomas Ledin-konsert eller att dansa våldsam pogodans när Sex Pistols kommer till stan.

}

\small{
\textbf{Frukt}
\label{7b0faed51fc6c55d2431ed677d0989ad}
 är ett slags godis, har forskarna nu konstaterat efter noggranna studier. Frukt skiljer sig dock från andra sorters godis genom att växa på träd och buskage. Vanliga former av denna konfektyr är:
 \begin{itemize}
 \item Banan \textsc{(s.~\pageref{aec7bd708ed2ad3435b9a9883ac7f45c})}
 \item Äpple
 \item Apelsin
 \item Päron \textsc{(se Päronhalva s.~\pageref{cc9c1bfa2ec4eaed89ca86a1b63e3a45})}
 \item Sviskon
 \end{itemize}

}

\small{
\textbf{Frukt är gott}
\label{4ab1c2bc560c69ffd1360e14fd9554b3}
 är den mest spelade sången i Sverige genom alla tider. Alla som vill lära sig ett instrument inom den kommunala musikskolan får börja med den låten. Bakgrunden är antagligen att låten anses enkel att lära sig med sina tre toner. Faktum är dock att \textit{Teenage Superstars} med Vaselines är ännu enklare med sina två toner så egentligen borde man lära sig den.

 Texten till Frukt \textsc{(s.~\pageref{7b0faed51fc6c55d2431ed677d0989ad})} är gott:
 \textlessi\textgreaterFrukt är gott
 Frukt är gott
 Äpplen och  bananer \textsc{(se bananas  s.~\pageref{ec121ff80513ae58ed478d5c5787075b})} smakar jättegott\textless/i\textgreater

}

\small{
\textbf{Fruktsallad}
\label{d7b2bae7bf161cb28b81db03874d3ecb}
 innehåller olika sorters frukt, men ingen sallad.

}

\small{
\textbf{Fryntlig}
\label{7735b91d0ba3b357bade2d695a0bb12a}
 Välvillig, tjock och röd - i ett ord!

}

\small{
\textbf{Ftw}
\label{830266bc8848e9e0b62835f6516d9327}
 betydde på det ascoola/brutala 90-talet Fuck The World och var då ett populärt uttryck bland kåkfarare, bikers och medlemmar i Driller Killer. Betydelsen ändrades under det urmesiga 00-talet till For The Win och är nu ett väldigt populärt uttryck bland internetungdomar och andra fjantiga grupper.

 HEAD2: OBS!!
 Slarva inte så det blir FTV. Det betyder Folktandvården och har aldrig, inte ens på 90-talet, varit populärt.

}

\small{
\textbf{Fuck Bowie}
\label{d2aa331a51db29a1affef46a2a8b362e}
 är en låt med Mob 47 \textsc{(s.~\pageref{9955900bded21660e7f4e15ae8d23e3a})}. Den är skriven av Mob 47-Åke \textsc{(s.~\pageref{486ee67ac39debabed3d92a7555dcebd})} och handlar om att han hatar David Bowie för att denne gör dålig musik och bara uppträder för att tjäna pengar. Låter återfinns exempelvis på deras samlingsskiva \textit{Back to attack}.

 HEAD2: TEXT
 Vers:
 David Bowie är en bög
 David Bowie är en stor skit hög
 Spelar jävla skitmusik
 Spelar bara för profit

 Refräng:
 Jag hatar David Bowie
 som lyssnar på \textit{Fuck Bowie} genom en risig bandare.]]

}

\small{
\textbf{Fucked in finland}
\label{a75a1bc0afe687bdde1f14a5a591a5cc}
 är en liveskiva med det svenska råpunkbandet Anti Cimex. Skivan gavs ursprungligen ut som vinylsingel av finska Arda records men har sedan dess kommit i flera nyutgåvor. Den spelades på Lepakko i Helsingfors 1992 och gavs ut samma år. Ljudet på skivan låter förjävligt och det är något oklart varför folk lägger pengar på den.

 Ett intressant stycke fakta \textsc{(s.~\pageref{fce663ae73dc87a727148bc3b94d1ffa})} är att sångaren, Jonsson, genomförde hela turnén i nerpissade byxor. På rikt.


 HEAD2:  Låtlista

 \textbf{Vinylversion:}
 Braincell battle (live)
 Dogfight (live)
 Make my day (live)

 \textbf{CD-version:}
 Braincell battle (live)
 Dogfight (live)
 Make my day (live)
 Under the sun (live)
 Daughters of pride (live)

}

\small{
\textbf{Fudge}
\label{7921903964212cc383bf910a8bf2d7f4}
 är en populär konfekt som folk trugar på en inte minst kring jultid. Konfekten är oftast guldbrun, kubformad och har en konsistens som kan uppfattas som lite suspekt, vilket man dock har överseende med eftersom den har en ganska okej smak. Fudge uppfanns av psych-coverbandet Vanilla Fudge för så länge sedan att det är svårt att föreställa sig.

}

\small{
\textbf{Fula fiskar}
\label{f2dd2a061e8121a392589d09cf5db207}
 Motsatsen till fina fisken.

 Se filurer \textsc{(se filur s.~\pageref{e308f4e2553faf188385f17ebda05242})}.

}

\small{
\textbf{Fulhybris}
\label{486a0bebebedab6491c566ef10e30a8d}
 Hybris hos en ful person. Det må så vara att skönhet ligger i betraktarens öga. Men det finns människor vilka de flesta betraktares ögon inte gärna tittar på. Fula personer.  Hos dessa fula personer finns en utbredd uppfattning om att de, genom deras utseende, dragit en kosmisk nitlott. I samma sociala kategori (fula), är tron på ett karmaliknande system, som kompenserar den fule personen med gudaliknande egenskaper lika utbredd. Oftast handlar det om att den fule personen ser sig själv som extremt intelligent. Detta i sin tur leder ofta till ett extremt jobbigt beteende (se storbossnörd) \textsc{(se storbossnörd s.~\pageref{456018ad01124baca4c32b6567fca7b8})}, vilket vidare bidrar till den fules sociala isolering, och därmed tillförs mer bränsle till den fules hybris. För hur smart måste inte Gud ha gjort en om man både är ful och ensam? Enligt Nissepedias källor har ingen kausalitet kunnat påvisas, där ful=smart.

}

\small{
\textbf{Fulla Fåglar}
\label{c74c968264428d9117c61fb6c570568f}
 Det finns en massa fulla fåglar. Somliga är individuellt fulla lite till och från, som till exempel skatan Elsa. Andra är mer genetiskt predisponerade att berusa sig, så kallade artbaserat fulla fåglar (AFF). Skåltrasten, Bofinkeln och Trattmåsen är några av de i Sverige vanligast förekommande AFF-arterna. Samtliga ses ofta både i skogen och på krogen, då de inte är det minsta blyga, och vanligen i sällskap med varandra. Allra fullast är dock Arne Anka, som sannolikt är en mutation.

}

\small{
\textbf{Fullkittad}
\label{e66a645b2ba13a48aa09d73519fcdaa1}
 Det räcker inte med att äga en sak, den måste utvecklas.

}

\small{
\textbf{Fulsnygg}
\label{361540c3374a05e68c1037107b4d1b33}
 är en person som både är ful och snygg på samma gång, något som kan skapa förvirring hos den som betraktar fulsnyggingen. Skolboksexemplet på fulsnygghet är den franske skådespelaren Gerard Depardieu, vars utseende, hans gnistrande leende och milda ögon till trots, alltid kommer att framkalla en känsla av ambivalens, då han i stället för en näsa har vad som förefaller vara en mandelpotatis storlek XL monterad mitt i fejset.

}

\small{
\textbf{Fultomten}
\label{c12b715eea245de7573c1972c35d4e80}
 är farbror, eller morbror, eller morfar, eller farfar, eller grannen, eller kanske rentav inhyrd okänd mansperson, eller i allra värsta fall pappa, som 24:e december tar en snaps, sätter på sig de röda byxorna, först bakfram, men sen rätt väg, tar sig en snaps, sätter på sig den röda jackan, dricker lite öl, kränger på sig kängorna, stoppar in en snus och tar sig en snaps, funderar på om det verkligen behövs ett lösskägg när man inte rakat sig på två veckor, rapar, tar sig en snaps och lite öl, kränger på sig lösskägget i alla fall, svär över katten som skitit på hallmattan, tar sig bara en liiiiten snaps, för man får ju inte vara full inför barnen, tar på sig den röda luvan med vit tofs, och går sen för att, högst oavsiktligt, göra julen evigt traumatisk för de stackars förväntansfulla tvååringarna.

}

\small{
\textbf{Furu}
\label{7b4f0a8661e2be694243977d38f88fac}
 Att ha åtminstone en del av sitt hem (helst en källare eller gillestuga) klädd i furu är för inredning vad det är för namn att heta Andersson eller för bilister att köra Volvo - Det svenskaste som finns. Att komma ner i en källare med väggar som lyser i det ljusa trämaterialet signalerar trygghet på samma sätt som att hitta Black Sabbath-vinyl på en skivloppis. Dalahästar och broderade dukar med kristna budskap på kan ta sig. Snackar vi svensk inredning snackar vi furu.

}

\small{
\textbf{Fyllevolontära}
\label{738cc82d889811ec70767e0ae7dbf047}
 När man är lite på lyset är det lätt att man kommer på idéer som verkar bra för stunden men som i backspegeln \textsc{(se inre backspegel s.~\pageref{4d9c85c411e32a3a87ec9b69b7b75b70})} ofta kan te sig mer tveksamma. Ett klassiskt exempel på detta är att fyllevolontära.


 HEAD3: Pedagogiskt exempelscenario
 Vi befinner oss på Augustibuller \textsc{(s.~\pageref{6072997c8cf781008ad5748003565c8a})} där en gråtfärdig 14-åring står vid campingentrén och inte blivit avlöst på två dagar för att träskpunkaren \textsc{(se träskpunkare s.~\pageref{484838b3db1adb135ea74d6fc61e44c0})} som skulle gjort det har drunknat i en barnpool fylld av urin \textsc{(s.~\pageref{524fd7acb94f9c2d879b5c1cf8335669})}. Förbudet mot glasflaskor efterlevs mycket sporadiskt men barnets \textsc{(se barn s.~\pageref{5dfcc0aab2f3db925b2d51ba73e48946})} möjligheter att göra något åt saken är ganska obefintliga. Men här kommer du, raj-rajande efter en lyckad spelning med (till exempel) Varukers, och får syn på den tårögda tonåringen. Dina solidaritetskänslor är starka efter att under en halvtimme ha sparkats på smalbenen av crustknytna \textsc{(se crustknytning s.~\pageref{f9f5d7389b273a02e992d998eaafcece})} kängor så du erbjuder dig genast att sköta barnets jobb medan denne letar på en vuxen. Du slår dig ned i en brassestol med knäckt rygg (alltså stolens) och vinkar glatt medan kavalkaden av slan \textsc{(s.~\pageref{caaad522de864ab45ed679c4a16edd8d})} som saknar festivalarmband tågar in på campingen. Barnet har tagit första bussen hem och din mage börjar kurra efter ett tag så du drar och klämmer en burgare.

 \textit{Fyllevolontära}

}

\small{
\textbf{Fyn}
\label{9854682d71fdb60a819d9188a846f42d}
 är en ö tillhörande konungariket Danmark \textsc{(s.~\pageref{5331d7fd27772396f412a5b6d19bad44})}. Om man betraktar Danmark som en stor röv \textsc{(s.~\pageref{08e887710123ae191d6b777e3e65170c})} ligger Fyn precis där ändtarmen normalt har sin mynning. Detta återspeglar sig också i öns ekonomi som till stor del bygger på att medelst håv och dansk kanot \textsc{(s.~\pageref{884458cb1a5fba0e5e027e7d1377bbc8})} fånga in de fekalier som flyter runt i havet efter att Roskildefestivalens bajamajor tömts. De näringsrika slaggprodukterna torkas sedan och blir till gödningsmedel på haschplantagerna \textsc{(se hasch s.~\pageref{1e93612a55f48e5fd9cbce22d0e71944})} som genererar öns huvudsakliga exportprodukt: vit reggae. Eftersom den globala efterfrågan på denna vara är ganska begränsad går Fyn regelbundet i statskonkurs och måste räddas av den välbärgade haschtomten \textsc{(s.~\pageref{3a3c1522c7155a18293fb1388055c13e})} och vinylfantasten Lars Krogh \textsc{(s.~\pageref{7a5a84f5d8e84be6ff55aa9709c3dacd})}.

}

\small{
\textbf{Fyra}
\label{7bdb5385ce8e0b1cbc7c15b1d71e8e7d}
 En fyra är en liten spritdrink och ett armgrepp i brottning. Av någon anledning känns denna kombination rättvisande och välmotiverad.


 Se även: Etta \textsc{(s.~\pageref{ba48f6c4097b7fc25ca11f1e544842d7})}, Tvåa \textsc{(s.~\pageref{84fcc0494ecf9f5af79fcd9bed184a9a})}, Trea \textsc{(s.~\pageref{6f94fdf535ab2e21147ea40ea920ca75})}, Femma \textsc{(s.~\pageref{d974e0811fe7a4d49a9062d33b66a88d})}, Sexa \textsc{(s.~\pageref{4b1fabe53857b0a2ace6ae22008fe13e})}, Sjua \textsc{(s.~\pageref{e7bf63fa6d0d29bd89c23f833b979a15})}, Åtta \textsc{(s.~\pageref{6fa68b0d02ec525fa72a51c13e5e3ed1})}, Nia \textsc{(s.~\pageref{04a481486dd84d7c8bfdfc89d38136a6})}.

}

\small{
\textbf{Fyrhjulsdrift}
\label{be42a23777ac1bfbd06d09a21ac33b8b}
 är den enskilt största anledningen till dödsolyckor i trafiken. Efter ihärdig propaganda från bilindustrin har vanliga konsumenter sakta invaggats i en falsk trygghet om att fyrhjulsdrivna bilar är säkrare att köra. Förblindad av dogmer om ABS, antisladd, styrservo, etcetera övervärderar föraren sin förmåga och slutar allt som oftast i en bergvägg eller röven på en älg.

 Den trafikant som värdesätter riktig säkerhet väljer istället bakhjulsdrivet. Med detta alternativ får du en tydlig indikation vid minsta halka och kan dessutom häva en sladd själv utan att behöva ställa dina förhoppningar till en millimeterstor kiseltransistor \textsc{(s.~\pageref{aaa0d78af49a2fbc2f7ad8fbb11de1aa})} som någon storbossnörd \textsc{(s.~\pageref{456018ad01124baca4c32b6567fca7b8})} med flottiga fingrar lödit ihop.

}

\small{
\textbf{Fyrtiotusen miljarder}
\label{c2160bffc9c5ca88e77204672e62e489}
 är ett kvantifierande kraftuttryck som myntades av Sverigedemokraten Margareta Sandstedt som skämtade till det lite i riksdagskammaren genom att räkna om 400.000 Euro till just fyrtiotusen miljarder SEK. Exempel på språkliga kontexter där uttrycket kan användas:
 - Jag hade tänkt ta med mig en flaska vin \textsc{(s.~\pageref{62911ad86d6181442022683afb480067})}, men det var fyrtiotusen miljarder folk före mig i kön så jag struntade i det.
 - Jag hade gärna följt med på utflykt, men jag har fyrtiotusen miljarder saker att hinna med idag.
 -Max, Karl och Emile Andersson!! Jag har sagt åt er fyrtiotusen miljarder gånger att den som plockat fram legot \textsc{(se lego s.~\pageref{3a22c9ea9a3039d180e0a514a5a3b619})} tar undan legot när den är klar!! (springer skrik-gråtandes mot vedboden)

}

\small{
\textbf{Fängelse}
\label{c8c141473b3ae76ede09c99b23d8501f}
 Ett fängelse är ett slags hus av betong där strulputtar \textsc{(se strulputte s.~\pageref{21651c95306d1b1e281443f8620910da})} bor.
 HEAD2: Thin Lizzys fängelse - sämst
 Thin Lizzys fängelse är bland de sämsta fängelsen som finns: Fångarna bryter sig ut lite hur som helst, inget vet riktigt var i stan fängelset ligger och det blir alltid trubbel.

}

\small{
\textbf{Färgskala}
\label{b3805ad68db6ccd76654a1106f5c2c52}
 Det finns jättemånga färger men vissa är viktigare än andra. Utan nedanstående färger hade Sverige fortfarande varit ungefär som på medeltiden \textsc{(s.~\pageref{88cbc30c5b233d97df68b5b041ac0655})}.

 HEAD2:  ASEA-grönt \textsc{(s.~\pageref{e5ce0e93ee9c54094b4ec1c2027272ca})}

 ASEA \textsc{(s.~\pageref{61585ca988d62029b6ee6adab6066c34})}











 HEAD2:  Telverksorange \textsc{(s.~\pageref{ca5043d4e81ab5163ab467d10a37cc95})}

 Televerket \textsc{(s.~\pageref{4cf398db49e1da53e4a4d3f34dce77e3})}







 ==Korvb-röd \textsc{(se \quotetext{korvbröd} s.~\pageref{6898888a74f0d42574012debf1a6d8f3})}

}

\small{
\textbf{Färskost}
\label{845fda8e8c2ab93f6e9948166ef471b1}
 är en diet som går ut på att enbart äta saker gjorda på färs. Vanligtvis består färsen av kött från djur eller fiskar, men det enda riktiga kriteriet för att en färs ska få kallas färs verkar vara att den är hackad eller mald så för allergiker \textsc{(se allergi s.~\pageref{23773a17729d8e7e24da798e97533aeb})} torde det gå utmärkt att köra på till exempel gröt- eller gurkfärs. Dieten har utvecklats av Nissepedias \textsc{(se Nissepedia s.~\pageref{62400dadecd90cb5cd39062abe5a3e4a})} medarbetare Ronny \textsc{(se Användare: Ronny s.~\pageref{c7fc87f27db026e1c60a6ac2cb1fd820})} och är en hårdare variant av hans gamla koncept \quotetext{bärs och burgare}. I valet mellan att hacka och mala rekommenderar han hackat, eftersom man får använda kniv då. Vad dieten ska vara bra till är oklart.

}

\small{
\textbf{Färst}
\label{5530a65addfadb2019a87b18347ab20b}
 Få, färre, färst.

 Användning: \quotetext{Av alla skandinaviska länder har Danmark färst invånare}

 Det är alltså fel att säga \quotetext{minst} i exemplet, för då menar att Danskarna är mindre i storlek än resten av skandinavierna. Sen antas det att danskar är mer småsinta, men det är en annan diskussion.

}

\small{
\textbf{Fågelmyra}
\label{044cc1889af3f8f726910a98218d6e2d}
 En fågelmyra är en myra som har befjädrade vingar. Förutom det är den som en helt vanligt myra.

}

\small{
\textbf{Fåglar som går}
\label{bdf29eb702510810d39c906ce7f47ad5}
 är något av det mest fåniga och skrattretande som finns. Ofta kan man se dem på stan - duvor och kajor som går utmed marken och letar mat. De har så små ben och otjänliga fötter. Vissa fåglar kan inte ens gå, som vi förstår ordet, utan liksom hoppar jämfota framåt eller \quotetext{knixar \textsc{(se knixa s.~\pageref{2d350f4a76f78d9017ed0f136cf81b88})}} där de står.

 Vissa fåglar, så som familjen vadare, har fått sitt namn genom ovanan att gå istället för att flyga. Andra har gått så mycket att evolutionen har gett upp på dem och tagit bort deras flygförmåga. Sådana fåglar har fått den träffande benämningen \quotetext{gåfåglar}. De ojämförligt största gruppen av fåglar som går är dock helt vanliga fåglar som kan flyga men som av en eller annan anledning väljer att promenera istället.

}

\small{
\textbf{För}
\label{5a98c81c7b5b60a5777a92b943f53a41}
 en är frammen på en skuta. Fören är skutans ansikte, kan man säga. Folk med begynnande storhetsvansinne \textsc{(s.~\pageref{2f9c0ea6231e1de87c97eab41410c795})} brukar ange fören som sin favoritdel av båten.

}

\small{
\textbf{För full för knull}
\label{45b93665c3da2e4d85cf421e627b72be}
 Prof. Etienne \textsc{(s.~\pageref{56957a267e57df32753cf7f3b8a603d8})} översatte i början av 1990-talet en mängd låttexter från engelska till svenska, i sin bok \quotetext{Alla språk utom svenska är skitspråk}. En av de texterna var en översättning av punkbandet Dead Kennedys (eller som Etienne översatte deras namn till \quotetext{Skjut Olof Palmes lik i ansiktet}) låt \quotetext{Too drunk to fuck}. På svenska döptes låten till \quotetext{För full för knull}. Textraden \quotetext{You ball like the baby in Eraserhead} var en alltför smal referens för bokens tilltänkta publik, därför tog sig Etienne friheten att ändra till \quotetext{Du knullar som Härenstam i Tuppen}. Då åsyftas alltså Lasse Hallströms superfilm Tuppen från 1981, i vilken Magnus Härenstam spelar tidskonrollanten Cederquist som förlustar sig med fabriksarbetande kvinnor i en svensk småstad på 1940-talet. Även om Etienne hyllat filmen som sådan, har han i flera texter kritiserat Härenstams prestation, med särskild fokus på att han inte lever ut tillräckligt mycket i sexscenerna (mer specifikt har Etienne vid upprepade tillfällen liknat Härenstam vid en rodnande korgosse). Således blev att knulla som Härenstam i Tuppen, likvärdigt med mutantbarnet i Eraserhead.

 \textbf{För full för knull:}

 Gick på ett party
 Dans, åll rajt!
 Jag drack 16 stöl
 Å starta upp en fajt

 Nu är jag matt
 Så jävla bull
 Staplar som en rese
 För full för ett knull

 För full för knull
 För full för knull
 För full för knull
 För full, för full, för full
 För ett knull

 Du är så smart
 En bra idé
 Att skita iiiii
 Värdinnans bidé

 Å jag blev less
 Vill döda dig sen
 Du knullar som Härenstam
 I Tuppen

 För full för knull
 För full för knull
 För full för knull
 Usch och fy och tvi
 För full för ett knull

 För full för knull
 För full för knull
 För full för knull
 Jag är slapp och risig och kall
 För full för ett knull

 På väg att dö
 Å hälsa hem
 Men blir glad
 Av att aldrig se dig igen

 Gå ner på mig
 Men jag vill spy
 Ska aldrig knulla igen
 Åh fy!

 För full för ett knull
 För full för ett knull
 För full för knull
 Å de va själva fan - å jävlar!
 Tappa portmonnän – åh nej å jävlar
 Å nu sket jag ner mig
 För full för ett knull
 Yeah, yeah
 Yeah, yeah
 Yeah, yeah

}

\small{
\textbf{För nit och redlighet i rikets tjänst}
\label{ddece89acb2446187503ff7cf88c9e1b}
 är en belöningsmedalj som delas ut till den som oförvitligt tjänat staten i över 30 år. Den präglas i 23 karats guld, väger 11 gram och har en diameter av 27,5 mm.

 HEAD2: Valfrihet

 Den som inte önskar få belöningen i form av en medalj kan istället välja på följande alternativ:
 \begin{itemize}
 \item En vacker skål av glas från Kosta Boda.
 \item Ett förgyllt armbandsur.
 \item Motsvarande värde i Abu Garcia-kepsar \textsc{(se Abu Garcia s.~\pageref{ebb8e709f4430ad487471fd1acdf28e2})}.
 \end{itemize}

 HEAD2: OBS!
 För nit och redlighet i rikets tjänst är ingen leksak! Den ska därför inte behandlas som en sådan, utan placeras på en väl synlig plats där den inte kan komma till skada. Undantaget är om man väljer Abu Garcia-alternativet. Abu-garcia-kepsar är lite som leksaker och kan behandlas hur som helst.

}

\small{
\textbf{Fördelar med att bo i gryt}
\label{87f4e00149377054658d1b0851c4718b}
 Gryt är en boendeform som funnits mycket längre än populära nymodigheter som bostadsrätt och villakollektiv. Trots den långa etableringstiden är det ändå nästan ingen människa som bor i gryt idag. Man kan tycka vad man vill om detta och vi på Nissepedia \textsc{(s.~\pageref{62400dadecd90cb5cd39062abe5a3e4a})} tycker att det är himla synd. Att bo i ett gryt har nämligen, bland annat, följande fördelar:
 \begin{itemize}
 \item Det är gratis.
 \item Det är svalt på sommaren.
 \item Tak, väggar och golv är av jord så inget blir smutsigare än det var från början.
 \item Ibland kommer det maskar och mulvadar krypandes i jorden så om du inte är vegan \textsc{(s.~\pageref{792fec82e3a0dcea1817fd9ebfaf1533})} serverar middagen sig själv.
 \item Ibland händer det att jägare går på grytjakt så om du faktiskt är vegan kommer ett ypperligt tillfälle att skrämma slag på din antagonist.
 \item Du behöver aldrig putsa fönstren eller måla om fasaden.
 \end{itemize}
 Som läsaren märker finns det i princip bara fördelar med att bo i gryt. Ska man säga något negativt skulle det i så fall vara att det kan finnas grävlingar där, och grävlingar biter tills det knakar.

}

\small{
\textbf{Förebyggande skallgångskedja}
\label{cbe91e410ec13c56545fce404b27eefe}
 Förebyggande skallgångskedjor bildas för att se till att personer som ännu inte försvunnit förblir oförsvunna. Den förebyggande skallgångskedjan drar runt på stan i armkrok tills de får syn på en tant med misstänkt förvirrande hatt eller en gammal gubbe som inte verkar fatta någonting. Den förebyggande skallgångskedjan tar då rygg på personen och följer efter den tills man bedömer att den trots allt är på rätt spår. Skulle det dröja för lång tid innan några positiva tecken visar sig fångas personen in och transporteras tillbaka till torget där den först upptäcktes  för att få ett nytt försök. Den förebyggande skallgångskedjan har fullföljt sitt uppdrag och kan fira med att ta ännu en snaps.

}

\small{
\textbf{Förlovad}
\label{47914f0532a1aaa7658751d9e64138cf}
 En gång i tiden förlovade man sig och sedan gifte man sig inom ett år.
 \textit{\quotetext{Vi ska gifta oss men inte just nu}} så att säga.
 I takt med att ungdomen valde sus och dus istället för Lars Levi Laestadius \textsc{(s.~\pageref{c91fcd34b5328c4a87e4ae93efa97bfc})} skrifter blev en förlovning nån slags tonårsgimmick.

}

\small{
\textbf{Förskinka}
\label{bd76a4426859d7abf0e581e94572bb77}
 Om man griljerar en grisröv mellan första och fjärde advent benämns denna förskinka.

}

\small{
\textbf{Första sjuan}
\label{7b7c558fc3f8d8557ba30b082e644ea1}
 Inom de flesta obskyra musikstilar börjar de flesta banden med att släppa en sjua \textsc{(s.~\pageref{e7bf63fa6d0d29bd89c23f833b979a15})}.

 Den här är alltid bättre än nästkommande släpp oavsett hur det låter, i alla fall om du frågar en konnässör. Framförallt så är den jättedyr och skivsamlare får halvfralla/smygväta om de skulle bläddra fram en sådan i en skivback.

 Exempel på första sjuor som är bättre än resten av bakkatalogen:

 Tragedy - Can we call this life?

 Totalitär - Multinationella Mördare

 Mob 47 - Kärnvapenattack

 Minor Threat - Minor Threat

 Earth Crisis - All Out War

 Rolling Stones - Come on b/w I want to be loved

 Atomångest-Kasta gatsten


 HEAD2:  Se även
 Första skivanalibi \textsc{(s.~\pageref{0fd7abea0db14d19df73202811130364})}

}

\small{
\textbf{Första skivanalibi}
\label{0fd7abea0db14d19df73202811130364}
 är mycket vanligt bland skinheads \textsc{(se skinhead s.~\pageref{a54bc1b5d472b5afed8e84004b6441c4})}, framförallt vad det gäller Skrewdriver.
 \quotetext{\textit{Men dom var inte nazister på första skivan}} säger alltid den strykrädde \textsc{(se strykrädd s.~\pageref{75bdfdb38f443fea6318f325667d096c})} skinnskallen som konfronteras av en grupp glada \textsc{(se spritfylla s.~\pageref{0668c687b51995118ec27cbf25061118})} lulebor \textsc{(se Luleå s.~\pageref{3cefb5ac35187749592f1ebb25472b99})}.

 Förstan skvianalibit fungerar även för den anti-auktoritäre socialisten med smak för progg, och t.ex. \quotetext{\textit{Men dom var inte stalinister på första skivan}} när han eller hon måste försvara sin Knutna Nävar t-shirt.

 Första skivanalibi kan även gälla kassetten \quotetext{Gällivarevisor} som faktiskt spelades in 20 år innan Nisse Hennix blev nazist.

 HEAD2:  Se även
 Svart alibi \textsc{(s.~\pageref{168a6ccab282409b534cf3a9fcdc7029})}

 Kvinnligt alibi \textsc{(s.~\pageref{60da199ecfe5b75a702ff11156c333df})}

}

\small{
\textbf{Förstklassig cricket}
\label{db8e1fedec050c25da8da646a16b7efa}
 är den ädlaste klassen av nationell cricket och består av matcher som spelas under minst tre dagar. Lagen är två till antal och består av vardera elva spelare. Förstklassiga matcher kan bara spelas mellan lag tillhörande samma nation. Den ädlaste klassen av matcher mellan nationer kallas istället \quotetext{test cricket \textsc{(s.~\pageref{6f4f77d351da2cf1ba5e0b4f445fdd78})}} och kan bara spelas av nationer som tilldelats \quotetext{test status} av Internationella Cricketrådet (fram till 1966 kallat \quotetext{Imperiets Cricketkonferens}). Termen förstklassig cricket definierades första gången 1947 under ett möte med Imperiets Cricketkonferens och klassificeringen kan bara ges särskilt kvalificerad lag.

}

\small{
\textbf{Förstoppad uv}
\label{06630b162e869a376076dda808c05e5f}
 ), flankerad av sina oförstoppade vänner. Studera uvens plågade uppsyn noga om du vill lära dig känna igen en uv som lider av förstoppning.]] En förstoppad uv är en uv \textsc{(s.~\pageref{45210da832f9626829457a65e9e7c4d0})} som har blivit förstoppad av att ha ätit för mycket uvgodis \textsc{(s.~\pageref{58de09e078ac891b067c0ec53d780b8a})}. En förstoppad uv särskiljs av den skarpsynte från vanliga uvar på dess plågade ansiktsuttryck och i panik \textsc{(s.~\pageref{bca410441b88e24768f3f385548edfbe})} uppspärrade ögon. Påträffar man en förstoppad uv är det enda humana tilltaget att skyndsamt infånga den och erbjuda den katrinplommon, eller om så krävs tvinga ner sådana i uvens inre. Detta löser tids nog upp den uvgodis[[propp]] \textsc{(se uvgodis s.~\pageref{58de09e078ac891b067c0ec53d780b8a})} som orsakar uven olägenheter. Om detta inte är möjligt kan man ge uven lavemang, men detta är en både kostsam och komplicerad procedur och bör därför undvikas i möjligaste mån.
 Se också: uppstoppad uv \textsc{(s.~\pageref{a562653cfd13c16d7f4d85967242ccdd})}.

}

\small{
\textbf{Förståsigpåare}
\label{ff91afb86ce86124b6a517f3eb37bc18}
 Det enda yrket som står till buds för någon med över 300 hp. Yrket tillkom under regeringen Persson som en arbetsmarknadsåtgärd för att avhjälpa den allt mer kritiska situation som det ökande antalet personer med kandidatexamen ledde till. Förståsigpåare får inte förväxlas med anställda hos institut och tankesmedjor \textsc{(s.~\pageref{c276b5997d5af80504f79b30d121cf62})}

}

\small{
\textbf{Försåvitt}
\label{2a29ecb116746203e9618d363fca88bd}
 är ett väldigt pråligt ord som alla borde använda mycket oftare. Det är synonymt med \quotetext{om} och \quotetext{ifall}. Ett exempel på hur det kan användas: \textit{\quotetext{Termen globalisering avser väsentligen den tänjningsprocess försåvitt förbindelserna mellan olika sociala sammanhang eller regioner blir till ett nätverk över hela jordens yta}}(Giddens 1996: 66). Ett annat exempel som är lättare att tillämpa i vardagen: \textit{\quotetext{Kära surkärring jag är gift med, jag vägrar komma ut från mitt hemliga gömställe försåvitt det inte står tolv frasvåfflor, en hink blåbärssylt, sju deciliter vispgrädde (halvvispad - sådär som jag tycker om det) och tre Arboga 10.2 \% på middagsbordet inom 30 minuter}} (Etienne \textsc{(se Prof. Etienne s.~\pageref{56957a267e57df32753cf7f3b8a603d8})} 1992: 88).

}

\small{
\textbf{Förtråkningsprocesser}
\label{270a286749a1959c7e5223277e548186}
 är väldigt vanliga inom bland annat kreativa verksamheter. Grundprincipen är att något som från början var spännande, nyfiket och kanske rentav lite farligt, går och blir tråkigare och tråkigare efterhand. Sossarna till exempel, ville ju störta kapitalet en gång i tiden. Men se på dem nu. Sven-Ingvars var i början ett riktigt fräsigt och egensinnigt popband. Tecknade serier, designers, TV-serier med mera har alla en tendens att drabbas av förtråkningsprocesser. En anledning kan vara att det helt enkelt inte finns hur många bra idéer som helst inom ett specifikt område. Men den vanligaste anledningen är att, typ, rockstjärnan fastnar i en vana för hur hen brukar göra för att det ska funka, rentav gärna för att det ska vara ett säkert kort (säkra kort är nästan alltid centrala i förtråkningsprocesser) som alla är nöjda med. Och eftersom människor ovanligt ofta är lata försöker man förenkla de säkra korten till en ganska andefattig tumregel.

 Den nu centrala frågan är: Kommer Nissepedia att drabbas av en förtråkningsprocess? Eller är den redan här? Man har börjat sortera ut en del osäkra kort, underliga infall och humor som inte uppfyller internationell ISO-standard på ett räddhågset, nästan gnälligt vis. \quotetext{Duger verkligen det här? Jag hade kul när jag skrev det, men är det inte lite dumt så här i eftertankens kranka blekhet? Nä jag tar nog bort det.} VAD FAN HAR KRITISKT GRANSKANDE EFTERTANKE PÅ NISSEPEDIA ATT GÖRA? undrar bekymrad gästskribent. Snart är väl nissepedia lika roligt som ett partimöte med socialdemokraterna.

 VI VILL HA UNDERBET KVAR! ROPEN SKALLA, UNDERBET ÅT ALLA!

}

\small{
\textbf{Förvirring}
\label{c502a6223b16f730a8900c12f2b10fec}
 Befinner du dig i en situation där du inte riktigt förstår vad som händer? Har fenomen som vart och ett är fullkomligt rimliga kombinerats på ett sätt som inte går att förklara? Har du svårt att sätta ord på vad du ser? Då är det troligt att du drabbats av förvirring.

 vid en kyrkorgel. Solklart fall av förvirring.]]
 Förvirring är ett utbrett fenomen som drabbar nästan alla som inte förstår vad som händer. Exakt vad det är har debatterats länge och någon full konsensus råder tyvärr inte. Fysiker förklarar det vanligtvis med att DNA:t hos den förvirrade har för små atomer, medan hippies \textsc{(s.~\pageref{4dc77d6258fd18e7c0dd5eece5c7c81c})} tror benhårt att det beror på att det är människans litenhet i förhållande till universum. Postmodernister tror som vanligt att allt egentligen förhåller sig tvärt om.

 Förvirring existerade redan långt innan den första hominiden vandrade ut över Pangea och slog ned sina bopålar. Redan under dinosauriernas tid ledde förvirring till att snurrigare arter blev dödade av rationellare. Tyvärr finns det inget helgarderande försvar mot förvirring. Sprit kan fungera ett tag för att få allt att verka rimligt, men tyvärr brukar förvirringen ofta komma tillbaka med dubbel kraft efter ett tag. Att meka bilar fungerar också skapligt så länge kärran inte är för ny. Den som drabbas av förvirring bör för sin egen och andras skull hålla sig undan.

}

\small{
\textbf{Galapagosudden}
\label{fbbeb30d63ece2cf40091ab6943e8a0f}
 är en kille från Åtvidaberg. Vissa påstår att han aldrig funnits. Det är dikt och förbannad lögn. Galapagosudden is the shit.

}

\small{
\textbf{Galna ko-sjukan}
\label{3f8bb9e74da4b0ebd54dc1ea5d30dd08}
 Well, never mind. We are ugly but we have the mu-sick.

}

\small{
\textbf{Gamle Ole}
\label{1092ad25ed7624715bba1a2a30eb8307}
 är något så ovanligt som en dansk ost. Man tänker att det inte borde finnas ost i Danmark \textsc{(s.~\pageref{5331d7fd27772396f412a5b6d19bad44})} eftersom det enda pålägg danskar har på sina smörgåsar är godis och alla åkrar är belamrade med besökare till Roskildefestivalen. Men det finns faktiskt några hektar oförstörd betesmark på norra Jylland (vilket förmodligen beror på att det ligger en tandläkare och en klädaffär i närheten, och danskar hatar som bekant att borsta tänderna och att inte vara nakna) och där betar några trötta kor. Osten karraktäriseras framförallt av sin starka lukt men har också en distinkt smak. Ska man bjuda på ostbricka \textsc{(s.~\pageref{bf06c995c523e159eb93017810ee8f44})} rekommenderas att man lämnar Gamle Ole utanför eller ställer på ett separat fat i ett eget rum så att den inte tar över smaken på andra ostar. Mads Mikkelsen \textsc{(s.~\pageref{1658b9125bfc1f75901858f0e8344337})} har kallat Gamle Ole \quotetext{mye, mye dejlig}.


 Vem Ole är eller var förtäljer inte historien, men kanske luktade han riktigt mycket ost.

}

\small{
\textbf{Gammpojkar}
\label{4dcf505f68cb2f0708155b78f56ad632}
 En man som trots sin ålder behållit sitt oberoende, man framlever alltså sina dagar som ogift och barnlös.
 Han bor allmänt kvar i fäderneshemmet (ofta är han hemmansägare) \textsc{(se hemmansägare s.~\pageref{c7b9079472aeb2dda9f09770208aba96})} ibland tillsammans med en bror som även han är gammpojk. Som gammpojk har man gott om tid att lägga på hembyggda  vedklyvar \textsc{(se vedklyv s.~\pageref{34782a935cffabcc8964e073c921a629})},  stövarjakt \textsc{(se stövare s.~\pageref{53be299bc9a8935b8740369c0bc69fd2})},starka drycker \textsc{(se brännvin s.~\pageref{ff49ececa32cff978496a39635496f46})} och radiosporten.
 Man färdas stundom längs byavägen på moped eller möjligen i dieselbil med lastgaller; \textsc{(se dieselbil med lastgaller s.~\pageref{73b1f975c67393304ff101482965163c})} på väg till konsumbutiken \textsc{(se konsumbutik s.~\pageref{70e4875f7c2c177596305006e46b7ca9})} för att handla kokkaffe, ärtsoppa på burk och kanske en korvsnärt \textsc{(s.~\pageref{6fb9ccfbd5699d12ff8d04b2a27852fb})}.

 HEAD2:  Gammpojkar i kulturen
 \begin{itemize}
 \item Esko Männikö har skildrat gammpojkar i norra Finland \textsc{(s.~\pageref{631d44eaa1254ff71a1e11ba021d1266})} i en fin bok.
 \item Euskefeurat har gett röst till gammpojken Leonard Larsson på plattan \quotetext{Hipp Happ}
 \item Frantz Kafka skrev uteslutande om gammpojkar, även om dessa var kontinentala sådana.
 \item Leif Bäckström i den svenska filmen \textit{Jägarna}, skickligt spelad av Lennart Jähkel.
 \end{itemize}
 HEAD2:  Synonymer till Gammpojk
 \begin{itemize}
 \item Gammelgoss
 \item Ungkarl
 \item Grovsingel
 \end{itemize}



 HEAD2:  Se även
 Gammstinta \textsc{(s.~\pageref{3ac827b92df0aaad75cde173b0c608bb})}

}

\small{
\textbf{Gammstinta}
\label{3ac827b92df0aaad75cde173b0c608bb}
 En gammstinta är en kvinna \textsc{(s.~\pageref{9a7760b2521c3471c47cd5d789a2d324})} som inte hittat någon lämplig karl att gifta sig med. Det kan bero på att hon har jättehöga krav eller att alla karlar i världen har en massa brister, eller någonstans mittemellan. Man kan tycka att varje gammstinta borde hitta sig en gammpojk \textsc{(se gammpojkar s.~\pageref{4dcf505f68cb2f0708155b78f56ad632})}, men det händer väldigt sällan.

 HEAD2: Gammstintor i kulturen
 Bridget Jones
 Elaine Benes

}

\small{
\textbf{Garden can}
\label{f2d91d42b4daffbbed6659b449fcb156}
 är en idrott som har ungefär två utövare och som uppfanns i Pajens \textsc{(se Paj s.~\pageref{0b438dd454bc6a17de239ebf0a46b91b})} morsas trädgård för kanske fem år sedan. Den går ut på att försöka kasta en snusdosa på sin polares skrev och att med fötterna förhindra att själv bli träffad i skrevet. Spelarna ligger på marken och kastar upp dosan i luften, så att den vänder och faller ner mot motspelarens pung. En lyckad träff kallas en pouch-hitter och ger, förutom glädje och ära, tre poäng. En lyckad lyra med fötterna förlenar spelaren tre poäng, medan en lyra med händerna inte ger någon poäng men kan vara ett nog så nödvändigt ingripande. Garden can är nära besläktat med det namnlösa spel som går ut på att alla står i en halvcirkel runt pajen och kastar äppelkart på honom så hårt de kan, medan han försöker försvara sig genom att slå på äppelkarten med skaftet till en crocketklubba. I det spelet utdelas dock inga poäng.

}

\small{
\textbf{Geezer Butler}
\label{a923b43d339c51d0835094d5aa59aefb}
 Terence Michael Joseph “Geezer” Butler är en brittisk man som inte alls är butler utan basist i hårdrockens grand old band Black Sabbath. I intervjuer är det lätt att missta herr Butler för en helt vanlig hippie \textsc{(s.~\pageref{14fd61fa8edcb67c5c7886f11af8431e})} som puffat på pipan lite för länge men om man läser texterna han skrev för Sabbath om religion, magi, krig och andra otäcka saker förstår man att han inte är en snubbe att leka med. Räcker inte det som bevis kan man googla fram bilder från runt 1980 där Geezer ofta manifesterade sin råhet med svarta spandexdräkter med eldflammor på. Känsliga rockers har varnats.

}

\small{
\textbf{Gekås}
\label{2dbf90982c35d6e7b8d3e171ccff40c5}
 Ullared AB grundades 1963 och är Skandinaviens största varuhus. Med fyra och en halv miljoner besökare årligen utgör det Sveriges \textsc{(se Sverige s.~\pageref{b1999637949ed135b2ca03f3a38460cc})} största utflyktsmål.

}

\small{
\textbf{Gengas}
\label{309c38192f208b8dd3bf1f58cc214016}
 är en gas som uppstår vid ofullständig förbränning av trä och kol. Men hjälp av ett gengasaggregat kan gasen renas och regleras på ett sätt som möjliggör att använda den som drivmedel till bilar. ”Det här låter ju som rena storfiskarhistorien”, tänker du nu. Men under det stora fosterländska kriget \textsc{(s.~\pageref{8e55572fc7b7490da402e43a822eb3da})} fanns i Sverige faktiskt fler än 70.000 bilar som drevs på detta sätt. Det är lätt att föreställa sig en romantiserad bild av hur föraren märker att mätaren börjar närma sig rött och då bara ler, öppnar handskfacket \textsc{(se handskfack s.~\pageref{651fd6de0e0851f657a1c9b76b76f692})}, plockar ut ett vedträ som han slänger in i kaminen i baksätet. Riktigt så fungerar inte gengas men man önskar att den gjorde det. I realiteten sliter gengas något så djävulskt på motorn och det händer ibland att aggregaten sprängs. Så sker till exempel i Åke Hodells bok \textit{Elddopet}, en bortglömd klassiker i den svenska litteraturen.

}

\small{
\textbf{Genuint snåla människor}
\label{b48797ecad31c4b98b780e115e70dcc1}
 är sådana som arbetar inom offentlig sektor och röstar på moderaterna \textsc{(se moderat s.~\pageref{c4564b188cb670841733a3ff923c2fb0})} för att de vill undgå att betala skatt. Typiskt kälkborgeri \textsc{(se kälkborgare s.~\pageref{0f34b469a48952e93688861083ace75a})}.

 En annan kategori genuint snåla människor stöter du på i sådana hem där pappersinsamlingen förvaras inne på toaletten. Samma kategori värmer bara upp ett rum i huset och klagas det på kylan kommer svaret \quotetext{Men gå till varmrummet då!}.

}

\small{
\textbf{George Best}
\label{f4288789b1401dc1595a0cb6f22d5b93}
 var världens besta fotbollsspelare. Han föddes i den besta delen av det brittiska samväldet; nämligen Nordirland år 1946 och dog i skam och förnedring 2005. George var en riktig idrottsman som inte behövde dopingpreparat eller hälsokost för att vara best. Enligt uppgift kunde han dricka brännvin \textsc{(s.~\pageref{ff49ececa32cff978496a39635496f46})} och spela fotboll \textsc{(s.~\pageref{961bd74d34872ff94a4df3a16119096e})} simultant.

}

\small{
\textbf{George Everest}
\label{36c3305a3fc7dbde2cdc868b00cc1af4}
 Lantmätare från Wales (4 juli 1790 – 1 december 1866, från 1861 \textit{Sir} George Everest). För allmänheten känd som den person Mount Everest \textsc{(s.~\pageref{b5b5d890ef4ff008c7821da350799545})} är uppkallad efter. För sina vänner främst ihågkommen som han som alltid slet av sig byxorna när han var full.

}

\small{
\textbf{Georgij Zjukov}
\label{92edaad13ba11b17963e644994fa1a6f}
 var en sovjetisk militärgeneral som bland annat listade ut hur man skulle inta Berlin och återta Leningrad. För detta och lite annat blev han utnämnd till \textit{Sovjetunionens hjälte} hela fyra gånger, två fler är självaste Jurij Gagarin \textsc{(s.~\pageref{65a8533cd59bc537e760118da0751c50})}. För att öka hajpen ytterligare lät kommunistpartiet ge ut serietidningen \textit{Rött Inferno}, som handlade om Zjukovs bedrifter. Precis som Haile Selassies \textsc{(se Haile Selassie s.~\pageref{9accb6bc1893f934737ecb7710da49a4})} mausoleum är Zjukovs idag en offentlig toalett.

}

\small{
\textbf{Gertrud}
\label{8598eece636f53052b0d6cc2cc2409da}
 är ett gammsvenskt namn som är en sammansättning av gammsvenskans ord för \quotetext{bjär} och \quotetext{skrud,} det vill säga OP-klänning.

}

\small{
\textbf{Gibba}
\label{a9e46bf0bf77017657c4c607ffe3a2ef}
 Att gibba är att spela datorspel. Man syftar sällan på spel som tetris, harpan eller sims när man talar om gibb. När man gibbar ska det gärna finnas ett element av konflikt och risk för död med i leken (märk väl, död kan inträffa i sims också, men det är inte spelets syfte). Spel som dota, unreal tournament, counter strike och andra våldsglorifierande hack n' slash-lir är tydliga exempel på gibbspel.
 HEAD3: Ursprung
 Ordets etymologi förklaras av användaren jumpcut på forumet flashback: \quotetext{\textit{Ordet }gib\quotetext{ kommer från }giblets\quotetext{ (http://en.wikipedia.org/wiki/Giblets) och innebär att man förvandlar motståndaren i ett FPS till just giblets när man fraggar dem, dvs dödar dem. Om jag inte minns helt fel så var det när Doom kom som man för första gången kunde gibba folk (eller var det Heretic? Lite rostig)}}.

}

\small{
\textbf{Gifta sig}
\label{fde23aab828cff50483a59fc662f8fa8}
 När giftemålet närmar sig och det är tager-du-denna-dags finns det en sak som är viktigare än allt annat att ta i beaktan - och detta gäller inte bara dig själv, utan är något som goda kamrater bör tänka på utifall att den tilltänkte brudgummen är en nära vän som man inte vill ska gå i fördärvet. Detta fenomen är även mera aktuellt i Norrbotten \textsc{(s.~\pageref{0e8c003b75982032cde152609ee94154})} än på andra ställen:

 \textbf{Gift dig ALDRIG med någon som kommer från en grop i skogen!}

 Speciellt om gropen är belägen i tät granskog. Bristen på ljus och vyer gör dessa \quotetext{gropmänniskor} galna och sinnesrubbade. Något som ofta inte märks förrän det är för sent. T.ex: allt var frid och fröjd tills en dag vid middagen. Pang! Yxa \textsc{(s.~\pageref{bd74f429522c7c1481fbba07187efc6b})} i huvudet.

 HEAD2: Kända exempel på varför man inte ska gifta sig med någon från en grop i skogen
 HEAD3: Från Karesuandotrakten?
 Suijavaara, Parkalombolo, Närvä, undvik flickebarn från dessa trakter.

 HEAD3: Från Norsjötrakten?
 Är du från Norsjötrakten bör du tänka efter både en och två gånger innan du gifter dig med en man som kallas \quotetext{Slägg-Birger}.

 HEAD3: Från Rökåtrakten?
 Om du har rötter i trakterna kring Rökå \textsc{(s.~\pageref{b06106b8f786098f1ff569a4f75dc3c8})} bör du kanske inte skaffa barn med din gemål om denne också är från bygden, av rent biologiska skäl. Byn bebos av 50 personer uppdelat på tre efternamn.

}

\small{
\textbf{Gism}
\label{025b6da73f168a6d2e766e79c9b2941a}
 Världens bästa band!

}

\small{
\textbf{Gitarr}
\label{a08bf8420208934bc59c7ed7385d4308}
 Ett lika vanligt som älskat stränginstrument vars form påfallande mycket påminner om hur kurviga kvinnor målas och skulpteras av medelålders \quotetext{livsbejakande} manliga \quotetext{konstnärer}. Många medelålders män \textsc{(se man s.~\pageref{39c63ddb96a31b9610cd976b896ad4f0})} har också ett slags nostalgiskt och känslofyllt förhållande till gitarren, trots att den är ett relativt vanligt förekommande instrument och en ägodel som påträffas i snart när vartannat hem i det västerländska samhället. Detta gäller även, eller kanske framförallt, om personen i fråga knappt kan spela gitarr och kanske aldrig ens ägt en sådan. Även om inte mannen kan spela gitarr för att rädda sitt liv hindrar detta honom inte för att tala om självklara klassiska gitarrer som \textit{Fender Stratocaster} och Gibsons \textit{Les Paul} och påminna omgivningen om världskända gitarrister med mindre imponerande förmågor när det gäller att faktiskt producera ny, kreativ musik - så som Eric Clapton och Sting. Det faktum att gitarren samtidigt påminner om en kvinnokropp och en fallos ger en viss hint om varför så många män ser på den med nostalgi och längtan tillbaka till det förflutna, vilket dock inte gör den komplexa sexual-psykologiska problematik som ligger bakom detta fenomen mer överskådlig.

}

\small{
\textbf{Gitarrkille}
\label{a1f5ce8e48efeec2e430afcfed2ef045}
 En gitarrkille har ofta hatt, väst och problem med sin självinsikt. Han kan vara i vilken ålder som helst, och nästlar sig gärna in på alla slags tillställningar. Han har inte hört talas om jantelagen och bör lugna ner sig och bete sig som folk. Han slaktar gärna den ena covern efter den andra, utan någon som helst tanke på att folk helst kanske skulle önska sig att han behöll sin kreativitet för sig själv och sitt pojkrum. En äldre gitarrkille kallas med ett finare ord för trubadur, och/eller ses som frontsångare i ett coverband.

 ----

 En gitarrkille bör hanteras hårt och bestämt; knip tillgångarna (gitarren) och höj musiken på stereon när du anar vad som kan komma att ske. Uppmuntra honom inte på något sätt om du vill ha en fortsatt trevlig kväll.

}

\small{
\textbf{Gitarrmys}
\label{9bf70010c64634e237ceef636561c4c9}
 är en populär aktivitet på lärarlagsfester och liknade tillställningar och går ut på att killen slash \textsc{(s.~\pageref{9fbbaa4cc515bc46e0c12e82a31df736})} tjejen med musikerambitioner i läraraget plockar fram en gitarr och spelar lite låtar som de andra kan sjunga med i. Oftast inmundigas alkohol före, men å andra sidan är det just gitarrmyset som förhindrar att alla blir grisfulla \textsc{(se grisfull s.~\pageref{80fc21ba5a45f2d0cd24855d78fa7246})} och idkar samlag med varandra. Många gånger startar gitarrmyset lite oväntat, eftersom folk varit för indragna i något samtal som börjar bli lite gapigt, när gitarrkillen slash \textsc{(s.~\pageref{9fbbaa4cc515bc46e0c12e82a31df736})} tjejen plötsligt utbrister i \quotetext{Halt!..........(Paus för att de andra på festen ska förstå vad som händer) .... Här får ingen passera!} och så är gitarrmyset igång. Senare kommer någon Cornelis Vreeswijk-låt och kanske, om man har tur, nån Tomas Ledin-bit.

}

\small{
\textbf{Gittan}
\label{867eaf107626e46efe4abf21d5ca704e}
 är ett kvinnonamn som betyder \quotetext{den som är en jävel på att lira ackegura}.

}

\small{
\textbf{Glad och pigg tjej}
\label{8ad47065f969f6c49b3a91a6dc8daf6f}
 En glad och pigg tjej är en kvinna \textsc{(s.~\pageref{9a7760b2521c3471c47cd5d789a2d324})} som tillhör en av Sveriges \textsc{(se Sverige s.~\pageref{b1999637949ed135b2ca03f3a38460cc})} mest utbredda sub-kulturer. Som den närbesläktade glada och spralliga tjejen är hon positiv och utåtriktad och nästan alltid redo att hitta på något kul, så som att gå och shoppa, laga middag och dricka ett glas vin tillsammans med väninnorna eller kanske gå en rask promenad. Hon sysslar kanske med mindfulness, har en spik- eller yogamatta, hänger tavlor av klassiska godisaskar så som Djungelvrål och Salta Katten och försöker att inte se så mycket på TV, även om det faktiskt blir en hel del trevliga pratprogram ändå. När den glada och pigga tjejen skriver brev eller på något vis presenterar sig på internet påpekar hon vanligtvis att hon är just en glad och pigg tjej och tillägger därefter sin ålder.

 Den glada och pigga tjejen:
 \begin{itemize}
 \item Jobbar i reception
 \item Äter gärna lätt och fräscht
 \item Bor ljust
 \item Läser magasin
 \item Avskyr svält och krig
 \item Ogillar sorgliga eller våldsamma filmer. Man tittar väl på film för att ha kul?
 \end{itemize}

}

\small{
\textbf{Glasbygdsball}
\label{fc73b06dec2dd6cd6ea004812c90d63e}
 Glesbygdsball \textsc{(s.~\pageref{f28697650d6058a85247bc65ae7166d3})}

}

\small{
\textbf{Glassbutt}
\label{769564f911c36d45768dc8ae69b8af0b}
 , (på engelska \textit{tupperware}), är ett kärl speciellt framtaget för att transportera lunchmat från hemmet till en annan plats. Ett genomsnittligt kärl rymmer mellan 0.5 och 1 liter. Överstiger kärlet 1 liter kallas det lunchtråg \textsc{(s.~\pageref{1e0e0470206e0f2baad8e628ba8f770c})}. Glassbutten uppfanns av en ren slump av den finske industridesignern Uri Marimekkonen (1912-1996) när han skulle konstruera en begravningsurna i det nya trendmaterialet plast. För att lansera den nya produkten genomförde Uri ett PR-jippo där han lät fylla de 500 första kärlen med glass. Produkten blev en succé och reklamkonceptet används fortfarande på sina håll i världen. IKEA försökte som vanligt plagiera designen och lanserade den billigare varianten glasspaketet. Det är inte på långa vägar lika funktionellt att transportera mat eller människostoft i.

}

\small{
\textbf{Glassbåt}
\label{c20258a87c6be491dbc073f946dd083c}
 är en skapelse som kombinerar det bästa av två världar. Glass: godisets närmaste släkting men inte alls lika tabubelagt. Båt: symbolen för drömmen om det fria livet på sjön. Med ett ytterhölje av kex kan du hålla i den utan att bli kladdig och slippa bry din lilla hjärna med vad du ska göra av pinnen. Till utseendet påminner den om Noaks ark, såsom den avbildades i Gustave Dorés bibel. Och det kan ju gå an, så väl som den smakar.

}

\small{
\textbf{Glasse}
\label{3b2b4e26097f9dc1b8baa4c53609563f}
 är företaget Triumfglass maskot. Han är en isbjörn och det bästa han vet är glass. Han har haft en lång och krokig karriär men mår just nu ganska bra. Triumfglass startades 1946 och strax där efter gjorde Glasse sitt intåg, till många barns \textsc{(se barn s.~\pageref{5dfcc0aab2f3db925b2d51ba73e48946})} lycka. År 2003 försvann han dock till synes spårlöst och sommaren var sig inte längre lik. Det visade sig att han kidnappats av det norska klasskonsortiet Diplom-Is, som lagt rabarber på Triumfglass och ersatt Glasses glada nuna med den betydligt fulare Eskimonika. Reaktionerna lät inte vänta på sig och i februari 2010 bytte Diplom-Is namn till Triumf Glass AB och Glasse är populärare än någonsin. Så kan det gå om man bråkar med Glasse.



 Kuriosa: 1961 ändrade Triumfglass namn på glassen Jätteneger till Chokladjoker.

}

\small{
\textbf{Glasögon}
\label{bb2f2cb84c42a821763d572f86b1e3c9}
 är en attiralj som i varierande grad förbättrar din syn.
 Motsatsen till glasögon kallas metanol.

}

\small{
\textbf{Glenn}
\label{3c784bff199ef62ecc2f3a988f395c62}
 Titeln Glenn tilldelas framstående göteborgare och relaterade personer inom bollsporter såsom gruppbollsparkning, bollkastning och ölbrännboll. Namnet påstås ha irländska rötter men verkligheten är något helt annat.  Glenn som namn kommer från det fornsvenska ordet för varvsarbetare, ett i Göteborg traditionellt högt aktat yrke. Detta ledde till att Glenn övergick i en titel, då svensk varvsindustri gick i putten efter att regalskeppet Wasa satte industrin på pottkanten, nådens år 1628 (fakta).

}

\small{
\textbf{Glesbygdsball}
\label{f28697650d6058a85247bc65ae7166d3}
 I vissa tätorter betonar man sin metropolitiska avart genom att uttryckligen förhålla sig dialektiskt till glesbygden omkring sig och genom att uttalat tycka att det är ”genuint” att komma från orter med få invånare. Om man faktiskt gör det kan man gôtta sig i att leva i villfarelsen att Jula är en klädaffär och att vintern är ett tillfälle att visa vart skåpet ska stå - ombord på en maskin som stadsinvånaren aldrig tidigare sett, men genom sin teoretiska förståelse av fyrtaktsförbränning ändå talar om som en redan kasserad ägodel. Personen från glesbygden har dock alltid ett övertag i sin mer genuina konstitution och otvungna personlighet.

}

\small{
\textbf{Glida under radarn}
\label{a14f07b32632e6696da7a9510112ddbf}
 Den provisoriska titeln på ett manuskript till landsförrädaren Stig Berglings ännu outgivna spänningsroman. Innehållet sägs vara av huvudsakligen självbiografisk karaktär, med färgstarkt skildrade internationella intriger och livsbejakande erotiska \textsc{(se erotik s.~\pageref{972f097461d1eab1c1ff104757bad922})} anekdoter.

 Enligt källor existerar texten i ett enda (ofullständigt) fysiskt exemplar, vilket tillföll Berglings polska ex-hustru i samband med parets andra skilsmässa 2004.

}

\small{
\textbf{Glima}
\label{b46db05ac9d1b0bb2019c68b73335729}
 är en form av ganska störtlöjlig vikingabrottning som går ut på att man sliter varandra i bältet tills den ena tävlanden ramlar, eller blir \quotetext{omkullglimad}. Dock populärt bland nazister.

}

\small{
\textbf{Glimröv}
\label{dd0a3a947a541f3d83b53a56be518062}
 Thor och Oden lär ha haft ståtliga glim \textsc{(se Glima s.~\pageref{b46db05ac9d1b0bb2019c68b73335729})}.

}

\small{
\textbf{Globen}
\label{c520b11670b9cef944588fe3849ce491}
 är ett slags enorm golfboll som man kan gå in i för att kolla på ishockey eller U2. Bollen står i södra Stockholm \textsc{(s.~\pageref{edcd259e0a03c7ab70feb186bae19f13})} och kan ses enda från Danderyd en klar dag. Globen anses sedan 2007 vara ett av Sveriges sju underverk \textsc{(s.~\pageref{f4f71e4db3f279d42d840c805d75820c})} eftersom den är så rund och golfboll-lik.  De som inte varit i Stockholm har förmodligen ändå sett Globen på TV i slutscenen till Björn Skift gamla kioskvältare \textit{Joker} som visas på TV4 med jämna mellanrum.

}

\small{
\textbf{Globetrotter}
\label{94ac1a29b3fdede652f45d5962de646f}
 En globetrotter kan vara en person som reser väldigt mycket och har genom det fått en självklar världsvanhet som är väldigt irriterande för globetrotterns nära och kära. De kan haspla ur sig grejer i stil med \quotetext{Åh det här är som när jag var i Tokyo förra hösten...} och resten av sällskapet tvingas nicka och le instämmande och känner sig som att de aldrig varit utanför 50-skyltarna.

 Globetrotter är också namnet på en av Volvos lastbilshytter som tillverkas vid företagets Umeåfabrik. Den är stor och rymlig och som gjord för att åka jorden runt i, men det är inte därifrån den fått sitt namn. I själva verket så är det Volvos smygkommunistiska fackledning som döpt den till Globetrotter som en hyllning till Leon Trotsky \textsc{(se Trotta s.~\pageref{918b2980ffb5f16acf768fa89f71021b})} och den globala revolutionen.

}

\small{
\textbf{Glop}
\label{aae22e6d62a99e31db1de383aa15e538}
 Ett stadie i en människa av mansköns liv vilket inträffar mellan gosse och karl.
 Moderna glopar hänger utanför Coop \textsc{(s.~\pageref{0b5cb0ec5f538ad96aec1269bec93c9c})} och har ofta wct-byxor \textsc{(se wctbyxa s.~\pageref{dc78f9615e53d0ddb525d3975197a781})}, huvtröja samt spottar obscent mycket på marken.

}

\small{
\textbf{Glottis}
\label{0b4c164a8f8b7eb1f7afe08e499b6381}
 består av av stämläppar (stämband) och röstspringa och finns nere i halsen på de flesta människor, som en dold liten mun \textsc{(s.~\pageref{6585f290ce92c3de5ff339920330e26f})}. Med hjälp av glottis formas ljud, så kallade allofoner, som sedan tillsammans skapar ord och meningar (morfem/lexem och satser). Vetskapen om detta gör lätt att språket genast känns lite äckligt. Usch.

}

\small{
\textbf{Glädjevetenskaper}
\label{7e4eadb905a6345ef2a6ce2b5b179847}
 är akademiska forskningsområden som skänker folk stor förnöjsamhet och hopp om en vackrare värld. Den största glädjevetenskapen är ortnamnsforskning som syftar till att utröna de etymologiska betydelserna bakom nordiska ortnamn. Ett enda ortnamn, som exempelvis Lövånger, kan sysselsätta en ortnamnsforskare i flera år och ändå skänka denne stor tillfredställelse när historien äntligen är klarlagd. Det finns ju flera hundra Lövånger i Sverige! Ortnamnsforskningens vita val är Medelpad. Vart ligger egentligen meden mellan padarna? Så där håller det på, till allas glädje.

}

\small{
\textbf{Gnagare}
\label{b5c3a0f14d5f76de604f5d8e4cc068ff}
 En gnagare är någon som, av någon anledning, alltid vill att AIK (Allmänna IdrottsKlubben, Solna, Stockholm) ska vinna när det är match. AIK vinner dock inte alltid, vilket gör gnagaren ledsen och nedstämd. Gnagaren går då hem till sitt krypin någonstans i Solna eller stannar ute och slåss med kniv. Det är vanligt att gnagaren är nynazist.

}

\small{
\textbf{Gnussa}
\label{054df2b2d5a6a72109b5ba59cbc991b9}
 är en term som används inom den samtida pardans-kulturen. Att gnussa innebär att två danspartners liksom gnider och smeker sig mot varandra medan de dansar. Speciellt kontakt mellan partnernas ansikten är viktigt för att det ska röra sig om gnussning.

 HEAD2: Trivia
 Det är strängt förbjudet att ovälkommet gnussa någon utanför dansbanan.

}

\small{
\textbf{Gobit}
\label{86063054a9db002c783f9cac4f459803}
 En gobit är den godaste delen av en viss matvara - det kan i princip vara vad som helst förutom grytor och soppor. \textit{Gobit} används ofta i samma betydelse som det mer \quotetext{fräcka} ordet \textit{godis}.

}

\small{
\textbf{Godisautomat}
\label{6b30cb1d93494052f2ac926395564e9e}
 En godisautomat är en smart mojäng som gör det möjligt att få en helt annan sorts godis än den man betalat för. Godisautomater finns i två utföranden: En som tar enkronor och en som tar femkronor. Man stoppar in det angivna myntet i en springa och vrider på ett handtag och vips så kommer fel sorts godis ut ur en lucka.

}

\small{
\textbf{Gokart för alla}
\label{b99dfe0e7ded4a07eab2b2958139fd8e}
 är namnet på ett företag som driver en gokartbana i Malsta, en jordbruksort \textsc{(s.~\pageref{3257bf804d763afce5a153f73ce80f7c})} utanför Norrtälje \textsc{(s.~\pageref{7527f7dad9445013a559dc7e2a91f3b3})} i Stockholms län.
 HEAD2: Banor och bilar
 Gokart för alla har två banor. En är för barn och en är för vuxna. Banan \textsc{(s.~\pageref{aec7bd708ed2ad3435b9a9883ac7f45c})} för vuxna är 460m lång. Håll i dig nu, för nu blir det lite komplicerat: Om man är över 135cm kan man få åka den den mellanstora gokartbilen (Dino medi) på den stora banan. Om man är äldre än fem år och under 135cm får man köra den lilla bilen (Dino mini) på den lilla banan. Om man är över 155cm lång (och, får man anta, över fem år gammal) får man köra den stora bilen (Dino maxi) på den stora banan.

 HEAD2: GPS-koordinater till Gokart för allas bana
 N 59.76750
 E 018.61617


 HEAD2: Tidtagning
 Elektronisk tidtagning görs med hjälp av Amb IT-system.

 Extern länk: Gokart för allas hemsida [http://www.gokartforalla.com/]

}

\small{
\textbf{Golden retriever}
\label{2b3d7c01f0a8a57d8fa2a18b54993a6b}
 Du och din partner har varit ganska lyckliga (så när som på några mer eller mindre allvarliga bråk om \quotetext{inställningen} och hur \quotetext{allvarlig} man är med allt det här) och har köpt in er i ett radhus med hjälp av era föräldrars besparingar. Nu visar det sig att din partner just blivit befordrad, så trampet av barnafötter får med andra ord vänta ännu några år. Men man ska väl inte skynda in i något heller, helst. Så... Ett steg i taget.

}

\small{
\textbf{Golf}
\label{c6cf642b8f1cac1101e23a06aa63600e}
 (egentligen \quotetext{moderatbandy} på svenska) är en skotsk sport där utövaren slår allt vad den orkar med en pinne på en liten boll för att se vart den tar vägen. Världens bäste golfare heter Kim Jong-Il, som enligt egen uppgift snittar 3 hole-in-one per runda.

}

\small{
\textbf{Google}
\label{c822c1b63853ed273b89687ac505f9fa}
 är namnet på en sökmotor på internet som fungerar på samma sätt som mer klassiska sökmotorer såsom Lycos och Altavista. Google erbjuder även andra tjänster på sidan om och kommer så småningom äga hela internet.

}

\small{
\textbf{Googol}
\label{5fa3d764647715575ab03e273f92fd47}
 En etta följt av 100 nollor. Det är det största talet som matematiker kan tänka sig. \quotetext{En etta följd av 101 nollor då?} tänker kanske ni, men då skakar bara matematikerna på huvudet och väser argt: \quotetext{Sluta!}. Eller nä då säger han eller hon att googolplex är det största talet, som är 10 upphöjt i en googol. Ja så måste det vara.

 Den här artikeln borde första skriva ut talet men det skulle vara asjobbigt, fatta att skriva en 1 etta följt av 0000000000000000000000000000000000000000000000000000..., äh ni fattar va?

}

\small{
\textbf{Gore-Tex}
\label{cc1e3b66bcda33218427995652e4e31a}
 är ett material som är vind- och vattenavvisande, men ändå \textit{andas}. Av Gore-Tex kan man bl.a. göra regnkläder. Man kan undra varför kläder behöver \textit{andas}, men prova jogga i galonbyxor så kanske ni förstår. Har en person mycket kläder med Gore-Tex i eller på är den personen friluftsintresserad, vegan \textsc{(se veganer  s.~\pageref{2a12d5d6ae91d2f4f7d9af3cef58e75c})}, båda eller en vanlig människa utan sinne för ekonomi.

}

\small{
\textbf{Gottis}
\label{90555c8edd726ba0b0a03d0676a4ae48}
 är ett slangord för godis och uppfannas av en föredetta uteliggartant som i ett inslag på Västerbottensnytt, mumsandes på lösgodis och liggandes i en säng i den friggebod hon just fått flytta in i, sa till intervjuaren att \quotetext{det är så skönt att ligga framför TVn och äta lite gottis.}

}

\small{
\textbf{Gouverment}
\label{9550c7da7c1951c5e35799616c4e5387}
 I don´t need sex, the Gouverment fuck me every day.

}

\small{
\textbf{Grabbrent}
\label{b9a8d4c49a300de05ce98dfa59b80ff8}
 När det är grabbrent finns det inga äckliga fläckar, inga kläder på golvet och inget som stinker. Endast vagt besläktat med tjejrent \textsc{(s.~\pageref{904a1c27d10e1b9d641f57c45953ec3f})}.

}

\small{
\textbf{Grafologi}
\label{1e07a05022fa87af0025f5d2a0dea18b}
 går ut på att studera sambandet mellan utseendet på handskriven text och egenskaper hos personen som skrivit. Den moderna grafologin uppfanns av Carl von Linné \textsc{(s.~\pageref{5e8380bf6b7ce99678e6752b6d9e709e})} efter att denna tröttnat på anklagelser om att vara usel på partytrick. Om många ord slutar på \quotetext{E} är det troligt att en kille skrivit medan många ord som slutar på \quotetext{A} tyder på att det är en tjej. Skriver man skrivstil har man troligtvis glasögon medan konstiga ord tyder på att man är utlänning. Korta meningar med ett hårt tryck mot papperet tyder på att författaren haft bråttom eller varit arg. Är texten skriven med sprayburkar har personen stora byxor.

 Kända grafologer:
 Hans Scheike
 Max Pulver

}

\small{
\textbf{Grammatik}
\label{7b921f88471155db0c3e3230378553a4}
 En mening består vanligen av predikat, subjekt och objekt. Detta uttrycks som så: \textit{Gillaruna?} med manligt resp. neutralt objekt: \textit{Gillaruan?} och \textit{Gillarure?}. Detta är även en så kallad Rhotacism. Ett förfall som pågått i årtusenden där -d- i det ursprungliga \textit{Gillarude?} blivit ett -r-.

 HEAD2: Sagt om grammatik

 \quotetext{Om språket är ett mäktigt vapen så är grammatiken dess kikarsikte} - Kråkan \textsc{(se Användare: Hawaii-kråkan s.~\pageref{a777da05d6e59c7961af7b56578d657a})}

}

\small{
\textbf{Grand Funk Railroad}
\label{fd425d96c8908215d6cdd7a72ac97c27}
 , eller GFR som fansen kallar det, är ett rockband från Flint i USA. De har gått till rockhistorien för att de var först med:

 \begin{itemize}
 \item Distad bas
 \item Att dra distratten förbi 7
 \item Långt hår
 \item Cowboyhatt
 \item Jeans \textsc{(s.~\pageref{a0f2589b1ced4decbf8878d0c3b7986f})}
 \item Strupsång
 \item Gitarrsolon
 \item Dubbeltramp
 \item Att ha en trummis som dör under mystiska omständigheter
 \item Powerackord
 \item Sjungande trummis \textsc{(s.~\pageref{86761451ff4f807963858d8a2afece37})}
 \end{itemize}

}

\small{
\textbf{Grants}
\label{74dc2f36b83c605847a3519729a18d11}
 Om man som fjortonåring har sin första fylla på en halv hela Grants så kräks man. Detta kan bota ens emetofobi \textsc{(s.~\pageref{afcfb287e0c9a3f9fa7a3e6e748afdcf})}. Eller förvärra den.

}

\small{
\textbf{Graviditet}
\label{abe06a47f108826313da397bb6d5056e}
 är en slags parasitsjukdom som drabbar nästan hälften av världens kvinnor. De drabbade får en böldliknande utväxt på magen. Smittar genom oskyddat sexuellt umgänge men endast kvinnor drabbas av parasiten \textsc{(se barn s.~\pageref{5dfcc0aab2f3db925b2d51ba73e48946})}.. Det finns profylaktisk behandling och även behandling då man väl drabbats. Risken finns att man efter 40 veckor drabbas av något än värre än en intern parasit som tar både plats och energi, då kommer parasiten efter att ha orsakat enorm smärta och stor skada på underlivet kräva mycket stor uppmärksamhet och kan kosta de drabbade enorma summor pengar!

}

\small{
\textbf{Gravitation}
\label{276211224224cdf8211307450ef6e915}
 Jordens dragningskraft är inte riktigt så konstant som man kanske kan tro. Det är högre gravitation i badrummet kring badrumsskåpet och toaletten än på andra ställen, feppla med din mobiltelefon kring toan eller plocka ur något ur badrums-skåpet skåpet så får du se ;)

}

\small{
\textbf{Grekiska statsobligationer}
\label{8640f62d0945ef7a0a7364dcd9c0d4c9}
 är ett oförtjänt baktalat sorts värdepapper. För den ekonomiske lekmannen räcker som förklaring att pappret berättigar ägaren till utdelning vid försäljning i den mån den grekiska ekonomin går med förtjänst. Här har vi ett land och örike som kan sägas vara den europeiska kulturens vagga, där demokratin och den logiska filosofin föddes, ett land vars skönhet lockat mången konstnär och skulptör att för en tid bosätta sig i ett högt beläget åsneskjul för att med detta som utgångspunkt kunna förkovra sig i den kulturskatt som dväljs i riket Grekland. Knossos, Sokrates, fetaost.. många är de uppslagsord som Grekland producerat sedan antikens storhetsdagar. Varför skola då just grekiska värdepapper icke vara tillförlitliga så som sparform? Om detta säger Prof. Etienne \textsc{(se Användare: Prof. Etienne s.~\pageref{a9878d2280e5a39becac8f73d113df91})} följande kloka ord: \textit{Köp, vid filosofens skägg! Köp!}

}

\small{
\textbf{Greven}
\label{330e9f7b3cb7b66c26ace97d18c3c8cd}
 Krilles pappa

}

\small{
\textbf{Grillfrack}
\label{d9e4382ef266d7e62d51649556a86061}
 är ett nyttigt plagg som kan användas i alla lägen, består till 100\% av militär överskottsull (färg grå). En grillfrack är innerfodret i någon permisrock, för att få permis måste man göra lumpen, ingen gör lumpen längre, ingen SAAB tillverkas heller längre, hindrar dock inte att de används och är bra. Jag använder grillfrackar svala sommarkvällar då man tvingas att vara ute bara för att man fått besök av grattisätare. Grillfrackar skall finnas till alla gratisätare så att man slipper kommentarerna. Man skall inte åka vattenskidor i grillfrack på midsommar, det blir både blöt och kallt samt något tungt både för vattenskidåkare och dragbåt.

}

\small{
\textbf{Grind}
\label{2568fa9f52b34de6328f5044555fe7b6}
 Ett slags dörr eller jättesnabb punk.

}

\small{
\textbf{Grisfull}
\label{80fc21ba5a45f2d0cd24855d78fa7246}
 ; kraftig alkoholberusning av ungefär samma typ som pissfull. En grisfull person har vanligtvis svårt att tala i sammanhängande meningar och kontrollera sin urinblåsa. Om flera grisfulla personer vistas på en liten yta är chansen stor att handgemäng kommer uppstå.

}

\small{
\textbf{Grissini}
\label{7f499faec0a16359a2e8cfba23960924}
 är en slags brödpinnar som består av mjöl och inget mer. Det smakar inget, man blir fet men inte mätt av det och det smular. Grissini står ofta framdukat på finare restauranger redan när man tar plats vid bordet. Tanken med detta är att det ska kännas lyxigt att man erbjuds detta gratis. Yes, en pinnformad mjölbit som kostar mindre än en fis i rymden \textsc{(s.~\pageref{6d5ad1e8996d7ec9d8ac6058649290c0})}, fan vad generöst. Den medvetne konsumenten bojkottar grissini och köper istället dess delikata kusin salta pinnar som smakar utsökt, är fria från fett och höjer stämningen på alla fester.

}

\small{
\textbf{Groggvirke}
\label{ba264d4eb820b4066de4c8723a08f824}
 Detta blir man bakis av och det bör därför undvikas.

}

\small{
\textbf{Gross}
\label{58d949771b2a49016259a9fb4fa7499e}
 Ett hundra fyrtio fyra stycken. Ett dussin \textsc{(s.~\pageref{e4616326552f9ff9435bf747d0495940})} dussin.

}

\small{
\textbf{Gruk}
\label{78233c5ad0b90efdffd147b849201ce4}
 är, som läsaren förmodligen redan gissat, ett danskt versmått. Snubben som uppfann det hette Piet Hein (1905-1996) och var på äkta danskt vis förutom poet även vetenskapsman, matematiker, filosof, författare, uppfinnare, konstnär och den som formgav Sergels torg. Precis som de flesta danskar. En klassisk gruk är kort och koncis och bör innehålla ett paradoxalt eller satiriskt inslag. Under Tysklands \textsc{(se Tyskland s.~\pageref{b1b58da783b6d5fa090f3015f1889869})} ockupation av Danmark \textsc{(s.~\pageref{5331d7fd27772396f412a5b6d19bad44})} blev gruken ett viktigt redskap för att gjuta mod i motståndsrörelsen (tyskar förstår som bekant inte ironi och man kunde därför trycka grukar helt öppet i dagstidningarna). Hein själv skrev över 7000 grukar som han gav ut i mer än 20 böcker. Sedan hans död har nyproduktionen av grukar kraftigt minskat, liksom Danmarks motstånd till nazism.

 HEAD2: En klassisk gruk
 \textlessi\textgreaterTil antropologiske forskere
 i det inden- og udenlandske
 kaster jeg hermed en handske:
 Jeg paastaar, at Nordmænd er norskere
 end nogen Danske er danske.\textless/i\textgreater

}

\small{
\textbf{Grunka}
\label{6b79b7e074be4c86299c3ee48160b626}
 (verb, infinitiv; avlett substantiv, \textit{grunk}) är en multipel handling där utövaren gråter högljutt samtidigt som den onanerar. Fenomenet har blivit vanligare på senare tid i och med att ens sexpartner avvisar alla närmanden. Gråten måste vara ärlig.

 Grunka kan även användas synonymt med pryl, vilket kan leda till språkförbistringar.

}

\small{
\textbf{Gruvuv}
\label{d70c14fa26991e152ab05c82569e5da4}
 en (Bubo Subterranis) är en sentida evolutionär utveckling av grottuven (Bubo Neanderthalis) och har väldigt bra mörkerseende. Den lever liksom sin numera utdöda föregångare i huvudsak på fladdermöss, men äter även bland annat shiitakesvamp. Jaktsäsongen på gruvuv är sedan 2008 förlagd till industrisemestern på grund av alltför många tidigare missöden av typen \quotetext{Skjut inte på dynamiten, din jävla idiot!} och \quotetext{Ser jag ut som en gruvuv, kanske? Det här ska skyddsombudet få höra.}. I Kiruna och Gällivare är gruvuvskrov en populär delikatess. Gruvuven är rätt impopulär bland stammare.[[file:Gruvuv.jpg\textbarright\textbarNissepedia]]

 Category:Bubologi \textsc{(s.~\pageref{2a0552d751fe7b13dfd1c180d2521392})}
 Category:Djurriket \textsc{(s.~\pageref{7198efe02efcb0dbd7953602b84a13d3})}
 Category:Mat \textsc{(s.~\pageref{54662e86f99c17a1f593cf0cd06f62ff})}

}

\small{
\textbf{Grå eminens}
\label{43d50b8900dac86ab3ad81a5dd5b8c9f}
 En \textbf{grå eminens} är en person som lever efter familjen Wallenbergs devis om att \textit{\quotetext{verka utan att synas}}; den som håller till i kulisserna och drar i trådarna. Ofta har den grå eminensen inte någon officiell makt \textsc{(s.~\pageref{7209d8106e8d1ab0fd106b96ac4a0c4c})} utan styr sin marionett med andra medel. Gemensamt för hela skrået är att man har någon form av diabolisk plan som man skrattar diaboliska skratt åt varje gång man kommer ett steg på vägen. Den första grå eminensen var kapucinmunken Joseph Le Clerc du Tremblay som var kardinal Richelieus rådgivare och arbetade för Frankrikes deltagande i trettioåriga kriget. Honom har ingen hört talas om, men om man istället tar Saruman och den där gamla alvkungen han förtrollat så förstår nog de flesta. Sveriges mäktigaste grå eminens är kungens polare Noppe Lewenhaupt. Postens frimärksserie \quotetext{Hästsport} ska till exempel ursprungligen varit Lewenhaupts idé - han ska till och med ha skrattat diaboliskt vid lanseringen.

}

\small{
\textbf{Gråmelerad T-shirt}
\label{49d0df4cc081e449116fbae9f536d9a7}
 Känner du fortfarande att du är ung och har koll på läget? Tror du att tonåringarna som morsar på dig gör det för att de tycker att du är cool som är något år äldre? I själva verket är det tio år sedan Korn hade en hit och ungarna hälsar bara som en ironisk kommentar till den där tribalen du har på armen. Hur kunde det bli så här? När tappade du greppet? Svaret stavas: den dag du började bära gråmelerad T-shirt utan tryck. Bekväm, neutral, billig. Där och då tog du steget och nu kan du aldrig återvända. Glöm inte MedMera-kortet nästa gång du ska fylla på garderoben.

}

\small{
\textbf{Gröt}
\label{9783ee2e0bcf28373d27292dafa35d44}
 uppfanns av Göran \textsc{(s.~\pageref{798906d6f87c98cb6c72c306560e30f4})} i filmen \textit{Tillsammans}. Det är iallafall vad man har trott hittills. Det finns dock färska paleoantropologiska fynd som tyder på att gröt för första gången synteserades när Husqvarnas \textsc{(se Husqvarna s.~\pageref{6671b561d336f97592b06a183ea47d3e})} första elspis damp ned på marknaden.

 Gröt används ofta som symbol för virilitet och styrka, samt som en mycket generell analogi för metabolismen. Exempelvis i meningar som \quotetext{ät upp din gröt nu, så du blir stor och stark som Mob 47-Åke \textsc{(s.~\pageref{486ee67ac39debabed3d92a7555dcebd})}}. Detta uttryck används oftast av far\&mor-föräldrar av manligt kön (dock ej postmoderna sådana \textsc{(se postmodern morförälder s.~\pageref{739c4c2e41c756708ce80adef26bf68b})}, troligtvis då deras egen kropp börjat förtvina vid barnbarnaskapet. Undersökningar har visat att 69\% av alla morfars tror att muskler består av havregrynsgröt och blodproppar orsakas av omogna lingon i lingonsylten. Farfars har i regel liknande åsikter, men skiljer sig på den punkten att det tror vithårighet orsakas av mjölköverskott istället för pigmentunderskott.

 Category:Mat \textsc{(s.~\pageref{54662e86f99c17a1f593cf0cd06f62ff})}

}

\small{
\textbf{Gubbafint}
\label{c47c78e7f558b5b25fdad7204fed9506}
 En gubbafint är en klassisk handbollsfint, lika enkel som briljant. I anfallsposition låtsas spelaren helt enkelt gå förbi försvararen på ena sidan, men byter sen ben och går på andra. Den som behärskar gubbafinten kan även använda den på fritiden. Till skillnad från kampsporter är det nämligen tillåtet att bruka handbollsrörelser utanför planen.
 . Försvararen blir förvillad och kan inte göra något.]]
 Länk till klockren gubbafint: http://www.youtube.com/watch?v=8kvaZVxAE_0 \textsc{(s.~\pageref{74c92ebadc082dc1ab40cdafe4950bd2})}

}

\small{
\textbf{Gubbrock}
\label{017518791bbb6d1db3fca1e31b678b4d}
 (av sv. 'gubbe' ung. \quotetext{äldre man} och eng. 'rock' ung. \quotetext{vaggande rörelse} el. \quotetext{klippa}) är rockmusik som spelas av gubbar och uppskattas av människor, så kallade gubbrockare, med tendenser mot gubbighet. Även om gubbrocken av naturliga skäl inte tillhör rockens mest vitala grenar är den dels bland de största inriktningar inom rocken eftersom alla rockband som består av män förr eller senare börjar spela gubbrock, dels den mest pondusfyllda rocken eftersom gubbar har spelat rock längst (bortsett från gummor, som kan ha spelat lika länge) och därför vet mest. Exempel på gubbrock är Ulf Lundell, John \quotetext{The fog \textsc{(s.~\pageref{576875ef0042ff21c04f5f1b9377d4e7})}} Fogetys soloprojekt, Roger Waters och David Gilmoures dito, Eldkvarn, AC/DC (som automatiskt blev gubbrock när den nya sångaren \textsc{(s.~\pageref{2e55dbe6a48745ced354e0dd04dd4b80})} började), allt som någonsin har spelats på Droskan i Umeå, Rolling Stones samt allt som Chips Kiesbye någonsin tagit i. Motörhead är inte, och kommer aldrig att bli, gubbrock. Motörhead är arbetarklassrock \textsc{(s.~\pageref{3268200a49db708660491e54e53c05c3})}.

}

\small{
\textbf{Gubbsova}
\label{6e106ba0fa4c271bb5d5c8fdf63694cd}
 Att gubbsova är att ta en liten tupplur, utan att för den skull sluta med den aktivitet man håller på med, t.ex. läsning eller datoranvändande. Sovställningen är nästan alltid oergonomisk då den intas i stundens hetta och inte efter nogrann planering. Anledningen till att det heter just \quotetext{gubbsova} är att det är populärt bland gamla människor, främst av hankön, s.k. \quotetext{gubbar}.

}

\small{
\textbf{Gubbsäker}
\label{e6cb916b91ceed5550ee4204e7b6c902}
 Att vara gubbsäker innebär att man kanaliserar själva essensen av att vara gubbe - en luttrad ovilja att bli påverkad av omvärlden och en utopisk önskan om att existera som en ö, avskiljd från samhället. Det kan handla om att bestämt förneka sanningshalten i vetenskapliga rön om att det är ohälsosamt att äta två paket basonfläsk och dricka fem old ox \textsc{(s.~\pageref{954ddcc10ad941a7ee93e0584ee6a78b})} om dagen, att vägra sluta säga negerboll \textsc{(se alternativa namn på bakverk s.~\pageref{95983a7c39a10b94946c312761bf3db6})} eller att bestämma sig för att tatuera Betty Boop på axeln vid 55-års ålder - till synes bara för att jävlas. Sen går man och snickrar \textsc{(se gå och snickra s.~\pageref{daafc4f3b248988c8f764ce014029905})}.

 I diskussioner tar sig gubbsäkerhet uttryck genom ett tvångsmässigt behov att agera djävulens advokat. Först lite för att retas, men sen när man inser att verkligen ingen tycker som en själv, stå fast vid korkade åsikter som att hålla med folkpartiet \textsc{(s.~\pageref{b692fa6a23fd557940474dc94909d80f})}, hylla abortmotståndare, eller hävda att Metallica gjorde sig bäst på Load och Reload, till det bittra slutet. För att kontra gubbsäkerhet är det säkraste kortet att lägga ner diskussionen. Detta brukar åtföljas av en ursäkt en eller två dagar efter att diskussionen tagit plats. Kanske hade till exempel Olof Rudbeckius d.ä. då kunnat säga till sin vördnadsvärde gäst Jean-François Regnard: \textit{\quotetext{- Eh, det där jag sa om att det mytomspunna öriket Atlantis faktiskt är det samma som Sverige \textsc{(s.~\pageref{b1999637949ed135b2ca03f3a38460cc})}... Alltså fan, jag hade druckit tio old ox och var bara gubbsäker. Atlantis är nog bara en saga ändå, och även om det fanns var det inte Sverige.}}

 Men dessa ord yttrades aldrig, vilket resulterade i ett av de mest passivt-aggressiva vänskapsförhållanden världen någonsin skådat. Men det är en annan historia.

}

\small{
\textbf{Gud}
\label{91e49146121c992aab11a19c77e26cf0}
 är redaktör till den bästsäljande antologin Bibeln \textsc{(s.~\pageref{7de7d2a7d608c9a2044f50688bc63e27})}. Boken har skapat kontrovers då Gud själv författade flera kapitel i den tillsammans med spökskrivare och det anses svårt att utröna vilka kapitel Gud ursprungligen avsåg att ta med, flera reviderade utgåvor förekommer. Debatten har försvårats av att Gud väldigt sällan ger intervju.

 HEAD2: Andra bedrifter
 Envisa rykten säger att Gud ska ha uppfunnit den moderna sjudagarsveckan.

 HEAD2:  Familj

 Gud är enligt de flesta pappa till Jesus \textsc{(s.~\pageref{110d46fcd978c24f306cd7fa23464d73})}, men vissa påstår att han har avsevärt fler barn än så.


 HEAD2:  Populärkulturella referenser

 Gud nämns vid namn i följande sånger:
 \begin{itemize}
 \item Ebba Grön - Häng gud
 \item Mob 47 - Religion är hjärntvätt
 \item Nick Cave \& the Bad Seeds - God is in the house
 \item David Sandström Overdrive - The god thing
 \item Eyehategod - My Name is God (I Hate You)
 \item Hillsong - God is Great
 \item Swans - Children of God
 \item Om - Unitive Knowledge of the Godhead
 \item Slayer  - Disciple
 \end{itemize}

 HEAD2:  Övrigt
 Ej att förväxla med Mob 47-Åke \textsc{(s.~\pageref{486ee67ac39debabed3d92a7555dcebd})}

}

\small{
\textbf{Gulsvansad ullapa}
\label{c9022fba896b3e8a41420242680d2480}
 Den gulsvansade ullapan (\textit{Cornelis ludopus}) är ett djur i grenen primater på djurrikets brokiga släktträd. Den gulsvansade ullapan finner den som har både tur och skicklighet i Braziliens djupa djungel. Forskare uppskattar att jordens totala bestånd av gulsvansade ullapor uppgår till mindre än tvåhundra djur. Anledningen till detta var länge en gåta för den internationella forskarvärlden. Två av de få akademiska experter som ägnade sig åt studiet av den gåtfulla apan, Dr. Martin Zimmerstein och Professor Helena Goodman, uttalade sig i en akademisk artikel publicerad i \textit{Nature} 1983 om att anledningen till apans alarmerande låga födelsetal och dito höga dödstal med stor säkerhet var att apans naturliga habitat blivit rubbat av den för Brazilien ekonomiskt viktiga skogsindustrin. Ett forskarteam vid Vancouver University i Kanada motbevisade dock att detta var huvudanledningen till artens bekymmer. Istället, visade man i forskning som väl motsvarade de allra strängaste akademiska krav på vetenskaplighet, var artens största nackdel och genetiskt nedförda akilleshäl att den helt enkelt var dum något så in i Norden. Det tyngsta beviset som man lade fram i artikeln (\quotetext{The Genetic Key to the Mortality Rate of\textit{ Cornelis Ludopus}}, \textit{Applied Biogenetic Science} 49 (1989) sid 34-56) var resultaten av en närstudie av en isolerad grupp individer, den enda som levde söder om Amazonasfloden. Den visade nämligen att gruppen, på grund av att de jagat individuellt efter att dess ledardjur dött och inte ersatts av ett annat, glömt bort vad deras huvudsakliga bytesdjur var liksom dess naturliga metod för att finna och döda det. På grund av detta svalt alla djuren i gruppen ihjäl eller dog av och svältrelaterade olyckor, så som att ramla ner från träd av utmattning. Vad som dock inte hjälper arten i dess förvirrade existens är att den inte har förstånd nog att rädas människor. I stamkulturerna i området kring den riegion där de största apgrupperna normalt håller till finns många berättelser om hur gulsvansade ullapor kommit intraskande mitt i människobyn. Detta har troligen skett med sådan frekvens för några hundratals år sedan att folket i en av stammarna i området talar om sig själva som \quotetext{folket som besöks av den efterblivna apan.} Apans dumhet har blivit så omtalad i forskarvärlden att den har använts av kreationistiskt inriktade forskare som ett bevis mot evolutionsläran. Man påstod att den gulsvansade ullapans så nära släktskap med andra primater, som uppvisar komplicerade sociala system och hög problemlösningsförmåga, enligt den evolutionära utvecklingsteorin omöjligt kan kombineras med sådan extremt låg intelligens. Enligt teorin om naturligt urval borde apans intelligens ha utvecklats i samma takt som apan genetiskt förändrats från dess att den skiljt sig biologiskt från andra idag levande primater. Denna teori har dock motbevisats i en rad artiklar av framstående genetiker.

 HEAD2: Föda och beteenden
 Apans naturliga föda består till största delen av fisk. I viss utsträckning äter den också mossa. För att döda fisken rusar apan omkring i grunt vatten och slänger sig efter fiskar. När den fångat en slänger den fisken långt in i skogen och går sedan och hämtar den. På tiden det tar för apan att hitta igen fisken hinner fisken dö av syrebrist. När fisken väl är död suger apan på den så länge att bara slamsor återstår. Dessa lägger apan av okänd anledning i en äcklig liten hög centralt i den yta där flocken lever.

 HEAD2: Gulsvansade ullapan i mytologin
 I flera av de stammar som lever i den Gulsvansade Ullapans närhet finns versioner av myten om de två bröderna Gazam och Korooko. Denna myt berättar hur en man vid namn Hangah hade två söner varav en, Gazam, av avund för sin broder Korookos nära förhållande till fadern slog sin
 bror i huvudet med en sten. Korooko blev av detta så dum att han inte kunde reda sig själv och blev ett ok på sin faders axlar. Fadern förargades eftersom han trodde att dumheten kom sig av att Korooko åt av den förbjudna svampen. Han förskjöt nu sin son, som gick att leva djupt inne bland träden, som aporna. Med tiden blev han som sina vänner aporna luden och en svans växte ut på honom. Han förökade sig med aphonor och blev fader åt den Gulsvansade Ullapan. Apan omtalas också i uttryck som används i språkområdet, så som \quotetext{att lura en ullapa,} vilket syftar till en enkel men osympatisk bedrift (jmf. \quotetext{att stjäla godis från ett barn}).

 HEAD2: Den Gulsvansade Ullapan i populärkulturen
 Sludge/stoner/industri/grunge-bandet \textsc{(se Sludge s.~\pageref{2ccd23d1cd0f95dc6984215a1f1b31ca})} Melvins' album \textit{(A) Senile Animal} kretsade kring den gåtfulla gulsvansade ullapan.

}

\small{
\textbf{Gunborg}
\label{9e29dc34382963ae7d76a742e98637a4}
 , förstår den som har lyssnat ordentligt på engelskalektionen, betyder \quotetext{kulsprutenäste} och är ett vanligt namn hos flickebarn till ivriga hemvärnsfrivilliga. Det vill säga till reaktionära bönder \textsc{(s.~\pageref{30a6fc00c9102680b8196b1b79935ec4})}.

 {{Utmärkt}}

}

\small{
\textbf{Gurka}
\label{1cf02b8eacd57c92e9df0a1a3eaa8946}
 En Gurka är en grönsak som faktiskt är grön. Den består mest av vatten och smakar inte så mycket. EU \textsc{(s.~\pageref{4829322d03d1606fb09ae9af59a271d3})} har lagt ner ofantliga resurser på att reglera gurkodlandet i Europa.

}

\small{
\textbf{Gurkmajonnäs}
\label{67accedc0d647557edea0dc0f54fe3be}
 Efter att Lyotard på 1970-talet skapat räksallad \textsc{(s.~\pageref{b7e642c04871e612342088c068b7bd65})} och hävdat postmodernismens intåg, kom på 1990-talet en replik från den filosofiska underdog-nationen Storbritannien. Det var socialdemokratins frälsare/Beelzebub Anthony Giddens som gick i polemik med Lyotards idéer i och med skapandet av gurkmajonnäsen, i boken \textit{Modernitetens följder}.

 Giddens menar som bekant att vi idag lever i en högmodern samtid, vilken bär drag av postmodernism men som inte alls är där Lyotard menar att den är (de stora narrativens död etc). Därmed var räksalladen, postmodernismen fulländad, något som inte kunde stå obesvarat. Giddens, alltid villig att kompromissa, skapade således gurkmajonnäsen. Den är på ett sätt (modernistiskt) ärlig med sitt innehåll, då majonnäs skrivs fram tydligt i namnet. Men den innehåller också en (postmodern) lexikal orimlighet, då \quotetext{gurk} kommer före \quotetext{majonnäs} trots att gurka inte är huvudingrediensen. Således bär gurkmajonnäsen spår av både modernitet och postmodernitet, utan att direkt tillhöra någondera.

}

\small{
\textbf{Gurkvatten}
\label{9df50ad83b16d973a84fa87a5a3c4cbc}
 I årtionden har gurkvatten varit en bestående del i det svenska cuisinet. Till vissas glädje. Till andras förtret. Här följer ett recept på gurkvatten. Skiva ca 1/3 gurka av EU-standard storlek i lagom tunna skivor. Stoppa ner gurkskivorna i en karaff. Häll på iskallt vatten. Sen är det klart. Och meningslöst.

}

\small{
\textbf{Gustav Vasa}
\label{aeb7f10919b25762e3d031a0b583a2e8}
 Enkel och praktisk frisyr. Låt håret växa så att det värmer öron och nacke. Klipp ett rakt snitt strax ovanför axlarna så att håret inte flottar ner din jacka (du tycker schampo är en myt). Klipp något kortare på framsidan så att du har fri sikt. Klart. Gjort på mindre än fem minuter och kräver inget gesällbrev för att få till.

}

\small{
\textbf{Gylfa}
\label{28165e284c81217ee310f87818e01161}
 Att gylfa är att snatta med hjälp av gylfen. Förövaren stoppar byxorna i sockarna och stoppar sedan in prylar i gylfen. Denna metod fungerar skitdåligt med Cheap Monday-jeans och boxershorts.

}

\small{
\textbf{Gå och köpa tidningen}
\label{86fb6ed06892c06b596fe55aa3468d7f}
 Att gå och \quotetext{köpa tidningen} är något som svenska män ofta gör på självaste julafton. Barnen tror att fadern gör just detta, medans frugan tror att de ska gå ut i garaget, ta på en tomtemask och komma tillbaka med julklappar. I själva verket går de över till sin andra familj för att fira julen där. Handlingen upprepas där för att gå tillbaka till ursprungsfamiljen, och sådär håller det på.

}

\small{
\textbf{Gå och snickra}
\label{daafc4f3b248988c8f764ce014029905}
 Kod för att gå ut i snickarboa och dricka ur en kvarting som man har gömd i ett kikarfodral. Populärt under släktmiddagar och andra krävande situationer.

}

\small{
\textbf{Göra rätt för sig}
\label{c8c01e0e8b4ad8e5ff6011b8af6405a5}
 Att göra rätt för sig är att alltid göra saker som man själv inte har någon glädje av, och att undvika sådant som att studera ämnen som kan anses vara oviktigta men intressanta. Bland saker man inte ska studera ingår alla humanistiska ämnen. Man måste tjäna sitt levebröd på sådant som inte är roligt. Alltså kan man inte samtidigt göra rätt för sig och arbeta som mångsysslarpensionär \textsc{(s.~\pageref{ce651324111b616e98f210ea8511ce75})}, barberare \textsc{(s.~\pageref{f5683c848bc74039df5513e734636898})} och så vidare. Det säkraste sättet för den som vill göra rätt för sig är att ta en sådan anställning genom vilken man kan bli medlem av ett fackförbund anslutet till Landsorganisationen, LO. Är man det är det i princip omöjligt att inte göra rätt för sig och då är man också värd päronhalva \textsc{(s.~\pageref{cc9c1bfa2ec4eaed89ca86a1b63e3a45})} till efterrätt. Ett bra steg på vägen mot att göra rätt för sig är att ta för vana att alltid använda handjagare \textsc{(s.~\pageref{a85b0d5ff42c09f4605cec0188b8dd6e})}.

 HEAD2: Att göra rätt för sig i litteraturen

 \textit{Storstrejken? Ingen har några minnen av något speciellt vad gäller Josefina Markström. Hon ogillade den väl, naturligtvis. Hon ansåg att alla skulle göra rätt för sig. Jo, ett kritiskt yttrande från henne om Bolagsledningen: He \textsc{(s.~\pageref{6f96cfdfe5ccc627cadf24b41725caa4})} var int rätt. Vad var inte rätt? Att sänka löningen med tjugofem procent. Då gjorde inte arbetsgivaren rätt för sig. Men en skall så, en annan skall skörda, tillade hon gåtfullt}

 - Musikanternas uttåg av P.O Enquist

}

\small{
\textbf{Göran}
\label{798906d6f87c98cb6c72c306560e30f4}
 är ett klassiskt svenskt namn som betyder \quotetext{lägga till båten}.

}

\small{
\textbf{Göras till åtlöje inför hela svenska folket}
\label{3ed7e29e1fe8e143fdb81bf836742e4b}
 Att göras till åtlöje inför hela svenska folket är det faktiskt inte många som varit med om, och som med allt annat som är exklusivt ska naturligtvis kungahuset ta för sig \textsc{(s.~\pageref{da38d3921d90c6551623165ebb693bb3})} även av detta. Kungen har till och med skapat en tradition av att på nyårsafton göra bort sig inför kanske inte precis alla svenskar, men en mycket stor del av dem i alla fall.
 HEAD2: Folk som gjorts till åtlöje inför hela svenska folket
 Trots det exklusiva elementet i denna aktivitet finns det ett och annat driftkucku på denna lista, som omfattar:
 \begin{itemize}
 \item Fotbollslandslaget (herrar)
 \item Hon som spydde i direktsändning på ZTV.
 \item Leila K
 \item Junilistan
 \end{itemize}



 Skall ej förväxlas med Rikspucko även om likheterna är många. Ett rikspucko kan medvetet offra sin själ för att få uppmärksamhet. Här återfinner vi karaktärer som Robinson-robban, Alex Schulman m.fl.

}

\small{
\textbf{Gösta}
\label{f5cd6403e7588f83f36e163936962a6e}
 betyder tjuvfiskare. Det var länge ett rent nidingsnamn men nu kan man ju heta lite vad som helst.

}

\small{
\textbf{Gösta "Snoddas" Nordgren}
\label{5cb1aa19b3f60a517978ebea69456dcf}
 Gösta \quotetext{Snoddas} Nordgren (1926-1981) var en framgångsrik bandyspelare som bland annat lirade med Bollnäs GIF och det svenska landslaget. Han var också Sveriges \textsc{(se Sverige s.~\pageref{b1999637949ed135b2ca03f3a38460cc})} första, och mest framgångsrika shockrockare \textsc{(s.~\pageref{0a5df81af5b35b7d9b48b9ab9e39b802})}, spred kaos och orsakade hjärtstillestånd i folkparkerna med sin syndiga och djävulsdyrkande refräng \quotetext{haderian hadera} i låten \textit{Flottarkärlek}. \quotetext{Snoddas} ska enligt sägnen ha segnat ner död under en innebandymatch han spelade med handikappade ungdomar på ett sjukhus i Vänersborg. Passande, på nåt vis.

}

\small{
\textbf{Göteborg}
\label{0e9b11e435dd9f73e87e868667e1d6f0}
 , också känt som \quotetext{Lilla London}, är rikets \textsc{(se Sverige s.~\pageref{b1999637949ed135b2ca03f3a38460cc})} andra stad. Här uppfanns den moderna ordvitsen och klasshatet.

 När det regnar och blåser väldigt mycket, då vet man att man är i Göteborg.

}

\small{
\textbf{Götebosska}
\label{20551bfa7f3d49ba95d5a63e19fcabdc}
 talas företrädelsevis av personer med anknytning till staden Göteborg \textsc{(s.~\pageref{0e9b11e435dd9f73e87e868667e1d6f0})}. Många har fått för sig att göteborgare är supertrevliga, och säger roliga saker, men det går så klart utmärkt att även vara dryg och otrevlig \textbf{Götebosska}.


 {\textbar class=\quotetext{wikitable}
 \textbar+ \textbf{Liten Parlör}
 ! Götebosska \textbar\textbar Svenska \textbar\textbar Förklaring \&nbsp;
 \textbar-
 \textbar E du go e hövvet eller? \textbar\textbar Är du helt jävla dum i huvudet \textbar\textbar \textit{förolämpning, ingen uppriktig undran över någons mentala kapacitet}
 \textbar-
 \textbar Flô daj din flane! \textbar\textbar Flytta på dig, fåntratt \textbar\textbar \textit{skämtsamt påpekande att någon är i vägen}
 \textbar-
 \textbar Skumma lô! \textbar\textbar Där fick du fikon! \textbar\textbar \textit{påstående att någon kan leta igenom könshåren, men ändå inte kommer hitta något}
 \textbar-
 \textbar Ha du bosstat bissarna? \textbar\textbar Har du borstat tänderna? \textbar\textbar \textit{barn har bissar, vuxna har gaddar}
 \textbar-
 \textbar Denna låten var gôr-bra! \textbar\textbar Det här var ett vackert musikstycke! \textbar\textbar \textit{'gôr' är ett vanligt kraftuttryck}
 \textbar-
 \textbar Ge maj en öl! Bördiga. \textbar\textbar Är du snäll och serverar mig en öl? \textbar\textbar 'bördig innebär givmild och god, som jorden''
 \textbar-
 \textbar Hon är ena lite tetig \textbar\textbar Hon har inte alla hästar hemma \textsc{(se alla x i y s.~\pageref{18d4689248d1c32d716dab95e7e57b17})} \textbar\textbar \textit{'tetig innerbär knepig eller konstig}
 \textbar}

}

\small{
\textbf{H\&Ms "rockiga" tjejkläder}
\label{f50a15920ea21e41e82b93a6e876ad6f}


}

\small{
\textbf{H.K.H Prins Ockelbocken}
\label{ffcde735ec52bf311bbbea8cd5ff867a}
 Vid midsommar i nådens år 2010 ändrades namnet Daniel Ockelbocken i folkbokföringsregistret till Daniel Westling Bernadotte. Trots vilda protester från Ockelborna tog hertigen av Västergötland mellannamnet Westling istället för Ockelbocken.

 Bemärk den förvånansvärt folkliga hammaren inom hjärtskölden på Daniels rykande färska vapensköld, den förklaras av hovet som en referens till Ockelbockens hemkommuns vapen. Eller så är den nya generationen monarker smygstrasserister.

}

\small{
\textbf{Ha bärs}
\label{a74b297c15834437ac2e49095492133c}
 Att \textbf{ha bärs} är det enklaste sättet att försäkra sig om en trivsam kväll/eftermiddag/älgjakt/skiftlagsfest/etc. På \textit{Systembolaget} kan man köpa nästan hur många bärs man vill, men ändå händer det ibland att man har slut när man vill ha en. Då måste man skaffa bärs på annat sätt och det lättaste är att fråga sig runt. Det låter ungefär så här:

 -Öhh, ha... haru bärs?

 Eller:

 -Haru bärs!?!! Ja får en va? ...du! ...amen du!


 HEAD2: Kända personer som har eller har haft mycket bärs
 Jens Spendrup
 Torsten Flink
 Homer Simpson



 Källa: Prof. Etienne \textsc{(se Användare: Prof. Etienne s.~\pageref{a9878d2280e5a39becac8f73d113df91})} - \textit{Hundra sätt att få ligga}. Allers förlag, Kalmar 2005.

}

\small{
\textbf{Habbadixen!}
\label{d04e4f3fe6f51cb772bff485609e96a5}
 Habbadixen är en interjektion som används vid bevittnande av sport på TV för att få en idrottsutövare att misslyckas med sitt företagande. Det uppfanns mig veterligen av min \textsc{(se Användare: HratvinnFlygur s.~\pageref{26c5d96dca8dfce84752fa1d4095fdb0})} morfar, Anders Selldén, under en tennismatch då trollformeln ledde till flera dubbelfel. Det har sedan dess med varierande grad av funktion använts vid olika idrotter, exempelvis skidskytte och dart.

 Exempel:
 \quotetext{Nu ska jag habbadixa Federer så att han gör ett dubbelfel; HABBADIXEN!}

}

\small{
\textbf{Hacka}
\label{40e8f50402b1aa765fa099581b7bd4bc}
 En hacka är ett spetsigt redskap som används för att gröpa ett hål i något; till exempel en bergvägg, en sjöis eller en fiende i kampen om makten i kommunistpartiet. Liksom många andra av människan uppfunna redskap kopierar hackan en teknik som redan finns utvecklad i djurriket. Hackspetten har, ända sedan den levde på havsbottnen, använt sin näbb till att picka hål i saker och äta upp vad som finns däri. Den första prototypen till den mänskliga hackan togs fram på medeltiden \textsc{(s.~\pageref{88cbc30c5b233d97df68b5b041ac0655})} och bestod av en vass sten som stoppades in i munnen. Den fungerade inget vidare. Det verkliga genomslaget kom först i och med uppfinnandet av träskaftet \textsc{(se träskaft s.~\pageref{1ab85ecd859ae682af47bb9334c7dac6})}. I senare tid kan hacka anspela på en icke ansenlig summa pengar, ibland förtjänad utom riksskatteverkets kontroll.

}

\small{
\textbf{Haddaway}
\label{8b9e898ec4637bc837bfd9302ab0813d}
 är en tysk eurodisco-artist, mest känd för låten \textit{What is love}. Efter genombrottet med debutsingeln har Haddaway haft svårt att upprepa braksuccén \textsc{(se braksuccé s.~\pageref{678371d35369d3d29afceb1445630833})}. Närmast var kan kanske 2008 då han i ett samarbete med Dr. Alban \textsc{(s.~\pageref{9756163bb9005234a901bcd148a44700})} gav ut \textit{I Love The 90s}. Trots detta har han ändå släppt två greatest hits-samlingar. Haddaway uppträdde i Umeå i maj 2010.

}

\small{
\textbf{Haiku}
\label{7e568fa3b73f720b15c2ed416e8b9db4}
 är en versform som är populär inom japansk poesi, men som har sitt ursprung i nordkinesisk poesi från tang-dynastins dagar. Haikun består av tre rader med två olika versmeter, den ena (rad ett och tre) på fem stavelser och den andra (rad två) på sju. Haikun är normalt avdelad i två led - rad ett och två bildar ett led och rad tre ett. Ofta byggs en viss stämning eller scen upp i den första raden, medan den andra introducerar ett nytt element i dikten, ofta som en oväntad förändring som på ett omedelbart och till och med dramatiskt vis kastar om den bild som läsaren har skapat i sitt inre. I någon av leden förekommer traditionellt en så kallad \textit{kigo}, det vill säga ett ord som på ett mer eller mindre subtilt vis situationerar dikten i en av årstiderna. I modern Haiku förkommer dock dikter utan kigo minst lika ofta som dikter med. Kigon kan vara en direkt referens till året, så som en månad, helgdag, festival eller annan speciell dag. Den kan också vara den mer subtil referens, så som närvaron av en viss, säsongsbunden växt (så som körsbärsblom), en referens till färgen av ett löv, immande andedräkt, närvaron av eller sången hos en säsongsbunden fågel och så vidare. Den medeltida Japanska diktaren Basho framhålls generellt av poesihistoriker och Haiku-älskare som den mest genialiske diktaren i haikuns långa historia.
 HEAD2: Exempel på Haikudikter med kigo

 Farsan, i gummistövlar,
 står på gårdsbacken och svär -
 packad och förkyld

 Isblandat störtregn
 har gjort jumpapåsen blöt -
 Friluftsdag i mars

 Ingen vill komma
 på min födelsedagsfest -
 För då är det påsk \textsc{(s.~\pageref{f8f0dd13b69a5c8ce56498e750551d3e})}

 Det luktar fiskrens
 ur gubbens mörka munhål-
 Svanarna flyttar


 HEAD2: Exempel på Haikudikter utan kigo
 Farsan och farfar
 drämmer näven i bordet -
 ryter åt farmor

 Brevbärarens fru
 Blir kanske inte så glad nu -
 För han ser död ut

 Halkar på död fisk
 och slår huvet i stenen -
 Farmor skrattar glatt

 Jag är förlamad
 ända från midjan och ner
 men ingen bryr sig.

}

\small{
\textbf{Haile Selassie}
\label{9accb6bc1893f934737ecb7710da49a4}
 bör främst kommas ihåg för sin långa titel:

 \quotetext{Hans kejserliga majestät, kejsare Haile Selassie I, Lejonet av Juda, Guds utvalde, konungarnas konung av Etiopien.}
 Han föddes med namnet Tafari Makonnen i Finska Karelen.

 Selassies flygplan sägs också ha funnits vid flygplatsen i Kärrgruvan men är i dagsläget nerbrunnet. Hans gravplats är numer en allmän toalett \textsc{(se ornässtugans dass s.~\pageref{17d2effff6c1590dbff6a7ac39f46a19})}.

}

\small{
\textbf{Haj}
\label{00ed0f1d6fa0f51775d9fd969adb4e3b}
 Gay delfin \textsc{(s.~\pageref{a62b1fca6b53d6670a84aa2c7b373b27})}

}

\small{
\textbf{Haka (vanlig)}
\label{3b8edf3dc8968e6b2805dc512af3b68c}
 En vanlig haka är en kroppsdel som är lite spetsig eller lätt rundad och återfinns allra längst ner och ut på ansiktet. Dess funktion är inte klart definierad, men på män kan den fungera som fästpunkt för ett pipskägg. Ovanför hakan finns munnen.

 Se även arselhaka \textsc{(s.~\pageref{9e4937897b8431412a9d8eb7561f5ec6})}

}

\small{
\textbf{Hakkors vi minns}
\label{93d98c61d8dac9b79008998fda15a363}
 Svastikor, men inte längre.

 De landmärken som hjälpt vagabonder på vägarna att hitta skydd under en dragig ladas tak en regnig sommarnatt. Liksom forna tiders luffarristningar. På en klippväg, eller var det en stor sten, minnet sviker, där stod den. Som en Bergslagens Tanums hede, sin mening höljd i dunkel: svastikan följd av texten \quotetext{SAAB}. Som så mycket annat gick just denna svastika ett grymt öde tillmöte - den mötesfria vägen - nollvisionen. Kommer människan någonsin igen kunna orientera sig mellan Norberg och Avesta?

}

\small{
\textbf{Hallonmobilen}
\label{42fc08eaaee2a76c23dd460dd547ab3e}
 (Oktober 2010 - Juni 2011) var en röd Volvo 740 \textsc{(s.~\pageref{e262951543da05bac43c7b87235a115c})} som bodde på Berghem i Umeå och var själva uppslagsverksdefinitonen av ett vrålåk. Bilen skrotades efter påtryckningar från samhället.

}

\small{
\textbf{Hallonvägens annaler}
\label{c34acf9a8058947d2f24d5b2d63ad639}
 var en av de framträdande bloggarna då bloggsfären växte explosionsartat under 2000-talets första årtionde. Redan den finurliga ordleken i bloggens titel introducerade den nyfikne läsaren för vad som komma skulle. Bloggen dokumenterade det intrikata samspelet mellan fyra unga män som bodde i ett hus i stadsdelen Berghem \textsc{(s.~\pageref{a6b1df39fa9b1b94dc92200594a8ccd6})} i Umeå. Här fick besökaren stifta bekantskap med huvudpersonerna, men också se små filmer i vilka deras vänner spelade luftgitarr \textsc{(s.~\pageref{0e2415e86edc316f5338964c6ef145b5})} på fyllan. Man provade också på andra välkända mediaformat, så som matlagningsprogram, i vilket en överförfriskad ung man sågs stå sans pants \textsc{(s.~\pageref{e690d08a3200d783d98b198f0354bc85})} och steka falukorv i köket.
 Bloggen blev föremål för den utbredda bloggdöd som härjade medialandskapet i slutet av 00-decenniet och somnade in under 2009, men ligger nu på \textit{lit de parade} för allmän beskådan här: [http://hallonvagentva.blogspot.com/].

}

\small{
\textbf{Halvfranskt band}
\label{6eb03d7bf5ff4e2689edb127e329ae26}
 Ett band där 50\% av medlemmarna heter Yvette eller Jean-Pierre.

}

\small{
\textbf{Halvtrött}
\label{07a5c50b32349c286c73a9ef44eec914}
 När en människa presterat halva sin fysiska förmåga kan denne påkalla det för andra genom att säga att de är halvtrötta. De kan prestera lika mycket en gång till för att sedan drastiskt måsta vila.

}

\small{
\textbf{Ham}
\label{79af0c177db2ee64b7301af6e1d53634}
 , född juli 1959 i Kamerun, död  19 januari 1983 på North Carolina zoo, var den första hominida apan i rymden. Två år gammal flög Ham till rymden i 16 minuter och 39 sekunder. Bedriften kan te sig ganska klen om man exempelvis jämför med Belka \textsc{(s.~\pageref{9f148033690e82549848dea862a1a9ee})}, fast det här var ju å andra sidan USA. Ham blev stor stjärna när han återvänt och spelade bland annat in film med Evel Knievel.

}

\small{
\textbf{Hamburgerkedjepersonalspersoner}
\label{95a22fdc49f91291ae743f1372bca323}
 (HKPP) är lätt hjärntvättade strebers, misfits eller bara allmänt otursförföljda mänskor. Har de varit anställda mer än tio minuter har de även begynnande symtom på skadlig stress. (För undantag se Frasses \textsc{(s.~\pageref{971e198d8fef127906319ec98ff657ce})}.) Tack vare den mänskliga självbevarelsedriften finns det många fler f.d. HKPP än aktiva, men de flesta får dras med livslånga men. Att ha en bakgrund som HKPP berättigar i sig inte till arbetsskadeersättning, men försäkringskassan brukar se mellan fingrarna och bevilja det ändå*.

 Större hamburgerkedjor får konspirationsteorikompatibla royalties från läkemedelsindustrin på försäljningen av blodtrycksmediciner (kunderna) och antidepressiva (personalen). Hamburgerkedjorna har via bulvaner börjat köpa upp läkemedelsindustrin i syfte att finslipa synergieffekterna. Ingvar Kamprad \textsc{(s.~\pageref{c5f2e9ee9a39f83c39079dbcf01d8809})} är med på ett hörn. Carl Bildt \textsc{(s.~\pageref{c612f1a697a7e5e1cadad85de394cea8})} ville, men fick inte.

 (* För den hemliga motståndsrörelsen inom Försäkringskassan se www.fkhandlaggaremotomansklighet.org. Efter dödsdomarna mot två öppet empatiska handläggare i Tumba byter hemsidan server ofta, använder bara krypterad kommunikation, och har ett visst samarbete med Flashback och Pirate Bay.)

}

\small{
\textbf{Hamlet}
\label{ea3596139530b2abe7089082ab57ecbd}
 är en pjäs av författaren och flintskallen William Shakespeare, Storbritanniens svar på Astrid Lindgren.

 HEAD2: Synopsis
 Hamlet är prins av Danmark \textsc{(s.~\pageref{5331d7fd27772396f412a5b6d19bad44})}, det är medeltid \textsc{(se medeltiden s.~\pageref{88cbc30c5b233d97df68b5b041ac0655})}. Han tycker inte att tillvaron är sådär jättesoft. Hamlets pappa har nyss dött och återvänder som spöke för att berätta för Hamlet att han blivit mördad av sin bror Claudius. Med det namnet kan man tycka att prins Hamlet borde ha listat ut att en man som heter Claudius är en skurk, kan man tycka, men tydligen ska hans farsa behöva återvända från andra sidan graven för att upplysa honom om detta. Ungdomar alltså, nåväl. Claudius har alltså dödat sin bror kungen och gift sig med Hamlets morsa Gertrude för att således bli kung själv. Förr i tiden krävdes det inte mer för att få den yttresta maktpositionen, jämte Gud fader, i ett land. Sjukt. Hamlet svär på sin fars skalle att hämnas denne och döda Claudius. Det hela blir dock rörigt tack vare Polonius, Claudius rådgivare, och far till Ofelia som Hamlet kärat ner sig i. Polonius försöker få reda på vad Hamlet har i kikaren genom att gömma sig bakom en gardin och tjyvlyssna när han och Ofelia pratar. Hamlet får för sig att det är Claudius skor som sticker fram där under gardinen och sticker ner honom med sin värja. Polonius dör, Ofelia och hennes bror Laertes svär att hämnas. Man skulle kunna bli filosofisk och kalla det en metahämnd, men det är inte säkert att det är så. Claudius övertalar Laertes att han och Hamlet ska duellera, och genom att förgifta Laertes värja ska Polonius jämna ut oddsen. Under duellen utbryter ett sånt där rökmoln som när de slåss i tecknade filmer och Hamlet lyckas på något vis sticka Laertes med hans egen förgiftade värja. Med sina sista andetag berättar Laertes om Claudius plan, varpå Hamlet dödar Claudius, bara så där. Sen dör Hamlet också av nån anledning. Sedan är pjäsen slut. Nej, justja, Gertrude dör också mitt i allt under oklara omständigheter. Så. Nu är det slut.

}

\small{
\textbf{Hammarby IF bandy}
\label{2f6b1282edcfc4b164f0f529b8e50d43}
 är det enda sportlag i världen som vinner även när det förlorar. Idag, till exempel, blev det 6-4 borta mot Haparanda-Torneo, men det visade sig att Bajen vann ändå.

 Se även: kubb \textsc{(s.~\pageref{de7f6954ec8c6e346b8ba18ae018d334})}, Korv i smörpapper \textsc{(s.~\pageref{401e9eb6cef7fa42d543ef85f5925021})}

}

\small{
\textbf{Handikappkod}
\label{308e9b8f35205f5d3a2347b469ad4fc2}
 på arbetsförmedlingen är strävans mål. Som att bli adlad eller föräras en doktorshatt ungefär.

}

\small{
\textbf{Handjagare}
\label{a85b0d5ff42c09f4605cec0188b8dd6e}
 är sådana verktyg och redskap som endast drivs av handkraft. Är man en riktig, redig karl/quinna och ingen Stockholmsborgare (numera \quotetext{Stockholmsprofil \textsc{(s.~\pageref{daaee4666c210c7a40537c2399f01556})}}) använder man handjagare, så är det bara. Medan skruvdragare och elhyvlar inte vill starta bara för att man glömt att ladda batteriet, som man också tappat bort, sviker handjagaren en inte förrän den gått sönder och då är det bara för att någon annan har använt den på fel vis eller låtit den ligga ute och bli rostig. När man använder en handjagare \quotetext{handjagar} man och detta leder ofta till att man får indianmuskler \textsc{(s.~\pageref{a0e24bd0dfe9431f72896e16614e79c0})}.

 Husqvarna \textsc{(s.~\pageref{6671b561d336f97592b06a183ea47d3e})} fick mycket kritik då de inrättade särskilda läger dit handjagande maskiner forslades för att användas tills de gick sönder. I efterhand har det visat sig att deras plan var att deras motordrivna diton sedan skulle regera över hushåll och trädgårdar världen över. Vissa handjagarredskap monterades t.om. medvetet ned, för att delarna sedan skulle användas som komponenter till Husqvarnas egna maskiner.

 Exempel på handjagare är spade \textsc{(s.~\pageref{ab1991b4286f7e79720fe0d4011789c8})}, yxa \textsc{(s.~\pageref{bd74f429522c7c1481fbba07187efc6b})}, fogsvans, hammare, skruvmejsel och hacka \textsc{(s.~\pageref{40e8f50402b1aa765fa099581b7bd4bc})}.

}

\small{
\textbf{Handskfack}
\label{651fd6de0e0851f657a1c9b76b76f692}
 et är ett utrymme i bilar som fungerar precis som bakluckan, fast inne i kupén och i mindre skala. Där förvarar man kartor över Gotland, skruvar som lossnat från interiören, ett avmagnetiserat kassettband med Mikis Theodorakis \textsc{(s.~\pageref{a51a602ccd87730203211131f20c5d94})}, eltejp, sand \textsc{(s.~\pageref{88336b5bb2a1cc21bac7cf33fd451270})}, nyckelringar, isskrapa halvtäckt av smält kexchocklad, fiskeflöten och kvitton. Hos rutinerade bilägare ligger där också en rostig morakniv och ett par repiga läsglasögon. Allt detta är till för att förenkla förarens möjligheter att ratta fordonet på ett säkert sätt. I finare bilar, som till exempel Volvo 740 \textsc{(s.~\pageref{e262951543da05bac43c7b87235a115c})}, har handsfackets lucka försetts med en uppfällbar sminkspegel och en jättedålig mugghållare, så att även den som sitter shot gun kan nyttja fackets praktiska möjligheter.

}

\small{
\textbf{Handvass}
\label{38cafce8227ecded58f7f84e0c4d7a68}
 Ett dialektalt ord från Västerbotten \textsc{(s.~\pageref{d4b008c5143dcffb6b8c35f3876c2a19})} som signalerar att en person är jävligt stark i nyporna. Handvassa personer lyckas alltid öppna syltburkar och kan vrida upp en vanlig kapsyl som vore det en skruvkapsyl \textsc{(se skruvkapsylöl s.~\pageref{6b1e7b5dfefe355c219b8bd7ff4db28c})}.

}

\small{
\textbf{Hardware}
\label{3ca14c518d1bf901acc339e7c9cd6d7f}
 är en post-apokalyptisk skräckfilm från 1990. Den handlar om en kille som går på skrotloppis och köper en gammal häftig robotdel åt sin konstnärliga flickvän som gärna bygger skulpturer av gamla robotdelar och skrot. Det visar sig att denna robotdel är en självreparerande och väl fungerande bit metall som snabbt bestämmer sig för att börja mörda allt levande i sin omgivning. Sen har konstnärstjejen en granne som fulonanerar och tittar på henne genom fönstret med en kikare \textsc{(s.~\pageref{e2b5cb2875a91cea4345226ce26ada44})}.

 Filmen hör till sub-genren skräckfilm med självreparerande robotar som skräckmoment. Den påminner i upplägget om filmen Virus som handlar om en båt där en självreparerande robot av misstag kraschat i sin resa till jorden för att utplåna mänskligheten innan vi hinner påbörja våran virusliknande spridning ut i rymden.

}

\small{
\textbf{Hasch}
\label{1e93612a55f48e5fd9cbce22d0e71944}
 (Fornpersiska, uttalas som Ernst-Hugo Järdegård skulle ha uttalat det) är ett slags brun koda från mellanöstern. Den är väldigt förbjudet i Sverige \textsc{(s.~\pageref{b1999637949ed135b2ca03f3a38460cc})} eftersom man blir så fnissig och lite pratig av att röka det. Som tur är kan man hinka i sig brännvin \textsc{(s.~\pageref{ff49ececa32cff978496a39635496f46})}, köra motorbåt och åsamka en gammal tant hjärnskakning genom att slå henne över huvudet med en metallstång \textsc{(s.~\pageref{6b45527e41ce216a150d4ac5950322bd})} under en redig spritfylla \textsc{(s.~\pageref{0668c687b51995118ec27cbf25061118})} istället, och på så vis åka i fängelse den lagliga vägen.

}

\small{
\textbf{Hashtag}
\label{b2d28ef83f2b78b798fc40f9ded46a5a}
 Har ersatt smileyn som uttrycksform i skrift.

 Ej att förväxla med Haschtag \textsc{(s.~\pageref{49194eebfa33965995db7accc1e1858c})}.

}

\small{
\textbf{Hasselbackspotatis}
\label{9354ad56ba8ac2ead1daeea852c88bec}
 En stackars bintje \textsc{(s.~\pageref{f21f4f64cb0df1775b5c2a7dc0d83c6c})} eller King Edward \textsc{(s.~\pageref{a081b9ae5423fc12a8439e33b2af8bed})} som någon sadist hackat upp till hälften för sitt eget höga nöjes skull.

}

\small{
\textbf{Havsmunk}
\label{81441ab6429e473c2a7679b4a54246e3}
 (\textit{Monachus monachus}) är en sälart \textsc{(se sälar s.~\pageref{ff14e847bd014e34af62f5c5855a1bdc})} som främst brukar ligga och glassa \textsc{(se glasse s.~\pageref{3b2b4e26097f9dc1b8baa4c53609563f})} i Medelhavet och östra Atlanten. Den kan väga nästan 300 kg och bli uppåt 30 år. Tyvärr blir de flesta havsmunkar bara någon månad gamla för naturen har försett arten med den rätt korkade vanan att föda ungarna i grottor där ingången ligger under havsytan. Vid höga vattennivåer översvämmas sådana grottor ganska ofta och familjen drunknar. Hårt och inte särskilt rättvist. Till skillnad från många andra sälar saknar havsmunken faktiskt öron.

}

\small{
\textbf{Hawaii-pizza}
\label{742e4954c36e42931521b0a417511c7c}
 tar dig med till en tropisk värld där livet är enklare, gladare och där ljumma nätter vibrerar av exotisk sensualitet. När man får sin Hawaii-pizza \textsc{(s.~\pageref{7cf2db5ec261a0fa27a502d3196a6f60})} serverad är det därför bara naturligt att för en stund sluta sina ögon samtidigt som man drar in doften av fett och mjuk ananas genom näsborrarna och drömma sig bort till orangea solnedgångar, det avlägsna ljudet av oljefatstrummor, stålsträngad gitarr \textsc{(s.~\pageref{a08bf8420208934bc59c7ed7385d4308})} och ljusa kvinnostämmor, till söta frukter och lekande delfiner \textsc{(se delfin s.~\pageref{a62b1fca6b53d6670a84aa2c7b373b27})}, till bar hud och djupa suckar.

 Av förklarliga anledningar är hawaii-pizzan populär som bakfyllekäk efter en redig vinfylla \textsc{(s.~\pageref{aee462fab19723e71e7f1f3302309d1e})}.

}

\small{
\textbf{He}
\label{6f96cfdfe5ccc627cadf24b41725caa4}
 (verb, infintiv) är ett ord på västerbottnisk bondska. Det är ett ersättningsord för alla verb som kretsar kring att placera ting på olika platser, och mycket, mycket mer. Det böjs He Hedde Hett.

 Exempel (Inom parentes översättning till rikssvenska):

 \begin{itemize}
 \item He på lyset! (Slå igång den elektriska inomhusbelysningen!)
 \item He igång biln'! (Starta automobilen!) \textsc{(se bil s.~\pageref{b3188f47d2eac7efc3f1258dc673a9fe})}
 \item He de hem! (Gå hem!)
 \item He de borda hygget! (Förflytta dig ifrån detta nyligen avverkade skogsområde)
 \item He int schwartmyren i ryggsäcka! (Stoppa inte svartmyror i min ryggsäck)
 \item He på nå pären! (Var god börja koka potatis.)
 \item Han derna Reinfeldt \textsc{(se Fredrik reinfeldt s.~\pageref{0c16c01849fc86b54e9e0e815490f747})} skull ha va hedd bort för länge sen. (Vår statsminister, Fredrik Reinfeldt, borde ha avgått för länge sedan)
 \item Nä, om man sku ha hett se hemåt vägen. (Nä om man skulle ha tagit och gått hem)
 \item N' Holger ha hett se åt Missenträsk. (Holger har begett sig till Missenträsk)
 \end{itemize}

 En dialog mellan två västerbottniska potatisbönder kan således se ut så här.

 - Ha du hett ner nå King Edward \textsc{(s.~\pageref{a081b9ae5423fc12a8439e33b2af8bed})} i pärbänka i år?
 - Nä, ja hedde bara ner Bintje \textsc{(s.~\pageref{f21f4f64cb0df1775b5c2a7dc0d83c6c})}.
 - Tro du he vall nå bra?
 - Nä fan heller.

 Märk väl att he inte är ett verb i den tredje strofen, utan en lokal variant för ordet \quotetext{det}.

 HEAD2: Övriga betydelser
 He kan också vara det onomatopoetiska ljud som kommer efter att en superskurk avslöjat sin diaboliska plan. Till exempel:

 -You see, I have a gift. An instinct for sensing people's weaknesses. Yours is women. Hers and mine are winning, whatever the cost. So when I arranged for that fatal overdose for the true victor at Sydney, I won myself my very own MI6 agent, using everthing at my disposal - her brains, her talent, even her sex, he he he! (Gustav Graves i \textit{Die another day})

 -Vad Lennart Holmlund inte vet är att det är jag, Randlofo \textsc{(se randolfo s.~\pageref{b8f0a32f840f1db27a2c12e17b640fb2})}, som lagt motionen i fullmäktige, och den innehåller en paragraf om att införa kommunala subventioner på guldfiskar he he he! (Randolfo på \textit{Mias grill})

 I engelskan betyder \textit{he} \quotetext{han.} Exempel, \quotetext{He dislikes trousers} (Sv. Han ogillar byxor).

}

\small{
\textbf{Headbanga}
\label{2846d09194c713ef6abd3c0c7eadbf5a}
 Att headbanga är att våldsamt nicka med huvudet i takt med hårdrocksmusik \textsc{(se hårdrock s.~\pageref{a4566a810e7ad85a57ddc84083a8139b})}. Vad nickandet signalerar är ett intensivt medhåll från lyssnarens sida. Vid varje takt nickar lyssnaren, som för att säga till artisten, \textit{ja det där var bra, och det där, och det där också, och den där med, och oj den var fräck, och den, och den, och den, och även det taktslaget fångade mitt intresse, och det, och det, bra så, bra så, japp, japp, japp} osv.

 Är man riktigt intresserad av att visa sin uppskattning för artisten står man längst fram och är en front row banger \textsc{(s.~\pageref{bd4740b18eb6aca9232b48ea476f7547})}.

}

\small{
\textbf{Heavy}
\label{7cfe64ea44dc3bbeb63b29ff3039a481}
 handlar om att vara förbannad och att vara en krigsherre. Och att lägga ner, typ... Typ som nån som är i ett krig och om han hade en yxa och ba högg en snubbe i huvet och yxan landade och man gör ett riff som är precis så. DET är heavy music.

 Källa: Matt Pike \textsc{(s.~\pageref{f1e5b05112d62b84340f4d287585d83d})} [http://www.youtube.com/watch?v=OjtS1FbYeIg].

}

\small{
\textbf{Hedersknyffel}
\label{b74d0cddfe72c60ae2fa8850f2bc3c45}
 http://rikard.info

}

\small{
\textbf{Hefaistos}
\label{7aee5e07996a14b4fd13e0f23aeb3a92}
 var smidets gud \textsc{(s.~\pageref{91e49146121c992aab11a19c77e26cf0})} i det gamla Grekland. När han föddes tyckte hans morsa att han var så vanskapt att hon kastade honom från Olympen, han föll i nio dygn och landade i havet. Fatta att ramla i nio jävla dygn. Jag tror inte att det finns någon aktivitet som är rolig i nio dygn - allra minst att ramla.

}

\small{
\textbf{Hegemoni}
\label{d6099ee381402a854e5ef9d61ac3c28c}
 Enligt Antonio Gramsci \textsc{(s.~\pageref{d4d0da57d321555b3550f1d7cffa3249})}, som lanserade begreppet i politisk bemärkelse, handlar hegemoni om idéer som har en dominant ställning i samhället. Dessa hegemoniska idéer ses som sunt förnuft, eller av naturen givna sanningar i samhället. Hegemoni är inom marxismen ett verktyg för de med makt att skapa samtycke om att det rådande samhället är det bästa. Men, eftersom idéerna är kulturella konstruktioner och inte fysiska institutioner (även om de för all del kan ta sig uttryck i sådana), kan man attackera de rådande idéerna utan att bli skickad i fängelse eller bli skjuten i huvudet av en snut, vilket händer om du attackerar statens fysiska institutioner.´Nedan presenteras en kortare lista över konstruerade sanningar som måste brytas ner för att åskådliggöra verklighetens egenliga beskaffenhet.

 Det är viktigt att ta för sig
 Ditt liv är speciellt
 Cadillac är finare än volvo
 Att organisera sig fackligt leder inte till något
 Det är finare med chorizo än falukorv
 USA \textsc{(se United States of America s.~\pageref{ade6b3bd5e720abb20ed8a9a4c6b9ae8})} är coolare än Tyskland \textsc{(s.~\pageref{b1b58da783b6d5fa090f3015f1889869})}
 Om du jobbar tillräckligt hårt kan du bli vad du vill
 Metallica är världens bästa band
 Kvinnor på chefspositioner skapar ett bättre samhälle

}

\small{
\textbf{Hej hej}
\label{0d474411c69210e6f0c68de34085f9d3}
 är en kurs som varannan kassörsperson som jobbar på coop har varit på. Där fick de lära sig att säga \textit{\quotetext{hej hej}} till alla kunder i kassan. Hej hej får inte låta glatt, och det ska vara en lätt uttråkad nedgång i satsmelodin på det andra hejet. Detta för att ingen störande mellanmänsklig kontakt, och absolut inte kommunikation, ska uppstå som tär på arbetstempot och vinstmarginalerna. Anställda som gör uppror mot hej hej deporteras till Kalix.

 Aktieägarna i Coop \textsc{(s.~\pageref{0b5cb0ec5f538ad96aec1269bec93c9c})} kallas för anställda direktörer och tjänar multum på hej hej. Bonuskunder i Coop kallas för medlemmar och har inget att säga till om. Annat var det förr.

}

\small{
\textbf{Hej hej skiva}
\label{d5429f5da84a15e8b14f7e09ace9c4cb}
 Som namnet antyder är detta en skiva.
 Vilkets slags skiva undrar du då? Jo, det är varken en bords- eller ostskiva, utan en alldeles suverän musikskiva med det svenska bandet \textit{Nic \& The Family} som utkom år 2004.

 Singeln \textit{Hej Monica} fick gott om speltid på radion och uppnådde sommarplågestatus.
 De riktiga fantasterna ser dock ofta något av de andra spåren som starkare.
 I en undersökning av inofficiell karaktär blev \textit{Djungelvrål} och \textit{Hej det är Nic...Klick} oftast nämnda som personliga favoriter bland de tillfrågade.

 Spelas med fördel på både för- och efterskiva. Däremellan står du alldeles för länge i kö för att köpa en alldeles för dyr redbull vodka och är alldeles för full till tonerna av modern dunka-dunka.

 I maj 2011 ryktades det om att bandet skulle släppa en singel i digitalt format efter ett långt uppehåll.
 Enligt uppgifter skulle uppföljaren heta \textit{Hej hej mp3-fil}, men med facit i hand kan vi konstatera att det rann ut i cybersanden.

}

\small{
\textbf{Hejåhå}
\label{872fb693de8a0eb88c6374ea0343b1c2}
 Det finns så mycket hejåhå. Håhåjaja.

}

\small{
\textbf{Helgarderat efterliv}
\label{9b0ade94f583f387a033b9b4da0da76e}
 Vill man vara säker på att ha något att göra efter döden \textsc{(s.~\pageref{6f3c270eb5b4d979c777b4ec26dd106f})} så är det bäst att helgardera. Ingen vet vilken av våra fem världsreligioner som har rätt så därför är det bra att inte stryka någon av dem mothårs.


 HEAD2: Kristendomen
 Vi börjar med kristendomen och då kan en bra idé vara att följa de tio budorden. Det första budordet \quotetext{Du skall icke ha andra gudar jämte mig} blir i denna hypotes lite knepigt att efterfölja, men om man ändå inte tror på någon av religionerna så har man trots allt ingen gud jämte någon annan. Ateism är alltså här att föredra, hur absurt det än låter. De andra budorden är väl inte helt omöjliga att efterfölja, hålla sig i skinnet och softa på söndagar \textsc{(se söndag s.~\pageref{85b2e5c3758394a24221d1abac79191a})}. Det här med att man inte får stjäla får väl ses i perspektiv \textsc{(s.~\pageref{1606dd19366985367d677f7b6de46e52})} mot hur mycket borgarna själ av oss hela tiden. Om ni måste svära så se till att göra det ordentligt och inte säga \quotetext{Herregud! \textsc{(se gud s.~\pageref{91e49146121c992aab11a19c77e26cf0})}} och annat trams.

 HEAD2: Judendomen
 I egenskap av abrahamitisk religion så har judendomen också tio bud, men dessa kompletteras av hela 603 (!) bud till. Tack och lov så handlar precis vart enda av dessa 603 om hur och vilken mat som man får äta. Alla matvaror som behandlas av dessa bud handlar om djur så vegetarianism är en väldigt enkel väg att gå, men mer om det senare.

 HEAD2: Islam
 I dagens samhälle är det svårt att be en massa gånger om dagen, så det är bäst att slå på stort. Ta en semester till Mekka och delta i circle pitten runt kaban så kommer nog Allah tycka att du är en helt ok lirare. Vill man verkligen vara på den säkra sidan kan man uppfylla de tre andra pelarna också. Allmosan är inga problem om man bor i ett land där man betalar skatt som går till samhällets utsatta. Än så länge gör vi det i Sverige \textsc{(s.~\pageref{b1999637949ed135b2ca03f3a38460cc})}, men se upp så att det inte blir blåval \textsc{(s.~\pageref{df5ac0401fcf8ea926b70eedfc67e82d})} igen. Fastan låter väl inte jätterolig, men tänk på att man får äta på natten. Se till att vända på dygnet under Ramadan. Trosbekännelsen blir knepig om man är ute efter att vara nere med alla religioner, så kompromissen får bli att man skriver den på en lapp och stoppar i bakfickan \textsc{(se bakficka s.~\pageref{d259b5ebe8541b74129f0c78a82335b7})}. Den sista pelaren är bönen och den hinns som sagt inte med i dagens samhälle, det får baske mig Allah ha överseende med.

 HEAD2: Hinduismen
 Den här religionen fungerar extremt bra att kombinera med de andra, då den faktiskt inte menar att den är överlägsen någon annan, väldigt sympatiskt. Däremot har den inget efterliv i samma mening som de abrahamitiska religionerna, utan man återföds. Dock kan man återfödas som olika saker som ändå får lov att ordnas hierarkiskt med kloakdjur \textsc{(s.~\pageref{592254da9a0f26310a65ff83f6d73c9e})} längst ner, gulsvansad ullapa \textsc{(s.~\pageref{c9022fba896b3e8a41420242680d2480})} någonstans i mitten och uv \textsc{(s.~\pageref{45210da832f9626829457a65e9e7c4d0})} längst upp. Det ultimata ska tydligen vara att inte återfödas alls, men det låter ju ta mig tusan helt befängt. Vill man slippa ett liv i träck så kan man ta till sig följande regel: Kor är heliga, så vegetarianism är fortfarande ingen dum idé. Sen har vi konceptet karma som kort går ut på att om man beter sig extremt osoft så lutar det åt att man blir ett kloakdjur \textsc{(s.~\pageref{592254da9a0f26310a65ff83f6d73c9e})}. Följer man budorden från de abrahmitiska religionerna så borde man redan ha ganska bra karma, så det är lugnt.

 HEAD2: Buddhism
 De två österländska religionerna är ganska lika så genom att vara nere med hinduismen så får man buddhism på köpet. För att vara störtsäker kan man ta till sig de fyra sanningarna: Det finns osofta företeelser, anledningen till dessa är begär, begär kan och måste tas bort och detta görs genom den åttafaldiga vägen. Till vårt försvar så är det väldigt lätt att gå fel på en åttafaldig väg så det sista får nog bero.

 HEAD2: Slutsats
 Fixar ni det här så kommer ni inte hamna i helvetet eller återfödas som något obehagligt. Är man däremot ute efter ett drägligt jordeliv rekommenderas en marxistisk tolkning av världen.

}

\small{
\textbf{Helgvolym}
\label{3539fdeb41a5b216f614b6ced9ff5cff}
 Uppskattningsvis 25 \% högre än vardagsvolym.

 \textbar

}

\small{
\textbf{Helvetet}
\label{9c1e91d22a5df2b06a57fba276f94b5c}
 är en plats som det finns många teorier kring. Den knastrande croissangen Jean Paul Sartre menade att helvetet var andra människor. Men det är det inte alls. Den italienske författaren Dante Alighieri var god nog att efter en resa till platsen i fråga skissa upp vad helvetet är och vad som finns där, i en av delarna i sin bokserie Den gudomliga komedin.

 Enligt Dante ligger helvetet under marken. Antagligen börjar helvetet efter jordlagren där man i genomskärning kan se skattkistor och dinosauriefossiler ligga och vila. Helvetet består av nio cirklar. Den yttersta cirkeln heter Limbo. Där hamnar alla som i Guds ögon är ganska snälla men inte kristna - så kallade nobla vildar. Stora personligheter som Orpheus, de sju dvärgarna i snövit och Don Dokken promenerar runt där nere, lätt konfunderade över vad fan de håller på med och när de får åka därifrån. Svaret är aldrig.

 Den andra cirkeln är den plats där de som syndat genom Lust befinner sig. Cleopatra och Paul Stanley (vid frånfälle) huserar i den obekväma miljön. Antagligen försedda med kyskhetsbälte, bara för att jävlas.

 I cirkel tre finner vi frossarna. Theoden ståthållare av Gondor och Micke Dubois drar fräckisar där i en evighet.

 I fyran hamnar alla giriga. Joakim von Anka och Guldivar Flinthjärta är i cirkelns centrum, locked in eternal combat, som de säger i engelskspråkiga länder.

 \quotetext{På femman} som de säger där, är de arga. Alla som någonsin rage-quittat efter en keff \textsc{(s.~\pageref{890a42bbf6c2e6888fb851dd76e1e980})} omgång starcraft sitter där. Många fjortonåringar.

 Sjätte cirkeln inhyser kättare. Mohammed och Karl Marx hinkar bärs under ett plommonträd.

 Cirkel sju är avsedd för våldsverkare. Hulk Hogan och Henry Kissinger går oändliga mängder ronder.

 I åttan är alla bluffmakare. Där sitter Göran Persson för att ha kommunaliserat skolan.

 Och i den innersta nionde cirkeln sitter satan själv, med Judas Iscariot och andra renommerade förrädare som Jussi och Grima Ormstunga.

 Och det är Helvetet.

}

\small{
\textbf{Hemlig agent}
\label{8eda5d731e1c289ada64034bd0bfc0ea}
 Det finns egentligen inte någon som vet vad en hemlig agent gör eftersom deras arbete är hemligt.


 HEAD2: Ett halvtråkigt skämt om hemliga agenter
 En man i skidmask rånar en bank, en gubbjävel i kön säger: \quotetext{Sluta upp genast, jag är hemlig agent!} Mannen i skidmasken blir rädd och sticker, varpå gubbjävelns assistent vänder sig till honom och säger \quotetext{Men Åke, du är ju inte hemlig agent, du är ju senildement!}

}

\small{
\textbf{Hemliga koder}
\label{bcdedbe776850b161921d9d244f82452}
 är sätt för slutna sällskap att kommunicera med varandra utan att utomstående förstår vad det är frågan om. Stonerskins \textsc{(se Stonerskin s.~\pageref{b94c65dba2990b3146c2bedf663e9989})} använder sig till exempel av frisyrer hämtade från \textit{Dödligt vapen}-serien för att signalera tillhörighet. För en oinvigd ser det bara ut som ett märkligt sammanträffande \textsc{(se märkliga sammanträffanden s.~\pageref{f46282d99158f351a81b9deaff157b4e})} men fyller alltså ett djupare syfte. Om du ser en svensk stålåsna \textsc{(se Volvo 740 s.~\pageref{e262951543da05bac43c7b87235a115c})} försedd med en bildekal \textsc{(s.~\pageref{7bdee158c028cef0e4bcd04631250180})} innehållandes gamla sosserosen vet du, om du känner till hemliga koder, att det sitter en av Konsums \textsc{(se konsumbutik s.~\pageref{70e4875f7c2c177596305006e46b7ca9})} stolta ägare \textsc{(se ägmästare s.~\pageref{8324518500d7e7ccd22ae364887d4476})} bakom ratten. \quotetext{Vad roligt!}, tänker du säkert nu, \quotetext{om jag lär mig dessa kryptiska signaler kan jag skaffa mig nya kamrater}. Visst kan du det, MEN tänk på att också fienden använder samma taktik. Om du till exempel möter en medelålders man på flerväxlad cykel \textsc{(s.~\pageref{cd75a1ec5d4b7caabeaaaf25edee0250})} är han med stor säkerhet medlem av någon av illuminatiavkommorna storfräsare \textsc{(s.~\pageref{4db17005692cd83e3e946a1311b81ed0})} eller ICA-personal i förklädnad \textsc{(s.~\pageref{f5a6964fb398df4c2da0d3bac3d8ed7a})}.

 HEAD2: Några till bara för att det är ett roligt tema

 Om du ser den gamla sosserosen i något annat sammanhang betyder det att personen har stomipåse.

 Om du ser en regnbåge är det förmodligen en leprechaun som signalerar till sina kumpaner att hen har grävt ner en ny kruka med guld.

 Gå och köpa tidningen \textsc{(s.~\pageref{86fb6ed06892c06b596fe55aa3468d7f})}.

 Om du är på PRO-möte och någon plockar ut löständerna och börjar putsa på guldtanden vet alla invigna att det blir päronhalva \textsc{(s.~\pageref{cc9c1bfa2ec4eaed89ca86a1b63e3a45})} efter kaffet så det gäller att tömma stomipåsen innan.

}

\small{
\textbf{Hemmansvärde}
\label{589faebe1f78d343077a0f8de5364ef8}
 Ett hemmansvärde är en diffus måttenhet populär i Västerbotten \textsc{(s.~\pageref{d4b008c5143dcffb6b8c35f3876c2a19})}. Den åsyftar värdet på hemmanet, alltså familjens gård, komplett med kräk, fuse och loge.

 Exempel:
 \quotetext{Ska du köpa ny kaffebryggare igen?}
 \quotetext{Ja det är ju inget hemmansvärde direkt.}

}

\small{
\textbf{Hemmets Härold}
\label{812e1c9a5a14e8d21ece7bfbdba893c4}
 är Pingstkyrkans \textsc{(se Ping s.~\pageref{df911f0151f9ef021d410b4be5060972})} skivbolag som faktiskt funnits i nästan hundra år. Utan Hemmets Härold skulle landets begagnatskivbutiker bara ha hälften så många sumpiga femkronorsbackar som står och tar plats helt i onödan. Från 1938 till 2007 var Svante Widén från Avesta producent på de flesta av bolagets inspelningar. Samarbetet avbröts dock när Svante dog.

 HEAD2: Klassiska namn på Hemmets Härold
 Pelle Karlsson \textsc{(s.~\pageref{1a8c873ff230698396c324f14c02b7fa})}
 Gitarrbröderna Värnamo
 Ny-David
 Pärla

}

\small{
\textbf{Hemohes}
\label{78daf59d2c820001becb4f44e9e89ab0}
 Populärt finskt groggvirke \textsc{(s.~\pageref{ba264d4eb820b4066de4c8723a08f824})}.

}

\small{
\textbf{Henke Larsson}
\label{6df12803e750ca6844878c81add0d73f}
 är en svensk fotbollsspelare och syndabock. Förutom några år i skottska \textsc{(se skottar s.~\pageref{c2e5f84c76d823ea9482387bfb950791})} Celtic där det gick hyfsat för Henke (ungefär lika prestigefyllt som att spela i div 2, norra Västmanland), är han av gemene man främst ihågkommen för sin fatala straffmiss i VM -94 som höll på att kosta Sverige \textsc{(s.~\pageref{b1999637949ed135b2ca03f3a38460cc})} bronset.

 HEAD2: Upprinnelsen till fiaskot
 Efter en heroisk insats av bland annat Håkan Mild, Martin Dahlin, Stefan \textsc{(s.~\pageref{2e970e822e1a8834203d06abb60f59ec})} Schwartsz och Kenneth Andersson, stod det efter ordinarie speltid 2-2 i kvartsfinalen mot Rumänien. Resultatet stod sig även efter förlängning och man blev tvungen att avgöra matchen på straffar. Eftersom alla kände på sig att Henke skulle suga på det här lät man honom ta första straffen för att få det överstökat. Och mycket riktigt. Henke slog en helt bedrövlig straff. Målvakten hade redan slängt sig åt ena hållet och låg närmast ihopkrupen bakom stolpen; en komapatient hade med lätthet satt straffen. Men inte Henke. Han tog istället fart och skickade iväg läderkulan långt upp på läktaren...

 Som tur var hade Sverige vid denna tid den eminente målvakten Tomas Ravelli, som i princip själv återställde ordningen genom några av tidernas snyggaste räddningar [http://www.youtube.com/watch?v=6jLjfsF1fHI].

}

\small{
\textbf{Heroin}
\label{66c4e6fa3361e746620634353e6f15e3}
 Horse, junk, dope och smack är alla smeknamn på den fruktansvärda drogen heroin. Nissepedias skribenter saknar alla ord för att beskriva den vita fasan. Istället lämnar vi den biten till en sann poet, Jerker i Rövsvett, som sjöng om heroin på Rövsvetts demo hunden beskyddar människan men vem beskyddar hunden? \textsc{(s.~\pageref{2d5984a809dd83cdd14ed73163d1b026})}.

 Heroin:
 Heroinet är ett dödligt gift
 beroendet efter mer bara växeer
 det har tagit död på många
 och fler står på tur
 du är nyfiken att testa tyngre
 langarna väntar på dej med nya tag
 med det är upp till dej om du vill må bra
 annars är det bara att säga nej!!!!!!!

 Han hittades död
 på en tunnelbanestation
 han flydde dit i nöd
 det blev hans sista destination

 Heroin - döden i en spruta
 heroin, så vill väl ingen sluuta

 I skolan var han kass
 han började röka brass
 ifrån sin oro ville han kuta
 det var då han började skjuta

 Han tog sin jungfrusil
 det var i mitten av april
 livet det stod still
 nu är han död som en sill

}

\small{
\textbf{Herrcykel}
\label{8da016d4ced142cc447d520eaa04c33e}
 Inte lika ballt som en Damcykel \textsc{(s.~\pageref{2f0f41314b4e4edb773a7ae87addc913})}.

 En herrcykel har högre utväxling än en damcykel, men det är inget folk bryr sig om idag. Detta förhållande borde appliceras på samhället i stort. Vi kan lära oss mycket av våra cyklar.

 Category:Trafik \textsc{(s.~\pageref{8a2f75cb2fdbbd1b67833430f8bc0f33})}
 Category:Samlevnad \textsc{(s.~\pageref{e8e398a6ccf3be3505d3117cbffa314b})}

}

\small{
\textbf{Herrkläder}
\label{2ca60777e59875280b6db600fc53569d}
 är en gren på mode-rikets brokiga träd. Inom gruppen herrkläder ingår sådana plagg som den modemedvetne i första hand associerar med  mannen \textsc{(se man  s.~\pageref{39c63ddb96a31b9610cd976b896ad4f0})}, men som mycket väl också kan bäras av andra, till exempel David Batra. Typiska herrklädesplagg är läderväst, belt-buckle, boots, skogshuggarskjorta, gummistövlar (utan färgglatt mönster), kepsar med olika företagslogotyper \textsc{(s.~\pageref{6a414633590fd4cd6d6ac64798d14c14})}, speedos \textsc{(s.~\pageref{22286b2c61cbd4c567b0999a958db3eb})} och ett par skitiga jeans.

 Utlottning av herrkläder brukar kombineras med älgköttsoppa och snapsvisor vid fester i hembygdsgårdar.
 HEAD2: Önskar du uppnå succé genom att klä dig i herrkläder?
 Då har du ett svårt dilemma framför dig, för detta är nämligen en självmotsägande paradox. Om du medvetet vill uppnå succé genom att klä dig så har du redan gjort det omöjligt att lyckas med ditt företag. Om du däremot slänger på dig det du hittar i tvättkorgen, bakar en snus och kliver ut i världen kommer du, med lite tur, att anses vara nonchalant maskulin så som till exempel Bon Scott, Thåström och Stor-Anders \textsc{(s.~\pageref{777d0562284d1dfba75c6f1b6297100d})}.

}

\small{
\textbf{Herrtoaletten i Lindellhallen}
\label{04ff06fb85370126485265886d1be53e}
 är en hygieninrättning vid Umeå Universitet. Här finns 12 bås och 5 handfat. Pissoarerna är borta sedan länge. Det bästa båset är det mittersta längst in till vänster. I det båset rasade en debatt om huruvida kapitalism eller socialism var det förnämaste produktionssättet VT 2010, socialism vann. Diskussionen är idag borttagen. Toaletten är också den enda platsen i Umeå där Svenskarnas Parti sprider sin propaganda. I ett annat av båsen har något geni (kanske en medlem av Svenskarnas Parti? Troligen inte, det är alldeles för fyndigt.) ristat det odödliga citatet \quotetext{Fan va det är gött att dynga, borde jag testa analsex?} Den avantgardistiska konstnärsduon Hökklo och Dolken \textsc{(s.~\pageref{6cd77f9a37b6f1bbf39abffa383e2b9e})} har haft vernissage här.

}

\small{
\textbf{Hertigen av Rothesay}
\label{4034a7db63c1263a0844259903a85a62}
 är prins Charles hovtitel i Skottland. I England kallas han \quotetext{prins av Wales}, men det tycker skottarna är att ta i lite väl. Charles bästa ovän är hans mamma som aldrig vill dö.

}

\small{
\textbf{Hertsökullar}
\label{01e4df862e1ef84d2cdf12093edd7373}
 är ett par högar av typ bajs och lera på bostadsområdet Hertsön i Luleå. På Hertsökullar är de populäraste aktiviteterna på vintern pulkåkning och hundrastande medan sommarens aktiviteter består av haschrökande \textsc{(se hasch s.~\pageref{1e93612a55f48e5fd9cbce22d0e71944})} och att tända eld på grejer. Från Hertsökullars högsta topp cirka 2 \textsc{(se tvåa s.~\pageref{84fcc0494ecf9f5af79fcd9bed184a9a})} meter över havet kan man se så långt som till SSAB ibland.

}

\small{
\textbf{Hertsön}
\label{e4b227ae8f6ccfed5aae6fdc7e655623}
 är det största bostadsområdet i Luleå \textsc{(s.~\pageref{3cefb5ac35187749592f1ebb25472b99})}. Under 70-talet skulle Stålverk 80 byggas i Luleå och stadens befolkning skulle komma att öka ofantligt så man byggde lite betongklumpar för att bemöta efterfrågan på bostäder. Hertsön var den största av betongklumparna och där skulle alla de nya arbetarna på Stålverk 80 konstruera barn och bli ansvariga konsumenter. Det blev dock inget Stålverk 80 och allt gick åt helvete så ett gäng fulla knivfinnar \textsc{(se finländare s.~\pageref{fc472090d678bd6f029cd80792f4a36d})} flyttade in på Hertsön istället. Knivfinnarna blev istället barnkonstruktörer och skapade mängder av små finnar. På nittiotalet anlände även kurder, somalier, tornedarlingar och lite knarklangare. På senare dagar har det även kommit rapporter om att Skåningar \textsc{(se Skåne s.~\pageref{a01d1167b9dcd72e212d876d672db261})} och Stockholmare \textsc{(se Stockholm s.~\pageref{edcd259e0a03c7ab70feb186bae19f13})} flyttat till Hertsön för att studera på Luleå Tekniska Universitet \textsc{(se Universitet s.~\pageref{11dfc744fa396b961a6cc9cf89c4eaea})}. Dessa har dock haft svårt att anpassa sig till sitt nya land och många har återvänt hem.

 På Hertsön finns mycket att se och uppleva. Bland annat Hertsökullar \textsc{(s.~\pageref{01e4df862e1ef84d2cdf12093edd7373})}, misstro och uppgivenhet, ett stall, ett koloniområde. Man kan bad på badhuset, låna böcker på bibblan och köpa folköl på Ica.

}

\small{
\textbf{Hildegard av Bingen}
\label{c7c6415c032f7d851cd2c0a11f40be0b}
 var en nunna som under medeltiden \textsc{(s.~\pageref{88cbc30c5b233d97df68b5b041ac0655})} skrev en massa trams om Gud \textsc{(s.~\pageref{91e49146121c992aab11a19c77e26cf0})} och sådant som man skrev om under denna period av allmän intellektuell regression \textsc{(s.~\pageref{83d2f1dedf9cade132d175d197430f11})}. Hon hyllas idag av många feminister för att hon var kvinna \textsc{(s.~\pageref{9a7760b2521c3471c47cd5d789a2d324})}.

}

\small{
\textbf{Hillevi}
\label{758961a2bf96af710999ea02164fb582}
 är ett svensk kvinnonamn som betyder ungefär \quotetext{Joe Hills minne ska alltid leva}.

}

\small{
\textbf{Hip-hop}
\label{66c22415908267e727d3fa4a63c16672}
 är en musikstil som är minst lika bred och mångfacetterad som doom \textsc{(s.~\pageref{b4f945433ea4c369c12741f62a23ccc0})}, men handlar till skillnad från doomens symbolekonomi av himlakroppar och mörk tomhet om damer med stora behag samt pengar \textsc{(se valuta s.~\pageref{cf1e2a0af4955aa7539b6e12e9d282e6})}. Den kännetecknas av pratsång ovanpå en matta av rytmiska ljud som utstöts i kupade händer som hålls upp mot ansiktet. Denna \quotetext{sång} framförs ofta av entreprenörer mot en fasad av  pizzaracer \textsc{(se pizzaracer s.~\pageref{19c6d2a54dcb50b16c3a3b7c6c8a1a09})} och obscent lättklädda flickor. Det anses mycket lyxigt och fint.

 HEAD2:  Hiphop och samhället

 Den vita arbetarklassen har sällan något till övers för hip-hop, medelklassen tycker att den är \quotetext{genuin} och överklassen är som vanligt helt världsfrånvänd och har inte hört talas om den.

 HEAD2:  Hiphopens utveckling

 Hip-hop fader är, som alla vet, Evert Taube med sin låt \quotetext{Den kinesiska muren \textsc{(se Kinesiska muren s.~\pageref{f5a025bc13e9af75673c6b3f647ebf5f})}}.[http://www.youtube.com/watch?v=QrgzNI98r78]
 Höjdpunkten inom genren nåddes dock 1992 av den amerikanska duon Kris Kross med låten \quotetext{Jump}.

}

\small{
\textbf{Hippie}
\label{14fd61fa8edcb67c5c7886f11af8431e}
 , kort för hipster. Vard.: ett nedsättande ord för allehanda träd-, varg- och kattkramare. På 1920-talet då hela hippievågen startade var det hippt med arier. På den vitala delen av kontinenten samlades entusiaster i små grupper som gemensamt benämndes \quotetext{Völkisch-}. Det hela slutade i ett abrupt nederlag efter ett kort men maffigt hippie-crescendo. Förgrundsfigurerna är fortfarande populära i Hippie-kretsar med namn som Adolf Hitler och Madame Blavatsky. Den första hippien med medial genomslagskraft var onekligen ovan nämnde Hitler. Gemensamt för hippies av både igår och idag är den bottenlösa relativismen. Ofta ingår en vurm för nakenhet och ungdomlighet. En illa dold fallenhet för blodsmystik, naturlighet och renhet är heller inte okänd. Hippierörelsen fick under 60- och 70-talen en nytt uppsving då under devisen \quotetext{All you need is love}. Stora tänkare från denna andra våg är Charles Manson och John Lennon vars tankar korsbefruktades intensivt. Vad som börjar med att unga pojkar brottas nakna på scout-liknande läger slutar inte sällan någonstans i Sydamerika med hopp om att sprida den germanska rasen medels incest. Den s.k gröna vågen kan sägas vara en syntes av dessa två första generationer av hippies. Storstadsbor en masse flyttade ut på \quotetext{landet} och rollspelade lokalbefolkning.

 Ett annat typiskt särdrag för hippies är den totala handfallenheten. Att man inte kan någonting döljs genom den spindelnätsliknande tankevärlden där \quotetext{rätt} och \quotetext{fel} inte existerar. Hellre än bra tycks vara parollen. Detta gör att hippies ofta återfinns i pseudoområden: exempelvis reklam och design. Här syns ständigt återkopplingen till den tidiga hippierörelsens frontfigurer, Göbbels och Hitler. Hitlers banbrytande gärning inom designen går knappast att ifrågasätta i korridorerna på designskolorna. Volkswagenbubblans hisnande och på samma gång läckert aerodynamiska linjer. Eller det legendariska samarbetet kring SS-uniformerna som aldrig tycks sluta influera film-, mode- och designvärlden. Man tror på \quotetext{något} - varför inte \quotetext{Blut und Boden}?


 Detta fortgår ständigt men det måste få ett slut.

}

\small{
\textbf{Hippies}
\label{4dc77d6258fd18e7c0dd5eece5c7c81c}
 Människor med långt hår som kan vara unga människor som går på kulturinriktade folkhögskoleprogram eller gamla människor som gillar vinfylla \textsc{(s.~\pageref{aee462fab19723e71e7f1f3302309d1e})} och pensionärserotik. Gemensamt har dessa olika hippies att de föreställer sig en bättre värld och för att uppnå denna är de beredda att sitta på gräsmattor och dricka vin. Ibland med könet naket och exponerat \textsc{(se kalle anka s.~\pageref{64db68f686a0ca4d9d641061cb3fdf13})}.

 HEAD2: Saker som hippes gillar
 \begin{itemize}
 \item Viss bra och en hel del dålig musik
 \item Fred
 \item Vin \textsc{(se vinfylla s.~\pageref{aee462fab19723e71e7f1f3302309d1e})} och hasch \textsc{(s.~\pageref{1e93612a55f48e5fd9cbce22d0e71944})}
 \item Skogsrave \textsc{(s.~\pageref{2180f77028a02c8fd94f622505937a53})}
 \item Knulla.
 \item Kultur
 \end{itemize}
 HEAD2: Saker som hippies ogillar
 \begin{itemize}
 \item Krig \textsc{(se krigsgrisen s.~\pageref{e95477b4d8cdc0372ffcac1d533623a2})}
 \item De flesta former av arbete \textsc{(se Tung industri s.~\pageref{454e5e8cb27bed118f0a6a1a01a6e6a9})}.
 \item Sport.
 \item Åtsmitande kläder
 \item Ny musik \textsc{(se Heavy s.~\pageref{7cfe64ea44dc3bbeb63b29ff3039a481})}
 \end{itemize}

}

\small{
\textbf{Hippopotamus}
\label{9b4609b17fea63f3f3f067fc2f465c6e}
 \textit{Hippopotamus amphibius}, eller flodhäst som den också kallas, är ett av de tuffaste djuren på jorden. Den degar för det mesta i vattnet men när den är på land kan den springa i 30km/h på korta sträckor. Den är ett av de djur som dödar flest människor i Afrika.

}

\small{
\textbf{Hipstermatic}
\label{d762519ff557d32ea6ec003d6d10b4d7}
 är en app som är vanlig bland urbana skithuven som tar kort på meningslösa saker. Eftersom motiven är meningslösa krävs något lite extra för att man ska kunna lägga ut sina bilder på facebook och få kommentarer om hur fina de är. Det är hipstermatic här kommer in i bilden, så att säga. Hipstermatic gör nämligen att bilderna påminner om ljusskadade foton som togs på den tiden när man ännu inte lyckats skapa kamerafilm som inte sög. På så vis signalerar fotografen att den kommer från en annan tid.

}

\small{
\textbf{Historiska händelser i badrum}
\label{883e86693d6804a30ae0d22311449058}
 \begin{itemize}
 \item Archimedes luktar så illa att han måste bada och upptäcker då sin egen princip.
 \item Gustav Vasa \textsc{(s.~\pageref{aeb7f10919b25762e3d031a0b583a2e8})} beträder Ornässtugans dass \textsc{(s.~\pageref{17d2effff6c1590dbff6a7ac39f46a19})} och Sverige \textsc{(s.~\pageref{b1999637949ed135b2ca03f3a38460cc})} påbörjar resan mot modern nationalstat.
 \item Frank Zappa får idéen till det där skivomslaget där han sitter på muggen.
 \item Den jätteäckliga scenen i filmen \textit{Trainspotting}.
 \item Polisen upptäcker varför Plura Jonsson alltid är så glad fastän han gör så tråkig musik.
 \item Ett av bibelns \textsc{(se bibeln s.~\pageref{7de7d2a7d608c9a2044f50688bc63e27})} märkligare avsnitt, \textbf{Jesaja 36:12 \textsc{(s.~\pageref{cddcbdb1e8a5df591e5efa642a584350})}}.
 \item Evelyn Waugh skyndar sig på påskdagen 1966 in på dass efter att ha deltagit i högmässan och avslutar där, sittandes, ett av Storbritanniens stilistiskt viktigaste författarskap genom att dö av ansträngning.
 \end{itemize}

}

\small{
\textbf{Hjulafton}
\label{0f994a1e8714e7c4e02d3dab466807c2}
 Tid för byte till sommar- eller vinterdäck.

}

\small{
\textbf{Hobofobi}
\label{e210eec4b750d8c50c4f98065f4d62fd}
 En hobofob är en person som är sjukligt rädd för att få sexuella inviter från småvuxna människor med stora, håriga fötter. Antagligen för att hen inte riktigt vet hur man artigt säger \quotetext{Nej tack.} i sådana situationer. Vetenskapen har hittills gått bet på att förklara hur hobofobin hos somliga individer kan blir så kraftfull, men man antar att det har en freudiansk koppling till att själv vilja ha riktigt håriga fötter men inte våga stå för det. Hobofobi räknas till de sociokulturellt inlärda fobierna, till skillnad från till exempel orchofobi, som har en starkt genetiskt-evolutionär bakgrund.

}

\small{
\textbf{Hockey}
\label{df0349ce110b69f03b4def8012ae4970}
 är en sport för killar där man kissar på varandra i duschen och runkar i grupp. Hockeyspelare har taskig musiksmak och klär sig ungefär som Peter Jihde.

}

\small{
\textbf{Hockeypulver}
\label{37757fcc292528e54d9fddb865947a98}
 är ett frätande ämne som uppstår som en slaggprodukt \textsc{(se slagg s.~\pageref{99084d72b557047a46c3b9ba2142afb1})} vid framställning av batterisyra. Vid torkning av ämnet bildas ett finkornigt pulver som används till att späda benmjöl vid industriell framställning av foderpellets.Hockeypulver kan i nödfall fungera som vitt pulver \textsc{(se tjackad s.~\pageref{b12ceb5f265e6ab9afcd2c662715e0b5})}.

}

\small{
\textbf{Hockeyröv}
\label{c904e78c8991794c8d598d44f0494f9c}
 En hockeyröv kännetecknas av att den är väldigt bred. Det enda som kan matchas i hur bred den kan bli är hur bred glipan kan bli, alltså avståndet mellan skinkorna. Se även Glimröv \textsc{(s.~\pageref{dd0a3a947a541f3d83b53a56be518062})}.

}

\small{
\textbf{Hold on}
\label{45c194a91bfef734109bb8ecaffa5561}
 \textit{Hold on} är det ûber-brutala Oi!-bandet Bonecruchers kanske enda kärlekssång.
 HEAD2: Refräng
 Hold on to me
 Please don't let me go
 Hold on to me
 Please don't let me go

}

\small{
\textbf{Holland}
\label{b95f379f6ae245614d2f949801524317}
 Ett stycke mark under havets yta. Holland är mindre än Bodens \textsc{(se Boden s.~\pageref{7c5dfb91b1d55bff98ec6d4faf83976b})} kommun så att erkänna denna markplätt som suverän nation är endast löjligt. Precis som sin granne Belgien \textsc{(s.~\pageref{f79ffe9e826a19f9f6a446c90e21c4e3})} är det mycket pedofiler och ravedansare i Holland.

}

\small{
\textbf{Holmfrid}
\label{3bf7f5142d389b9508caad2e1c5d2aec}
 är ovanligt som tilltalsnamn. Många som har haft detta namn har varit lagda åt att bo på öar i hus med torn på.

}

\small{
\textbf{Holmsunds tropikhus}
\label{5b087d935637ad4d1823cf48036e9be6}
 är enligt dess hemsida \quotetext{ett kunskapscentrum där man kan fascineras av livets stora, unika mångfald} som vill verka för en \quotetext{nyfiken natursyn som också tar hänsyn till det levande livet omkring oss.} Inte minst är Holmsunds tropikhus ett klockrent föremål för en Nissepediaartikel \textsc{(se Nissepedia s.~\pageref{62400dadecd90cb5cd39062abe5a3e4a})} efterdom det uppfyller alla önskvärda kriterier för en sådan: Den har en ganska rolig hemsida [http://holmsundstropikhus.se], är en lite provinsiell företeelse och har med djur att göra.
 HEAD2: Historia

 Holmsunds tropikhus låg förr på bottenplanet i gamla sporthallen i Holmstund men har sedemera flyttat till egna lokaler, varpå den gamla sporthallen brann ner (eller upp, beroende på perspektiv) \textsc{(se perspektiv s.~\pageref{1606dd19366985367d677f7b6de46e52})}. Man skriver på hemsidan (notera festligt syftningsfel) att \quotetext{på den tiden var det en uppskattad utställning med enbart uppstoppade djur som flitigt besöktes, inte minst under sommarmånaderna.}
 HEAD2: Nutid
 Nu kan dock besökare av Holmsunds tropikhus, förutom uppstoppade djur, se levande ormar och \quotetext{olika slags leddjur}. Det kostar 40 spänn för ungdomar att gå in. För vuxna vill de ha 60. Man har stängt måndag \textsc{(s.~\pageref{1086ddd192a30419d01e5c28b74cab2f})} - tisdag \textsc{(s.~\pageref{47ee958d272b159c6a0dfb024c6f9155})} och öppet lördag \textsc{(s.~\pageref{8d203c09d6ebbc3a0d797e14178798a0})} - söndag \textsc{(s.~\pageref{85b2e5c3758394a24221d1abac79191a})}. Vad som sker där emellan förtäljer inte hemsidan, vilket lätt gör en lite misstänksam och orolig.

}

\small{
\textbf{Holmund}
\label{6cd117ad7911565dfdf49640c9e5ab01}
 är ett slags mun \textsc{(s.~\pageref{6585f290ce92c3de5ff339920330e26f})} som har fått sitt namn efter Lennart Holmlund \textsc{(s.~\pageref{26d063a59c90487b11c8f5b4fa9af348})}, socialdemokratiskt kommunalråd i Umeå \textsc{(s.~\pageref{bd1e37dc477bb704c667ed1a4606df71})} kommun. Den skiljer sig från en vanlig mun i det att den ofta uttalar sig av sig själv, utan direkt koppling med hjärnans olika centra för logisk avvägning och framtidsplanering. Holmunden uttalar sig om saker den inte vet något om, säger korkade saker och förolämpar folk åt höger och vänster och blir speciellt aktiv i närheten av olika former av lokala media. Medan hjärnan under en intervjusituation processar input och skickar varningssignaler till olika delar av de kognitiva och motoriska systemen (\quotetext{Varning! Här krävs taktisk försiktighet!}) uttalar sig Holmunden vitt och brett om allt mellan himmel och jord: Whitney Houston, offentlig konst, italienska kvinnor, Håkan Juholt \textsc{(s.~\pageref{7ea89926158056e22e152ccb16d816b2})}, Ahlgrens bilar, skidlegenden Assar Rönnlund och det nuvarande beståndet av fiskmås, allt har holmunden något att uttala sig om, och inget vet den.
 HEAD2: Externa länkar
 Här kan man studera en Holmund som löper amok i etermedia.[http://www.folkbladet.nu/340242/det-var-det-korkaste-jag-har-hort?mobil]

}

\small{
\textbf{Homepage of the de Maré family}
\label{b03020dfc75184e87337722c8645bb5b}
 är en hemsida på Swipnet på World Wide Web \textsc{(s.~\pageref{3b7d657e8b7bf25a9d524b60d9bb17df})} som handlar om familjen de Maré och mycket, mycket annat. Familjen de Maré är, i motsats till vad hemsidans titel kan få en att tro, en svensk familj. Familjens rötter går enda tillbaka till 1640-talet då Jacques le Bel de Maré invandrade till vårt avlånga land och lade grunden för detta brokiga släktträd, som man kan se här [http://home.swipnet.se/de_mare/page6.html]. Några av de saker man erbjuder sidans besökare är snapsvisor, gitarrackord, limerickar, färg- och stilanalys, tankar om livet, \quotetext{gruk \textsc{(s.~\pageref{78233c5ad0b90efdffd147b849201ce4})}} (?) samt ordspråk och knasigheter. Sidan skapades av Erik och Monika, som bor i Malmö och har arbetat för Malmö kommun - Monica som informationssekreterare och Erik som stadsjurist. Fast numera är de, enligt hemsidan, glada pensionärer. Monica sysslar med färg– och stilanalys och Erik med \quotetext{diverse aktiviteter.} Båda gillar att åka på bilresa och återkommer hela tiden till Meran i Sydtyrolen i Italien, där klimatet är subtropiskt.

 Se hemsidan här: [http://home.swipnet.se/de_mare/index.html]

}

\small{
\textbf{Homi K. Bhabhas son}
\label{66ce2281df988914500cb1c269d7418f}
 Den extremt supersmarta och coola akademikern Homi K. Bhabha har en son. Sonen, Satya Bhabha, är skådespelare i Hollywood och i filmen Scott Pilgrim Vs. The World ser han ut som på bilden nedan. Nissepedias redaktion ringde upp Homie-Kay (pappa Homis alias när han agerar agent åt sin son) för att få höra det senaste om Satyas framgångar. Tyvärr ville inte Homi prata om det alls. Han ville bara prata om Mimikry.

}

\small{
\textbf{Homi k. bhabhas son}
\label{66ce2281df988914500cb1c269d7418f}


}

\small{
\textbf{Honkön}
\label{204e209b96ab0d93124f83ebe1dd4b03}
 Leif \quotetext{Honkön} Holmqvist, legendarisk transsexuell hockeymålvakt född -42. Har nästan vunnit VM. Har bland annat hånglat upp Foppa och Wayne Gretzky när ingen sett på. Förnekar detta om någon frågar, men rodnar. Har även blivit misstagen för Christina Husmark-Persson \textsc{(s.~\pageref{26162477ef162a13e8321aeecdc8259c})}.

}

\small{
\textbf{Hoppet}
\label{0bcd709c17bb88e3f7e862c4252d293f}
 är det sista som överger människan. Och är därför en svikare.

}

\small{
\textbf{Horgalåten}
\label{6514f071de56f9f33e6df4cd42b24a5d}
 är en svensk hambo utan känd upphovsmakare. Tillsammans med \textit{Idas sommarvisa}, \textit{Skala banan}, \textit{Den blomstertid nu kommer} och \textit{Hasta mañana} utgör den stommen i svensk grundskolas musikundervisning alltsedan krigsslutet. Den saknar ursprungligen sångtext men det är idag vanligt att man sjunger om hur ett gäng ungdomar i Hårga by i Hälsingland ger sig ut på dans en midsommarafton där de förtrollas av en speleman med bockfot som får dem att dansa sig till döds.


 Kebnekaise har spelat in en utmärkt version av låten på sin skiva \textit{II}.

 Iron Maiden blev så gripna av melodin att man gjorde en hel temaskiva om Horgalåten år 2003.





 Noter till Horgalåten: [http://www.folkwiki.se/Musik/1038]

}

\small{
\textbf{Hotell Gripen}
\label{b2f675b76432ab4dafb41f8b683bd35a}
 slangord för Arrest, häkte eller fängelse.

}

\small{
\textbf{Huckleberry Grylls}
\label{370888c70b3153be96b5ecc8f27e0633}
 är också \textsc{(se Marmaduke Grylls s.~\pageref{ba16bd8c7fd5e72c30e72dfda9492a8c})} son till äventyraren och överlevnadsexperten Bear Grylls.

}

\small{
\textbf{Hudiksvall}
\label{2c6ab7d529960d633a50e7a7692626dc}
 är en stad i Sverige, unik i sitt slag för att det är den enda stad i Sverige som upplevt ett regelrätt krig i modern tid.

}

\small{
\textbf{Hugo Alfvén}
\label{1e854569d1a1129571bc9d734430c8d8}
 Musiker som 1954 blev den första svensk att ge ut en stereoinspelning på skiva. Låten var \textit{Midsommarvaka} och bolaget Swedish Society Discofil. Nu vet ni.

}

\small{
\textbf{HUMlab}
\label{450ad80b62e2d6878e218e63bc52f246}
 Hmmmm.... [http://www.humlab.umu.se/]

}

\small{
\textbf{Humlan}
\label{113017afa0a9549cccc931300ba2edb3}
 är det finurliga namnet på fiket i Humanisthuset på Umeå Universitet. Här jobbar en kort tant som älskar Zlatan Ibrahimovic, samt ett antal män varav en mycket påminner om Super Mario. En kopp te/kaffe får du för 7 kronor och en helt ordinär ostmacka för en överkomlig tia \textsc{(s.~\pageref{e7292d5ba58672ce7f6fc3c0b646ab63})}. Snuset kostar 45. Besökaren kan också köpa enklare maträtter så som micro-uppvärmd lasagne, falafel i pitabröd, fetaost-foccacia och så vidare. Vid sådana tillfällen betalar du till föreståndarinnan som passar på att dra ett skämt eller två och rör dig sedan omedelbart till höger om disken för att inte störa hennes arbete. Efter en stund kommer ett avgrundsvrål i vilket din maträtt ropas ut och då gäller det att vara snabb att rusa fram och snappa upp den så att inte nästa vrål kommer precis när du anländer till disken, då detta kan orsaka permanenta hörsel- och till och med hjärnskador.
 HEAD2: Nya stolar?
 Fiket har troligen fått nya stolar.

}

\small{
\textbf{Humor}
\label{7a5b5501705f07767ec8bed743855a01}
 är lite svårt att definiera, men för enkelhetens skull kan vi säga att humor är allt som lockar till skratt. Det mesta är roligt beroende på perspektiv \textsc{(s.~\pageref{1606dd19366985367d677f7b6de46e52})} och roligast av allt är Hasseåtage [http://www.youtube.com/watch?v=1w0l9GNi0OU].

}

\small{
\textbf{Hunden beskyddar människan men vem beskyddar hunden?}
\label{2d5984a809dd83cdd14ed73163d1b026}
 \quotetext{Hunden beskyddar människan men vem beskyddar hunden?} är namnet på Tranåsrockarna Rövsvetts första demokassett från 1984. Den innehåller klassiker som \quotetext{Heroin}, \quotetext{Jehovas vittnen}, \quotetext{Kontaktproblem} och en rasande cover på \quotetext{Cadillac}. På en slovakisk blogg konstateras det att detta släpp, tillsammans med de två första sjuorna är det bästa de gjort.

 HEAD3: Konvolut och nedladdning
 Nedan kan man avnjuta omslagoch konvolut, i riktigt stor, mullig storlek. Och [http://open.spotify.com/album/3v1t2RG1GemqkTYF1rEuEW här] kan man lyssna på nästan (inte \quotetext{Cadillac}, tyvärr) alla Rövsvetts låtar de gjorde mellan 1984 och 1987, samlade på skivan \quotetext{Boll-Mats bjuder på bullkalas \& kaffe}.




 HEAD3: Låtlista
 Sidan 1

 Cadillac (Intro) - 1.51
 Jehovas vittnen - 1.08
 Snutar - 1.17
 Kontaktproblem - 1.50
 Nazzesvin - 0.59
 Knäcka dej - 1.23

 Sidan 2

 Världskrig III - 1.09
 Likna dej - 1.47
 Oskyldiga människor - 1.11
 Makt - 1.13
 Heroin - 2.05

}

\small{
\textbf{Hundkäx}
\label{11eaa9cc2162f3db2975453feb96eee7}
 är en växt som det finns fullt av i svenska ko- och fårhagar. Den växer och frodas i hela Norden med sin gröna skälk och vita blommor. Då växten är giftig är det jävligt dumt att äta den men om man prompt måste göra det rekommenderas att först koka den ca 45 minuter för att få bort det bittra ur smaken. Däremot går det utmärkt att mata kaniner och andra gnagare med den. Ser man svinstora hundkäx är det antagligen i själva verket björnloka. Hundkäx kan med fördel ingå i en midsommarstång eller som en av sju blommor att sova med under kudden. På vissa platser kallas hundkäx för hundshecks, provocerande nog.

}

\small{
\textbf{Hundra sätt att få ligga}
\label{f64dcb0fe20cd7d4b041678939fbc038}
 \textit{Hundra sätt att få ligga} är ett verk av den boklärde småskurken \textsc{(se småskurk s.~\pageref{c25031c5d78d9ad6fae8ab8f08d5e9dd})} Prof. Etienne \textsc{(se Användare: Prof. Etienne s.~\pageref{a9878d2280e5a39becac8f73d113df91})}. Som titeln antyder rör det sig om en handbok i konsten att få till det sexuellt. Prof. Etienne lägger stor vikt vid att använda rakvatten, hävda sin sociala rang genom att ljuga som en häst galopperar och samtidigt tränga sig på så mycket det bara går. Ett annat knep som förs fram är att åka till Danmark \textsc{(s.~\pageref{5331d7fd27772396f412a5b6d19bad44})} och medverka i skapandet av friluftsporr.

 HEAD2: Reaktioner på boken
 Björn Ranelid \textsc{(s.~\pageref{b374a5d86cf98cd5ba2a0ff96d5a9e97})} blev kränkt.
 Expressen var försiktigt positiva.

}

\small{
\textbf{Hur man bär iväg med femton kilo kopparskrot och kommer undan med det}
\label{c3e0853a184f697f629db9b282307b10}
 Som alla moderna människor är Prof. Etienne \textsc{(se Användare: Prof. Etienne s.~\pageref{a9878d2280e5a39becac8f73d113df91})} en stor vän av den fria marknaden. Men koppar väger ganska mycket, så femton kilo kan därför bli väldigt tungt. I sin bok \textit{Hur man bär iväg med femton kilo kopparskrot och kommer undan med det} delar professorn med sig av sina bästa tips på hur man kan tjäna en extra slant utan att riskera att bli haffad eller skada ryggen. Här delar han med sig av de bästa knepen helt gratis till Nissepedias \textsc{(se Nissepedia s.~\pageref{62400dadecd90cb5cd39062abe5a3e4a})} läsare.

 \begin{itemize}
 \item Trotta \textsc{(s.~\pageref{918b2980ffb5f16acf768fa89f71021b})} det styrande partiet i din bygd och inför ankeborgslagstiftning \textsc{(s.~\pageref{8598b83afdaf4801d1875e348df3ea53})} i kommunen.
 \end{itemize}

 \begin{itemize}
 \item Lär dig behärska härskarteknikerna. Blir du ertappad av en kvinna är det bästa sättet att uppträda som om hon inte förstår vad du håller på med. Klappa henne lätt på huvudet och förklara med babyröst att det är så här vuxna karlar jobbar.
 \end{itemize}

 \begin{itemize}
 \item Gräv ner det i närheten och rita en skattkarta som du säljer till kopparslagaren. Har du tillgång till pergament kan även arkeologer bli presumtiva köpare.
 \end{itemize}

 \begin{itemize}
 \item Kör på rainbow riders-stuket \textsc{(s.~\pageref{e31ec5cf9cada9eaab86c175a39aa3e6})}. Då vågar INGEN bråka med dig.
 \end{itemize}

 \begin{itemize}
 \item Ta med dig en läskback, en propellerkeps \textsc{(s.~\pageref{34087753e20a67ca90f6c51bcae4528e})}, en kasse lump och en gummiklubba och starta en auktion på plats. Betala dina fattigaste vänner en struntsumma för att gå runt med plakat på rygg och bröst och göra reklam så kommer det snart folk. Medan du väntar på att tillräckligt med folk ska samlas bjuder du ut lumpet i kassen.
 \end{itemize}

 På grund av att boken i Prof. Etiennes mått mätt sålde relativt bra planeras en uppföljare, \textit{Hur man är en vedervärdig medmänniska och kommer undan med det}. Publiceringsdatum har vid ett flertal gånger blivit uppskjutet på grund av våldsamma och känslostormande protester. Pär Ström har skrivit förord.

}

\small{
\textbf{Hur man klär på sig}
\label{21788ea76185e3ef4fcd025172b03791}
 Att klä på sig är något de flesta gör varje morgon då de placerar dessa ihopsydda tygstycken på sina kroppar. Vi på Nissepedia \textsc{(s.~\pageref{62400dadecd90cb5cd39062abe5a3e4a})} har full förståelse för människor som finner vardagen komplicerad, så här kommer en bra guide:

 HEAD2: Att klä på sig på underkroppen

 1. Börja med underkläderna, stoppa in fötterna i de två parallella hål som sitter nertil, genom det stora hålet som finns upptill.
 2. Stoppa din ena fot i det ena hålet, och den andra i det andra.
 3. För att skilja fram från bak är tricket \quotetext{gult fram - brunt bak}
 4. Dra upp underkläderna tills det tar stopp och känns smutt \textsc{(s.~\pageref{d9114ffee4f2dcee302ae2b19ce5eea9})}.
 5. (Är det kallt ute tar du här långkalsonger/mamelucker, det följer samma princip som ovan)
 6. Sen tar du på dig byxorna/kjolen.
 7. Har du byxor gör du som med underkläderna och eventuella långkalsonger. Skulle du i förvirring stoppa båda benen i ett byxben så är det ingen fara, alla gör fel ibland. Det är bara att börja om.
 8. Kjolen är lite knepigare, den är nämligen svårt att skilja på upp och ner. De flesta kjolar är dock större nertil, så utgå från det.
 9. Stoppa in benen i det mindre hålet.
 10. Dra upp kjolen så högt som du känner är lämpligt.
 11. Sen var det sockarna, här är det enkelt, de har nämligen bara ett hål!
 12. Stoppa in den ena foten i hålet i den ena sockan, dras tills det tar stopp.
 13. Upprepa på fot nummer två.
 14. Grattis! Du har nu en påklädd underkropp och slipper gå omkring Kalle anka \textsc{(s.~\pageref{64db68f686a0ca4d9d641061cb3fdf13})}.

 HEAD2: Att klä på sig på överkroppen

 1. Börja med att skilja på tröjans fram och baksida. Eventuella logotyper och tryck sitter ofta på framsidan.
 2. Skulle tröjan råka ha tryck på båda sidor, leta efter tvättlappen.
 3. Stoppa in huvudet i tröjans största hål. (Memorera följande punkter för här slutar du kunna läsa)
 4. Stick in den ena armen i det mindre hålet längst ut.
 5. Upprepa med andra armen.
 6. Stick in huvudet i hålet i mitten.
 7. Planerar du att ha flera lager tröjor så upprepa proceduren.
 8. Grattis! Du är nu fullt påklädd och redo att möta världen.

}

\small{
\textbf{Hur man ritar en snygg tromb}
\label{066bc35a936df08d5aeeb88f96144eaa}
 Hur ritar man en snygg tromb, är det många som undrar. Här finner du svaret.
 HEAD2: Tillvägagångssätt
 Det finns två tillvägagångssätt. Det ena lite enklare än det andra.
 Det första sättet att rita en tromb:
\begin{enumerate}
\item Tag fram papper och penna.
\item Dra ett streck över arket, ganska långt ner. Detta är horisontlinjen. Om du gör vissa upphöjningar i den kan dessa föreställa berg.
\item Börja en bit ner under linjen och rita ovaler med pennan lätt tryckt mot pappret och utan att lyfta den.
\item Arbeta dig nu långsamt uppåt och låt tromben gradvis expandera åt sidorna. Tromben ska ha formen av en trattkantarell.
\item Öva denna teknik och experimentera med olika hårt tryck på pennan och täthet mellan sträcken. Ju hårdare tryck och ju större täthet, ju tätare ser den framrusande tornadon ut att vara, men blir det för tätt och för hårda streck kan den se \quotetext{död} och \quotetext{platt ut}. När du har bemästrat denna teknik kommer du att kunna rita tromber som ser så levande ut att de tycks kunna svepa med sig allt i rummet omkring teckningen.
\end{enumerate}

 Det andra, lite mer avancerade sättet innebär mer förberedelse men är annars likt den första tekniken.
\begin{enumerate}
\item Tag fram papper och penna.
\item Börja nu med att rita föremål som har slukats upp av tromben. Kanske har den fört med sig en intet ont anande kvinnlig filmstjärna från 50-talet eller en Volvo 740? \textsc{(se Volvo 740 s.~\pageref{e262951543da05bac43c7b87235a115c})} Kanske har den rent av fått med sig träd och byggnader? Det finns nästan inga gränser för naturens krafter. Tänk på att de föremål som befinner sig längre upp i tromben bör vara mindre eftersom den normalt sett betraktas från en punkt som är bara obetydligt högre belägen än horisontlinjen.
\item När du är nöjd med föremålen, gör som ovan: Börja en bit ner under linjen och rita ovaler med pennan lätt tryckt mot pappret och utan att lyfta den.
\item Arbeta dig nu långsamt uppåt och låt tromben gradvis expandera åt sidorna. Tromben ska ha formen av en dragbasun \textsc{(s.~\pageref{0315aaaabb57a67312aa3316fd2006e1})}.
\item Öva, öva och öva. Du kanske inte blir nöjd med din första tromb, men ge inte upp! Redan efter några timmars arbete kommer du att se skillnad från de första skisserna och snart kommer du att kunna rita tromber som är så naturtrogna att de ser ut att komma rakt emot en. Och glöm inte bort att studera bilder på tromber och att dra lärdom av dessa. Alltid finns det någon ny detalj att lära sig av.
\end{enumerate}

}

\small{
\textbf{Hur man ritar ett snyggt lodjurshuvud}
\label{85f12831da9d7403326be028c34be8a9}
 Många har skrivit in till nissepedia \textsc{(s.~\pageref{62400dadecd90cb5cd39062abe5a3e4a})} och undrat hur man ritar ett snyggt lodjurshuvud \textsc{(s.~\pageref{e906cd95a540df9b16d0460fb4cf0adc})} och nissepedia-teamet hörsammar läsarnas önskningar. Här kommer en enkel, pedagogisk beskrivning hur detta går till.
 HEAD2: Förberedelser och grundstruktur
\begin{enumerate}
\item Ta fram ett bra papper (se pappersform) \textsc{(se pappersform s.~\pageref{37edcb2e533bd9c3e51f475c598b8671})} och en eller flera bra blyertspennor.
\item Rita en boll (cirkel). Detta är din prototyp till lodjurshuvudet. Om du vill kan du här använda ett runt föremål som mall, till exempel en femma \textsc{(s.~\pageref{d974e0811fe7a4d49a9062d33b66a88d})}.
\item Rita nu öron ovanpå huvudet. Öronen ritar du genom att efterlikna uppochnedvända V:n.
\item Rita djurets nos genom att göra ett litet o eller en uppochnedvänd triangel mitt i den större cirkeln.
\item Rita djurets ögon strax ovanför nosen genom att efterlikna små mandlar. Om du ritar ett stort lodjurshuvud kan du rent av använda mandlar som mallar för ögonen. OBS! Rita dem inte för högt upp! De ska nästan vara i linje med nosens högsta punkt.
\item Nu är det dags att rita djurets mun \textsc{(s.~\pageref{6585f290ce92c3de5ff339920330e26f})}. Åstadkom detta genom att rita ett lite utdraget uppochnedvänt V under djurets nos.
\end{enumerate}
 HEAD2: Detaljer
 Nu har du grunden till ditt lodjurshuvud. Nästa steg är att lägga till sådana detaljer som verkligen får lodjurshuvudet att \quotetext{leva}.
\begin{enumerate}
\item Börja med att dra horisontella \textsc{(se hårizont s.~\pageref{6840fd7c6b0803c2f77031daa8e6f801})} streck från nosen och ut mot cirkelns kanter. Detta är lodjurets morrhår.
\item Rita pupillerna i djurets ögon. Tänk på att de måste vara på samma sida om ögat så att lodjuret uppfattas som att det ser på något byte i fjärran och inte är vindögt.
\item Rita lodjurets kindhår och karaktäristiska örontofsar. Låt pennan löpa upp och ner respektive från sida till sida med lätt tryck. På så vis ser strecken ut som hår.
\item Lägg på skugga under och på ena sidan av lodjurshuvudet.
\end{enumerate}

 HEAD2: Övning, övning, övning!
 När du har utfört dessa steg har du ett klart huvud. Tänk dock på att det tar tid att bli skicklig på att rita snygga lodjurshuvuden. Du kanske inte blir nöjd med ditt första försök, men fortsätt träna! Studera bilder och logotyper på skoterkepsar \textsc{(se Butiker som bara säljer skoter-kepsar s.~\pageref{1104d57d523c5abf0a8273fff6b5fdd7})} där lodjurshuvuden förekommer och prova dig fram. Efter många timmars träning kommer du att kunna åstadkomma ett lodjurshuvud som är så skickligt utfört att det ser ut som att det kan \quotetext{kliva ut} ur teckningen vilken sekund som helst.

}

\small{
\textbf{Hurry hurry}
\label{6dd85b66907b57f2e0f1c80c850bf487}
 Kan användas som \quotetext{ja jag var nog lite hurry-hurry} , \quotetext{jag vet inte vad som händer men det verkar hurry-hurry} och är ett milt utryck för galenskap eller tokerier. I bemärkelsen galen på riktigt men kanske inte nedstämd och deprimerad  utan mer manisk, uppspelt, arg och utlevande. Det kan också vara ett uttryck för stor förvirring.
 Besläktade uttryck är curry-curry och kockobello.

}

\small{
\textbf{Husqvarna}
\label{6671b561d336f97592b06a183ea47d3e}
 tillverkar nyttomaskiner. Bland annat så kan du slippa köpa hushållsnära tjänster genom att köpa deras automagiska gräsklippare, Automower \textsc{(se Robotgräsklippare s.~\pageref{108731cb2300809a0968a111a229c3af})}. Husqvarna tillverkar också trä- och stålbearbetningsmaskiner som drivs av testosteron och han-könceller. Detta exkluderar över halva befolkningen, men Husqvarnas produkter säljer ändå som smör.

 Husqvarna är ett modernt företag. Husqvarna går att följa på Facebook, Youtube, Twitter och Flickr. Husqvarna är den enda twittrare som Annika Lantz följer (även om hon gör det i smyg). Ett jäkla häftigt företag alltså.

 Företaget stod under hård kritik när de drev sin kampanj mot helmanuella verktyg. Se Handjagare \textsc{(s.~\pageref{a85b0d5ff42c09f4605cec0188b8dd6e})}.

}

\small{
\textbf{Huvud}
\label{e906cd95a540df9b16d0460fb4cf0adc}
 et är den del av kroppen som sitter längst upp. Det är i huvudet som det mesta pågår - tankar, ätande/drickande, hörsel, balans, syn, nickmål och så vidare. I vårt mindre grannland Danmark \textsc{(s.~\pageref{5331d7fd27772396f412a5b6d19bad44})} används huvudet som närstridsvapen och kallas då \quotetext{dansk skalle.} Så blev de också ockuperade av tysken under det stora fosterländska kriget \textsc{(s.~\pageref{8e55572fc7b7490da402e43a822eb3da})}.
 HEAD2: Symbolism
 Den som vill vara i fred eller signalera till sin omgivning att han eller hon menar allvar kan med fördel låta trä upp ett antal avhuggna huvuden på pålar på balkongen eller i trädgården.

}

\small{
\textbf{Huvudduk}
\label{1361e6bca1e024f434201ad9b4512cfa}
 Att uppbära huvudduk är icke förbehållet muslimska damer \textsc{(se kvinna s.~\pageref{9a7760b2521c3471c47cd5d789a2d324})}, vilket många tycks tro. Huvudduken finns i alla upptänkliga färger och former och bärs av bikers, folk som lagt sig till med det ryska babusjka-stuket \textsc{(s.~\pageref{82b3ee065c37c75fdee61cdf1edd9705})}, Suicidal Tendencies, rappare \textsc{(se Hip-hop s.~\pageref{66c22415908267e727d3fa4a63c16672})}, sjörövare, cancerpatienter \textsc{(se kräfta s.~\pageref{31d4f9ec82e212d1a52dc283f7335710})} och tennisspelare. Huvudduken består ofta i ett slags tygstycke som läggs på hjässan och sedan binds ihop någonstans, vanligtvis i nacken. Huvudduken skyddar bäraren från att utsättas för starkt uv-ljus \textsc{(s.~\pageref{2239e04c73609ab9e8cc9b359552fa81})}, från lättare fallande föremål och människors dömande blickar (om man till exempel, likt skådespelaren och TV-personligheten Ulf Larsson, är skallig). Den som enkelt och smidigt vill skaffa sig en huvudduk kan leta fram en gammal snusnäsduk från tiden då man bar sådan tillsammans med jeansjacka, vika den som en triangel, lägga den på huvudet \textsc{(se huvud s.~\pageref{e906cd95a540df9b16d0460fb4cf0adc})} och sedan knyta samman de tre ändarna så att näsduken bildar en liten \quotetext{hätta}.

}

\small{
\textbf{Huvudlös gestalt}
\label{9aa22f15cb8c6338a648f08c3e81bcbb}
 En stämning som bara kan råda en sådan dag
 när du har intalat dig själv att det faktiskt kan ske
 Du står där som förstenad, din blick är fäst vid hedens södra hörn
 Är det din egen skräck som gör att du ser vad du ser

 Huvudlös gestalt
 En mardröm på två ben
 Gestalt
 En dagdröm blir din död

 \quotetext{Försvinn}, en panisk önskan, en bön riktad till... vem?
 Du sluter dina ögon och öppnar dem igen
 Åh nej, ännu närmare, ännu närmare för varje gång
 Hur kan, hur kan den röra sig så snabbt

 Huvudlös gestalt
 En mardröm på två ben
 Gestalt

 En dagdröm blir din död, likt för länge sen
 på platsen där du nu ligger i gräset
 1216. Dådet har etsat sig fast
 Din fortsatta existens består av scener
 som spelas om och om igen

}

\small{
\textbf{Huvudsida}
\label{f8568a1054e23bd254848c040955e2f0}
 HEAD2:  Välkommen till Nissepedia

 I denna fria databas kan du ta del av kunskap, som om den inte berikar ditt liv åtminstone inte gör det sämre. Vet du inte riktigt vilken kunskap du letar efter rekommenderar vi att du trycker på knappen \quotetext{Slumpsida} i menyn till vänster, så kommer garanterat något intressant att dyka upp. Mycket nöje!


 \textbf{NYHETER}

 \textbf{Uppdaterad version av wikin} Januari 8 2014
 De som har servrarna för Nissepedia har uppdaterat mjukvaran för wikin och den nya versionen stödjer inte vår gamla hederliga layout längre.
 Efter mycket trix har jag fått fått fram något som åtminstone liknar den gamla designen. En bra sak är att loggan nu syns i hörnet. Det gjorde den inte förr tror jag.
 Säg till om något har fått sönder. Jag har sett en massa saker som var paj som jag har fixat.
 /Potmo

 \textbf{Begränsning av nyregistreringar} Mars 28 2013
 För att registrera sig behöver man ett hemligt lösen, endast betrott liten skara.
 En så länge är det bara H-Kråkan men jag räknar kallt med att han sprider det till JPN och Ekis Erövraren.
 /Potmo

 \textbf{Captcha för registrering} Januari 5 2012
 Eftersom bottarna fortsätter att förstöra för oss så har jag lagt in \quotetext{captchas} för att registrera nya användare.
 Det innebär att när man registrerar sig så måste man svara på en enkel mattefråga. Annars får man inte vara med.
 Jag inser att detta kan anses vara en aning elitistiskt och utesluter några av våra yngre skribenter. Jag överväger ett annat plugin där man får se 12 hundar eller katter
 och måste klicka på alla hundar.
 För er som redan är registrerade skall det inte vara någon skillnad.
 Detta öppnar även för att vi i framtiden kan tillåta att man skriver här utan att vara registrerad men då måste man fylla i captchan.
 /Potmo

 \textbf{Fler admins} Oktober 16 2011
 Utan att ha rådfrågat någon av dem så har jag utnämnt Hawaii-kråkan och Jons Polare Nicke till admins. De kan nu leva rövare.
 Vi får se om de upptäcker det. Varför har jag valt dem?
 Hawaii-kråkan är första användaren och Nicke är uppenbarligen polare med min polare Jon så han kan man lita på.
 /Potmo

 \textbf{Logo III} Juni 7 2011
 Jag har försökt ändra så att vi får den nya logon men lite andra saker ändrades på samma gång.
 Jag vet inte riktigt vad jag gjort och jag är jävligt sur just nu
 Det verkar som att vissa sidor råkat bli gula och jag vet inte varför. Jag vet heller inte varför det ser lite olika ut här och var nu. Säg till om ni vill ha tillbaka så att det blir precis som förr eller om jag kan behålla det som det är
 /Potmo
 EDIT: Nej fan det blev för kasst så jag tog tillbaka det hela. Jag får försöka med logon igen i morgon. Satan!

 \textbf{Logo II} Maj 16 2011
 Snart tar vi och stänger Logotävlingen. Tyvärr har jag fått några motorer på posten så jag måste fippla med dom men snart ska jag lägga upp det vinnande bidraget.
 Glöm inte att posta förslag här  [http://www.nissepedia.com/index.php/Logof\%C3\%B6rslag]


 \textbf{Logo} Jan 30 2011
 Vi behöver en Logo. Använd den [http://www.nissepedia.com/index.php/Logof\%C3\%B6rslag] här sidan för att posta förslag

 \textbf{Facebook} Jan 30 2011
 Man kan nu like:a artiklar på facebook med den lilla knappen som syns på varje sida

 \textbf{Botattack} Xxx 00 2010
 Det har varit en bot som har fipplat med sidorna här och jag orkar inte rulla tillbaka hela tiden. Numer måste man vara en användare (alltså vara inloggad) för att redigera sidor
 Det ser ut som att jag lyckats ta bort allt skräp \textsc{(s.~\pageref{75f1a5320951ea0dd9aa3c0eaba2c2c7})} förutom diskutionssidorna och några användarsidor
 Förhoppningsvis så kan vi slå på att man kan redigera poster utan att regga sig senare. Vi vill ju vara öppna.

 \textbf{Slumpurl}  Xxx 00 2010
 Nu ska ni hardcore-nissepedianer kunna komma till slumpsidan genom att använda nissepedia.org.
 Går man till nissepedia.com så kommer man hit till förstasidan som vanligt.\textbf{

 ----

 }Enkla sätt att skapa nya sidor'''

 Alternativ 1: Skriv bara in nissepedia.com/index.php/vilken_ny_artikel_du_vill_göra

 Alternativ 2: Sök på önskad artikel och om den inte finns kan du skriva den själv genom att trycka på Skapatesttest

}

\small{
\textbf{Hybridbandoneon}
\label{24185b698b77395758cbee6eb50fd59b}
 är lättare att lära sig
 än vanligt bandoneon.

}

\small{
\textbf{Hyena}
\label{ba213b8c28962d5b00140bdc076796c6}
 [[File:TheHyena.jpg\textbarthumb\textbaralt=Puzzle globe logo\textbarWu-Tang clan poserar med hyena. Från vänster RZA, GZA, Method Man, Raekwon, Ghostface Killah, Inspectah Deck, U-God, Masta Killa, Ol' Dirty Bastard]]
 Hyenan (\textit{Hyaenidae}) är det djur som är mest crust av alla djur som man kommer på sådär på rak arm. Till utseendet ser det ut att vara ett hunddjur men tillhör egentligen grenen kattliknande rovdjur (\textit{Feliformia}). Arten delas upp i underarterna jordvarg, brun hyena, strimmig hyena och fläckig hyena. Pälsen är grov, fläckig eller strimmig och kan ibland vara försedd med små lappar som det står Conflict, Extinction of Mankind eller Discharge på. Den kan inte dra in sina klor och har en skalle som på många vis skiljer sig från liknande rovdjurs, inte minst vad gäller de grova käkarna. Ordet hyena kommer från grekiskans \textit{hýaina} som är ett nedsättande ord för svin.
 HEAD2: Föda
 Jordvargen lever uteslutande på termiter och är immun mot det gift som vissa djurarter utsöndrar - Ja, hyenan till och med uppskattar det, sägs det. De andra underarterna lever mest av att sno kadaver från andra djur. De behöver inte dricka eftersom vätskebehövet täcks av födan (ruttet blod).

}

\small{
\textbf{Hyperhidros}
\label{fa707ad681c717c82885405791ef88aa}
 \textit{Hyperhidros} är den latinska termen för kraftiga svettningar. Detta symptom orsakas av sådant som feber, fetma, förvirring \textsc{(s.~\pageref{c502a6223b16f730a8900c12f2b10fec})}, sömnbrist, korpfotboll och ansträngning \textsc{(se göra rätt för sig s.~\pageref{c8c01e0e8b4ad8e5ff6011b8af6405a5})}. I Sverige \textsc{(s.~\pageref{b1999637949ed135b2ca03f3a38460cc})} är hyperhibros främst associerat med Adde Malmberg \textsc{(s.~\pageref{1390facdddaee5ed00a964fbe93b30b9})} som uppvisar samtliga av ovanstående åkommor, och mer därtill.

}

\small{
\textbf{Hysterika}
\label{72cb251805523a222408d28bcd0d4955}
 En hysterika är en kvinna \textsc{(s.~\pageref{9a7760b2521c3471c47cd5d789a2d324})} som ofta går bananas \textsc{(s.~\pageref{ec121ff80513ae58ed478d5c5787075b})} över vad som uppfattas som struntgrejer av omgivningen (omgivningen utgörs då ofta av män). Empiriska studier visar att hysterikor i nio fall av tio är roligare och ballare än snälla tjejer \textsc{(s.~\pageref{0aa3c8d228095f6fc73eccb8c92b8c81})}.

}

\small{
\textbf{Hytta med näven}
\label{dabb9466fffc72b8eec1d4616f32d62e}
 Att \textbf{hytta med näven} är ett klassiskt tecken för att visa missnöje. Blir du omkörd av en sportbil? Hytt med näven! Kallar snorungar dig för dinosauriefossil? Hytt med näven! Har bolaget höjt priset på Kir? \textsc{(se Kir s.~\pageref{002e1a6e54da86cabc77fbb474c2df49})} Hytt med näven! Hejar din måg \textsc{(s.~\pageref{55752d6920060b54fd689faee4ed037b})} på Djurgården? Hytt med näven! Säger någon att den nya sångaren \textsc{(s.~\pageref{2e55dbe6a48745ced354e0dd04dd4b80})} ät bättre än Bon Scott? Hytt med näven!

 Det finns alltid en anledning att hytta med näven!

 Att hytta med näven ska inte förväxlas med att veva med kängnäven \textsc{(s.~\pageref{0b5f330433e1fc19a412718dba802627})}.


 Källa: Prof. Etienne \textsc{(se Användare: Prof. Etienne s.~\pageref{a9878d2280e5a39becac8f73d113df91})} - \textit{Barnagans förträffliga pedagogik}. Rabén\&Sjögren, Uppsala 1998.

}

\small{
\textbf{Hyvel}
\label{1668e2298ba60f14922e2cca6aa96538}
 Finns både kort, lång och osthyvel \textsc{(s.~\pageref{c09f8306965e1344e1102a46d084cab9})}.
 Se även rubank \textsc{(s.~\pageref{b1c373a9ae319af9e1bf15a62fdf85cf})} och putshyvel \textsc{(s.~\pageref{82aace730b3087db7cfc8b4ed5d7dae0})}.

}

\small{
\textbf{Háfrónska}
\label{c60b58326f9cf3e7f8b02e1878c464f0}
 , även kallat högisländska, är en konstgjord form av isländska \textsc{(se Island s.~\pageref{3dbacfd76b0a040ccad1eacb20def4c8})} som har som mål att vara helt fri från utländskt inflytande i form av låneord, böjningar etc. ”Det är väl någon stollig ordförande i Vatnajökulls hembygdsförening som hittat på det här”, tänker ni säkert nu. Men i själva verket är det den inte alls särskilt isländske men väldigt stofila belgaren \textsc{(se Belgien s.~\pageref{f79ffe9e826a19f9f6a446c90e21c4e3})} Jozef Braekmans som satt ensam på sin kammare i typ tio år och snickrade på detta. Att Braekmans både är belgare och vurmar för att rena det nordiska kulturarvet gör naturligtvis att Nissepedias \textsc{(se Nissepedia s.~\pageref{62400dadecd90cb5cd39062abe5a3e4a})} läsare, som vi alla vet till största del består av kulturmarxister, genast börjar associera honom med nyliberalism \textsc{(s.~\pageref{a562ace16486d966be4513ea22aee287})} och glima \textsc{(s.~\pageref{b46db05ac9d1b0bb2019c68b73335729})}. Om detta är sant eller inte har vi egentligen ingen aning, men flaggan som Braekmans valt till symbol för sitt projekt minskar inte direkt misstanken.

}

\small{
\textbf{Häcklefjäll}
\label{40a7322a2ef5adb9efd69969d8f28f1e}
 Svenskt namn på den isländska vulkanen \textit{Hekla}, vars krater enligt nordisk folktro är porten till helvetet. Det låter rätt larvigt, men är alltså egentligen ganska ball.

}

\small{
\textbf{Hällefors}
\label{e144fd5ba5ee4d7c395f18c9b1a4cd1f}
 är en kommun i nordvästra Västmanland. Den är framförallt känd för att ha gett upphov till hälleforskavajen, ibland felaktigt benämnd Helly Hansen-jacka.

}

\small{
\textbf{Hänga på låset}
\label{92458169b7b1c64adcfb1ccced81c249}
 När någon suktar att inhandla något, eller befinna sig, på en plats som har en låst dörr och etablissemanget inte riktigt hunnit öppna så har konsumenten bara ett val, att hänga på låset. Denna praktik är socialt accepterad på t.ex. Konsum \textsc{(se Konsumbutik s.~\pageref{70e4875f7c2c177596305006e46b7ca9})}, kombinationsaffärer \textsc{(s.~\pageref{0a2777bf1366a8a9a5b8eab9ca1496a1})} och butiker som bara säljer skoter-kepsar \textsc{(s.~\pageref{1104d57d523c5abf0a8273fff6b5fdd7})} men brukar ses snett på vid anrättningar som serverar eller säljer alkohol \textsc{(s.~\pageref{11c589cba1a208e0359048a78e6b88b8})}, t.ex. Rött \textsc{(s.~\pageref{dacd03b85a85d8c8b67c702e1872c498})} och Systembolaget. Det är synnerligen populärt vid reor.

 HEAD2: Etmyologi
 Idag utförs sällan praktiken till fullo, dvs de flesta står bara och slår dank \textsc{(s.~\pageref{eee1edb16ac8987af66023852db6c513})} utanför inrättningen. I uttryckets begynnelse var det annorlunda. Under ransoneringens dagar flödade inte brännvinet så som det gör idag, utan tillgången var strikt begränsad. För att få ut det mesta av detta så började bonddrängar att dricka upp sin sprit kvällen innan nästa ransonering började gälla och gick därför ner till handlaren och somnade, hängandes på låset, i väntan på att få ut mer brännvin \textsc{(s.~\pageref{ff49ececa32cff978496a39635496f46})} och kunna fortsätta bläckan.

}

\small{
\textbf{Härja}
\label{3e3f117bf13ea6ac00fdfc6b183c6146}
 Ibland är saker och ting inte som dom ska. Det finns så mycket hejåhå \textsc{(s.~\pageref{872fb693de8a0eb88c6374ea0343b1c2})}. Härja!

}

\small{
\textbf{Häst}
\label{b4c608370b339da095c5f8db7fab0945}
 Gulligt djur med hovar och mjuk och luddig päls, gillar att äta hö och att reta andra djur.

}

\small{
\textbf{Hästhandlarplånbok}
\label{2f8fbda5296f2f6cab04d88082ed9015}
 En hästhandlarplånbok är en valutaförvaringsaccessoar \textsc{(se valuta s.~\pageref{cf1e2a0af4955aa7539b6e12e9d282e6})} av det rejälare slaget. Ett typiskt exemplar är fullt till bristningsgränsen med visitkort \textsc{(s.~\pageref{07901352e2cfb4976fc023413c86711e})}, medlemskort, lånekort, bankomatkort, bilder på tjejer i kortkort, körkort, plåster, fiskelina, snabbmatsförpackningar med salt och peppar,  lite blandade diversesaker och fyrtiotusen miljarder \textsc{(s.~\pageref{c2160bffc9c5ca88e77204672e62e489})} kvitton. Namnet kommer sig av att det främst är hästhandlare och andra skojare som begagnar sig av den.? Nej, det är en helt vanlig hästhandlarplånbok.]]
 
 Tanken är att när kunden återvänder med länsman, rosenrasande över att den präktiga draghästen han trott sig köpa visat sig vara en blind mulåsna, ska hästhandlaren kunna blåneka till att något avtalsbrott begåtts. Hästhandlaren tar upp sin hästhandlarplånbok som med sin väldighet inger respekt och visar att handlaren minsann har alla sina papper \textsc{(se viktiga papper s.~\pageref{810193ff4e7ae05223a81e960d806ddf})} i ordning. Han gräver runt en stund i börsen och säger sedan självsäkert att han inte hittar något kvitto över affären. Detta förklarar han med att hästskrället var så eländigt malätet att han inte brydde sig om att ta betalt utan faktiskt gav bort den till nya ägaren av ren vänlighet.

 Vill man göra sin egen hästhandlarplånbok tar det ungefär fem år att samla på sig tillräckligt mycket trovärdigt skräp \textsc{(s.~\pageref{75f1a5320951ea0dd9aa3c0eaba2c2c7})} att stoppa i den.

}

\small{
\textbf{Hästkista}
\label{d2833489548c728566cda8554195c199}
 Innehav av hästkista berättigar en person (ibland med sällskap) till företräde i kön till det etablissemang där hästkistan är utfärdad. Systemet infördes på begäran av väktarfacket när det blev förbjudet att ras-, köns- och åldersdiskriminera. Det finns inga glasklara regler för hur man går till väga för att erhålla hästkista, men vissa sätt har större utdelning är andra. Det enklaste sättet är att på ett diskret vis hinta om att man kan tänka sig att ligga med någon i bandet. Vill man inte det kan man istället ligga med kulturstockholm \textsc{(s.~\pageref{fdb13bcf260377fba1d947da7c739223})}, det är lite krångligare men fungerar i slutändan lika bra. Båda dessa knep kräver dock ett visst förarbete, vilket man inte alltid har tid med eftersom det är förfest. Det bästa knep som återstår då är att helt sonika hävda att man står på hästkistan fastän man inte gör det. Stega kavat fram till vakten, morsa och säg \quotetext{Tjena chefen! Martin Lundvall, jag står på hästkistan}. Det är inte helt säkert att namnet du drar till med finns med, så nu gäller det att ta det kallt. Skulle det saknas måste du nu på ett självsäkert sätt kräva att få prata med någon i bandet för att reda ut detta missförstånd. Bandmedlemmar tjänar ändå aldrig några pengar på entré så får du tag på en kommer hon eller han med stor sannolikhet att legitimera din scam. Går inte det återstår bara att ta sats och springa. Göm dig i en folksamling, vänd jackan ut och in och hoppas på det bästa. Samtliga knep fungerar bästa om man är lite full \textsc{(se vältagravstensfull s.~\pageref{38b98456134208680c37fb10c911ee6a})}.

}

\small{
\textbf{Håbroa}
\label{628d44b3823f1b84e256a202243f1a68}
 är en bro över sjön Gäsen utanför Kärrgruvan. När man fiskar vid Håbroa får man alltid napp. Brogrunden utgörs av två betongfundament som fixerar två stålbalkar över sjön. Stålbalkarna täcks på ovansidan av brädor som gör bron utnyttjbar för både fotgängare, bilar och mopeder. Vid normalt vattenstånd går det att ro med en vanlig eka under bron, så länge man duckar för stålbalkarna. Bron är varken någon arkitektorisk skönhet eller milstolpe i ingenjörskonst men är väl värd ett besök ändå.


 Källa: Henning Mankell - Hundarna i Riga.

}

\small{
\textbf{Håkan Juholt}
\label{7ea89926158056e22e152ccb16d816b2}
 hade en god chans att bli Sveriges \textsc{(se Sverige s.~\pageref{b1999637949ed135b2ca03f3a38460cc})} nästa stadsminister som partiledare för Socialdemokratiska Arbetarpartiet (sic!). När detta tillkännagavs jublade Sveriges samlade satirtecknare och karikatyrmålare, och firade sedan i dagarna tre. Idag är dock Juholt inte längre partiledare på grund av att han åt den sista tårtbiten på Sossarnas kongress och blev sedan systematiskt utmobbad med stor hjälp av massmedia.

 medarbetare.]]

 HEAD2:  Goda skäl som talade för Juholt som statsminister
 \begin{itemize}
 \item Karln har mustach
 \item Han är typ inte höger
 \item Hans favoritfågel är pingvin \textsc{(s.~\pageref{a5c3190fd8fc0a6cbf0cb645b8add9d0})}
 \end{itemize}

}

\small{
\textbf{Håkan Uvholt}
\label{90861b90e6e0e41e144d313703eb453e}
 hade varit statsminister i en bättre värld. I den bästa av världar finns ingen statsminister.

}

\small{
\textbf{Hålkort}
\label{e361854e951ea42015b029ab30581844}
 . Översatt till svenska lyder texten: \textit{\quotetext{Nissepedia} a.(rtikel) Pudaslåda}]] Hålkort används inom informationsteknologin för att hantera stora mängder information. Det uppfanns av IBM som sålde hålkort och tillhörande maskiner till Nazityskland, som använde dessa i logistiken kring koncentrations- och fångläger för att hålla reda på vilka fångar som förts var och vilka som var vid liv respektive döda. Idag används hålkort i huvudsak inom den civila sektorn. Uppkomsten av hålkortsnätet, som tillåter hålkortsoperatörer från skilda delar av jorden att utbyta information med varandra, har radikalt förändrat förutsättningarna för denna informationstekniks fortlevnad och har gjort att den idag är en av det franska Minitel \textsc{(se videotex s.~\pageref{4dbd2386bf7ea2190fb4d03e6efb4775})} stora utmanare.

}

\small{
\textbf{Hårdrock}
\label{a4566a810e7ad85a57ddc84083a8139b}
 är ett slags sjukt rå rock för den lite tuffare publiken, men det är mycket mer än så. Det är vänskapsband som sträcker sig över kontinentalplattor och en lukrativ marknad för de som broderar backpatches. Hårdrockaren är en person som går sin egen väg och inte låter sig kuvas av samhällets normer och förväntningar. Medan samhället talar arbetslinjen \textsc{(s.~\pageref{a7d4c1873c9542a1c6a48a1e52bdb823})}, amorteringar och anställningsbarhet bara garvar hårdrockaren, bär jeansshorts och lyssnar på Saxon \textsc{(se Anglosax s.~\pageref{75591674b0deca83291ccfef6f4f557c})}. Detta särskiljer hårdrockaren från andra invånare i den semirurala bruksort där hårdrockaren bor. Men den kanske tydligaste sak som skiljer hårdrockaren från vanliga människor är de picturediscs som hårdrockaren lägger sina surt förvärvade pengar på. För den genomsnittliga människan är en picturedisc en lite fånig och ofta betydligt sämre skiva i jämförelse med normala plattor, men inte för hårdrockaren, som sätter värde på att kunna ta fram skivan och njuta av den genom att titta på den coola och lite skrämmande bilden.
 HEAD2: Hårdrockare i utbildningsvärlden
 Hårdrockaren är i regel en bra student i de flesta akademiska ämnen eftersom hårdrockaren letat sig till akademin pga sitt intresse för medeltiden, nordiska språk osv och har skaffat sig språkkunskaper genom att läsa liner notes och analysera låttexter medan andra studenter sprungit på disco och fjantat sig. Paradoxalt nog är det som hindrar hårdrockaren från en akademisk karriär det faktum att hen är upptagen just med att läsa liner notes, analysera låttexter, titta på picturediscs och klippa av ärmar och ben på div denimplagg.
 HEAD2: Trivia
 Ingen hårdrockare har någonsin kallat AC/DC för AC/DC, Judas Priest för Judas Priest, Black Sabbath för Black Sabbath eller Iron Maiden för Iron Maiden - för hårdrockaren är det DC, Priest, Sabbath och Maiden som gäller.

}

\small{
\textbf{Hårdrockare med gomspalt}
\label{0be312858534ef68f6f1b96d0a28d2fd}
 En hårdrockare med gomspalt är en hårdrockare vars gom inte vuxit ihop på normalt vis under graviditetsvecka 10-12. Hårdrockarens gomspalt kan omfatta både den mjuka och den hårda gommen, bara den mjuka gommen eller muskulaturen i denna (så kallad Submuskulös gomspalt, förkortat SMG). Tudelad gomspene förekommer också i sällsynta fall. Den moderna plastkirurgin är idag så avancerad att den i de flesta fall, ibland med hjälp av logoped, kan förhindra nedsatt talutveckling så väl som synliga ärrbildningar hos hårdrockaren. I svårare fall kan hårdrockaren få en protetisk obturator, vilket är en anordning av härdat silikon och titanium som stänger spalten mellan de nasala och orala kaviteterna. Även en Lathmansoperation kan vara nödvändig. En Lathmanordning opereras då in i munnen \textsc{(se mun s.~\pageref{6585f290ce92c3de5ff339920330e26f})} då hårdrockaren är i fyra- \textsc{(se fyra s.~\pageref{7bdb5385ce8e0b1cbc7c15b1d71e8e7d})} eller femårsåldern och fixerar de två delarna av gommen så att de slutligen växer samman, varvid anordningen kan avlägsnas.

}

\small{
\textbf{Hårdrockare och vitaminer}
\label{67e2a9b9917f6f8da2dceb3873ff10f2}
 För att hårdrockaren ska må bra och kunna göra sin grej är det viktigt att hårdrockaren har ett bra vitaminintag. Vitaminer är livsnödvändiga organiska ämnen; en del är sådana som hårdrockarens kropp inte kan bilda själv utan måste intagas genom hårdrockarens föda. Brist på något vitamin ger upphov till specifika bristsjukdomar hos hårdrockaren, som därmed kan missa  Motörhead i ishallen i Gävle. Vitaminer kan delas in i två olika typer, vattenlösliga och fettlösliga, där överskottet av de fettlösliga kan lagras i fettet i hårdrockarens kropp medan de vattenlösliga försvinner med hårdrockarens urin \textsc{(s.~\pageref{524fd7acb94f9c2d879b5c1cf8335669})}. Hårdrockaren bör få i sig rekommenderade nivåer av:

 \begin{itemize}
 \item a-vitamin
 \item b-vitamin
 \item c-vitamin
 \item d-vitamin
 \item e-vitamin
 \end{itemize}

 inklusive alla undergrupper till dessa. Det viktigaste för hårdrockaren är att ha ett varierat näringsintag och att äta minst en frukt \textsc{(s.~\pageref{7b0faed51fc6c55d2431ed677d0989ad})} om dagen, gärna två. Vitamintillskott kan också vara aktuellt, speciellt om hårdrockaren har en specialdiet eller inte får så mycket solljus, kanske för att hårdrockaren satt upp en sån där kombinerad flagga och affisch med omslaget till \textit{Powerslave} över fönstret. Sådana preparat kan dock vara dyra och innehåller är inte alltid ämnesnivåer som passar den enskilda hårdrockaren. Det är därför bra om hårdrockaren konsulterar sin läkare eller dietist, som kan hjälpa hårdrockaren att finna det tillskott som funkar bäst.

 HEAD2: Rekommenderat vitaminintag
 Hårdrockaren bör dagligen få i sig följande vitaminer:

 \textlessp\textgreater\&nbsp;\textless/p\textgreater
 \textlesstable cellspacing=\quotetext{3} cellpadding=\quotetext{4} width=\quotetext{600} border=\quotetext{0} style=\quotetext{background-color:\#fbe4eb;border-style:none;width:100\%;}\textgreater
 
 \textlesstr\textgreater
 \textlesstd style=\quotetext{border-top-color:\#ffffff;border-right-color:\#ffffff;border-left-color:\#ffffff;border-top-style:solid;border-right-style:solid;border-left-style:solid;border-top-width:1px;border-right-width:1px;border-left-width:1px;vertical-align:top;}\textgreater\textlessspan style=\quotetext{font-family:Trebuchet MS;}\textgreater\textlessstrong\textgreater\textlessspan style=\quotetext{font-size:10pt;}\textgreaterHårdrockarens \&nbsp;fettl\&ouml;sliga\&nbsp; vitaminer\textless/span\textgreater\textless/strong\textgreater\textless/span\textgreater\textless/td\textgreater

 \textlesstd style=\quotetext{border-top-color:\#ffffff;border-right-color:\#ffffff;border-top-style:solid;border-right-style:solid;border-top-width:1px;border-right-width:1px;vertical-align:top;}\textgreater\textlessspan style=\quotetext{font-family:Trebuchet MS;}\textgreater\textlessstrong\textgreater\textlessspan style=\quotetext{font-size:10pt;}\textgreaterRDI-\&nbsp;\&nbsp;\&nbsp;\&nbsp; v\&auml;rden\&nbsp;\&nbsp;\textless/span\textgreater\textless/strong\textgreater\textless/span\textgreater\textless/td\textgreater
 \textlesstd style=\quotetext{border-top-color:\#ffffff;border-right-color:\#ffffff;border-top-style:solid;border-right-style:solid;border-top-width:1px;border-right-width:1px;}\textgreater\textlessspan style=\quotetext{font-family:Trebuchet MS;}\textgreater\textlessstrong\textgreater\textlessspan style=\quotetext{font-size:10pt;}\textgreaterODI-v\&auml;rden\textless/span\textgreater\textless/strong\textgreater\textless/span\textgreater\textless/td\textgreater
 \textlesstd style=\quotetext{border-top-color:\#ffffff;border-right-color:\#ffffff;border-top-style:solid;border-right-style:solid;border-top-width:1px;border-right-width:1px;}\textgreater\textlessspan style=\quotetext{font-family:Trebuchet MS;}\textgreater\textlessstrong\textgreater\textlessspan style=\quotetext{font-size:10pt;}\textgreaterTerapeutiska v\&auml;rden\textless/span\textgreater\textless/strong\textgreater\textless/span\textgreater\textless/td\textgreater
 \textless/tr\textgreater
 \textlesstr\textgreater

 \textlesstd style=\quotetext{border-top-color:\#ffffff;border-right-color:\#ffffff;border-left-color:\#ffffff;border-top-style:solid;border-right-style:solid;border-left-style:solid;border-top-width:1px;border-right-width:1px;border-left-width:1px;vertical-align:top;}\textgreater\textlessspan style=\quotetext{font-family:Trebuchet MS;}\textgreater\textlessspan style=\quotetext{font-size:10pt;}\textgreaterA-vitamin\textless/span\textgreater\textless/span\textgreater\textless/td\textgreater
 \textlesstd style=\quotetext{border-top-color:\#ffffff;border-right-color:\#ffffff;border-top-style:solid;border-right-style:solid;border-top-width:1px;border-right-width:1px;vertical-align:top;}\textgreater\textlessspan style=\quotetext{font-family:Trebuchet MS;}\textgreater\textlessspan style=\quotetext{font-size:10pt;}\textgreater0,8 mg = 2664 I.E\textless/span\textgreater\textless/span\textgreater\textless/td\textgreater
 \textlesstd style=\quotetext{border-top-color:\#ffffff;border-right-color:\#ffffff;border-top-style:solid;border-right-style:solid;border-top-width:1px;border-right-width:1px;}\textgreater\textlessspan style=\quotetext{font-family:Trebuchet MS;}\textgreater\textlessspan style=\quotetext{font-size:10pt;}\textgreater10 000 - 50 000 I.E\textless/span\textgreater\textless/span\textgreater\textless/td\textgreater
 \textlesstd style=\quotetext{border-top-color:\#ffffff;border-right-color:\#ffffff;border-top-style:solid;border-right-style:solid;border-top-width:1px;border-right-width:1px;}\textgreater\textlessspan style=\quotetext{font-family:Trebuchet MS;}\textgreater\textlessspan style=\quotetext{font-size:10pt;}\textgreater10 000 -\&nbsp;100 000 I.E\textless/span\textgreater\textless/span\textgreater\textless/td\textgreater
 \textless/tr\textgreater
 \textlesstr\textgreater

 \textlesstd style=\quotetext{border-top-color:\#ffffff;border-right-color:\#ffffff;border-left-color:\#ffffff;border-top-style:solid;border-right-style:solid;border-left-style:solid;border-top-width:1px;border-right-width:1px;border-left-width:1px;vertical-align:top;}\textgreater\textlessspan style=\quotetext{font-family:Trebuchet MS;}\textgreater\textlessspan style=\quotetext{font-size:10pt;}\textgreaterD-vitamin\textless/span\textgreater\textless/span\textgreater\textless/td\textgreater
 \textlesstd style=\quotetext{border-top-color:\#ffffff;border-right-color:\#ffffff;border-top-style:solid;border-right-style:solid;border-top-width:1px;border-right-width:1px;vertical-align:top;}\textgreater\textlessspan style=\quotetext{font-family:Trebuchet MS;}\textgreater\textlessspan style=\quotetext{font-size:10pt;}\textgreater10 \&micro;g = 400 I.E\textless/span\textgreater\textless/span\textgreater\textless/td\textgreater
 \textlesstd style=\quotetext{border-top-color:\#ffffff;border-right-color:\#ffffff;border-top-style:solid;border-right-style:solid;border-top-width:1px;border-right-width:1px;}\textgreater\textlessspan style=\quotetext{font-family:Trebuchet MS;}\textgreater\textlessspan style=\quotetext{font-size:10pt;}\textgreater400 - 600 I.E\textless/span\textgreater\textless/span\textgreater\textless/td\textgreater
 \textlesstd style=\quotetext{border-top-color:\#ffffff;border-right-color:\#ffffff;border-top-style:solid;border-right-style:solid;border-top-width:1px;border-right-width:1px;}\textgreater\textlessspan style=\quotetext{font-family:Trebuchet MS;}\textgreater\textlessspan style=\quotetext{font-size:10pt;}\textgreater400 -\&nbsp;2800 I.E\textless/span\textgreater\textless/span\textgreater\textless/td\textgreater
 \textless/tr\textgreater
 \textlesstr\textgreater

 \textlesstd style=\quotetext{border-top-color:\#ffffff;border-right-color:\#ffffff;border-left-color:\#ffffff;border-top-style:solid;border-right-style:solid;border-left-style:solid;border-top-width:1px;border-right-width:1px;border-left-width:1px;vertical-align:top;}\textgreater\textlessspan style=\quotetext{font-family:Trebuchet MS;}\textgreater\textlessspan style=\quotetext{font-size:10pt;}\textgreaterE-vitamin\textless/span\textgreater\textless/span\textgreater\textless/td\textgreater
 \textlesstd style=\quotetext{border-top-color:\#ffffff;border-right-color:\#ffffff;border-top-style:solid;border-right-style:solid;border-top-width:1px;border-right-width:1px;vertical-align:top;}\textgreater\textlessspan style=\quotetext{font-family:Trebuchet MS;}\textgreater\textlessspan style=\quotetext{font-size:10pt;}\textgreater10 mg = 100 I.E\textless/span\textgreater\textless/span\textgreater\textless/td\textgreater
 \textlesstd style=\quotetext{border-top-color:\#ffffff;border-right-color:\#ffffff;border-top-style:solid;border-right-style:solid;border-top-width:1px;border-right-width:1px;}\textgreater\textlessspan style=\quotetext{font-family:Trebuchet MS;}\textgreater\textlessspan style=\quotetext{font-size:10pt;}\textgreater200 - 800 I.E\textless/span\textgreater\textless/span\textgreater\textless/td\textgreater
 \textlesstd style=\quotetext{border-top-color:\#ffffff;border-right-color:\#ffffff;border-top-style:solid;border-right-style:solid;border-top-width:1px;border-right-width:1px;}\textgreater\textlessspan style=\quotetext{font-family:Trebuchet MS;}\textgreater\textlessspan style=\quotetext{font-size:10pt;}\textgreater100 - 2000 I.E\textless/span\textgreater\textless/span\textgreater\textless/td\textgreater
 \textless/tr\textgreater
 \textlesstr\textgreater
 \textlesstd style=\quotetext{border-top-color:\#ffffff;border-right-color:\#ffffff;border-left-color:\#ffffff;border-top-style:solid;border-right-style:solid;border-left-style:solid;border-top-width:1px;border-right-width:1px;border-left-width:1px;vertical-align:top;}\textgreater\textlessspan style=\quotetext{font-family:Trebuchet MS;}\textgreater\textlessspan style=\quotetext{font-size:10pt;}\textgreaterK-vitamin\textless/span\textgreater\textless/span\textgreater\textless/td\textgreater

 \textlesstd style=\quotetext{border-top-color:\#ffffff;border-right-color:\#ffffff;border-top-style:solid;border-right-style:solid;border-top-width:1px;border-right-width:1px;vertical-align:top;}\textgreater\textlessspan style=\quotetext{font-family:Trebuchet MS;}\textgreater\textlessspan style=\quotetext{font-size:10pt;}\textgreater80\&nbsp;\&micro;g\textless/span\textgreater\textless/span\textgreater\textless/td\textgreater
 \textlesstd style=\quotetext{border-top-color:\#ffffff;border-right-color:\#ffffff;border-top-style:solid;border-right-style:solid;border-top-width:1px;border-right-width:1px;}\textgreater\textlessspan style=\quotetext{font-family:Trebuchet MS;}\textgreater\textlessspan style=\quotetext{font-size:10pt;}\textgreater65 - 85\&nbsp;\&micro;g\textless/span\textgreater\textless/span\textgreater\textless/td\textgreater
 \textlesstd style=\quotetext{border-top-color:\#ffffff;border-right-color:\#ffffff;border-top-style:solid;border-right-style:solid;border-top-width:1px;border-right-width:1px;}\textgreater\textlessspan style=\quotetext{font-family:Trebuchet MS;}\textgreater\textlessspan style=\quotetext{font-size:10pt;}\textgreater30\&nbsp;-\&nbsp;1600\&nbsp;\&micro;g\textless/span\textgreater\textless/span\textgreater\textless/td\textgreater
 \textless/tr\textgreater
 \textless/tbody\textgreater

}

\small{
\textbf{Hårdrockism}
\label{d685c148ce01867274a363fb51082a7d}
 är ett samlingsnamn för åsikter, beteenden och politiska uttryck som på ett eller annat vis diskriminerar eller har för avsikt att skada hårdrockare. Extrema grupper av hårdrockister finns i vårt samhälle idag, men de är ganska få och egentligen inte den stora faran för hårdrockaren \textsc{(se hårdrock s.~\pageref{a4566a810e7ad85a57ddc84083a8139b})} - det är vardagshårdrockismen som är det verkliga hotet mot vårt demokratiska samhälle: När hårdrockaren går in på H\&M får hen veta från en kylig expedit att alla jeansvästar plötsligt tagit slut både på lagret och hos leverantören. När hårdrockaren makar sig fram till DJ-båset på Berns och vänligt ber att \textit{South of Heaven} eller \textit{Wolverine Blues} spelas på helgvolym \textsc{(s.~\pageref{3539fdeb41a5b216f614b6ced9ff5cff})} blir hen bryskt avvisad och inte sällan hånad. Hårdrockaren kanske kommer till Emmabodafestivalen och får sig en kalldusch när det visar sig att inte ett enda av alla band på festivalen spelar dieseldoftande thrash metal. Exemplen är oräkneliga.
 HEAD2: Anti-hårdrockism
 Den enda riktigt framgångsrika sättet att bekämpa hårdrockismens fula tryne/månghövdade hydra är, förutom att möta människor ansikte mot ansikte och sprida kunskap, den amalgamerande gemenskap och enighet som hårdrocksvärlden erbjuder. I den finner många styrkan att mot alla odds fortsätta åka på Whitesnake på Gavliahallen, bära jeanshorts i oktober och köpa repade picturedics för ohemula \textsc{(se ohemul s.~\pageref{91b8873590abd15ec344c2ba93d015cd})} summor.

}

\small{
\textbf{Hårdrocksholk}
\label{94ca04c333903f31bb9186478c3d8945}
 En hårdrocksholk är ett arkitektoniskt koncept av trä som är avsätt för inackordering av fjäderfän och har en viss anknytning till en rå subkultur där män helt klart är överrepresenterade. Den består av en lådformad kropp med ett sadeltak och ett hål som tjänar som dörr. Till yttermera visso ska hårdrocksholken ha en pinne avsett för förenkling av den inneboendes in- och utträde. Ofta placeras den i ett träd i närhetan av ett bostadshus så att man kan beskåda livets gång i naturen medan man äter sin frukost. Det är viktigt att man årligen ser till att holken sitter stadigt fast och att den inte anfrätes innifrån, vilket kan leda till att en hel årskull går om intet. Hårdrocksholken bör också vara ljudisolerad med ett par meter liggunderlag tätt rullad kring själva hus-godset så att andra vilda djur inte störs av det aldrig avstannande dånet av \textit{No Sleep 'til Hammersmith} som tenderar att komma ut från holken, då många ekotoper kan ta stor skada av extremt ösig ljudpåverkan.

}

\small{
\textbf{Hårizont}
\label{6840fd7c6b0803c2f77031daa8e6f801}
 är (tillsammans med Studio 88 \textsc{(se åttioåtta s.~\pageref{accedee29196c933bdfee548d120a3c0})} Umeås mest dubiöst döpta affärsverksamhet. Butiken är en frisersalong, vilket förklarar förekomsten av \quotetext{hår} i namnet. Det är efter det man blir riktigt fundersam. Hår i vad? Syftar namnet på horisonten? I såna fall varför? Blir man klippt utsträckt i horisontalläge? Eller kan man bara få platta frisyrer som liksom smeker horisonten? Om det inte är horisonten som åsyftas, är \quotetext{zont} istället en lite tuffare stavning av sånt? Vad är då sånt? Vatten? Hårgelé? Uvsmör? \textsc{(se Uvsmör s.~\pageref{240a6e2f1169dc87b9533f6b9c7b0aec})} Det hela är väldigt oklart, vilket inte har hindrat Hårizont från att etablera tre (3!) boutiquer i Umeå.

}

\small{
\textbf{Högtalartips}
\label{67d1cdf9ebf847fa5430e998da2b7085}
 Om du i vänskapskretsen har en person som ser sig själv som högtalarexpert och du bjuder in denna person (som i nittionio fall av hundra är en man) kan du få högtalartips. När din bekante kliver in i lägenheten eller stugan eller vad du nu bor i säger han att du har högtalarna på fel ställe, skulle få fetare bas om du bara bytte element, att du måste ha bättre kablar och nya slutsteg. Om du gör som han säger, bedyrar han ihärdigt, kommer allt bli så bra, så bra. Högtalarexperten känns igen på att han äger högst fyra skivor och att ungefär hälften av dessa (2) är samlingsskivor.

}

\small{
\textbf{Hökklo och Dolken}
\label{6cd77f9a37b6f1bbf39abffa383e2b9e}
 är ett enigmatiskt konstnärspar från Umeå som likt sin kollega Banksy ännu inte röjt sina identiteter för omvärlden, trots enträgna övertalningsförsök från konstvärlden. Också likt Banksy använder Hökklo och Dolken allmänna utrymmen för presenterandet av sin konst, speciellt herrtoaletten \textsc{(se Herrtoaletten i Lindellhallen s.~\pageref{04ff06fb85370126485265886d1be53e})} under Umeå universitets bibliotek.
 HEAD2: Hökklo och Dolken på 2010-talet
 Hökklo och Dolkens konstproduktion avtog tvärt i slutet av 00-talet. Enligt envisa rykten som cirkulerar inom den mer fashionabla konstvärlden har konstnärsduon i all vänskaplighet gått skilda vägar och den ene ska idag experimentera med handbroderi och försörja sig som textilkonstnär, medan den andre är livskonstnär, och mer därtill.

}

\small{
\textbf{Hönsgård}
\label{3f284439cd46e0b187d34410aa79b2fb}
 En hönsgård är ett inhägnat område avsett för höns, ej uv \textsc{(s.~\pageref{45210da832f9626829457a65e9e7c4d0})}. En bra hönsgård bör innehålla pinnar som hönsen kan sitta på, en dammig plats som hönsen kan lerbada i, ett tak som hönsen kan sitta \textsc{(s.~\pageref{123c3e95c62201513a344526a2fec502})} under, samt ett tråg \textsc{(s.~\pageref{1e0e0470206e0f2baad8e628ba8f770c})} som hönsen kan äta ur när de ledsnar på att picka efter mask. Är det många tjejhöns kan det vara bra med ett rede som dom kan värpa i också. Eftersom i princip alla andra djur är hönsens naturliga fiender bör hönsgården vara inklädd i någon form av stängsel eller nät både på sidorna och ovanpå som skydd mot inkräktare. Vanligast är att man använder hönsnät till det. Brädorna som håller stängslet på plats ska alltid vara röd \textsc{(se rött s.~\pageref{dacd03b85a85d8c8b67c702e1872c498})}. Förr eller senare kommer man dock garanterat tröttna på att bara äta ägg \textsc{(s.~\pageref{128a5feb8e12d0aa622e0298a8332980})} varje dag och vilja göra sig av med hönsen och därför kan det även vara praktiskt med en dörr eller lucka på hönsgården som man kan öppna för att låta fjäderfäna irra sig ut i skogen. Rockgruppen Imperiet belyste problematiken med att äga för många höns i sin sång \textit{Var e vargen}.

}

\small{
\textbf{I otid}
\label{0df53cd7731137e17b1eaa91497a4427}
 Försenad norrlänning. Alternativt ett SJ- eller Norrtågståg som är oframme fast det borde varit det.

}

\small{
\textbf{I'm a punk, and I like sham. Cockney rejects, are the worlds greatest band.}
\label{b975eeec75f3f25ea7adbbb906ae390d}
 \textit{\quotetext{I'm a punk, and I like sham. Cockney rejects, are the worlds greatest band} ær den skønaste textraden i Austin-bandet Big Boys låt }Fun, Fun, Fun\quotetext{ från deras 12} EP med samma namn, slæppt 1982. Olyckligtvis føljs den ovanstående skøna textraden av \textit{\quotetext{But I like Joy Division, and Public Image too}}. Smolk i bægaren. Den førsta raden ær dessutom stulen från ovan næmnda Cockney Rejects.

}

\small{
\textbf{I'm so tired I could sleep on a clothesline}
\label{998a9c7bb8f515055143365ddc2f5e36}
 Detta ursprungligen londonesiska uttryck kan te sig svårbegripligt för den utan inblick i det tidiga nittonhundratalets brittiska samhälle. I Londons east end levde under denna tid de särskilt fattiga, så som prostituerade och sjömän med träben. Inte sällan var dessa två samhällsgrupper så barskrapade att de inte hade någon fast bostad utan var tvungna att driva runt och sova varhelst de kunde. Ett alternativ, om man hade en penny på sig, var att gå till en lokal där man satte sig ned på en bänk, flera personer i rad, för att sova. Vad som höll de sittsovande personerna upprätt var ett klädstreck, fastbundet vid en punkt på vardera sida av de sovande. När tiden för väckning var inne, knöt helt sonika den som ägde etablissemanget upp en av knutarna som höll klädstrecket på plats och lät de slumrande falla till golvet.

}

\small{
\textbf{Ibof}
\label{6d7d119f17cba7cc9edb4448587cf1f5}
 Att ha en ibof är som att ha en fobi \textsc{(s.~\pageref{deaf5f7387941b1c8f557f135d4c370a})}, fast tvärtom. Man är, eventuellt nyfiket, obekymrad över företeelser eller situationer som borde skrämma skiten ur en. Vill man ange en egenskap eller ett tillstånd i stället för en åkomma säger man iboffisk. Kan även användas som skällsord \textsc{(s.~\pageref{e0fc85fd2f5249557257965783ac136e})}, typ \quotetext{Du är ju helt jävla borgiboffisk}. En iboffiker är en person som har en eller flera iboffer.

 Iboffer kan vara individuella eller kollektiva. Psykmediciner används ibland för att åstadkomma konstgjorda iboffiska tillstånd. Den effektivaste behandlingen vid iboffer brukar vara inhämtning och analys av kunskap. Detta kan dock kompliceras av att många iboffiker även har infofobi.

 Några vanliga iboffer är:


 Amerrkibof

 Bildtibof/Borgibof/Fredrikibof (Olika namn för ungefär samma åkomma.)

 Putinibof

 Tillsatsibof/Eibof

 Facebookibof

 Googlibof (De har strukit det där med 'don't be evil'. Makes you think.)

 iIbof

 Bilibof

 Nuclibof

 Konsultibof

 Fraibof

 Ikeibof

 Disnibof

 Växthuseffektibof

 Juholtibof

 Baraentillibof

 Dilevibof

 Ciggibof

 Bankibof

 Hundraåttiknyckibof

 Dödibof

}

\small{
\textbf{Ibofobi}
\label{f9d4e2aabedde507e2b60e29ae399247}
 Rädslan för palindrom.

}

\small{
\textbf{Ica prästost}
\label{646d397ab705b1ab8b709ebc16f876b5}
 Denna prästost är matvarukedjan icas egna. Den är sämre än alla andra prästostar som finns till. Efter att ha inmundigat en leksands knäckemacka med smör, ica prästost och röd paprika på, ska estradören, renässansmannen tillika vetenskapsmannen Carl von Linné ha skaldat:
 \textlessi\textgreater\quotetext{Thänna prästost från icander
 göra vålhd på smakens löhkar
 then smaaka som a4-ark med snorloska på
 thät ej alls i munhålan pöhkar.}\textless/i\textgreater

}

\small{
\textbf{ICA-personal i förklädnad}
\label{f5a6964fb398df4c2da0d3bac3d8ed7a}
 är temat i Elvis Presleys låt \textit{Devil in disguise}. Mer exakt handlar låten om hur upprörd Elvis blir över att viktiga postärenden numera sköts av vanlig ICA-personal. Det är, tycker Elvis, ungefär lika rimligt som att en stighatare \textsc{(s.~\pageref{2a0e911b72b5555cedc4dcd9094c6b86})} skulle vara skogvaktare eller att pedofiler fick starta förskolor. Men tack vare Jan Björklund \textsc{(s.~\pageref{0b9b757044804b9be0e218acdad358cc})} överträffar dikten verkligheten och allt detta är numera sant, till Elvis stora sorg.


 Förtydligande: Elvis har alltså inget emot ICA-personal. Han tycker bara att de, likt skomakaren, ska bli vid sin läst.

}

\small{
\textbf{ICARUS}
\label{1ea8b3ac0640e44c27b3cb8a258a87f8}
 raket försök 5 är en klar seger trots ett misslyckat första försök då stubinen inte fick igång raketbränslet, men efter att ha borrat ut även denna raketen lite till flög den iväg med bravur, förvisso efter att ha stått och rynkigt en liten stund, dessvärre åter fans den inte men man hittade den första raketen Tsygan II mellan 150 - 200 meter från avfyrnings platsen.

 Tidigare raketer:

 Tsygan II \textsc{(s.~\pageref{da919dfb81083059022a634b495dac7d})}

 LANCELOT II \textsc{(s.~\pageref{386a45bf415cd217bf0eb4ab02876db8})}

 THOR \textsc{(s.~\pageref{575e22bc356137a41abdef379b776dba})}

 ISIS \textsc{(s.~\pageref{529419940a585fb2a83765b2ca5cc091})}

}

\small{
\textbf{IChastity}
\label{5743324709834e88a3e168e78ec3e977}
 En tillvalsprodukt till de nuförtiden så populära IBook, IPhone och IPad. IChastity är ett modernt kyskhetsbälte som går från en USB-port till ett bilbatteri, vidare till bärarens könsorgan (med så kallad krokodilklämma/roach clip). En krokodilklämma/roach clip kopplas till organet och vidare till bilbatteriet. Om användaren får för sig att surfa på sådan där pörr \textsc{(s.~\pageref{5faa435e2f0af7617816f0cade262581})} och sedan följer upp detta med att ta på sig själv så börjar krokodilklämmorna nudda vid varandra och självbesudlaren får upprepade stötar. Varför man överhuvudtaget har en USB-port kopplad till denna fantastiska apparat vet nog bara den där nissen som det finns en fet nissepedia-artikel om som gjort jättemycket feta grejer.

 Apple forskar i nuläget på hur man ska gå till väga för att tillverka en produkt som även fungerar för världens alla masochister.

}

\small{
\textbf{Ida}
\label{7f78f270e3e1129faf118ed92fdf54db}
 Fräcking med klös.
 Bor i Apudden.
 Tränar stepp på söndagar.
 Beroende av Earl Gray med mjölk.

}

\small{
\textbf{IHarness}
\label{34704b519a92563ce8c997b0ee8c4901}
 är en Apple-produkt som tillåter en användare av Apples populära läsplatta att hänga från taket ovanför plattan så att det går lättare att samtidigt skriva och ha skärmen parallell med anletet. Med ett enkelt handgrepp drar sig användaren upp i luften efter att ha trätt denna finurliga sele över kroppen. Cirkapris för en ny sele är fyrtiotusen miljarder \textsc{(s.~\pageref{c2160bffc9c5ca88e77204672e62e489})}.

}

\small{
\textbf{IKalix 2.0 (Tigerspank)}
\label{61d4d278d16e7c9cb84c496a10e001e2}
 {{DISPLAYTITLE:iKalix 2.0 (Tigerspank)}}iKalix 2.0 (Tigerspank) är det hemliga namnet på Kalix kommuns hemliga strategi för att vända den negativa befolkningsutvecklingen, framtagen när man insåg att man inte hade råd med ett äventyrsbad \textsc{(s.~\pageref{8e36481b72c8061bb9ff74c1df3b0b66})}. Tack vare hjältemodiga insatser från Wikileaks har dock dokumentet kommit till allmänhetens kännedom. Det hela går ut på att döpa om kommunala egendomar och verksamheter till \quotetext{ungdomligare, häftigare namn}. (Redan där sket det sig. Vem säger häftigt nuförtiden?) Äldreboendet Skogsgläntan ska döpas om till Gandalf House, Sporthallen till Quidditch Arena och kommunhuset till The Ultimate WoW Quest. Kommunaltjänstemännen ska kallas appar i stället, och kommunalrådet för smartphone. Ja, ni förstår. Men så lätt låter sig inte dagens ungdom luras, viket den hemliga arbetsgruppen egentligen inser. Därför har man även som plan B lämnat in en ansökan att kulturminnesmärka hela kommunen, i hopp om att åtminstone kunna locka lite turister när allt vanligt folk flyttat eller dött.

}

\small{
\textbf{Ikea-lassagne}
\label{28aaa4eddd1bb68b419467f77c4dae58}
 En Ikea-lassagne är en maträtt. Man tager gammal köttfärssås, kokar lite maggaroner och en liten kastrull ostsås. Sedan kan de som skola äta själva bygga ihop sin lassagne på tallriken, som bokhyllan Billy, fast ätbar.

}

\small{
\textbf{IMat}
\label{16090887ac47813a2a485ca55a96b4bb}
 är Apples och Oatlys kärleksbarn efter diverse misslyckade affärer, såsom iDryck och iPad.

 Category:Mat \textsc{(s.~\pageref{54662e86f99c17a1f593cf0cd06f62ff})}

}

\small{
\textbf{Inandningsljud}
\label{a4dab892b7b69a2bfdf8044431f2505b}
 Den  Västerbottniska \textsc{(se Västerbotten s.~\pageref{d4b008c5143dcffb6b8c35f3876c2a19})} motsvarigheten till svenskans \quotetext{ja}.
 HEAD2: Inandningsljud på internet
 På internet symboliseras inandningsljudet på följande vis: \textgreatero\textless
 O:et är förstås munnen och taggarna luft som dras in.

}

\small{
\textbf{Incitament}
\label{f9896a922c4b9345ceebc37009eaf545}
 är ett påhittat ord av dom rika för att dölja för arbetare vad chefer egentligen håller på med.

 Studier har visat att om man säger till en arbetare att man tänker ta hans pengar för att man vill ha mer av dom själv utan att göra mer blir arbetaren arg. Om man istället säger till arbetaren att man behöver hans pengar som ett incitament för att sköta företaget är chansen större att arbetaren inte förstår vad det handlar om och låter det hela bero.

 Källa: Internet

}

\small{
\textbf{Indianmuskler}
\label{a0e24bd0dfe9431f72896e16614e79c0}
 får man när man bedriver högre form av träning så som cykling, löpning, skidåkning, spjutkastning, fäktning, lägring, jakt av bäver och annat kritter, bergsklättring m.m.

}

\small{
\textbf{Indianrobot}
\label{ca7bb1abd48444105aa85ddc8adfd29a}
 En indianrobot är till lika delar indian och robot. Ofta består de av två frigolitsaker sammansatta med en brödpåseklammer och en rosa fjäder längst opp. Indianroboten ägnar sig ofta åt allt möjligt.

}

\small{
\textbf{Indiska}
\label{6dea39561dabb0c7475e798d473c2806}
 är en kombinationsaffär \textsc{(s.~\pageref{54328b839527f9917e5d057845b4fc5c})} som tillhandahåller kläder, heminredning och enklare livsmedel till genusvetare, konstskolestudenter och liknande. Som sig bör säljer man endast damkläder och hänvisar eventuella män i butiken till Dressmann \textsc{(s.~\pageref{02ee8e32b89869fffd11aceb4f2e1c10})} strax intill.

}

\small{
\textbf{Individ}
\label{41beed76a0af9b4f550f7ebdecd3e700}
 kan man va när man redan har det bra.

 Individen är högerns subjekt och dyrkas å det grövsta.

 Individen har aldrig aldrig aldrig åstadkommit något.

 Folk som ser sig själva som individer är per definition \textbf{dumma i huvudet \textsc{(se huvud s.~\pageref{e906cd95a540df9b16d0460fb4cf0adc})}}.

 Individer utbrister gärna saker som:
 \quotetext{Måste vi vara kön? Kan vi inte vara individer?}

 \quotetext{Klass hit och klass dit, kan vi inte bara vara individer?}.

 När detta inträffar bör man vara på sin vakt för då har du antagligen att göra med en tokliberal \textsc{(s.~\pageref{531cb70b602e3f3c32d40bac64400830})}.

}

\small{
\textbf{Inga lejon}
\label{c106255d8cfe0239db341a2e52e007b3}
 Det korrekta att svara när någon uttalar det självklara påpekandet \quotetext{vackert väder} på en solig dag.

}

\small{
\textbf{Inge}
\label{0fcfe249cf1f2c42bf15a36a19301ef8}
 kallar sig själv Inge Ansvar. Andra kallar honom Inge Hals.

}

\small{
\textbf{Ingemar Stenmark}
\label{989f4dba6b1fb2e5920e2c251fd693a2}
 kommer från Joesjö utanför Tärnaby, Lappland. Han är en jävel på att åka skidor utförs, och har vunnit så många medaljer att han skulle bryta nacken om han tog på sig alla samtidigt. Stenmark är konungariket Sveriges \textsc{(se Sverige s.~\pageref{b1999637949ed135b2ca03f3a38460cc})} genom tiderna största idrottsman, inget snack. Frågar man karln själv skulle han säga att han väl är \quotetext{helt ok} på att åka skidor. Som att säga att kinesiska muren är \quotetext{ganska lång} eller att kapitalismen är ett \quotetext{ganska dåligt} samhällssystem. Som att säga att HratvinnFlygur \textsc{(se User: HratvinnFlygur s.~\pageref{31c19e82288ba7034064ee9b096bd7cf})} har \quotetext{ganska PK} avslutningar på sina Nissepediainlägg.


 HEAD2: Nissequotes
 Hans lakoniska stuk har gett upphov till en rad kända citat, t.ex.:

 \begin{itemize}
 \item \quotetext{Hä ä bar å åk}
 \item \quotetext{Hä gå int förklar för en som int begrip}
 \item \quotetext{\textgreatero\textless} (Inandningsljud) \textsc{(se Inandningsljud s.~\pageref{a4dab892b7b69a2bfdf8044431f2505b})}
 \end{itemize}

}

\small{
\textbf{Ingvar Carlsson}
\label{9e00e2656c1088e7ac39efe26da62d05}
 (1934 - när som helst nu) är en svensk socialdemokratisk politiker som var Sveriges \textsc{(se Sverige s.~\pageref{b1999637949ed135b2ca03f3a38460cc})} statsminister mellan 1986 och 1991. Efter avklarat uppdrag som folkvald upptäckte han punken och började ställa till osämja i grannskapet genom att spela Black Flags \quotetext{Damaged} (ibland, men mycket sällan, även \quotetext{First four years}) på helgvolym \textsc{(s.~\pageref{3539fdeb41a5b216f614b6ced9ff5cff})} dygnet runt. Han ska även ha setts på en av Henry Rollins spelningar i Stockholm, försökandes stagedive:a iklädd en pappskiva med partilogotypen.

 [http://www.youtube.com/watch?v=ghTLuF2jlTA]

}

\small{
\textbf{Ingvar Kamprad}
\label{c5f2e9ee9a39f83c39079dbcf01d8809}
 (a.k.a Ingvar Mein Kampfras) är en av världens rikaste kapitalister. Han är det tack vare sina företagarskills, men framför allt tack vare stulet mervärde från arbetare över hela världen, inte minst i Asien. Trots att han är mångmiljardär så har han fått ett rykte om sig att vara snål. Snålheten är, i bästa fall, vansinnigt ambivalent. Han har klippkort på den lokala restaurangen där han bor i Schweiz, men har inga problem med att köpa en svindyr vargpälskappa. Hans image inbegriper också en folklighet, även den påklistrad. Han vill framstå som vilken människa som helst, förutom då att han har fyrtiotusen miljarder \textsc{(s.~\pageref{c2160bffc9c5ca88e77204672e62e489})} på banken. Han fick problem med detta när det uppdagades att han med kreativ bokföring sket i att betala skatt och likförbannat ville ha pension från svenska staten. Att folk blir förvånade över att en riking skiter i att betala skatt är konstigt i sig. Men Kamprad är bannemig inte känslokall, trots att han är kapitalist. Han har en gång uttalat sig om hur ledsen han blir när de som arbetar åt honom går med i facket då han tycker att IKEA \quotetext{är som en stor familj}.

}

\small{
\textbf{Innebandy}
\label{2a139376ee082fb1c7bba32a10af3714}
 är en lek som går ut på att man med hjälp av en plastklubba ska fösa en bisarr plastboll in i ett mål. Som hinder har man ett motståndarlag bestående av glaslirare som skriker att domaren är bög så fort man nuddar dom utan att bli utvisad. För att lag från hela landet ska kunna träffas har man hittat på en klubb som heter Superligan som ser till att alla får leka med alla. Fastän det heter Superligan har man inte superhjältekläder på sig utan färdigslitna jeans \textsc{(s.~\pageref{a0f2589b1ced4decbf8878d0c3b7986f})} och Champion-hood. Innebandyklubbor finns att köpa i alla välsorterade leksaksaffärer.

 HEAD2: Kuriosa
 Musikgruppen Takidas medlemmar lärde känna varandra när dom lekte innebandy på samma ställe, vilket speglar musiken väl. Har i radio bland annat omnämnts som \quotetext{svensk rock när den är som bäst från Ånge}.

}

\small{
\textbf{Inre backspegel}
\label{4d9c85c411e32a3a87ec9b69b7b75b70}
 kallas den spegel på bilar som sitter inne i kupén. Formen är ofta rektangulär och färgen på fästet svart eller krom. Dess funktioner är att man enkelt ska kunna spana på brudar i baksätet och ha en central punkt att fästa sin wunderbaum \textsc{(s.~\pageref{cdcb21ec6725e89f532eff2b6504ed46})} på.

 Enligt §18 i Trafikbalken är det lag på att alla bilar i Sverige ska ha inre backspegel. Skulle detta saknas får man dock ingen påföljd. Förmodligen för att man inte vill jävlas \textsc{(se jävelskap s.~\pageref{46845591177f16920cd586a5baf5a625})} med alla ägmästare \textsc{(s.~\pageref{8324518500d7e7ccd22ae364887d4476})} som kör dieselbil med lastgaller \textsc{(s.~\pageref{73b1f975c67393304ff101482965163c})}.

 I sydeuropeiska länder ser lagen annorlunda ut och den ses mer som ett schysst komplement men inte alls nödvändig. Mången spanjack eller italienare bryter därför bort den när bilen rullar ut från försäljaren för första gången med motiveringen \quotetext{Det som är bakom är redan passerat.}

}

\small{
\textbf{Institut och tankesmedjor}
\label{c276b5997d5af80504f79b30d121cf62}
 Sammanslutningar för bilförsäljare, hästhandlare, kulaker \textsc{(s.~\pageref{c17322f1f8b87ec8fc35538dbe1e9668})} och pissliberaler. Använder någon, i exempelvis en insändare, en hänvisning till institut, allra helst stavat på engelska, då har man att göra med en charlatan. Motsatsen är givetvis rediga och hederliga sammanslutningar, exempelvis Landsorganisationen. Ofta så har dessa institut och tankesmedjor engelska namn, trots att de kanske håller till i Schweiz. Andra symptom kan vara blekta tänder. Ofta använder man sig gärna av akademiska titlar som kanske var viktiga någon gång då Per Albin levde, exempelvis Fil. kand. i idéhistoria mfl.

}

\small{
\textbf{Insändarsignaturer}
\label{ba7f775c4e72c16cb824cc84d6bef76d}
 Insändare är, som alla vet, det effektivaste sättet att bilda opinion runt en viktig fråga. För att ge extra trovärdighet åt ditt inlägg är det alltid en bra idé att skriva under meddelandet med en snygg signatur. Den klassiska \textit{Vän av ordning} fungerar naturligtvis fortfarande, men i en globaliserad värld kan det ibland löna sig att prova något nytt. Slösa dock inte tid på att återuppfinna hjulet utan låt Nissepedia \textsc{(s.~\pageref{62400dadecd90cb5cd39062abe5a3e4a})} guida dig genom de nya slagfärdigheterna:

 \begin{itemize}
 \item Jag och många andra
 \item Missnöjd skattebetalare
 \item Guds sändebud på jorden \textsc{(se Användare: Prof. Etienne s.~\pageref{a9878d2280e5a39becac8f73d113df91})}
 \item Aldrig mera sosse
 \item Djurvän och varghatare
 \item Stolt köttätare
 \item Lennart Holmlund \textsc{(s.~\pageref{26d063a59c90487b11c8f5b4fa9af348})}
 \item Lär av de äldre
 \end{itemize}

}

\small{
\textbf{Intellektuell regression}
\label{83d2f1dedf9cade132d175d197430f11}
 är en företeelse som förekommer periodsvis i västerlandets historia och består i att alla plötsligt blir dumma i huvudet \textsc{(se huvud s.~\pageref{e906cd95a540df9b16d0460fb4cf0adc})}.

 HEAD2: Under medeltiden
 Medeltiden \textsc{(s.~\pageref{88cbc30c5b233d97df68b5b041ac0655})} anses vanligtvis innebära en sådan period eftersom man under denna tid lät sig styras helt av Bibeln \textsc{(s.~\pageref{7de7d2a7d608c9a2044f50688bc63e27})} och påven, istället för att som hos de gamla grekerna \textsc{(s.~\pageref{4a5fb3d6ce79b5ff43b33f8f7d843672})} låta det intellektuella samtalet och kritiskt tänkande stå i första rummet. Som tur var följde renässansen på detta mörka kapitel i västerlandets historia och spred åter vetenskapens och det självständiga tänkandets ljus över de kristna träskaften \textsc{(se träskaft s.~\pageref{1ab85ecd859ae682af47bb9334c7dac6})}.

 HEAD2: Under republiken och restorationen
 I den engelskspråkiga världen ledde Oliver Cromwells republik och den efterföljande restorerandet av monarkin, med Karl IIs kröning, till ett bakslag för det allmänna intellektuella klimatet. Puritanerna lät efter att de tillsammans med sina bundsförvanter gjort sig av med Karl I inte vänta på sig vad gällde att skrota alla musikinstrument i kyrkor och kapell eftersom de hatade musik, liksom att stänga alla teatrar och kaféer, eftersom de hatade kultur och sällskapliga samtal om livets många knepigheter. Efter Cromwells död och efter att hans son jagats på flykten öppnades teatrarna igen och man skapade kultur som handlade om överklass-snusk så som ytliga urbana charlataner som jagade in lika ytliga men kurviga damer bakom skärmar i kulisserna. Man skrev ironisk eller lätt erotisk \textsc{(se erotik s.~\pageref{972f097461d1eab1c1ff104757bad922})} poesi om lättidentifierade överklasstanters utseenden och började så smått knåpa ihop grunden för den moderna rasismen.

 HEAD2: I postmodernismen
 Nästa stora period av intellektuell regression började efter det andra världskriget, men blommade inte ut förrän det kalla krigets slut 1989. Nu annonserade Fukuyama \quotetext{historiens död} och nyliberaler världen över ansåg sig sitta på den enda sanningen om hur allt ska vara, vilket de, konstigt nog, uttryckte som att ideologierna var döda \textsc{(se döden s.~\pageref{6f3c270eb5b4d979c777b4ec26dd106f})}. Denna sanning består i att 'åt de som har skall vara givet' och 'åt de som inte har ska vara givet napalm och fosforbomber'. Det intellektuella samtalet består idag av förfasande över frekvensen av mörkhyde i västvärlden, att lönearbetare ställer krav på att faktiskt bli betalda för sitt arbete och att vänstern har mage att föreslå en lite annorlunda världsbild än den som går ut på ett slags förvirrad läsning av Darwins teori om det naturliga urvalet (i vilken själva \quotetext{urvalet} går ut på militära angrepp från skyn och ett slags långsam svält orsakad av \quotetext{frihandel}). I Sverige \textsc{(s.~\pageref{b1999637949ed135b2ca03f3a38460cc})} tillhör bland annat Dagens Nyheter \textsc{(s.~\pageref{b159d08de8d21d8a6d79374b02793693})} denna intellektuella regressions mest högljudda tillskyndare.

}

\small{
\textbf{International cloud atlas}
\label{9e27ba0dde9e80b43e62c72513b2d534}
 är en bok med bilder på olika molnformationer. Den publicerades första gången 1896 men har sedan dess kommit i flera nyutgåvor med fler moln \textsc{(s.~\pageref{9da1014bea9aa67f9cae12e619d34aae})}. För att ett moln ska hamna i atlasen måste det ha en helt unik formation och godkännas av \textit{Cloud appreciation society}. Senast det skedde var 1951 när man valde in formationen \quotetext{Cirrus intortus}. Just i detta nu förs diskussioner om att välja in den nyupptäckta formationen \quotetext{Undulatus asperatus}.


 http://nephology.eu/ \textsc{(s.~\pageref{8d572a3d63cb3e7ae6b0e10dc88e18ca})} International Cloud Atlas på Internet

}

\small{
\textbf{International harvester}
\label{7e828c23d32c8161fd6c4e35697f9d82}
 är ett amerikanskt företag som tillverkar traktorer och andra redskap för jordbruk. Man tillverkade även bilar \textsc{(se bil s.~\pageref{b3188f47d2eac7efc3f1258dc673a9fe})} och lastbilar fordom. Bilar som är heta som tusan, därför ofta röda \textsc{(se rött s.~\pageref{dacd03b85a85d8c8b67c702e1872c498})}.
 Det är även det ursprungliga namnet på psychrockgruppen Träd, Gräs och Stenar \textsc{(s.~\pageref{82a271b29bea1b3fd0073fe6668179bd})}.

}

\small{
\textbf{Internet}
\label{c3581516868fb3b71746931cac66390e}
 Av pederaster och sociopater - för pederaster och sociopater.

 Internet byggdes av den amerikanska armén med syfte att föra ett lågintensivt krig mot god smak. Såhär retrospektivt så kan man säga att man lyckades med sina intentioner. Vi lever numera i en gränslös värld. Då inte på det fina viset där man slipper visa pass och där herrar på rad skriver på ett konvelut. Nej, utan på det där lite olustiga och ganska svårgreppbara viset. Att var och varannan männska inte bara är singer/songwriter utan dessutom swinger eftersom alla de traditionella instanserna för kvalitetskontroll av både Neil Young-epigoner och sex helt försvunnit. Man kan säga att internet är ett solipsistiskt paradis för världens alla rape-dejtande datanördar. Visst folk har väl under längre tid haft mer eller mindre exotiska böjelser men nu har man fräckheten att prata om dom. Det värsta med hela den vämjeliga historia som kallas internet är att det inte längre går att vara i opposition mot något - allt är ju okej.

}

\small{
\textbf{Intersektionalitet}
\label{6dc08633cdbf83eb418ea31ef0302c51}
 Att beakta: \textsc{(se beakta s.~\pageref{5fb8066c875cfced859cf8968e991628})} Kön, genus, könsuttryck, funktionsskillnad, social klass, ställning i produktionen, etnisk tillhörighet, hudfärg, sexualitet, könsidentitet, art och hårfärg \textit{samtidigt}. Som ni säkert förstår kan det bli ganska rörigt.

}

\small{
\textbf{Ipad}
\label{09401fded433c34709fd1f1872728162}
 , eller paddan som den kallas i folkmun är en sorts datamaskin. Den kan både beskrivas som en förväxt iPhone utan möjligheten att ringa, eller som en halv laptop där tangentbordet bantats bort. Den har pekskärm och är därför obrukbar i planeten jordens nordligare delar, om man inte har pekskärmshandskar. Fyller man sin iPad med rock har man rätt att kalla den rockPad \textsc{(se rockpad s.~\pageref{e714d651acb78d6c2d39110f84a4ef8f})}.

}

\small{
\textbf{Iphone}
\label{0b3f45b266a97d7029dde7c2ba372093}
 Vuxennapp! \textsc{(se Vuxennapp s.~\pageref{d842277d157268c94379dbd3624b8e4c})}

}

\small{
\textbf{Irländsk diskussion}
\label{d832a3c58a177f7c838e1307dbbecbb6}
 är ett annat ord för ett gammalt hederligt rallarslagsmål. Anledningen till detta är att irländare ofta slåss medan de diskuterar.

 Se även: lätt misshandel \textsc{(s.~\pageref{da5052972c3a081d8e951c69da453722})}

}

\small{
\textbf{Irländsk konfetti}
\label{149459a4b475b90c8513551228efc472}
 är ett annat ord för stenar, flaskor och andra föremål som kastas hejvilt under upplopp och gatustrider.

}

\small{
\textbf{Irländsk parkering}
\label{792b9feba1f9e0ca7db5f858b4a1b40c}
 Att parkera på irländskt vis är att låta omgivningen, t.ex. träd, tegelväggar och andra bilar, sköta inbromsningen.



 {{Utmärkt}}

}

\small{
\textbf{Isaac Johannes Lamotius}
\label{58101caec425f1b4e8cd3ee2380fd060}
 (1653–1710) var den siste europe som rapporterats ha sett en levande dront. Alla ska vi gå till historien för någonting.

}

\small{
\textbf{Isbjörnsvin}
\label{2879df543437c30c0a2d0dfaf8649ac7}
 , eller Ursus som grekerna envisas om att kalla det, är en fantastisk vodka-baserad dryck som smakar slånbär och som gör att man \textsc{(s.~\pageref{39c63ddb96a31b9610cd976b896ad4f0})} blir poetisk. Dagen efter, å andra sidan, vill man dö. Isbjörnsvin är förebehållet den internationella arbetarklassen, inte minst som ett svar på att te \textsc{(s.~\pageref{569ef72642be0fadd711d6a468d68ee1})} är medelklassens dryck \textit{numero uno \textsc{(se etta s.~\pageref{ba48f6c4097b7fc25ca11f1e544842d7})}}, som de säger.

}

\small{
\textbf{Isebrogen}
\label{01a2b628d6489eb2a02433604c7fd86f}
 är Islands \textsc{(se Island s.~\pageref{3dbacfd76b0a040ccad1eacb20def4c8})} flagga och har så varit allt sedan självständigheten från Danmark \textsc{(s.~\pageref{5331d7fd27772396f412a5b6d19bad44})} 1944. Under kolonialtiden var alla riktiga flaggor förbjudna och danskarna tillät bara att man hissade sopsäckar eller boxerkalsonger i XXL, såsom man gjorde hemma på Själland respektive Jylland. Motivet föreställer Islands enda fyrvägskorsning, omgärdat av fyra glaciärer.

}

\small{
\textbf{Ishacka}
\label{542fa66b98a928c7d702a666c97ce418}
 Effektivt redskap för att hålla efter brillkommenister.
 Kan enligt uppgift även användas vid frisksport.

}

\small{
\textbf{ISIS}
\label{529419940a585fb2a83765b2ca5cc091}
 försök nr 4 gick åt skogen.

 Tidigare raketer:

 Tsygan II \textsc{(s.~\pageref{da919dfb81083059022a634b495dac7d})}

 LANCELOT II \textsc{(s.~\pageref{386a45bf415cd217bf0eb4ab02876db8})}

 THOR \textsc{(s.~\pageref{575e22bc356137a41abdef379b776dba})}

}

\small{
\textbf{Island}
\label{3dbacfd76b0a040ccad1eacb20def4c8}
 .]]
 Island är ett nordiskt land som ligger i vattnet brevid Norge. På Island finns geisrar, polacker, glaciärer och en anrik historia med en massa vikingar i.
 HEAD2: Ekonomi
 Islands ekonomi är baserad på fiske, turism och att låna pengar från Storbrittanien (UK) \textsc{(se UK s.~\pageref{c2431a9d201559f8de1dcfb6a9dd3168})} som de sedan inte betalar tillbaka.
 HEAD2: Politik
 Island är i likhet med de andra nordiska länderna socialdemokratiskt, skrällen är att de har en brud som statsöverhuvud. Parlamentet utses genom allmänna val, men eftersom alla islänningar hatar valar är det nästan ingen som röstar. Rejkavik-borna har valt en borgmästare vars första punkt på dagordningen är att bygga ett äventyrsbad \textsc{(s.~\pageref{8e36481b72c8061bb9ff74c1df3b0b66})} på Kefflavik (OBS! Sant!)
 HEAD2: Kultur
 Islands kultur består till 80\% av Björks förvånansvärt bra techno-pop och till 20\% av Sigur Rós post \textsc{(se posten s.~\pageref{cd13d688571681e426231485b732444b})}.
 HEAD2: Island i sportvärlden
 Island deltar sällan i sportevenemang, men har spöat Sverige i både fotboll och handboll en gång. De är dock hejare på glima \textsc{(s.~\pageref{b46db05ac9d1b0bb2019c68b73335729})}. Den ständigt kontroversielle shackspelaren Bobby Fischer spelade schack mot dåvarande världsmästaren Boris Spassky på Island '72. Fischer, slut som människa, återvände till Island 2005 efter att ha blivit landsförvisad från resten av världen.
 HEAD2: Kända islänningar
 Björk
 Snorre Sturlasson
 Egil Skallagrimsson
 Halldór Laxness
 Bubbi Morthens

}

\small{
\textbf{ISO 216}
\label{2f6bbcf45fd8afaa6eed567fcfe4c722}
 är en internationell standard för pappersform \textsc{(s.~\pageref{37edcb2e533bd9c3e51f475c598b8671})}. Tyvärr är den inte internationell på riktigt då Nordamerika inte är med på tåget, vilket är helt särske \textsc{(s.~\pageref{552a5aad891937bf760fb193900ea140})}.

}

\small{
\textbf{Italienska svordomar}
\label{3a490d7017c99929180c9d80e52e5926}
 är rejäla grejer. Där är det inget larv med något \quotetext{heliga blå \textsc{(se franska svordomar s.~\pageref{b842b5b380758ed7e123daede577617b})}} eller \quotetext{skam på torra land}. Till och med amerikanernas \quotetext{könsligt umgänge med en moder} står sig slätt \textsc{(s.~\pageref{a9cde01124ca41f23d6044b3ba27b979})}. Eftersom italienarna över lag är ett väldigt religiöst folk blandar man gärna in gud \textsc{(s.~\pageref{91e49146121c992aab11a19c77e26cf0})} eller jungfru Maria i sammanhang kring sex \textsc{(se erotik s.~\pageref{972f097461d1eab1c1ff104757bad922})} och avföring. \quotetext{Dio cazzo porco cristo culo!} betyder ungefär \quotetext{guds kuk i Jesus \textsc{(s.~\pageref{110d46fcd978c24f306cd7fa23464d73})} grisarsle} och används mycket i allmänna sammanhang på fotbollsläktare \textsc{(se fotboll s.~\pageref{961bd74d34872ff94a4df3a16119096e})}. För direkta uppmaningar riktade till personer är \quotetext{Vai in culo!} (\quotetext{Gå in i arslet!}) och \quotetext{Cazzo durro dell cavallo!} (\quotetext{En hård hästkuk (i din mun)) \textsc{(se mun s.~\pageref{6585f290ce92c3de5ff339920330e26f})} mycket vanliga. Som ni märker är det lite annan nivå än }kyss Karlsson\quotetext{ och }far åt Häcklefjäll \textsc{(s.~\pageref{40a7322a2ef5adb9efd69969d8f28f1e})}\quotetext{.


 För den som vill konstruera sina egna italienska svordomar kan följande lilla ordlista vara mycket användbar:

 Dio (Gud)
 Cristo (Jesus)
 Madonna (jungfru Maria)
 ostia (hostia, nattvardsbröd)

 cazzo (kuk)
 fica/figa (fitta)
 chiavare (knulla)
 coglione (testikel)
 culo (röv)
 merda (skit, som massord)
 pirla (skit, om person)

 boia (bödel)
 cane (hund)
 porco/porca (svin)
 troia (sugga)
 HEAD2: Trivia
 Det italienska cosmic doom-bandet \textsc{(se doom s.~\pageref{b4f945433ea4c369c12741f62a23ccc0})} Ufomammuts gitarrist och bassist har i samtal med en av Nissepedias medarbetare varmt förordat användningen av interjektionen \textit{Bella fika!} vilket betyder ungefär }vacker framstjärt! \textsc{(se framstjärt s.~\pageref{843a33cdc4d90488cea3030f8b941e08})}"

}

\small{
\textbf{Ivar Bryntse}
\label{a5e922bb2ad7c32a2419b5ba3afcdc99}
 (1917-2003) uppfann den första hydraliska palltrucken \textsc{(s.~\pageref{1315ee23693c2c382b0e6b878be74cbc})} år 1947.
 Denne man förtjänar någon form av postum utmärkelse eller högtidsdag.

}

\small{
\textbf{Ivar Lo Johansson}
\label{9313ba1f83fee1fe49b9f0d0c5ea41c8}
 (1901 - 1990) var sveriges första punkare. Hans arv förvaltas idag av hans sonson Johan Johansson. Category:musik[[Category:litteratur]] \textsc{(se Category:musik s.~\pageref{38cce583d2d3675d645425cb435aa2bb})}

}

\small{
\textbf{Ivriga små bävrar}
\label{6d10ab1ba7bd378ba7cc1629ddf2bbde}
 är centerpartiets \textsc{(se centerpartiet s.~\pageref{e331dec360e356adc1e2db36fe9a9f3f})} officiella begrepp för vad som normalt kallas egenföretagare. Där många ser en kulturarbetare med viscerala stressyndrom ser alltså centerpartisten en ca 75 cm hög amfibisk gnagare \textsc{(s.~\pageref{b5c3a0f14d5f76de604f5d8e4cc068ff})} som har stor negativ inverkan på biotoper där den lever. Ett annat sätt att tolka detta begrepp är via slangbetydelsen av ordet \textit{bäver}. Där många ser en 34-årig, studieskuldsbetyngd massör som är tre övertidstimmar från att gå in i väggen med dunder och brak ser alltså Maud Olofsson \textsc{(s.~\pageref{eb913a2e9be929654908a05017401bd6})} en behårad kroppsöppning \textsc{(se framstjärt s.~\pageref{843a33cdc4d90488cea3030f8b941e08})}.

}

\small{
\textbf{J.R.R Tolkien}
\label{3f0b7fcbd9fa7369ca314a46c280b67e}
 John Roland Razorhoof Tolkien (1892-1973) var en brittisk författare och språkvetare som skapade de populära böckerna om Midgård (Middle Earth). Böckerna handlar om ett antal fantastiska väsen, hoberna, som endast Tolkiens fantasi kunnat föda fram: De är som korta människor med hår på fötterna.
 Triologin om härskarringen anses av finniga litteraturvetare i femtonårsåldern vara kärnan i Tolkiens författarskap. Den handlar om en ring som några hober ska kasta ner i en vulkan. En trollkarl är med och så finns det orcher (upphottade troll). Sex är det dåligt med, men en del våld kan som tur är utlovas.

 HEAD2: Trivia
 \begin{itemize}
 \item De flesta av Tolkiens böcker handlar om ringar.
 \item Tolkien rökte pipa.
 \item Åke Ohlmarks \textsc{(se Tolkien och den svarta magin s.~\pageref{b2cc087ddfcc973d83fe146ca31fe88e})} har länge svurit på att Tolkien gick på Subutex.
 \end{itemize}

 HEAD2: Sagt om Tolkien
 '“The man once wrote: Do not meddle in the affairs of wizards, for they are subtle and quick to anger. Tolkien had that one mostly right.'
 ― Jim Butcher
 '\quotetext{Asså Tolkien är ju så jävla dum i huvet \textsc{(se huvud s.~\pageref{e906cd95a540df9b16d0460fb4cf0adc})}}' - M. Frygell

}

\small{
\textbf{Jackie Howe}
\label{8c5310106f65a85e6a8b06acbaf29e72}
 (26 Juli 1861 – 21 Juli 1920) var en australiensisk \textsc{(se australien s.~\pageref{e727d8d1b3162a732c7f706d55de64f3})} fårklippare och fackföreningsledare. Howe är framförallt känd för att han år 1892 slog världsrekord i antal klippta får under en dag respektive en vecka. Dagsrekordet Howe satte uppgick till 321 får på strax under 11 timmar och stod sig ända fram till 1950 när Ted Reick slog det. Reick använde dock elektrisk klippare medan Howe på sin tid enbart jobbade med handkraft \textsc{(se handjagare s.~\pageref{a85b0d5ff42c09f4605cec0188b8dd6e})}. Hans veckorekord på 1437 klippta får är fortfarande obesegrat. En bronsstaty föreställande Howe klippandes ett får finns rest i det lilla australiensiska samhället Blackall.

}

\small{
\textbf{Jackson Pollock}
\label{ccd4f29c0e00799c7319efcd15069fa0}
 , född 28 januari 1912 i Cody, Wyoming, USA, död 11 augusti 1956 i Springs, New York, var en amerikansk \quotetext{målare}, centralfiguren inom den abstrakta expressionismen med sin \quotetext{action painting}. Pollocks teorier byggde i stort på att göra sitt liv så enkelt och bekvämt som möjligt och en stor del av detta mål var att försöka övertyga konstvärlden om att han var duktig. Man kan likna Pollocks verk lite vid produkten en fotograf producerar \textsc{(se fotografering s.~\pageref{176551844874f34f5bb9a9d0ac93f99a})}. Medan fotografens jobb består i den komplicerade konsten att trycka på en knapp tog Pollock lättja till en ny nivå. Han lät placera duk på golvet och gick sedan mest runt i sin studio med färg droppande från olika saker. På grund av detta passerade det abstrakt måleriet fotografiet efter den klassiska devisen \quotetext{mest utdelning för minst ansträngning} då han kunde göra en bild under kortare tid än en fotograf kan i och med den komplicerade fotoframkallningsprocessen som består i att gå till en butik och få bilder utskrivna.

 Än idag kan man se resterna efter Pollocks runtflanerande på olika museum.

 Jackson Pollocks action-painting var också en föregångare till action-filmen (som uppfanns under 80-talet i och med filmen Terminator) men idag är denna koppling mer eller mindre bortglömd och oftast tillskrivs tv-serien COPS som action-filmens urfader.

}

\small{
\textbf{Jacques Touillaud}
\label{0cb54fc44bc0ed93ced4ad4e981ea30b}
 var gaffare vid inspelningen av James Bond-filmen \textit{Moonraker} (1979). Han var även chefselektriker vid inspelningen av den betydligt mindre framgångsrika \textit{En fluga i soppan} (Brust oder Kule) från 1976, men denna films begränsade försäljningsframgångar ska på inget vis härledas till Jacques, vars arbetsinsats ska ha varit så nära felfritt som det går.

}

\small{
\textbf{Jag}
\label{6d3b29d3effedec81efd70167d6bf670}
 är bäst. Men det finns andra artikelförfattare som tycker att de är överlägset sämst.

}

\small{
\textbf{Jag ska bara bli full först}
\label{66ecf23dc53d37f0509f569153ddbb6f}
 Sex vackra små ord, som kan hjälpa dig i de flesta situationer. Ursprungligen var det Epikuros \textsc{(se epikurism s.~\pageref{198603b3bd87cb821515314304b24181})} som myntade uttrycket, fast på grekiska, men det var Prof. Etienne \textsc{(s.~\pageref{56957a267e57df32753cf7f3b8a603d8})} som gjorde det till ett kommersiellt mantra i och med sin psykoterapeutiska talkshow, \textit{Baby, du borde institutionaliseras}. I talkshowen, som sändes efter pörren \textsc{(se pörr s.~\pageref{5faa435e2f0af7617816f0cade262581})} på TV 1000, tog professorn emot gäster med olika psykologiska problem. Under första halvan av programmet lyssnade mest värden på sina gäster. I den andra halvan var det dags att ställa diagnos och rekommendera vidare behandling. Men innan det var det alltid dags för programmets mest populära segment, \textit{Jag ska bara bli full först}, i vilket Prof. Etienne söp sig redlös under 20 minuter i sällskap med sin porslinshund Trigger, och inte sällan tillsammans med sina gäster.

 Uttrycket och sysslan att faktiskt bli full innan man gör viktiga saker, har sedan dess blivit en smash hit världen över. Ofta kan man höra folk säga det innan man byter däck på bilen, gräver ett dike, tillagar moules frites, går ut i skogen och letar dyckert \textsc{(s.~\pageref{4e5643f8df0b0729ba0c40470cc43d69})}, eller innan ett parti intensiv men slafsig älskog. En viss backlash kom i och med att skolungdomar började bara bli fulla först, innan de gjorde läxan. Detta resulterade i en förlorad generation illitterata suputer som vid ett tillfälle gav SD över tio procentenheter i en gallupundersökning.

 Men trots det fortsätter folk att säga, \textit{Jag ska bara bli full först}, innan de borstar tänderna, kör hem svärmor från släktmiddag, renoverar badrummet, eller packar in ungarna i familjens volvo 740 \textsc{(s.~\pageref{e262951543da05bac43c7b87235a115c})} och drar till Legoland \textsc{(se lego s.~\pageref{3a22c9ea9a3039d180e0a514a5a3b619})} för lite dansk semester \textsc{(s.~\pageref{d23078bd5d0733edeb88fccd11d196e1})}.

}

\small{
\textbf{Jakobsbergs sportfiskeklubb}
\label{f9c353edd2f9474b34a0bf241dad3895}
 är en fiskeklubb grundad år 1954. Föreningens huvudsakliga aktivitet är sportfiske men även pimpling och mörtsump förekommer. Föreningens logotyp består av en hink med fisk och några måsar som cirkulerar ovanför. Förutom att inte få napp är medlemmarnas största nemesis försurningen av de svenska sjöarna. Klubben bytte därför namn år 2008 till Ph-värdets vänner. Under tävlingar i klubben är det inte tillåtet att fiska med bilbatteri \textsc{(s.~\pageref{3894bc588cff2547bbfa28aaf455313a})} eller anglosax \textsc{(s.~\pageref{75591674b0deca83291ccfef6f4f557c})}.

}

\small{
\textbf{Jan Björkblund}
\label{51df1401461fb9b68821ef30a288c1d3}
 är en sagofigur som ofta förekommer i godnattsagor och vaggvisor. Grundberättelsen, av vilken det finns olika varianter, går så här: När det lilla barnet \textsc{(se barn s.~\pageref{5dfcc0aab2f3db925b2d51ba73e48946})} har borstat sina små vita mjölktänder, kysst mor och far godnatt, repeterat sin katekes och lagt sig mellan lakanen flyger Jan Björkblund in genom det öppna fönstret och ställer språkkrav. Han läser från en lista med krav på motprestationer på invandrare och för varje steg på listan blir barnets ögonlock \textsc{(s.~\pageref{4a83c9b0ac915b5affcb5ddcb17f2823})} lite tyngre. Den lille, som hört allt detta fyrtiotusen miljarder \textsc{(s.~\pageref{c2160bffc9c5ca88e77204672e62e489})} gånger i sitt oansenliga liv, slappnar av och drömmer sig bort. Nu somnar barnet in och Jan Björkblund, som ser att sitt arbete är gjort, utsvävar populistiskt genom fönstret igen för att ställa språkkrav i en barnkammare någon annanstans.

}

\small{
\textbf{Jan Björklund}
\label{0b9b757044804b9be0e218acdad358cc}
 (fp) är riksdagspolitiker och uppvisar den intressanta kombinationsdiagnosen \quotetext{liberal} och \quotetext{sjuk i huvudet}.

 HEAD2: Jan Björklund i mytologin
 Se Jan Björkblund \textsc{(s.~\pageref{51df1401461fb9b68821ef30a288c1d3})}

}

\small{
\textbf{Jan Guillou}
\label{63f2c8aba9686bc92efeb7eb21e35156}
 Jan Oskar Sverre Lucien Henri Gillou är en svensk författare, assäist, krönikör och debattör. Han föddes i nådens år 1944, antagligen av en kvinna som befruktats av en man, och visade redan i tidig ålder prov på sin goda förmåga att berätta \textsc{(s.~\pageref{4f84e02a70b3bbb57fa83da31bf7a16f})}. Vad få vet är att Guillou förutom att vara författare, assäist, krönikör, dragbasunist \textsc{(se dragbasun s.~\pageref{0315aaaabb57a67312aa3316fd2006e1})}, luftgitarrdomare \textsc{(se luftgitarr s.~\pageref{0e2415e86edc316f5338964c6ef145b5})} och debattör också är uppfinnare av ett föremål som används för avrättningar, nämligen yxan \textsc{(se yxa s.~\pageref{bd74f429522c7c1481fbba07187efc6b})}.

}

\small{
\textbf{Jan Wilsgaard}
\label{213a0ab775d17e92dfd78748a2a1bc3b}
 Estetikens urfader.  Designade bland annat Amazon, 240 \textsc{(se Volvo 240-serien s.~\pageref{9a8db892f6b42596f2abce57f62a6399})} och 740 \textsc{(se volvo 740 s.~\pageref{e262951543da05bac43c7b87235a115c})}.

}

\small{
\textbf{Japan}
\label{578ed5a4eecf5a15803abdc49f6152d6}
 är två öar längst bort i Asien, om man mäter från London som man brukar. Där bor de flesta av världens japaner i nybyggda höghus av gummi eller i anrika samurajtempel av trä. När som helst kan hela skiten sopas bort av en flodvåg eller jordbävning så japanerna lever sina liv i ett sinnessjukt tempo för att ändå hinna med så mycket som möjligt. Dom har typ växlar på rulltrapporna och ris som kokar sig självt efter att man svalt det och robothundar som bara behöver rastas en gång om året och såna sjuka grejer. Riktigt hektiskt. Konjekturalt \textsc{(se Konjektural s.~\pageref{97b105b94d1adc2125ccd7409f18beda})} nog har detta stressande gjort att typ alla japaner blir 150 år gamla. Enda rimliga förklaringen är att de lever efter samurajernas hederskodex \textsc{(s.~\pageref{dec840b19d3e79e3b3ce89b1995bafd9})}.  Ekonomin består främst av guldpengar som samlas in i tv-spelen \textit{Super Mario Bros} 1-3. Utöver detta finns även yrkena ”valforskare” och kamikazepilot, som också verkar vara ganska stressigt.

}

\small{
\textbf{Jeans}
\label{a0f2589b1ced4decbf8878d0c3b7986f}
 är ett slags slitstark byxa som idag är väldigt populär över stora delar av världen, men inget land mäter sig mot Kanada där man är som tokig i jeans \textsc{(se Kanadensisk frack s.~\pageref{365eda0bac157feb7ee22ceb7f07e17d})}. Jeans associeras ofta med ungdomskultur och rockmusik och detta är i mycket Grand Funk Railroads \textsc{(se Grand Funk Railroad s.~\pageref{fd425d96c8908215d6cdd7a72ac97c27})} förtjänst då de var de första i musikvärlden att bära jeans. Att jeans är populärt just i musikvärlden är inte ologiskt då jeans liksom rockmusiken ofta förknippas med en individualistisk livsstil och utmanande av etablissemangets konventioner, vilket till exempel gestaltas med all önskvärd tydlighet på omslaget till Bruce Springsteens skiva \textit{Born in the USA} (1984).

}

\small{
\textbf{Jeansröv}
\label{3de9cbeefa5338dd6d965236eb7a8aa8}
 Det som nittio procent av all amerikansk rockmusik från 80-talet handlade om. Till exempel Bruce Springsteens \textit{Born in the USA \textsc{(se United States of America s.~\pageref{ade6b3bd5e720abb20ed8a9a4c6b9ae8})}}

}

\small{
\textbf{Jemen}
\label{908706852c5107d727d8d0eeffe8782d}
 -Jemen är ett land i Mellanöstern.
 -Yeah man!

 Se även: Oman \textsc{(se Oman s.~\pageref{ad437fd2f44c7b3f8208a162604d81d0})}

}

\small{
\textbf{Jens}
\label{e19457c81e62b6bb21e9031a5a187cdf}
 är ett mansnamn som är relativt vanligt hos personer som är med på olika sorters läger, så som pingisläger, scoutläger och skidläger, och där är föremål för ens odelade irritation.
 Namnet är en korrumperad form av ordet \quotetext{jeans \textsc{(se jeansröv s.~\pageref{3de9cbeefa5338dd6d965236eb7a8aa8})}}.

}

\small{
\textbf{Jerry Williams}
\label{7a4fce22b1c66561eff5401a0c92e003}
 Sveriges \textsc{(se Sverige s.~\pageref{b1999637949ed135b2ca03f3a38460cc})} motsvarighet till Lemmy \textsc{(s.~\pageref{6cc2f8758343439728f308f08a4a8fad})}. Har turnerat sedan koppardaler var en accepterad valuta \textsc{(s.~\pageref{cf1e2a0af4955aa7539b6e12e9d282e6})} och skiter i hur ljudet låter så länge det svänger. Allt du vet om rock har du lärt dig av Jerka.

}

\small{
\textbf{Jesaja 36:12}
\label{cddcbdb1e8a5df591e5efa642a584350}
 Men Rab-Sake svarade: »Är det då till din herre och till dig, som min herre har sänt mig att tala dessa ord? Är det icke fastmer till de män som sitta på muren och som jämte eder skola nödgas äta sin egen träck och dricka sitt eget vatten?»

}

\small{
\textbf{Jesus}
\label{110d46fcd978c24f306cd7fa23464d73}
 Kristus (f.0 - d.33) var son till gud \textsc{(s.~\pageref{91e49146121c992aab11a19c77e26cf0})} och \quotetext{Jungfru} Maria. Med sådana inflytelserika föräldrar var det naturligtvis inget problem för Jesus att komma in på den tidens byggprogram och sedan att få jobb som snickare. Jesus gick som prao och handjagade \textsc{(se handjagare s.~\pageref{a85b0d5ff42c09f4605cec0188b8dd6e})} ett tag men fick sedan arbete som informationsansvarig på sin farsas firma och reste runt på firma-åsnan och höll presentationer och föredrag för judar och fariséer \textsc{(s.~\pageref{4503817c4c5816b5f30fffc91f66ac28})} på konferensmiddagar. Vid ett sådant tillfälle ska Jesus ha bjudit på odeklarerade mat- och starkvaror. De romerska myndigheterna hade tampats med detta utbredda problem i flera år och var nu så less att de ville statuera ett exempel. Man ställde därför Jesus inför rätta, men än en gång kom farsan till undsättning och såg till så Jesus fick en plats som vice VD i vad som då blev familjeföretaget.

}

\small{
\textbf{JO))))N}
\label{2c874694453385960dd7f4ccc9a2e54f}
 är ett droneprojekt som består av en man och hans elorgel, spelandes tonen C\#.

}

\small{
\textbf{Joelbrevet}
\label{a4958b91cbc4cc67e342f00bc8f4d206}
 kan vara det finaste stycke tankar, känslor och idéer som någonsin fästs på papper \textsc{(se pappersform s.~\pageref{37edcb2e533bd9c3e51f475c598b8671})}. Brevet är skickat från Oslo, Norge till Umeå, Sverige \textsc{(s.~\pageref{b1999637949ed135b2ca03f3a38460cc})}. Från en kär vän, exilumebon Joel Brändström, till en annan, radioprofilen och villaägaren Torbjörn \textsc{(s.~\pageref{c3e6fb6fb2b655457597f063bd9392e8})} Rolandsson. På den senares begäran har jag, Ronny, transkriberat brevet i sin helhet.

 HEAD2:  Innehåll
 \textlessi\textgreater\quotetext{Kära vän

 När du öppnar det här brevet har jag sedan länge bott i Oslo. Det är då min innerliga önskan att du ska tänka på mig med en förnimmelse av samhörighet och oavvislig kärlek, just så som jag i skrivande stund tänker på dig. Jag vet att du i ditt innersta stundom känner sorg över de svårigheter som har drabbat dig och som möjligen kommer att följa dig genom livet. Ja, att du kanske också av och till genomilas av förtvivlan, men vet då att förtvivlan enligt uttryck från den store författaren Thomas Mann \textsc{(se Man s.~\pageref{39c63ddb96a31b9610cd976b896ad4f0})} inte är slutet. När människan vaknar upp har hennes blick klarnat.

 När vi för länge sedan, kära vän, brukade vandra från rollspelslokalen till affären och vi samspråkade om drakar, seriefigurer och tärningsslag var det en fröjd att märka hur lätt du förstod och tog till dig allt. Jag vet att du kommer att lyckas i livet. Du kommer alltid att vara till stor heder för dina vänner och din familj och allt som du fått löfte om ska tillfalla dig.

 Även om du inte uppsöker gudstjänsten kommer du att finna stor tröst i mina avslutande ord från evangelisten Markus. Det är bättre för dig att gå in i Guds \textsc{(se Gud s.~\pageref{91e49146121c992aab11a19c77e26cf0})} rike enögd än med båda ögonen i behåll. Varje människa måste saltas med eld. Jag glädjer mig redan nu, långt i förväg, åt alla framgångar som oundvikligen ligger framför dig.

 De varmaste hälsningar från din vän

 Joel Brändström}\textless/i\textgreater

 HEAD2:  Mottagande
 Enligt säkra källor ska Torbjörn \textsc{(s.~\pageref{c3e6fb6fb2b655457597f063bd9392e8})} Rolandsson vid mottagandet av detta brev suttit i köket, druckit en kopp kaffe och under läsningen känt kärlekens värme sprida sig i hela sin kropp. Vid fullbordad läsning av dokumentet ska han ha gått in på sin kamrat John Anderssons rum för att i sin helhet recitera brevet, eftersom det var så förbaskat fint och bra.

}

\small{
\textbf{Johan dahlberg}
\label{11023feb5a10d8d6fc311c732ca7b077}


}

\small{
\textbf{Johan Dahlberg}
\label{11023feb5a10d8d6fc311c732ca7b077}
 Adjektiv för att beskriva någons förtida åldrande samt allmänna gubbighet.Detta yttrar sig i daglig fönsterpost, ilsket bläddrande i lokaltidningen och en total oförståelse av moderna människor.

 Många uppfattar meningen \quotetext{Fan vad Johan Dahlberg du har blivit på sistone!} som tungt kränkande.

}

\small{
\textbf{Johan Skytte}
\label{bfb1371ca26a0f7f54616d1076e7adf1}
 (1577-1645) var friherre och allmän vis man. Han undervisade bland annat Gustav II Adolf, men glömde ge denne lektionen om vikten av att undvika krig i dimma. Skit som händer. På 1600-talet gick det bra att syssla med lite vad som helst bara man var smartare än allmogen, eller i alla fall låtsades att man var det. Johan Skytte nyttjade detta privilegium till att, bland mycket annat, inrätta Skytteanska skolan i Lycksele. Skolans uppgift var att utbilda missionärer som hade till uppgift att konvertera så många samer som möjligt, och på det viset kolonialisera hela Umeå Lappmark. Det lyckades med facit i hand ganska väl.

}

\small{
\textbf{Johann Neumann}
\label{6df8131236a0fd857e27e9cbba3bb95e}
 , född 10 september 1949 i Österrike. För många känd som \quotetext{Iprenmannen}, men framförallt ihågkommen för sin fantastiska rollprestation som \quotetext{Ödlan} i \textit{Jönssonligan på Mallorca}. Då det ibland kan gå något decennium innan Neumann hittar ett nytt filmmanus som är bra nog för att han ska ställa upp händer det att han får det lite kärvt ekonomiskt. Därför är han även utbildad konservator och urmakare. Två typiskt österrikiska sysselsättningar som med lätthet kan kombineras om en kund till exempel beställer en uppstoppad uv \textsc{(s.~\pageref{a562653cfd13c16d7f4d85967242ccdd})} med tidtagning.

}

\small{
\textbf{Johannes Brost}
\label{cc2c0840e6426345d582a25e02674787}
 Sveriges grand old man när det kommer till a-hootin' and a-hollerin' \textsc{(s.~\pageref{1928c39ea0f58992a3e5f53d143a23ff})}. Började på 60-talet med att supa med Rolling Stones och fortsatte med det fram till det glada 80-talet när Stones började suga och han gick över till att äga upp \textsc{(se ägmästare s.~\pageref{8324518500d7e7ccd22ae364887d4476})} alla andra i \textit{Gäster med gester} istället. Efter en hektisk inspelning av succéfilmen \textit{Black Jack} behövde Brost varva ned och tog jobb som bartender på en finlandsfärja där han stannade i nästan 10 år. Som om inte allt detta vore nog har han dessutom med sina åtta gånger svenskt rekord i att medverka i \textit{Fångarna på fortet}.

}

\small{
\textbf{John Kellogg}
\label{553eafeebd3ccaab41e38937d5ab7c27}
 var en amerikan som är mest känd för att ha uppfunnit konceptet frukostflingor. Vad han är mindre känd för är att han var en ihärdig förespråkare för yoghurtlavemang, någon som han menade lämnade tarmen \quotetext{skinande ren}. Kellogg tyckte även att man borde hindra flickor från att onanera genom att utsätta klitoris för frätande syra. För pojkar menade Kellogg att det var lämpligt att sy ihop förhuden. Han var även, föga förvånande, vegetarian.

}

\small{
\textbf{Johnny Takter}
\label{7cceecf64f19c092b60fd77b28ee337d}
 (f. 1958) är en svensk travkusk och alla spelares skräck. Tippar man Takter som vinnare kan man ge sig sjutton på att han kör bort sig, men om man räknar ut honom lyckas han alltid vara en mullängd före tvåan. Därav uttrycket ”vara lika opålitlig som Johnny Takter”. Han ser ut som en vanlig travkusk med hjälm och ridstövlar och allt sånt där, men ändå är det något som inte stämmer. Kanske har han en enäggstvilling som heter Assar Takter som tar hans plats ibland och kör som Farmor Anka. Eller så hatar han det monetära samhället och går emot oddsen med flit. Eller så har han helt enkelt bara lite för roligt på förfesterna innan loppen ibland och råkar dricka för mycket hemohes \textsc{(s.~\pageref{78daf59d2c820001becb4f44e9e89ab0})}.

}

\small{
\textbf{Johnskrove}
\label{92a6f4a71ab0087f48ba4aab7db89bdb}
 En Johnskrove är en maträtt som påminner om en calskrove \textsc{(s.~\pageref{84ff54e779ee49fdad21e17c20f14453})}, men består istället av någon som heter John inbakad i en calzone. Ordet kan också syfta på ett skrovmål inuti en person som heter John \textsc{(se Användare: HratvinnFlygur s.~\pageref{26c5d96dca8dfce84752fa1d4095fdb0})}.

}

\small{
\textbf{Jolle}
\label{4fe195f73917395e8a5851dc036ef8bc}
 En jolle är en liten båt man hoppar över till om man är på en stor båt som börjar sjunka. Den är jätteliten så alla får inte plats i den, tyvärr.
 HEAD2: Alternativa betydelser
 Jolle kan också syfta på en marijuana-cigg \textsc{(s.~\pageref{2bcc66e1261fa5199a4f4decf2720ef5})}. Den räcker oftast till alla som vill ha, men kan paradoxalt nog vara anledningen till att båten sjunker.

}

\small{
\textbf{Jon}
\label{006cb570acdab0e0bfc8e3dcb7bb4edf}
 är ett tilltalsnamn för män och betyder en atom eller molekyl som antingen har ett underskott eller överskott av elektrisk laddning. Känner man någon som heter Jon och som har dåligt med pengar kan man skämtsamt kalla honom fattig-hjon, men inte så ofta att det blir tradigt eller att jonen i fråga blir ledsen.

}

\small{
\textbf{Jon och de strejkande spanska fotbollsspelarna}
\label{1095a308163d34df5779fdb208092277}
 är ett trallpunkband från Norberg med omnejd. Bandet låg i nittiotalets början på skivetiketten Owlnest, men akterseglade \textsc{(se aktersegla s.~\pageref{fd1077f333993f7f88b3b43533db9b98})} dessa efter försäljningsfloppen \quotetext{I skymningstid} och började släppa sina album på det egna bolaget, BMI records, som samfinansierades av en rysk oljegark \textsc{(s.~\pageref{d78fbbc214d52206f58476f02f66f0b6})} vid namn Tjarbjarij Rolanskolnikov, som senare spårlöst försvann med bolagets kassaskrin. Bandet gick till den svenska publikens hjärtan i och med samlingskivan \textit{Definitivt femti spänn - igen}, där man bidrog med en låt om krig och elände vid namn \textit{Om kriget kommer \textsc{(s.~\pageref{86325b0844aed9a3678fc492c795ba16})} imorgon, vill du ligga med mig då?}.

}

\small{
\textbf{Jonas}
\label{9c5ddd54107734f7d18335a5245c286b}
 Visst finns det alltid minst en Jonas i varje vänskapskrets? Ofta finns det ett helt gäng Jonas och då får man kalla dem Jonas J, Jonas S, Jonas med kniven, Lill-Jonas och så vidare.

}

\small{
\textbf{Jonas Claesson}
\label{65b9252dc2a0fa610d59d72854440ae7}
 är Sveriges i särklass bästa bandyspelare någonsin. Han är svensk bandys \quotetext{Stor Grabb \textsc{(se Stora Grabbars och Tjejers Märke s.~\pageref{3b527f8b13885eb277c77de4b1f51658})}} nummer 202 och försörjde sig under sin tid som bandyspelare delvis med att hyra ut släpvagnar åt sin brorsas bensinmack. Han spelade med Vetlanda, men framförallt med Hammarby IF bandy \textsc{(s.~\pageref{2f6b1282edcfc4b164f0f529b8e50d43})} och så landslaget då, såklart.

}

\small{
\textbf{Jonathan Guy}
\label{e311c3151a8f2d7afe7fd5b2432739b1}
 var den första engelsmannen att födas i Kanada. Han var son till Nicholas Guy och dennes fru, vilka tros vara de första engelsmännen som gjorde barn i Kanada.

}

\small{
\textbf{Jons polare Nicke}
\label{ad2945c8325d2a4b16b1d6f1aadf7d0a}
 är en man i sina bästa år som vet lite av varje, och gärna delar med sig av det. Han gillar psykadelisk rock och sånt.

}

\small{
\textbf{Jonsson}
\label{f9a15f696d32dfff3f37dcdedfc334df}
 Tomas Jonsson, mest refererad som Jonsson

 Vokalisten i det ökända bandet Anti-Cimex.
 Utövare av den mest brutala, råa rösten i kängens historia

 Han lever än idag. Med en konstant haschpsykos delar han ut tidningar och gömmer sej i buskar
 för att hoppa fram och skrika åt den första med anti-cimex merch hans ser!

}

\small{
\textbf{Jordbruksort}
\label{3257bf804d763afce5a153f73ce80f7c}
 En jordbruksort är ett antal åkrar där det också finns en väg och minst två postlådor.

}

\small{
\textbf{Josef fritzl}
\label{3f89c7ae4575ada3faf793ddb812a509}


}

\small{
\textbf{Josef Fritzl}
\label{3f89c7ae4575ada3faf793ddb812a509}
 Flerfaldigt prisbelönt familjefar från Österrike \textsc{(s.~\pageref{c7c58270ca7c339e744580f8a1bc04d2})}.

}

\small{
\textbf{Joseph Lucas}
\label{6c9df58fe0e834429e7767eed88b3138}
 (1834-1902) var en brittisk affärsman som genom sin firma Lucas Electric tillverkade elkomponenter åt finbilsmärken \textsc{(se putsbilar s.~\pageref{cb242351ab9f9d6b8a5afe8bed7b2dbd})} som Rolls-Royce, Jaguar, MG, Rover och Triumph. Vissa läsare känner honom förmodligen bättre vid hans smeknamn \quotetext{mannen som uppfann mörkret}. Som alla andra brittiska uppfinningar var hans elsystem nämligen onödigt komplicerade och sämre än exempelvis tyska motsvarigheter. Bland annat använde sig Lucas av positiv jord, vilket gjorde att elsystemen blev extra känsliga för korrision och lätt fick glappa kontakter (erkänner villigt att jag inte har en aning om vad korrision är men det låter inge bra). Till en början var detta inget större problem eftersom bilarna kördes av anställda chauförer som ofta hade en hel del dötid att meka på medan gentlemännen spelade hästpolo, hade möte med äventyrsklubben eller duellerade. Lucas själv avfärdade synpunkter på hans bilbelysning med att \textit{en gentleman kör inte bil efter mörkrets inbrott}. Lucas Industries levererade elsystem åt brittisk bilindustri in på 1980-talet då eldelen av firman såldes.

 Lucas avled den 27 december 1902 i sviterna av den tyfus han ådragit sig efter att ha avböjt ett glas vin och istället druckit smittat vatten. Hans barn ärvde dock firman och såg till att den dåliga tekniken levde vidare.

}

\small{
\textbf{Joulupukin}
\label{6729e155693750ec13cab64230b0971e}
 maa on Rovaniemellä asuu paikassa. Santa tykkää syödä puuroa ja veistää puukenkiä. Sivusto on myös koti tomtemor maa, joka tykkää kokata puuroa ja kävellä tukkii. Molemmat nauttivat pelaa suklaa pyörän Kiviks markkinoilla.

 Svensk översättning:
 Santa bor i Rovaniemi, är det land platsen. Santa gillar att äta gröt och rista trätofflor. Platsen är också hem tomtemor ett land som gillar att laga gröt och gå i träskor. Båda gillar att spela på choklad hjulet Kiviks marknad.

}

\small{
\textbf{Jourgatans ringlivs}
\label{74ac221e332167adfb88de92a2f25878}
 Vid en första anblick kan man tro att Ringatans Jourlivs är en ren pissbutik \textsc{(s.~\pageref{3be611a8fcd1a49d8aa59b77092c1bbe})}. De obscent höga folkölspriserna \textsc{(se folkölspriser s.~\pageref{88a1df48cd2dd24a91bc440e98851393})} och de oergonomiska hyllorna skänker inte något förtroendeingivande intryck. Man bedrar sig dock - inte nog med att Jourgatans Ringlivs troligen är den sista butiken i västvärlden som säljer pörr \textsc{(s.~\pageref{5faa435e2f0af7617816f0cade262581})} på vhs \textsc{(s.~\pageref{115751c3a0f5273a9e039b96b44250ae})}, man har även ett digert läsksortiment samt ett stort utbud av tobaksalternativ \textsc{(s.~\pageref{ce7311323773c585240e411b26fa7551})}. Oavsett om man är där för att hämta paket eller lämna in tipset är det alltid med ett visst vemod man lämnar butiken igen.
 HEAD3: Kuriosa:
 \begin{itemize}
 \item Man får två stora stark på Toscana vägg i vägg för samma peng som ett 6-pack på Ringlivs.
 \item Innan Toscana blev Toscana var det ett postkontor som rånades med jämna mellanrum av lokala busar \textsc{(se strulputte s.~\pageref{21651c95306d1b1e281443f8620910da})}.
 \item Ringlivs däremot var en välsorterad Konsumbutik \textsc{(s.~\pageref{70e4875f7c2c177596305006e46b7ca9})}.
 \end{itemize}

}

\small{
\textbf{Joxa med trasan}
\label{9723c16a2bdc64fedaed62da101691e5}
 Fotboll \textsc{(s.~\pageref{961bd74d34872ff94a4df3a16119096e})}.

}

\small{
\textbf{Jubal}
\label{5f7b7046fcb07abbe2448d98106037a0}
 är en biblisk \textsc{(se bibeln s.~\pageref{7de7d2a7d608c9a2044f50688bc63e27})} figur som enligt samma bok är stamfader till alla som spelar harpa och röker pipa. Att inte fler frifräsare predikar om honom tycker vi på Nissepedia \textsc{(s.~\pageref{62400dadecd90cb5cd39062abe5a3e4a})} är ganska B.
 Stonerrockens skyddshelgon.

}

\small{
\textbf{Judith Butler}
\label{259e843d41b50af014211ce25e80f6b5}
 är en amerikansk feministisk retoriker och filosof vars verk \quotetext{Könet brinner} och \quotetext{Genustrubbel}, med sina post-strukturalistiska förklaringar av kön och genus, skänkt stor lycka och framförallt otaliga huvudvärker till studenter över hela världen. Hennes i sammanhanget kändaste teori är perfomativitetsteorin som går ut på att kön/genus är något som spelas upp eller iscensätts, och då inte som sociologer som Goffman pratar om roller utan på ett ungefär fyrtiotusen miljarder \textsc{(s.~\pageref{c2160bffc9c5ca88e77204672e62e489})} gånger mer invecklat sätt. Som ett intressant stycke kuriosa fick hon denna snilleblixt då hon, efter att ha läst Hegel \textsc{(se Friedrich Hegel s.~\pageref{eadd964c0c4d2479bfe20e49c8921e77})} en hel dag, satt och bärsade och kollade på dragshow. En annan rolig grej är att hon i åtminstone ett sammanhang varit tvungen att säga att hon visst har en kropp.

}

\small{
\textbf{Julia Svan}
\label{63e9b7cb330e99fc4f7c4afdb7756056}
 är dotter till Gunde Svan \textsc{(s.~\pageref{f80f1875ab3ebccf935723ba83b6da63})}.

}

\small{
\textbf{Julian Assange}
\label{6e801fdf16285877d3644a86221e9d60}
 är en australiensisk man som processar mot länsstyrelsen \textsc{(se Processa mot länsstyrelsen s.~\pageref{0ae3fdeda52fe82800b04c624330139c})}. Han har även långt gångna planer att omvandla Ecuadors ambassad till en pudaslåda \textsc{(se Pudaslåda s.~\pageref{6a56958e2057dd500650e2be8049e033})}.

}

\small{
\textbf{Julpizza}
\label{fb3f91462b5fb94fc46e1bf72cb4f1d7}
 En julpizza innehåller saffransås, mos, köttbullar, prinskorv \textsc{(s.~\pageref{981d6501577f8e905435799959f99cb2})}, rödbetssallad, julskinka, rökt kalkon, rökt lax, anjovis, ägg \textsc{(s.~\pageref{128a5feb8e12d0aa622e0298a8332980})} med räkor i majonnäs. Skapelsen toppas med en skiva renstek. Den uppfanns i [http://www1.arvidsjaur.se/kamera.html Arvidsjaur] som har en uppsjö av pizzerior av varierande kvalité.



 HEAD2:  Källor

 [http://www.kuriren.nu/nyheter/default.aspx?articleid=6180168 Källa]

}

\small{
\textbf{Juridisk rådgivare}
\label{786e814361b08457a160c3e758833e1e}
 När man tittar på en film från USA är det alltid poliser \textsc{(se polis s.~\pageref{fa296149fa58bfd4408e407cc3fd3be5})} med, och så snart någon blir gripen kräver hen alltid att få ringa sin advokat. Vem fan har sin egen advokat? Advokater är ju snordyra och ber dig dra åt helvete \textsc{(se helvetet s.~\pageref{9c1e91d22a5df2b06a57fba276f94b5c})} så snart du slutar betala. Den som av någon anledning behöver gå i strid med lagen gör istället klokt i att skaffa sig en juridisk rådgivare. Till skillnad från advokat är juridisk rådgivare nämligen inte en skyddad titel så vem som helst kan verka inom skrået utan att behöva förhålla sig till dammiga idéer såsom hederskodex och yrkeslegitimation. Generellt behöver juridiska rådgivare inte heller kvitto på sina tjänster utan tar gärna ersättningen i trivselskrot \textsc{(s.~\pageref{6235563333e8dc26c9fc54e9e70c85ed})}, snus eller kattungar \textsc{(se kräftbete s.~\pageref{76499bc9cc050bed2beb8e36dd601066})}. De flesta har ganska mycket dötid att göra av med så det behöver inte bli så fasligt dyrt.

 Storheten hos en juridisk rådgivare ligger kanske inte främst i sin kunskap om paragrafer och vägledande domar utan mer i sin känsla för att ta ut svängarna och uppmuntra till sådant som andra inte skulle ha tänkt på. Den juridiska rådgivaren är dock ett tveeggat svärd och man bör vara beredd på att uppdragen ofta följer modellen \textit{ett steg bak, två steg fram}. Man bör också vara beredd på att framstegen kan te \textsc{(s.~\pageref{569ef72642be0fadd711d6a468d68ee1})} sig tvetydiga för utomstående. Till exempel när din juridiska rådgivare deklarerat att du inte accepterar löneförhöjningen på 2,5\% utan vill ha en ökning i samma takt som industriavtalet och får detta beviljat. När industriavtalet visar sig vara lägre än 2,5\%. Observera att din juridiska rådgivare aldrig kan hållas ansvarig för dåliga råd. Som man bäddar får man ligga och sover man i sovsäck får man sällan ligga.

}

\small{
\textbf{Jutis}
\label{e22cbc6c7bf2a278ba4374732fc1a6ae}
 är en by utanför Arjeplog, Norrbotten. Den är förmodligen mest känd för den s.k. kallade \quotetext{Jutiskalendern} där ortens män vek ut sig för den goda saken.[http://www.pitea-tidningen.se/nyheter/artikel.aspx?ArticleId=4769882]
 I Jutis finns även en tatueringsstudio.[http://www.imtattooing.com/]

}

\small{
\textbf{Järnspett}
\label{6cbe55f18d91c10e3307681ab810fd74}
 är ett gammalt beprövat verktyg som man kan använda till allt mellan himmel och jord bland annat göra hål i backen, använda som tyngd vid tillverkningen av pressgurka och kasta på dumma grannar. Men dess verkliga syfte är att luta mot glappa dörrar så som till vedbodar, ladugårdar och dass. Det har många finesser.

 Det finns ett talesätt som säger: Det finns många ting på ett spett, men det gör det inte så man kan ju undra vad det talesättet egentligen betyder.

 Precis som ett smidesstäd \textsc{(s.~\pageref{c3fed5991476f5dfb387dac8e88be084})} är spettets tydligaste egenskap dess oförmåga att flyta, så ser du något som flyter så är det troligen inte ett spett.

}

\small{
\textbf{Järspumpa}
\label{5cc916ec4b962224ba0be67a982f1587}
 En järspumpa är en olycklig syntes av pumpa och järs, men så är det här i herrens lustgård så det är inget att hålla på att göra sig omöjlig över.

}

\small{
\textbf{Jättedvärguv}
\label{4521dd9d46dad823cc2c4542223cf99d}
 en är en av våra mest motsägelsefulla uvar \textsc{(se uv s.~\pageref{45210da832f9626829457a65e9e7c4d0})}. Den är både stor och liten på samma gång, men utan att vara lagom.

 HEAD2: Föda
 Jättedvärguven äter uteslutande vegetarisk mat. Den äter också kopiösa mängder möss, sorkar, lämlar och annat uvgodis \textsc{(s.~\pageref{58de09e078ac891b067c0ec53d780b8a})}.

 HEAD2: Rede
 Den gillar att häcka vid bergsidor som ligger i totalt lä från vinden. Den har också ofta reden på högfjällslätter där det blåser något ohemult \textsc{(se ohemul s.~\pageref{91b8873590abd15ec344c2ba93d015cd})}.

 HEAD2: Häckning
 Uvmamman häckar hela tiden. Uvpappan häckar också hela tiden. Dock ryms bara en uv i redet samtidigt.

}

\small{
\textbf{Jättemyrslok}
\label{c6ca587e20a7103f6ea7f656968165bd}
 (\textit{Myrmecophaga tridactyla}) är ett djur inom familjen myrslokar, den största faktiskt. Den bor antingen i träskmarker eller på savannslätten \textsc{(se slätt s.~\pageref{a9cde01124ca41f23d6044b3ba27b979})}. Med svansen inräknad kan den blir nästan två meter lång. Den har inga tänder men den kompenserar detta med en jättelång tunga, som den använder till att äta myror med. När en unge föds ställer sig mamman på den och börjar riva den med sina enorma klor, precis som hon gör när hon ska riva upp en avokado[http://www.dn.se/nyheter/sverige/myrsloksunge-fodd-for-forsta-gangen-i-sverige]. I Sverige finns det två vuxna jättemyrslokar, dom heter Benita och Rozinski. Jättemyrslokens avsmalnande huvud gör att den var omöjlig att hålla kopplad, fram till dess att jättemyrsloksselen \textsc{(se jättemyrslokssele s.~\pageref{2c3f0277fc39cc38aa6c2420dee8c081})} uppfanns, och den har därför ännu inte blivit så populär bland barnfamiljer.

}

\small{
\textbf{Jättemyrslokssele}
\label{2c3f0277fc39cc38aa6c2420dee8c081}
 Jättemyslokssele är en uppfinning som gör det möjligt att hålla en jättemyrslok \textsc{(s.~\pageref{c6ca587e20a7103f6ea7f656968165bd})} kopplad. Selen uppfanns av Jan Guillou \textsc{(s.~\pageref{63f2c8aba9686bc92efeb7eb21e35156})} i samarbete med ett forskningsinstitut i Schweiz \textsc{(s.~\pageref{1686d12779e6c4574c24716d01189820})}.

}

\small{
\textbf{Jättemyrslokssäle}
\label{ccb71091d33335402ba73e0d778e3a31}
 Jättemyrslokssele \textsc{(s.~\pageref{2c3f0277fc39cc38aa6c2420dee8c081})}

}

\small{
\textbf{Jävelskap}
\label{46845591177f16920cd586a5baf5a625}
 är sånt man kan hitta på när man vill busa. Ibland händer det att det drabbar andra och det är lite olyckligt men ofta oundvikligt. Det första kända jävelskapet utfördes av en lokal byfåne i Chan Chan, nuvarande Peru \textsc{(s.~\pageref{32e8419a7ecb8f918c70fdadf783e3d8})}. Byfånen lär ha fångat ett bältdjur \textsc{(s.~\pageref{e57c0e34724e888178ffeff956101271})} och sedan gömt det i sin farfars mockasinlåda.

 HEAD2: Välkända jävelskap:

 \begin{itemize}
 \item Lasse Brandeby lurar Robert Gustavsson att dricka sig full i tv-serien Rena Rama Rolf.
 \end{itemize}

 \begin{itemize}
 \item Jöns Skalman får i uppdrag att rita ett första utkast till Vasaskeppet och lägger till ett extra kanondeck. Ingen upptäcker sprattet innan det är för sent.
 \end{itemize}

 \begin{itemize}
 \item Den första januari 1994 intar över 3000 beväpnade zapatister en rad olika städer och byar i Chiapas, Mexiko, för att dagen efter lämna dem igen.
 \end{itemize}

 \begin{itemize}
 \item Noshörningskungen Rataxes låter aldrig Babar vara ifred.
 \end{itemize}

 \begin{itemize}
 \item Direktörn \textsc{(s.~\pageref{307279f4fa4053e1e1aebc1649eea8b8})} anlitar baltiska svartarbetare för att, på pin tji, flyttstäda Lars rum när han är bortrest.
 \item Umeå kommun anlitar balticgtuppen för att skämta till det lite med sina invånare inför \textit{Kulturhuvudstad 2014}. Ingen är riktigt klar över vad skämtet består i. Men det är ju så det är med postmodern kultur så allt är i sin ordning. Att pengarna går in i fickorna på Krister Olsson är en del av det postmoderna skämtet.
 \end{itemize}

}

\small{
\textbf{Jörg Haider}
\label{f40f229da31c801dcc9f969165e7e31d}
 Bög, rattfyllerist och miljonär, såklart från Österrike \textsc{(s.~\pageref{c7c58270ca7c339e744580f8a1bc04d2})}.
 Dessutom kryptofascist.
 Haider bekräftade det som alla länge misstänkt, att folk med högerextrema böjelser oftast även har andra böjelser. Som tur är körde han av vägen med sin sportbil 2008.

}

\small{
\textbf{Jörgen}
\label{92592dfe96a3cac8ff3eae81584f9b42}
 är ett namn som funnits i Norden enda sedan Pommern var en del av Svea rike. Namnet är ett slanguttryck för singularformen av \quotetext{österrikare}.

}

\small{
\textbf{Kabyliska}
\label{42e8883f69e4ff3efc97d6d79efd9e54}
 är ett språk som talas av folkslaget Kabyler som bor i Algeriet, men själva menar de att de bor i landet Kabylien. Det talas av mellan 5 och 7 miljoner människor. Kabylerna har också undersökts av bondsonen, dåvarande antropologen, senare sociologiprofessorn och numera döingen Pierre Bourdieu. Vad han kom fram till är oklart, men bevisligen var det inte helt tokigt då han blev en högt aktad akademiker.

}

\small{
\textbf{Kaffe}
\label{a51a0cac0ce374a853d2359417debc28}
 är en magisk dryck som görs på rostade bär (folk tror oftast det är bönor men det är bär), och härstammar likt Miriam Makeba från Afrika. Den innehåller allt som en välanpassad människa behöver för att fungera på topp. Enligt en studie som publicerats på internet så fyller 0,5 l kaffe dagsbehovet av vitamin C, D, A och K samt de flesta spårämnen och energibehovet hos ett genomsnittligt vuxet barn.

 HEAD2: Serveringsförslag
 Kaffe är en dryck som går att kombinera med andra bra saker såsom kubb \textsc{(s.~\pageref{de7f6954ec8c6e346b8ba18ae018d334})} eller brännvin \textsc{(s.~\pageref{ff49ececa32cff978496a39635496f46})}. Kaffe dricks med fördel på fat \textsc{(s.~\pageref{068c450db48e3bfe2c97cd3ea4c0e083})}.

 HEAD2: Kaffe och samhället
 Arbetarklassen dricker stora mängder kaffe, man skulle kunna säga att vård, skola, omsorg och industri inte hade fått nåt gjort utan denna dryck. Medelklassen dricker också kaffe, men då är det såklart specialare som cappucino, wiener melange och latte macchiato. Café Latte har t o m blivit medelklassens främsta betecknare. Överklassen dricker inte kaffe, ännu en anledning att hata den.

}

\small{
\textbf{Kaffe kask}
\label{f017294802d98446f6b5e9c0cc37d6a1}
 Kaffekask \textsc{(s.~\pageref{ec14c8f3bcd9c059cdec884372eef42a})}

}

\small{
\textbf{Kaffefisk}
\label{af1258c212f378e0d974ac807a91ab79}
 är en sockerbit som varsamt sänkts ned till hälften i en slät kopp kaffe \textsc{(s.~\pageref{a51a0cac0ce374a853d2359417debc28})} och sedan lika försiktigt tagits upp igen. Nu har sockerbiten sugit åt sig kaffe och, likt Jan Björklund \textsc{(s.~\pageref{0b9b757044804b9be0e218acdad358cc})}, blivit ljusbrun. Kaffefisken kan man sedan kasta åt en unge eller en hund som man vill ska bli beroende av kaffe och/eller socker.

}

\small{
\textbf{Kaffekask}
\label{ec14c8f3bcd9c059cdec884372eef42a}
 Recept på rusdrycken kaffe kask.

 1. Fixa kaffe.
 2. Lägg ett mynt i en kopp, förslagsvis en tia \textsc{(s.~\pageref{e7292d5ba58672ce7f6fc3c0b646ab63})}.
 3. Slå i kaffe tills myntet inte syns.
 4. Slå i klar sprit tills myntet syns igen.
 5. Fyllna till.

}

\small{
\textbf{Kaffeost}
\label{4137c7d464c0451a1aecee557d79c868}
 är pricken över I:et, den lilla detaljen som förvandlar en kopp kaffe till en festmåltid.
 Man bör känna till att det finns två skolor av kaffeost, den östra vilken förekommer i östra delen av republiken Norrbotten föredrar ostar gjorda på kommjölk, stora som en tolva \textsc{(s.~\pageref{75e2490604087d3d303b09a98803a16b})} och gräddade i ugn. Den västra skolan föredrar ostar av getmjölk i storlek som en sjua \textsc{(s.~\pageref{e7bf63fa6d0d29bd89c23f833b979a15})}.
 Värdig representant för den östra skolan är Kangososten medan den västra företräds av Svartbergets.

 På grund av det rådande skymmningsläget kan det vara bra att veta hur man gör sin egen kaffeost ifall dom \textsc{(s.~\pageref{dd988cfd769c9f7fbd795a0f5da8e751})} lyckas skära av transportkanalerna.
 HEAD2: Recept på Kaffeost

 1 paket mjölk
 1 paket grädde
 löpe

 Blanda mjölk och grädde i en kastrull och värm försiktigt under omrörning till fingervärme. Lyft av värmekällan och tillsätt en tesked löpe, rör om och lägg på ett lock. Låt stå. Lyft på locket,ser det grynigt ut? -Bra. Häll allt i en finmaskig sil och pressa ut så mycket vassle som möjligt. Lägg på ett kaffefat, pressa med en tyngd och låt stå svalt över natten.

}

\small{
\textbf{Kafkafoni}
\label{c0b24f40409a2941ea69033b5a3a600e}
 Att lyssna på flera ljudböcker av författaren Franz Kafka på samma gång.

}

\small{
\textbf{Kakluckan}
\label{7fd014af9490d51f96eba2368ecffc71}
 Kaklucka är en medicinsk term för människans mun \textsc{(s.~\pageref{6585f290ce92c3de5ff339920330e26f})}.

}

\small{
\textbf{Kaknästornet}
\label{ffe3ac06a304714dcee7cbbdfeb20d84}
 är tillsammans med Globen \textsc{(s.~\pageref{c520b11670b9cef944588fe3849ce491})} Sveriges svar på Paris' Eifeltorn, Londons Big Ben och Kuala Lumpurs Petronas Twin Tower. Med sin majestätiska höjd på 155 meter tornar byggnaden upp sig över Ladugårdsgärdet i Stockholm. Det invigdes 1967 av den dåvarande kommunikationsministern Olof Palme \textsc{(s.~\pageref{702b78623785546fb9c9890222376178})} och blev snart ett attraktivt besöksmål för tillresta turister. Det är brunt och gjutet av betong. Förutom att placera Stockholm på världskartan och fungera som ett av rikets viktiga och identitetsskapande landmärken är tornets funktion att fungera som restaurang, presentshop och radio- och TV-mast. Härifrån sänds bland annat UR, TV4 och ett antal reklamfinansierade lokalradiostationer till Stockholmarnas glädje.

 Kaknästornets hemsida: [http://www.kaknastornet.se/]

}

\small{
\textbf{Kaksmälla}
\label{fbd452dca27fd571b729571263c7e2bb}
 är det tillstånd av dåsighet och lätt illamående som inträder då man stoppat in lite för många kakor i kakluckan \textsc{(s.~\pageref{7fd014af9490d51f96eba2368ecffc71})}. Inträffar ofta efter ett bullmangel \textsc{(se mangel s.~\pageref{ecc5b41821ed829b0c3fb48d4d5389ed})}.

}

\small{
\textbf{Kalaskula}
\label{e889c1a4915c4b4aad08d49192e79276}
 kallas en mage som är alldeles uppspänd, hård och rund. Kalaskulan är ett attribut som i dagens samhälle ofta innehavs av herrar i 40-års åldern. Ursprungligen var kalaskulan en sorts dyrkan av urmodern, Gaia. I jägar- och samlarsamhället skaffade sig alla som kunde en ordentlig kalaskula för att hylla kvinnans fertilitet och förmåga att skapa nytt liv. När en medlem (man eller kvinna) i stammen skaffat sig en mage som motsvarade den hos en kvinna i åttonde månadens graviditet samlades stammen för att dansa och äta. Man dansade så ogenerat som man bara kunde göra i en tid utan modebloggar och popfanzines och förlustade sig efteråt på saltat grävlingkött, blötlagd lök och dryck gjord på jästa sviskon.

 Av oklara anledningar har den kvinnohyllande delen i att ha en kalaskula försvunnit under historiens gång. Snarare är kalaskulan idag en hyllning till den fetlagde gubben, då en ogenerad övervikt hos herrar sägs signalera \quotetext{pondus}, samtidigt som den hos en kvinna signalerar \quotetext{stigmatiserad bärshagga}.

}

\small{
\textbf{Kalixare}
\label{4b4f527a0ea88003ec5806fd29f6b6fe}
 En kalixare är att kupera en kortlek genom att ta ut en bunt kort ur mitten av kortleken och placera dessa längst upp. Det är annars brukligt att kupera genom att lägga den nedre halvan av kortleken längst upp. En bra kalixare har lite snärt och flärd.

 HEAD2: Etmyologi
 Kalixare heter som det gör för att människor från Kalix brukar kupera kortlekar så, i alla fall på jägarregementet i Kiruna på 50-talet.

}

\small{
\textbf{Kalle anka}
\label{64db68f686a0ca4d9d641061cb3fdf13}
 är ett sätt att klä sig där överkroppen är påklädd och underkroppen bar. Klädseln är flitigast använd på festivaler och svensexor men passar i princip till alla tillfällen. Motsatsen till att klä sig Kalle Anka är att klä sig Mimmi Pigg \textsc{(s.~\pageref{47a20f7432f125f29ac8d0101be60ad7})}.
 HEAD2:  Se även
 Sans pants \textsc{(s.~\pageref{e690d08a3200d783d98b198f0354bc85})}.

}

\small{
\textbf{Kalle ankamat}
\label{372adec9335daa7cc03b9e921bdebd83}
 Typisk Kalle Ankamat är:
 \begin{itemize}
 \item Sviskon
 \item Läskeblask
 \item Aladåb
 \end{itemize}

 Kalle Ankamat förekommer frekvent i Ankeborg men sällan annorstädes.

}

\small{
\textbf{Kalle Ankas pocket}
\label{9d006a462dd4bddaa433ca106bef2a34}
 \textit{Kalle Ankas pocket} (hädanefter refererad till som KAP) är en tidskrift och systerpublikation till \textit{Kalle Anka \textsc{(se kalle anka s.~\pageref{64db68f686a0ca4d9d641061cb3fdf13})} \& CO}. I KAP erbjuds dock betydligt mer djuplodande porträtt om Kalle och hans värld, och det ställs därmed också större krav på läsarens intellekt. I KAP räcker det inte att bara titta på bilderna, som i en vanlig trerutorsstripp där kusin Knase halkar på ett bananskal \textsc{(se banan s.~\pageref{aec7bd708ed2ad3435b9a9883ac7f45c})}. Istället förväntas läsaren själv delta i skapandet av upplevelsen genom att läsa i pratbubblorna och memorera vad som hände mer än tio sidor bakåt. I gengäld blir helhetsintrycket desto starkare och trogna läsare som verkligen ansträngt sig berättar om spirituella upplevelser där de verkligen trott sig sitta i gamla 313 eller stulit en paj hos Farmor Anka. De lite tjockare permarna på KAP gör också att de klarar sig betydligt längre än andra magasin inne på muggen innan de blir vattenskadade.

}

\small{
\textbf{Kanadensisk frack}
\label{365eda0bac157feb7ee22ceb7f07e17d}
 Jeans, jeansskjorta och jeansjacka.

}

\small{
\textbf{Kanotjeans}
\label{6424bde3cd847250427bb9c21fb5f840}
 Ett par kanotjeans är ett par jeans \textsc{(s.~\pageref{a0f2589b1ced4decbf8878d0c3b7986f})} som användaren bara har när denne snickrar på sin kanot. Det är som ni förstår ett synnerligen exklusivt plagg, men bli inte förvånade om typ Acne \textsc{(s.~\pageref{450e166c06161deddfc97749332c61cb})} eller Denim Demon börjar sälja förnötta kanotjeans inom några år. Den enda som Nissepedia obetalda researchare hittat som äger ett par kanotjeans, är Slisken \textsc{(se Användare: Slisken s.~\pageref{0434b6e7c92786761d7fb5b1e5e0dd3d})}.

}

\small{
\textbf{Kappgöra}
\label{144987cb6ef404616d5a658fe53ef258}
 Att kappgöra något är när minst två personer gör något samtidigt på ett tävlingsinriktat manér. Det går t.ex. att kappgrilla, kappdricka sprit, kappköra bil, kappskriva avhandling och kappvara dryg.

 I september var fjärde år så kappröstar hela svenska folket om vilka som ska få kontrollera deras liv de fyra kommande åren.

}

\small{
\textbf{Kapten Haddocks samlade svordomar}
\label{86bca93bf3f7151c4a37189a6bba1047}
 Kapten Haddocks samtliga svordomar i hittills två översättningar till svenska.

 Karin och Allan B. Janzons översättning:

 \begin{itemize}
 \item Anfäkta och anamma
 \end{itemize}

 \begin{itemize}
 \item Anamma och regera alla världens andemakter
 \end{itemize}

 \begin{itemize}
 \item Amiral Nelsons alla bomber och granater och kanoner
 \end{itemize}

 \begin{itemize}
 \item Amöbor
 \end{itemize}

 \begin{itemize}
 \item Apsvansade analfabeter
 \end{itemize}

 \begin{itemize}
 \item Apspektakel
 \end{itemize}

 \begin{itemize}
 \item Asgamar
 \end{itemize}

 \begin{itemize}
 \item Avskyvärt
 \end{itemize}

 \begin{itemize}
 \item Babian
 \end{itemize}

 \begin{itemize}
 \item Bandit
 \end{itemize}

 \begin{itemize}
 \item Banditskojaren
 \end{itemize}

 \begin{itemize}
 \item Bestar
 \end{itemize}

 \begin{itemize}
 \item Bladlus
 \end{itemize}

 \begin{itemize}
 \item Blindstyren
 \end{itemize}

 \begin{itemize}
 \item Blixt, brak och dunder
 \end{itemize}

 \begin{itemize}
 \item Blodsugare
 \end{itemize}

 \begin{itemize}
 \item Blåkullatomtar
 \end{itemize}

 \begin{itemize}
 \item Blötdjur
 \end{itemize}

 \begin{itemize}
 \item Bomber och granater och krevader
 \end{itemize}

 \begin{itemize}
 \item Bondlurkar
 \end{itemize}

 \begin{itemize}
 \item Bovar
 \end{itemize}

 \begin{itemize}
 \item Bålnötter
 \end{itemize}

 \begin{itemize}
 \item Bödel
 \end{itemize}

 \begin{itemize}
 \item Deghögar
 \end{itemize}

 \begin{itemize}
 \item Det var som topp tusen tunnor
 \end{itemize}

 \begin{itemize}
 \item Drummel
 \end{itemize}

 \begin{itemize}
 \item Drönare
 \end{itemize}

 \begin{itemize}
 \item Dunder och brak
 \end{itemize}

 \begin{itemize}
 \item Du är en pesto...klyster
 \end{itemize}

 \begin{itemize}
 \item Dyngspridare
 \end{itemize}

 \begin{itemize}
 \item Dårhushjon
 \end{itemize}

 \begin{itemize}
 \item Död och pina
 \end{itemize}

 \begin{itemize}
 \item Eländiga kryp
 \end{itemize}

 \begin{itemize}
 \item Enögda kannibal
 \end{itemize}

 \begin{itemize}
 \item Erbarmliga plattfötter
 \end{itemize}

 \begin{itemize}
 \item Far och flyg
 \end{itemize}

 \begin{itemize}
 \item Fega ynkryggar
 \end{itemize}

 \begin{itemize}
 \item Fåntrattar
 \end{itemize}

 \begin{itemize}
 \item Fähund
 \end{itemize}

 \begin{itemize}
 \item Fördömda kräk
 \end{itemize}

 \begin{itemize}
 \item Förpiskade luspudlar
 \end{itemize}

 \begin{itemize}
 \item För hundra gubbar
 \end{itemize}

 \begin{itemize}
 \item För sjutton hakar
 \end{itemize}

 \begin{itemize}
 \item Gamla knölsvan
 \end{itemize}

 \begin{itemize}
 \item Gangster
 \end{itemize}

 \begin{itemize}
 \item Gargantuaner
 \end{itemize}

 \begin{itemize}
 \item Giftblåsan
 \end{itemize}

 \begin{itemize}
 \item Gnom
 \end{itemize}

 \begin{itemize}
 \item Gråsuggor och spindelapor
 \end{itemize}

 \begin{itemize}
 \item Gravade oxsvansar
 \end{itemize}

 \begin{itemize}
 \item Grobianer
 \end{itemize}

 \begin{itemize}
 \item Groteskt
 \end{itemize}

 \begin{itemize}
 \item Grottmänniskor
 \end{itemize}

 \begin{itemize}
 \item Gråsuggor och dreglande paddor
 \end{itemize}

 \begin{itemize}
 \item Gräsligt
 \end{itemize}

 \begin{itemize}
 \item Gyckel, båg och ordkonster
 \end{itemize}

 \begin{itemize}
 \item Hamstrare
 \end{itemize}

 \begin{itemize}
 \item Huggormars avföda
 \end{itemize}

 \begin{itemize}
 \item Idiot
 \end{itemize}

 \begin{itemize}
 \item Injsögangster
 \end{itemize}

 \begin{itemize}
 \item I alla milda makters namn
 \end{itemize}

 \begin{itemize}
 \item Jäklar
 \end{itemize}

 \begin{itemize}
 \item Jäkla general
 \end{itemize}

 \begin{itemize}
 \item Jäkla odjur
 \end{itemize}

 \begin{itemize}
 \item Jämmer och elände
 \end{itemize}

 \begin{itemize}
 \item Kanalier
 \end{itemize}

 \begin{itemize}
 \item Kanapéer och konjakskransar
 \end{itemize}

 \begin{itemize}
 \item Kannibal
 \end{itemize}

 \begin{itemize}
 \item Karnevalsdiktatorn
 \end{itemize}

 \begin{itemize}
 \item Kloakdjur
 \end{itemize}

 \begin{itemize}
 \item Kloakråtta
 \end{itemize}

 \begin{itemize}
 \item Klorlutsoppan
 \end{itemize}

 \begin{itemize}
 \item Kramsfågelmördare
 \end{itemize}

 \begin{itemize}
 \item Krevader och kanoner
 \end{itemize}

 \begin{itemize}
 \item Kryddkrämare
 \end{itemize}

 \begin{itemize}
 \item Krämare
 \end{itemize}

 \begin{itemize}
 \item Legodrängar
 \end{itemize}

 \begin{itemize}
 \item Lekstugefasoner
 \end{itemize}

 \begin{itemize}
 \item Lumphandlare
 \end{itemize}

 \begin{itemize}
 \item Luspudlar
 \end{itemize}

 \begin{itemize}
 \item Luskungar
 \end{itemize}

 \begin{itemize}
 \item Lymmel
 \end{itemize}

 \begin{itemize}
 \item Lögnhalsar
 \end{itemize}

 \begin{itemize}
 \item Lönnmördare
 \end{itemize}

 \begin{itemize}
 \item Maskar
 \end{itemize}

 \begin{itemize}
 \item Milda makter
 \end{itemize}

 \begin{itemize}
 \item Murmeldjur
 \end{itemize}

 \begin{itemize}
 \item Möshumlor
 \end{itemize}

 \begin{itemize}
 \item Nu blommar asfalten
 \end{itemize}

 \begin{itemize}
 \item Nu går skam på torra land
 \end{itemize}

 \begin{itemize}
 \item Ockrare
 \end{itemize}

 \begin{itemize}
 \item Oduglingar
 \end{itemize}

 \begin{itemize}
 \item Orangutang
 \end{itemize}

 \begin{itemize}
 \item Parasiter
 \end{itemize}

 \begin{itemize}
 \item Pestråtta
 \end{itemize}

 \begin{itemize}
 \item Pest och pina
 \end{itemize}

 \begin{itemize}
 \item Pillerbagge
 \end{itemize}

 \begin{itemize}
 \item Pirater
 \end{itemize}

 \begin{itemize}
 \item Pistaschgubbar och surkålsstuvning
 \end{itemize}

 \begin{itemize}
 \item Plattfötter
 \end{itemize}

 \begin{itemize}
 \item Potatisgrisar
 \end{itemize}

 \begin{itemize}
 \item Pottsorkar
 \end{itemize}

 \begin{itemize}
 \item Pyromanapa
 \end{itemize}

 \begin{itemize}
 \item Rena rävgiftet
 \end{itemize}

 \begin{itemize}
 \item Rena sammansvärjningen
 \end{itemize}

 \begin{itemize}
 \item Rena snurren
 \end{itemize}

 \begin{itemize}
 \item Rena vansinnet
 \end{itemize}

 \begin{itemize}
 \item Rotborstar
 \end{itemize}

 \begin{itemize}
 \item Rullsvansapa
 \end{itemize}

 \begin{itemize}
 \item Råttsvansar
 \end{itemize}

 \begin{itemize}
 \item Rödskinnen
 \end{itemize}

 \begin{itemize}
 \item Rötägg
 \end{itemize}

 \begin{itemize}
 \item Sakramentskade sumprunkare
 \end{itemize}

 \begin{itemize}
 \item Sabotörer och pestråttor
 \end{itemize}

 \begin{itemize}
 \item Sladderfågel
 \end{itemize}

 \begin{itemize}
 \item Sillmjölkar
 \end{itemize}

 \begin{itemize}
 \item Skabbhalsar
 \end{itemize}

 \begin{itemize}
 \item Skabbråttor
 \end{itemize}

 \begin{itemize}
 \item Skamligt
 \end{itemize}

 \begin{itemize}
 \item Skojare
 \end{itemize}

 \begin{itemize}
 \item Skottkolvar
 \end{itemize}

 \begin{itemize}
 \item Skurkar
 \end{itemize}

 \begin{itemize}
 \item Skunkdjur
 \end{itemize}

 \begin{itemize}
 \item Slavhandlare
 \end{itemize}

 \begin{itemize}
 \item Slyn-yngel
 \end{itemize}

 \begin{itemize}
 \item Snack och strunt och snack
 \end{itemize}

 \begin{itemize}
 \item Snorvalp
 \end{itemize}

 \begin{itemize}
 \item Snyltror på ädel tistelstam
 \end{itemize}

 \begin{itemize}
 \item Sopprötter
 \end{itemize}

 \begin{itemize}
 \item Sumpråtta
 \end{itemize}

 \begin{itemize}
 \item Sötvattenspirater
 \end{itemize}

 \begin{itemize}
 \item Tusan
 \end{itemize}

 \begin{itemize}
 \item Tjockskalle
 \end{itemize}

 \begin{itemize}
 \item Tjurskalle
 \end{itemize}

 \begin{itemize}
 \item Tjuvar
 \end{itemize}

 \begin{itemize}
 \item Tryffelsvin
 \end{itemize}

 \begin{itemize}
 \item Tångräka
 \end{itemize}

 \begin{itemize}
 \item Urfånigt
 \end{itemize}

 \begin{itemize}
 \item Uslingar
 \end{itemize}

 \begin{itemize}
 \item Vagabonder
 \end{itemize}

 \begin{itemize}
 \item Vandaler
 \end{itemize}

 \begin{itemize}
 \item \textbf{Vegetarian}
 \end{itemize}

 \begin{itemize}
 \item Vidriga apmänniska
 \end{itemize}

 \begin{itemize}
 \item Vidriga varulvar
 \end{itemize}

 \begin{itemize}
 \item Vinlus
 \end{itemize}

 \begin{itemize}
 \item Vrakplundrare
 \end{itemize}

 \begin{itemize}
 \item Vrålapor
 \end{itemize}

 \begin{itemize}
 \item Vårtsvin
 \end{itemize}

 \begin{itemize}
 \item Åsneskallar
 \end{itemize}

 \begin{itemize}
 \item Ärkebanditer
 \end{itemize}

 \begin{itemize}
 \item Ärkeboven
 \end{itemize}

 \begin{itemize}
 \item Ärkelögnare
 \end{itemize}

 Per Kellbergs översättning:

 \begin{itemize}
 \item Anåda och anagga
 \end{itemize}

 \begin{itemize}
 \item Avsigkomne
 \end{itemize}

 \begin{itemize}
 \item Blixt och dunder och miljoner tyfoner
 \end{itemize}

 \begin{itemize}
 \item Blixt och skatter
 \end{itemize}

 \begin{itemize}
 \item Bläckfiskar
 \end{itemize}

 \begin{itemize}
 \item Din olycka
 \end{itemize}

 \begin{itemize}
 \item Din skråpuk
 \end{itemize}

 \begin{itemize}
 \item Ditt bjäbbande spöke
 \end{itemize}

 \begin{itemize}
 \item Ditt enödga murmeldjur
 \end{itemize}

 \begin{itemize}
 \item Era tjattrande kariblar
 \end{itemize}

 \begin{itemize}
 \item För bövelen
 \end{itemize}

 \begin{itemize}
 \item Fördubblade babianer
 \end{itemize}

 \begin{itemize}
 \item Fördubblade galenskaper
 \end{itemize}

 \begin{itemize}
 \item För höge farao
 \end{itemize}

 \begin{itemize}
 \item Gorillor
 \end{itemize}

 \begin{itemize}
 \item Iglar och blodsugare
 \end{itemize}

 \begin{itemize}
 \item Jag ska göra papegojpaj av hela rasket
 \end{itemize}

 \begin{itemize}
 \item Knölfötter
 \end{itemize}

 \begin{itemize}
 \item Kornblixtar och tyfoner
 \end{itemize}

 \begin{itemize}
 \item Kors i världen
 \end{itemize}

 \begin{itemize}
 \item Kreatur
 \end{itemize}

 \begin{itemize}
 \item Lomhörda lymmel
 \end{itemize}

 \begin{itemize}
 \item Maneter
 \end{itemize}

 \begin{itemize}
 \item Markattor
 \end{itemize}

 \begin{itemize}
 \item Monster och morsgrisar
 \end{itemize}

 \begin{itemize}
 \item Passgångarna
 \end{itemize}

 \begin{itemize}
 \item Pottsorkar och palsternackor
 \end{itemize}

 \begin{itemize}
 \item Sjögurkor
 \end{itemize}

 \begin{itemize}
 \item Småfjantar
 \end{itemize}

 \begin{itemize}
 \item Stendöva stallknekt
 \end{itemize}

 \begin{itemize}
 \item Stinkbomben
 \end{itemize}

 \begin{itemize}
 \item Tokskallar
 \end{itemize}

 \begin{itemize}
 \item Vedervärdige
 \end{itemize}

 \begin{itemize}
 \item Vid alla nerlusade sopprötter
 \end{itemize}

 \begin{itemize}
 \item Vid alla piskande råttsvansar
 \end{itemize}

 Källa[http://www.sallander.nu/tintin/text/svor.html]

 En nyöversättning av Tintin pågår.[http://www.asterion.se/tintincomparison.html] Om översättarna ändrar Haddocks klassiskt återkommande svordomar, såsom \quotetext{bomber och granater}, väntar troligen åtal för kulturmord.

 Kuriosa rörande svordomarna saxat från Wikipedia: \textsc{(se Wikipedia s.~\pageref{12672e79b01e9ca7018105efb0ef871c})}

 \textit{Kapten Haddocks svordomar (som naturligtvis inte översatts ordagrant) finns i sin franska originallydelse katalogiserade och förklarade i alfabetisk ordning i en volym på 93 sidor av Albert Algoud: Le Haddock illustré, L'intégrale des jurons du capitaine, ISBN 2-203-01710-4 (\quotetext{Haddock i bild; Kaptenens samlade svordomar})}]

}

\small{
\textbf{Karbinhake}
\label{1c28d3787ab93954ddaa1cd87597a83c}
 En karbinhake användes förr i tiden, precis som namnet antyder, för att haka fast gevär (karbiner) på. Idag används karbinhaken i stort sett uteslutande av veganpunkare vilka genom att fästa sina nycklar i haken och sedan haka fast den i sina byxor använder nycklarnas högljudda rasslande som ett sorts parningsläte, då användarna hoppas att oväsendet ska dra omgivningens uppmärksamhet till deras oansenliga ändalykter. Vad karbinhakens uppfinnare hade sagt om hur den används idag är oklart, men man kan tänka sig att inte hade varit übernöjd direkt.

 HEAD2: Karbinhaken inom idrotten
 Karbinhaken är också populär bland bergsklättrare som använder dessa för att liksom sätta fast sig i berget så att de inte ramlar och slår ihjäl sig. Det bör dock höjas ett varningens finger mot att sätta fast sig med karbinhaken man har sina nycklar i, för det är lite jobbigt att hämta dem igen.

}

\small{
\textbf{Karel Gott}
\label{4be86a4c12ad816d5fca53719a4da3a5}
 (f. 1939), även kallad \quotetext{The Golden Voice of Prague} och \quotetext{Sinatra of the East} (på sin hemsida [http://www.karelgott.net/]), är en tjeckisk musiker som kom på plats 13 i Eurovisionschlagerfestivalen 1968.

}

\small{
\textbf{Karl Marx}
\label{e93e240d0bcb37baa1a0169598a9145d}
 Holländsk doom metal-sångare. Hellre än bra.

}

\small{
\textbf{Kasta ut barnet med badvattnet}
\label{b88127ad3d7f8a41176e115ca8fb7c9f}
 Kommer vara betydligt vanligare om Kristdemokraterna \textsc{(s.~\pageref{18a843e4776b5003d411ce0148bab148})} får som de vill.

}

\small{
\textbf{Kastrup}
\label{28456a2f66ab9d5114033a146b593850}
 ær en flygplats i Danmark \textsc{(s.~\pageref{5331d7fd27772396f412a5b6d19bad44})}, nordens lustigaste och obehagligaste land. På Kastrup kan man surfa internet på 7-eleven, men det ær inte superfett, før man kan inte sitta ner nær man gør det. Man kan æven stå och snacka på kass engelska med en dam i biljettkassan och førsøka gøra sig førstådd. Det ær inte alltid jættelætt. Ibland kan man också inleda diskussioner med personalen på engelska, bara før att efter några meningar bryskt gå øver i svenska, då personalen ganska lætt mærker att man ær svenne. Om man vill gå och bajsa på herrtoan går det riktigt bra. Det ær næmligen den finaste hitills dokumenterade platsen på Kastrup. Rymligt, ljust och med en intressant diskussion om huruvida catalonien ær en del av Spanien \textsc{(s.~\pageref{84c63835ca2fcac8636cf7d36aa48fa4})} eller ej klottrad på toadørrens insida(det sitnæmnda gæller dock endast den mittersta av de tre herrtoaletterna).

 Utanfør det stora komplexet finns en fontæn som inte sprutar men fortfarande ær lite fuktig. Den kan man halka och slå sig på.

}

\small{
\textbf{Katolik}
\label{75d0665472f571956570c00f3fccfbc2}
 En katolik, eller papist \textsc{(s.~\pageref{9e0a37391a1827c035ec21cb07a39853})} som man också kan kalla dem, är en kristen person som inte vill göra avkall på två av sina största intressen, nämligen att dricka brännvin och ha så mycket oskyddad sex som är praktiskt möjligt. Katoliken ägnar sig åt detta i vanlig ordning och biktar sig sedan för sin präst och blir då förlåten för att ha sex och vinfylla \textsc{(s.~\pageref{aee462fab19723e71e7f1f3302309d1e})} som huvudintressen. Den katolskt kristne kan kanske ses som en mindre seriös troende av sina protestantiska bröder och systrar, men detta är inte helt sant. Katoliken är nämligen en engagerad motståndare till homosexualitet, abort, kondomer, icke-traditionella familjekonstellationer och mycket annat som, kan man väl tycka, det krävs lite jävlaranamma för att vara emot så här ett decennium efter millenieskiftet. Katoliken är dock bergfast i sin beslutsamhet när det gäller dessa saker, bara inte just när det gäller att dricka alkohol \textsc{(s.~\pageref{11c589cba1a208e0359048a78e6b88b8})} och knulla runt med folk (av motsatt kön).

}

\small{
\textbf{Katt}
\label{0fd9accd1d8c95e86a96f681b6805948}
 (obestämd form singular; plural kateter) är ett fyrbent djur inom familjen morrhåringar. Mången katt har lurvig päls vilket gör det lätt att även som lekman kunna åldersbestämma den, ju tovigare päls desto äldre. Katten föds genom att modern efter befruktning hostar upp ungen i form av en hårboll som sedan planteras i kattsand \textsc{(s.~\pageref{6e6a2ba3be745f1d81eb854ceb010c98})} för att ligga där och gro några veckor. Träffar man en katt bör man tilltala den Misse eller Jamis då arten haft väldiga svårigheter att inrätta sig i 1900-talets Du-reform. På landet är följande kattnamn frekvent förekommande: Fräsen, Sotis, Maja / Majsan, Pelle, Katta, Kattskrället, Kattfan samt alla namn som innehåller ett eller flera S då katter lystrar till detta.[http://www.agria.se/Agria/text.nsf/id/4621] Som kuriosa kan nämnas att Sven-Jerry är det frekventaste kattnamnet i Skåne enligt Skånes kattklubb.[http://www.skkk.se/index1.php]



 Andra kattsorter:
 Katekes
 Katamaran
 Katrinplommon

}

\small{
\textbf{Kattbrosch}
\label{a3d22061739404335c4675738803b886}
 er är ofta av lackat trä med en tydlighet över sig. Det kan vara ett par distinkta morrhår och eller en  färgglad detalj. Kattbroschtillverkarna vänder sig i första hand till den konstnärliga och lekfulla kundkretsen.

}

\small{
\textbf{Kattguld}
\label{aa62881467ab5ca635f7606d3ffd1321}
 är en populär sulfidmineral som finns lättillgänglig över större delen av jorden. Precis som vanliga katter \textsc{(se katt s.~\pageref{0fd9accd1d8c95e86a96f681b6805948})} och vanligt guld framstår kattguld vid en första anblick som väldigt fint. Men vid en närstudie märker man att det är rätt överskattat och inte värt besväret. Carl von Linné \textsc{(s.~\pageref{5e8380bf6b7ce99678e6752b6d9e709e})} var så klart tidigt framme och forskade i kattguldets mysterium och kom bland annat fram till att det inte är särskilt hälsosamt att äta och att barn börjar gråta om man slänger bitar större än en knytnäve på dem. Den franske \textsc{(se frankrike s.~\pageref{8a28b520a53cd68763ebf19b5599412b})} upptäcktsresanden Jacques Cartier (1491-1557) föddes tyvärr innan Linné upptäckt dessa egenskaper hos metallen och trodde att han hittat en förmögenhet när han under en resa i nuvarande Kanada seglade förbi en strand täckt av kattguldsklimpar. Han släpade hem en hel skeppslast till Frankrike \textsc{(s.~\pageref{8a28b520a53cd68763ebf19b5599412b})} där kungen blev måttligt imponerad vid upptäckten att han langat upp en brakmiljard för att Cartier skulle segla halva jorden runt och hämta hem sten lika uppskattad som ringaren i Notre-Dame. Från denna utflykt härstammar uttrycken \quotetext{guld från Kanada} och \quotetext{Fransk upptäcktsresande}.

}

\small{
\textbf{Kattkvinnan}
\label{78747f76146d2f60b39c8a4d8c1bdcd0}
 Aspirerar på att vara en alv, skriver långa dikter och cyklar runt iförd multikultikläder.
 Förutom att föra alvernas talan är hon någon form av kattombudskvinna som lockar till sig intet ont anande katter, likt zombies vandrar fryntliga bondkatter mot hennes regnbågs och enhörningsbeprydda boning för att sedan försvinna.

}

\small{
\textbf{Kattsand}
\label{6e6a2ba3be745f1d81eb854ceb010c98}
 är en massa avsedd att förenkla hanteringen av kattexkrementer inomhus. Det är inte säkert att kattsand just är sand \textsc{(s.~\pageref{88336b5bb2a1cc21bac7cf33fd451270})}, det kan också vara gjort av pappersmassa eller träprodukter. När det gäller \quotetext{vanlig} kattsand, kan man dela upp dessa i två olika grupper, klumpande \& icke-klumpande. Klumpande sand bildar klumpar, mer eller mindra hållbara, när urin eller annan vätska kommer i kontakt med sanden. Bra kattsand dammar så lite som möjligt, är lätt att bära hem och håller lukten bra.


 Det är viktigt att katten \textsc{(se katt s.~\pageref{0fd9accd1d8c95e86a96f681b6805948})} trivs med sin kattsand för att inte riskera att den börjar förrätta sina behov annorstädes. Då katter \textsc{(se katt s.~\pageref{0fd9accd1d8c95e86a96f681b6805948})} saknar utvecklat språk och bara kan kommunicera genom jamande och spinnande är det dock svårt att fråga den. Ett vanligt sätt att ta reda på vilken typ av sand den föredrar är att prova sig fram. Är katten \textsc{(se katt s.~\pageref{0fd9accd1d8c95e86a96f681b6805948})} en diva och vägrar använda någon av sandtyperna kan man enkelt straffa den genom att spruta en klick senap i röven.


 HEAD2:  Icke klumpande

 Miss Cat t ex är en icke klumpande kattsand. Urin \textsc{(s.~\pageref{524fd7acb94f9c2d879b5c1cf8335669})} och avföring måste tas bort oftare än för klumpande kattsand, kattlådan behöver också rengöras oftare. En fördel med denna sand är att den inte är lika farlig för små kattungar som kan äta kattsand - den bildar inga klumpar i den lilla magen.



 HEAD2:  Klumpande

 Till denna grupp hör t.ex. Everclean, Pussi \textsc{(se framstjärt s.~\pageref{843a33cdc4d90488cea3030f8b941e08})}, Zoogillets, Katty och Husse. Kattsanden består av bentonitlera som klumpar när den kommer i kontakt med urinen. Klumparna liksom avföringen plockas bort var dag, och ca 1 ggr i veckan töms hela kattlådan och rengörs med såpa och vatten.



 HEAD2:  Pappersmassa

 Toa-lätt tillhör denna grupp. Toa-lätt är en restprodukt vid pappersmassatillverkning, och består av ren cellulosa och kaolin. Detta torkas, och i säcker hittar vi kattägare cylinderformade pellets. Toa-lätt är förnyelsebar, dammfri och lätt att bära hem. Vill man göra egen pappersmassa pular man bara ner några äggkartonger i en matberedare och blåser på för fullt i någon minut.



 HEAD2:  Trä

 Katt Skogsströ och Wood är två exempel på träprodukter. En ganska ny produkt som kom i slutet av 2001 är PeeWee som består av speciell kattlåda och träbaserat strö. Kattlådan får en påtaglig skogsdoft, och när urin kommer i kontakt med dessa material, smulas det sönder, och faller till botten av lådan. Det är komposterbart och lätt att bära hem. Bränslepellets fungerarar också på detta sätt och är dessutom billigt. Ett varningens finger bör dock höjas mot att elda med den efteråt då det kan lukta nå förjävligt. Bor man i Kärrgruvan kan man gå till brikettfabriken med en plastpåse och sno åt sig lite spån där. Bor man på annan ort kan man vintertid uppsöka närmsta riksväg och vänta på att en flisbil \textsc{(s.~\pageref{89900467e74c1de354e483c90b816b0e})} kör i diket.



 HEAD2:  Alternativa metoder

 Många katter är faktiskt så pass smarta att de kan tränas att gå på en vanlig toalett. Little Kwitter erbjuder ett toalettträningskit \textsc{(se Toaletträningsskita s.~\pageref{88f608faecb7d2695f7b1627751c32d8})} för katter äldre än tre månader. Fördelen med denna metod är att pengar och energi inte behöver ödslas på att köpa eller stjäla ny kattsand hela tiden. Som ägare behöver man bara se till att hålla badrumsdörren öppen och spola lite nu och då. Metoden rekommenderas inte om ägaren är en så kallad \quotetext{vilskitare} då det är väldigt svårt att lära katter att stå i kö.


 Har man möjlighet att hålla katten utomhus ibland kan man ta med den till en lekpark. Studier har visat att katter mer än gärna skiter i vanliga sandlådor.


 Samtliga av ovanstående tips och råd gäller bara för så kallade \quotetext{innekatter}. För vanliga normala katter är det bara fantasin som begränsar var dom uträttar sina behov.


 Källa: Silverbibeln \textsc{(s.~\pageref{7de7d2a7d608c9a2044f50688bc63e27})}

}

\small{
\textbf{Kattsläkt}
\label{10807598833f69b271dba94b1aea7199}
 , familjeband längre bort än kusinnivå.


 Källa: Sara Lidman \textit{Nabots sten}

}

\small{
\textbf{Kattstrypare}
\label{fc44dd925aa7fea6a0bce9504c6fe652}
 är en typ av plastband som är nästan lika användbara som gaffatejp. Bandet har en liten ögla i ena ändan, och när man drar den andra ändan igenom går den inte att dra tillbaka sen. Det är väldigt användbart om man vill spänna fast någonting men inte är så bra på att knyta. Till exempel stänkskärmar, handtag, valurnor eller trumstativ. Men hur kan det komma sig att det heter kattstrypare undrar kanske den vetgirige? Om det bara går att dra bandet åt ett håll kan man ju bara strypa en (1!) katt sen är bandet oanvändbart? Det är förvisso sant, men plast är fortfarande så pass billigt att banden säljs i storpack och har ett styckepris på några få ören så det behöver inte bli så dyrt.
 som saluför kattstrypare.]]


 HEAD2: Kattstrypare i populärkulturen
 I Sid Vicious tolkning av sången \textit{My way} sjunger han \quotetext{I find it all so amusing. To think, I killed a cat}.

}

\small{
\textbf{Kaviar}
\label{2a1df6259cb37397d58a554b5265afea}
 är ett smörgåspålägg bestående av torskrom. Kaviar finns även i Sovjet men där blandar man inte i krämiga ingredienser som gör att det passar på äggsmörgås. Kaviar betyder egentligen \quotetext{havande stör}. Kaviar smakar gott, särskilt gott är det om män eller kvinnor glömmer kaviar i mustaschen för att 3 timmar senare slicka sig om överläppen. Givet på kavjarmacka \textsc{(s.~\pageref{bb9f1374df7e64fbede92c51856896b2})}

}

\small{
\textbf{Kavjarmacka}
\label{bb9f1374df7e64fbede92c51856896b2}
 är vad Nisse äter till frukost varje dag. Ibland lyxar han på med ett glas O'boy också. Mackdelen av kavjarmackan kan med fördel vara av typen knäckebröd men allt går.

}

\small{
\textbf{Keff}
\label{890a42bbf6c2e6888fb851dd76e1e980}
 är ett begrepp som står i ett motsatsförhållande till smutt \textsc{(s.~\pageref{d9114ffee4f2dcee302ae2b19ce5eea9})}. Exempel på keffa företeelser är ryggont, munsår och jeans som är trasiga på fel ställe (dvs runt skrevet).

}

\small{
\textbf{Kelsey Grammer}
\label{de7280f47372899caeb3d8db67ed727d}
 är en känd underhållare vars repertoar i huvudsak går ut på att imitera den parsisik-indiska postkoloniala teoretikern Homi K. Bhabha, vars son Satya Bhabha \textsc{(se Homi K. Bhabhas son s.~\pageref{66ce2281df988914500cb1c269d7418f})} många lärt känna och kommit att älska genom rikstelevisionen. Grammer har flera gånger turnerat med en uppsättning där han förklädd till Bhabha, iförd flaskbotten-glajjor och lösskägg, råkar ut för de mest dråpliga och slap-stickartade situationer. Bland annat fastnar \quotetext{Bhabha} med foten i en hink och spiller vindaloo på skjortslaget. I slutakten ser vi ett bejublat inhopp av Bhabha Jr. som sin fars bästsäljande bok, \textit{The Location of Culture} (1994). Under 2012 turnerade uppsättningen i Europa och sålde bland annat ut Millenium Arena och Holmsunds tropikhus \textsc{(s.~\pageref{5b087d935637ad4d1823cf48036e9be6})}.
 HEAD2: Trivia
 Holmsunds tropikhus ligger i Holmsund.

}

\small{
\textbf{Kennet}
\label{eb251e3745d960e2100c5435a32764c4}
 92\% av alla smala, glasögonprydda män i 45-50 årsåldern heter Kennet antingen i förnamn eller i mellannamn.
 Kennets anseende upprätthålls av Svenska Kennetklubben \textsc{(s.~\pageref{13ceb6e883dd0d39e9246cfe89bd17bd})}

}

\small{
\textbf{Kent}
\label{564f10260067a9b0c8d8e206ecdb49c6}
 är framförallt ett förnamn  som är exklusivt reserverat för män. Det är en kortform av det äldre namnet Kentaur som betyder ungefär ”hästarsle”.  Dessutom är det ett grevskap i Storbritannien där canterburyscenen skapade magisk progg på 60-talet OCH ett ciggmärke speciellt framtaget för den som bara vill röka fimpar OCH ett anskrämligt popkoncept som går ut på att sjunga slumpmässigt utvalda ord ur SAOL med ljus stämma - i Eskilstuna.

}

\small{
\textbf{Kenta Gustafsson}
\label{b2e62eef29bb3f50253d932b26d4de76}
 Svensk bard och vädergud. Gillar solsken och jästa drycker, sjunger hellre än bra.

}

\small{
\textbf{Kepsar med olika företagslogotyper}
\label{6a414633590fd4cd6d6ac64798d14c14}
 är klassiska accessoarer för den som vill framhäva en avslappnad och skön stil. Vilket företag som återges på kepsen spelar mindre roll men logotypen bör gärna vara utfasad och inaktuell, exempelvis rekommenderas Lantmanna istället för Odal butik.[1] Skärmen bör antingen helt sakna böjning eller ha så kallad \quotetext{sadeltaks-böj}. Tidigare hade många kepsar genomskinliga plastskärmar men detta har tyvärr försvunnit i modern tid. Passformen justeras bakpå av ett elastiskt band på billigare modeller medan större företag föredrar att justera sina kepsar med kardborre eller plastknäppning. Blir man erbjuden en keps med företagslogotyp bör man alltid ta emot den då man aldrig kan få för många. Detta kan te sig något komplext då man helst alltid bör bära samma keps år ut och år in. Men så kan det vara.

 Se även: Butiker som bara säljer skoter-kepsar \textsc{(s.~\pageref{1104d57d523c5abf0a8273fff6b5fdd7})}

 [1] Numer är båda namnen dock godtagbara då företaget haft den dåliga smaken att åter byta namn, denna gång till Granngården.

}

\small{
\textbf{Kermit Roosevelt}
\label{8773865e547ab36619f2ee50ffb11ac2}
 är Theodore Roosevelts son. Han sköt sig själv 1943.

}

\small{
\textbf{Kerry King}
\label{d6b6ab73f6de8a63b008622c780b0ad5}
 är en förebild för tunnhåriga män världen över. När han inte är det spelar han, till skillnad från Larry David, i SLAYER!!!!!!!!!!!!!!!

}

\small{
\textbf{Keytar}
\label{dd8acdce45608c8b0924e7d1bc6c9124}
 är ett musikinstrument som kombinerar digitalsynthens plastiga bingo lotto-ljud med gitarrens utseende och attityd. Tack vare dess axelband kan keytaren hängas på musikanten, som nu får samma rörelsefrihet som en gitarrist. Det inbjuder exempelvis till att gunga i takt med basisten och gitarristen på refrängen i \textit{Rocking all over the world}, eller att gunga med i groovet på en bingo lotto-version av \textit{Smoke on the water} (RIP Jon Lord). Keytarens främsta uppgift är nämligen att underlätta för musikanten att på ett naturligt sätt få sträcka på benen ibland. Keytarmusiker är påfallande ofta ändre herrar som spenderat större delen av karriären sittandes på en svettig skinnpall lite i skymundan bredvid trummisen. De feta åren är sedan länge förbi men gubbens kroppshydda \textsc{(s.~\pageref{032eed30d2aad3425b9139aafdd6740f})} växer däremot stadigt. Så åderförkalkningen är ett faktum och inte blir det bättre av att turnera i halvfulla ishallar. Framåt sista tredjedelen av setet halar keytarmusikern därför fram sin keytar och stegar ut på scenen för att efter 25 år äntligen få känna på det värmande rampljuset.

 Det tydligaste exemplet är naturligtvis Robert Wells som varit dålig i mer än ett decenium och därför borde fått sjukpension för länge sen. Men en nitisk handläggare på Försäkringskassan lusläste de nya sjukskrivningsreglerna och skickade på stackaren en keytar istället.

}

\small{
\textbf{Khadaffi}
\label{33585042455a434620aa25936b85826b}
 Moammar. Jävligt svårt att veta hur man ska stava hans efternamn. Död. Lika bra det, så slipper vi problemet med att stava hans efternamn.

}

\small{
\textbf{Kicki Danielsson}
\label{b4646e392dda635159575835254d4ef1}
 En drink bestående av billigt vitt vin (helst ur bag-in-box) \textsc{(se bag-in-box s.~\pageref{1fdd5e1bb07154385669cd70e53bd354})} och 7-up (uttalas /ʃup:/), gärna med en falukorvssnärt \textsc{(s.~\pageref{6fb9ccfbd5699d12ff8d04b2a27852fb})} på kanten. Den är extremt lättdrucken och bäddar för en helt vansinnig fylla.

}

\small{
\textbf{Kikare}
\label{e2b5cb2875a91cea4345226ce26ada44}
 är en av de mest psykedeliska uppfinningar den mänskliga hjärnan hittat på. Genom att titta på en sak genom en kikare blir den (saken) genast mycket större - fast den egentligen inte är det. Tar man bort kikaren går allt tillbaka till det normala och saken får sin ursprungliga storlek igen. Men tittar man i kikaren igen så, vips!, är saken jättestor igen. Det behövs inte ens någon konverteringstid. Enligt den tyske \textsc{(se tyskland s.~\pageref{b1b58da783b6d5fa090f3015f1889869})} paleontologen \textsc{(se paleontologi s.~\pageref{7dfd5e638a8efdca671f24f6bfef45fb})} Wilhelmus Wurst kan detta mycket väl vara orsaken till att dinosaurierna var sådana jättar. Alla dinosaurier gick runt med kikare och till sist upphörde verkligheten att existera och ersattes av jättevärden.

 HEAD2: Missbruk
 Om man vänder kikaren bak och fram och tittar blir alla saker istället jättesmå.

}

\small{
\textbf{Kim}
\label{fb1eaf2bd9f2a7013602be235c305e7a}
 är ett namn som är vanligt bland manhaftiga kvinnor och feminina män.

}

\small{
\textbf{Kim Long}
\label{9f5d79563005dcda7f9197851998b23f}
 är en restaurang på bostadsområdet Hertsön \textsc{(s.~\pageref{e4b227ae8f6ccfed5aae6fdc7e655623})} i Luleå \textsc{(s.~\pageref{3cefb5ac35187749592f1ebb25472b99})}. Kim Long ägdes tidigare av en bunt asiater, mest troligt kineser men detta är ej bekräftat. Nu ägs den av ett gäng från balkan som gör pizza \textsc{(s.~\pageref{7cf2db5ec261a0fa27a502d3196a6f60})} istället för kinamat. Varför restaurangen fortfarande heter Kim Long beror enligt obekräftade källor på att balkangänget i hemlighet sammarbetar med syd-koreanska underrättelsetjänsten för att komma i kontakt med en nord-koreansk statstjänsteman som bedriver hemlighetsstämplat arbete på Hertsön. Man hoppas kunna locka honom med det korean-klingande namnet Kim Long. Själva namnet Kim Long är en hyllning till asiens längsta man Kim Long il, 167 cm lång.

}

\small{
\textbf{Kimba}
\label{6b7b6a244906fd84bb5690bacd47e878}
 är en by/bensinmack som ligger i södra Australien \textsc{(s.~\pageref{e727d8d1b3162a732c7f706d55de64f3})} och har 686 invånare. Kimba är världsberömt av två anledningar, nämligen:
 \begin{itemize}
 \item Det ligger halvvägs genom Australien \textsc{(s.~\pageref{e727d8d1b3162a732c7f706d55de64f3})}, från öst till väst och \textit{vice versa}.
 \item Här finns en jättestor skulptur av en utvecklingsstörd fågel som möter besökaren som anländer till byn/bensinmacken med sin vindögda, hotfulla blick.
 \end{itemize}
 Förutom att sälja snask och dricka till människor som i sitt tröstlösa sökande efter en mening med livet valt att bila genom Australien, försörjer man sig i Kimba på att odla vete - en näringstradition som sträcker sig tillbaka till slutet av artonhundratalet då vita anlände och slog ihjäl alla aboriginer i området, började odla vete och lät resa en jätteskulptur av en utvecklingsstörd fågel.

}

\small{
\textbf{Kineseri}
\label{1f4cf2e5ffaa23a61ff5edd509c8c10f}
 betyder att krångla till saker i onödan. Ordet härstammar från kinesers ständiga önskan att göra det lätta extremt svårt och krångligt. Kinesiska muren är ett klassiskt exempel på kineseri. Istället för att bjuda över mongolerna på lite ödla på pinne \textsc{(s.~\pageref{9159305fc9033e2af7fee11a993874d9})} och äckligt risvin för att prata om saken, skulle kineserna prompt bygga en fet jävla stenmur runt hela sitt land. Kina är för den som missat det typ hur stort som helst.
 De kinesiska språken är ett annat exempel. Istället för att göra ett talat språk till standard och låta mer eller mindre bisarra dialekter \textsc{(s.~\pageref{1e35ddb6700b133ed0f9f078815422b8})} existera inom landets gränser, har Kina skitmånga talade språk, utan något krav på att alla ska kunna tala ett gemensamt språk. Så om två kineser träffas på stan kan de inte prata med varandra. Däremot kan de chatta med varandra med typ sms eller flaskpost, eftersom Kina har ett standardiserat skriftspråk. Snyggt jobbat, Kina. Verkligen, smart tänkt. Ni kan inte prata med varandra, men ni kan skriva på små lappar. Skitbra.

}

\small{
\textbf{Kinesiska muren}
\label{f5a025bc13e9af75673c6b3f647ebf5f}
 är en rap av MC Evert och handlar om hur Chi-Huang-Ti, kung av Tsin, lät bygga den kinesiska muren. Texten rappas över minimalistiska beat \textsc{(s.~\pageref{ed6966981beb595bd0a50e4371805dec})} bestående av bjällror på fingrarna och gong gong. Flera teorier florerar om att texten i själva verket är en dold kritik mot modernare härskare. Vidast spridd är teorin att texten i själva verket behandlar Göran Perssons instiftande av myndigheten Forum för levande historia.

}

\small{
\textbf{King Edward}
\label{a081b9ae5423fc12a8439e33b2af8bed}
 är ett brittiskt mansnamn som betyder ungefär \quotetext{kung över alla som heter Edward}. Vanliga smeknamn är \quotetext{Kingen} och \quotetext{Eddie baby}.

 Det är också namnet på en könlös potatissort. Läckergommen Carl von Linné beskrev rotfrukten i ett brev till vännen Peter Artedi med orden: \quotetext{Thänna potät äro mycken ghod att ät. Ock brennvin brygg när käringa trät}.

}

\small{
\textbf{Kinka}
\label{1536ff745bb5a1b8bc504c51685530af}
 Att kinka innebär att urinera och onanera samtidigt. Idealet för vana utövare är att uppnå klimax precis när sista droppen runnit ut.

}

\small{
\textbf{Kinneviksliljan}
\label{1f7dfd6912a2bacc6b5262bc01793758}
 Tidigare mera känd som Flikenliljan (1865)och efter det som Fagerstaliljan (1886). Man har felaktigt antagit att liljan föreställer Ormbärsörten, \textit{Paris Quadrifolia}. Senaste forskning visar dock att \quotetext{liljan} är flertalet blodkärl som förenas i en maffig propp \textsc{(s.~\pageref{8bde165d41e3a0363954a27cc7164e2e})} i aorta.

}

\small{
\textbf{Kir}
\label{002e1a6e54da86cabc77fbb474c2df49}
 - enkelhet med fransk finess och svensk tradition. Från Finland.

}

\small{
\textbf{Kisar}
\label{918ae8307ab41b5ab2b23af26748e276}
 Bögvördigt tilltalsord

}

\small{
\textbf{Kissemiss}
\label{6c1d5156bc862124ee5d4b945c89423c}
 Urin \textsc{(s.~\pageref{524fd7acb94f9c2d879b5c1cf8335669})} på golvet bredvid klosetten.

}

\small{
\textbf{Kisskorv}
\label{6e019e7e730af9d790b981ebcccb7ace}
 Exakt vad en kisskorv är vet nog bara den hårdrockare \textsc{(se hårdrock s.~\pageref{a4566a810e7ad85a57ddc84083a8139b})} som hade detta skrivet på sin jeansväst då han besökte Hultsfredsfestivalen år 1998, troligtvis för att lyssna på musik.

}

\small{
\textbf{Kiwi}
\label{de5949721e6352f01dfef317c3e898a8}
 n är en ganska menlös, men gullig, fågel. I motsats till typ alla andra fåglar kan den inte flyga, inte ens hoppa kan tänkas.
 Den är också  utrotningshotad \textsc{(se utrotningshotade djur  s.~\pageref{24a427a5537c2c8918cfa213ae099a74})} tack vare att dess bruna och ludna ägg är så söta och fulla av vitaminer.

}

\small{
\textbf{Kladdi Mittänän}
\label{733a8a539042e9b49e71bf75cb93a379}
 är Finlands sämsta konstnär. Hans tavlor är så dåliga att Prof. Etienne \textsc{(se Användare: Prof. Etienne s.~\pageref{a9878d2280e5a39becac8f73d113df91})} lät publicera bilder på flera verk för att använda som praktiska övningar i sin bok \textit{Barnagans förträffliga pedagogik \textsc{(s.~\pageref{9e0723018fdd5ac13da751c48083a4e3})}}. Till försvar för hans misslyckade livsval lär hans mor ha myntat begreppet ”det är inte utsidan som räknas”.

}

\small{
\textbf{Klasse Möllberg}
\label{db97de5ede53615920b9d1e44c276fba}
 (f. 1948), måste nog de flesta hålla med om, har passerade sitt zenith med det klassiska barnprogrammet Trazan och Banarne som han gjorde tillsammans med Lasse Åberg, och hans kändisskaps sol har stadigt dalat sedan dess med vissa perioder av tillfällig uppgång så som under 2009 då Möllberg medverkade i TV4s underhållssatsning \quotetext{Hjälp, jag är med i en japansk tv-show}. Många TV-tittares favorit-\quotetext{asse} har därför varit \textbf{L}asse och inte \textbf{K}lasse. Detta kan vissa andra tycka är Klasse oförtjänt, inte minst eftersom Klasse enligt wikipedia har svensk, fransk och USAsk skidlärarexamen [http://sv.wikipedia.org/wiki/Klasse_M\%C3\%B6llberg]. Vidare är han en hejjare på gura - det var ju trots allt Klasse som axlade gitarren i Electric Banana Band tillsammans med Janne Schaffer!!

 HEAD2: Kontrovers
 Vintern 2004 uppstod rykten om att Trazan och Banarne skulle vara ett näsligt varumärkesintrång på Edgar Rice Burroughs klassiska böcker om Tarzan. I en utdragen rättsprocess som bevakades väl i etermedia avkrävdes Möllberg stora summor i skadestånd till Burroughs dödsbo. Han friades dock i högsta domstolen och den anklagande sidans advokater fick näsor och öron avskurna, i enlighet med djungelns lag.

 Klasses hemsida: [http://www.klassemollberg.com/]

}

\small{
\textbf{Klassikerbilssjälvmord}
\label{8d866f91bbb57d360447acf0fadaec45}
 Ett klassikerbilssjälvmord är en mycket vanligt förekommande mansdöd. Många är de män som när fyrtio- eller framförallt femtioårskrisen inträffar skaffar sig ett bilvrak med klassikerstatus. Garaget städas ur och proffsverktyg (migatronicsvetsmaskin, speedglasshjälm och en stor uppsättning makitaverktyg) för en förmögenhet införskaffas. Nu påbörjas arbetet med en sprudlande entusiasm. Motorn monteras ned och tvättas in i minsta detalj, ramen lyfts och underredet skrapas rent från underredsmassa. I stort sett hela bilen monteras ned och placeras i godislådor och banankartonger. Ungefär vid detta läge börjar dock entusiasmen tryta, och problemen hopa sig. Man har skaffat sotningssats till en 57a när man i själva verket äger en 55a och nyansen på den dyra lackfärgen man köpt är ju en liten ton fel. Det var inte så självklart och enkelt som det från början verkade att renovera en klassisk bil till nyskick. Framförallt var det kanske inte det man behövde för att komma igenom sin femtioårskris. I själva verket så blir den bra mycket värre och grubblerierna och prestationsångesten växer lavinartat. Det hela känns oöverskådligt och utmynnar i en destruktiv spiral. Det är här jaktgeväret eller i vissa fall hängsnaran kommer in i bilden. Mången kortklippt permanentad hustru har funnit sin älskade hängande i takbjälken bredvid en avstannad renovering.

 Vi här på Nissepedia \textsc{(s.~\pageref{62400dadecd90cb5cd39062abe5a3e4a})} vill ju gärna se något positivt i allt, och i det här fallet är det självklart våran miljö som drar det längsta strået. Tycker du denna artikel var deprimerande så råder vi dig att läsa den allmer positiva putsbilar \textsc{(s.~\pageref{cb242351ab9f9d6b8a5afe8bed7b2dbd})}. Känner du själv en oro för att hamna i potentiellt klassikerbilssjälvmordsscenario råder vi dig att hålla dig borta från publikationer såsom Classic Motor och förbund såsom Norrlands Motorhistoriker

}

\small{
\textbf{Klasskamp}
\label{8355704f5b9cf122fda38c67d8bf801c}
 Brännbollsturnering. Föddes med yxa och påk.

}

\small{
\textbf{Klassmarkör}
\label{6a9c0c6836a0777442468f821837e795}
 er är något som faller sig naturligt för de flesta och som resten lägger merparten av sina pengar på att skaffa sig.

}

\small{
\textbf{Klo}
\label{642b921dce80b8405b7f32f9974c5a40}
 är foten på en uv \textsc{(s.~\pageref{45210da832f9626829457a65e9e7c4d0})}. Den består av spetsiga, farliga tår som kan slajsa upp fejset på den som vill ha bråk hur lätt som helst. Den kan också lyfta upp saker, typ sköldpaddor att släppa i huvudet på Aischylos \textsc{(se de gamla grekerna s.~\pageref{4a5fb3d6ce79b5ff43b33f8f7d843672})}. Då kallas det gripklo. Det kallas även gripklo om det är klon på det mytologiska monstret grip som åsyftas. I Skellefteåtrakten, där man alltid ska va så jävla speciell, betyder dock klo skithus. Så nu vet ni.

}

\small{
\textbf{Kloakdjur}
\label{592254da9a0f26310a65ff83f6d73c9e}
 är benämningen på en undergrupp av däggdjur. Det som särskiljer kloakdjuren från de andra däggdjuren är att kloakdjuren lägger ägg \textsc{(s.~\pageref{128a5feb8e12d0aa622e0298a8332980})}. Anledningen till att de kallas just kloakdjur är att honornas urogenitala system och anus mynnar ut i samma öppning. Öppningen började först skämtsamt kallas kloaken av Carl von Linné, för att sedan bli, inte bara öppningens namn, utan benämnande för hela subgruppen.
 Idag finns endast två sorters kloakdjur, myrpiggsvinet och näbbdjuret. Båda toppade flera gånger Carl von Linnés \textsc{(se Carl von Linné s.~\pageref{5e8380bf6b7ce99678e6752b6d9e709e})} lista \quotetext{Thät mäst bhuskis diur uti thenna mystiska wärld} som årligen publicerades i \textit{Then Swänska Argus}.

}

\small{
\textbf{Klossa}
\label{0e57164ee7dd893f0f88aac39c7806b7}
 Att bajsa en stor kloss av skit.

}

\small{
\textbf{Klä ut sig till ett djur}
\label{78663eff2fe898143e822b7f9d4851f7}
 Att klä ut sig till ett djur är ett av världshistoriens äldsta spratt. Sprattet går ut på att skämtaren iklär sig kläder eller en dräkt som påminner om ett djurs dräkt. Poängen är inte alltid att folk ska missta skämtaren för ett djur, utan att skämtaren ska framstå som lite lustig eftersom han eller hon ju i själva verket inte alls är ett djur.
 HEAD2: Skämtet i mytologin
 Att klä ut sig till ett djur är ett väl använt skämt i mytologiska texter som Gilgamesh-eposet från Mesopotamien, i vilket Gilgamesh klär sig i ett lejonskinn, och i skandinavisk mytologi, i vilken halvguden Loke klär sig i fågelskrud och flyger omkring och således får de andra asarna att vråla av skratt. I Bibeln \textsc{(s.~\pageref{7de7d2a7d608c9a2044f50688bc63e27})} återfinns berättelsen om spilevinken Jona som skojar till det genom att liksom \quotetext{klä sig} i en valfisk.

 HEAD2: Skämtet i modern tid
 I modern tid har vi sådana exempel som när Lukas Moodysson tog emot ett pris för \quotetext{Fucking Åmål} på guldbaggegalan iförd ett slags diadem med djuröron på och ett spratt av fotbollslaget Arsenal i vilket några av lagets stjärnor iklädde sig djurdräkter [http://fotboll.expressen.se/internationellt/1.1784539/har-ar-arsenals-stjarnor-i-djurdrakter].

}

\small{
\textbf{Klädsamt ful}
\label{645b7342c8cabb097bb1b2bafda2ba13}
 Att vara klädsamt ful är när en person är ful, enligt konstens alla regler, men att det ändå lyckats bli deras \quotetext{grej}. Cronos i Venom är typexemplet av någon som är klädsamt ful. Sveriges \textsc{(se Sverige s.~\pageref{b1999637949ed135b2ca03f3a38460cc})} motsvarighet till Cronos, Janne Schaffer, är ett annat slående exempel.
 Engelsmän är i regel, men med undantag, klädsamt fula.

}

\small{
\textbf{Knallhatt}
\label{a4e8971c8e546927a05727ad92eba532}
 kan man kalla en person som är ett träskaft \textsc{(s.~\pageref{1ab85ecd859ae682af47bb9334c7dac6})} eller som är särske \textsc{(s.~\pageref{552a5aad891937bf760fb193900ea140})}. Om till exempel någon man känner och ska sova hos har låst in sina nycklar i bilen \textsc{(se bil s.~\pageref{b3188f47d2eac7efc3f1258dc673a9fe})} när det är skitkallt ute och man har släpat omkring på två backar öl i snöoväder och allsköns jävelskap \textsc{(s.~\pageref{46845591177f16920cd586a5baf5a625})} bara för att plötsligt inse att man inte kan komma in i värmen kan man kalla snubben som låst in nycklarna sina för knallhatt.

}

\small{
\textbf{Knark}
\label{bebc5e7342ca2f076b3d32ed6c557398}
 är en drog som din mamma varnat dig för men din vänner påstår gör dig häftig. Vem du väljer att lyssna på är ett personligt val.

 Knark kan leda till omåttligt kändisskap och pinsamma dödsfall.

}

\small{
\textbf{Knarktron}
\label{4742d2020e0af603c44c9ad2db56aa3a}
 En knarktron är en tron som till 90\% eller mer består av knark.
 HEAD2: Knarktroner i populärkulturen
 Det engelska hårdrocksorkestern Electric Wizard besjunger denna ovanliga tron i titelspåret på albumet \quotetext{Dopethrone} från 2000.

}

\small{
\textbf{Knixa}
\label{2d350f4a76f78d9017ed0f136cf81b88}
 Att knixa är att rytmiskt och snabbt böja på knäna medan man står på stället.

 HEAD2: Folk som knixar
 \begin{itemize}
 \item Pundare
 \item Strömstarar
 \end{itemize}

}

\small{
\textbf{Knocking on heavens door}
\label{1cabaffcd7be1dc9cee62705d0dd6594}
 är en låt med den brittiska musikgruppen Mungo Jerry. Den skrevs ursprungligen av den amerikanske knarkhippien \textsc{(se hippie s.~\pageref{14fd61fa8edcb67c5c7886f11af8431e})} Bob Dylan, som åkte runt med en glapp elgitarr och spelade den 20 år tidigare. Det är det dock ingen som kommer ihåg idag. Låten återfinns på Mungo Jerrys skiva \textit{Together again} från 1981, varifrån den släpptes som singel. Den nådde bland annat förstaplatsen på Sydafrikas topplista, och denna historiska placering gjorde att Mungo Jerry nu kunde stryka a cappellaversionen, rapversionen, midiversionen och discoremixen av \textit{In the summertime} från sina greatest hits-skivor, vilkets sågs som en delseger för världens alla Mungo Jerryhatare \textsc{(s.~\pageref{31191f71efcd2dbcc9e02a0ce88c5943})}.

}

\small{
\textbf{Koalor}
\label{ae59a7cf8bdd53755a9056d86f88e5e8}
 är ett slags australiensisk \textsc{(se Australien s.~\pageref{e727d8d1b3162a732c7f706d55de64f3})} trädbjörn \textsc{(se träbjörn s.~\pageref{3fa1e4f2d866814bf69e29479762b85a})} med najs andedräkt som låter förvånansvärt konstigt.

 Källa:
 [http://www.youtube.com/watch?v=qIuHICSahzc]

}

\small{
\textbf{Kockobello}
\label{87464dbe1f4053eaf434b95c4e6b38ab}
 Den riktiga betydelsen är vacker kokosnöt (exotisk ko) och så kan uttrycket fortfarande användas.
 Den vanligaste betydelsen är dock knäppskalle.

}

\small{
\textbf{Kodnacke}
\label{4e5843d28601c986086af2994feddde3}
 är ett skadligt värk- och anspänningstillstånd i halskotpelaren. Det uppstår när man återkommande, med huvudet i onaturlig vinkel, stirrar stint in i t.ex. tuggummihyllan på ICA för att inte råka lära sig andras kortkoder och lösenord. Är man utbränd riskerar man dessutom att blanda ihop vems kod som är vems, och kanske få gå hem utan ketchup och pölsa.

 Att inte stirra på tuggummina, typ, anses som brott mot kodiketten och lätt perverst. En sådan person kallas för kodfluktare. Personer som valt att inte dölja sina lösenord för varandra kallas för kodsystrar eller kodbröder. Kodincest händer så lätt på firmafester. Det är inte olagligt, men ofta väldigt pinsamt på måndag.

}

\small{
\textbf{Kokt}
\label{19637e73db4270e8d526bb79b56852b7}
 Blöt och sladdrig, istället för stekt och god.

 OBS. Gäller dock ej ägg \textsc{(s.~\pageref{128a5feb8e12d0aa622e0298a8332980})}.

}

\small{
\textbf{Kolonialdricka}
\label{07e759cc1c08884f798b5bac20e06f9e}
 Länge var kolonialdricka per preference gin och tonic. Ibland kunde en och annan kopp te slinka ner i en sentimental brittisk strupe, drömmandes om afternoon tea någonstans i Kent \textsc{(s.~\pageref{564f10260067a9b0c8d8e206ecdb49c6})}. På senare tid har kolonialdrickat Indian Pale Ale blivit populärt, främst bland hippies. Skrapar man på rödbrända bukens yta finner man dock att den utan konkurrens vanligaste kolonialdrickan var en enkel gin och flodvatten. Tillredningen av drycken illustreras utmärkt av Humpfrey Bogart i filmen The Queen of Africa. Glaset fylls till hälften med flodvatten, gin efter humör. Och då ska det vara riktigt flodvatten, inget fancy fjällbäcksvatten. Det ska vara det mest stillastående vattnet du kan hitta i en riktigt långsam flodkrök i centralafrika, helst riktigt nära ett ebolautbrott. Gin tar nämligen bort allt dåligt här i världen. Sedan filmen Gudarna måste vara tokiga så har Coca Cola blitt populärt i många före detta kolonier. När Lenin så förtjänstfullt menade att imperialismen är kapitalismens högsta stadium var han dead on. Vad är blott en coca cola?

}

\small{
\textbf{Kom och dansa samba! Caramba!}
\label{d42d47b9c159c367e02a3c829df91c90}
 är en låt skriven av låtskrivaren, mångsysslarpensionären \textsc{(se mångsysslarpensionär s.~\pageref{ce651324111b616e98f210ea8511ce75})} och föredetta stadsjuristen Erik de Maré.
 Låten handlar om att ha roligt och dansa samba. Här kan du se texten till låten och också sjunga med till musiken: [http://home.swipnet.se/de_mare/page12.html]

}

\small{
\textbf{Kombinationsaffär}
\label{54328b839527f9917e5d057845b4fc5c}
 En kombinationsaffär är en affär som erbjuder en ofta osannolik kombination av varukategorier. En legendarisk kombinationsaffär är Randolfos \textsc{(se Randolfo s.~\pageref{b8f0a32f840f1db27a2c12e17b640fb2})} affär på Haga i Umeå. I den säljer han hårdrock \textsc{(s.~\pageref{a4566a810e7ad85a57ddc84083a8139b})} och akvariefiskar. Ett annat exempel på kombinationsaffär är \textit{Norrtälje teknik och inredning} som säljer mobiltelefoner och stentroll. Anledningen till sådana kombinationer är att de ofta drivs av ett gift par som på ålderns höst, på grund av ett kognitivt fel, har bestämt sig för att driva affär. Maken sysslar med sitt stora intresse, till exempel mobiltelefoner, medan fruntimret sysslar med sitt, vilket ofta är stentroll.

 Ett annat riktigt skruvat exempel är \textit{Elvisboden} i Norberg som kör en trippelkombination. Förutom att saluföra Elvis Presley-memoribilia är affären även massageklinik och begravningsbyrå. Ägaren heter Runo och han älskar Elvis och har varit begravningsentreprenör i hela sitt liv. Här bor också Runos likbil.

 I Malå \textsc{(s.~\pageref{41da4620e87888eaaeafcb3004a8d177})} finns en sund och pragmatisk kombinationsaffär, nämligen fryppelkombon jourlivs, skoterklädesförsäljning, bensinmack och frisör. Detta grundas i släktskap. Frisörsalongen är dock inte kvar idag.

}

\small{
\textbf{Kombinationsaffärer}
\label{0a2777bf1366a8a9a5b8eab9ca1496a1}
 Kombinationsaffär \textsc{(s.~\pageref{54328b839527f9917e5d057845b4fc5c})}

}

\small{
\textbf{Kommunist}
\label{fd9bf7896d396992b29d542a0200b800}
 är ett skällsord \textsc{(s.~\pageref{e0fc85fd2f5249557257965783ac136e})} som ofta används av tokliberaler \textsc{(se tokliberal s.~\pageref{531cb70b602e3f3c32d40bac64400830})} och som etymologiskt betyder \quotetext{någon som frivilligt delar med sig} och \quotetext{någon som ogillar krig}. Med detta ord vill tokliberalen antyda att givmilda personer som ställer sig tveksamma till krig som affärsidé på ett eller annat vis är skyldiga till folkmord och förtryck.
 Kommunist är motsatsen till småborgare.

}

\small{
\textbf{Kommunister}
\label{41fe84ccaa18761f33bdd4c793780286}
 Den extra kroppshydda kommunal- och landstingsråd lägger sig till med efter ett par mandatperioder på posten.

}

\small{
\textbf{Kommunistglasögon}
\label{1bc58f6f6a934c05a63add653dbeadf0}
 Runda brillor, vilka skänker bäraren en aura av upplysthet. Han/hon har skådat ljuset som vanliga pantade knegare \textsc{(s.~\pageref{98d0a7dac261debb934a16b7041ef22f})} inte kan se. Detta ljus ser man lättast i universitets \textsc{(se universitet s.~\pageref{11dfc744fa396b961a6cc9cf89c4eaea})} korridor F bortanför F243. Brillkommenister \textsc{(s.~\pageref{baa057b925aa9c528ab62b48fb8cdc05})} har ofta kommunistglasögon

}

\small{
\textbf{Kommunslogan}
\label{b80d94cf2d085d692dd87e1f5bdeaa59}
 Alla kommuner i Sverige \textsc{(s.~\pageref{b1999637949ed135b2ca03f3a38460cc})} värda namnet har en kommunslogan \textsc{(se kommunistglasögon s.~\pageref{1bc58f6f6a934c05a63add653dbeadf0})}. Här kommer de bästa eller sämsta beroende på perspektiv \textsc{(s.~\pageref{1606dd19366985367d677f7b6de46e52})}.

 1. Alingsås - Utan oss vore korv och mos bara korv
 2. Ska du handla i Ludvika imorgon? Handla i Ludvika idag istället!
 3. Ljusterö  - Just de'ru!
 4. Fullt blås, Borås!
 5. Kristinehamn - Staden med massor av p-platser i centrum
 6. Motala - Why not?
 7. Fagersta – Här får du livstid
 8. Nordmaling - Sveriges godaste dricksvatten
 9. Vindeln - En sista utväg
 10. Bjurholm - Bäverstaden
 11. Skellefteå - Inte bara en sjukdom
 12. Malå \textsc{(s.~\pageref{41da4620e87888eaaeafcb3004a8d177})} - Snart vänder det
 13. Norrtälje \textsc{(s.~\pageref{7527f7dad9445013a559dc7e2a91f3b3})} - varken bra eller dåligt
 14. Trollhättan - där drömmar dör!
 15. Lilla Edet - en svag del av Trollhättan
 16. Filipstad - inte bara den blåsta nazisten på youtube!
 17. Robertsfors - Sveriges bästa kommun om cirka 85 år!
 18. Ale - En härlig blandning av kemikalieindustri och riktigt pantade nazister!
 19. Täby - entreprenörskap!! entreprenörskap!! entreprenörskap!! wohooouu uläääää!!!11!
 20. Piteå - Lilla Frankfurt
 21. Borås - Borås – of course not
 22. Krokom - Center of Innovation
 23. Sundsvall - Rastahundens hemkommun!
 24. Kronoberg \textsc{(se Växjö s.~\pageref{2fc07b846123d1c41a4c7eb55c40df40})} - Duger det åt Electric Wizard duger det åt dig.

}

\small{
\textbf{Kompisvännen}
\label{53c3944aba4b9e4cb9a376ce624589f9}
 är en slags pleonasm, men ändå inte. Ordet används som benämning på en kvinnlig vän som man ej stiftat bekantskap med, men ändock har telefonkontakt med.

 Vid kontakt med kompisvännen bör man helst inte använda sig av språkliga byggstenar som kommatering och versaler \textsc{(se majuskler s.~\pageref{cc540e015f5457a65e5c31c0cb947227})}, då det kan få er relation att ta oönskade vändningar.

 Exempel på användning vid SMS-kommunikation: vad har du på dig kompisvännen

}

\small{
\textbf{Komvux}
\label{493b2ab873be1315fde1ef6e6f6fa0ec}
 är en fin inrättning för dig som spenderade din gymnasietid med braj \textsc{(se hasch s.~\pageref{1e93612a55f48e5fd9cbce22d0e71944})} istället för relevant kurslitteratur.
 Har du klippt dig och siktar mot universitetet \textsc{(se universitet s.~\pageref{11dfc744fa396b961a6cc9cf89c4eaea})} är Komvux ett mellansteg du måste uthärda.
 Har du däremot inte klippt dig utan tvärtom börjat odla bajskorvar i nacken \textsc{(s.~\pageref{f2714801a00494f2ef070519d2ffd3df})} och börjat uppskatta keramik är det folkhögskola du ska satsa på.

}

\small{
\textbf{Kongo-fett}
\label{207713bbdcf294064643ad49207140b0}
 Det är oklart när den superba skoprodukten Kongo-fett slutade säljas i Sverige, men beskrivningen nedan ger en fingervisning om att när Belgien gav upp kolonin 1960 så var det kanske inte lika lätt att \quotetext{finna} råvarorna till produkten. Såvida det exotiska afrika inte bara var ett slugt säljtrick från några fula fiskar \textsc{(s.~\pageref{f2dd2a061e8121a392589d09cf5db207})} i Örebro.

 Direkt saxat från burken:

 \quotetext{HENRIKSSONS TEKN. FABR. A:B ÖREBRO

 \textlessi\textgreaterHENRIKSSONS KONGO-FETT är sammansatt av i Kongo \textbf{funna} och av oss importerade fett- och gummiämnen. Då vi härmed erbjuda \textbf{vår allra sista} uppfinning (!?) av läderkonserveringsmedel är det i medvetandet av, att intet i den svenska marknaden hittills salubjudet läderbalsam kan mäta sig med vårt kongo-fett. Övertygad härom blir var och en som provar det. Kongo-fett gör skodonen hållbara, mjuka och absolut vattentäta. OBS! KONGO-FETT ingnides väl å skodonens såväl ovanläder som sulor, och bör lädret under och efter impregneringen försiktigt uppvärmas så att fettet fullständigt intränger däri.}\textless/i \textgreater

 Om man hittar en övervintrad burk Kongo-fett i en sekundär-hand boutique så bör man, ifall man är vid sina sinnes fulla bruk, inte applicera det som glidmedel vid samlag.

}

\small{
\textbf{Konjektural}
\label{97b105b94d1adc2125ccd7409f18beda}
 Hypotetisk möjlig men ändå helt otänkbar. Som Anton Abele \textsc{(s.~\pageref{0906f6e1d290c547e1fb93c6ff6a0b44})}, med andra ord.

}

\small{
\textbf{Konsonaut}
\label{8d31fd4f0f32e0f3b93be020b6edd51c}
 Yrkesbeteckning på den som uppfann finskan.

}

\small{
\textbf{Konsum på Haga}
\label{946b975c583aac460524a871a0653c28}
 i Umeå är den finaste butiken i hela stan. Långt från Coop stormarknads brölande masspsykos och ICA gourmets megalomana arrogans, utgör Konsum på Haga en stilla bukt i ett annars rasande hav. Trånga passager där varorna lutar sig ut över en från sina rangliga hyllor ger en pittoresk inramning till ens vistelse i butiken. Man kan där köpa alla konventionella basvaror, men inte så mycket mer [1]. Detta kan läggas butiken till last, men ärligt talat, vem behöver tillgång till ädelost \textsc{(s.~\pageref{8aeecc3a132ce5d2e562fa7a2ca29a06})} och fårfiol i sitt närområde? Inte du, i alla fall. Lägg ett sexpack Tingsryd 2.8\%, ett halvt kilo potatis och en gullök i kundvagnen, lägg upp på bandet, hälsa på Majvor i kassan, gå hem och njut över helhetsupplevelsen att shoppa loss på Konsum på Haga.

 [1] Men med detta sagt är det på sin plats att poängtera att man som enda matvaruaffär i Umeå med omnejd saluför, eller åtminstone har salufört, kaffe från områden i Chiapas Mexiko.

}

\small{
\textbf{Konsum på haga}
\label{946b975c583aac460524a871a0653c28}


}

\small{
\textbf{Konsumbutik}
\label{70e4875f7c2c177596305006e46b7ca9}
 En Konsumbutik är en butik som drivs av Konsum och som innehåller allt man kan tänkas behöva. Cocktailpinnar, sandaletter, luftfuktare, svanväska \textsc{(s.~\pageref{f5cd47fc9fb6544d2d9e10009334bece})}, hackebiff, djungelolja, bingolotter, järnspett \textsc{(s.~\pageref{6cbe55f18d91c10e3307681ab810fd74})}, picknickbog \textsc{(s.~\pageref{696b50bd1480adf411314859b3464652})}, gitarrlektioner, speedos \textsc{(s.~\pageref{22286b2c61cbd4c567b0999a958db3eb})}, Kongo-fett \textsc{(s.~\pageref{207713bbdcf294064643ad49207140b0})}, spikmatta, svag mat \textsc{(s.~\pageref{0526761422547962084b0c0c3701cf91})}, alkisschäfer \textsc{(s.~\pageref{347febbc28041eae88556d2e7ced587b})}, kubb \textsc{(s.~\pageref{de7f6954ec8c6e346b8ba18ae018d334})}, luktsudd, uvgodis \textsc{(s.~\pageref{58de09e078ac891b067c0ec53d780b8a})}, uvsvane \textsc{(s.~\pageref{c5081b14cdeb1ff42b655213e80c9d51})}, bil- och cykelparkering, svanskrove \textsc{(s.~\pageref{e543ead268a283bfdb5ea638d6cca4a2})}, transkrove \textsc{(s.~\pageref{1188281a09fb681b922e45663e5ffc4b})}, makadam \textsc{(s.~\pageref{d358df22f11c57bbd1d5718b9b474b26})}, luffarschack \textsc{(s.~\pageref{f6d93abef5b64200d5b82fa5752de6b3})}, tråg \textsc{(s.~\pageref{1e0e0470206e0f2baad8e628ba8f770c})}, frisedel \textsc{(s.~\pageref{ec0e47809187866739cd1a19e3d1ed37})}, dagisplats. Allt finns i Konsumbutiken.

}

\small{
\textbf{Konventikelplakatet}
\label{f8aed9f48b1d86e6566033d0464906d7}
 är en förordning från den 12 januari 1726 som förbjöd bönemöten i hemmet (undantaget familjeandakter).

}

\small{
\textbf{Kopi Luwak}
\label{cf99daa7757f89df31ccd16a13b7ab10}
 är världens dyraste kaffe för att bönorna har gått igenom en mungos tarmsystem. Hur dyrt är det då, frågar man sig? Jo, det är jättedyrt, närmare bestämt 6100 kr/kg.

}

\small{
\textbf{Kopparorm}
\label{b8f4fa38453856ba979bc2898e116e5a}
 är den fjärde ballaste sortens orm \textsc{(s.~\pageref{51b0b5a943ae2b04076f7a6cb037afd6})}. Skalan för balla ormar är precis som de olympiska medaljerna klassificerad efter metallens ballhet. Guldplatsen har spottkobran eftersom den har coolast tänder, avståndsattack och hypnotisk förmåga. Silverplats har anakondan eftersom den är så stor, svart, sväljer saker utan att tugga och har fått en Nicholas Cage-film \textsc{(s.~\pageref{691536bd318544d7e5ffe1ee895caefd})} uppkallad efter sig. Brons tar den svenska huggormen på grund av sitt uppkäftiga sicksack-mönster på ryggen. Kopparplatsen innehas av den så kallade kopparormen, som inte ens är en orm \textsc{(s.~\pageref{51b0b5a943ae2b04076f7a6cb037afd6})}. Det är egentligen en ödla som saknar extremiteter, vilket man tycker borde diska den från ormlistan. Men i och med att kopparormen har utnyttjat ormens \textsc{(se orm s.~\pageref{51b0b5a943ae2b04076f7a6cb037afd6})} kanske mest framstående egenskap, lömskhet, för att mygla sig in på listan får den vara med ändå.


 {{Utmärkt}}

}

\small{
\textbf{Korp}
\label{9461f507c1c3b6bff113be5583717658}
 ; synonym för bovaktig, gemen, lömsk eller bedräglig person. Kan också beteckna ett klassiskt redskap vid arbete med gruvdrift.

 För information om fågeln, se körp \textsc{(s.~\pageref{4e3912b2f76d00abac5253e33813a431})}.

}

\small{
\textbf{Kortbyxor}
\label{ee7b70c698267646b43630ec7a53fb81}
 , eller shorts som anglofiler kallar det, är byxor som inte sträcker sig till anklarna, utan slutar i höjd med knäna. Är kortbyxorna längre än knäet så är det för korta långbyxor, clamdiggers, highwater-byxor eller capribyxor. Den enda personen som borde tillåtas ha så långa \quotetext{kortbyxor} är Mob 47-Åke \textsc{(s.~\pageref{486ee67ac39debabed3d92a7555dcebd})}. De bästa kortbyxorna är avklippta jeans, eller sportkortbyxor i kulörta färger. Avklippta mjukisar är också med beröm godkänt. Tumregeln är ju kortare desto bättre, men syns könsorganet är det att ta i.

}

\small{
\textbf{Korv i smörpapper}
\label{401e9eb6cef7fa42d543ef85f5925021}
 är den matvara som främst bidragit till konungariket Sveriges \textsc{(se Sverige s.~\pageref{b1999637949ed135b2ca03f3a38460cc})} utveckling.
 Någon gång i mitten av 1900-talet amerikaniserades dessvärre korvätandet och smörpappret byttes ut mot ett bröd.

}

\small{
\textbf{Korv med bröd}
\label{8ed6a229bd465c6f2a0a73f65534056b}
 är en populär maträtt som är både billig och lätt att laga. Receptet uppfanns av en den tyske emigranten Charles Feltman (1841-1910) på Coney Island \textsc{(s.~\pageref{3dbacfd76b0a040ccad1eacb20def4c8})} i USA år 1871. Upptäckten gjorde Feltman av en slump när han skulle skära en brödskiva men hade en så slö kniv att han bara kom igenom till hälften. Blind av raseri tog Feltman en kryddig bratwurst och mosade ner i brödskrevan. Hans lille son Horst kom då inknallande i köket och tyckte det såg \quotetext{wunderbar!}t ut och övertygade sin fader om att saluföra rätten kommersiellt. Från början gick affärerna inte så bra, mycket på grund av att han på tyskt übermenchmaner envisades med att sälja korvarna kalla. När han skrev om receptet så att det istället blev varm korv med bröd var succén \textsc{(se braksuccé s.~\pageref{678371d35369d3d29afceb1445630833})} dock snabbt ett faktum. Till Sverige kom korv med bröd först år 1897 i och med världsutställningen i Stockholm, innan dess åts korv i smörpapper \textsc{(s.~\pageref{401e9eb6cef7fa42d543ef85f5925021})}. Sedan dess är inget sig likt och i dag finns korv med bröd i alla möjliga urspårade varianter, så som Tegare \textsc{(s.~\pageref{61a9e94d20a0e011579891609fa7d765})} och French hotdog.






 HEAD2: Recept på korv med bröd
 1. Klyv en bit bröd till hälften (numera finns också färdigkluvna, så kallade \quotetext{korvbröd} \textsc{(s.~\pageref{6898888a74f0d42574012debf1a6d8f3})} att köpa i de flesta matvarubutiker).
 2. Lägg korven i brödet.
 3. Garnera efter egen smak. Senap och ketchup brukar vara populära kryddor, och allt utöver det är egentligen mest tänkt för storfräsare \textsc{(s.~\pageref{4db17005692cd83e3e946a1311b81ed0})}.

}

\small{
\textbf{Korv-Ivars}
\label{e42e3fd4f6b398bd3bb69c234431269d}
 var en legendarisk korvmojje vid busstationen i Skellefteå som enligt fleras utsago ska ha serverat landets finaste parisare \textsc{(s.~\pageref{5aca28013b9a7e4088e7fb228f3e4827})}, fram tills det att mojjen av oklara skäl behövde byta ägare och snart blev ett mer konventionellt gatukök med samma namn, men utan landets finaste parisare.

}

\small{
\textbf{korvbröd}
\label{6898888a74f0d42574012debf1a6d8f3}
 Korvbröd

}

\small{
\textbf{Korvbröd}
\label{6898888a74f0d42574012debf1a6d8f3}
 är en speciellt djup röd som alla nämnvärda konstnärer har på palletten. Namnet kommer sig av att pigmentet i färgen traditionellt kommer från den högt järnhaltiga jorden i den ukrainska provinsen Korvb, som gränsar till Ryssland. Numer framställs dock syntetiskt pigment industriellt, men vissa användare menar att denna färg troligen kommer att förlora sitt färgdjup efter ett antal år. Det syntetiska pigmentet har dock endast framställts sedan 1989, så än är det för tidigt att säga om denna risk föreligger eller ej.


 {{Utmärkt}}

}

\small{
\textbf{Kovändning}
\label{f5443b8a36759ce46480e6a7992cc4f2}
 Att hastigt och olustigt vända sig, både åsiktsmässigt och geografiskt. Den anspelan som namnet har på bovin kreatur härstammar från kors traditionella position som kälkborgare \textsc{(s.~\pageref{0f34b469a48952e93688861083ace75a})}.

}

\small{
\textbf{Krav}
\label{36fc01ffadfd76625315893ed67f635c}
 Svenska KRAV har en dold agenda

 Officiell Version

 Krav, formellt KRAV ekonomisk förening, svenskt organ som utvecklar regler för ekologisk odling, förädling och tjänster. Krav är en ekonomisk förening som har till uppgift att få till stånd en ökad ekologisk odling, djurhållning, hållbart fiske och förädling genom att tillhandahålla trovärdiga regler, kontroll och märkning av ekologiska produkter. I samma syfte ska Krav bedriva en aktiv informationsverksamhet. Kravmärket på en produkt anger att den är framställd enligt Kravs kriterier.

 Krav bildades för att skapa en trovärdig märkning av ekologiska livsmedel och förenkla för konsumenter att göra \textsc{(s.~\pageref{c4774ec92abe06f5664e18f44446d7e7})} en miljöinsats genom sina dagliga inköp. Målet är att bedriva en långsiktigt hållbar och ur konsumentens synvinkel förtroendeingivande produktion av livsmedel och andra produkter av hög kvalitet.

 Nisses research har visat:

 Krav är bara ett hittepå, de svenska lagarna tvingar bönderna till samma miljökrav. Således behövs inte KRAV-märkningen. Däremot har alla KRAV-produkter en touch av bajs smakmässigt, som Jules Verne påtalat i olika oberoende smaktester.

 http://www.krav.se/sv/

}

\small{
\textbf{Kravspecifikation}
\label{b6873dbaad6d1ae16eb34efac4218c11}
 Något \textsc{(se gobit s.~\pageref{86063054a9db002c783f9cac4f459803})} måste uppfyllas, men det är väldigt oklart vad.

}

\small{
\textbf{Krigsgrisen}
\label{e95477b4d8cdc0372ffcac1d533623a2}
 är en tecknad gris med stora betar och hjälm som sedan 1977 varit heavy \textsc{(s.~\pageref{7cfe64ea44dc3bbeb63b29ff3039a481})} metalbandet Motörheads maskot. Egentligen heter den Snaggletooth men det är det inte så många som kallar den. Valet av maskot tros hänga samman med att Lemmy är väldigt intresserad av krig och råa grejer. Ungefär hälften av alla hårdrockare har krigsgrisen tatuerad någon stans på kroppen (och den andra halvan har spader ess). Om man frågar Lemmy vad han tycker om att folk har deras krigsgris intatuerad svarar han belåtet att han tycker det är \quotetext{fuckin' great}.
 , kedja och järnkors.]]

}

\small{
\textbf{Krille}
\label{a49833282b3e64d27fb1d08ceb6b2538}
 Röker hasch och spelar Wow

}

\small{
\textbf{Kriminalroman}
\label{7651f0db40825f508a14f8111a05711e}
 En kriminalroman, eller deckare som det också kallas, är en bok som heter något coolt och suggestivt som typ \textit{Innan frosten} eller \textit{Leopardens öga} men sedan visar sig handla om en föråldrad polis med njursvikt som alla andra kriminalromaner. Nästan alla svenska kriminalromaner är omåttligt populära i Tyskland \textsc{(s.~\pageref{b1b58da783b6d5fa090f3015f1889869})} på grund av att man där ändrar omslaget så att det föreställer ett rött hus, för detta tilltalar av någon anledning den tyska mustigheten \textsc{(s.~\pageref{682ccd5fdc3aff0c97e8845c3d6b6ca8})}.

}

\small{
\textbf{Kristdemokraterna}
\label{18a843e4776b5003d411ce0148bab148}
 är ett svenskt parti för människor som inte gillar:

 \begin{itemize}
 \item Kvinnor
 \item Ungdomar
 \item Bögar
 \item Flator
 \item Transpersoner
 \item Queer
 \item Ensamstående föräldrar
 \item Vilda djur (framförallt rovdjur)
 \item Arbetarklassen
 \item Sjuka människor
 \item Politiker (detta kan te \textsc{(s.~\pageref{569ef72642be0fadd711d6a468d68ee1})} sig en smula orimligt, men det är helt i deras linje)
 \item Hedningar
 \end{itemize}

 De gillar däremot:

 \begin{itemize}
 \item Kyrkor
 \item Pengar
 \item Färgen beige
 \item Familjer
 \item Att tala i tungor
 \item Verklighetens folk \textsc{(s.~\pageref{9404cebf965f11b660c6ad262d3c2433})}
 \item Framgångsteologi \textsc{(s.~\pageref{0c3b399eedaaef3220f1e59f6d138055})}
 \end{itemize}

}

\small{
\textbf{Kristen klädstil}
\label{ef5cd8cd88d4ae3951776d72a8cd2a28}
 För snart tvåtusen år sedan red en excentrisk man in i Jerusalem på en åsna. Fattiga judar strödde palmbland framför honom och ropade \textit{Hosianna, Davids son!}. Man lyssnade till mannens hoppingivande men också omstörtande ord. Han talade om sin faders rike och om att älska sin nästa, även de som lever i synd och skam. Snart stod mannen inför den ångerfulle Pontius Palatius, som inte såg någon annan utväg än att ha honom uppspikad på ett kors och pinad i trenne dagar. Så gav herren sin förstfödde son för syndernas förlåtelse. De som idag minns detta fantastiska evangelium klär sig i vissa kläder och undviker andra kläder. Den som rättar sig efter bibeln \textsc{(s.~\pageref{7de7d2a7d608c9a2044f50688bc63e27})} har svårt att däri se vilka kläder som herren godtar och vilka som förarga honom, eftersom folk i den tid då bibeln skrevs endast gick iklädda säckväv \textsc{(se särk s.~\pageref{7a522dc7e11bd1136642b3452855c1d6})} eller i bästa fall toga. Men på något vis har den svenska frikyrkorörelsen kommit fram till följande:

 HEAD2: Kläder som herren godtaga
 För män gäller:
 \begin{itemize}
 \item Chinos
 \item Seglarskor
 \item Enfärgad eller diskret randig skjorta
 \item Enkel kavaj
 \item Slips
 \item Poncho av säckväv
 \end{itemize}
 För kvinnor gäller:
 \begin{itemize}
 \item Långkjol eller byxdress
 \item Högt knäppt blus
 \item Låga pumps
 \item Ett stort halsbandssmycke. Gärna ett kors, förstås.
 \item Poncho av säckväv
 \end{itemize}
 HEAD2: Kläder som förarga herren
 \begin{itemize}
 \item Keps
 \item Jeans \textsc{(s.~\pageref{a0f2589b1ced4decbf8878d0c3b7986f})}
 \item Kort kjol
 \item T-shirt
 \item Vindjacka. Det ser för sjaskigt ut!
 \end{itemize}

}

\small{
\textbf{Krister}
\label{6e117d314c94f5fa1600abb83f952112}
 är ett mansnamn som kommer sig av ett skämt om en kines som försöker köpa klister i en konsumbutik \textsc{(s.~\pageref{70e4875f7c2c177596305006e46b7ca9})} men inte kan göra sig förstådd.
 Nästan alla på nissepedia \textsc{(s.~\pageref{62400dadecd90cb5cd39062abe5a3e4a})} ser naturligtvis på sådana skämt som förkastliga. De är reaktionära och tillhör det föregående århundradet, är den officiella hållningen.

}

\small{
\textbf{Krister Sturmark}
\label{a1bd5d5fcede775c9cedc4aa0cd32ed8}
 Talesperschon för det autistiska Sverige \textsc{(s.~\pageref{b1999637949ed135b2ca03f3a38460cc})}.

}

\small{
\textbf{Kristerstånd}
\label{eacb7b3161f8220a0858dcc056a2fc8c}
 är en benämning på den del av framdelen på ett par byxor, där gylfen sitter, där det ofrånkomligen bildas ett veck som står ut. Ordet är etymologiskt besläktat med en av Rådmansö SKs föredetta b-lagsmålisar som kännetecknades av pipig röst och, enligt envisa rykten, smått tilltaget kön.

}

\small{
\textbf{Kristoffer}
\label{b1bebf7f19c7345d261dd0f1f7746f00}
 betyder såklart någon som offras till minnet av Jesu \textsc{(se Jesus s.~\pageref{110d46fcd978c24f306cd7fa23464d73})} korsfästelse.

}

\small{
\textbf{Kroft}
\label{25bc57a09a9adb39248c272032b0bde3}
 Kraftigt soft. Motsatsen till Bull \textsc{(s.~\pageref{c4ceb152db108935c71875ae7eaeaaec})}.

}

\small{
\textbf{Krokben}
\label{85c111491df5c9adeb8d907f3203238f}
 är ett av våra mest tidlösa skämt. Skämtet går ut på att man sätter ut sitt ben framför någon som går förbi så att denne snubblar, att man går bakom och sparkar till någons fot så att den hakar i det andra benet så att personen snubblar eller att man kör in foten mellan någons ben från sidan så att denne snubblar. Poängen är alltså att man medelst sitt eget ben får någon att snubbla, vilket ju är hejdlöst roligt.
 HEAD2: Klassiska krokben
 Carl von Linné \textsc{(s.~\pageref{5e8380bf6b7ce99678e6752b6d9e709e})} ska enligt egen utsago ha lagt en del krokben. Bland annat ska den svenske vetenskapsmannen ha lagt krokben för en same under sin norrländska resa och för en gås under en kort vistelse i Lund under sin studietid. Även Artur Hazelius \textsc{(s.~\pageref{cfe3ab83bbf192ab78a5b06cdd7cbf9f})} ska flitigt ha lagt krokben för folk runtom i landet.

}

\small{
\textbf{Krokodiljägare}
\label{0e2ecd15474fec79c6cf263b058afd94}
 är en yrkesgrupp man väldigt sällan stöter på i de svenskspråkiga delarna av världen. Vissa menar att det beror på att det inte finns några krokodiler här, medan andra snarare anser att det är ett resultat av att svenska krokodiljägare gjort ett lite väl bra jobb. Hur man än ser på saken är alla överens om att krokodiler är jävligt balla eftersom dom är typ som dinosaurier fast inte utdöda.

}

\small{
\textbf{Kroppshydda}
\label{032eed30d2aad3425b9139aafdd6740f}
 Namnet på den bastkjol Fantomens kamrat Guran går klädd i.

}

\small{
\textbf{Krus}
\label{2a95ddf371e46d685f45c0f173f8b7e2}
 Ett krus är ett dryckskärl som är mer rejält än nått litet fjantigt champangeglas men också mer behändigt än ett tråg \textsc{(s.~\pageref{1e0e0470206e0f2baad8e628ba8f770c})}. Kruset är ofta gjort av lera och är försett med ett redigt handtag där en smedslabb går in utan problem. Kruset kan användas i slagsmål om så behövs, men kan framförallt höjas mot taket så att ölet skummar i en skål för brödra- och dryckesskap.

 HEAD2: Krus i litteraturen
 \textit{Hemsöborna} av Absinthpundaren tillika författaren August Strindberg inleds med meningen: \quotetext{Han kom som ett yrväder med ett höganäskrus i en svångrem om halsen}. Vad som nu menas med det.

}

\small{
\textbf{krypa ihop i soffan som en katt}
\label{4d1d36fa3c68844e34049d8b7db95af2}
 Ofta kryper smala tjejer upp i soffor som katter. Där värmer de sina händer på tekoppar \textsc{(se te s.~\pageref{569ef72642be0fadd711d6a468d68ee1})}. De talar kanske franska och säger sig älska ost men ingen har sett dem äta någon.

}

\small{
\textbf{krypa upp i soffan som en brugd}
\label{ea7efaf2898e75a3e2cd2bb76fc18568}
 Att sitta i soffan, gapa och bara ösa in olika födoämnen i munnen \textsc{(se mun s.~\pageref{6585f290ce92c3de5ff339920330e26f})}.

}

\small{
\textbf{krypa upp i soffan som en uv}
\label{ae8f7b4e20767fca9e50538495ec4954}
 Folk med storhetsvansinne \textsc{(s.~\pageref{2f9c0ea6231e1de87c97eab41410c795})} kryper ofta upp i soffor som uvar \textsc{(se uv s.~\pageref{45210da832f9626829457a65e9e7c4d0})}. Där läser de tjocka bruna böcker och hoar för sig själva av förnöjdhet. De talar kanske tyska och äter mörkt, redigt bröd som de klappat på en ostkiva på.

}

\small{
\textbf{Kräfta}
\label{31d4f9ec82e212d1a52dc283f7335710}
 Urtidsdjur som är populärt som tilltugg eller elakartad sjukdom som ofta leder till döden \textsc{(s.~\pageref{6f3c270eb5b4d979c777b4ec26dd106f})}.

}

\small{
\textbf{Kräftbete}
\label{76499bc9cc050bed2beb8e36dd601066}
 är sådant som kräftor \textsc{(s.~\pageref{ae3aef2c38d9c11d2147301f1022b04e})} äter. Kräftfiskaren lägger kräftbetet i en kräftmjärde eller modifierad mörtsump som de läckra skaldjuren vandrar in i och sen inte hittar ut ur. För att många kräftor ska gå dit är det viktigt att ha ett bra kräftbete. Det absolut bästa är nyfödda kattungar \textsc{(se katt s.~\pageref{0fd9accd1d8c95e86a96f681b6805948})} som dränkts dagen innan och hunnit ligga och götta till sig lite. Rutten fisk eller en skopa blandade sopor \textsc{(se skräp  s.~\pageref{75f1a5320951ea0dd9aa3c0eaba2c2c7})} fungerar också, men inte lika bra. Kräftbete är ett mycket viktigt inslag i att hålla den ekologiska balansen av sommarkatter under kontroll och kan därför berättiga till EU-bidrag \textsc{(se EU s.~\pageref{4829322d03d1606fb09ae9af59a271d3})}.

 Källa: Hedkärra högre läroverk.

}

\small{
\textbf{Kräftor}
\label{ae3aef2c38d9c11d2147301f1022b04e}
 är små röda gynnare som springer runt på botten av sjöar och nyper i saker. Dess blod är blått och dess syn mycket dålig. De lever i flock tills de blir riktigt gamla, då vandrar de ut till havet för att vara ifred och kontemplera. Såna individer kallas krabbor. Utvecklingsmässigt befinner sig kräftan på ungefär samma nivå som Europas människor under medeltiden \textsc{(s.~\pageref{88cbc30c5b233d97df68b5b041ac0655})}, vilket bland annat märks på att den äter ruttna kadaver och kan drabbas av pest. Det är tillåtet att jaga kräftor med alla till buds stående medel.
 HEAD2: Disambiguation
 Kräfta kan också vara en sjukdom, vilket förklarar uttrycker \quotetext{att stupa i kräfta.}

}

\small{
\textbf{Kränkande}
\label{5311bb8220aa4c45c14a860bfaa3b0db}
 Kränkt \textsc{(s.~\pageref{e8b03aca868764904f2741ed0b21fb17})}

}

\small{
\textbf{Kränkt}
\label{e8b03aca868764904f2741ed0b21fb17}
 På 1900-talet innebar en kränkning typ att bli avklädd och piskad offentligt. Såhär 100 år senare innebär det att någon med enormt uppblåst ego tvingas möta den bistra verkligheten.

 HEAD2:  Några handfasta exempel

 \quotetext{Björn Ranelid \textsc{(s.~\pageref{b374a5d86cf98cd5ba2a0ff96d5a9e97})} anser sig oerhört kränkt av bevisen på att han är förståndshandikappad}
 \quotetext{Orgasmatron Andersson \textsc{(s.~\pageref{992f857a2415202c7eb4b9f973ea11a0})} känner sig oerhört kränkt efter att ha förväxlats med the Fog \textsc{(s.~\pageref{576875ef0042ff21c04f5f1b9377d4e7})} på omslaget till \textit{Cosmos Factory}}

}

\small{
\textbf{Kråkslott}
\label{eb22b3e5b2c5c71fdff3899ed6cc3ecc}
 Ett kråkslott är en byggnad som pga sin fula fasad (liksom kråkornas bo) blir kallat just för \quotetext{Kråkslott}. Varför är ännu okänt men rykten säger att det ofta bor just ruggugglor \textsc{(se rugguggla s.~\pageref{6af1b4e2e210d1aef03643fb57c86bc2})} eller fyllkajor i just dessa, det händer även att de träffas och um-gås.

}

\small{
\textbf{Kubb}
\label{de7f6954ec8c6e346b8ba18ae018d334}
 är ett rejält fikabröd som utgör en egen gren på gofikats släktträd. Kubb är närmast besläktat med bulle. Kubb, speciellt mandelkubb, äts genom att kubben doppas i kaffe \textsc{(s.~\pageref{a51a0cac0ce374a853d2359417debc28})}, likt en kaffefisk \textsc{(s.~\pageref{af1258c212f378e0d974ac807a91ab79})}. Man kan med fördel äta kubb när man kollar på bandy. Det är viktigt att verkligen låta kubben suga till sig tillräkligt med kaffe. Ett typiskt nybörjarmisstag är att bara snabbdoppa \textsc{(s.~\pageref{017d4cf4a8dc7d8d4801b949df3e3f6e})} eller inte alls doppa, varvid kubben helt förlorar sin mening.

}

\small{
\textbf{Kukenkillar}
\label{3cf0284428a6f396e261986d14927a1b}
 gillar dressmann \textsc{(s.~\pageref{02ee8e32b89869fffd11aceb4f2e1c10})} och hårpomada. Slitz är en jäla bra tidning med många intressanta artiklar och riktigt bra raggninsrepliker. Kukenkillar har svårt för andra än kukenkillar.

}

\small{
\textbf{Kulaker}
\label{c17322f1f8b87ec8fc35538dbe1e9668}
 Rika bönder \textsc{(s.~\pageref{30a6fc00c9102680b8196b1b79935ec4})} som älskar fri rörlighet \textsc{(s.~\pageref{4360ec2152e06fe380908bc2b75f2cb3})}, sin oxpiska i äkta läder och drömmer om en återgång till medeltiden \textsc{(s.~\pageref{88cbc30c5b233d97df68b5b041ac0655})}. Kulaker finner man huvudsakligen i Skåne \textsc{(s.~\pageref{a01d1167b9dcd72e212d876d672db261})} där dom lever gott på EU:s arealsstöd. Man kan dock finna kulaker så långt norrut som Västerbotten \textsc{(s.~\pageref{d4b008c5143dcffb6b8c35f3876c2a19})} där  Centerpartiets \textsc{(se Centerpartiet s.~\pageref{e331dec360e356adc1e2db36fe9a9f3f})} numera före detta översterabbin häckar.

}

\small{
\textbf{Kulturstökigt}
\label{9fd9e5739d4dbe0c1e0742eb2abcbb8e}
 är ett sanitärt stadie som ofta påträffas hemma hos kulturintresserade och akademiker. Eftersom individerna i dessa grupper är så fruktansvärt upptagna hela dagarna med att ta del av saker som inte är så viktiga men väldigt intressanta så hinner de helt enkelt inte städa \textsc{(s.~\pageref{793f88411898643a984c343fa86deb5e})} hemma. Vanliga saker som ligger framme i en stor röra är uppslagna böcker, tekoppar, en sån där krokidocka i ljust trä, akustisk gitarr (ibland), tvätt, skivfodral, en trasig stol från någon dyr designer. Eftersom dom som har kulturstökigt hemma är så upptagna hela tiden med intressanta saker så skäms dom inte för det.

}

\small{
\textbf{Kusin}
\label{f7f20d5744925e2e72e5524035a162be}
 En kusin är en person som ofta upplevs som mycket stötande, men som du enligt lag måste träffa ett par gånger om året. Kusinen är intesserad av sådant som fyrhjulingar, techno och dumma idéer, om han är av manskön, och hästar \textsc{(se häst s.~\pageref{b4c608370b339da095c5f8db7fab0945})} och kläder om han är av kvinnokön. Din farmor/farfar eller mormor/morfar verkar av någon outgrundlig anledning tycka mer om kusinen än dig, trots ovan nämnda, fullgoda skäl till det motsatta. Kusinen får alltid fetare julklappar än du.

}

\small{
\textbf{Kvalitativ tisdag}
\label{6217b20de776eea9406d83189219cd0a}
 att ligga på sofflocket och dricka tingsryd, tittandes på Himmlerdokumentär

}

\small{
\textbf{Kvantitativ innehållsanalys}
\label{8cbd40215a0453bdd47cd6ef47c53ec2}
 är en metod som syftar till att möjliggöra kvalificerade generaliseringar av material av större omfång, att tydliggöra de generella strukturer forskaren identifierar. Metoden används för att göra systematiska klassificeringar och beskrivningar av kommunikationsinnehåll utifrån vissa förutbestämda, tydligt angivna, kategorier i ett kodschema. För att detta ska lyckas krävs ett systematiskt och formaliserat upplägg av analysen där forskaren arbetar med variabler som fungerar som standardiserade frågor hon eller han ställer till innehållet i en text. För att en kvantitativ innehållsanalys ska fungera så bra som möjligt finns det framförallt fyra begrepp forskaren hela tiden bör förhålla sig till för att uppnå pålitliga resultat, dessa begrepp är: objektivitet, systematik, kvantitet och manifest innehåll. Objektivitet innebär att innehållsanalysen ska vara utformad på ett sådant sätt att den enkelt kan upprepas av andra forskare och fortfarande komma fram till samma resultat. Systematik har jag redan nämnt tidigare och handlar om att analysens tillvägagångssätt ska vara väl definierat. Ett systematiskt genomförande är ett viktigt sätt för att uppnå objektivitet. Det tredje begreppet, kvantitet, innebär att variablerna i kodschemat ska vara kvantifierbara och kunna sammanställas på ett sätt som möjliggör fastställanden av statistiska samband i form av omfång eller frekvens. Det sista av dessa fyra viktiga begrepp är manifest innehåll som handlar om vikten av att avgränsa sin analys till sådant som går att klart utläsa ur texten utan att någon kvalitativ tolkning ska behövas för att använda sig av sina variabler. Detta undviks genom att variablerna och förklaringarna till hur dessa ska tillämpas formuleras på ett sätt som gör dem så entydiga som möjligt. Professorerna Olle Findahl \textsc{(s.~\pageref{433e1dc6d01073b9b2b4a5a6294d0597})} och Birgitta Höijer \textsc{(s.~\pageref{9c402d608d4c87384133bd5f8b522574})} skriver dock i sin bok \textit{Text och innehållsanalys – En översikt av några analystraditioner} att forskaren redan vid en genomgång av det manifesta innehållet många gånger tvingas till en viss form av subjektiv tolkning för att förstå innehållet.  Så snart forskaren vill gå längre i sin analys än att titta på ordens längd eller antalet ord i en artikel blir analysen i viss mån kvalitativ. Findahl och Höijer skriver att lösningen på detta blir att forskaren bestämmer sina tolkningar i förväg genom att göra dessa vid konstrueringen av variablerna.  Tolkningen görs så att säga i förväg innan själva analysen genomförs.


 HEAD2:  Kritik

 Vad gäller kritiken mot kvantitativ innehållsanalys har den i den litteratur jag tagit del av främst rört tolkningsfrågor. Om man exempelvis har ämneskategorierna ”viltjakt” och ”brott” och en tidningsartikel berör jaktbrott, hur ska den då klassificeras? Forskaren kan aldrig i förväg skydda sig fullständigt från tolkningsdilemman utan att göra antalet kategorier i kodschemat lika många som antalet artiklar i materialet.  Detta är ett problem för all forskning som vill, så att säga, mäta frekvenser och volymer av en text i en positivistisk (idén om objektiv kunskap) forskningstradition. Än mer problematiskt blir det om forskaren vill registrera någon form av perspektiv \textsc{(s.~\pageref{1606dd19366985367d677f7b6de46e52})} som exempelvis positivt eller negativt. Vad som bedöms som positivt eller negativt är givetvis en subjektiv tolkning och detta påverkas inte av att forskaren definierar sina tolkningar redan vid konstruktionen av variablerna. Forskaren kommer många gånger antagligen behöva göra helhetsbedömningar av en text för att tillämpa sina variabler på den. Resultaten av undersökningen kan ofta dessutom bara tolkas i termer av omfång och man kan fråga sig om två positiva nyhetsartiklar om ett ämne verkligen betyder dubbelt så mycket som en negativ?

}

\small{
\textbf{Kvarka}
\label{845b5a3b4cbfd68185b5bc6877f01a42}
 är en luftvägssjukdom hos hästar. Den kan vara riktigt äcklig och lyder under epizootilagen \textsc{(s.~\pageref{bb3fa656326993784ac864edc12a2373})}.

}

\small{
\textbf{Kvicktänkt}
\label{f06ed437f6ad7eeafae17b1a824bf4ee}
 kallar man personer som är lite bättre än andra på att lösa problem. Den kvicktänkte låter sig inte luras av borgerliga lösningar som att lämna in bilen på verkstad eller betala för snöskottning. Istället kopplar hen förbi elfelet med hjälp av ett gem och lägger ut en bro av lastpallar från dörren till trottoaren.

 Se även Quicktänkt \textsc{(s.~\pageref{27b95c92dee2e106157ff07529c6f059})}

}

\small{
\textbf{Kvinna}
\label{9a7760b2521c3471c47cd5d789a2d324}
 En kvinna är en människa av honkön. Man föds dock inte till kvinna, sägs det. Man \textsc{(s.~\pageref{39c63ddb96a31b9610cd976b896ad4f0})} blir det.

}

\small{
\textbf{Kvinnlig författare-knepet}
\label{2df7cf3cc32dd55b7c833e6220d42c4a}
 Ett beprövat knep hos unga män som önskar att få ligga med medelklasstjejer. Tillfrågad om vem som är den unge mannens favoritförfattare drar han till med en kvinnlig sådan, för att framstå som en fördomsfri och härlig person som man kan diskutera Mare Kandre och Virginia Woolf med under filten när hösten faller på. Detta, hoppas han, ske leda till att den unga kvinnan \textsc{(se kvinna s.~\pageref{9a7760b2521c3471c47cd5d789a2d324})} erbjuder sig att ligga med honom. Detta händer också, men inte så ofta. Det beror på hur han ser ut.
 HEAD2: Exempel på kvinnliga författare
 \begin{itemize}
 \item Sylvia Plath
 \item Susanne Brøgger
 \item Toni Morrison
 \item Hanne Vibeke-Holst
 \item Joyce Carol Oates
 \item Margaret Atwood
 \item Carolyn Cassidy
 \item Ayn Rand \textsc{(s.~\pageref{08bc6fbe57f6ad5c154a3d93954fbb9b})}
 \item Selma Lagerlöf
 \end{itemize}

}

\small{
\textbf{Kvinnligt alibi}
\label{60da199ecfe5b75a702ff11156c333df}
 Ett kvinnligt alibi behöver varje organisation som vill framstå som modern och jämställd trots att makten till 99,9\% ligger i händerna på samma gubbar som i resten av världen. Inget är så uppfriskande som en fräch tjej i kavaj och vit blus.

}

\small{
\textbf{Kvinnokläder}
\label{d09ebf3842ad7452891cf646bf47b3a0}
 är en gren på mode-rikets brokiga träd. De plagg som ingår i gruppen kvinnokläder är sådana som den modemedvetne samhällsmedborgaren i första hand associerar med kvinnan, men som i vår postmoderna värld ingalunda bärs uteslutande av kvinnor bara för det, skall understrykas. Typiska kvinnokläder är kjol, sjal, huckle, blus och små hattar.

 HEAD2:  Undergrupper
 \begin{itemize}
 \item \textit{heroin chic}
 \item Tantkläder \textsc{(s.~\pageref{876b87b4d70dd7225b5b665038841361})}
 \item H\&Ms \quotetext{rockiga} tjejkläder \textsc{(s.~\pageref{f50a15920ea21e41e82b93a6e876ad6f})}
 \item Det ryska babusjka-stuket \textsc{(s.~\pageref{82b3ee065c37c75fdee61cdf1edd9705})}
 \item Multikultikläder \textsc{(se Multikulti s.~\pageref{25eea9148080d30d384ce1c1277ef126})}
 \end{itemize}


 HEAD2: Hur klä sig i kvinnokläder?
 Vill du känna dig omtyckt och uppskattad och dra avundsfulla blickar till dig där du går fram mellan boutiquerna? Undrar du hur du kan uppnå detta genom att skaffa dig en garderob kvinnokläder? Tänk då först och främst på att du med dina kläder visserligen vill göra ett outplånligt intryck, men att din dräkt inte får bli för skrikig. Du vill imponera, inte revoltera! Välj en diskret men \textit{chic} hatt, en enfärgad blus, kanske med en kattbrosch \textsc{(s.~\pageref{a3d22061739404335c4675738803b886})}, en bekväm men stilig byxa \textsc{(s.~\pageref{bd74f429522c7c1481fbba07187efc6b})} eller kjol, gärna ett par påkostade pumps och en liten väska att hänga från handleden.   Se även: herrkläder \textsc{(s.~\pageref{2ca60777e59875280b6db600fc53569d})}.

}

\small{
\textbf{Kälkborgare}
\label{0f34b469a48952e93688861083ace75a}
 är människor som tror sig vara lite förmer men egentligen sitter i samma skit som alla andra. De röstar gärna borgerligt, trots att de inte tjänar på det, för att det känns lite finare. Kälkborgaren älskar hockey \textsc{(s.~\pageref{df0349ce110b69f03b4def8012ae4970})} och arbetslinjen \textsc{(s.~\pageref{a7d4c1873c9542a1c6a48a1e52bdb823})}. Martin Timell är en typisk kälkborgare.

}

\small{
\textbf{Källkritik}
\label{f7d7e62bc82fef217332d721333008ea}
 I nissepediasammanhang \textsc{(se nissepedia s.~\pageref{62400dadecd90cb5cd39062abe5a3e4a})} har detta begrepp endast och enbart att göra med folks gnäll på undermåligt bordsvatten.

}

\small{
\textbf{Kängcrustare}
\label{9df07d07054775ade51db363e067695f}
 är en Crustare \textsc{(s.~\pageref{58197b69d4be0c18c21c6e8d3f950270})} som hävdar att han inte är Bög \textsc{(s.~\pageref{a1bd23bf21b5add8ef4fceba9c763237})}, därav tilläggsnamnet \quotetext{Käng}, vilket förargar Crustare \textsc{(s.~\pageref{58197b69d4be0c18c21c6e8d3f950270})}, Kängare och Träskpunkare \textsc{(s.~\pageref{484838b3db1adb135ea74d6fc61e44c0})} lika mycket.

}

\small{
\textbf{Känslo-Oi!}
\label{85cd7d30a3dde018f06b9ade6dfad5ae}
 är en så smal gren inom oi! och street-punken att inget nu aktivt band kan kategoriseras inom denna kategori. Istället bygger Känslo-Oi!-fansen sin subkultur på utvalda, speciellt blödiga, låtar från sagda musikstilar.
 HEAD2: Exempel på Känslo-Oi!
 \begin{itemize}
 \item Cock Sparrer- Platinum Blonde
 \item Bonecrusher - Don't give up on me
 \item The Crack - Don't You Ever Let me Down
 \item The Templars - My Saving Grace
 \item Cock Sparrer - We're Coming Back
 \end{itemize}

}

\small{
\textbf{Kärlek}
\label{1f9f2c658ab9f6e0136ec032f5d88783}
 Den kosmiska kraft som får den älskande att gravitera mot älskade, patrioten till fosterlandet, och tysken till korv.

}

\small{
\textbf{Kökssoffan}
\label{d1c2d6488fde9b41b5c6b2a03c5fd79c}
 är en bra uppfinning. Det är en kombination av soffa, köksstol, garderob, säng, bokhylla, bäddsoffa, förråd och fåtölj. Kökssoffan passar utmärkt för många saker, den är dock främst skapt för att ligga på. Många tror att den är till för att sitta på, det är en av anledningarna till att den är så skön att ligga på. Till skillnad från en vanlig soffa som man ligger \quotetext{i} är kökssoffan en möbel man ligger på. Den är även hårdare än en vanlig soffa vilket ger omväxling. Kökssoffan inbjuder till liggande ätning vilket kan vara nödvändigt i dagens högersamhälle. Man bör dock se upp med liggande ätning vilket kräver viss träning om man inte vill ha en fläck på alla kläder, strax under halskragen. Kökssoffan är en utmärkt plats för funderingar. Kökssoffan bör vara av trä och innehålla en tunnare madrass eller dyna. För att helt tjäna sitt syfte bör det även finnas en kudde och en filt till hands i anslutning till kökssoffan. I kökssoffan, under det så kallade sofflocket, kan man oftast förvara allt möjligt tex skridskor, julpynt, uppstoppade fåglar och bortstoppade uvar \textsc{(se bortstoppad uv s.~\pageref{86574b11bb49a6f8e32d9f716676236a})}..

}

\small{
\textbf{Köping}
\label{13b4a7ffd8f6569bf6643881dbafb812}
 (sööpiŋɠ) är Västmanlands största stad. I stadskärnan bor cirka 150.000 människor men om man även räknar in närförorterna blir invånarantalet nästan 250.000. En stor del av befolkningen härstammar från Finland \textsc{(s.~\pageref{631d44eaa1254ff71a1e11ba021d1266})}, men inflyttning sker även från Kolsva \textsc{(se helvetet s.~\pageref{9c1e91d22a5df2b06a57fba276f94b5c})} och Arboga. GPS är att rekommendera för den som besöker staden för första gången och inte vill förirra sig bland de vindlande boulevarderna som utgör centrum. Stadens välstånd byggde länge på skivbolaget Birdnests världsledande ställning som leverantör av trollpunk \textsc{(s.~\pageref{5e806ae90a53e9328e1e467a4d7b1b37})}, men detta förändrades snabbt när Ulke lade ifrån sig gitarren. Nu för tiden bärs ekonomin istället upp av Volvofabriken där majoriteten av världens kvalitetsväxellådor monteras. Birdnest har stängt sin affär, så den som önskar en komplett diskografi med Dennis \& dom blå apelsinerna är numera istället hänvisad till Tradera där Birdnest-Stempa säljer av restlagret av det en gång blomstrande imperiet.

}

\small{
\textbf{Körp}
\label{4e3912b2f76d00abac5253e33813a431}
 (Corvus corax) är en stor, svart kråkfågel. Den återfinns över hela norra halvklotet och är därmed den mest utspridda av alla kråkfåglar. Körpen är en av de absolut intelligentaste fågelarterna och använder sin talang till att föra andra fåglar bakom ljuset och leva utanför samhällets normer i möjligaste mån.

}

\small{
\textbf{Közösülés}
\label{0ebb408ff797d9cc15611b4f0c691685}
 Ungerska för \textit{knulla}. Ordet används påfallande ofta i samband med förtäring av Ungerns nationaldryck, Törley gala \textsc{(s.~\pageref{896d6fc796f5eb5ed2924f0b4c6bc540})}.

}

\small{
\textbf{L}
\label{2db95e8e1a9267b7a1188556b2013b33}
 HEAD3: L
 Länsman, bör man undvika.
 HEAD3: LL
 Detta hette varuhuskedjan Willys en gång i tiden.
 HEAD3: LLL
 Lars Levi Laestadius \textsc{(s.~\pageref{c91fcd34b5328c4a87e4ae93efa97bfc})}, en man  vars verk Nissepedias skribenter borde ägna mer tid åt än att sitta här och hitta på dumheter.

}

\small{
\textbf{L..........O}
\label{d6f530884d12f84f5c67aa91c4d6603d}
 är inte en hjärnskadad som försöker prata om Landsorganisationen, utan en kortform för fotboll som ofta används i SMS och olika chatrum på World Wide Web \textsc{(s.~\pageref{3b7d657e8b7bf25a9d524b60d9bb17df})}.

}

\small{
\textbf{Ladies in the city}
\label{e11e416f1ddc5c72727819384c85700a}
 \quotetext{Ladies in the city} (i folkmun LIC eller LIX) är en solkig real-life bar-version av \quotetext{Sex and the City} ägd av ett yngre Ryskt heterosexuellt par. Baren är lokaliserad på adressen Fischergasse 1 i Wien, andra distriktet. På veckodagarna hänger ofta Ukrainska och Ryska studenter där och då är det ett himla håll igång. De har helt ok toaletter och oresonabelt pris på öl men drinkarna är billiga och klientelet oslagbart.

}

\small{
\textbf{Lady Gray}
\label{8888280c03e22bec646398342fa20dd6}
 En riktig karl dricker Lady Gray till morgonfikat och så är det med de(t \textsc{(se te s.~\pageref{569ef72642be0fadd711d6a468d68ee1})}.

}

\small{
\textbf{Lagen om profitkvotens fallande tendens}
\label{16f3ed66024b2e53814d6f0949886431}
 När kapitalisterna säger att det \quotetext{krisar} så är det denna lag som verkar.

}

\small{
\textbf{Laissez-faire}
\label{68f6c515dec88c0215c4ce05a1166c85}
 (Fr. \textit{laissez} ung. \quotetext{skövla} Fr. \textit{faire} ung. \quotetext{Afrika}) är något som diggas utav bara helvete på ledarplats i Dagens Nyheter \textsc{(s.~\pageref{b159d08de8d21d8a6d79374b02793693})}.

}

\small{
\textbf{Lakritstroll}
\label{7d95faffde1363bedb69dce2da3947b5}
 är ett troll gjort av lakrits. I serietidningen Pellefant figurerar ett sådant. Bland finlandssvenskar är det en nedsättande benämning ämnad för svarta människor. Varför någon av finsk härkomst skulle ogilla något gjort av lakrits (eller mörkhyade människor för den delen) är dock \textit{helt} obegripligt.

}

\small{
\textbf{LANCELOT II}
\label{386a45bf415cd217bf0eb4ab02876db8}
 (RC II) avfyrades ca ½ timme efter Tsygan II \textsc{(s.~\pageref{da919dfb81083059022a634b495dac7d})} och va ett ännu större fiasko. Den blåste ut munstycket ca 1,5 m upp i luften. Munstycket återfanns ca 4 m från avfyrningsplatsen. Även LANCELOT II skickades upp på Åkerö i Leksand.

}

\small{
\textbf{Landslagsuppehåll}
\label{9933f58575b9eae3c41a3a58fe3ca808}
 är när de bra fotbollsligorna tar paus för att spelarna ska hem för att lira med sitt, i de flesta fall, betydligt sämre landslag. I Sveriges fall innebär det att spelarna får styrk av lag i andra B-nationer, spelar oavgjort med C-nationer, och vinner med 1-0 mot micronationer. En gång vann man över Spanien och en gång spelade man oavgjort mot Brasilien. De enda som gillar landslagsuppehåll är England eftersom dom alltid tror att dom ska vinna.

}

\small{
\textbf{Lappkast}
\label{9dd6698c53a9d42abffb80092f739ae2}
 Ett lappkast är att som skidåkare vända sig genom att vända ena skidan 180 grader och sen vända resten av sig.

 Lappkastet är en av många fina knep som Carl von Linné \textsc{(s.~\pageref{5e8380bf6b7ce99678e6752b6d9e709e})} lärt sig av samerna och sedan spridit i resten av Sverige.

 Ett retoriskt lappkast är däremot något helt annat. \quotetext{Nä, men DU är det!} är ett exempel på ett sådant.

}

\small{
\textbf{Lapplands Väsby}
\label{e2b43d11fea7c3193e91e202faa8d226}
 Enligt genuina fjollträskbor början på Norrland. (Fast Väsby Sameby håller inte med.) Ta pendeln från centralen mot Märsta och hoppa av strax innan. Mycket billigare än att åka till Åre eller Riksgränsen.

}

\small{
\textbf{Lapplisor}
\label{0d3e4c1085e1d029818497b4c7e624f9}
 Moderna publikaner, nytt namn, samma skit.
 Glöm inte bort att du alltid kan överklaga och att dom inte är några snutar, hur gärna dom än tycks tro det.
 Folk som tycker att det är jobbigt med lapplisor är troligen vanliga hederliga arbetare som måste parkera på gatan utanför sin surt förvärvade hyresrätt.

}

\small{
\textbf{Lappskojs}
\label{0d0eb99c8a08ae96acd7226a3cfec257}
 Lika delar renskav och potatismos.
 I nödfall kan det rena ersättas med corned beef.

}

\small{
\textbf{Lars Krogh}
\label{7a5a84f5d8e84be6ff55aa9709c3dacd}
 är en dansk man som ger ut garagerock och som ogärna inte röker stora mängder hasch \textsc{(se stenad s.~\pageref{dec4a3a91f0f2bf8dcf033a8cfeaa554})}. Han ska enligt envisa rykten vara extremt gammal och är, som sagt, dansk.

}

\small{
\textbf{Lars Levi Laestadius}
\label{c91fcd34b5328c4a87e4ae93efa97bfc}
 Predikant från Jäckvik, Lappland men slog igenom stort i Pajala, Tornedalen. Startade även en rapkarriär under artistnamnet LL Cool L och hade en hit under sommaren 1820 med låten \quotetext{Preachin' on til da break of dawn}. Tappade helt koncepten av all uppståndelse efter att han av en olyckshändelse uppfann jeansbyxan \textsc{(se jeans s.~\pageref{a0f2589b1ced4decbf8878d0c3b7986f})}.

}

\small{
\textbf{Lars-Åke Lagrell}
\label{b50e45965cb37d676d63eb0e77c6253f}
 är ca ett millenium gammal och sedan 1991 ordförande för Svenska fotbollsförbundet. Han är den enskilt största anledningen till att Sverige missade fotbolls \textsc{(se fotboll s.~\pageref{961bd74d34872ff94a4df3a16119096e})} VM i Sydafrika 2010.

}

\small{
\textbf{LAS}
\label{1102b0bde793717c4382ffd6894155fc}
 (Lagen om anställningsskydd) är det som gör att man kan ha ett arbete och ändå kunna gnälla på chefen. När borgarna skrotar LAS är det hög tid att ta upp vapen mot kapitalet.

}

\small{
\textbf{Laserögon}
\label{7d642f9221f16b36fa9d731166ba3416}
 är ett effektivt vapen i kampen för rättvisa, men är få förunnat. Med laserögonen kan man sätta den mest råbarkade kapitalist i brand med ett enkelt lyft på skyddsglasögonen. Men stor makt medför stort ansvar och man får se till så att man inte missbrukar sina laserögon till en massa jävelskap \textsc{(s.~\pageref{46845591177f16920cd586a5baf5a625})} och lurendrejeri.

}

\small{
\textbf{Lasse Åberg}
\label{52700e0748c7b425d15a4cb4f342389f}
 Mångsysslande livskonstnär och Alfons Åbergs frånvarande mamma.

 Åberg är dessutom jättegammal. Han är äldre än t.ex. svensk abortlagstiftelse, aga, andra världskriget, färg-TV, AIDS och Nissepedia \textsc{(s.~\pageref{62400dadecd90cb5cd39062abe5a3e4a})}. Belinda Olsson tycker att det är konstigt att han ens lever. Lasse Åberg är antagligen sveriges mest namnkunnige näthatare.

}

\small{
\textbf{Lasse-maja}
\label{91a370ce9cf03385b907174bfbb104e9}
 , eg. \textit{Lars Molin}, född 5 oktober 1785, död 4 juni 1845, från Ramsberg i Västmanland är en svensk äventyrare och småskurk \textsc{(s.~\pageref{c25031c5d78d9ad6fae8ab8f08d5e9dd})}. Namnet Lasse-Maja erhöll han efter att ha genomfört stötar förklädd i kvinnokläder \textsc{(s.~\pageref{d09ebf3842ad7452891cf646bf47b3a0})} och han betraktas idag som en av Sveriges första transvestiter. Hans charmigt vågade förhållande till lag och ordning förärade honom en andraplats när Kooperativa Förbundet \textsc{(s.~\pageref{818bca76a3c15e92a5fa7042b1f69758})} utsåg västmanlands häftigaste brottslingar (vann gjorde Sigvard Thurneman) \textsc{(se Sigvard Thurneman s.~\pageref{f9661f47746535d9b19e7f86bbf41dbd})}.


 Vid hans födelsekyrka i Ramsberg finns en minnessten:
 \textit{I Ramsberg hans vagga}
 \textit{I Arboga hans grav}
 \textit{I rättsprotokollet hans minne}

}

\small{
\textbf{Ledingreppet}
\label{528126abac2b649bb4fdf7bd2764726f}
 är när en kille står bakom en tjej med armarna runt hennes midja (kan med fördel också vila hakan på hennes axel \& liksom titta fram) i samband med framträdanden av rockartisten Tomas Ledin, helst under någon riktigt fin och romantisk låt. Han har nog en tröja knuten i midjan också för att han blev lite varmt där efter \quotetext{Det finns inget finare än kärleken}. Ledingreppet kan utföras i vilka mysiga musikaliska sammanhang som helst (och kallas då folkparksgrepp), men är vanligast förekommande på Ledinspelningar.

 HEAD2: Regionala skillnader
 Norr om Skellefteå kallas det Bo Kasper-greppet (verb: ”att Bo Kaspra”) pga BKA kommer från Piteå.

}

\small{
\textbf{Left-handa}
\label{ccb801aab0e3e02a30e125d2c414410e}
 Att left-handa är att snatta genom att pröjsa för nåt billigt med den högra handen och att hålla något dyrt i den vänstra, utom synhåll för kassapersonal. Har man stora händer kan man sno en liter Brämhultsjuice utan problem, har man små kan man sno en folköl. Är man fiffig så kan man left-handa bra kombinationer, t.ex. en fralla och en tub Tartex.
 HEAD2: Trivia
 Det svenska death'n'roll-bandet Entombed har skapat en temaskiva om att left-handa.

}

\small{
\textbf{Leggings}
\label{ba2d335fc737a1517329bf1ee71f77bc}
 är en postmodern variant av långkalsonger. Av någon anledning är det socialt accepterat att traska runt på offentliga platser iförd dessa leggings medan den som åker ner på macken i långkallingar blir bemött med stor misstro.

}

\small{
\textbf{Lego}
\label{3a22c9ea9a3039d180e0a514a5a3b619}
 är ett byggmaterial uppfunnet av dansken Godtfred Kirk Christiansen. Det uppfanns just efter andra världskrigets slut och var tänkt att ersätta tegelstenen, som Christiansen tyckte var skrattretande föråldrad, vid återuppbyggandet av ett Europa som låg i spillror. Danmark \textsc{(s.~\pageref{5331d7fd27772396f412a5b6d19bad44})} hade inte liksom Sverige \textsc{(s.~\pageref{b1999637949ed135b2ca03f3a38460cc})} järnmalm att sko sig på och man ville därför ta fram något som tillät att man fick sin del av den ekonomiska kakan. Innan så hann ske greps dock Christiansen på grund av alkoholkonsumtion och närheten till Tyskland \textsc{(s.~\pageref{b1b58da783b6d5fa090f3015f1889869})} av storhetsvansinne \textsc{(s.~\pageref{2f9c0ea6231e1de87c97eab41410c795})} och lät utropa republiken Legoland och i dess hjärta slå upp sen stor jävla giraff av lego. Sedan dess har Legoland varit en av landets stora attraktioner och har årligen tiotusentals besökare, men inte minst på grund av försäljningen av droger, som sker helt öppet på republikens huvudgata, pusherstreet, har relationerna till Danmarks myndigheter tidvis varit ansträngda och hot om \quotetext{normaliseringsprojekt} har avlöst varandra.

}

\small{
\textbf{Leif}
\label{8f8e2804ec8d4ceb435968cf30044844}
 (2008 - ??) är en man i sina bästa dagar som är en av de få medlemmarna i det mytomspunna motorcykelgänget Rainbow Riders \textsc{(s.~\pageref{54b5b4739e6bc150148c5019e1793413})}. Han bär rosa solhatt. Han bär träskor. Han kör kalle anka \textsc{(s.~\pageref{64db68f686a0ca4d9d641061cb3fdf13})} utomhus och nedlåter sig till att dra till en bakaxel ibland när han tycker att det inte har gjorts rätt. Men hans huvudsakliga fokus ligger helt på jordgubbar.

}

\small{
\textbf{Lelle dollars grabb}
\label{3eb708b04f17bbf88765af5c56a68039}
 ...har ett sjukt träffsäkert vollyskott och en tendens att nicka in hörnor. Hans blonda lockar gör att han framstår som trovärdig. Härstammar till skillnad från de flesta svenskar från Skellefteå.

}

\small{
\textbf{Lemmy}
\label{6cc2f8758343439728f308f08a4a8fad}
 Ian \quotetext{Lemmy} Kilmister är beviset mot tesen att \quotetext{Det enda som är säkert är döden \textsc{(s.~\pageref{6f3c270eb5b4d979c777b4ec26dd106f})}}.
 Detta är på grund av att han enda sedan 60-talet bara inmundigat amfetamin, whisky, lite engelsk skräpmat och cigaretter utan filter.

}

\small{
\textbf{Lemmy-bas}
\label{8788bd84a131e0d292f8d966b03745d4}
 En \textbf{Lemmy-bas} (ibland också kallad \quotetext{krafsbas}) är ett musikinstrument som har som uppgift att bringa med tyngd \textsc{(se heavy s.~\pageref{7cfe64ea44dc3bbeb63b29ff3039a481})} till en rockgrupp. En Lemmy-bas skiljer sig från en vanlig bas i det att den låter mer som ett brus än ett dunkande. För att uppnå detta är baskroppen utskuren ur en halvrutten tallstam som sedan konserverats med tapetklister. De fyra strängarna är uppifrån i tur och ordning gjorda av taggtråd, elstängsel, startkabel och gummisnodd.


 HEAD2:  Kända brukare av Lemmy-bas är:

 Alla kängband \textsc{(se Lista över dis-namn s.~\pageref{466d0ad02bac3d76b9aecb05ac68c5ca})}
 Lemmy \textsc{(s.~\pageref{6cc2f8758343439728f308f08a4a8fad})}
 HEAD2: Se även
 Saltbas \textsc{(s.~\pageref{f3110a9d4fef2f2adad9020c8b59249a})}

}

\small{
\textbf{Lena}
\label{8dbc672497bdc46f88e864bb1121232c}
 är ett vanligt namn bland kvinnliga entreprenörer, speciellt inom detaljhandel i vilken kvinnokläder \textsc{(s.~\pageref{d09ebf3842ad7452891cf646bf47b3a0})} saluförs, men också inom kombinationsaffärs \textsc{(se kombinationsaffärer s.~\pageref{0a2777bf1366a8a9a5b8eab9ca1496a1})} och stentrollsaffärsmiljön \textsc{(se stentrollsaffär s.~\pageref{f832a905e0a0a857d0d7eae1520c14b1})}. Speciellt framgångsrika entreprenörer får efter skriftlig ansökan till länsstyrelsen beviljat att tillägga ännu ett namn till huvudnamnet \textit{Lena}. Populära tillägg är Anna- och Eva-.

}

\small{
\textbf{Lenin-Churchhill aka Mysgubbe}
\label{80e5c2587ab5af8ef44ba0dde04327bf}
 Lenin-Churchill är smeknamet på en flaska Breznak (\textlessi\textgreaterBřezňák\textless/i \textgreater på tjeckiska). Det kommer sig av att den till synes goa mysgubben på etiketten, hållandes ett pilsnerstop i ena handen och en tänd cigarr i den andra, ser ut som en korsning av de två historiska herrarna från förra seklet. Om än några dussin kilo tyngre.
 HEAD2: Från tågkonduktör till varumärke

 Mysgubben hette egentligen Victor Cibich (1856-1915) och prydde Breznaks flaskor redan år 1906. [http://www.breznak.cz/?cibich]. Han var under sitt tämligen korta liv, Cibich dog vid en ålder av endast femtionio år, vida känd i sin by med omnejd för sin öl och matentusiasm. Tydligen skall Victor varit en så pass trevlig människa att ha att göra med så att styrelsen för Breznaks bryggeri föreslog honom som logo för dess pilsner år 1906, mot att han fick trettio stop i veckan i resten av sitt liv på deras bekostnad.

 Bryggeriet valde att göra ett uppehåll med den till synes välbärgade Cibich (dock arbetade han under större delen av sitt liv endast som tågkonduktör) på etiketten under eran av statssocialism, 1945-1990.

 HEAD2: Mysgubben till de svenska massorna
 
 De ölentusiaster som kan räknas in bland småfolket och studenter gladdes rejält när Breznak introducerades i Systembolagets ordinarie sortiment (nr 1611) hösten 2009 till det humana priset av 31.21 riksdaler per liter, 10.30 per flaska. Ett vakuum hade nyligen uppstått bland budgetpilsnern från Tjeckien då Primator höjts från 10.10 riksdaler flaskan till 12.50. Nu kunde man åter få en låda Tjeckisk pilsner för mindre än fem Jennys. Den är sommaren 2010 fortfarande den enda Tjeckiska pilsnern under tolv kronor flaskan.[http://systembolaget.se/SokDrycker/Produkt?VaruNr=1611\&Butik=0\&SokStrangar=\%u00d6L\%3aTjeckien\%3aAlla+storlekar\%3a\%3a\%3aBeska\%3aFyllighet\%3aS\%u00f6tma\%3a]

 Breznak har gjort världen lite mindre miserabel sedan 1753. Bryggeriet ligger kvar i nordliga Böhmen men ägs sedan 2009 av ölkoncernen Drinks Union där Heiniken är huvudägaren. I hemlandet bryggs den även som mellanöl (4.1\%), mörk lager och en starkare variant på 6.5\%. På tapp är den troligen så god så man inte vet vart man skall sätta munnen.

 En fascinerande egenskap hos Breznakflaskor är att de likt mörk materia finns, men för vanliga dödliga är mycket svårt att upptäcka. Detta stärks av af Hinkens tes: \quotetext{Hur mycket man än städar, återfinner man likväl ännu en flaska}.

}

\small{
\textbf{Lennart Holmlund}
\label{26d063a59c90487b11c8f5b4fa9af348}
 AKA V75kungen. Denne man/bulldog är på papperet kommunalråd i Umeå \textsc{(s.~\pageref{bd1e37dc477bb704c667ed1a4606df71})}. När han inte åker på solsemester till Thailand brukar han och hans bäste vän Krister \textsc{(s.~\pageref{6e117d314c94f5fa1600abb83f952112})} Olsson bada bastu, finljuga \textsc{(s.~\pageref{4eee5e7eab6f049c4084d3a5161016f9})} och styra Umeå med järnhand. Hans födointag består uteslutande av Ahlgrens bilar \textsc{(se bil s.~\pageref{b3188f47d2eac7efc3f1258dc673a9fe})} och choklad.
 HEAD2: Missförstånd
 Många har misstagit Angry Anderson, sångare i det australiensiska \textsc{(se australien s.~\pageref{e727d8d1b3162a732c7f706d55de64f3})} outlaw-bandet Rose tattoo, för att vara Lennart Holmlund, vilket orsakat Angry obeskrivliga svårigheter i sin karriär liksom i sitt sällskapsliv. Detta har många gånger förvärrats av att Holmlund utgivit sig för att vara Anderson - häromsistens på Wacken, då Holmlund dök upp tillsammans med Krister Ohlsson, Prof. Etienne \textsc{(se Användare: Prof. Etienne s.~\pageref{a9878d2280e5a39becac8f73d113df91})} och två kommuntjänstemän. Tillsammans framförde de Rose tattoos 'We Can't be Beaten' framför en lyckligt ovetande samling mustiga tyskar.

}

\small{
\textbf{Leopolds Kongo}
\label{9a135c36b0d8d599c2777b54c10ceb6d}
 Kung Leopold II av Belgiens Kongo är en de få saker Belgien \textsc{(s.~\pageref{f79ffe9e826a19f9f6a446c90e21c4e3})} bidragit med till världshistorien. I Leopolds Kongo, eller Fristaten Kongo, som var Leopolds personliga egendom och existerade mellan 1885-1908, dödades mer än 10 miljoner människor genom regelrätta avrättningar och mord eller svält, tvångsarbete och av detta orsakade sjukdomar, men det gjorde inte så mycket tyckte kung Leopold, som aldrig var där och egentligen aldrig lämnade sina egna ägor där han cyklade omkring på en enorm trehjuling.

}

\small{
\textbf{Lex Maskinen}
\label{84866cb0977404135f39ba81c7d98bf3}
 Om ingen dans -\textgreater Våldtäktsman

 Om dans -\textgreater Inte våldtäktsman

}

\small{
\textbf{Lex pudas}
\label{8f67692fefbd914aeb678a89a978df9c}
 Folke Pudas \textsc{(s.~\pageref{d9f740fcf09e3cbdcaba15cf92783958})}

}

\small{
\textbf{Lexe Crustare}
\label{4dc09ec7fd33842218230329beb42691}
 är en humanoid varelse som vältras i Olofströms djupaste kloaker.

}

\small{
\textbf{Libanon}
\label{73a3ab1ca04609e0540e692a4cc5f286}
 Ett land långt borta.
 I Libanon bor dels libanser, vilket är rekoderligt folk som kan dricka sprit utan att balla ur och har fördelaktigt utseende.
 Dessutom bizarra minoriteter såsom maroniter och syrianer.
 Libanons flagga innehåller en tall vilket såklart för tankarna till Malå \textsc{(s.~\pageref{41da4620e87888eaaeafcb3004a8d177})}, historiker menar att Beirut kan ha grundats av Malåbor vilka tagit en tur på rullskidor.

}

\small{
\textbf{Liberalfeminism}
\label{f3f6938f916c732c921fbeee9d82f818}
 är en ideologi som i mycket bygger på feminismen \textsc{(se feminism s.~\pageref{44a20d8673cfd6258002acb74ec2f83e})}, som i korthet är en politisk inriktning som motsätter sig Bert Karlssons idé om att alla kvinnor ska vara förtryckta och inte ha några mänskliga rättigheter, och nyliberalismen \textsc{(se nyliberalism s.~\pageref{a562ace16486d966be4513ea22aee287})}, som kan sammanfattas som en frihetsälskande ideologi som framhåller rätten hos efterblivna rika personer i slip-over att sko sig på korp-fotbollspelares arbete. Denna fusion av kvinnorättskamp och napalmvurmande nyliberalism tar sig uttryck i krav på:

 \begin{itemize}
 \item att fler kvinnor ska vara oförskämt rika och ägna sig åt att sänka andras löner
 \item att fler kvinnor i västvärlden ska få lika stor chans som sina manliga vänner att idka oljautvinning i Nigerdeltat
 \item att även kvinnor ska få fara omkring som jihun i mellanöstern och tortera män, kvinnor, barn och djur utan att det ska bli nåt jävla tjafs i vänstermedia.
 \end{itemize}

}

\small{
\textbf{Life just be that way, I guess.}
\label{11f000e9e3f96eae83688170fc343ec7}
 \quotetext{Life just be that way, I guess.} är ett citat från TV-serien The Wire, säsong 1 avsnitt 1. Det användes då som ett svar på påpekandet att en person fick öknamnet \quotetext{Snot} för att han glömt sin jacka och börjat snora.

 Citatet, taget ur sitt sammanhang i varierande grad, går att tillämpa på det mesta som vi människor stöter på här i livet. T.ex. om man handlat specerier och kassen går sönder utanför affären och alla ens konserver och medelst patentkork \textsc{(s.~\pageref{1e39785f5bab52f931dac485727645b6})} förslutna kärl sprids över parkeringen kan man, istället för att svära eller nåt annat teatralt, muttra \quotetext{Life just be that way, I guess.} och gå vidare med sitt liv.

}

\small{
\textbf{Ligga med kulturstockholm}
\label{fdb13bcf260377fba1d947da7c739223}
 Att ligga med kulturstockholm är en upplevelse som är möjlig för subkultur-trendiga norrländska killar, och kvinnliga opublicerade poeter med flätor, stickade kläder och pipröst. Detta kan vara en perspektivgivande erfarenhet och tillföra visst förnyat självförtroende, men man gör det dessvärre bara en gång, för sedan är man förbrukad. Den enda möjligheten att få obegränsad sexuell tillgång till kulturstockholm är att döpa om sig till en siffra och starta en blogg.

}

\small{
\textbf{Likgömmarmössa}
\label{493768cf234228879c93276799fb0787}
 En likgömmarmössa är det som en del människor kallar sotarrolle, alltså en vanlig mössa som inte går ner över öronen. En riktig likgömmarmössa är svart för att passa in med övrig klädsel som är bruklig när man pysslar med tvivelaktigheter.
 HEAD2: Att ha mössan DC
 Att ha mössan DC innebär att rulla upp sin likgömmarmössa ett halvt varv till och placera den långt bak på huvudet, lite som sovjetiskt marininfanteri bar sina baskrar fordomtida. Precis som otyget straight edge kan man skylla detta mode på Ian Mackaye.

}

\small{
\textbf{Lilla lovis}
\label{da7d27472ac524ce215f6fcccd531261}
 Många påstår att Lilla Lovis är könsrock. Andra påstår att hon inte alls är det, och att hon är vidrig. Lilla Lovis själv påstår att hon spelar rock och pop. Hur det ligger till med det hela överlåter vi åt experterna på Flashback.org

}

\small{
\textbf{Lillgammal}
\label{d30762f704f0731fb3b08cd0846128d4}
 Om ett barn kan kallas lillgammalt är detta enligt nyblivna föräldrar och idiotiska mor- eller farföräldrar något av en jackpot. Lillgammla barn kallas så för att de utan en strimma av skam smörar för äldre för att få uppmärksamhet. Det lillgamla barnet lägger huvudet på sned och uttalar (högt så alla ska höra) klokheter så som att man ska \quotetext{ta hand om varandra} och annan skit som den plockat upp från Disneyprogram men själv inte praktiserar. Det lillgamla barnet är med förskräckande stor sannolikhet din kusin \textsc{(s.~\pageref{f7f20d5744925e2e72e5524035a162be})}.

}

\small{
\textbf{Lillnöjd}
\label{b5af5d6e8bb3aa0c9b9c7444bb8b3e50}
 , men för The Boss himself är varje gig en ny utmaning.]]
 Känsla av vällust efter någon form av glädjande och lite oväntad bragd. Som att Hammarby IF bandy \textsc{(s.~\pageref{2f6b1282edcfc4b164f0f529b8e50d43})} efter tio förluster på rad vinner en, i de flesta avseenden, obetydlig match. Eller att man än en gång lyckas laga sin International Harvester \textsc{(se international harvester s.~\pageref{7e828c23d32c8161fd6c4e35697f9d82})} med lite ståltråd och några välriktade hammarslag istället för att köpa nya delar. Eller att någon du ogillar räknar fel på fyrtiotusen miljarder kronor \textsc{(se Fyrtiotusen miljarder s.~\pageref{c2160bffc9c5ca88e77204672e62e489})}.

}

\small{
\textbf{Lim}
\label{499e0d7e3f2f15a72fb4d114388bcb0a}
 är praktiskt om man är lite dåsig men inte kan somna.
 - Ta en klick stor som en tumnagel och sprid ut på en bit aluminiumfolie
 - Elda under med en tändare
 - Andas in ångorna som bildas
 - Sov!

}

\small{
\textbf{Limerick}
\label{09ef45a3371030c09e8497a2b562d5e0}
 är en diktform populäriserad genom Hasseåtage. Tydligen kommer den från den irländka byn Limerick, men eftersom det inte är så många som hört talas om denna säkert fantastiska plats är det heller inte så många som associerar diktformsjäveln med den. Anyway.. en limerick består av fem rader och är lite rolig. De två första och den femte består av rimmande rader på åtta eller fler stavelser. De två där mellan består av fem eller sex \textsc{(se erotik s.~\pageref{972f097461d1eab1c1ff104757bad922})} stavelser. Den första ska innehålla ett ortsnamn och en referens till en viss person som dikten handlar om.

 Till exempel:

 \textlessi\textgreaterDet bodde en snubbe i Malå
 Som hade en så vidrig stortå
 Att vart han än gick
 Ropa alla shit!
 Herr tåbira är ute och går!
 \textless/i\textgreater

 Och:

 \textlessi\textgreaterJohn Anscha va en grabb från Malå sta'
 Han skida' till Ume på under en da'
 Han läste en bok
 Blev besatt som en tok!
 Han hade upptäckt hur mäktig marxismen va'.
 \textless/i\textgreater

 Och:

 \textlessi\textgreaterJon han kom från ett kärrgruvehöl
 Han gilla' brodyr, klasshat \& öl
 Ibland med elegans
 Dansa'an Pulpdans
 Men ibland sluta dansen i en blodpöl
 \textless/i\textgreater

}

\small{
\textbf{Linda Norrman Skugge}
\label{141f8115f3f69da45ddb845f4575ac21}
 är liberal hypokondriker och nästan alltid arg på något. Först hatade hon alla som var normala och var typ familjer och hade barn \textsc{(s.~\pageref{5dfcc0aab2f3db925b2d51ba73e48946})} sen fick hon barn \textsc{(s.~\pageref{5dfcc0aab2f3db925b2d51ba73e48946})} själv och nu hatar hon alla som inte älskar barn \textsc{(s.~\pageref{5dfcc0aab2f3db925b2d51ba73e48946})}. Hade Norrman Skugge framlevt sina dagar en generation tillbaka hade hon klassats som hysterika \textsc{(s.~\pageref{72cb251805523a222408d28bcd0d4955})}.

}

\small{
\textbf{Lista på korsningar av frukt och fisk}
\label{fc226bf1567a0c95dc81da4185ca317c}
 \begin{itemize}
 \item Braxpäron
 \item Laxvinbär
 \item Fläderflundra \textsc{(s.~\pageref{a7bcb065c7d65d219825c737ce6d18fa})}
 \item Brugd-sviskon \textsc{(s.~\pageref{a6a1c3bbd109173fc773aec1dc6754e0})}
 \end{itemize}

}

\small{
\textbf{Lista på kristna rockband}
\label{10661f12937e040980e5afdb417a3ba7}
 Är du nyfiken på rock som för ut Guds \textsc{(se Gud s.~\pageref{91e49146121c992aab11a19c77e26cf0})} ord och inte bara en massa hat och livsfrånvänd propaganda? Då är du välkommen att undersöka följande band, av vilka de flesta spelar rap-metall. Vilket blir ditt CCM-band? \textsc{(se CCM s.~\pageref{b42f1990c0cee8758b64584877d69b93})} Kanske blir du inspirerad och startar ett eget kristet rockband med dina vänner?
 HEAD2: Rockband i herren Jesus
 \begin{itemize}
 \item Resurrection Band
 \item Everlife \textsc{(s.~\pageref{235344472d00f90174cd1c9e10b20b5e})}
 \item Falling up
 \item Forever Changed
 \item Advent
 \item Jerusalem
 \item Blindside
 \item New Born Soul
 \item Bleach
 \item Carisma
 \item Manafest
 \item Leviticus
 \item Pillar
 \item Superhero
 \item X-sinner
 \item Servant
 \item Soapbox
 \item Zao
 \item POD (Payable on Death)
 \item Stryper
 \item Prussian Blue
 \end{itemize}
 HEAD2: Kristen rock på radio
 Här kan den som vill streama kristen rock-radio: [http://www.christianrock.net/]

}

\small{
\textbf{Lista över anständig mat}
\label{da5ba10000e04e58f0fb6c0e23cc0106}
 \begin{itemize}
 \item Kornmjölsgröt
 \item Blöta
 \item Smolanedi
 \item Fisk och potatis
 \item Palt
 \item Köttsoppa
 \item Brutte
 \end{itemize}

}

\small{
\textbf{Lista över dis-namn}
\label{466d0ad02bac3d76b9aecb05ac68c5ca}
 Ett dis-namn är ett namn en orkester tar för att hylla Stoke-On-Trents bästa band: Discharge \textsc{(s.~\pageref{7084c38f1708430f138336428e4ac7cb})}. Här följer två listor på dis-namn, en med upptagna och en med lediga. Det är lite blandat svenska och engelska, men det gör väl inget? Om ni ska starta ett käng/crust-band så ta gärna ett namn, men hör av er så flyttar vi det.

 HEAD2: Upptagna
 \begin{itemize}
 \item Discharge
 \item Disclose
 \item Dischange
 \item Dispose
 \item Disorder
 \item Disaffect
 \item Dispense
 \item Diskelmä
 \item Disträ
 \item Disklass II
 \item Disculpa
 \item Disrupt
 \item Diskonto
 \item Disbrubtum
 \item Disaccord
 \item Disarm
 \item Dissober
 \item Diskent
 \item Disbrutal
 \item Diskrieg
 \item Disfear
 \item Disfornicate
 \end{itemize}


 HEAD2: Lediga
 \begin{itemize}
 \item Diskursanalys
 \item Diskir
 \item Disbärs
 \item Diskmedel
 \item Distinktion
 \item Distingerad
 \item Diskettstation
 \item Disneyland
 \item Diskussion
 \item Distributionsapparat
 \item Disputationsfest
 \item Disproportionerlig fördelning av jordbruksareal
 \end{itemize}

}

\small{
\textbf{Liten, liten mössa}
\label{cce31c1a6e8f4e68e19ca9a0a45823d4}
 Dvärgnäbbmus av honkön \textsc{(s.~\pageref{204e209b96ab0d93124f83ebe1dd4b03})}. Trippar lätt på tå. Går ej att ha på huvudet om man inte har riven parmesan där, och står väldigt stilla. Ej att förväxla med Mimmi Pigg \textsc{(s.~\pageref{47a20f7432f125f29ac8d0101be60ad7})}. Ej heller att förväxla med liten mössa, som går att ha på huvudet, särskilt på barn \textsc{(s.~\pageref{5dfcc0aab2f3db925b2d51ba73e48946})} och hipsters.

}

\small{
\textbf{Livets hårda skola}
\label{96466df38aab9756d72b1401c47319d6}
 [http://bengt-livethrdaskola.blogspot.com/] är en blogg. Där kan man bland annat se en film med stillbilder på vargar i 10 min. Då får du även lyssna på vargar, men dessvärre avbryter de vargarna 5:10 in i vargsången för att spela \quotetext{I vargens spår} med Roger Pontare. Vargarna återkommer 9:35 så att du får 25 sec till att njuta av dem.
 Bloggen drivs av Bengt 53år från Kvissleby, 14,5 km söder om Sundsvall.

}

\small{
\textbf{Livslektion}
\label{47e02c3f8cc50359d7814ac6883ae606}
 En livslektion skiljer sig från sitt tråkigare syskon skollektion i det att den kunskap man då får lära sig faktiskt är användbar. Ofta ges livslektionen över generationsgränserna, från gammal till ung. Ibland är livslektionerna så bra att de förs med över ännu en generationsgräns. De handlar ofta om hur den yngre bör leva sitt liv och vilka val som är bra att fatta och inte. Ibland är de rakt på sak och ibland mer kryptiska och filosofiska.

 HEAD2: Exempel
 \quotetext{Du borde bli reparatör, för två saker är säkra: Folk kommer inte att börja göra saker för hand, och maskinerna kommer alltid att gå sönder.} - Dag Örnered

 \quotetext{När jag slutade gymnasiet så sa min pappa till mig: Louise, har jag berättat för dig om n'Oskar? Han gjorde som så att han alltid tog bästklabben först, för på det viset fick han alltid ta bästklabben. Det kan ni tänka på.} - Louise Renberg

}

\small{
\textbf{Livsnjut}
\label{94b97eab956178ad369c0cd28d3debde}
 är något man bär med sig hela livet, en gåva och kan avnjutas utan några som hällst förutsättningar förutom din blotta existens.
 Ett livsnjut kan vara att tex
 - klia sig när det kliar
 - gå till sömns sjukligt trött
 - onna
 - nysa
 - pilla ut den där snorproppen
 - klossa
 Listan kan göras lång och ibland kan man till och med få för sig att det är Livsnjutet som håller oss vid liv.

}

\small{
\textbf{Ljuga}
\label{22c0de23c07cb8bfc2818368265696f8}
 behöver man som allmänhet kunna göra när det blir fråga om skuld.
 Ljuger gör man för att slippa skäll, pinsamheter eller i värre fall stryk eller fängelse.
 Det är en överlevnadsstrategi att kunna ljuga, de allra flesta gör det med jämna mellanrum medan vissa gör det för att de tycker att det är roligt.

}

\small{
\textbf{Ljuv}
\label{632bf5372f37d760ebb25b34ab411f71}
 (adj. -t, -a, -are, -ast) är ett adjektiv som används för att signalera att något eller någon är härligt, vackert och eller njutbart på ett uvrelaterat \textsc{(se uv s.~\pageref{45210da832f9626829457a65e9e7c4d0})} vis.

}

\small{
\textbf{Lobotomobil}
\label{66fb436eb11096dc74ee9956dc85b6b4}
 \textbf{Lobotomobilen}, medicinvetenskapens motsvarighet till batmobilen, transporterade Walter Jackson Freeman II's land och rike runt i jakt på orginal, karaktärer och andra filurer \textsc{(se filur s.~\pageref{e308f4e2553faf188385f17ebda05242})}.
 Lobotomobilen var en fransktillverkad Citroën 2CV Fourgonnette, möjligen årsmodell 1952.

 Med hjälp av lobotomobilen lobotomerade Freeman ca 3400 intet ont anande medborgare, däribland John F. Kennedys syster Rosemary.
 Lobotomi går ut på att man skär av anslutningarna i prefrontal cortex. Namnet kommer från grekiskans λοβός (lobos): lob; τομή – (tomē): skära/kapa.
 Lobotomi botar enligt Psyciatric Dictionary 1970:
 :\quotetext{Prefrontal lobotomy is of value in the following disorders, listed in a descending scale of good results: affective disorders, obsessive-compulsive states, chronic anxiety states and other non-schizophrenic conditions, paranoid schizophrenia, undetermined or mixed type of schizophrenia, catatonic schizophrenia, and hebephrenic and simple schizophrenia. Good results are obtained in about 98 percent of cases, fair results in some 35 percent and poor results in 25 percent are thereabouts.}
 Hur de fick ihop procentsatserna tvistar fortfarande de lärda om.

 Freeman uppfann inte lobotomin men förfinade den. Orginalversionen uppfanns av en portugis vid namn António Egas Moniz som fick ett halvt nobelpris 1949 för besväret (tillsammans med Walter Rudolf Hess (yes han hette så).
 Egas Moniz kallade sitt ingrepp leukotomi där leukos är grekiska för ren eller vit. Leukotomin var lite bökigare eftersom man var tvungen att borra hål i kraniet först.
 Freemans ingrepp, transorbital lobotomi, gick ut på att man tog en syl och knackade in den i tårkanalen med en gummihammare.
 Väl inne i prefrontal cortex så vispade man helt sonika runt lite med sylen. Lite som att borsta tänderna. Fast med en syl. I hjärnan.
 Att Freeman inte var kirurg utan psykolog gjorde inte så mycket då han bara tog 25 dollar för besväret.

 Trots att svensken Snorre Wohlfahrt utvärderade tidiga resultat redan 1947 och kom fram till att det inte var så smart lobotomerades 4500 svenskar mellan 1945 och 1966. Mest kvinnor.
 Den amerikanska cyberneticspionjären Norbert Wiener kommenterade ämnet med att om man vill att patienterna skulle vara enklare att ha att göra med så blir det ännu enklare om man tar död på dem.
 Sovjet förbjöd lobotomi 1950 eftersom de kom fram till att det \quotetext{gjorde galna personer till idioter}.

}

\small{
\textbf{Logoförslag}
\label{d7d1d8dfc87b4b4c50e8a838a77f320e}
 Lägg in bilder på logos som ni tycker Nissepedia ska ha

 Här är två förslag från mig. Förslag ett kallar jag \quotetext{hornet} och denna är inspirerad av musikens förunderliga värld. Likt hornet sprider sina toner tänker jag mig att Nissepedia sprider ljuv \textsc{(s.~\pageref{45210da832f9626829457a65e9e7c4d0})} kunskap.

 Det andra förslaget kallar jag \quotetext{wig-wam} och där har jag inspirerats av indianernas förunderliga värld. En tipi är varm, välkomnande och dekorativ; ungefär som Nissepedia. Dessutom är den uppfunnen av indianer och såna är ju ganska okända men ändå väldigt spännande; precis som det ofta känns när man läser en ny artikel på Nissepedia.
 /\textit{Kråkan}

 Jag tycker ju att båda är nå ohemult \textsc{(se ohemul s.~\pageref{91b8873590abd15ec344c2ba93d015cd})} vackra, men har ett speciellt öga för hornet. /Nicke

 Jag gillar hornet bäst i nuläget. Nu är jag ingen grafiker, men sheiken och drottningen har ju en speciell plats i alla nissepedianers hjärtan. /Den inbillade sjuke

 Om nissepedia hade varit en psych-skiva hade hornet gardvis övergått i en svamp som det hade kommit kärlek, symboliserat av regnågar, ur. Men nu är ju Nissepedia en kunskapsbank, så.. Carry on. /N

 Jag tycker fortfarande att hornet är ballast, men här har jag gjort en bild i vilken jag utgått från nisse- och -ped- morfemen i ordet nissepedia, alltså ung. \quotetext{ett land där folk som tänder på nissar vill bo.}  /N

 Och  bild nummer fyra \textsc{(se \quotetext{krypa upp i soffan som en uv}  s.~\pageref{19b5a50efac38521422d78d6c11eac7a})} talar nog för sig själv..
 t
 Är det inte en tornuggla? /J.A

 Ehum! Jag vill påminna Andersson om att: \quotetext{Uvar (latin: bubo) är ett släkte fåglar som vid en första anblick kan likna en uggla men egentligen är något helt annat} och att \quotetext{uvar kännetecknas av sin ... allmänna mäktighet.}

 Ja det där fjäderfät ser då synnerligen mäktigt ut. Beroende på vilken approach man vill ha på Nissepedia skulle jag säga att förslag nummer tre är antingen fulländat eller väldigt bisarrt. /Kråkan

 Här vill jag passa på att inflika att fulländat och bisarrt inte utesluter varandra. Men ja, alltså arslet blir lite mycket imo. Jag vill slå ett slag för hornet. Tut tut! Här kommer kunskap! /TB

 Jag tycker som sagt att hornet är fett som fan, och tycker att man lite uppsluppet skulle kunna fotoshoppa på en tomteluva på den kring jul och så vidare. /N

 Det verkar råda en ganska stor enighet om vilket av nuvarande bidrag som ligger i topp. Om ingen inkommer med någon häcklefjällsdjup \textsc{(se häcklefjäll s.~\pageref{40a7322a2ef5adb9efd69969d8f28f1e})} protest de närmaste dagarna kanske vi tar och väljer den då? /Kråkis

 Ja! Ta hornet. Nissepedia - In the shadow of the horn /John

 Nu har vi väl bestämt oss för hornet? Vi borde ha den i hörnet, som på \textlessstrike\textgreaterWikipedia\textless/strike\textgreater. /JPN

 Ja ta och fixa det nu Nisse! /Kråkan

 Skulle inte några tutstreck mellan tratten och texten vara fint? /jonarn

 Sätt igång för tusan! /Viggen

}

\small{
\textbf{Looppedal}
\label{e8090e1ce86d2924eb44581bf23dab63}
 En looppedal är ett verktyg för fattiga bandledare som inte har råd att hyra sig en egen orkester.

 Category:Redskap \textsc{(s.~\pageref{bd9feb5165ba18711b823cea4058095a})}

}

\small{
\textbf{Lothar}
\label{a2c85ab64a0c7a0197c17fd3eefe47d5}
 är den engelska bondkatt \textsc{(s.~\pageref{0fd9accd1d8c95e86a96f681b6805948})} som blev den första dokumenterade Lolkatten[http://sv.wikipedia.org/wiki/Lolcat]. När han var ca två månader gammal så togs en bild på honom poserande på hans ägarinnas axel som sedan skulle bli känd som den första bilden föreställande en Lolkatt.

 Han var under sin livstid en riktig linslus och detta ledde till att ett rykte florerade om att han skulle vara gay bland hanarna vid hans revir på Winchester Street, Hampshire. Lothar gillade de facto att posera med familjemedlemmarna men försökte under den senare delen av sitt liv tvätta bort den oseriösa stämpeln han fått genom de foton som togs av honom som ung. För Lothars del var det aldrig tal om att göra någon modellkarriär då han inte var en raskatt, pga detta var han periodvis deprimerad och missbrukade kattmynta för att orka med att sköta sina dagliga rutiner i hushållet såsom att jaga möss och garnnystan. Genom att ofrivilligt undvika en modellkarriär levde han dock i det stora hela ett förhållandevis lugnt kattliv och somnade in vid en ålder av arton år.

 Lothars minne bevaras endast av Internet då han själv inte erkände sitt faderskap till några kattungar under sin livstid.

 Lothar har även gett namn åt en av warcraftuniversats största krigare, sir Anduin Lothar, knight champion of the kingdom of Azeroth under hordens första invasion av hans hemvärld och supreme commander of the armies of Loarderon under hordens andra invasion. Denne Lothar dog i strid, man mot man, med den väldige orchen Warchief Ogrim Doomhammer, just innan alliansen segrade och den mörka portalen stängdes.

}

\small{
\textbf{Lucia}
\label{3ba430337eb30f5fd7569451b5dfdf32}
 är en högtid som sanktionerats av svenska staten allt sedan folkskolereformen år 1842. Med anledning av att det blev obligatoriskt att gå i skolan behövde staten klargöra vilka ämnen alla svenska barn förväntades kunna. Eftersom det skulle bli så tråkigt att bara räkna matte och läsa om Nils Holgersson hela dagarna kom staten på att man också skulle ha undervisning i musik. På vårterminen bestämde man att eleverna ska öva på \textit{Idas sommarvisa}, \textit{Skala banan}, \textit{Horgalåten \textsc{(s.~\pageref{6514f071de56f9f33e6df4cd42b24a5d})}}, \textit{Den blomstertid nu kommer} och \textit{Hasta mañana}. På höstterminen hade man dock ingen naturlig högtid att öva inför, och det var då någon kom på att man kunde damma av den gamla luciatraditionen. Tidigare hade lucia firats stort i Sverige men detta åkte ut med buller och bång när Gustav Vasa vevade igång reformationen. Nu började dock en intensiv propagandaapparat att arbeta för att återinföra traditionen. För att göra högtiden populär även hos vuxna (som egentligen tycker det är skittråkigt att lyssna på bräkande barnkörer) finslipade man den en aning och lade till att det var helt okej att sitta i särk \textsc{(s.~\pageref{7a522dc7e11bd1136642b3452855c1d6})} och spy ner sig. Det blev braksuccé \textsc{(s.~\pageref{678371d35369d3d29afceb1445630833})} på en gång och lucia firas allt jämt sedan dess.

}

\small{
\textbf{Luciavaka}
\label{3854b9ddd1f195d1344f77747760aac9}
 Traditionen att fira lucia \textsc{(s.~\pageref{3ba430337eb30f5fd7569451b5dfdf32})} härstammar lite blandat från folktro och kristendom. De förkristna elementen är så klart ballare då man trodde att djuren kunde tala under lucianatten och att övernaturliga makter var i rörelse eftersom det är det mörkaste dygnet på hela året. För att inte råka ut för motpåven i Gränna \textsc{(se Motpåven i Gränna s.~\pageref{35d44c815567a4c3ba5fdba1ab1cec21})}, Anton Lavey \textsc{(s.~\pageref{869cf213daf268853824c26db9960ab7})}, Jubal \textsc{(s.~\pageref{5f7b7046fcb07abbe2448d98106037a0})} eller någon annan kristen mörkerman denna natt satt människorna uppe hela natten och vakade vid fönstret. Liksom alla andra förkristna högtider inkluderar även luciavakan att deltagarna dricker stora mängder brännvin \textsc{(s.~\pageref{ff49ececa32cff978496a39635496f46})} sittandes i särk \textsc{(s.~\pageref{7a522dc7e11bd1136642b3452855c1d6})}. Genom sin popularitet har luciavakan även fött andra jultraditioner såsom häxblandning, bockbränning och skinnsbergslucia \textsc{(s.~\pageref{1195cf539556d5d28ac8418c613f2676})}.

}

\small{
\textbf{Luffarmacka}
\label{0f7c04cb797f372ed3f219808e0af5b3}
 Hamburgare \textit{sans} kött.

}

\small{
\textbf{Luffarschack}
\label{f6d93abef5b64200d5b82fa5752de6b3}
 En är kryss, en är ringar. Reglerna varierar.

 HEAD2: En historia om luffarshack
 När min \textsc{(se Användare: HratvinnFlygur s.~\pageref{26c5d96dca8dfce84752fa1d4095fdb0})} farsa var skolungdom spelade de alltid luffarshack på roliga timmen. En gång så spelade han mot sin mattelärare, Leif Hedström, och vann. Gubbjäveln surnade och fortsatte att utmana farsan varje roliga timme fram till examen för att kunna säga \quotetext{Jaha du Botte \textsc{(s.~\pageref{59d505d6448b03a0331e6fc09a69d3b9})}, vad står det? 29-1?}

}

\small{
\textbf{Luffarskål}
\label{20e976e9aebea02afe6cb514d7cf302e}
 Att äta i luffarskål, eller irländskt hovporslin som det också kallas, är att knipa ihop låren medan man sitter och sedan hälla ner maten i knät och äta direkt ur skrevet.

}

\small{
\textbf{Luftgitarr}
\label{0e2415e86edc316f5338964c6ef145b5}
 kallas den uråldriga blandning av teater, mim och dans som går ut på att medelst kroppen imitera solot i \textit{November rain} eller annat gitarrmättat musikstycke.

 Ursprungligen handlade luftgitarr bara om framförande för utövarens egen höga njutning. Men sedan den borgerliga revolutionen trätt in och raderat feodalsamhället förändrades även luftgitarrens roll och blev nu en individens kamp med alla mot alla. Som i så många andra tävlingar där det visuella styr koras vinnaren i luftgitarrtävlingar av en jury. Vinner gör den som mest får det att se ut som att hon/han faktiskt håller i en gura och dessutom inte spelar fel. I juryn till luftgitarr-SM 1991 ingick bland annat Jan Guillou \textsc{(s.~\pageref{63f2c8aba9686bc92efeb7eb21e35156})} och Micke \quotetext{Svullo} Dubois, den senare i egenskap av flink luftbassist tillika komiskt geni. Vann gjorde det året Lena PH:s ex-snubbe Martin Björk. 1984 vann Kalle Moraeus.

}

\small{
\textbf{Luktagott}
\label{f9613f1654fe61d6a5c0787c85daeeaf}
 är sånt som storfräsare \textsc{(s.~\pageref{4db17005692cd83e3e946a1311b81ed0})} har på sig för att till och med lukta märkvärdigt. Som alla vet är de bästa dofterna egentligen de från bränd linolja och Kir \textsc{(s.~\pageref{002e1a6e54da86cabc77fbb474c2df49})}, men vitsen med luktagott är inte att det ska lukta bra utan dyrt och borgerligt. Det mesta luktagottet görs därför på exklusiva och dyra saker såsom flodkaninens \textsc{(se flodkanin s.~\pageref{5b2fd3512fae865f843dfe95c778fa07})} hypofys, saffran, kopparrör och jordgubbar. För att ytterligare öka luktagottets status ges essenserna namn som leder in brukarens fantasi på mytologiska spår såsom \quotetext{\textit{daggvåt ängsmark}}, \quotetext{\textit{änglafjärt}} eller \quotetext{\textit{kungen av Danmark \textsc{(s.~\pageref{5331d7fd27772396f412a5b6d19bad44})}}}. Naturligtvis är det mest kälkborgare \textsc{(s.~\pageref{0f34b469a48952e93688861083ace75a})} som ägnar sig åt detta, i ett desperat försök att få ligga mer. Vilket de förmodligen får eftersom andra kälkborgare också är ena ytliga jävlar.

}

\small{
\textbf{Luleå}
\label{3cefb5ac35187749592f1ebb25472b99}
 Lule (Luleå i  Storsvensk \textsc{(se storswänsk s.~\pageref{716f41dcabef6599bcf08334a8a6ae27})} stavning) är en  stad i Norrbotten \textsc{(s.~\pageref{0e8c003b75982032cde152609ee94154})} som bebos av en hel del tuffa typer av alla kön. I Lule kan folk av olika trosinriktningar såsom veganism \textsc{(se veganer  s.~\pageref{2a12d5d6ae91d2f4f7d9af3cef58e75c})}, lutheranism och frifräsare  umgås under samma tak utan att handgemäng uppstår.
 Övriga landet har ofta svårt att hantera närheten av en lulebo då de ofta är ett gladlynt och skämtsamt folk som gärna klär av sig.
 Lulebon vägrar att kompromissa med sig själv och säger sin åsikt med rak rygg och högt hållet huvud. Således följer han/hon sitt Lule Hockey med en självplågares fulla entusiasm. Poporkestern Toto har spelat inte bara en utan två gånger i Lule. Medelpersonen från Lule har cirka 105 högskolepoäng utfärdade av Umeå Universitet i humaniora, beteendevetenskap, socialt arbete och/eller från en lärarutbildning.

}

\small{
\textbf{Lunchklubben}
\label{8759e8d10b4cf0f3916334daae2ceb6b}
 är ett socialt fenomen på Umeå Universitet som består av att vissa studenter äter lunch på samma ställe varje dag kl 12.00, och den platsen är Humlan \textsc{(s.~\pageref{113017afa0a9549cccc931300ba2edb3})}. Klubben är öppen för alla och inga krav ställs på medlemmarna, förutom att de helst ska vara socialt normalbegåvade.

 HEAD2: Populära samtalsämnen
 \begin{itemize}
 \item Samtida händelser, gärna med stöd ur en dagstidning
 \item Storlek och innehåll i matlådor
 \item Kvalitét på kebab-, steglibab- \textsc{(se steglibab s.~\pageref{1031ea42e093eb6ec5a2c30b2430e4eb})} och/eller falafelrullar \textsc{(se falafel s.~\pageref{b2d6ec45472467c836f253bd170182c7})}
 \item Vädret \textsc{(s.~\pageref{495336a8161e7cecbcc37f2f9a7745f3})}
 \end{itemize}

}

\small{
\textbf{Lundgren}
\label{6c15999d6cf4d190d74fe573657c92b9}
 är en av de äldsta inventarierna i stadsbilden av centrala Norberg. Han har sedan inlandsisen smälte ansvarat för att kundvagnarna utanför Konsum och Ica kommer tillbaka på sin plats. Om arbetet utförs på konsultbasis eller ideellt är oklart. Det är också oklart om någon någonsin faktiskt bett honom att göra detta. Lundgren går alltid oklanderligt klädd i kavaj och kepsar med olika företagslogotyper \textsc{(s.~\pageref{6a414633590fd4cd6d6ac64798d14c14})} och skulle med lätthet kunna ta plats i vilken historia som helst om Kapten Stofil. I väntan på att nya kundvagnar ska köras tillbaka händer det att han unnar sig ett pipstopp och går en sväng med händerna på ryggen.

 Enligt samstämmiga uppgifter från en källa var Lundgren mods på 1960-talet.

}

\small{
\textbf{Lundin Petroleum}
\label{69b17fd76232d179bca83c8eadcc12ed}
 Lundin Petrolium är en en gren av FNs Barnfond UNICEF och har alls inget med hänsynslös exploatering och etnisk rensning att göra. Organisation ägnar sig mycket åt att pumpa upp olja för att av den kunna tillverka mat åt fattiga barn i Afrika. Vår egen Carl Bildt är en av organisationens mest envetna tillskyndare och arbetar oförtröttligt i anletets svett för organisationens bästa.

}

\small{
\textbf{Lurkuk}
\label{27d9dc2d37102d0877e7034edd1c9c87}
 En lurkuk är en man, påfallande ofta heterosexuell och medelålders, som lovar kvinnor i sin närhet guld och gröna skogar mellan lakanen, men i slutändan inte får upp den. Detta beror ofta på fylla/trötthet, nervositet eller att åldern helt enkelt tagit ut sin rätt på svällkropparna. Det finns olika strategier för att kompensera detta, varav de vanligaste är lite tafatt oralsex eller att somna. Bland våra kändare lurkuk återfinns framförallt Ulf Lundell och även Max Weber \textsc{(s.~\pageref{2fb708fe6352f97cd7e4fe5bab54a88f})}.

 HEAD2: Strategier för att upptäcka lurkukar
 En ganska felsäker tumregel är att; ju mer karlfan lovar och bedyrar sin virilitet trots eventuell fylla, desto troligare att det mesta han kommer att åstadkomma är att dregla dig lite i underlivet innan han somnar.

}

\small{
\textbf{Lurmus}
\label{23f18296e8df765844117b713fb4613f}
 \textit{pl. lurmöss} är en person som skapar en förhoppning att en ska få ligga med denna för att i sista sekunden avvisa en.
 Snälla killar som aldrig får ligga \textsc{(s.~\pageref{630d0607c17e587ef244461bbafe9b4b})} går ofta på lurmustricket.

 HEAD2: Kända lurmöss
 Skogsrået \textsc{(s.~\pageref{d370fd7de29a36c04c0c0c43ac488f11})}
 Näcken
 Rörmokare
 Poliser

 Se även: lurkuk \textsc{(s.~\pageref{27d9dc2d37102d0877e7034edd1c9c87})}

}

\small{
\textbf{Luís Figo}
\label{b0e0e67d9390871802e90a12876cdcff}
 , eller Figo som han kallas, är en portugisisk fotbollsspelare som var väldigt duktig på det han gjorde. Han tröttnade på Portugal och började spela i den katalanska klubben FC Barcelona där han utvecklades till en riktig stjärnspelare i mittfältet. Spaniens näst största stad blev dock för liten för Figos svällande ego och han nappade på ett erbjudande att börja spela i Real Madrid. För att konceptualisera precis hur stora rivaler dessa klubbar är kan ni tänka er att det är troligare att en sparkad brittisk kolgruvearbetare ger chefen en kram och säger \quotetext{You did your best!} än att två supportrar till de olika klubbarna sätter sig ner och tar en bärs tillsammans. Frågan som spanska sportjournalister ställde var om ens herren vår Gud \textsc{(s.~\pageref{91e49146121c992aab11a19c77e26cf0})} själv skulle vara förmögen att förlåta Figo. Barcelonas fans visade vad de tyckte om beslutet genom att hiva ett grishuvud på Figo när denne lade en hörna under sin första match med Real mot Barca.

}

\small{
\textbf{Lycksele}
\label{b6298c88c17c25794096f7a6d2a7baff}
 är ett sött och lätt efterblivet samhälle i västerbottens inland. Som så mycket annat ligger det vid vatten. Varje vinter gör man en tiotals meter hög och ständigt ejakulerande kommunal snökuk (officiellt kallad ispelare) ute vid stranden. Vilket ortens feminister ställer sig lätt frågande till. Det lokala patriarkatet tycker att feministerna har snuskig fantasi.

 [[File:feminist.jpg\textbar90 px\textbarSkeptisk feminist]]  [[File:snökuk.png\textbar180 px\textbarSnökuk]]

}

\small{
\textbf{Lykke Li}
\label{a58871a224762ba5c2ae8fd07afdef19}
 är barn till ZilverZurfarn från Dag Vag och Kärsti Stiege från Tant Strul.

}

\small{
\textbf{Lyteskomik}
\label{fb086a53905285406abfb800f0a82bf9}
 är en uråldrig del av vårt kulturarv men på grund av postmodernism och annat fanstyg får man att inte skratta åt Anton Abele \textsc{(s.~\pageref{0906f6e1d290c547e1fb93c6ff6a0b44})} eller tjocka dvärgar på små motorcyklar.

}

\small{
\textbf{Lägg dig inte i}
\label{c2852775cbc92085b0a61d5912eba396}
 Sa han som sket i sängen.

}

\small{
\textbf{Lägre Sorbiska}
\label{303f755b0e855272f58060b6bf78fe94}
 är ett slaviskt språk som talas av typ 14 000 människor. Det är populärt i den tyska staden Cottbus.

}

\small{
\textbf{Läppar som prinskorv}
\label{d01958f01259206b3a8724070d49aa47}
 (uttalas med fransk brytning) är ett skinhead från det, i korvsammanhang, anrika landet Tyskland \textsc{(s.~\pageref{b1b58da783b6d5fa090f3015f1889869})}. Närmare bestämt bor han i Hamburg där han lägger mycket tid på att följa det lokala fotbollslaget \textsc{(se fotboll s.~\pageref{961bd74d34872ff94a4df3a16119096e})} \textit{FC St. Pauli} och på att kolla in kassa tyska punkband. En genomsnittlig dag dricker Läppar som prinskorv ungefär 5 bärs av det lokala märket Astra och röker ett paket cigg. Har han inget bättre för sig kan det också hända att han drar nån skabbig gaddning på sin askgrå hud. Precis som man föreställer sig är han ganska trind om buken. Det mest iögonfallande med Läppar som prinskorv är dock hans fylliga läppar som alltid är glansiga och väldigt stora. Man blir som lite nyfiken. Varför är dom alltid lika glansiga som flottiga prinskorvar? Men det vågar man inte fråga för Läppar som prinskorv är trots allt skinhead och dom brukar sällan uppskatta sådana frågor oavsett om dom är sharp eller inte.

}

\small{
\textbf{Lärare}
\label{b73161d722478b506765d029253c4519}
 är ett ädelt yrke som länge hade hög status i och med historiska föregångare som Thomas av Aquino \textsc{(se chapeau de paysan s.~\pageref{27aa75146d9ab723d1423168a2539d5d})} och varma mediala representationer som Ingemar Bergmans gullgubbe Caligula i filmen \textit{Hets}. Med åren har lärarkårens status chanserat. Rent historiskt befinner sig nu lärare vid en kritisk nollpunkt i ryktbarhet. Nedan följer en historisk redogörelse för hur det västerländska läraryrket en gang var, och hur det sedermera kommit att bli.

 HEAD3: Antiken
 De gamla grekerna \textsc{(s.~\pageref{4a5fb3d6ce79b5ff43b33f8f7d843672})} höll lärarna högt. Sokrates föreläsningar om ditten och datten gjorde honom till en togaklädd rockstjärna. Platon, en annan farbror som lärde, var så begeistrad av Sokrates att han utformade nästan alla sina kommande böcker i dialogform, där han och den då avlidne Sokrates hade påhittade samtal om gamla idéer. Det var även så fenomenet \quotetext{coverband} uppstod.

 HEAD3: Medeltiden
 På medeltiden bodde de flesta lärare i kloster. Kloster på den tiden var mäktiga institutioner där alla var feta som broder Tuck och söp hela tiden. Runt omkring klostren brann ett pestdrabbat Europa. Att lära sig om det ingick dock inte i den högst begränsade läroplanen, som mest avhandlade korvstoppning och att ordagrant kopiera enorma textsjok för hand.

 HEAD3: Renässansen
 Under renässansen flyttade lärarna ut från sina kloster och tillbaka ut i samhället. Eller en del av samhället i alla fall. De som hade råd med lärare på den tiden var italienska merkantila högdjur och tyska hertigar som fick läsa \textit{Fursten} av Machiavelli och lära sig räkna medelst abakus. Rika, inte fattiga, med andra ord.

 HEAD3: Upplysningen \& Romantiken
 Här började det ta fart som satan för lärarna. Upplysningen innebar att \quotetext{vetenskap} var inne och att samhället förändrades en del (fler fick tillgång till skolor). Därför blev läraren det hetaste sedan nån spillde senap på en varmkorv för första gången. Alla ville bli lärare som gick runt och berättade saker för folk. Rika som fattiga, alla ville ha en lärare!

 Romantiken var en idéströmning som löpte parallellt med upplysningen och hade lite andra ideal, typ att man var andlig och gillade naturen som estetiskt föremål - inte som ett monster som skulle betvingas med eld och järn. Men de gillade också lärare, i den känslosamma filosofiska formen, ungefär som Robin Williams i \textit{Döda poeters sällskap}.

 HEAD3: Moderniteten
 Allt skulle industrialiseras och mätas och så. Läraren fortsatte rocka loss i samhället. Visst, nu var lärarna fler och behövde undervisa barn från lägre samhällsskikt, men deras rockstjärnestatus levde vidare. Det här skulle kunna ses som läraryrkets guldålder. Många kunde bli det, ingen ifrågasatte en och lönen var rätt bra. Man fick också slå barnen.

 HEAD3: 1970 - 2002
 En inflation av lärare orsakade en urholkning av läraryrkets guldiga aura. Dessutom hade tjejer börjat bli lärare, något som i regel pajar ett yrkes status. På 90-talet behövdes inte längre någon riktig utbildning för att bli lärare, bara en extrem hängivelse till en hobby. Typ om man hade en LUF-pin kunde man titulera sig mattelärare och om man hade håriga handflator och gillade pörr \textsc{(s.~\pageref{5faa435e2f0af7617816f0cade262581})} kom en heltidsanställning som biologilärare som ett brev på posten. Det var även här elevernas respekt för lärarna försvann, då folkpartister och porrsamlare sällan har någon längre utbildning i pedagogik.

 HEAD3: Nu och vidare
 Vad framtiden bär i sitt sköte för läraryrket är svårt att sia i. I Sverige finns nu ett krav på lärarlegitimation, vilket innebär att 90-talets entusiaster pressas ut från arbetsmarknaden. Detta skulle kunna resultera i att lärarkåren blir mer professionell och håller högre standard. Men i och med att antagningspoängen för lärarutbildningen är lägre än dito för dikesgrävare, kommer antagligen ingen större förändring att ske ändå.

}

\small{
\textbf{Läsesalsflört}
\label{8a9d9225bb219791bfea63b3c3eab47e}
 En läsesalsflört uppstår när en person fattar tycke för en annan person i en läsesal, oftast på ett universitet \textsc{(s.~\pageref{11dfc744fa396b961a6cc9cf89c4eaea})}, och bestämmer sig för att göra slag i saken. Slaget i saken består de flesta gånger i att överlämna en lapp på sitt amorösa intresses läsebänk, där man skriver typ: \quotetext{du är söt, vill du ta en fika någon dag? :)} och sedan nedtecknar sitt telefonnummer. Nästan ingen av dessa lappflörtar (obs! ej menat som rasistiskt tillmäle mot Sveriges ursprungsbefolkning) resulterar i mer än en eftermiddags höjt självförtroende hos mottagaren och minst en veckas dödsångest hos avsändaren.

}

\small{
\textbf{Lätt misshandel}
\label{da5052972c3a081d8e951c69da453722}
 Hästbett, purple nurple, knuffar i kombination med kränkande skällsord yttrade med avsikt att såra, öppen handflata i ansiktet, baksidan av handen i ansiktet, tjuvnyp på kärlekshandtagen, hajkbox, solar plex-slag, att fällas medelst krokben \textsc{(s.~\pageref{85c111491df5c9adeb8d907f3203238f})} eller tillhygge, luggning, spark i arslet, box på axeln eller i magen (inte för hårt),  kasta ett litet föremål på nåns pung \textsc{(se Garden can  s.~\pageref{f2d91d42b4daffbbed6659b449fcb156})}, skjuta någon med soft air-gun eller häftapparat (på långt avstånd) och att skrika någon ashögt i örat så det gör ont är alla exempel på lätt misshandel. Alla dessa former av lätt misshandel förekommer allt som oftast i fylleceller (förslagsvis i Norrtälje) \textsc{(se Norrtälje s.~\pageref{7527f7dad9445013a559dc7e2a91f3b3})} och på defensiva linjen under en handbollsmatch.

 \textbar

}

\small{
\textbf{Lättlagade festrätter från Anderssons skafferi}
\label{9ce1ec72a4e839231c4a61599642efce}
 är en kokbok med enklare rätter som kan serveras såväl till vardag som till fest. Den gemensamma nämnaren är att de alla komponerats av den Malå-ättade \textsc{(se Malå s.~\pageref{41da4620e87888eaaeafcb3004a8d177})} hippiekocken Ramses Andersson med målet att skapa en spirituell dimension av ätande. Alla rätter rekommenderas att serveras till tonerna av Enya och/eller panflöjt \textsc{(s.~\pageref{ce107b52f922e556b394fa5303dc6b0f})}. Nissepedia \textsc{(s.~\pageref{62400dadecd90cb5cd39062abe5a3e4a})} publicerar här ett urval av recepten men för en komplett sammanställning får ni köpa boken. Alla rätter är veganska \textsc{(se vegan s.~\pageref{792fec82e3a0dcea1817fd9ebfaf1533})}, så när som på några som innehåller ägg \textsc{(s.~\pageref{128a5feb8e12d0aa622e0298a8332980})}.
 HEAD2: Ägg i babaganoush
 1. Mosa en aubergine till oigenkännlighet.
 2. Tryck ner fem skalade och hårdkokta ägg så att ungefär halva kroppen döljs i sörjan.
 \textit{\textbf{Tips: } Har du några sverigeflaggor från en gammal prinsesstårta  sparade kan du sätta dessa i äggen så blir det ännu trevligare. }

 HEAD2: Rutten frukt
 1. Köp ett nät klementiner med kort datum.
 2. Låt stå i solen tills skalen bytt färg minst två gånger.
 \textit{\textbf{Tips: } Bär gärna kläder från Ed Hardy \textsc{(s.~\pageref{e3b7de0302ffdd8ec9ad544dba1d5b3d})} medan du äter ifall magen får svårt att processa alla smakupplevelser.}

 HEAD2: Irish hotdog
 1. Borra ut ett hål i en rova.
 2. Placera en halv purjolök eller annat korvsurrogat i hålet.
 \textit{\textbf{Tips: } Önskas dressing så dra en snorloska i hålet först. }

 HEAD2: Punchrullar från Rusta
 1. Ta dig till Rusta
 2. Langa upp en guldtia \textsc{(s.~\pageref{e7292d5ba58672ce7f6fc3c0b646ab63})} i kassan
 \textit{\textbf{Tips: } Servera gärna tillsammans med sandkakor \textsc{(s.~\pageref{fe0b18b5cc74dcf22faf367e45df6e7d})}. }

 HEAD2: Ägg á la Slayer
 1. Koka några ägg riktigt hårda.
 2. Skiva upp och lägg ut i ett pentagrammönster på en Slayerskiva.
 \textit{\textbf{Tips: } Lyssna på SLAYER!!!! }

}

\small{
\textbf{Lättnad}
\label{c591923999933bd79701bef0f2af2dc0}
 är en känsla som ofta infinner sig efter en kortare eller längre period av oro. Lättnadskänslan upplevs normalt som positiv eller mycket positiv, men medför ibland förlorad kontroll av den anala kroppsöppningen, så att en liten fis i vissa extrema fall kan slippa ut. Enligt forskare ska detta vara förklaringen till att extremt höga gaskoncentrationer uppmättes i skandinaviska storstadsregioner den dag då Ålandskrisen \textsc{(s.~\pageref{967c6b3cd72e6de161ca9e911779795a})} fick ett lyckligt slut.

 HEAD2: Historiska exempel på situationer då man upplevt lättnad
 \begin{itemize}
 \item Lasermannen grips
 \item Maud Olofsson \textsc{(s.~\pageref{eb913a2e9be929654908a05017401bd6})} avgår
 \item Stig-Helmer får tillbaka sitt förlorade bagage i \textit{Sällskapsresan \textsc{(s.~\pageref{1023ca20cc8ad5b3f0233d023ad01bf5})}}
 \item Nikolaj Valujev \textsc{(s.~\pageref{85363e70d015f18efa110613d79baf4a})} slutar boxa folk i huvudet och börjar istället leta efter snömannen.
 \item Det visar sig i den första boken om Spöket Laban att Laban är ett snällt spöke.
 \end{itemize}

}

\small{
\textbf{Lättöl}
\label{7bc0cdfbdabfa9d70195842ad3e460f2}
 är en form av bärs \textsc{(se Ha bärs s.~\pageref{a74b297c15834437ac2e49095492133c})}, vilken till skillnad från resterande släktträdets magnifika utlöpare, maximalt innehåller löjeväckande 2.25 \% alkohol. Lättöl inmundigas oftast av medelålders gubbar som gift sig, skaffat tre barn (Josef, Leo och Vladimir) - men förbjudits av sin partner från att dricka starköl varje dag. Även om det vore fullt möjligt att hinka läskeblask, vatten eller svagdricka till fiskgratängen på tisdagarna, klamrar sig den medelålders gubben fast vid den enda kvarvarande spillran av sin, i forna dagar skimrande, maskulinitet - det faktum att han åtminstone \textit{älskar smaken} av öl. Ibland träffas medelålders gubbar på ett slags minikonvent, så kallade \quotetext{konferenser}, på vilka det är legio att dricka lättöl till lunchen och på kvällen köpa prostituerade.

 I Danmark \textsc{(s.~\pageref{5331d7fd27772396f412a5b6d19bad44})} har lättöl varit olagligt sedan 1920-talet. I Tyskland \textsc{(s.~\pageref{b1b58da783b6d5fa090f3015f1889869})} har det aldrig funnits en enda lättöl.

}

\small{
\textbf{Lådaktivism}
\label{d9856ea12ca9b158867cd209bf4df7f9}
 Vid flera tillfällen i Sveriges historia har olika grupperingar och personer testat varianter av att placera en person i en låda som medel för att nå sina mål. Folke Pudas använde sin pudaslåda \textsc{(s.~\pageref{6a56958e2057dd500650e2be8049e033})} för att processa mot länsstyrelsen \textsc{(s.~\pageref{0ae3fdeda52fe82800b04c624330139c})} och vid två tillfällen har terrorister gjort försök att kidnappa rika eller mäktiga personer för att sedan placera dom i liknande lådor, i dom fallen talar man om lådterrorism snarare än om lådaktivism.

 Det ena fallet är när Fabian Bengtsson kidnappas och hölls instängd i en låda i 17 dygn, i ett försök att kräva Fabians familj på fem miljoner euro och fem miljoner kronor, som misslyckades då Fabian släpptes utan att lönesumman betalades ut. Ett annat fall är Norbert Kröcher som byggde en låda tänkt för att placera Anna-Greta Leijon i efter att hon kidnappats, en kupp som aldrig genomfördes.

}

\small{
\textbf{Långa namn}
\label{a1b555b977d20f62cd2eaa01091e08fb}
 Pippilotta Viktualia Rullgardina Krusmynta Efraimsdotter Långstrump, är långt. Hon är påhittad och har rött hår.
 Diego María de la Concepción Juan Nepomuceno Estanislao de la Rivera y Barrientos Acosta y Rodríguez, är också ett fruktansvärt långt namn. Han var mexikansk konstnär och kallades understundom Diego Rivera.

}

\small{
\textbf{Långfredagen}
\label{d497ff3e8ae0d52f8b75f4c698d7d293}
 är en del av påsken \textsc{(se påsk s.~\pageref{f8f0dd13b69a5c8ce56498e750551d3e})} och är den dagen då Jesus \textsc{(s.~\pageref{110d46fcd978c24f306cd7fa23464d73})} Kristus hängde på korset och var nära på att dö av de pinor som Pontius Pilates utsatte honom för. För att fira detta låter folk sig själva och sina barn ha fruktansvärt tråkigt.
 HEAD2: Tips på aktiviteter
 \begin{itemize}
 \item Se Werner Herzogs \textit{Fitzgeraldo}
 \item Spela svälta räv \textsc{(s.~\pageref{f5ba1e0ca45e2d553c6282cb290878dd})} och fia med knuff.
 \item Lyssna med ett halvt öra på SRs Trädgårdsdags eller \textit{Vapen \& ammunition} av \quotetext{sveriges \textsc{(se sverige s.~\pageref{b1999637949ed135b2ca03f3a38460cc})} största rockband}, Kent \textsc{(s.~\pageref{564f10260067a9b0c8d8e206ecdb49c6})}.
 \item Skjut omkring farmor i \quotetext{finrummet} i hennes rullstol
 \item Dö lite inuti.
 \end{itemize}

}

\small{
\textbf{Lårkaka}
\label{335b43d2df58a135dbc44ed782cbae24}
 är en kaka som tillreds av lår och med fördel också serveras på ett lår. Då kakan har ett högt energivärde är det vanligt att den förtärs i samband med fysik aktivitet så som idrott.

}

\small{
\textbf{Lögnologi}
\label{41f0f3903e3f302437d4c1ec1aa1df33}
 är läran om allt bedrägligt här i världen. Alla större universitet med självaktning bedriver både forskning och undervisning i lögnologi. Det kan dock vara svårt att avsiktligt hitta till lögnologiska institutionen och de kurser den ger, eftersom den i enlighet med gammal tradition sätter en ära i att kamouflera sig och verka vara något helt annat. Går du en kurs i nationalekonomi eller meteorologi? Var inte så säker. För att inte tala om kvantfysik, den internationella lögnologkårens mest framgångsrika practical joke någonsin. De (mycket få) som lyckas genomskåda kvantfysiken, och dessutom kan bevisa det, tystas snabbt genom att bjudas in till lögnologins innersta cirkel, där festerna är RIKTIGT roliga.

 Tre av lögnologins största forskningsgrenar är statistik, estetik och kulturhistoria. Intressanta forskningsrön är till exempel:

 Att de ideala proportionerna mellan olika sanningsgrader i mellanmänsklig kommunikation är: rak sanning, 22 \%; redigerad sanning, 37 \%; småljug, 39 \%, och fullljug, 0,2 \%. (Statistik är, trots att det är en välkänd form av lögn, inte medtaget i underlaget eftersom det skulle leda till skumma feedbackloopar.)

 Att smicker är vackrare än både mobbing och lösmustasch.

 Att gud ljuger för Adam redan i andra kapitlet i första mosebok: ”Du får äta av alla träd i trädgården utom av trädet som ger kunskap om gott och ont. Den dag du äter av det trädet skall du dö.” Inte dör Adam inte, åtminstone inte just då. Möjligen skulle man kunna hävda att gud inte ljög, utan bara var felinformerad. Men då faller ju en del av konceptet.

 Lögnologins filosofiska gren är så gott som omöjlig att skilja från annan filosofi, men man har åtminstone producerat frasen \quotetext{Jag ljuger, alltså finns jag inte.} Det finns en minoritet bland forskarna som hävdar att hela filosofikonceptet, snarare än kvantfysiken, är lögnologernas mest framgångsrika practical joke genom tiderna.

}

\small{
\textbf{Löneförmån}
\label{1e6dac24312efb97f6006f4c3915adee}
 Allt du kan bära är löneförmån.

}

\small{
\textbf{Lördag}
\label{8d203c09d6ebbc3a0d797e14178798a0}
 är enligt många den bästa dagen, för då är man ledig och har dessutom ännu en hel ledig dag att se fram emot (dvs. söndagen) \textsc{(se söndag s.~\pageref{85b2e5c3758394a24221d1abac79191a})}. Lördagen är enligt Christian Information Service Homepage \textsc{(s.~\pageref{41f6a92690af566fe26cfeb327f82eb5})} den egentliga sabbatsdagen. Därför har man i Umeå med omnejd delat ut flygblad i vilka vikten av att fira sabbaten på lördagen framgår med all önskvärd tydlighet. På lördagen får man äta lördagsgodis och spela på tipset.

}

\small{
\textbf{Lövestad var ej järnvägsknut}
\label{6e03a09b023b1a25b6f683e4ee1b1e98}
 Lövestad. YA publicerade häromdagen ett vykort från Lövestad från förra seklets början. Vi påstod då att Lövestad var en handelsmetropol och järnvägsknut. Den påpasslige läsaren Håkan Olsson, Torparbron, påpekar att Lövestad aldrig var \textsc{(s.~\pageref{b2145aac704ce76dbe1ac7adac535b23})} någon järnvägsknut - samhället hade bara en linje: Ystad-Eslöv. Detta räckte dock för att skapa ett blomstrande stationssamhälle.

 -Ystads Allehanda 22/1 2010

}

\small{
\textbf{Mackshopping}
\label{f6cbaa37785643222fb462b32a199d29}
 Vanligt vid haschrökeri och sådana dagar då fulla människor inte i förväg planerat att bli fulla, vilka också råkar vara de enda två tillfällen då man fortfarande drar sticka för att bestämma vem som ska tvingas göra något jobbigt. Väl på macken köper den fulle hutlösa mängder folköl för jättemycket pengar. Den stenade \textsc{(se stenad s.~\pageref{dec4a3a91f0f2bf8dcf033a8cfeaa554})} har ett helt annat sätt att ta sig an situationen: hen tittar på allt i butiken och tar ett individuellt beslut för varje vara, dvs köpa eller inte köpa. Normalt slutar det med en jättepåse med plockgodis, en Pepsi Max samt en Nöt-Crème att suga i sig på vägen hem, men det är inte ovanligt att ett gäng kabinhakar, 20 m metrev samt en presentinslagen domkraft åker med. För folk som är nyktra och över fyrtiofem år fungerar bensinmacken på samma vis som skivaffärer fungerar för oss andra: Här finns ett ställ fullt med 20 olika compact discs med Jill Jonsson och Tomas Ledin \textsc{(se Ledingreppet s.~\pageref{528126abac2b649bb4fdf7bd2764726f})} som det bara är att välja och vraka från.

}

\small{
\textbf{Mad by proxy}
\label{106ac65bc2652f248fd8d12c85233ad2}
 Umgås man med genuint glada människor (alltså inte telefonförsäljare) blir man glad. Umgås man med ledsna människor blir man lätt lite dyster. Umgås man med tokar vet ingen vad som kan hända.

}

\small{
\textbf{Mads Mikkelsen}
\label{1658b9125bfc1f75901858f0e8344337}
 (född 22 november 1965) är Danmarks \textsc{(se Danmark s.~\pageref{5331d7fd27772396f412a5b6d19bad44})} största skådis. Han slog igenom 1996 med filmen \textit{Pusher}, som enkelt kan beskrivas som en dansk motsvarighet till \textit{Sökarna}. Därifrån var vägen till Hollywood spikrak och han har bland annat spelat in \textit{Nattens Engel \textsc{(s.~\pageref{b502941eb96ddba32dd5e652b647f350})}} och en reklamfilm för reseföretaget Ving sedan dess. Hans fans kallas \quotetext{mikklare} och på Mads-konvent är det vanligt att dessa träffas nakna på en strand och super hejdlöst tills tidvattnet spolar bort dem. Han är dubbad pilsnerdræng \textsc{(se danska hedersbetygelser s.~\pageref{799941a6e98a1446da72ff5483c6503d})} vid det danska hovet, som är den näst finaste utmärkelse en privatperson kan få i landet.

}

\small{
\textbf{Maginotlinjen}
\label{a745c13371be11c7bf2e84f9d6f4dd51}
 En historisk, militär försvarslinje längs Franska gränsen mot Tyskland \textsc{(s.~\pageref{b1b58da783b6d5fa090f3015f1889869})}. Under första världskriget stod denna linje som försvar mot ett tyskt anfall från öster men någon osedvanligt förslagen tysk general kom på genidraget Schlieffenplanen som gick ut på att gå genom Belgien \textsc{(s.~\pageref{f79ffe9e826a19f9f6a446c90e21c4e3})} och således gå runt denna försvarslinje. Belgarna bjöd inget motstånd och vips var Frankrike ockuperat. Nu blev inte det här sista gången som Frankrike behövde försvara sig mot den tyska mustigheten \textsc{(s.~\pageref{682ccd5fdc3aff0c97e8845c3d6b6ca8})}, utan två decennier senare fick de en andra chans. Tyskarna rustade på nytt för krig och Fransmännen skulle försvara sig mot ännu en ockupation. En fransk general lär ha sagt \quotetext{Ok, de gick runt Maginotlinjen förra gången, men jag tycker vi testar, varför skulle de gå genom Beligen igen?}. Tyskarna lär ha skrattat under hela marschen genom Belgien ända tills stöveltrampen  åter ljöd på Champs-Ellysés.

}

\small{
\textbf{Main Page}
\label{0745d0b57dcc4000e96812839a349e18}


}

\small{
\textbf{Maine Paige}
\label{90b94cdc06b03dd185b6dade1a721d60}
 Elaines dyslektiske lillebror.

}

\small{
\textbf{Maj-Björn}
\label{48c9303e748c4c60b486f3ae20632b66}
 är ett original som främst rör sig i kretsarna kring Juneporten, Jönköping. Maj-Björn har tidigare suttit på mentalsjukhus och kan uppfattas som en \quotetext{kuf}. Han ses ofta med långt rosa hår (ofta med utväxt), för kort tröja (där magen tittar fram) samt färdandes på cykel. Maj-Björn är på intet sätt farlig för allmänheten men kan dra blickar till sig samt skapa debatt bland åskådarna.

}

\small{
\textbf{Majstång}
\label{a8f36927b7ad0430235d48561aa57200}
 1. Vanligen redskap för att på avstånd plocka upp majs.

 2. Även mycket fånig benämning av midsommarstång då midsommar \textsc{(s.~\pageref{e7055fcb068b695ddcfb7e34bb4866a7})} inte på något vis infinner sig i Maj. Det som för det nya seklets generationer verkar fånigt förklaras med att ordet \quotetext{maja} (verb) härom förra seklet användes för att beskriva att man dekorerade dörrar och stänger med något vackert ur växtriket, t.ex. kransar av gröna löv.[http://www.svenskaakademien.se/web/Ordlista.aspx]

}

\small{
\textbf{Majuskler}
\label{cc540e015f5457a65e5c31c0cb947227}
 Stora bokstäver. Versaler. Även populärt kallat majuskeljävlar.
 Används bland annat i meningsbegynnelser, i början av egennamn och när man vill förmedla textbaserade skrik.

 Exempel 1:
 JAG ÄLSKAR DIG ALMA SKA VI LEKA RYSKA POSTEN PÅ LÖRDAG

 Exempel 2:
 HEJ LO VAD HAR DU PÅ DIG

 Exempel 3:
 Micke och Molle anses av många vara den bästa filmen som någonsin har gjorts.

}

\small{
\textbf{Makadam}
\label{d358df22f11c57bbd1d5718b9b474b26}
 har fått sitt namn efter den skotska stenmagnaten Ian MacAdam och är egentligen mer en metod än en produkt. Då MacAdam upptäckte att man genom att lägga en sten i ett uvbo \textsc{(se uv s.~\pageref{45210da832f9626829457a65e9e7c4d0})}, kunde reta upp uven så till den milda grad att han så småningom krossade stenen, kom han på att denna metod skulle kunna mekaniseras och inbringa deg. MacAdam tog patent på metoden världen över och uvarna blev, i denna roll, arbetslösa. Produkten makadam är krossad sten som man lägger på vägen så att den blir smutt \textsc{(s.~\pageref{d9114ffee4f2dcee302ae2b19ce5eea9})}. Det tillverkas numera i arbetsläger som t.ex. Storsien \textsc{(s.~\pageref{83f914b5fcd131d3ed802b838cce4aaf})}.

}

\small{
\textbf{Makt}
\label{7209d8106e8d1ab0fd106b96ac4a0c4c}
 är ett ord som slängs omkring lite hursomhelst i dessa tider så vi på Nissepedia väljer att reda ut begreppet en gång för alla. Makt är, i dess allra enklaste mest renodlade form att få sin vilja igenom trots en annan parts ovilja. Säg att du vill äta varma mackor till middag men din partner mycket hellre vill ha våfflor. Blev det varma mackor? Grattis! Du har makt.

 Vi dyker djupare. Om vi analyserar samhället och ser till dess olika nivåer så ser vi att vissa människor har mer makt än andra. Politiker får väldigt ofta sin vilja igenom, statstjänstemän, byråkrater och poliser likaså. Vad beror det på? En modern analys av saken hade sagt att de alla tillhör det som vetenskapen kallar \quotetext{staten}. Staten har, till skillnad från dess undersåtar, tillgång till våld \textsc{(s.~\pageref{c01df500e07826fb356183119ff0d07c})}. Eller stryk det. Båda parter har tillgång till våld, men bara staten har tillgång till \textit{legitimt} våld i form av ordningsmakten. Less på att betala hyra och slutar med det? Vips kommer det beväpnade män och/eller kvinnor hem till dig och ser till att du börjar med det igen. Försöker du göra samma sak när någon slutar betala dig så hamnar du i fängelse. Det är skillnanden, typ.

 En postmodern analys är annorlunda och det kan vara svårt att hitta ett konkret verkligt sammanhang där det går att applicera. Lugn, vi på Nissepedia är mycket kompetenta kunskapare. Du föds, läkaren håller upp din kletiga kropp och utbrister \quotetext{Grattis! Ni har fått en pojke/flicka!} - Makt. Du blir sedan sjuk och en läkare konstaterar att du har öroninflammation och börjar äta penicillin - Makt. Kunskap är makt, som Thomas Hobbes sa på 1600-talet. Man kan redan nu börja undra hur nyskapande alla postmodernister är, men tid är väl bara på låtsas om man frågar dom. Makt är alltså två saker, det ena är kunskap, för inte fan vet du bättre än en läkare så då gör du som han säger. Det andra är definitionen. Att kunna fastställa något som det ena eller det andra är maktutövning. När man blir definierad som något, exempelvis särske \textsc{(s.~\pageref{552a5aad891937bf760fb193900ea140})} så tillskrivs man en uppsjö av egenskaper som man kanske inte har, det är inte så kul alla gånger.

}

\small{
\textbf{Malå}
\label{41da4620e87888eaaeafcb3004a8d177}
 (umesamiska: Máláge) är en ort i Västerbottens \textsc{(se Västerbotten s.~\pageref{d4b008c5143dcffb6b8c35f3876c2a19})} inland.

 Malå fick sitt namn efter att några samer kastat malätna renskinn i det närliggande vattendraget Malån, antagligen på fyllan.
 Förr i tiden fanns det en blöjfabrik här, men den bommade igen. Numera kretsar denna industriort kring sågverket \textsc{(se sågverk s.~\pageref{39a99a78876fd85985cc06fa0baa3c1a})} och verkstadsindustrin Hultdins, som gör gripklor. Malå hade ett tag kommunsloganen \textsc{(se kommunslogan s.~\pageref{b80d94cf2d085d692dd87e1f5bdeaa59})} \quotetext{Makalösa Malå!}. Det roliga är den höga koncentrationen ungkarlar, eller som det heter på orten, gammpojkar \textsc{(s.~\pageref{4dcf505f68cb2f0708155b78f56ad632})}. Malå är även ett demokratiskt föredöme som får Sjöbo att blekna. Man folkomröstar gärna och folket röstar rätt, varenda gång! Ett stycke Schweiz mitt i de västerbottniska skogarna. Till exempel röstade enade Malåbor igenom ett bestämt NEJ till att sörlänningarna skulle dumpa skiten från deras kärnkraftverk här.

 HEAD2: Kända personer från Malå med omnejd

 Stor-Stina \textsc{(s.~\pageref{7f22dae34dafed9e1a1b3f2689f3793a})} (Långa Lappflickan)
 Syster Ingegerd
 Lennart Gustavsson (v)

 HEAD2: Bra suparställen i Malå

 Valfritt vindskydd i skidbacken Tjamstan \textsc{(s.~\pageref{76f026797a3d868f6a32a26b28f76f8e})}
 Omklädningsrummen vid Solviksbadet
 Bastun vid sågen
 Hemma hos någon eller i någons bil
 Malå Hotell

 HEAD2: Sagt om Malå

 \quotetext{Till yttermera visso voro skidorna klistervallade ok jaag tvangs skriva under en lapp på att skreva under lappen för att han skulle återbörda mina pjäxor} - Carl Linnæus, Flora Lapponica: s. 184.

 \quotetext{Du får en femma och en mosbricka om du kör mig härifrån till länsgränsen} - Manne Rydberg \textsc{(s.~\pageref{f97cda7739d86ca5430897eac7f614f7})}

}

\small{
\textbf{Malåbo}
\label{076b6b6cf4db46055b3f7dac198a482b}
 \textit{En gång malåbo, alltid malåbo}

 -Gammalt västerbottniskt ordspråk

 En malåbo står alltid rakryggad och arbetar för sin älskade orts väl och ve. Den som av någon anledning lämnar bygden står alltid upp för Malå trots sin fysiska frånvaro. Ett lysande exempel på en sådan hedervärd person är gunstig junker John Andersson i Umeå. Det bör dock tilläggas att ortens ungdomar ofta har ett oidipuskomplex \textsc{(s.~\pageref{58fae9174c458ab30624d6c4f38da4f8})} till sin hemby, men detta ordnar upp sig på äldre dar.

}

\small{
\textbf{Malålistan}
\label{b818e42e31c2b8a8569bda7dc092b772}
 är ett politiskt parti verksamt främst i Malå \textsc{(s.~\pageref{41da4620e87888eaaeafcb3004a8d177})}. Partiets ordförande är moderaten \textsc{(se moderat s.~\pageref{c4564b188cb670841733a3ff923c2fb0})}, tillika spaägaren Arne Hellsten som ville verka under annan flagg. Rött och blått övergavs för en purpurnyans och partiet (eller listan) sa sig verka för Malås bästa. Malås bästa råkade som av en händelse vara samma sak som företagarnas bästa mest hela tiden. Mycket märkligt \textsc{(se Märkliga sammanträffanden s.~\pageref{f46282d99158f351a81b9deaff157b4e})}.

}

\small{
\textbf{Malåparkering}
\label{2a202b3b74185add38dd9ccf7251a8b5}
 En Malåparkering är att parkera mot körriktningen, eller att parkera på huvudled. Bakgrunden till att Malåbor gör dessa parkeringar är att de är så oerhört sugna på att äta på korvkiosken att de inte har tid att följa diverse trafikförordningar.

}

\small{
\textbf{Malårca}
\label{9b3a75564b97a443bb8208edd0a10c15}
 Har ni tänkt på hur jävulskt drygt det är att åka på charter? Det hade i alla fall ett driftigt gäng i Malå \textsc{(s.~\pageref{41da4620e87888eaaeafcb3004a8d177})} som bestämde sig för att ta russinen ur kakan och ha charter hemma i byn istället. Hyr en industrilokal, fyll den med sand, höj termostaten och börja blanda paraplydrinkar. Vips: \textit{Malårca}

}

\small{
\textbf{Man}
\label{39c63ddb96a31b9610cd976b896ad4f0}
 En typisk man är Stor-Anders \textsc{(s.~\pageref{777d0562284d1dfba75c6f1b6297100d})}.

}

\small{
\textbf{Manet}
\label{f95133906d1eba86b61fc05be6aecd9c}
 DET ÄR ÅLDERSGRÄNS 18 PÅ DENNA ARTIKEL! TÄNK PÅ BARNEN SOM MOB 47-ÅKE \textsc{(se Mob 47-Åke s.~\pageref{486ee67ac39debabed3d92a7555dcebd})} SKULLE HA SAGT!!

 Manet är lätt ett av djurrikets mest doomiga djur. Det finns en manet som heter \textit{Nemopilema Nomurai} som kan bli 2 \textsc{(se tvåa s.~\pageref{84fcc0494ecf9f5af79fcd9bed184a9a})} meter i diameter och väga upp till 200 pannor. Maneter är gjorda av ett slags slime, som i och för sig är mer sludge \textsc{(s.~\pageref{2ccd23d1cd0f95dc6984215a1f1b31ca})} än doom \textsc{(s.~\pageref{b4f945433ea4c369c12741f62a23ccc0})}, men å andra sidan kan de påträffas mycket djupt ner i havet där det är kallt och mörkt och jävligt och det är så högt tryck att ens huvud \textsc{(s.~\pageref{e906cd95a540df9b16d0460fb4cf0adc})} sprängs om man går ut ur sin ubåt och försöker simma omkring. Först sprängs cyklopet så att glaset far rakt \textsc{(s.~\pageref{92be9c2f6a2fa0abd7fbcbebc76531ea})} in i ögonen på en och sen sprängs, som sagt, huvudet. Djupt där nere pumpar sig maneten fram och dödar genomskinliga djuphavsfiskar medelst ett slags celler som fungerar typ som små mikroskopiska harpuner som pumpar in gift i bytet. Maneten har ingen mun \textsc{(s.~\pageref{6585f290ce92c3de5ff339920330e26f})} och behöver ingen eftersom den typ är en mage och bara behöver ligga intill sitt byte så smälter det och så suger maneten upp det och åker vidare på sin ständiga jakt på genomskinliga djuphavsfiskar.

}

\small{
\textbf{Mangel}
\label{ecc5b41821ed829b0c3fb48d4d5389ed}
 är ett mängdmått för stora kvantiteter. För att få full betydelse måste manglet kombineras som suffix med en betydelsebärande del. Om man exempelvis skickar ner ett helt paket korv i stekpannan har man ett korvmangel, om man skottar hela gården istället för bara en liten stig \textsc{(s.~\pageref{2e9b1ac56ea26932bf0aff53fe48a533})} har man ett snömangel, och om man ser alla avsnitt av Benny Hill på raken har man ett buskismangel.
 HEAD2: Dansk trivia
 På danska betyder \quotetext{mangel} brist (på något), eftersom dansken alltid ska vara lite annars.

 Källa: Snöwall/Wahlnöt - Den vedervärdige mannen från Säffle \textsc{(s.~\pageref{e6c829ebc03d7696483c60996b81e40b})}.

}

\small{
\textbf{Manglar som ägg}
\label{7b1e91fdfd952485ddd3bc6ef4e40b3c}
 \textsc{(s.~\pageref{128a5feb8e12d0aa622e0298a8332980})} är en samlingsskiva på vinyl med det svenska mangelbandet Protes Bengt \textsc{(s.~\pageref{921bbbbb29de88e13256319e8559ccc4})} och innehåller hela deras diskografi. Skivan är högst troligt en av Benny Bus \textsc{(s.~\pageref{a8289efd495ef49dbe0225de89f7f019})} favoritskivor.

 Låtlistan ser ut som följande:
 Bengt E Sängt Cassette (1986)
 A1 	  	Bengt E Sängt
 A2 	  	Ett Sex-Pack Ägg \textsc{(s.~\pageref{128a5feb8e12d0aa622e0298a8332980})}
 A3 	  	T-Sprit
 A4 	  	Manglar Som Ägg \textsc{(s.~\pageref{128a5feb8e12d0aa622e0298a8332980})}
 A5 	  	Klämd Lem
 A6 	  	Flänsost
 A7 	  	Nisse
 A8 	  	Sås I Håret
 A9 	  	Pungklåda
 A10 	  	Slicka Mina Ägg \textsc{(s.~\pageref{128a5feb8e12d0aa622e0298a8332980})}
 A11 	  	Släptask
 A12 	  	Smal Anal \textsc{(se tvåa s.~\pageref{84fcc0494ecf9f5af79fcd9bed184a9a})}
 A13 	  	Hopplös
 A14 	  	Skateboard Punx
 In Bengt We Trust 7" (1985)
 B1 	  	Påskstämning \textsc{(se Påsk s.~\pageref{f8f0dd13b69a5c8ce56498e750551d3e})}
 B2 	  	Vattenkammad
 B3 	  	Jojo-Punks
 B4 	  	Trimmad Älg
 B5 	  	Äggkopp
 B6 	  	Skalad Gurka \textsc{(s.~\pageref{1cf02b8eacd57c92e9df0a1a3eaa8946})}
 B7 	  	Hopplös
 B8 	  	Total-Slakt
 B9 	  	B.S.B.
 B10 	  	18 Meloner
 B11 	  	Ovärt
 B12 	  	Krossa Er
 B13 	  	Urinsvägsinfektion \textsc{(se Urin s.~\pageref{524fd7acb94f9c2d879b5c1cf8335669})}
 B14 	  	Mangel \textsc{(s.~\pageref{ecc5b41821ed829b0c3fb48d4d5389ed})}
 B15 	  	Ägg \textsc{(s.~\pageref{128a5feb8e12d0aa622e0298a8332980})}
 B16 	  	Radioaktiv
 B17 	  	U-Bat
 B18 	  	OB-Inlägg
 B19 	  	En Kasse Bira
 B20 	  	Rännskita
 B21 	  	Snuken Bakom Knuten
 B22 	  	Kul I Hjul
 B23 	  	Bibeln E Min Lag
 B24 	  	Konfirmationsläger
 B25 	  	Tumstock
 B26 	  	Satsa
 B27 	  	Smörjd
 B28 	  	Inte Banga Ur
 B29 	  	Rösta På Dig Själv
 B30 	  	Grymt Sa Bengt
 B31 	  	Hjälp Snuten
 B32 	  	Vingmutter \textsc{(s.~\pageref{3d474f53ae61d29d3b924d44c21410b5})}

}

\small{
\textbf{Mango safe}
\label{b894f492cc904146f5f0bbd787247b4a}
 är ett slags lösenord som räddar en fisande person från att bli slagen av människor som drabbas av fisen. Genom att uttala \quotetext{mango safe} innan någon av offren hinner säga \quotetext{door knot} upprättas amnesti för fisaren.

}

\small{
\textbf{Mani}
\label{07cd55c7b42715ec44c133a6a165e8d2}
 i färd med ett ganska taffligt försök att gestalta \textit{Cosmos factory}.]]
 Att ha mani på något är att vara nästan lite \textit{för} intresserad. Är man till exempel intresserad av insekter är det fullt normalt att tattuera in The Locust logotyp på armen och ha en affisch på en jättestor gräshoppa hemma. Om man däremot flyttar ut i skogen under sommarhalvåret för att bo i en myrstack börjar intresset övergå i mani. Tycker man att Creedence Clearwater Revival \textsc{(se the fog s.~\pageref{576875ef0042ff21c04f5f1b9377d4e7})} är ett svängigt band är det fullt normalt att köpa alla deras skivor och ha en av deras låtar som ringsignal. Om man däremot börjar odla helskägg och ha en racercykel inomhus för att snabbt kunna gestalta omslaget till \textit{Cosmos factory}, och börjar varje dag med att klättra upp på hustaket för att spela luftgitarr \textsc{(s.~\pageref{0e2415e86edc316f5338964c6ef145b5})} och sjunga \textit{Rambel tamble}, ja då har det börjat övergå i mani. På portugisiska betyder mani \quotetext{kung}, ett tydligt tecken på hur rimligt det egentligen är med monarki. Att ha mani på något ska inte förväxlar med att vara en vanlig gammal hederlig stofil.

}

\small{
\textbf{Manne Rydberg}
\label{f97cda7739d86ca5430897eac7f614f7}
 är 33 år och kommer från Norberg men nu bor han i Stockholm.
 Han spelar mest på travet men han spelar också på allt annat som man kan vinna pengar på.
 Han vill inte inte jobba. Han orkar inte.
 Han har mycket skägg och är ganska tjock och hans förebild är Kalle anka \textsc{(s.~\pageref{64db68f686a0ca4d9d641061cb3fdf13})}. Han har likadana kläder som honom, fast han har byxor.

}

\small{
\textbf{Mannen}
\label{e16b547d8b55a95897104201e63b8853}
 Snark lixom.

}

\small{
\textbf{Manowar}
\label{ac62eaec6dc3e81da86dfbb5252c0ffc}
 Amerikanskt rockband som med lika delar wagnervurm, thule-covers och manlig fåfänga delar Nissepedia \textsc{(s.~\pageref{62400dadecd90cb5cd39062abe5a3e4a})} och världen liksom Sagån klyver Svealand i en bebolig del och den själsliga öken som breder ut sig med runstenarna. Bandets kreativa höjdpunkt nåddes med skivan \quotetext{Sign of the Hammer} där DeMaio briljerar i hednisk Mythos och mustiga bassguitarriff. Skivorna \quotetext{Fighting the world} och \quotetext{Kings of metal} slog ner som Balders död i åttitalets pojkrum. Dessa två skivor innehåller de flesta av bandets stora verk, \textit{Hail and Kill}, \textit{Carry On}, \textit{Metal kings} m.fl. Här börjar bandets kreativa motor Joey DeMaio närma sig allkonstverkets ideal som trots de tidigare skivornas geist inte kunnat uppfyllas. Nämnvärt är preludiet i Hail and Kill där en aura av äkttysk \textsc{(se tyskland s.~\pageref{b1b58da783b6d5fa090f3015f1889869})} violoinvibrato tycks sväva som en salig ande, den rättframma hyllningen och presentationen av övermänniskan: författaren själv. Konstnärligt så inleder dock bandet nu en snart 30-årig ökenvandring. Mer eller mindre tramsiga rekordförsök sätter sordin på festen. Datorernas intåg i pojkrummen och nittiotalets vitmaktvåg har dock skapat ett nytt träsk i vilket bandet slagit ny rot. \quotetext{May your swords stay wet, like a young girl in her prime} som sångaren Eric Adams så förtjänstfullt sjunger. Joey DeMaio är smartare än aids.

}

\small{
\textbf{Manuel}
\label{96917805fd060e3766a9a1b834639d35}
 är namnet på den kypare som spelas av Andrew Sachs i den brittiska TV-serien \textit{Fawlty Towers}. Tack vare sin spanska brytning anses Manuel vara nästan lika festlig som en skotte \textsc{(se skottar s.~\pageref{c2e5f84c76d823ea9482387bfb950791})} i kilt, enligt den vita arbetarklass som de flesta Oi!-band härstammar från. Kanske blir brytningen extra festlig i och med att Sachs egentligen härstammar från Tyskland. Det kommer vi aldrig få veta säkert förens Nissepedia \textsc{(s.~\pageref{62400dadecd90cb5cd39062abe5a3e4a})} lanseras på spanska. Förmodligen är så inte fallet eftersom man i större delen av Spanien \textsc{(s.~\pageref{84c63835ca2fcac8636cf7d36aa48fa4})} valde att dubba och döpa om honom till italienaren Paolo och i Katalonien till mexikanen Gonzales.

}

\small{
\textbf{Margaret Thatcher}
\label{0bdaa2c5b2f4fb15d678c3e54c10d347}
 Kärring som knullade den brittiska arbetarklassen på det ena onämnbara sättet efter det andra. När hon kolar blir det fest, och undrar man hur det ligger till det med den saken så kan man surfa in på den här sidan [http://www.isthatcherdeadyet.co.uk]. Hon har också, verkar det som, varit en förebild för Maud Olofsson \textsc{(s.~\pageref{eb913a2e9be929654908a05017401bd6})}. Bara det är värt en hel del förakt.

 Men kanske mest känd är hon för att ha varit med i gruppen på J. Lyons and co. som uppfann mjukglassen, eller i alla fall hur man blåste upp glass med luft så att folk får mindre götta för pengarna. Lika osympatiskt som resten av hennes gärning.

}

\small{
\textbf{Margit Sandemo}
\label{d4a62753375ff2e975534b9ca740fd28}
 Författare som är Norges \textsc{(se Norge s.~\pageref{aa03c4e9d9f8011f9b0102380b029256})} svar på Storbritanniens J.R.R Tolkien \textsc{(s.~\pageref{3f0b7fcbd9fa7369ca314a46c280b67e})}. Sandemos böcker handlar om isfolket och innehåller till många läsares glädje en hel del ångande erotik \textsc{(s.~\pageref{972f097461d1eab1c1ff104757bad922})}. Bokserien om isfolket består av hela 47 böcker och gavs ut mellan 1982 och 1989, vilket betyder att Sandemo, vars namn för övrigt kommer från den opopulära norska maträtten sandmos, skrev i genomsnitt en miljon böcker om året.
 HEAD2: Vid sidan av författandet
 Vad många inte vet om Sandemo är att hon vid sidan av författandet också är en av de främsta företrädarna för pösbyxemärket Wu-wear. Detta upptar sedan sent nittiotal all Sandemos tid. Hon åker runt i världen och för märkets talan i olika sammanhang.

}

\small{
\textbf{Marknadskläder}
\label{7008bef5ef993c8d280afb9964711f82}
 På marknader, alltså inte \quotetext{marknaden} utan marknaden; kan man utöka sin garderob med:
 \begin{itemize}
 \item Foppatofflor
 \item Fleecetröjor med indianmotiv,
 \item T-shirts med
 \begin{itemize}
 \item \quotetext{Skojiga} texter
 \item Taffliga efterapningar av kända varumärken
 \item Indianmotiv
 \item Uvar \textsc{(se Uv s.~\pageref{45210da832f9626829457a65e9e7c4d0})} i solnedgång
 \item Leggings \textsc{(s.~\pageref{ba2d335fc737a1517329bf1ee71f77bc})} i skinnimitation
 \end{itemize}
 \item Cowboyhattar \textsc{(se Cowboyhatt s.~\pageref{229b4cd9f3e94e4794ffba4ed0bea704})} i skinn
 \end{itemize}

 De som handlar kan grovt indelas i tre kategorier:
 \begin{itemize}
 \item Tjackad\textbarAmfetaminister
 \item White trashfetton
 \item Kids som inte vet bättre
 \item Ironiska medlemmar av medelklassen
 \end{itemize}

}

\small{
\textbf{Marks balkong}
\label{5540b686b9c7d3d578f494ff71f49fe9}
 var ett nöjesetablissemang i Umeå som hade sin storhetstid sommaren 2010. En bidragande orsak till ställets popularitet tros ha varit att man valde att bortse från det rökförbud som annars gällt på svenska krogar sedan 2005. Ursprungligen lanserades verksamheten som ett alternativ för de människor som önskade rulla hatt \textsc{(s.~\pageref{7c7afc9fb7bb52962f954c0cb548c10c})} på någon av stans terasser med serveringstillstånd men saknade kapital för att kunna köpa bärs där. På Marks balkong serverades öl till självkostnadspris och inredningen bestod av två campingstolar och en banankartong. På grund av den positiva responsen från allmänheten förfinades konceptet i mitten av sommaren med en stoppad tvåsitssoffa som närmast gav känslan av en lyxig lounge. Ägaren behöll dock principen med självservering och priserna låg därmed kvar på den mycket uppskattade nivån. Vid entrén välkomandes besökarna av den alltid lika glada värdinnan Minmin som såg till att ingen behövde sitta själv.

 HEAD2: Nöjesutbud
 För underhållningen svarade ägarens dator där sommarhitten \textit{Holiday} av Pink mountaintops gavs stort utrymme. Även balkonggrannarna till vänster kunde ibland titta ut och bjuda på en underhållande skröna eller dråplig anekdot. Temakvällar förekom också såsom exempelvis grekisk afton med soltorkade tomater och vitt bröd. Men framförallt uppskattade nog besökarna varandras sällskap. Marks balkong hade nämligen en närmast oförskämt hög andel trevliga människor som gäster.

 HEAD2: Nuvarande situation
 I nuläget ligger Marks balkong på is då ägaren till konceptet saknar lokaler med de rätta förutsättningarna.

 HEAD2: Marks balkong i kulturen
 Mark har målat en tavla av en man som sitter på en balkong och dricker folköl. Utropspriset för verket är fyrtiotusen miljarder \textsc{(s.~\pageref{c2160bffc9c5ca88e77204672e62e489})}.

}

\small{
\textbf{Marmaduke Grylls}
\label{ba16bd8c7fd5e72c30e72dfda9492a8c}
 är son till äventyraren och överlevnadsexperten Bear Grylls.

}

\small{
\textbf{Marsvin}
\label{9140164273c3e97c542cfa529d0456fb}
 är ett djur med ett \quotetext{dointande} läte.
 De \quotetext{dointar} ofta när de är hungriga/rädda.

 I Danmark är marsvin tumlare.

}

\small{
\textbf{Martine}
\label{d8ea363365c125fbeae837cec35df042}
 var den första franska apan i rymden. 7 mars 1967 gav sig Martine ut på sin historiska resa till rymden där hon levde i flera dagar innan någonting gick snett \textsc{(s.~\pageref{64b2eafa388ee50a226adc9013644f08})}.

}

\small{
\textbf{Maski Hallonen}
\label{3750bcd77f861642a358a89c2cd43829}
 (1921-1947) blev som tjugoåring tvångskommenderad av finska armén att utvandra till Sverige tillsammans med ett hundratal andra unga finnar. Deras uppdrag var att samla pengar till krigsinsatsen på hemmafronten, kosta vad det kosta ville. De unga männen sökte jobb på flera industrier, men hade ingen tur. I slutändan blev krigarna tvungna att plocka bär för att skaffa pengar till fosterlandet.
 I Sverige har en ordvits uppstått kring Hallonen, där man kallar honom Finlands \textsc{(se Finland s.~\pageref{631d44eaa1254ff71a1e11ba021d1266})} sämsta bärplockare på grund av hans namns lustiga betydelse på svenska. Sanningen är den att Hallonen skötte sig bra. Inte bäst, men bra. Vad han tyckte om vitsandet uppdagades aldrig då han kort efter sin återkomst till Finland \textsc{(s.~\pageref{631d44eaa1254ff71a1e11ba021d1266})} avled i sviterna efter ett fall av parasitsjukdomen testikulär maggot.

}

\small{
\textbf{Masochism}
\label{02b7b2b6c595195943b35ce38df20afd}
 Aj är skoj.

 Eller, mer nyanserat, att må bra av att må dåligt. Vanligt i norra norrlands inland, kulturella kretsar och efter fem sorters sprit, varav minst en utländsk. Ouzo framför stafflit i Pajala lovar gott.

}

\small{
\textbf{Masonitemuseet i Rundvik}
\label{69525d5014f8a47a8f5d954c5c998a73}
 \textbf{Masonitemuséet i Rundvik} är Sveriges enda museum som enbart handlar om masonit. Museet är baserat i en skolsal på övervåningen i Rundviks gamla skola. Det var här företaget Masonite, ägt av amerikanen William H. Mason (som var chefsingenjör hos Thomas Edison), licenserade Europas första fabrik för tillverkning av masonitskivor till Nordmalings Ångsåg Aktiebolag år 1929. Här kan besökare exempelvis se gamla reklamanslag som meddelar att masoniten ersätter inte bara trä, utan även plåt och marmor.

 Masonitemuseet invigdes på Rundviksdagen 2004 och hålls öppet vid större arrangemang eller efter överenskommelse. Prata med Sune eller Lennart så löser det sig.

 Källa: Jonas Fröberg  \textit{Masonit - De oanade möjligheternas material}.

}

\small{
\textbf{Matematikmaskinnämnden}
\label{796b008ec3601ee7d89ea8a2df6725e5}
 Statlig myndighet som fram till 1963 ansvarade för att förse Svea rike med datorer. Lades ned av Tage Erlander som en del i förarbetet inför valrörelsen 1964. Strategien var lyckad och fick Folkpartiets \textsc{(se Folkpartiet s.~\pageref{b692fa6a23fd557940474dc94909d80f})} storbossnörd \textsc{(s.~\pageref{456018ad01124baca4c32b6567fca7b8})} Bertil Ohlin att ragequitta flera debatter.

}

\small{
\textbf{Math metal}
\label{b3c2c3d43173af93b0d1230b09bcf157}
 är metal som till 99\% eller mer bygger på avancerade och snabbt spelade skalor. Math metal bör undvikas när någon kan se, men kan avnjutas i lönndom.

}

\small{
\textbf{Matilda}
\label{91f2b7dfd8fc3d900133c356f92c4e20}
 är det forndanska namnet för en serie mejeriprodukter som bland annat innehåller matlagningsgrädde och bordssmör.

}

\small{
\textbf{Mats Lundgren}
\label{aa1c612378b96902d2dce79820708a94}
 är en busschaufför i Umeå som lessnade på den outhärdliga sommarvärmen. Det fascistiska bussbolaget Nobina tillåter inte sina manliga busschaufförer att ha shorts på sommaren då det inte är en del av uniformen. Den driftige Lundgren upptäcker då att kjol är tillåtet och kör på det istället.

 Hade ett sådant pris funnits hade Lundgren snarast blivit tilldelad Folke Pudas-priset \textsc{(se Pudaslåda s.~\pageref{6a56958e2057dd500650e2be8049e033})} för sina insatser i kampen mot byråkratin.

 [http://www.svd.se/nyheter/inrikes/busschauffor-bar-kjol-pa-jobbet_103281.svd]

}

\small{
\textbf{Matt Pike}
\label{f1e5b05112d62b84340f4d287585d83d}
 är världens gulligaste och bästa gitarrist, frontman i High on Fire och en ägmästare \textsc{(s.~\pageref{8324518500d7e7ccd22ae364887d4476})} av rang. Tidigare svingade han yxan i stonertrion Sleep och levererade då mästerverket Dopesmoker \textsc{(s.~\pageref{cebc8a343bbfefbfac0078fcd926a0e0})} vilket vi alla är honom tacksamma för idag. Har man inte hört Dopesmoker är man en tönt och borde inte få vistas bland folk.
 HEAD2: Styrkor och svaga punkter
 Det Matt Pike enligt egen utsago är bäst på är att lira gura och att vara en god människa och det hoppas vi att han ska fortsätta med länge än. Det han är lite mindre bra på är att kröka \textsc{(se spritfylla s.~\pageref{0668c687b51995118ec27cbf25061118})}, så det har han slutat med (typ).
 HEAD2: Sagt om Matt Pike
 \quotetext{Det står fan en asskön snubbe där ute och ba matar slayerriff fast på ett 70-talssätt} - P Pajen \textsc{(se Paj s.~\pageref{0b438dd454bc6a17de239ebf0a46b91b})} Pajunen.
 \quotetext{Det är fan paddan i pannrummet \textsc{(s.~\pageref{f3e250508788285c8d4b2ea74db0c6e7})} från början till slut som gäller när den karln är inblandad.} - P Pajen Pajunen.

}

\small{
\textbf{Maud}
\label{07830bf6d135d130a8566fe94ba59eb8}
 är näringslivs-tugg för \quotetext{don't get me started.}

}

\small{
\textbf{Maud Olofsson}
\label{eb913a2e9be929654908a05017401bd6}
 Maud \textsc{(s.~\pageref{07830bf6d135d130a8566fe94ba59eb8})} Olofsson (även smeksamt kallad Mao Näringslivsson) är en ivrig bäver \textsc{(se Ivriga små bävrar s.~\pageref{6d10ab1ba7bd378ba7cc1629ddf2bbde})} från Robertsfors som hatar arbetarklassen. Tack vare sitt rabbinskap i  bondeförbundet \textsc{(se centerpartiet s.~\pageref{e331dec360e356adc1e2db36fe9a9f3f})} har hon lyckats dupera  kulakerna \textsc{(se kulaker s.~\pageref{c17322f1f8b87ec8fc35538dbe1e9668})} att stödja henne ett flertal mandatperioder.
 Tack och lov har hon avgått. En del firar detta med stora smörgåsar, dagobertmacka, smörgåstårta \textsc{(s.~\pageref{b81fe66f0f43489f730fd6baa91a12f7})} eller knäcke med  dubbelsovel \textsc{(se dubbelsovla s.~\pageref{4a58428516d8ba930242406ad6073922})}.

 HEAD3: se även
 \begin{itemize}
 \item Violas \textsc{(s.~\pageref{bc0b8c20b7ac9de2bb42b4c7285e93ba})} klädbutik
 \end{itemize}

}

\small{
\textbf{Maud olofsson}
\label{eb913a2e9be929654908a05017401bd6}


}

\small{
\textbf{Max Weber}
\label{2fb708fe6352f97cd7e4fe5bab54a88f}
 var en tysk professor som anses varit medgrundare till samhällsvetenskapen sociologi. Vad vanligt folk minns Weber för är dock inte hans akademiska arbete, utan för att han uppfann klotgrillen. Anledningen till att det var just Weber som uppfann detta var att han kom på att man kunde använda gallret från rationalitetens järnbur i grillarna.

 Weber plågades hela sitt liv av problem med det sexuella, främst den egna potensen. När de flesta andra led i tystnad över dessa problem tyckte Weber att det var en bra idé att diskutera saken med sin dyra moder. Hade Sigmund Freued varit samtida med Weber hade antagligen denne gått bananas \textsc{(s.~\pageref{ec121ff80513ae58ed478d5c5787075b})} över detta.

 HEAD2: Nissequotes
 \quotetext{Passera dansgolvet till förmån för buffén - det är en social handling som inte går av för hackor!}

 \quotetext{Försökte ligga med Marianne igår. Fick inte upp den. Åh mamma, vad tror du att detta beror på?}


 Se även:
 Bokbål \textsc{(s.~\pageref{95ba942fddfa43299693158f961bfa99})}

}

\small{
\textbf{Medelklassvänner}
\label{bbf0e28d1f8015c14119873cc4a518ae}
 En vänskap man gjort genom exempelvis ett gemensamt intresse för trädskällare eller modern litteratur. Allt känns bra tills man för första gången blir hembjuden till den nya bekantskapen. Har man inte en halv miljon i studieskulder, skjorta från dressmann \textsc{(s.~\pageref{02ee8e32b89869fffd11aceb4f2e1c10})} och upplever ifyllandet av blanketter för ROT-avdraget som sitt största problem kan stämningen lätt bli lite stel. Det kan man lösa genom lite lekar och spel.

 ==

}

\small{
\textbf{Medeltiden}
\label{88cbc30c5b233d97df68b5b041ac0655}
 var, som namnet antyder, ingen braksuccé \textsc{(s.~\pageref{678371d35369d3d29afceb1445630833})}. Man levde på kottar och bär, var sjuk hela tiden och var träl åt någon rik. \quotetext{Alla som gillar medeltiden är inte idioter, men alla idioter gillar medeltiden}, som Carl von Linné \textsc{(s.~\pageref{5e8380bf6b7ce99678e6752b6d9e709e})} träffsäkert lär ha beskrivit det. På medeltiden leve Petrus de Dacia \textsc{(se Petrus de dacia s.~\pageref{e07b2cf719fa237191665c127c7080c2})} och Hildegard av Bingen \textsc{(s.~\pageref{c7c6415c032f7d851cd2c0a11f40be0b})}.
 et som hela byn åt ur. Situationen ser ytterst sugig ut och är typisk för en vanlig medeltidsdag.]]
 Källa: Internet

}

\small{
\textbf{Mejram}
\label{c2aec38c403dbce7fcbf17ef2a5305f1}
 På mejramsförpackningen står det att den passar utmärkt till ärtsoppa, det är lögn. Mejram är bra om man vill att maten ska smaka mögel utan att va möglig. Timjan, däremot, ska man ha i ärtsoppan.

}

\small{
\textbf{Merchband}
\label{dd9d8bdba3839bffebf775a9f6aa51c9}
 Ett merchband är ett band som tillhandahåller ett brett utbud av merch i form av tshirts, backpatches, muggar osv. men som sällan släpper nytt material eller mest spelar dålig rock.

 HEAD2: Förtydligande exempel
 \begin{itemize}
 \item Ghost
 \item The Misfits
 \item Exploited
 \item The Adicts
 \item Chicago Bulls
 \end{itemize}

}

\small{
\textbf{Metaforiska klädesplagg}
\label{c670bf663e4ed36a579c5f5cf9d55f20}
 En udda retorisk tradition är att ta på sig olika typer av metaforiska klädesplagg.

 Lista på plagg:

 Spenderarbyxor
 Offerkofta
 Hårdhandskar
 Betongkeps
 Kritikerglasögon
 Nattmössa
 Träfrack
 Rödstrumpor
 Tvångströja
 Skitstövel
 Silkesvantar
 Sportmössa

}

\small{
\textbf{Metallstång}
\label{6b45527e41ce216a150d4ac5950322bd}
 Med ett schysst järnrör slår man världen med häpnad.

 \textbf{Källa}:Sockerconny

}

\small{
\textbf{Metalmynt}
\label{66b54fb5f4c4b119a30452f71d678055}
 är ett slags poletter av plast som används som valuta \textsc{(s.~\pageref{cf1e2a0af4955aa7539b6e12e9d282e6})} vid köp av öl på diverse hårdrocksfestivaler inom den Europeiska Gemenskapen. Hårdrockaren byter in sina vanliga konventionella pengar mot dessa metalmynt och kan sedan förse sig med starkvaror i barer eller öltält runt om på festivalområdet. Utanför detta är metalmyntet värt väldigt lite och accepteras sällan eller aldrig som pengar i det vanliga civilsamhället. Den kände författaren Professor Etienne \textsc{(se Användare: prof. Etienne s.~\pageref{a9878d2280e5a39becac8f73d113df91})} ska enligt egen utsago vid ett tillfälle ha försökt föra in falska metalmynt till ett värde av femton kubikmeter öl på den belgiska \textsc{(se Belgien s.~\pageref{f79ffe9e826a19f9f6a446c90e21c4e3})} hårdrocksfestivalen deathfest. Då interpol fått nys om kuppen tvingades han dock överge dessa i ett irrigationsdike och fly ut ur landet gömd i ett hölass.

}

\small{
\textbf{Micke Alonzo}
\label{4bd27d3cf2641236f956496c779a0dc2}
 Sångare i det legendariska punkbandet KSMB och Sveriges \textsc{(se Sverige s.~\pageref{b1999637949ed135b2ca03f3a38460cc})} genom tiderna mest underskattade rockband: Stockholms \textsc{(se Stockholm s.~\pageref{edcd259e0a03c7ab70feb186bae19f13})} negrer. Han har mycket kort, nästan osynligt hår och har problem med vissa feminister \textsc{(se feminism s.~\pageref{44a20d8673cfd6258002acb74ec2f83e})} som han tycker har gått lite för långt.

 HEAD2: Privatliv
 Stora obesvarbara frågor väller upp inom Micke Alonzo. Han känner att han är som en roderlös båt som guppar på ett ödsligt hav. Ibland går vågorna höga och hotar att kantra den lilla båten. Andra gånger ligger ytan blank som en spegel och havet, enormt och tomt, breder ut sig i alla väderstreck. Alonzo ser sin spegelbild i havet. Han rör vid sitt ansikte och frågar viskandes; Vem är jag? Vad är meningen med allt?

}

\small{
\textbf{Midsommar}
\label{e7055fcb068b695ddcfb7e34bb4866a7}
 Fånig benämning på ett fylleslag som inte på något vis infaller i mitten av sommaren.

}

\small{
\textbf{MIG-svets}
\label{077609c718e131aefe3b21b4b34be728}
 Det är ingen idé att försöka MIG-svetsa en målad yta eller något annat material än stål och järn, för det går inte. MIG-svetsen är självmatande och gasdriven och finns att få med olika strömstyrka. Den lämpar sig till vardagligt bruk och är en bra svets att börja med om man vill prova att svetsa. MAN SKA ALLTID HA SKYDDSHANDSKAR PÅ SIG ANNARS KAN MAN TAPPA HÄNDERNA!!!!1!!!11! Det är viktigt att se till att det man svetsar är jordat. MIG-svetsen, eller miggen som den initierade oftast säger, uppfanns av Carl von Linné \textsc{(s.~\pageref{5e8380bf6b7ce99678e6752b6d9e709e})} under dennes långa och händelserika Norrlandsresa.

}

\small{
\textbf{Mikis Theodorakis}
\label{a51a602ccd87730203211131f20c5d94}
 är Greklands främste tonsättare. Det spelar ingen roll vilken ton man frågar om så sätter han den på en gång. Man ba \quotetext{Ja men den där då?} och så pekar man på en svart tangent på pianot mitt bland alla andra. Han ba \quotetext{det är c-moll}, utan att blinka. Inte ens om man krånglar till det och säger \quotetext{amen rita ett G i fis-dur då}. Då tar han fram papper och penna och ritar \textit{exakt} som det ser ut. Vill man veta mer om Mikis förehavanden kan man fråga Nissepediaanvändaren HratvinnFlygur \textsc{(se Användare: HratvinnFlygur s.~\pageref{26c5d96dca8dfce84752fa1d4095fdb0})} för han är en av nordens främsta  kännare om gubben.

 Theodorakis är lika grekisk som en grekisk bondsallad och spelas flitigt på alla rikets charterorter. Otaliga gånger har, på Retsina och Rhaki, grisfulla \textsc{(se grisfull s.~\pageref{80fc21ba5a45f2d0cd24855d78fa7246})} brittiska kvinnor dansat på, och ramlat ner från, borden på bodegor lite varstans i den grekiska övärlden till tonerna av Theodorakis \quotetext{Zorba}.

}

\small{
\textbf{Miljöbil}
\label{d458e7b6350f49c57c4f4e4e77ed5cc0}
 Det lobbas hårt av både stat och kapital för att alla ska köpa sig en miljöbil. Anledningen till detta är att det inte längre går att blunda för att jorden håller på att gå under på grund av koldioxidutsläppen, men att göra något så vansinnigt som att producera och konsumera mindre, det kommer inte på fråga! Miljöbilar är enligt etablissemangets definition elbilar, etanolbilar och snåla dieselbilar med partikelfilter. Här manar Nissepedia \textsc{(s.~\pageref{62400dadecd90cb5cd39062abe5a3e4a})} till eftertanke och ber er ponera \textsc{(s.~\pageref{81de0f38ad2cd422870c2e70763f3510})} följande: Vilken bil är bättre för miljön? Den bil som körs eller den bil som står still? Självklart är det bilen som inte brukas och således är den sanna miljöbilen en riktig jävla rishög. Denna bil har en livslängd på max ett halvår efter inköp om inte dess ägare är en jävel på att skruva, sen är det raka spåret till skroten där den aldrig mer är Moder Jord till last. Slutet gott, allting gott.

 Bland de mer kända miljöbilarna finns Hallonmobilen \textsc{(s.~\pageref{42fc08eaaee2a76c23dd460dd547ab3e})} som satte Umeå norr om älven i skräck unders dess halvår i livet. PG Pogo från Missbrukarna lär ha sagt att äga en miljöbil är det mesta punk han håller på med nuförtiden, följt av ett karakteristiskt hjärtligt skratt.

}

\small{
\textbf{Miljöpartiet}
\label{3e11b29518eeea19128b64869699f363}
 är ett politiskt parti skapat av den norske entreprenören Jam-Ole Rip Tit. Namnet \quotetext{Miljöpartiet} är ett anagram av grundarens eget namn men betyder egentligen ingenting.
 Två prickar satts över o:et som en hyllning till musikgruppen Rob n Raz.
 Jam-Ole Rip Tit bor numera på Bahamas.

}

\small{
\textbf{Mimmi Pigg}
\label{47a20f7432f125f29ac8d0101be60ad7}
 är ett sätt att klä sig där underkroppen är påklädd och överkroppen bar. För att Mimmi Pigg-klädnaden ska bli fulländad bör man även komplettera med ett par väldigt fula klackskor.

}

\small{
\textbf{Min bästa byxa}
\label{d713d68db15d469d6e39abacefefb3ab}
 är en svensktillverkad byxa \textsc{(s.~\pageref{bd74f429522c7c1481fbba07187efc6b})} som vann stor popularitet under 1970-talet. Byxan tillverkas av det patenterade materialet Avesta jet-tex och ska enligt uppgift ha en \quotetext{suverän passform} och vara ytterst komfortabel. Konsumenten hade också många härliga färger att välja mellan. Och den är maskintvättbar!

 .]]

}

\small{
\textbf{Min kära gamla soppeskål}
\label{42ce7b9516a04ae0a46ccb0de720f3c3}
 är en sång av Sveriges \textsc{(se Sverige s.~\pageref{b1999637949ed135b2ca03f3a38460cc})} proggflaggskepp Philemon Arthur \& the Dung. Låten släpptes för första gången år 2002 på bandets skiva \textit{Får jag spy i ditt paraply?} (Silence records), och blev en omedelbar hit bland landets batikklädda befolkning. Sången premierades framförallt för sitt allmogebudskap och sin enkla refräng som var lätt att komma ihåg och sjunga med i. Philemon Arthur \& the Dung hade inte haft en sådan mosterhit sedan 1992 års \textit{Plocka päron \textsc{(s.~\pageref{4db1fecfcb40624ab38021166b8aaa05})}}.

 På Österlen i Skåne fick sången sådant genomslag att Peps Persson bildade ett särskilt sällskap, Soppeskålsorden, vars medlemmar betraktade föremålet i visan som en helig gral som skulle återbörda alfalfagroddar på menyn i varje skolmatsal. Denna sammanslutning hade som mål att finna soppeskålen och föra hem den till Silence musikstudio i Koppom.

 HEAD2: Text
 \textit{Min käre gamle soppeskål jag har dej så kär}

}

\small{
\textbf{Minnesmonumentet över soldater från Kanadas ursprungsbefolkning som stupade i Andra världskriget}
\label{cd30bfe578c89775aa1a0a8fc043614d}


}

\small{
\textbf{Minusmat}
\label{0d15d5e8e5494ca8b24a8c372754b7d0}
 Mat som ger mindre energi vid intaget än det går åt när man äter.
 Resultatet blir att man blir smalare av att äta än att ligga på soffan.

 Exempel:
 \begin{itemize}
 \item Ris(äta med pinnar)
 \item Pommes frites( äta en och en)
 \item Kräftor (jättepilligt med skalet)
 \item Myror (för björnar)
 \end{itemize}

}

\small{
\textbf{Missanpassad}
\label{40350e8687998c87216814f22822c90b}


}

\small{
\textbf{Mjukis}
\label{2e68c3d8cac8cbb7f9488d36090323a6}
 : Ett adjektiv som ofta används för att beskriva någon vars främsta karaktärsdrag är snällhet. I de flesta fall används det på ett uppskattande sätt. I dessa fall främst av kvinnor om män, då kvinnor ofta redan antas vara mjukisar och det därmed blir överflödigt att kommentera eftersom det inte är ett brott mot en vedertagen norm. När det används av män om andra män är ofta tonfallet hånfullt, då mjukisen ofta \quotetext{går hem} hos det motsatta könet, vilket sällan uppskattas av andra män i mjukisens närhet. För att exemplifiera skillnaden i diskussionen om mjukisen könen emellan följer här två fabricerade yttranden.
 \begin{itemize}
 \item Yttrande A: Betty, milkshakekonnässör, till sina väninnor, Magda, Penny och Sigrid om mjukisen Eugene: \textit{\quotetext{Åhhh! Eugene är en sån mjukis. Man vill bara krama honom!}} (följt av uppskattande fnitter).
 \end{itemize}

 \begin{itemize}
 \item Yttrande B: Zacke, innebandyforward, till sina vänner, Jacke och Tåbbe om mjukisen Eugene: \textit{\quotetext{Asså Eugene! Vilken jävla mjukis! Palla inte ens å se när t-rexen ba käkar upp killen på skithuse' i jurassic park utan å börja lipa!}} (följt av revirmarkerande snusspottande och hånfullt skrockande).
 \end{itemize}

 Exempel på mjukisar: Drängen Alfred i Emil, Lille Skutt i Bamse, Björne i Björnes Magasin, Mart i DLK.

 Uppmärksammas bör att mjukis i pluralform, mjukisar, lätt kan blandas ihop med slangbenämningen på mjukisklädsel \textsc{(s.~\pageref{57a78bf29e9f6fb6a4dba89fc21bc897})} (mjukisar). Förvirring av begreppen bör särskilt undvikas då mjukisar inte alltid bärs av just mjukisar. Många gånger bärs mjukisbyxor av sportutövande ungdomar, en demografisk grupp som i stort sett aldrig uppvisar mjukisens mest prominenta egenskap; snällhet.
 Den mjukis som oftast bär mjukisbyxor är den långtidsarbetslösa mjukisen, som nio gånger av tio har gått en estetisk utbildning på gymnasiet för att sedan komma ut i ett kargt samhällsklimat. Mjukisen misslyckas i samhället på grund av dess oförmåga att \quotetext{vässa armbågarna} och \quotetext{ta för sig}. När den sortens mjukis använder mjukisbyxor kallar den inte detta plagg för \quotetext{mjukisar}, som den sportande ungdomen, utan snarare \quotetext{fisbyxor}.

}

\small{
\textbf{Mjukisklädsel}
\label{57a78bf29e9f6fb6a4dba89fc21bc897}
 är ett samlingsbegrepp för mjukisbyxor och mjukiströjor. Plaggen ska vara tillverkade i minst 50\% bomull (max 80\%), resten polyester och med fördel vara matt färgade i antingen grått eller svart. Mjukisbyxan är idag en stapelvara både för den unge ambitiöse sportutövaren, såväl som för utförsäkrade, arbetslösa och folkölsalkoholister.

 Mjukiströjan försvann i stort sett helt från marknaden under 90-talet och används idag så gott som uteslutande av två grupper där den första är folk som tycker det är ballt med retro. Den andra gruppen som fortfarande storkonsumerar mjukiströjor (uteslutande i kombination med mjukisbyxor) är handbollsmålvakter, vilka alltid har tjänat som fanbärare för mjukismode.

}

\small{
\textbf{Mjukpack cigg}
\label{afed83191faa2d3d33ba83795792dbc1}
 \textbf{Mjukpack för cigaretter} är en benämning för den behållare som flera typer av cigaretter säljs i. Mjukpacket skiljer sig mot andra ciggpack genom sitt mjuka material, andra ciggpack är för det mesta hårda. Mjukpackets främsta egenskap utgörs dock av att det alltid finns minst en cigg kvar i det. Mjukpack går utmärkt att förvara i sin bakficka \textsc{(s.~\pageref{d259b5ebe8541b74129f0c78a82335b7})}.

}

\small{
\textbf{Mjölkskorv}
\label{0340144b99f5ede3c88664d279049104}
 är en åkomma som visar sig såsom små kladdiga, gulbruna fjäll i hårbotten. Varför det uppstår vet man inte riktigt. Kanske är det bara så att vissa har mer otur än andra. I Skåne, den lantliga nynazismens hemland, är mjölkskorv en delikatess som hör varje ålagille till. Den fattigaste personen på orten fångas in och grävs ned till brösthöjd på en åker. Där får hen sedan stå i ungefär två veckor utan att tvättas. När tiden är mogen skördas skorven genom att man skrapar försiktigt i hårbotten med ett stämjärn. Det blir alltid lite kvar men det är byns ungar snart framme och slickar bort. Skorven blandas sedan med röd mjölk på samma sätt som man rör ihop O´boy och festdeltagarna får varsitt glas som dom kan doppa sina utrotningshotade ålar i och suga i sig av den goda skorven.

 Det latinska namnet på mjölkskorv är, ballt nog, \textit{Crusta lactea} (OBS sant!).

}

\small{
\textbf{Mjölkuv}
\label{0b7347112ec2721aa5c81d126b089dc7}
 Historien om Mjölkuvens (\textit{ Bubo Lacteus }) uppkomst är inte bara en historia om hur problematiskt det kan vara när vilda djur blir tamdjur, utan även en historia om en man som milt uttryckt var helt jävla från vettet.

 HEAD2: Mjölkuvens uppkomst
 Den Västerbottniske \textsc{(se Västerbotten s.~\pageref{d4b008c5143dcffb6b8c35f3876c2a19})} mjölkbonden \textsc{(se bönder s.~\pageref{30a6fc00c9102680b8196b1b79935ec4})} n'Lennart Andersson från Vindeln hade en gård med 30 kor under 1900-talets början. En dag när han mockade dynga ute i fuset så hände något märkligt, korna revolterade \textsc{(se naturens dialektik s.~\pageref{8688bb10239ff053879f9219f8191bd2})}. I ett av kaos av dynga, mjölk och utslagna tänder hittades senare bonden i fosterställning av sin hustru a'Matilda, men korna var sedan länge borta. Familjens redan fattiga och eländiga situation förvärrades gravt av förlusten av hela sin produktion, men så lätt skulle banne mig inte n'Lennart ge upp. Ett försök att börja hållas med kor på nytt gjordes, men sviterna av den tidigare incidenten hade gett n'Lennart vad den moderna vetenskapen kallar Post-traumatisk stress \textsc{(s.~\pageref{e10a36f1a5231e597daf8f42dc1ab55a})}, på den här tiden sa man dock att han var \quotetext{schvag åt nervern}. n'Lennart och a'Matilda satte sig ned vid köksbordet med varsin bullskiva och började fundera över hur de nu skulle göra. Det spekulerades i får, getter och till och med hundar, men inga djur verkade dugliga som mjölkproducenter för n'Lennart. De går till sängs och senare på natten vaknar n'Lennart ur en dröm och skrei \quotetext{UV \textsc{(se uv s.~\pageref{45210da832f9626829457a65e9e7c4d0})}} och springer i strumpläst och nattsärk till skogs. a'Matilda väntar oroligt på sin make, som hon nu insett är helt bananas \textsc{(s.~\pageref{ec121ff80513ae58ed478d5c5787075b})}, i en hel vecka innan han återigen står på farstukvisten. Hans nattsärk är ett minne blott och den sargade kroppen vittnar om fasliga olyckor. På hans armar sitter två vidunderliga fjäderfän. \quotetext{Jag lyckades, Matilda, jag lyckades!}. a'Matilda har svårt att se hur denne nakne man som ser ut som hade han spenderat veckan i en köttkvarn ska ha lyckats med något överhuvudtaget. n'Lennart ser sin hustrus skepsis och springer ut i fuset med fjäderfäna för att återvända med en spann full med vad som i alla fall liknar mjölk. \quotetext{Uvmjölk, Matilda! Vi kommer bli rika, RIKA!}. Aldrig hade a'Matilda ångrat att hon lovat prästen att stå denne vansinnige man bi som just nu, men hon höll god min och lät n'Lennart fortsätta sitt vansinniga projekt. Uvmjölken blev inte någon braksuccé \textsc{(s.~\pageref{678371d35369d3d29afceb1445630833})}, men a'Matilda ville inte låta sin makes psykos gå till spillo så hon satte sig och kärnade och vips: Uvsmör \textsc{(s.~\pageref{240a6e2f1169dc87b9533f6b9c7b0aec})}. Uppehället var äntligen tryggat för familjen Andersson. Som en minnesbeta står det på parets gemensamma grav på Vindelns kykrogård \quotetext{När livä ge er uvmjölk, gör uvsmör \textsc{(s.~\pageref{240a6e2f1169dc87b9533f6b9c7b0aec})}}.

 HEAD2: Mjölkuven idag
 Familjen Anderssons uvgård drivs fortfarande av n'Lennart och a'Matildas ättlingar i Vindeln. Några djur har dock rymt och ett litet bestånd har upprättat sig längs Vindelälven. Dessa ser precis ut som vanligare uvar \textsc{(se uv s.~\pageref{45210da832f9626829457a65e9e7c4d0})}, men har betydligt större uvjuver och kladdigare fjäderdräkt.

}

\small{
\textbf{Mob 47}
\label{9955900bded21660e7f4e15ae8d23e3a}
 är ett svensk råpunkband från Stockholm som hållit på sen -81. Deras texter handlar mest om vad de tycker om saker. Det här tycker dom inte om: systemet, hela jävla samhället, David Bowie, rustning, bylingen \textsc{(s.~\pageref{b54c92e1b1671e982dc24eefae2edce1})}, politiker, religion, krig, lögner, rasistiska regimer, maktmissbruk, kärnvapen och mycket mer. Det här tycker dom om: resning mot överheten, frihet, fred, cellbyggnader, nedrustning, ändringar i systemet, rättvisa, barn, att inte rösta och inget mer. Dom tycker också att alla djurförsök är mord. För att kunna sjunga om andra saker än vad dom tycker har dom ett sidoprojekt som heter Protes Bengt \textsc{(s.~\pageref{921bbbbb29de88e13256319e8559ccc4})}.

 Deras vanligaste logo är ett kranium med tuppkam.

 HEAD2: Medlemmar
 Mob 47-Åke \textsc{(s.~\pageref{486ee67ac39debabed3d92a7555dcebd})} - Gitarr \textsc{(s.~\pageref{a08bf8420208934bc59c7ed7385d4308})}
 Chrille - Trummor
 Johan - Bas

 Det är flera andra som också varit med men de har slutat.

 HEAD2: Diskografi
 \textit{Hardcore attack} kassett utgiven 1984
 \textit{Kärnvapen attack} 7\quotetext{ utgiven 1984
 \textit{Dom ljuger igen} 7} utgiven 2008

 Dom har även varit med på massa, massa samlingsskivor.

}

\small{
\textbf{Mob 47-Åke}
\label{486ee67ac39debabed3d92a7555dcebd}
 Gud.

 Ej att förväxla med gud \textsc{(s.~\pageref{91e49146121c992aab11a19c77e26cf0})}.
 ]]

}

\small{
\textbf{Mobiltelefon}
\label{82c0f48fde46b1501b45b8fc1f821d8f}
 Handikapphjälpmedel för folk som inte kan ropa tillräckligt högt.

}

\small{
\textbf{Modem}
\label{b66ea6cf3f88d28d51f61b7574788808}
 Ett modem är en liten låda som gör att alla 1:or och 0:or som susar runt i kablar och luften förvandlas till roliga bilder på djur i din dator. Utan modem skulle inte hemsidor som Passagen och Blabbermouth ha några besökare och då skulle allt arbete som hårdrockare lägger ned ideellt på att få fram det senaste om Lemmys \textsc{(se Lemmy s.~\pageref{6cc2f8758343439728f308f08a4a8fad})} påstådda inkontinens vara förgäves.

}

\small{
\textbf{Moderat}
\label{c4564b188cb670841733a3ff923c2fb0}
 betyder ursprungligen återhållsam, men i politisk parlans benämner ordet medlemmar av högerpartiet Moderaterna. Det finns dock inte mycket som är återhållsamt med detta gäng storkskurkar \textsc{(se småskurk s.~\pageref{c25031c5d78d9ad6fae8ab8f08d5e9dd})}, kälkborgare \textsc{(s.~\pageref{0f34b469a48952e93688861083ace75a})} och stråtrövare \textsc{(s.~\pageref{49bcf5791fb20ffce4a0150d38dee0ac})}. Moderaterna kännetecknas nämligen av en osund kombination av dåligt tålamod och fullkomlig oförmåga (eller ovilja) till att tänka utanför sammanhang som samlar färre än 8-10 personer. Detta är anledningen till att de inte kan betala TV-avgift, inte förstår sig på skatter, offentlig sektor, klassperspektiv och så vidare, förordar svågerekonomi och glatt påhejar överklassens härjningar runt om i landet.
 HEAD2: Två sorters moderater
 Det finns två sorters moderater. Dessa är:
 \begin{itemize}
 \item De som är genuint onda.
 \item De som är särske \textsc{(s.~\pageref{552a5aad891937bf760fb193900ea140})}.
 \end{itemize}
 Den första kategorin förstår mycket väl att deras politik förstör liv och gör vardagen omöjlig för flertalet människor, men eftersom de är onda tycker de att detta är bra, i enlighet med modern satanism i vilken själviskhet och hänsynslöshet som bekant förordas. Den andra kategorin består av sådana som är så urbota dumma att de inte förstår detta utan på allvar tror att privatiseringar, Corporate social responsibility \textsc{(s.~\pageref{14eb18288303a7408a099b83b5af7f08})}, entreprenörskap \textsc{(se Entreprenad s.~\pageref{2d3b60492ed3cebe0a3cf341bc5b20b5})}, friskolor och allsköns jävelskap \textsc{(s.~\pageref{46845591177f16920cd586a5baf5a625})} är bra för allmänheten. Denna sorts moderater är ofta inavlade Täby-bor eller löst folk som har hjärntvättats medelst ett slags märklig maskin som partiets härförare Fredrik Reinfeldt \textsc{(se Fredrik reinfeldt s.~\pageref{0c16c01849fc86b54e9e0e815490f747})} förvarar i en skokartong i ett mycket högt torn i sin borg \textit{Zhraflindhur} i Mordor, som han kom vandrande ifrån.

}

\small{
\textbf{Moderat likvidations moralkodex}
\label{621d0d857446dbec9c850ec6c4f15594}
 Inbäddat i Malmörockarna Moderat Likvidations texter finns ett tydligt moralkodex. Till skillnad från gängse bild, var (den nuvarande sättningen räknas givetvis inte, då Cliff inte medverkar) inte Moderat Likvidation några nihilister. Nej, de var unga män med strikta ideal.

 HEAD3: Empiriskt material:

 - Låten Anti-fag tar enligt bandet själva avstånd från rökning (\quotetext{fag} är slang för cigarett på brittisk engelska). Inte svårare än så.

 - I Proggrebell räknar sångaren Fjalle upp en serie icke-önskvärda karaktärsdrag som han finner hos proggare, punkarnas antites. Fjalle beskriver proggaren: \quotetext{Du är en proggrebell, knullar med tjejen, gillar blommor och bin. Och dricker rödvin}. Sexuell promiskuitet är alltså inget som högaktas av Moderat Likvidation. Fjalle säger som synes också att rödvin är töntigt. Alltså, man ska inte ha sex och man ska inte dricka alkohol.

 - Köttahuve är en låt om folk som äter för mycket kött. Det ska man inte göra enligt Moderat Likvidation. \quotetext{Köttahuve, stick hem!} som de själva hade uttryckt saken.

 - I Brända celler spelar aggrokvartetten en bit, där texten handlar om hobbyprojekt. \quotetext{Ge mig krök, ge mig lim, så orkar jag en dag till} skriker Fjalle och sprider värme i alla hobbysnickares hjärtan.

 HEAD3: Moderat Likvidations kodex:

 - Ligg inte runt!

 - Rök inte cigaretter!

 - Ät inte kött!

 - Drick inte alkohol!

 - Snickra hemma, DIY!

}

\small{
\textbf{Moderator}
\label{0408f3c997f309c03b08bf3a4bc7b730}
 Moderatkrigare med leveln \textit{Warlord} eller högre. Har hög attack i när- såväl som eldstrid och kan skjuta laser med sitt fruktade (s)kullcrusher-svärd, men saknar helt försvar och babblar istället om att motståndaren önskar ha det som i Gulag.
 HEAD2: Grundvärden
 \begin{itemize}
 \item Att. 10
 \item Def. 2
 \item Mana. 6
 \item Char. 1
 \end{itemize}
 HEAD2: Levels
 Om du använder till exempel \quotetext{parrot attack} för att öka din moderators Mana till +6 kan den levla till \textit{Harbinger}, men om den uppnår -10 i karisma blir den automatiskt Maria Abrahamsson.

}

\small{
\textbf{Moksha}
\label{ce9ad8fab25a32847194eb0e62278ee9}
 (Мокша) är ett språk som talas i republiken Mordovia som omsluts av Ryssland. Språket har 500.000 talare och en Wikipedia-avdelning som är mindre än Nissepedia \textsc{(s.~\pageref{62400dadecd90cb5cd39062abe5a3e4a})}.

}

\small{
\textbf{Moleman}
\label{7bf8ed05e9c4f709665060377d8fc686}
 är en av bikaraktärerna i TV-serien \textit{Simpsons}. Han har stora tjocka glasögon och råkar ofta hamna på fel plats vid fel tillfälle. Hans film \textit{Man getting hit by football} blev utsedd till bästa film på Springfields filmfestival. I ett avsnitt ligger han med Lars Ulrichs farmor. Precis som i fallet med karaktären Rainier Wolfcastle, som är en karrikatyr av kroppsbyggaren Arnold Schwarzenegger, är Moleman också inspirerad av en verklig person. Konkreta bevis saknas än på att förlagan också legat med Lars Ulrichs farmor, men nog kan man skönja vissa drag.

}

\small{
\textbf{Moln}
\label{9da1014bea9aa67f9cae12e619d34aae}
 är anhopningar av vattenbaserad kondens och inget annat. Molnen skapas av att solen värmer upp vatten som på så vis blir varmare än den omkringliggande luften, varpå vatten kondenseras och stiger upp genom atmosfären. En viss del av det kondenserade vatten som molnen består av kommer också från Malå sågverk \textsc{(s.~\pageref{39a99a78876fd85985cc06fa0baa3c1a})}. När molnfronter kolliderar överstiger fuktigheten en viss procent och större vattendroppar bildas, dessa blir då så tunga att de faller ner mot marken. Detta kallas i folkmun för regn \textsc{(s.~\pageref{03456beeae643b4c33b17500a17d1d1e})} och är något helt normalt och inget att hetsa upp sig över. Jag har hört om en stad ibland molnen, men avfärdat förekomsten av en sådan som hörsägen. Vill man veta mer om moln kan man gå till sitt lokala bibliotek och låda International cloud atlas \textsc{(s.~\pageref{9e27ba0dde9e80b43e62c72513b2d534})} där några entusiaster skrivit ned allt som går att veta om fenomenet.

}

\small{
\textbf{Molotov cocktail}
\label{7d215b4a5cc923f86bb4c229af6e2d61}
 s känner ju alla till som växt upp utanför Lundsbergs låtsasvärld. Vad desto färre känner till är föregångaren \quotetext{Molotovs brödkorgar}, som finnarna kallade de klusterbomber Sovjet ovänligt nog släppte över landet och själva kallade \quotetext{matleveranser} när resten av världen frågade vad dom höll på med.

}

\small{
\textbf{Moona röv}
\label{622167eadeb75b63388d3f3ea45ab8a3}
 Att moona röv är ett av de absolut äldsta sätten att driva gäck med sin omgivning. Byxorna \textsc{(se Min bästa byxa s.~\pageref{d713d68db15d469d6e39abacefefb3ab})} halas helt enkelt ned så att röven \textsc{(se praktarsle (positiv) s.~\pageref{ec1f5a634088019acf000718397d0b8a})} tittar fram. Ett lyckat moon resulterar oftast i att den som blir moonad hyttar ned näven \textsc{(se hytta med näven s.~\pageref{dabb9466fffc72b8eec1d4616f32d62e})}, medan resten av åskådarna skrattande tittar på, och då har moonaren fått sin belöning. Nyckeln till skämtets popularitet är att det har så få variabler att det i princip inte kan utföras fel. Komikern Adde Malmberg \textsc{(s.~\pageref{1390facdddaee5ed00a964fbe93b30b9})} har exempelvis uteslutande kört denna repertoar de senaste 15 åren.

}

\small{
\textbf{Mosaiska lagen}
\label{ec63c79de03c8d40017c6841d80ebe30}
 är någon form av proportionsriktlinjer för hur en moussaka bäst tillagas. Om man följer mosaiska lagen och inte fullkomligt förlorar fokus under resans gång så utlovas ett delikat resultat.

 Vissa påstår dock att moussaka, hur väl tillagad den än är, är jätteäckligt. Man skulle kunna kalla dem för undantagen som bekräftar lagen

}

\small{
\textbf{Moses Hightower}
\label{37077c4fe659ccfaa14c8ec9723bedc4}
 Kadett, senare Konstapel, senare Sergeant Moses Hightower är en karaktär i Polisskolanfilmerna. Han är jättelång och jättestark och har en mäktig röst. Trots alla dessa egenskaper jobbade han innan polisutbildningen som, håll i er, florist!

}

\small{
\textbf{Motoruv}
\label{de6d9e6951e530b768e9fbbc39302515}
 En motoruv är en uv \textsc{(s.~\pageref{45210da832f9626829457a65e9e7c4d0})} som försetts med motor för att gå snabbare. De finns med upp till 17 hästkrafter \textsc{(se häst s.~\pageref{b4c608370b339da095c5f8db7fab0945})}.

}

\small{
\textbf{Motpåven i Gränna}
\label{35d44c815567a4c3ba5fdba1ab1cec21}
 Efter att påve Benedictus XVI meddelat sin avgång, den 11 februari i herrens år 2013, uppstod en schism i den kristna världen. I Vatikanen verkade allt vara business as usual - en ny påve skulle tillsättas, och kardinalerna lovade att han också skulle vara nazist, precis som den förra. Där kunde allt ha varit frid och fröjd, men i omvärlden kokade missnöjet. Den katolska kyrkan framstod för resten av världen som försvagad. För första gången på 800 år hade en påve frivilligt avgått. Vad sänder det för signaler, när killen just under Jesus i den kristna hierarkin inte tycker att det är så kul längre, och hellre startar en kombinationsaffär \textsc{(s.~\pageref{54328b839527f9917e5d057845b4fc5c})} eller nåt?


 Skolastiker skyndade till biblioteken för att finna prejudikat för motaktioner mot den romersk-katolska överheten. Och de som sökte, skulle också finna. Inom den katolska tron finns en stolt tradition av så kallade motpåvar, eller antipåvar. De är män som vägrar acceptera att de inte är minst lika balla som han i Vatikanen, och utser sig själva till påvar. Under historiens gång har en mängd motpåvar existerat, där de med starkast anspråk på riktig makt var de som tillsattes i Avignon, av den franske kungen Filip IV på 1300-talet. Nu för tiden är motpåvarna en sorglig skara sektmedlemmar, geografiskt avlägsna från maktens kärna. Men det skulle komma att förändras, 2013 i Gränna.


 Varför står religionen utanför marknaden, frågade sig många kristna svenskar? Är inte förfallet inom katolicismen ett tecken på att det blivit för slappt rent konkurrensmässigt i den själsliga världen? I Gränna grunnade Alf Svensson, svensk kristendoms grand old man, tillsammans med sina rådgivare, över dessa frågor. Till slut fattade han ett beslut. Den 14:e februari 2013 utropade sig Alf Svensson till påve, under namnet Albion Venerabilis I. Runt sig hade han en konklav nytillsatta kardinaler, bestående av Marcus Birro och Paolo Roberto. Tillsammans skulle de utmana Vatikanen. Budskapet var att uppvärdera påvedömet, fixa nya kläder, bättre pensionsförmåner, lite mer sex appeal. Regeringen Reinfeldt uttalade sitt aktiva stöd för initiativet, då det bara kunde innebära en frisk fläkt med lite gammal hederlig konkurrens i det där murkna örnnästet. En dag efter sitt tillkännagivande hade Albion Venerabilis I både hunnit erhålla en bannbulla från Vatikanen, sms:at en egen tillbaka, och meddelat sin medverkan i melodifestivalen 2013. Under deltävlingen i Skellefteå, i Sveriges enda sanna bibelbälte, skulle Albion Venerabilis I avslöja sina budord. Paulo Roberto skulle spela ackegura och Marcus Birro hade skrivit text. Spänd förväntan följde.

}

\small{
\textbf{Mount Everest}
\label{b5b5d890ef4ff008c7821da350799545}
 är med sina 8848 meter över havet jordens högsta berg. Namnet Mount Everest är dock kattpiss i jämförelse med det tibetanska namnet \quotetext{Jo-mo glang-ma ri} som översätts till det oerhört psych-iga och övermäktiga \quotetext{Universums moder}. Typiskt att nån jävla britt \textsc{(se George Everest s.~\pageref{36c3305a3fc7dbde2cdc868b00cc1af4})} döpte det efter sig själv istället.

}

\small{
\textbf{Multikulti}
\label{25eea9148080d30d384ce1c1277ef126}
 Innebär i korthet att blanda saker som inte hör ihop på ett smaklöst sätt.
 Kvinnor i begynnande medelålder har en förkärlek för detta smaklösa michmach och suckar lyckligt varje gång de får bevittna ett \quotetext{spännande möte mellan olika kulturer.} Det må vara pyttipanna med chokladsås. Det må vara jazzmusik på panflöjt \textsc{(s.~\pageref{ce107b52f922e556b394fa5303dc6b0f})}. Det må vara vad skit som helst som inte passar ihop.

}

\small{
\textbf{Mun}
\label{6585f290ce92c3de5ff339920330e26f}
 En mun (alt. stav. \quotetext{mund}) är ett hål i ansiktet. Man hittar den nedanför näsan \textsc{(se näsa s.~\pageref{eb02670054310d89c985dfe12c3ba7b8})} och ovanför  hakan \textsc{(se haka (vanlig)  s.~\pageref{3b8edf3dc8968e6b2805dc512af3b68c})}. Man använder munnen till att tala, äta, vissla, spela dragbasun \textsc{(s.~\pageref{0315aaaabb57a67312aa3316fd2006e1})}, pussas och snusa. Det enda dåliga man gör med munnen är att räva \textsc{(s.~\pageref{c13f687883c1eb0be3be218fff63e6b8})}. Munnen bör rengöras morgon och kväll om man vill undvika fetor ex ore \textsc{(s.~\pageref{d3b96d618fb972d12fb0cdfdeaf13a98})}. Vissa målar munnen röd, medan enstaka människor uppgraderar den till näbbmun \textsc{(s.~\pageref{9e3395be14cf14f92e8cd1e93eb7599b})}, men detta är ganska onödigt tycker många.
 HEAD2: Exempel på munnar
 8)
 8(
 8D
 8I
 8s
 8/
 8\
 8*
 8w (om man \textsc{(s.~\pageref{39c63ddb96a31b9610cd976b896ad4f0})} har näbbmun) \textsc{(se näbbmun s.~\pageref{9e3395be14cf14f92e8cd1e93eb7599b})}

}

\small{
\textbf{Mungo Jerryhatare}
\label{31191f71efcd2dbcc9e02a0ce88c5943}
 är människor som anser att den brittiska musikgruppen Mungo Jerry är den sämsta skit som hänt världen sedan Adolf Hitler. Dom tycker att \textit{In the summertime} är så fantastiskt värdelös att lobotomi plötsligt inte verkar så omodernt när dom hör den. Det mest provocerande, tycker Mungo Jerryhatarna, är att Mungo Jerry tack vare låten fått släppa mer än 20 skivor till av ren skit. Varför kunde inte medlemarna nöjt sig med att leva på STIM-pengarna från all radiotid låten får? Måste de dessutom turnera år efter år och spela okända låtar i 80 minuter för att sedan köra \textit{In the summertime} som andra extranummer. Mungo Jerryhatarna blir helt tokiga när dom tänker på detta och känner för att skjuta alla som heter Mungo i förnamn.

}

\small{
\textbf{Muno}
\label{47685c52bd097e4b2845db5c63047551}
 är en röd skabbräv som bor vid mynttorget i Norberg. Det är han som ser till att alla bussarna kommer och går i tid. Han lever på korv och mos från Myntgrillen och är allas vän.

}

\small{
\textbf{Musikhögskolemusik}
\label{28e79281eebe4f13098c88a3e51b6b8e}
 Folk som spelar musik på musikhögskola gör inte det för att dom gillar musik eller ens är intresserade av musik. Vad dom då gör där vet dom inte ens själva så tiden fördrivs genom att bilda olika lösa konstellationer inom genren Musikhögskolemusik.
 HEAD3: Säkra tecken för att identifiera musikhögskolemusik:
 \begin{itemize}
 \item Basisten har en bas med fler än fyra strängar.
 \item Gitarristen bär sin  fullkittade \textsc{(se fullkittad s.~\pageref{e66a645b2ba13a48aa09d73519fcdaa1})} gitarr strax under hakan.
 \item Matilda är en trevlig tjej, så man har flöjt \textsc{(s.~\pageref{15912990d2886bb707954c9a4e933bc0})} med.
 \item Stor vikt läggs vid avancerade arrangemang, framförande åsidosätts fullständigt.
 \item Bandnamnet innhåller något i stil med
 \begin{itemize}
 \item \quotetext{project}
 \item en sifferkombination
 \item intern humor
 \item Om man mot all förmodan spelar inför publik soundcheckar man i tre timmar och gnäller på allt tekniskt som går att gnälla på från telekablar till belysningen på toaletten.
 \end{itemize}
 \end{itemize}

}

\small{
\textbf{Musselini}
\label{6b19e50717b8e033d311707b4fa69f9d}
 (eng. \textit{Mickeylini}) var en karaktär skapad på Walt Disneys initiativ år 1938, efter att Il Duce samma år infört raslagar i Italien. Då den senare inte stod i någon högre kurs hos allmänheten i USA tyckte anti-semiten Disney att en charmoffensiv kunde vara på sin plats där den Italienska regimen genom karaktären på helsidor i olika serietidningar förklarades vara en bra politisk förebild. Italienska propagandaministeriet avsade sig alla kopplingar till kampanjen för att inte de diplomatiska förbindelserna med USA skulle påverkas negativt, men under bordet postades bilder på Il Duce till Disneys studios för att underlätta arbetet. Italienarna ville inledningsvis att deras slogan \textit{Libro e moschetto — fascista perfetto} (\quotetext{Bok och musköt - den perfekte fascisten})  även skulle figurera i kampanjen men detta avslogs av Disneys kampanjansvariga. [http://www.amazon.com/Under-Mussolini-Decorative-Propaganda-Collection/dp/8820215772/ref=sr_1_2?ie=UTF8\&s=books\&qid=1272202657\&sr=1-2]

 Då kampanjen inte rönte någon större framgång återvände Disney redan år 1939 till att bädda in smygrasism och konservativa ideal i sina större produktioner, sin anti-semitism lät Walt ligga i skymundan fram till sin död 1966.

 På sistone har extremhögern inspirerats av Disney och försökt med liknande propagandaknep i samma anda för att deras förebilder skall nå ut till en yngre skara av potentiella rekryter.[http://www.youtube.com/watch?v=KTu7YqOz-9k]

}

\small{
\textbf{Mustasch}
\label{78fe8e02985abb5090cb3f33ac2842d4}
 Det som sållar agnarna från vetet.

 Om en vill ha högre lön är det bra att ha en sådan.

}

\small{
\textbf{Muttersvarvare}
\label{78ccc210543eb2a008f7999e1f1f905b}
 Nedvärderande benämning på den svenska uppfinningen skiftnyckeln \textsc{(se skiftnyckel s.~\pageref{4eeef21c5cf8813ae82d2882f54c8e28})}. Alla vet att det är fasta nycklar \textsc{(s.~\pageref{ad577d76d7747bfd314d442197fc8587})} som gäller.

}

\small{
\textbf{Myntsamleri}
\label{e2819396b2fc6c24db2e11971493ff87}
 Få men hängivna är de som ägna sig åt myntsamleri ädla konst. Många tror motsatsen, då livet i västvärlden, som du kanske noterat, i mångt och mycket går ut på att skaffa sig pengar - antingen genom förvärvsarbete eller genom att kanalisera pengar som andra skapar genom förvärvsarbete ner i sin egen maffiga hästhandlarplånka \textsc{(se hästhandlarplånbok s.~\pageref{2f8fbda5296f2f6cab04d88082ed9015})}. Detta är dock inte myntsamleri, utan en helt andra företaganden, vars uppslagsord är löneslaveri respektive entreprenörskap \textsc{(se entreprenad s.~\pageref{2d3b60492ed3cebe0a3cf341bc5b20b5})}. Myntsamleriet går dock också ut på att samla pengar, men dessa måste, för att det ska röra sig om myntsamlande, vara oanvändbara som valuta \textsc{(s.~\pageref{cf1e2a0af4955aa7539b6e12e9d282e6})}. Hittar myntsamlaren ett värdelöst mynt blir han eller hon alldeles till sig och tar fram en pincett med vilken myntsamlaren varsamt flyttar det till en för ändamålet avsedd bok som han eller hon sedan trycker mot sin barm av glädje och själslig salighet.

}

\small{
\textbf{Myrstackarna}
\label{4510b93a9b9e44f51a730d04a8cf30a9}
 Myrstackare (singularis) Myrstackarna (pluralis)
 Inom svensk fauna välkänt begrepp på de myror som lever utanför den stora myrstacken. Vanligast är att de förvisats p g a dispyter med drottningen, eller att de levt ett liv fyllt av oegentligheter. Myrstackens lagar och förordningar är inte alltid så lätt att förstå för enskilda individer, och orsakar många gånger ett arbetsamt liv med många frågetecken. Den enskilde myrstackaren har inte en kotte att rådfråga och väl nere i de mörka gångarna ser han inte ett barr, därför väljer många självmant att flytta till förortens mindre samhällen, uppbyggda av ett friare umgängesklimat med nästan obetydliga förordningar.
 Det rapporteras att några myrstackare dykt upp på sidan 4 i Västerbottens Folkblad på lördagar, men det är ännu obekräftade uppgifter.

}

\small{
\textbf{Myskoxe}
\label{dcf78e29a42cfbae34e73e0e1559e7c9}
 Myskoxar är ett djur som är väldigt utrotningshotat i Sverige \textsc{(s.~\pageref{b1999637949ed135b2ca03f3a38460cc})}. Ibland finns det hela sju individer i Sverige, och ibland finns det inga alls! Det här beror på att hjorden promenerat över till Norge.

 HEAD2: Taxonomi
 Carl von Linné \textsc{(s.~\pageref{5e8380bf6b7ce99678e6752b6d9e709e})} döpte Myskoxen till \textit{Ovibos moschatus} som betyder \quotetext{den mystiska oxen} på latin. Det mystiska ligger i att de inte säger så mycket och är svåra att kommunicera med. von Linné försökte fråga oxarna varför de höll på som de gjorde, men ingen svarade. Mystiskt.

 HEAD2: Föda
 De äter lite vad som helst som finns i de fjälltrakter de lever i, mest lavar och mossor. Ibland äter de ingenting alls under långa perioder, för att någon skrattat åt dem och kallat dem feta. Det här är skitprat då alla myskoxar vet att de bara har en massa päls och späck för att skydda mot kylan.

 HEAD2: Beteende
 Myskoxar lufsar mest omkring och äter, sover och parar sig som alla djur gör. Men ibland avbryts denna annars så trevliga vardag av ett rovdjursangrepp och då blir myskoxarna jävligt arga. Kommer en vargflock och försöker käka upp myskoxkalvarna ställer sig de vuxna djuren i en ring runt de små med hornen utåt och stångar skiten ur varje varg som försöker med nåt. Denna mur av horn är helt ogenomtränglig för alla  rovdjur men en del, som t.ex. männskan, har lärt sig använda skjutvapen och använder dessa för att göra processen kort med oxarna. Det här bör ses på som fegt och alla jägare borde istället gå till angrepp med spjut eller nåt annat primitivt för att besegra denna skapelsens krona. Myskoxar som lever i fångenskap hatar människan av denna anledning och tar därför varje tillfälle de får att stånga ihjäl dom. Tack och lov finns staket som istället får utstå detta våld.

}

\small{
\textbf{Myspedofiler}
\label{2cb4b0ffec59535759020365fd8c7cd1}
 håller föreningslivet igång. Äldre män som gillar att umgås med unga gossar utan att för dens skull kliva över gränsen är ryggraden i de folkrörelser som skänker oss så mycken glädje.

}

\small{
\textbf{Myspyssockerkaka}
\label{e60a4b5830b6a6836f1b30c0642c88b0}
 är en de bakverk vars recept ingick i det klassiska häftet Vegankokboken \textsc{(s.~\pageref{8c52672f38e80fd29ac8dbc6dbc47008})}, författad av Kristoffer \textsc{(s.~\pageref{b1bebf7f19c7345d261dd0f1f7746f00})} Åberg. Kakan innehåller bland annat vetegroddar och är perfekt när man vill \quotetext{myspysa}, det vill säga fisa ut nedbrytningsgaser[http://youtu.be/v8GgGq5YipQ] i glada medelklassvänners \textsc{(se medelklassvänner s.~\pageref{bbf0e28d1f8015c14119873cc4a518ae})} lag.

}

\small{
\textbf{Mystiska band}
\label{fcb2f458be772f1df5e5246cec50c9a9}
 Vissa band väljer i början av sin karriär att uppträda förklädda för att på så vis skapa en hype kring sig själva. Ingen vet vilka de är. De som är coolast i scenen har sina aningar men säger inget till oss vanliga töntar. Ofta hittar bandet också på skojiga rykten om sig själva, som man låter sprida så att folk ska tycka att man är tuff.
 HEAD2: Exempel på band som är eller har varit mystiska
 \begin{itemize}
 \item SunnO)))
 \item Lordi
 \item Onkel Kånkel
 \item Final Exit
 \item Slipknot
 \item Goat
 \item Syphilitic Vaginas
 \item Ghost
 \item Rednex
 \item Hyenaz in the desert
 \item Antirep \textsc{(s.~\pageref{eb508922f79a60f76b0278edfea25a8c})}
 \end{itemize}

}

\small{
\textbf{Myt}
\label{d05f285ddc99f57480177b78a759aa6c}
 En myt är en historia som inte är sann, den har likheter med skröna \textsc{(s.~\pageref{c51cd220359f9f2755e98dcce2251e5c})} och det har egentligen ingen som helst betydelse vad skillnaderna är eftersom de båda rätt och slätt bara är hittepå eller bluff med i bästa fall en smula sanningshalt.

 HEAD3: Exempel på Myter:
 Människan landsteg på månen 1969.

 Olof Palme \textsc{(s.~\pageref{702b78623785546fb9c9890222376178})} sköts av en ensam galning.

 Man får kramp och dör om man badar inom 2 timmar efter att man ätit mat.

 Killen med brytaren i ask-koppen på sin epa.

 Om man kör 200Kmh så kan man se sin egen bil i backspegeln.

 Kör man i 400Kmh så möter man sig själv på tillbakavägen.

 Det är nyttigt med fisk.

 Frukt är godis.

}

\small{
\textbf{Mytoman}
\label{3ce225af6d01a55b418d7d83b938f26e}
 En mytoman är en person som inte kan hålla sig till sanning, en mytoman talar på ett vilseledande sätt ärligt om sina problem och skryter gärna om sina luftslott och tillgångar som inte existerar.

 Är man mytoman så har man allt fast man inte har någonting, man är egentligen en väldigt lycklig människa som är olycklig. Tror man på en mytoman så är man en såkallad mytorolis som är latin och betyder Lättlurad god vän.

 Tror ni mig inte så ljuger jag ännu mer.

 Mytomaner har som system att vilseleda dig för egen vinning.

 En mytoman kan tex säga till dig att han/hon väntar in 150 000Kr på sitt konto från en överföring från riksförbundet för motionärer men behöver låna 40 kronor av dig för att kunna ta bussen ner till banken, eller kanske ännu lämpligare - 120 till en taxi.

 En mytoman skriver kvittot i snön och låter solen vara vittne.

 Mytomanens favoritslogan: Tror ni mig inte så ljuger jag ännu mer.

 Källa: Ordförande i mytomanklubben.
 Tror ni mig inte så ljuger jag ännu mer.

}

\small{
\textbf{Mäklarbricka}
\label{818c8a106ef37f756992c962950fd74b}
 En mäklarbricka är en bricka \textsc{(s.~\pageref{0072a4ca9825dbec7dfa6e6cb9b23022})} med vodka \textsc{(s.~\pageref{234eaecc0efd4460dfeb921a50feea08})} och groggvirke \textsc{(s.~\pageref{ba264d4eb820b4066de4c8723a08f824})}, samt en eller två champagneflaskor och är en populär beställning/klassmarkör \textsc{(s.~\pageref{6a9c0c6836a0777442468f821837e795})} på krogar runt Stureplan i Stockholm. Mäklarbrickan delas sedan med folk runt det bord vid vilket beställaren sitter, speciellt om bordsgrannarna är damer och beställaren herre. Mäklarbrickan säger som inget annat \quotetext{jag har pengar,} \quotetext{jag har social status,} \quotetext{jag har en lätt utvecklingsstörning.}

}

\small{
\textbf{Mäklarsvenska}
\label{af79a693fcf194b09982947ae3fa07b0}
 Ett yrkesspråk som kan liknas vid knoparmoj eller, syftet är att utomstående inte skall begripa innebörden. Omfattande Nissepediaforskning har uppenbarat några glimtar.
 \textit{\quotetext{Äldre ytskikt}}- Du måste byta panel, antagligen köksinredning också.
 \textit{\quotetext{Tillfälle för den händige}}- Vattenskada.
 \textit{\quotetext{I populära.....}}- Det kan bli svårt att sälja detta hus.

}

\small{
\textbf{Män är djur, tycker inte du det?}
\label{a064fbffd5eda4ac5320b7c565c6f549}
 \quotetext{Män är djur, tycker inte du?} \textsc{(s.~\pageref{7b3f13fdbf56a65af8ed05e40a8259bc})}

}

\small{
\textbf{Män är djur, tycker inte du?}
\label{7b3f13fdbf56a65af8ed05e40a8259bc}
 \quotetext{Män är djur, tycker inte du?} är ett citat sagt i en dokumentär om kvinnojouren i Umeå och används ofta av anti-feminister (eller jämställdister som de själva vansinnigt nog kallar sig) för att påpeka att alla som menar att män skulle hålla på med tvivelaktigheter (eller rättare sagt, att den rådande maskulinitetskulturen i samhället innehåller problematiska element) är helt skogstokiga. För övrigt belyser reaktionerna på citatet mer än något att nittiotalet är över och med det ironins död.

}

\small{
\textbf{Mänsklighetens historia}
\label{5d87ba4132f8bdfa8c6294514c570c3f}
 är historien om de som har och de som inte har, och hur de som inte har vill ha mer. De som har är dock ena riktigt giriga jävlar, så detta är något komplicerat. Låt Nissepedia \textsc{(s.~\pageref{62400dadecd90cb5cd39062abe5a3e4a})} upplysa dig!


 HEAD2: Stenåldern, mänsklighetens vagga

 Mänskligheten började när aporna klättrade ner från träden och bestämde sig för att göra något bättre på dagarna. Då började det som idag kallas stenåldern. Människorna softade i grottor och krubbade på det som fanns tillgängligt, fanns det inget tillgängligt så var det kört. För att människorna skulle bli lite mer sofistikerade än de bajskastande bestar som de härstammade ifrån behövdes prylar, t.ex. stenyxor, flintaknivar och annat man kan göra av stenar. Vad den ack så finurliga människan gjorde var att bearbeta naturen. Hon började alltså att arbeta. Det sociala klimatet under den här tiden antas ha varit ganska soft.

 Och tiden gick...

 HEAD2: Agrikulturens födelse

 Sen slutade människor bo i grottor och kladdade ihop ler och långhalm och började softa i hyddor. Nån smart en kom på att om krubbet fanns tillgängligt hela tiden så skulle det inte köra ihop sig lika ofta. Nån spillde antagligen vete eller nån annan växt i jorden utanför, och vips var jordbruket uppfunnet. Då kunde folk på något outgrundligt vis baka bröd och göra varma mackor, den softaste av maträtter.
 Det sociala klimatet var fortsatt soft.

 Och tiden gick...

 HEAD2: Feodalismen

 På något mystiskt vis uppfanns handkvarnen och plogen och då började en viss typ av människor tycka att en annan typ av människor skulle arbeta på deras åkrar. Människor slutade vara softa bönder \textsc{(s.~\pageref{30a6fc00c9102680b8196b1b79935ec4})} och blev godsherrar och trälar. Godsherrarna hade det ganska soft och trälarna hade det ganska, ja, träligt. Godsherrarna brukade säga till trälarna att de skulle beskydda dem om nån annan godsherre skulle komma och bete sig osoft och krigas. Det kan verka ganska präktigt av godsherren att göra det här, men alla godsherrar var polare och skulle aldrig kriga på riktigt, utan gjorde det här för att lura trälarna.
 Det sociala klimatet kan sammanfattas med att godsherrarna åt varma mackor med nån riktigt god och illaluktande ost, medans trälarna åt blöta frallor med hushållsostkanter.

 Och tiden gick...

 HEAD2: Industrialismen

 Någon kom på den briljanta idén att det behövdes maskiner för att göra prylar. Det här kallas idag för industrialiseringen och är ursprunget till kapitalismen. Istället för att folk kunde softa och ta hand om varandra skulle de gå till fabriker och göra användbara saker. I teorin är det väl inget fel på det kan en tycka, för det går att göra skitfräna prylar typ elgitarrer och hoppborgar. Problemet är bara att prylarna ska säljas så att personen som säger sig äga fabriken och maskinerna ska få stålar. Av de här stålarna ser inte de som stått vid maskinerna och gjort prylar särskilt mycket så de får bo i baracker och äta gröt. De som säger sig äga fabrikerna och maskinerna, kapitalister eller bruksdisponenter eller saft- och cidermagnater eller va fan det nu kallas, får svinmycket stålar och kan bo i herrgårdar och äta mat som slutar på bokstaven é, typ filé och paté.
 Det sociala klimatet var då, som ni kanske fattar, raka motsatsen till soft.

 Och tiden gick...

 HEAD2: Det senmoderna kapitalistiska samhället

 Nu är vi framme i nutiden, i det senmoderna samhället \textsc{(s.~\pageref{1a4c3b1112bd2b510a8c47eff69397b8})}. Här kostar allting \textsc{(s.~\pageref{2ea7603b8880ffdf729128008f5d252d})} pengar och alla måste arbeta med nånting för att ha råd att köpa minsta pryl eller tjänst. Kapitalisterna, bruksdisponenterna och saft- och cidermagnaterna ser till att folk betalar för att ha någonstans att bo, mat på bordet och en ens lite dräglig tillvaro. Kapitalisterna gillar att vi köper saker så därför ser de till att det finns en massa fräcka prylar som kostar en massa stålar, t.ex. skaljackor i Gore-Tex \textsc{(s.~\pageref{cc1e3b66bcda33218427995652e4e31a})} eller cyklar med 21 växlar \textsc{(se Flerväxlad cykel s.~\pageref{cd75a1ec5d4b7caabeaaaf25edee0250})}. Prylar är som sagt fett, men det kostar stålar och därför måste vi arbeta med nånting. I en del av världen så görs inga prylar längre utan folk har låtsasjobb som bloggare, akademiker \textsc{(s.~\pageref{05c87f09f25afaf9f129a84230ab1008})} och spelutvecklare. Istället så ser kapitalisterna till att prylarna görs på ställen där arbetare får betalt i gröt och blöta frallor, alltså där världen ser ut som i den tidigare epoken. Kapitalisterna gör så här för att de ska kunna åka på resor i rymden, köpa tavlor som nån döing målat eller bada i diverse jästa drycker, och sånt kostar svinmycket stålar.
 Sammanfattningsvis så är ingenting soft med kapitalismen. Säg en dålig sak och det är kapitalismens fel, garanterat. Skulle det vara så att du kommer på en grej som kan verka ganska soft med kapitalismen så skulle det kunna vara så mycket bättre om ingen riking skulle ha vinst för den softa grejen.

 HEAD2: Framtiden

 Många skäggiga män har i sina dagar siat om vad som komma skall, t.ex. Nostradamus eller Rasputin. Till den här kategorin skäggiga män hör inte Karl Marx, vars ande spökskrivit denna text. Han ville inte skriva recept för framtidens soppkök, vad som nu menas med det. Vad som dock är säkert är att något måste hända. Snart. Gärna imorgon. Detta för att det dialektikiska \textsc{(se dialektik s.~\pageref{5c0ded4e9796ad82ecd11d1a0010bf6b})} förhållandet mellan kapital och arbetare kommer att syntetiseras och skapa något annat. Vad det blir är oklart, men annorlunda, det lär det vara. Vänta bara.

}

\small{
\textbf{Märkliga citat}
\label{25bc76cd6d1dc844459c060c414d6b12}
 är mindre textstycken som uttalats av en person på ett, för andra, förvirrande \textsc{(se förvirring s.~\pageref{c502a6223b16f730a8900c12f2b10fec})} sätt. Ofta är det dock inte så dramatiskt som det verkar. Det kan till exempel vara så att citatet plockats ur sitt samanhang och därför låter helt orimligt. Många politiker hänvisar till denna förklaring och alla minns vi väl Prof. Etiennes \textsc{(se Användare: Prof. Etienne s.~\pageref{a9878d2280e5a39becac8f73d113df91})} förvirrade uttalande om att ”Jag ska göra kräftbete \textsc{(s.~\pageref{76499bc9cc050bed2beb8e36dd601066})} av vartenda jävla tamdjur i kommunen!” under ett torgmöte i sin kampanj för att bli borgmästare i Ödeshög. Lokaltidningen gjorde först en stor affär av detta men fick sedan föra in en dementi när professorn hotat redaktionen till livet och förklarat att han faktiskt var full \textsc{(se jag ska bara bli full först s.~\pageref{66ecf23dc53d37f0509f569153ddbb6f})} när uttalandet gjordes. Så fel det kan gå om man saknar kontext.  Poesi kan också framstå som väldigt märkligt för många, men ofta känner man då inte till att texten kommer från en hårdrockslåt där vitsen inte är att det ska låta rimligt utan bara tufft som fan. Men sedan finns det faktiskt också en beskärd del uttalanden där ingen kontext i världen kan få raderna att framstå som något annat än genuint märkliga.
 HEAD2: Märkliga citat
 \begin{itemize}
 \item Är det här Smögerrak? /\textit{Anonym basis i svenskt punkband}
 \item Alla är unika utom du. /\textit{Anonym professor \textsc{(se Användare: Prof. Etienne s.~\pageref{a9878d2280e5a39becac8f73d113df91})} }
 \item En slemmig torsk i en brödrost. /\textit{Micke Alonzo \textsc{(s.~\pageref{4bd27d3cf2641236f956496c779a0dc2})} }
 \item Varför ska jag må dåligt när inte Roger Moore dåligt? /\textit{Dirtpunkaren \textsc{(se träskpunkare s.~\pageref{484838b3db1adb135ea74d6fc61e44c0})} Ramone}
 \end{itemize}

}

\small{
\textbf{Märkliga sammanträffanden}
\label{f46282d99158f351a81b9deaff157b4e}
 är när två \textsc{(se tvåa s.~\pageref{84fcc0494ecf9f5af79fcd9bed184a9a})} eller flera saker på något sätt hänger ihop, till synes utan rimlig anledning. Finns en rimlig anledning är det bara ett vanligt sammanträffande. Nya märkliga sammanträffanden inträffar hela tiden men det gäller att vara uppmärksam för att upptäcka dem. Ser man ett bör man rapportera det snarast till berörd myndighet.

 HEAD2: Exempel på märkliga sammanträffanden
 \begin{itemize}
 \item Maud Olofsson \textsc{(se Maud olofsson s.~\pageref{eb913a2e9be929654908a05017401bd6})} deklarerar att hon ska lämna offentligheten dagen innan Clarence Clemons dör \textsc{(se döden s.~\pageref{6f3c270eb5b4d979c777b4ec26dd106f})} under mystiska omständigheter.
 \item Många chefer kallas för \quotetext{Adolf}.
 \item Det är mest idioter som lyssnar på dålig musik.
 \item Olof Palmes \textsc{(se Olof Palme s.~\pageref{702b78623785546fb9c9890222376178})} gata ligger nästan precis där han blev mördad.
 \item Göran Greijder och King Buzzo i Melvins har samma frisyr.
 \item Bärsen smakar bäst fredag \textsc{(s.~\pageref{80d41f1e0b14eacb229eea9618632e88})} eftermiddag.
 \item Många artiklar här på nissepedia \textsc{(s.~\pageref{62400dadecd90cb5cd39062abe5a3e4a})} har att göra med uvar \textsc{(se uv s.~\pageref{45210da832f9626829457a65e9e7c4d0})} och uvrelaterade \textsc{(se uv s.~\pageref{45210da832f9626829457a65e9e7c4d0})} saker.
 \item Nissepedias niohundraelfte artikel handlar om Porsche 911 \textsc{(s.~\pageref{8750acbc992e40a29eeca5c71a21d4b8})}.
 \item Oljegarker \textsc{(se Oljegark s.~\pageref{d78fbbc214d52206f58476f02f66f0b6})} som delar ekonomiska intressen med Carl Bildt (M) brukar normalt inte stöta på patrull i svensk utrikespolitisk.
 \item Den amerikanske skådespelaren och skämtaren David Koechner är i princip identisk med Ulf Larsson, svensk skämtare, skådespelare och före detta programledare för \textit{Söndagsöppet}.
 \item Timbros VD heter Staffan UVell.
 \end{itemize}

}

\small{
\textbf{Mästaren på karate}
\label{b7e7db272d1a1ee2193cbac1acc89d47}
 \textit{'Shang-Chi's händer, Mästaren på karate}' (\textit{The Deadly Hands of Kung Fu}) var en serietidning om kampsportsspecialisten Shang-Chi som står i dialektikskt \textsc{(se dialektik s.~\pageref{5c0ded4e9796ad82ecd11d1a0010bf6b})} förhållande till sin onde far, Fu Manchu. Tidningen kom ut ett tag på sjuttiotalet, blev ingen braksuccé \textsc{(s.~\pageref{678371d35369d3d29afceb1445630833})} och lades ner, bara för att komma tillbaka med buller och bång på nittiotalet. Men då, som så mycket annat på nittiotalet, i färg!

}

\small{
\textbf{Måg}
\label{55752d6920060b54fd689faee4ed037b}
 En måg är en man som envist framhärdar med att han på något krångligt vis är din ingifta släkting. Ofta kräver mågen saker av dig, inte sällan är det nåt slags jovialiskt sällskap han är ute efter. Han pratar med dig om takisolering eller segelbåtar när han ser dig på släktmiddagar, medan du i ditt stilla sinne tänker \quotetext{vem är du och vad gör du här?}

}

\small{
\textbf{Målvakt}
\label{4df849b3692970a531db4cc1d7264c94}
 En målvakt är en idrottsutövare som hatar bollar och puckar. Så fort en boll eller puck kommer nära målvakten blir den förbannad och gör allt den kan för att schasa bort eller fånga in föremålet. Väl infångat tar målvakten i allt vad den kan för att slänga bort bollen/pucken till en annan målvakt, så det är inga direkta vänskapsband kollegor emellan. Tekniken varierar inom olika sporter men vanligast är att målvakten använder händer och fötter för att få vara ifred. Det avgrundsdjupa hatet mot allt bollformat bottnar oftast i att målvakten var tjock som barn och brukade bli retad genom att kallas boll- eller puckformat pucko.

}

\small{
\textbf{Måndag}
\label{1086ddd192a30419d01e5c28b74cab2f}
 är veckans första och kanske sämsta dag (vissa menar, med starka argument, att söndagen är sämst) samt en låt av rockgruppen Stockholms Negrer, i vilken Micke Alonzo \textsc{(s.~\pageref{4bd27d3cf2641236f956496c779a0dc2})} stog för sången.

}

\small{
\textbf{Mångsysslarpensionär}
\label{ce651324111b616e98f210ea8511ce75}
 En mångsysslarpensionär är en person som på grund av sin ansenliga ålder inte arbetar på riktigt utan bara på låtsas. Mångsysslarpensionären fyller sin tillvaro med ett stort antal ofärdiga projekt, så som båtrestauration, mattväveri, nedtecknandet av hembygdshistoria, låtskriveri, hembryggeri, dagdriveri och myntsamleri \textsc{(s.~\pageref{e2819396b2fc6c24db2e11971493ff87})}.
 HEAD2: Mångsysslarpensionären i kulturen
 Vissa har i J.R.R Tolkiens \textsc{(se J.R.R Tolkien s.~\pageref{3f0b7fcbd9fa7369ca314a46c280b67e})} karaktär Gandalf sett själva ärketypen av en mångsysslande pensionär.
 Andra omnämner Lasse Åberg \textsc{(s.~\pageref{52700e0748c7b425d15a4cb4f342389f})} som en typisk sådan.

}

\small{
\textbf{Mårten}
\label{45aa782921c38b69d6d11e7d7680afe8}
 är ett gremanskt mansnamn och är en böjning av den engelska frasen \quotetext{more tea?}.

}

\small{
\textbf{Mås}
\label{04f599c35052d2060c70cb99b09f94dd}
 Du vet det där jävla fågelfät som förr eller senare dyker upp när man är ute med sina polare i skärgården för att få sig lite välförtjänt lugn och ro och som gapar och lever om något så jävligt att man funderar på att åka hem igen, men det kan man inte för man har redan åkt så långt att då skulle de kännas som att man har gjort så mycket förgäves - handlat, fixat soppa till snurran, kollat vädret, fixat käk, vart på systemet, lånat en flytväst, fått skjuts hela vägen ut till Furusund, lastat båten, rullat ut alla segel, lagt ut, hissat seglen, småpyst i gångfart ut ur viken och äntligen fått lite styrfart när man kommer till fjärden, men då är det nån jävla stockholmare \textsc{(se storswänsk s.~\pageref{716f41dcabef6599bcf08334a8a6ae27})} med typ en V12a som tvunget ska åka förbi precis så långt bort att man inte kan kasta en ölburk på dem och men ändå tillräckligt nära för att skapa vågor så man håller på att ramla ur båten? Den är med största säkerhet en fiskmås (\textit{Larus canus}). Generellt kan man säga att måsarna häckar cirkumpolärt mellan 50- och 70-breddgraden utom på Grönland och i östra Kanada. Den häckar i hela Skandinavien och förekommer såväl vid kusterna som inne i landet. Längden är 40-44 cm, vingspannet 99-108 cm och den har en vikt på ungefär 360 gram.

}

\small{
\textbf{Mördare}
\label{716a6d16202e71ffec6de5365b142b55}
 är ett yrke som går ut på att döda andra människor. Ni kanske tänker att det inte är särskilt lukrativt med mord, men ni anar inte. Det finns många olika sorters mördare och störst av alla mördare är Jacob Wallenberg!

}

\small{
\textbf{Nagellackor}
\label{843cada6d0ac0ef461c38eaeee51d030}
 Nagellackan är en insekt besläktad med kackerlackan.
 Dess latinska namn är Colorae Neonicus.
 Den är en obehaglig insekt som lever på nagellack. När den äter fäller den ut en sugsnabel som sprutar ut ett gift liknande myggans som gör att nagellacket aldrig torkar. Sedan suger den upp en milliliter av innehållet och därefter går den och lägger ägg.
 Nagellackan behöver inte para sig det det räcker att den får i sig pigment så kan den lägga befruktade ägg.
 Nagellackan lägger 75-114 ägg i veckan, aldrig mer.
 Den föds helt genomskinlig men varje gång den äter nagellack byter den färg till samma färg som nagellacket och kissar ut den gamla färgen.
 Den är otroligt plågsam för kisset fastnar och går i princip aldrig bort.
 Den går nästan bara att bekämpa med aceton.

}

\small{
\textbf{Nalta}
\label{fff42153008c054c6224eed0b8a1374b}
 En mycket exakt måttenhet. Definitivt större än en pellagrut, men mindre än ett uns \textsc{(s.~\pageref{07525332b1617934911c9fbadb3a304e})}.
 Om man tar \quotetext{harta bortu harta} och sedan tar \quotetext{harta bortu he \textsc{(s.~\pageref{6f96cfdfe5ccc627cadf24b41725caa4})}} så är man ganska nära.
 Det vill säga: Nalta ≈ 0,25

}

\small{
\textbf{Namnsdagar}
\label{00ac9aeb322a7abaa79330987b0b414e}
 __NOTOC__
 Detta är en lista över \textbf{namnsdagar} sorterat efter datum.
 HEAD2: Mars

 {\textbar class=\quotetext{sortable wikitable}
 ! style=\quotetext{width:110px;} \textbar Datum
 ! style=\quotetext{width:300px;} \textbar Namnsdag
 \textbar-
 \textbar 1 mars \textsc{(s.~\pageref{e8710cb3944e7fe309d31173d3cbdb5e})} \textbar\textbar Albin \textsc{(s.~\pageref{899dffe3a176a36256e0a00fb97698d3})}, Elvira \textsc{(s.~\pageref{059273779b2ffa92e46e85c25c2dc34b})}
 \textbar-
 \textbar 2 mars \textsc{(s.~\pageref{07cdd6d8f631ba978c68684408ab3bbe})} \textbar\textbar Ernst \textsc{(s.~\pageref{f66f8c3c3246e6819aaa792638114eda})}, Erna \textsc{(s.~\pageref{035b3c6377652bd7d49b5d2e9a53ef40})}
 \textbar-
 \textbar 3 mars \textsc{(s.~\pageref{04c4bbc80cc362ef7672d5ad00ec5d0e})} \textbar\textbar Gunborg \textsc{(s.~\pageref{9e29dc34382963ae7d76a742e98637a4})}, Gunvor \textsc{(s.~\pageref{fb79ca08a5e3691af03f6dee15e8d4cd})}
 \textbar-
 \textbar 4 mars \textsc{(s.~\pageref{40517fef36e8e1191b97faddbc3c9500})} \textbar\textbar Adrian \textsc{(s.~\pageref{8c4205ec33d8f6caeaaaa0c10a14138c})}, Adriana \textsc{(s.~\pageref{a01183854d7e784f0455d559f4327d55})}
 \textbar-
 \textbar 5 mars \textsc{(s.~\pageref{c4016f58b8659a82a739bac99c50ec54})} \textbar\textbar Tora \textsc{(s.~\pageref{8303220f39c4f57e9499733006a1d3cc})}, Tove \textsc{(s.~\pageref{a2649d165191e9a91e0e645fb6e18bc6})}
 \textbar-
 \textbar 6 mars \textsc{(s.~\pageref{9e259e935894ab1bfb3bc422eb63a280})} \textbar\textbar Ebba \textsc{(s.~\pageref{21e88735035e9a7d22e7cbc76ed7b478})}, Ebbe \textsc{(s.~\pageref{a7cb5eddea4d149b3850d4089ba8f9d8})}
 \textbar-
 \textbar 7 mars \textsc{(s.~\pageref{594e8766ec5d02122bc9b24ab5c09f4e})} \textbar\textbar Camilla \textsc{(s.~\pageref{b077f51ff36868f21ea52956adcf7ff4})}
 \textbar-
 \textbar 8 mars \textsc{(s.~\pageref{56bc55fdf7815224cd4edbeebba85b7a})} \textbar\textbar Siv \textsc{(s.~\pageref{7b68742c36b75259702f1d732b528d2b})}
 \textbar-
 \textbar 9 mars \textsc{(s.~\pageref{169f369ad7b1c5e3e392f5f0440d842d})} \textbar\textbar Torbjörn \textsc{(s.~\pageref{c3e6fb6fb2b655457597f063bd9392e8})}, Torleif \textsc{(s.~\pageref{52605bb6053b9166d6feab04e45f76a6})}
 \textbar-
 \textbar 10 mars \textsc{(s.~\pageref{f48caff2008e2581e45edb7763743668})} \textbar\textbar Edla \textsc{(s.~\pageref{d0d9f775c1afe3749752c9d141af933c})}, Ada \textsc{(se Ada (namn) s.~\pageref{288c9cda5e6d6df7a02e0a0af70adda7})}
 \textbar-
 \textbar 11 mars \textsc{(s.~\pageref{80e0bc02e79148f4795f37635060b271})} \textbar\textbar Edvin \textsc{(s.~\pageref{31fbf11311684a7ecb580b8188f52df8})}, Egon \textsc{(s.~\pageref{17374c698909042377e8db0abd350040})}
 \textbar-
 \textbar 12 mars \textsc{(s.~\pageref{a2d30427be88040ab57f4b2435223eac})} \textbar\textbar Viktoria \textsc{(se Viktoria (namn) s.~\pageref{9ed58b33ac5b5298d1f9ab5c6c364176})}
 \textbar-
 \textbar 13 mars \textsc{(s.~\pageref{ccaf4acb17d187fa762c0dac103d37de})} \textbar\textbar Greger \textsc{(s.~\pageref{dddd7d1d4dc7dd7da97ddc7736e58869})}
 \textbar-
 \textbar 14 mars \textsc{(s.~\pageref{06f2a81a0c836caddbc1204416b057b4})} \textbar\textbar Matilda \textsc{(s.~\pageref{91f2b7dfd8fc3d900133c356f92c4e20})}, Maud \textsc{(s.~\pageref{07830bf6d135d130a8566fe94ba59eb8})}
 \textbar-
 \textbar 15 mars \textsc{(s.~\pageref{00a65a53e538d3da45d771c909a26ede})} \textbar\textbar Kristoffer \textsc{(s.~\pageref{b1bebf7f19c7345d261dd0f1f7746f00})}, Christel \textsc{(s.~\pageref{a725770e19d65a594f68265b5224e4d1})}
 \textbar-
 \textbar 16 mars \textsc{(s.~\pageref{82c19596aaa17e57ac74d81121dfef34})} \textbar\textbar Herbert \textsc{(s.~\pageref{74b0328a08e7d9e213b1ea77ba32198d})}, Gilbert \textsc{(se Gilbert (namn) s.~\pageref{e5dfcfe21665a66683dae10938792737})}
 \textbar-
 \textbar 17 mars \textsc{(s.~\pageref{c77fd32fe1a46333442b9030e35110c7})} \textbar\textbar Gertrud \textsc{(s.~\pageref{8598eece636f53052b0d6cc2cc2409da})}
 \textbar-
 \textbar 18 mars \textsc{(s.~\pageref{1b5b72932c210e009dd5944ff3f1428f})} \textbar\textbar Edvard \textsc{(s.~\pageref{46e6f90a359cdf6256efa1366ea27f5e})}, Edmund \textsc{(s.~\pageref{d3f8a05c465714e2c09d214af0e88897})}
 \textbar-
 \textbar 19 mars \textsc{(s.~\pageref{28ca4285740ab79ef17d1c2d06ca6fa3})} \textbar\textbar Josef \textsc{(s.~\pageref{6d55e02f385f45605879fbeb1bb42fb9})}, Josefina \textsc{(s.~\pageref{dcfc866c201ce95a295d28be30eb0c47})}
 \textbar-
 \textbar 20 mars \textsc{(s.~\pageref{87f11293cd6eb66f2f2bd79250f47d71})} \textbar\textbar Joakim \textsc{(s.~\pageref{5198db489023c272241a66ad5acda359})}, Kim \textsc{(s.~\pageref{fb1eaf2bd9f2a7013602be235c305e7a})}
 \textbar-
 \textbar 21 mars \textsc{(s.~\pageref{d951b2f91a374189b3df036af239db20})} \textbar\textbar Bengt \textsc{(s.~\pageref{2be19239ba225a12dbd56ea424a66672})}
 \textbar-
 \textbar 22 mars \textsc{(s.~\pageref{07bce34af1a883d3f8e0c67978a9983c})} \textbar\textbar Kennet \textsc{(s.~\pageref{eb251e3745d960e2100c5435a32764c4})}, Kent \textsc{(se Kent (namn) s.~\pageref{40186fbb22508f36f2e1ef476956ed80})}
 \textbar-
 \textbar 23 mars \textsc{(s.~\pageref{263d6549477693429212b6feaf9e6031})} \textbar\textbar Gerda \textsc{(s.~\pageref{d629de2ecfc6d6640872b938ec187716})}, Gerd \textsc{(se Gerd (namn) s.~\pageref{c5e66122b13d3f1b3d16a48821edfff0})}
 \textbar-
 \textbar 24 mars \textsc{(s.~\pageref{e1813255d4250bcecbe89e1b2e89fcd7})} \textbar\textbar Gabriel \textsc{(se Gabriel (namn) s.~\pageref{36402cbdf2bde63cf553515dc139e97e})}, Rafael \textsc{(s.~\pageref{9135d8523ad3da99d8a4eb83afac13d1})}
 \textbar-
 \textbar 25 mars \textsc{(s.~\pageref{b85c2bbab45eb9a47f5bb4c9cd1da8c7})} \textbar\textbar \textlessspan style=\quotetext{color:\#888;float:right;}\textgreater\textit{(ingen namnsdag)}\textless/span\textgreater Marie bebådelsedag \textsc{(s.~\pageref{09aa5d77d08379344ca22284b0c50630})}
 \textbar-
 \textbar 26 mars \textsc{(s.~\pageref{4ef3ff57f45405c10a252cf70310ea1d})} \textbar\textbar Emanuel \textsc{(s.~\pageref{a80e1c212420901edde8bbeb64037593})}
 \textbar-
 \textbar 27 mars \textsc{(s.~\pageref{0dd330ae542925a3cf67035ae5a05bb7})} \textbar\textbar Rudolf \textsc{(s.~\pageref{c1ed4c1369f6af7c64dce16701d5383e})}, Ralf \textsc{(s.~\pageref{3cca634013591eb51173fb6207572e37})}
 \textbar-
 \textbar 28 mars \textsc{(s.~\pageref{44e08058939cad2be3a9cf90a74021b3})} \textbar\textbar Malkolm \textsc{(s.~\pageref{4843714ed38d3debdc701ecb0275afca})}, Morgan \textsc{(s.~\pageref{0dae4a923e4ae71d0a8960c6f89c3c18})}
 \textbar-
 \textbar 29 mars \textsc{(s.~\pageref{c7a567cbf574d226556231bb10f854d5})} \textbar\textbar Jonas \textsc{(s.~\pageref{9c5ddd54107734f7d18335a5245c286b})}, Jens \textsc{(s.~\pageref{e19457c81e62b6bb21e9031a5a187cdf})}
 \textbar-
 \textbar 30 mars \textsc{(s.~\pageref{d1ac42bcb599797508b463b645ca83a5})} \textbar\textbar Holger \textsc{(s.~\pageref{3acdd51ce152dbee8f6f7fd98e687d83})}, Holmfrid \textsc{(s.~\pageref{3bf7f5142d389b9508caad2e1c5d2aec})}
 \textbar-
 \textbar 31 mars \textsc{(s.~\pageref{2b73c5b037f29b0037ceedeaf4e22710})} \textbar\textbar Ester \textsc{(se Ester (namn) s.~\pageref{48bfd6fa1ebdadcf8ddd9e05b5c6ac9d})}
 \textbar}

}

\small{
\textbf{Nattens Engel}
\label{b502941eb96ddba32dd5e652b647f350}
 är en dansk vampyrfilm från 1998. Den är så sanslöst dålig att det är i det närmaste omöjligt att minnas vad den egentligen handlar om. En scen är däremot oförglömlig och det är den enda scen i vilken engelska talas. I scenen befinner sig en vampyr i en korridor och en vampyrjägare dyker upp med en revolver i högsta hugg. Vampyrjägaren säger då \quotetext{Hey you, fuckface! Smile.} och skjuter vampyren.

}

\small{
\textbf{Nattväktarstat}
\label{8709c886f2bf43f44af8bcb5958bc30f}
 En nattväktarstat är en stat som istället för att lägga sig i sina medborgares alla förehavanden har ett avslappnat förhållandesätt till sina skyldigheter. Nättvaktarstatens jobb går ut på att sitta på ett kontor och glo på tv-skärmar där inget händer. Plötsligt hörs ett märkligt ljud och nattväktarstaten går och undersöker detta. Den skriker \quotetext{Vem där!?} och lyser med en ficklampa i en tom lagerlokal. Den dricker sedan svart automatkaffe, sätter sig i kontorstolen och drar en djup suck.

}

\small{
\textbf{Naturen}
\label{77942174bc778b3f7c7fd1f10134d30a}
 är en spännande plats där saker går till nästan som de gjort sedan urminnes tider. Här bor alla djur och alla växter, samt svampar och bakterier. Naturen blev till stor del inaktuell för människan i och med att det moderna samhället började ta form.

 HEAD2: Naturen i den politiska filosofin
 Den brittiska filosofen Thomas Hobbes hade inte mycket gott att säga om naturen. Han menade att där pågick ett allas krig mot alla och att människan var människans varg. Naturen gör alltså att människor hatar varandra och inte har tänkt hjälpa sin nästa ett jota, och det här är förklaringen till varför alla campingsemestrar slutar i bråk. Livet i naturen lär Hobbes träffsäkert ha beskrivit som kallt, brutalt, förjävligt och framförallt kort. Hobbes motiverade därför att människan bör acceptera en stat som skyddar och vägleder sina medborgare, annars har de bara ihjäl varandra.

 Det finns dock de som tycker att naturen är lite fetare, som till exempel den ryske biologen och filosofen Peter Kropotkin. Han såg naturen som en plats för samarbete där alla tar hand om varandra och är snälla. Han bevisförde detta genom att studera myror och en massa andra djur som inte bråkar så mycket med varandra utan hjälps åt. Olika arter har dock en tendens att bråka något djävulskt med varandra, men i och med att det djur som Kropotkin främst riktade sig till var människan så tyckte han att tesen höll. Han tyckte inte att någon stat var nödvändig utan förespråkade en mindre formell organisering.

 HEAD2: Naturen i samhällsdebatten
 Naturen är ofta en riktlinje för vissa debattörer, måhända lite konservativa (milt uttrycket). De ser till naturen som ett rättesnöre, eller försöker så gott det går. Populära ämnen där detta tas upp är t.ex. diskussioner om könsroller och gay-äktenskap. Dåförtiden var nämligen kvinnan hemma i grottan och sopade med en kvast gjord av en gren, förstår ni, och därför ska det också alltid vara så! Så kan ett argument låta. Det här begränsas inte ens till hur människan antas ha betett sig fyrtiotusen miljarder \textsc{(s.~\pageref{c2160bffc9c5ca88e77204672e62e489})} år sen, utan även hur andra djur beter sig och har betett sig. Att två vuxna människor  Städar inte karlfan hemma? Det är för att på stenåldern (alltså, när människan levde i naturen) var han ute och jagade och honhatar varandra men prompt ska dela vardag låter förstås rätt vansinnigt, men så har typ svanar \textsc{(se svan s.~\pageref{f80f1875ab3ebccf935723ba83b6da63})} alltid gjort och därför borde vi göra likadant, för annars är det inte \textit{naturligt}. Det finns extremt många bizarra beteenden som finns hos andra djur som absolut inte lika lätt skulle kunna rättfärdigas hos människan, som t.ex. att döda sin partner efter samlag eller att döda sina barn för att de utmanar ens position i familjen. Människan har också under sin mångtusenåriga histora utvecklat en massa askonstiga beteenden som inga andra djur är särskilt förtjusta i, t.ex. att kolla på TV och köra bil. Trots detta tillåter dessa debattörer detta fastän det är så långt i från stenåldersmänniskorna att det är rent löjligt. Det är så att man undrar om dessa debattörer inte har något annat i kikaren än att förespråka en naturlig människan, men varför misstro dem?

 HEAD2: Naturen i media
 Media har en tendens att prioritera mänskliga förehavaden i sin nyhetsrapportering, men Umeås ena dagstidning Västerbottens-Kuriren råder bot på detta. I nästan varje upplaga finns i alla fall en kort artikel om vad som egentligen händer i naturen, typ att en hök dödat en duva, eller att en järv tagit en ren. Ibland finns lite mer sorgsamma reportage som det om talgoxen som förirrat sig till en gatbrunn, men det gick bra.

}

\small{
\textbf{Naturens dialektik}
\label{8688bb10239ff053879f9219f8191bd2}
 Det som inträffar när en blomma växer i betong.

 För att lite mer gå in på vad denna dialektik \textsc{(s.~\pageref{5c0ded4e9796ad82ecd11d1a0010bf6b})} innebär så är tesen naturen, anti-tesen civilisationen och syntsesen? Den ser ni här ovan.

}

\small{
\textbf{Naturhistoriska museet}
\label{e8743a78ba58fa893902b75a5ddfa461}
 Låt dig inte luras av den i jämförelse pampiga fasaden. Ta istället första chansen du har att kolla vilken matsäck som skickats med dig. Detta är utgångspunkten för dina val från nu till ni åker hem.
 HEAD2: Klassamhället

 Risig matsäck = Glida iväg från gruppen och försöka hitta rådjuret som har två huvuden
 Helt okej matäck = Häng på. Har du tur kommer du hamna i ett läge där du kan dra in på shoppen och använda resten av de pengar dina föräldrar kastar kring sig till att köpa en ihålig gummidinosarie. Gör en mental notering - de i klassen som drog iväg för att kolla på rådjuret istället för att äta matsäck kommer förr eller senare spöa skiten ur dig. Inget kommer vara som förr.

 Se även: Storfräsare \textsc{(s.~\pageref{4db17005692cd83e3e946a1311b81ed0})}

}

\small{
\textbf{Nedlagda industribyggnader}
\label{cf7af2374abf8d8d6ccc035c2e0eb3be}
 är betongkolosser som byggdes för länge sedan innan Socialdemokraterna gick på myterna om \quotetext{tjänstesektor} och \quotetext{incitament \textsc{(s.~\pageref{f9896a922c4b9345ceebc37009eaf545})}}. Utan nedlagda industribyggnader skulle Sverige \textsc{(s.~\pageref{b1999637949ed135b2ca03f3a38460cc})} inte ha några konsthallar och ungdomar skulle inte ha någon stans att öva sig på att panga rutor. Dessutom skulle landets fotografer \textsc{(se fotografering s.~\pageref{176551844874f34f5bb9a9d0ac93f99a})} inte ha några motiv att ställa ut i konsthallarna.

}

\small{
\textbf{Nelsoncertifikat}
\label{2445c44b60e201b29745dd85f9fdcbb4}
 Ett Nelsoncertifikat är en märkning av kommersiella produkter som garanterar att varan inte innehåller noshörningshorn. Märkningen är uppkallad efter hela svenska folkets gullegris noshörningen Nelson \textsc{(se noshörningen nelson s.~\pageref{e439707db1c491d30a2ac06e71632fe6})} som dog en allt för tidig död efter att Jan Björklund \textsc{(s.~\pageref{0b9b757044804b9be0e218acdad358cc})} sparkat honom i skallen för att demonstrera arbetslinjen \textsc{(s.~\pageref{a7d4c1873c9542a1c6a48a1e52bdb823})}. Certifikatet består av ett emblem med Nelsons ledsna ansikte med en såg i hornet. I princip alla produkter som inte är traditionell kinesisk medicin \textsc{(s.~\pageref{b910794bdd09f8d29cf4b7e9a3fe966e})} kan bli nelsoncertifierade, men då krävs först att man betalar ansökningsavgiften på 50.000 kronor till \textit{Prof. Etiennes gulliga djurstiftelse \textsc{(se Användare: Prof. Etienne s.~\pageref{a9878d2280e5a39becac8f73d113df91})}} som står bakom certifikatet. Nya medlemmar värvas genom ett telemarketingföretag som ringer runt till företag och frågar: \quotetext{-Vadå, gillar du inte Nelson, eller?}. Stiftelsen innehar inte 90-konto.

}

\small{
\textbf{Neontetra}
\label{460642ed82ecf1d38b42a3f0a1f9afe3}
 n (\textit{Paracheirodon innesi}, plural neontetris) \textsc{(se tetris s.~\pageref{f76e534251c8595a9746fde225f9289b})} är den billigaste akvariefisken som man kan köpa på en djuraffär, men så är den också så liten att om den inte hade haft en så starkt färgad skrud hade man inte kunnat se den. Om man fixar en stor glasburk eller ett litet akvarium och skickar ner lite sand \textsc{(s.~\pageref{88336b5bb2a1cc21bac7cf33fd451270})} och grus, vatten och ett gäng neontetror har man vips! ett akvarium som kan skänka mycket glädje åt det enklaste hushåll och som pappa kan vila blicken på efter en lång och strävsam arbetsdag.

}

\small{
\textbf{Nia}
\label{04a481486dd84d7c8bfdfc89d38136a6}
 Titellösas tilltalsord innan Du-reformen och den del av kroppen som Pluton Svea vill skjuta Leif Loket Olsson i.


 Se även: Etta \textsc{(s.~\pageref{ba48f6c4097b7fc25ca11f1e544842d7})}, Tvåa \textsc{(s.~\pageref{84fcc0494ecf9f5af79fcd9bed184a9a})}, Trea \textsc{(s.~\pageref{6f94fdf535ab2e21147ea40ea920ca75})}, Fyra \textsc{(s.~\pageref{7bdb5385ce8e0b1cbc7c15b1d71e8e7d})}, Femma \textsc{(s.~\pageref{d974e0811fe7a4d49a9062d33b66a88d})}, Sexa \textsc{(s.~\pageref{4b1fabe53857b0a2ace6ae22008fe13e})}, Sjua \textsc{(s.~\pageref{e7bf63fa6d0d29bd89c23f833b979a15})}, Åtta \textsc{(s.~\pageref{6fa68b0d02ec525fa72a51c13e5e3ed1})}, Tia \textsc{(s.~\pageref{e7292d5ba58672ce7f6fc3c0b646ab63})}, Elva \textsc{(s.~\pageref{788bd84addbcf8f1767869759d4a2ad9})}.

}

\small{
\textbf{Nicholas Cage-film}
\label{691536bd318544d7e5ffe1ee895caefd}
 En Nicolas Cage-film är en spelfilm som på något vis är kopplad till skådespelaren Nicolas Cage - i de flesta fall genom att Nicolas Cage är med i filmen, men det är också möjligt att han varit inblandad i att göra filmer, och om så är fallet är även dessa Nicolas Cage-filmer. Vad som utmärker en Nicolas Cage-film är att den oftast är obegripligt fånig och dålig. Exempel som Gone in Sixty Seconds, National Treasure, Face Off, Con Air, Next och Season of the Witch talar för sig själva. Detta gör att många fascineras av Nicholas Cage och de filmer i vilka han onekligen fortsätter att figurera. Ingen gillar honom. Han är en bedrövlig skådespelare som med hjälp av några av samtidsfilmhistoriens största konstnärer (till exempel Charlie Kaufman) marakulöst men temporärt rykts upp till en acceptabel nivå för att genast åter sjunka tillbaka ner till sin normala, absurt låga nivå av skådespeleri. Det sägs att John Travolta efter inspelningen av Face Off blev så förvirrad av det faktum att Nicolas Cage forfarande erbjuds roller att han sökte sig till den Scientologiska kyrkan för att få svar på många av de frågor om livet och varat som dök upp under de grubblerier som Cages karriär föranledde hos Travolta, som ju som bekant själv uppvisar en högst tveksam konstnärlig förmåga.


 HEAD2: Nicolas Cage i populärkulturen
 Medallion (pre-production)
 
 2012 Ghost Rider: Spirit of Vengeance (filming)
 Johnny Blaze / Ghost Rider
 
 2012 The Croods (post-production)
 Grug (voice)

 2011 Trespass (post-production)
 Kyle

 2011 The Hungry Rabbit Jumps (completed)
 Nick Gerard

 2011 Drive Angry 3D
 Milton
 2011 Season of the Witch
 Behmen
 2010 The Sorcerer's Apprentice
 Balthazar
 2010 Kick-Ass
 Damon Macready / Big Daddy
 2009 Astro Boy
 Dr. Tenma (voice)
 2009 The Bad Lieutenant: Port of Call - New Orleans
 Terence McDonagh
 2009 G-Force
 Speckles (voice)
 2009 Knowing
 John Koestler
 2008 Bangkok Dangerous
 Joe
 2007 National Treasure: Book of Secrets
 Ben Gates
 2007 Next
 Cris Johnson
 2007 Grindhouse
 Fu Manchu (segment \quotetext{Werewolf Women of the SS})
 2007 Ghost Rider
 Johnny Blaze / Ghost Rider
 2006 The Wicker Man
 Edward Malus
 2006 World Trade Center
 John McLoughlin
 2006 The Ant Bully
 Zoc (voice)
 2005 The Weather Man
 David Spritz
 2005 Lord of War
 Yuri Orlov
 2004 National Treasure
 Benjamin Franklin Gates
 2003 Matchstick Men
 Roy Waller
 2002 Adaptation.
 Charlie Kaufman / Donald Kaufman
 2002 Sonny
 Acid Yellow
 2002 Windtalkers
 Sergeant Joe Enders
 2001 Christmas Carol: The Movie
 Jacob Marley (voice)
 
 2001 Captain Corelli's Mandolin
 Captain Antonio Corelli
 
 2000 The Family Man
 Jack Campbell
 
 2000 Gone in Sixty Seconds
 Memphis Raines
 
 1999 Bringing Out the Dead
 Frank Pierce
 
 1999 8MM
 Tom Welles
 
 1998 Snake Eyes
 Rick Santoro
 
 1998 City of Angels
 Seth
 
 1997 Face/Off
 Castor Troy / Sean Archer
 
 1997 Con Air
 Cameron Poe
 
 1996 The Rock
 Dr. Stanley Goodspeed
 
 1995 Leaving Las Vegas
 Ben Sanderson
 
 1995/I Kiss of Death
 Little Junior Brown
 
 1994 Trapped in Paradise
 Bill Firpo
 
 1994 It Could Happen to You
 Charlie Lang
 
 1994 Guarding Tess
 Doug Chesnic
 
 1993 Deadfall
 Eddie
 
 1993 Red Rock West
 Michael Williams
 
 1993 Amos \& Andrew
 Amos Odell
 
 1992 Honeymoon in Vegas
 Jack Singer
 
 1991 Zandalee
 Johnny Collins
 
 1990 Wild at Heart
 Sailor Ripley
 
 1990 Fire Birds
 Jake Preston
 
 1990 Industrial Symphony No. 1: The Dream of the Brokenhearted (TV movie)
 Heartbreaker
 
 1989 Tempo di uccidere
 Enrico Silvestri
 
 1988 Vampire's Kiss
 Peter Loew
 
 1988 Never on Tuesday
 Man in Red Sports Car (uncredited)
 
 1987 Moonstruck
 Ronny Cammareri
 
 1987 Raising Arizona
 H.I. McDunnough
 
 1986 Peggy Sue Got Married
 Charlie Bodell
 
 1986 The Boy in Blue
 Ned Hanlan
 
 1984 Birdy
 Al Columbato
 
 1984 The Cotton Club
 Vincent Dwyer
 
 1984 Racing with the Moon
 Nicky/Bud
 
 1983 Rumble Fish
 Smokey
 
 1983 Valley Girl
 Randy
 
 1982 Fast Times at Ridgemont High
 Brad's Bud (as Nicolas Coppola)
 Källa: IMDB

}

\small{
\textbf{Nigeria}
\label{f174114ad69e0db66eff6090be60b5e5}
 Land i västra Afrika. För att trygga världsfreden bildade man 1958 den så kallade \quotetext{Brev- och Frimärkesunionen} med Lichtenstein.
 Fursten av Lichtenstein hade genom ett vänligt brev från Nigerias dåvarande president uppmärksammats på ett väldigt arv. Detta blev början på den djupa vänskapen mellan de två staterna.En gemensam valuta (Shilling Banco) infördes på 2000-talet. Ekonomer världen över ser detta som förutsättningen för dagens blomstrande ekonomi. Under 2010-talet har nya exportvaror tillkommit, många inom IT-sektorn. Bland annat så sköter Microsoft allt sitt säkerhetsarbete ifrån \quotetext{Afrikas Silicon Valley}. Nigeria har fått sitt namn efter floden Niger. Niger blev känt för bland annat Lichtenstein genom skotten Mungo Park.

}

\small{
\textbf{Niklas}
\label{87c7a90aa97629c950fb781d466f5c65}
 är ett mansnamn och en sammansättning av \quotetext{nick} som betyder att bli träffad i huvudet \textsc{(se huvud s.~\pageref{e906cd95a540df9b16d0460fb4cf0adc})} med en fotboll \textsc{(s.~\pageref{961bd74d34872ff94a4df3a16119096e})}, och \quotetext{LAS}, dvs lagen om anställningsskydd.

}

\small{
\textbf{Nikolaj Valujev}
\label{85363e70d015f18efa110613d79baf4a}
 är en rysk tungviktsboxare som i väst går under namnet \quotetext{the beast from the east.}  Förutom sina tungviktstitlar och framförallt sin imponerande kroppshydda \textsc{(s.~\pageref{032eed30d2aad3425b9139aafdd6740f})} på 150kg och 213cm över havet är Nikolaj Valujev känd i Europa genom att många föräldrar använder honom skrämma för att sina barn då de är olydiga, inte vill sova eller inte vill äta upp maten: \quotetext{Äter du inte genast upp maten kommer Nikolaj Valujev och boxar dig i huvudet \textsc{(se huvud s.~\pageref{e906cd95a540df9b16d0460fb4cf0adc})},} kan det heta. Idag har Valujev till många boxares lättnad \textsc{(s.~\pageref{c591923999933bd79701bef0f2af2dc0})} lagt handskarna på hyllan för att istället leta efter snömannen i Sibiriska grottor[http://www.dn.se/nyheter/varlden/yetin-draghjalp-i-valet].

}

\small{
\textbf{Niledent}
\label{6fb36b1de07f94109de7882a96e189d7}
 Sveriges enda jakaranda-färgade dentalprodukt.

}

\small{
\textbf{NissCon 2011}
\label{5e4b46c004e3fa5ca2271078b209cdc0}
 Förslaget om att ha ett Nissepedia-LAN har uppdagats. Använd denna sida för att diskutera. Jag är otroligt positivt inställd till idén och öppnar mer än gärna upp mitt hem för att hålla i detta event. /John

}

\small{
\textbf{Nisse Schwarz}
\label{5af492cc4c5ae8563564d2a80c2d4f56}
 var en kulturjournalist på expressen vars minne ska ha kallats Nissepedia \textsc{(s.~\pageref{62400dadecd90cb5cd39062abe5a3e4a})} av kollegan Eric Erfors \textsc{(s.~\pageref{77b4a42490103982dba1f4bafed8a276})}. [http://bloggar.expressen.se/thomasmattsson/2011/05/fran_nisse_schwarz_till_arne_weise_pa_krogen_i_gar_kvall/]

}

\small{
\textbf{Nissepedia}
\label{62400dadecd90cb5cd39062abe5a3e4a}
 60\% nonsens, 20\% lögner, 20\% übersmal trivia.
 HEAD2: Artiklar
 Nissepedias i särklass bästa artikel är den om finskt sämskskinn \textsc{(s.~\pageref{ecdf6b5129df6ebb83a9b381b4b33553})}, tätt följt av den om doktorander \textsc{(se doktorand s.~\pageref{308932f67f983fbb70157f1a481f51ea})}.
 HEAD2: Ödmjuka jämförelser med omvärlden
 Nissepedia är i nuläget (2011-07-24 19:36) med sina 1 \textsc{(se etta s.~\pageref{ba48f6c4097b7fc25ca11f1e544842d7})} artiklar större än vad Wikipedia är på språken Moksha \textsc{(s.~\pageref{ce9ad8fab25a32847194eb0e62278ee9})}, som talas av 500.000 ryssar, Lägre Sorbiska \textsc{(s.~\pageref{303f755b0e855272f58060b6bf78fe94})} med ungefär 14 000 talande, den Holländska dialekten \textsc{(se dialektik s.~\pageref{5c0ded4e9796ad82ecd11d1a0010bf6b})} Zeeländska (220.000 talande), Kabyliska (3,1 miljoner talande), kreolspråket Tok Pisin som talas av mellan 5 och 6 miljoner papuaner och det austronesiska språket Banjaresiska med drygt 6 miljoner talande. Detta mot Nisspedias 60 medarbetare (i teorin, i praktiken typ fem \textsc{(se femma s.~\pageref{d974e0811fe7a4d49a9062d33b66a88d})}.

}

\small{
\textbf{Nissepedia på fyllan}
\label{582e983a1997c0e094f2107e6550db4e}
 Det är då man skriver det man annars inte vågar. Exempelvis artiklar om Fariséer \textsc{(s.~\pageref{4503817c4c5816b5f30fffc91f66ac28})} eller vilda spekulationer om uvar \textsc{(se uv s.~\pageref{45210da832f9626829457a65e9e7c4d0})}. Nissepedia \textsc{(s.~\pageref{62400dadecd90cb5cd39062abe5a3e4a})} är generellt mycket roligare lite lagom bärsfull \textsc{(se bärsfylla s.~\pageref{9380b60f9ee744b9acf978fe6f1a9545})}.

}

\small{
\textbf{Nissepediadagen}
\label{b8efbfc8d1c6498d04b980443fcab17e}
 , som infaller en gång om året, nämligen den 28e juli, firas över alla kartlagda delar av riket Sverige \textsc{(s.~\pageref{b1999637949ed135b2ca03f3a38460cc})} genom att familjen eller kollektivet samlas framför datorn och med tindrande ögon tar del av all kunskap som här mot ingen annan avgift än ett leende, ett snällt ord då och då, erbjuds allmänheten. Vissa kanske låter sina smaklökar retas av de talrika uppslagen om maträtter och låter göra några parisare \textsc{(s.~\pageref{5aca28013b9a7e4088e7fb228f3e4827})} eller en läcker uvsvane \textsc{(s.~\pageref{c5081b14cdeb1ff42b655213e80c9d51})}, till exempel, medan andra gör en utflykt till Holmsunds tropikhus \textsc{(s.~\pageref{5b087d935637ad4d1823cf48036e9be6})}, åker en sväng med buss 85 \textsc{(s.~\pageref{c7f78d6a64e6921e84be4513166cdade})}, shoppar loss riktigt på britts mode \textsc{(s.~\pageref{4222116edbe095681ea4a4513b21bd44})} eller besöker världens näst största byggnad \textsc{(s.~\pageref{3a66962b6aa9d503a3076b68cc261d23})}. Åter andra kanske väljer att fira dagen genom att i sin ensamhet grunka \textsc{(s.~\pageref{6b79b7e074be4c86299c3ee48160b626})}, skrunka \textsc{(s.~\pageref{7e3152e0cbea2212bd02444c45ee00db})}, bröka \textsc{(s.~\pageref{60862d3b986c7bbedc86064c842c5a6c})} och/eller brunka \textsc{(s.~\pageref{e9b217feddf7c7c0f64400fe683de947})}.

}

\small{
\textbf{Nissepedianostalgi}
\label{19ad57413f916e65c01d02a5e394421f}
 Tänk första gången man läste om Randolfo \textsc{(s.~\pageref{b8f0a32f840f1db27a2c12e17b640fb2})}. När man gjorde sin första bekantskap med dubbelsovling \textsc{(se dubbelsovla s.~\pageref{4a58428516d8ba930242406ad6073922})}. En sådan svårbeskriven lycka det var när ögonen för första gången vilade på texten om johnskroven \textsc{(se johnskrove s.~\pageref{92a6f4a71ab0087f48ba4aab7db89bdb})}

 det var bättre förr \textsc{(s.~\pageref{c7d3f908ea2aaab4cab2730336769b70})}

}

\small{
\textbf{Nissepediaslump}
\label{91de3fe4aa80ac35adb1ccf29fba1ea5}
 Enligt inofficiella statistiska analyser närbesläktad med märkta kort och magnetiska tärningar. En av spelets vinnare är Glasse.

}

\small{
\textbf{Nissepediasöndag}
\label{0d3526159b7e7ace4e029fd98092e832}
 På söndagkvällar poppar det gärna upp en oändlig massa nissepediaartiklar \textsc{(se nissepedia s.~\pageref{62400dadecd90cb5cd39062abe5a3e4a})}, när veckan tidigare förefallit lite torr.

}

\small{
\textbf{Nitlott}
\label{e4c4cf16474fad71a92b94f8008dc2f7}
 Lotterispel arrangerat av Ägg tapes \& records till förmån för träskpunkare \textsc{(s.~\pageref{484838b3db1adb135ea74d6fc61e44c0})} vars bidrag tagit slut. Nitlotteriet arrangerades första gången på Augustibuller 2004 och första pris var Anti Cimex-Cliffs gamla skinnpaj. Varje lott kostar tio kronor och det går bra att betala kontant eller i folköl. Föreningen bakom nitlotteriet saknar i sann punkanda 90-konto hos postgirot och det mesta av de insamlade pengarna försnillas.

}

\small{
\textbf{Nolla}
\label{21cdce774e105d593f6ea43014412b28}
 En nolla är ett träskaft \textsc{(s.~\pageref{1ab85ecd859ae682af47bb9334c7dac6})}, någon som är särske \textsc{(s.~\pageref{552a5aad891937bf760fb193900ea140})}, eller inte har \quotetext{alla x i y \textsc{(s.~\pageref{18d4689248d1c32d716dab95e7e57b17})}}.
 Ett exempel på en nolla är Tomas Bodström (s).

}

\small{
\textbf{Nollpresterare}
\label{846150b14321acb5fe3a8be14131b5b5}
 är ett begrepp som används inom den karga och cyniska högskolepolitik som Jan Björklund \textsc{(s.~\pageref{0b9b757044804b9be0e218acdad358cc})} lagt grund för under sin tid som utbildningsminister. Begreppet används för att underlätta de godtyckliga försämringar som sagda minister oförtröttligt och mot större delen av den samlade universitetsvärldens vädjanden kämpar för att införa. I detta sammanhang är en nollpresterare någon som skriver in sig på en kurs men inte slutför den och därmed inte får några högskolepoäng, men ordet har också kommit att användas utanför detta sammanhang. Det används också om personer - framförallt unga män - som i sin tillbakalutade existens inte presterar mycket mer än unken luft och experience points i något dataspel förlagt i ett parallellt, mindre komplicerat universum. Nollpresterare är vanliga bland annat i stadsdelen Berghem \textsc{(s.~\pageref{a6b1df39fa9b1b94dc92200594a8ccd6})} i Umeå och, kanske framförallt, i Norrtälje \textsc{(s.~\pageref{7527f7dad9445013a559dc7e2a91f3b3})} kommun.

}

\small{
\textbf{Nordisk kombination}
\label{b37e2fc95fc925a0807286d346d381ab}
 är en av de mer obskyra OS-sporterna. Grenen går ut på att man både hoppar backhoppning och åker längdskidor. Först drar man upp i backen och ser vem som hoppar längst och den som vinner får starta först i längden. Man kan skratta och tycka att det är en ganska fånig kombo, men om man tänker igen så känns det faktiskt ganska hårt. Bra mycket ballare än sprint, vattenskidor, puckelpist och andra grenar som är mer som lekar för vuxna.


 HEAD2: Nationella varianter
 I Danmark finns det aldrig någon snö så där kör man istället \textit{dansk kombination}, som innebär att man kör säckhoppning \textsc{(se säcklöpning s.~\pageref{b455d309a790e837f6b436258869ff50})} med rullskridskor på fötterna. Grenen har inte OS-status.

}

\small{
\textbf{Norge}
\label{aa03c4e9d9f8011f9b0102380b029256}
 är ett land som ligger i närheten av det hos många nissepedialäsare betydligt mer välkända Sverige \textsc{(s.~\pageref{b1999637949ed135b2ca03f3a38460cc})}. Norge är bergigt, ligger nära havet och är väldigt smalt, vilket ofta leder till att folk misstar det för Chile. Vill man kontrollera vilket av dessa länder man är i kan man enkelt göra det genom att ställa en fråga till någon i lokalbefolkningen. Får man svaret på norska är det troligen Norge man är i.
 HEAD2: Kultur
 Karl Ove Knausgård kommer från Norge, vilket tydligt framgår i den kritikerrosade romansviten \textit{Min Kamp}. Även Knut Hamsun och Edvard Munch ska visst komma från detta land.
 HEAD2: Ekonomi
 Norge har jättebra ekonmi. Det har de fått genom att saluföra olja och tjocktröjor. Därför är alla norrmän och -kvinnor väldigt nöjda med sig själva och sitter mest på bron och dricker öl och lyssnar på true Norwegian black metal hela dagarna.
 HEAD2: Religion
 Norge har länge varit kristet, men idag bekänner sig de flesta till satanismen.

}

\small{
\textbf{Normal dip}
\label{0880d46cbfb34f0c6712f9c00edae537}
 Anagram för palindrom.

}

\small{
\textbf{Normalt}
\label{5c455ca1c87070883ff0a4c13ae8937f}
 \textbf{ALLT} är normalt, den som påstår annat är en farlig nej-sägare.

}

\small{
\textbf{Norra Beteendevetarhuset}
\label{cbd36c73a33d2d245392215cfab45b64}
 , också känt som Batonghuset, också känt som Snuthuset. Det är ett hus på Umeå Universitet där polisstudenterna håller till. Här har de grupprum som ingen utom snutstudenterna får boka, men som likförbannat alltid står tomma. Rekordet på att vistas i snuthuset utan att höra ett \quotetext{Hur var det här då?} är 1 minut och 36 sekunder.

 Ett intressant stycke fakta \textsc{(s.~\pageref{fce663ae73dc87a727148bc3b94d1ffa})} är att det tar kortare tid att bli snut än att bli förskollärare.

}

\small{
\textbf{Norrbotten}
\label{0e8c003b75982032cde152609ee94154}
 Även känt som \quotetext{paradiset} och omnämnt i första mosebok.
 Är man från Norrbotten är man automatiskt lite bättre, lite snyggare och framförallt mycket tuffare än alla andra.Samtidigt löper man statistiskt sett 110\% större risk än Sveriges \textsc{(se Sverige s.~\pageref{b1999637949ed135b2ca03f3a38460cc})} övriga befolkning att vara en arbetslös pornografikonsument med rovdjursskelett i garderoben.

}

\small{
\textbf{Norrländska}
\label{e9a8473de49e4580345e0db21ff5c1df}
 \quotetext{Norrländska} är en ihopklumpande beteckning på alla dialekter som talas i Sveriges nordligaste län. Beteckningen är tillskriven av en oförstående storswänsk \textsc{(s.~\pageref{716f41dcabef6599bcf08334a8a6ae27})} kolonialmakt som inte har något sinne för nyanser överhuvudtaget. Att exempelvis inte kunna höra skillnad på Älvsbymål och Pitmål är väl en grej, men att inte bemöda sig skilja på exempelvis Lulemål och Ångermanländska är ett tecken på mental retardation och hög intellektuell viskositet \textsc{(s.~\pageref{17328a3aa2e9e596e033ccebf7995cc1})}.

}

\small{
\textbf{Norrtälje}
\label{7527f7dad9445013a559dc7e2a91f3b3}
 , eller Norrtelje som det ibland stavas, än en stad i Uppland \textsc{(s.~\pageref{0f24a6eb0b60bdcd74885743cb7099d8})} och huvudort i Norrtälje kommun, vilket säger sig självt. Norrtälje ligger cirka sju mil från Stockholm \textsc{(s.~\pageref{edcd259e0a03c7ab70feb186bae19f13})} och ca åtta mil från Uppsala \textsc{(s.~\pageref{1db4e388df1df7057b8f3d984c65ee88})}.Norrtäljes stadsvapen är ett ankare, den traditionella symbolen för hopp, vänt upp-och-ned. År 2005 hade Norrtälje lite över 16200 invånare. Staden grundades redan 1622 av Gustav II Adolf men kring den finns fornlämningar från tidig järnålder och folkvandringstiden. Norrtälje kan stoltsera med en av världens högsta densiteter av stentrollsaffärer \textsc{(se stentrollsaffär s.~\pageref{f832a905e0a0a857d0d7eae1520c14b1})} samt en för regionen rekordlåg utbildningsnivå bland befolkningen. Norrtälje är också scen för motorcykelsammankomsten Custom Bike Show som hålls första lördagen i juni och är Skandinaviens största event för hembyggda motorcyklar. Nedan följer ett slags litterär stadsvandring genom staden, häng med!

 HEAD2: Stadsvandring genom Norrtälje stad
 Vi kliver iland på borgmästarholmen i Norrtäljes östra del och har gästhamnen framför oss där överklassens jävla segelbåtar guppar. På andra sidan Norrtäljeviken har vi stadsdelen Grind \textsc{(s.~\pageref{2568fa9f52b34de6328f5044555fe7b6})} som blickar ut mot havet och vars skylt av naturliga skäl brukade bli snodd när det var punkspelningar i staden när det begav sig. Längre upp har vi Kvisthamra. Här bor det reklamare och företagare som har flyttat in från Stockholm. På andra sidan kärleksudden ligger adelsläkten Löwens gamla herrgård, Björnö. Vi går vidare!

 Vi har nu passerat Kärleksudden där man kan bada eller köpa en öl, men det är ofta mycket barnfamiljer där på dagen. Vi går nu genom Socitetsparken - Norrtäljes svar på vad som i alla andra svenska städer kallas \textit{folk}park. Här har det utkämpats tallösa slagsmål och parken har i slutet av 90-talet stormats av piketsnut från Täby eftersom det hade kommit till deras vetskap att ungdomar var där och hade picknick. Nämen se där har vi S/S Norrtälje och bakom den står ju Pajen på kajen \textsc{(s.~\pageref{813e80b092f456fc81da1b8e0e83a273})} och vinkar åt oss! Vi vinkar tillbaka, men går vidare.
 Där har vi Socitetsbacken och här har vi Landkrabban där ett oräkneligt antal 20åriga flickor arbetar och säljer rattstora bullar till andra 20åriga flickor. Längre bort skymtar Roslagsskolan där den övre medelklassens ungdomar går. Där finns musikklass och speciella klasser för \quotetext{Bäst i klassen}-tjejer som är sina föräldrars ögonstenar och som kommer att arbeta på Landkrabban. Intill ligger En liten smula där 20åriga flickor och pensionärer lade sina surt förvärvade pengar på handgjord konfektyr, vilket dock inte hindrade stället för att kånka. Vi går förbi Frälsningsarmébacken och på Stora torget hälsar vi på den tonåring som inte gått på Roslagsskolan men som står i \quotetext{Korvleones} grillvagn och tjänar tio och femti i timmen på att mata turister från Stockholm med korv. Vi kommer till Gröna ön där alkisarna hänger, hälsar på Arne Katten, Uffe Sotaren, Pinnen, Törstiga Törnan, Fylle-Gerd och de andra. På busstationen står det kickers som fortfarande finns kvar här och ett antal galna tanter som skriker och bråkar om en tom plastpåse. Vi passerar nu klassgränsen i Norrtälje och har kommit till underklassens halva av staden, den västra delen.
 Vi går nu mot Sandkilen där det förut fanns ett bilgarage med en vägg av små glasrutor som man kunde kasta sten på. Här ligger också Norrtelje Tidning och bakom den ett industriområde där VPK-lokalen låg innan kommunen rev den och VPKarna fick flytta in i ett garage vid norra industriområdet. Brevid ligger Rikets Sal. Ännu lite längre bort har vi kåken, där Hagamannen bland andra sitter.
 Nu har vi kommit till Knutby torg, vars förut så exotiska namn nu klingar lite sämre. Här ligger Järnia och där kan vi handla spik och tapetväv på löpmeter. Här ligger Lidl, Preem, Willys och Rusta. Bakom dessa finns ännu ett industriområde, med bland annat Samhall, en skrot och bikerlokalen. Vi korsar Vätövägen och kommer till Lommarvägen. Här bor det socialfall, långtidssjukskrivna knegare, ensamma mammor och nyalända invandrare. Parkerar man på Mekonomens parkering efter stängningstid får man böter inom några minuter eftersom någon i hyreshuset brevid ringer vaktbolaget. Här bor subotexpundare. Här kan man köpa droger och billig sprit. Bensinmacken brevid har blivit rånad två gånger de senaste åren.  Vi går till Vargheden, där man kan spela fotboll \textsc{(s.~\pageref{961bd74d34872ff94a4df3a16119096e})} eller smälla smällare och ställa till med jävelskap \textsc{(s.~\pageref{46845591177f16920cd586a5baf5a625})}. Lite längre bort hör vi stojet från Lommarbadet där man hittat vässade armeringsjärn i en klump cement nedsänkt vid bryggkanten, rakblad i vattenrutschkanan och en avhuggen hand i vassen. Bortom Lommabadet finns en vikingagrav, en hästskola och ett högstadium där arbetarklassen och invandrarnas barn går. Sen finns det inte mer.

}

\small{
\textbf{Norsjöblicken}
\label{4444470133e0178de88aa4daa4d63769}
 är ett fenomen som härstammar från den västerbottniska \textsc{(se västerbotten s.~\pageref{d4b008c5143dcffb6b8c35f3876c2a19})} byn Norsjö. Det klart bristande intellektet hos byns invånare leder till en svårighet att svara på de lättaste av frågor. Frågar t.ex. någon utböling en Norsjöbo: \quotetext{Var ligger Frasses? \textsc{(se Frasses s.~\pageref{971e198d8fef127906319ec98ff657ce})}} så svarar inte Norsjöbon \quotetext{Vid busstationen}, utan svarar med en blick som liksom är fäst i fjärran. En blick lika tom som skallens innanmäte. En blick lika tom som ett bullfat efter ett bullmangel \textsc{(s.~\pageref{ecc5b41821ed829b0c3fb48d4d5389ed})}.
 Norsjöbygden är även känd för sin höga koncentration av orgelbyggare samt att ha folkomröstat huruvida man vill ha ett Systembolag eller inte. Den oinsatte kan då lätt tro att Norsjöbon inte behöver bolaget då han kokar sin sprit själv, sålunda är inte fallet.

}

\small{
\textbf{Noshörningen nelson}
\label{e439707db1c491d30a2ac06e71632fe6}
 var den första trubbnoshörningen att födas i fångenskap i Sverige. Nelson föddes 6 februari 1995 på Kolmårdens djurpark, men ödet hade försett honom med en obotlig hjärnsjukdom och han dog 20 februari samma år. Svenskarnas oförmåga att hålla sig sansade när det gäller känsliga frågor gjorde att Nelson aldrig fick vara ifred och det är troligt att mediedrevet gav honom stressfrakturer som skyndade på dödsbudet. Liket valsade runt i landet tills det år 2000 kremerades och fick sin sista vila i en glassbutt \textsc{(s.~\pageref{769564f911c36d45768dc8ae69b8af0b})} i en klippskreva på Lars Vilks konstverk Nimis.


 Historien om Nelson visar på ett tydligt sätt vilket mesigt land Sverige är som inte kan presentera ballare djurstjärnor. SVT:s \textit{Uppdrag Granskning} ägnade nyligen ett program åt att lägga fram bevis för att historien om Nelson var en strategisk PR-kampanj iscensatt av Miljöpartiet \textsc{(s.~\pageref{3e11b29518eeea19128b64869699f363})}. Bland annat visades stillbilder på en full Per Gahrton som kastar flugsvamp på en dräktig flodhäst \textsc{(se Hippopotamus  s.~\pageref{9b4609b17fea63f3f3f067fc2f465c6e})} och Birger Schlaug har bevisligen bott i Kolmårdens djurgropar under långa perioder.

 Svenska Nelson-klubbens hemsida, som idag för en tämligen tynande tillvaro: [http://hem.passagen.se/nelsons/]
 .]]

}

\small{
\textbf{Nu går slakten på Bomans vind!}
\label{0c4856aa623689a6f23126f6eb223c5a}
 är ett kraftuttryck som används i Dala-Floda med omnejd för att markera att något är nära förestående, ungefär som en kombination av de lite populärare \quotetext{det var nära ögat} och \quotetext{snart går tåget}. Bakgrunden till visdomsorden är en grisbonde i Dala-Floda som hette Boman och tydligen brukade slakta på vinden.

}

\small{
\textbf{Nudist}
\label{6283bcc4698e84866e85e43a77418abe}
 En nudist är någon som inte gillar att bära kläder av någon sund eller tvångsmässig anledning. Vissa nudister bär kläder utomhus men inte inomhus och vice versa. Andra vägrar konsekvent att bära kläder och ställer sig på så sätt utanför den sociala normen som de som bär kläder skapat. [http://www.youtube.com/watch?v=bAucG0QmC4A]

 Det finns ingen lag i Svea rike om att man måste bära kläder, men om man inte för tillfället är Knug eller Drottning (och därmed är immun mot åtal och s.k. allmän praxis av allehanda slag) så kan man bli åtalad för förargelseväckande beteende, vilket sorteras under störande av allmän ordning, ifall man visar sig naken inför någon eller några som pliktskyldigt kan tänkas påkalla farbror blå.[http://www.naturistnet.org//main.php?site_id=33]

 Unga män som gillar att sitta \textsc{(s.~\pageref{123c3e95c62201513a344526a2fec502})} hemma endast i kalsongerna på sin lediga tid kvalar in under kategorin semi-nudister.

 Se också Freikörperkultur \textsc{(s.~\pageref{40cdc17a157b501b2c84835ce6204f9c})}

}

\small{
\textbf{Nudlar}
\label{15194322d5cca8ebdd1638f4ac845817}
 säljs ofta som erbjudande om tre förpackningar till en kostnad av mellan tio och femton kronor. Det finns många olika smaker.

 HEAD2: Tillagning
 Man kokar nudlarna i vatten, tillsammans med det mystiska innehållet i medföljande kryddpåse, populärt kallad currykondom \textsc{(s.~\pageref{cd987f145bcbad879ca394d2d22e8ae6})}.

 HEAD2: Ursprung
 Snabbnudlarna uppfanns 1958 av Momofuku Ando. Lustigt namn. Låter lite som någon som idkar samlag med sin mormor.

 HEAD2: Populära Nudelsmaker

 \begin{itemize}
 \item biff
 \item kyckling
 \item fläskkött
 \item räkor \textsc{(se räka s.~\pageref{2e1cdd6fa81f4968c8c527854e0c629b})}
 \item grönsaker
 \item champinjon
 \item kryddstark
 \end{itemize}

 HEAD2: Innehåll
 På något magiskt vis (kanske på grund av ett märkligt sammanträffande \textsc{(se märkliga sammanträffanden s.~\pageref{f46282d99158f351a81b9deaff157b4e})} har nudelförpackingarnas (85 g) innehållsförtecking exakt samma näringsvärden, trots att de innehåller kryddpåsar med olika smak.

 {\textbar class=\quotetext{wikitable} align=\quotetext{left} border=\quotetext{1}
 \textbar-----
 \textbar align=\quotetext{center} \textbar Energivärde \textbar\textbar 1900 kJ/450 kcal
 \textbar-----
 \textbar align=\quotetext{center} \textbar Protein \textbar\textbar 12 g
 \textbar-----
 \textbar align=\quotetext{center} \textbar Kolhydrater \textbar\textbar 57 g
 \textbar-----
 \textbar align=\quotetext{center} \textbar Fett \textbar\textbar 21 g
 \textbar}










 Detta kan betyda att alla veganer \textsc{(s.~\pageref{2a12d5d6ae91d2f4f7d9af3cef58e75c})} kan dra något gammalt över sig.

}

\small{
\textbf{Nutleys fru}
\label{0a6824de53984f1d29c42ca39c6eb180}
 (5 februari 1964 i Kortedala i Göteborg) är en svensk skådespelare som bland annat gjort sig känd genom att medverka i Colin Nutleys \textsc{(se Colin Nutley s.~\pageref{b7e4eb146052f2edb273b55e35f4f078})} filmer. Hon debuterade i en uppsättning av Klas Klättermus vid Teater Västmanland redan vid fem år ålder. Efter att ha gått scenskola började Nutleys fru arbeta vid Kungliga Dramaten 1990 och fick sitt stora genomslag med filmen Änglagård (1992). Hon fick rollen i denna kassasuccé efter att Nutley sett henne på en reklamaffisch, vilket är ett lika sant som obehagligt stycke information.

}

\small{
\textbf{Nya Zeeland}
\label{cc538f38b19598eab98f434ece99de60}
 Oceaniens Norge. Välmående, orimligt kristet, starka band till \textit{Sagan om ringen}. Hit flyttade alla de brittiska upptäcktsresande som var för fega för att bo i Australien \textsc{(s.~\pageref{e727d8d1b3162a732c7f706d55de64f3})}. Till skillnad från Australiens är Nya Zeelands fauna nämligen inte totalt livsfarlig utan istället gullig, mjuk och vänskaplig. Som alla civiliserade länder kör man bil på vänster sida och har Elizabeth den andra, ”av Guds nåde, Förenade konungariket Storbritannien och Nordirlands och hennes övriga riken och territoriers Drottning, Samväldets överhuvud, Troslärans försvarare” som statsöverhuvud. Till frukost äter man traditionellt kiwifrukt och till lunch kiwifågel. Middagen kan variera lite men utgörs ofta av en gröt baserad på dessa två ingredienser. Och till det så klart en kopp te \textsc{(s.~\pageref{569ef72642be0fadd711d6a468d68ee1})}. Vanliga namn: Edmund Hillary \textsc{(s.~\pageref{8c30ada4e29fa9820d3f6850dc843b0c})}.

}

\small{
\textbf{Nyliberalism}
\label{a562ace16486d966be4513ea22aee287}
 Ideologi som förr kallades nazism.

}

\small{
\textbf{Näbbmun}
\label{9e3395be14cf14f92e8cd1e93eb7599b}
 är ett lika vanligt som populärt inslag i Hollywood och är för många biljetten in i filmbranschens glamorösa värld. Näbbmunnen består av organisk vävnad och silikon och ersätter den vanliga munnen \textsc{(se kakluckan s.~\pageref{7fd014af9490d51f96eba2368ecffc71})}. Näbbmunnen är som en kort näbb med vilken filmstjärnan kan plocka upp frön och korn från marken eller andra ytor. Världens idag kanske mest välkända näbbmun finns att återse i Angelina Jolies ansikte.

}

\small{
\textbf{När livet blir alldeles för mycket - Prof. Etiennes bästa gömställen, i urval}
\label{84261fbc3868ffe933f92a40b9f62c16}
 Efter ett stort antal publicerade titlar riktade till en mogen läsarskara beslöt sig Prof. Etienne \textsc{(s.~\pageref{56957a267e57df32753cf7f3b8a603d8})} för att ge något till de som lånar ut jorden till oss - barnen \textsc{(se barn s.~\pageref{5dfcc0aab2f3db925b2d51ba73e48946})}. Under sin karriär hade Professorn, på grund av sina samhällsomvälvande idéer, flera gånger behövt gömma sig på olika sätt på olika platser. I boken \quotetext{När livet blir alldeles för mycket - Prof. Etiennes bästa gömställen, i urval}, presenterar Etienne sina smartaste knep för att undvika att bli upptäckt.
 
 \begin{itemize}
 \item Måla dig brungul i ansiktet och gå in på ditt lokala théhus/opiumhåla för att försvinna i ett kakafoniskt mish-mash av dunkla karaktärer från den fjärran östern.
 \item Raka munkfrisyr och slink in i en procession bestående av franciskanermunkar nästa gång en sådan drar förbi din by.
 \item Under diskhon brukar fungera, om din familj inte har sin källsortering där. Om dina föräldrar källsorterar, be dem vänligt men bestämt att upphöra med det, då all separation av sopor från varandra baserat på människoskapade kategorier är en synd mot Gud.
 \item Undvik ventilationstrummor. De är inte lika trevliga, rena och fria från smittobärande ohyra som på film, något jag fick lära mig den hårda vägen 1987, då jag ådrog mig en riktigt ihärdig släng fläcktyfus i samband med ett inbrott i livrustkammaren.
 \item Den höfyllda oxkärran är en lika klassisk som effektiv utväg om du befinner dig i en belägrad rural medeltidsmiljö och vill undkomma den alltid lika ondskefulle dansken. Kryp helt sonika upp i höet och drag in samtliga lemmar. Hoppa triumferande ut först då du befinner dig ur räckhåll för dina plågoandar.
 \item Förklä dig till uv \textsc{(s.~\pageref{45210da832f9626829457a65e9e7c4d0})} och kryp in i en svan \textsc{(s.~\pageref{f80f1875ab3ebccf935723ba83b6da63})}. Det lilla knepet räddade mig från en bunt manikeistpatrask i södra Schleisen för ett antal decennier sedan.
 \item Om du har fördelen att vara kaffer, låt \textit{alltid} skuggorna vara ditt hemliga gömställe.
 \item Lägg dig under en filt.
 \item Vi har mycket att lära av djuren i allmänhet och de som lever i gryt i synnerhet. Underskatta aldrig värdet av ett riktigt djupt gryt. Ut och gräv ditt eget, bums!
 \end{itemize}

}

\small{
\textbf{Näs-flås}
\label{f8dc13f6d2787745d138f6bead8bdb0f}
 är ett samlingsbegrepp för de samtidigt subtila och påträngande pip- och pustljud som kommur ut ur självomedvetna människors näsor. Företeelsen är i Sverige \textsc{(s.~\pageref{b1999637949ed135b2ca03f3a38460cc})} framförallt förknippad med landsbygdsminister Eskil Erlandsson som uppvisar detta och mycket mer därtill.

}

\small{
\textbf{Näsa}
\label{eb02670054310d89c985dfe12c3ba7b8}
 är en av andningssystemets två ventiler, kan man säga, och sticker ut på central plats i anletet på ett lite osvenskt och uppkäftigt vis. Likt en snorkel pyser den ut förbrukad luft och hämtar ner ny till lungorna. Näsan har för säkerhets skull två luftgångar, ifall en skulle blockeras av ett främmande föremål. De två gångarna mynnar ut i näsborrarna som på insidan är försedda med varsin ring av hårstrån. Näsan är formad lite som en takås, vilket gör att vatten och vind effektivt hindras från att tränga in i människan.
 HEAD2: Näsan i litteraturen
 Näsa är huvudkaraktär i Nikolaj Gogols mästerliga novell \textit{Näsan}.

}

\small{
\textbf{Nästan obegagnad}
\label{bcc061cb9bf17532daff34259a1b2e36}
 I skick som ny.

}

\small{
\textbf{Näverkoja}
\label{936bd6bdc86110bbd95ffdd7a5b7f946}
 En näverkoja är en koja helt byggd av näver. Där inne kan man sitta och ha det bra när regn och rusk drar omkring utanför. En hektar björkskog är tillräckligt för att kunna bygga varsin näverkoja åt en hel familj. Blir det lite näver över kan man använda den för att tillverka en näverask att ha sin bolmört i, en näverslips \textsc{(s.~\pageref{b8fca0a91648db5415350e52dc2e9c94})}, eller en liten strut att ha på huvudet.

}

\small{
\textbf{Näverslips}
\label{b8fca0a91648db5415350e52dc2e9c94}
 Slipsen, denna betecknare för maskulinitet och respektabilitet, har inte alltid sett ut som de släta och lite glansiga modellerna som vi är vana vid idag. I gamla tiders Sverige \textsc{(s.~\pageref{b1999637949ed135b2ca03f3a38460cc})} fanns inte lyxlirarmaterial \textsc{(se storfräsare s.~\pageref{4db17005692cd83e3e946a1311b81ed0})} som siden. Behovet av att piffa till sig innan man gick på dans fanns dock, och det var då någon handfallen hantverkare satte sig ner och flätade en näverslips. Idag är näverslipsen vanlig bland människor som vill vara \textit{vildmarkschic}.

}

\small{
\textbf{O}
\label{d95679752134a2d9eb61dbd7b91c4bcc}
 framför ett ord betyder att det blir negativt, t.ex bra-obra, dåligt-odåligt, äten-oäten.

}

\small{
\textbf{Obba}
\label{ea0c519f5a3b140cda0b4ff24c701b87}
 e ba.

}

\small{
\textbf{Oberoende olympiska deltagare}
\label{7ff55a9d2c10f447f4cfe4127a41fa06}
 är sanna internationalister som vägrar att tävla i olympiska spelen under en enskild nations flagga. De tävlar därför som oberoende eller autonoma om man så vill. Oberoende har en stadig trupp och har ställt upp i flera OS.

 HEAD2: Totalt medaljtilldelning

 \begin{itemize}
 \item 1 silver (Jasna Šekarić, 10 m luftpistol)
 \item 2 brons (Aranka Binder, 10 m luftgevär. Stevan Pletikosić, 50 m gevär)
 \end{itemize}

}

\small{
\textbf{Occupy Umeå}
\label{aef77b09cb0a602bb20318e5e01e8361}
 var en av de mer slagkraftiga proteströrelserna i Umeå under året 2011.

}

\small{
\textbf{Odiagram}
\label{01cd5463aa252bcb6c6a4834a3eb528b}
 Motsatsen till diagram \textsc{(s.~\pageref{d08cc3195cba7dd812ab0652a68bdeda})}.

}

\small{
\textbf{Odon}
\label{26f12a6ce29ddb468be92bd630ef02af}
 är ett bär som växer vilt lite varstans i Norden. Till utseendet påminner det om blåbär men smakmässigt är det betydligt mera trist och vattnigt. Odonbärets storhet ligger istället i dess lämplighet som bas för vinbryggning. Detta faktum i kombination med att namnet låter ganska mycket som Oden har lett till att bäret tidigt blev ett vida spritt afrodisiakum på svenska höskullar och kökssoffor \textsc{(se kökssoffan s.~\pageref{d1c2d6488fde9b41b5c6b2a03c5fd79c})}. Den gamle växtofilen Carl von Linné (RIP) skrev i \textit{Flora Oeconomica} år 1749 att bären \quotetext{behaga barn och kalkone-ungar, men åstad komma ofta någon upphetsning}. Varför Linné valde en flock nykläckta kalkoner som referensgrupp vet vi inte, men plötsligt blir det inte lika märkligt att många trodde på troll och häxor \textsc{(se trollpunk s.~\pageref{5e806ae90a53e9328e1e467a4d7b1b37})} vid denna tid.

}

\small{
\textbf{Offofili}
\label{a2a67452f0c9f4c7c44f6ae6dcf4b355}
 Ur boken 'Udda, udda - anonyma berättelser' utgiven på Kindbergs förlag 2010:

 \quotetext{Jag har en bekännelse.

 De gånger jag dristar mig till att berätta, brukar jag mötas av oförstående blickar och frågor. Ibland av rena fördömanden. Några gånger har människor sagt det rent ut till mig: Du är pervers!

 Jag är offofil. Jag tänder på kontorsutrustning.

 Det är en sensuell njutning att lägga handen på en riktigt välgjord textilryggspärm. Jag kan gärna ha ett par extra i bokhyllan, bara för att. Jag är en kännare av olika perforeringar i kollegieblock, och de olika handlag som krävs för en snygg avrivning. Jag vet inte hur många block jag köpt i onödan genom åren, bara för att få känna känslan av ännu ett i min hand.

 Och det är inte bara block och pärmar som lockar. Otaliga är de timmar jag researchat de snyggaste och tystaste skrivarna. Jag har sett ut en särskild modell, en Canon, som jag ska köpa om jag blir rik. Mmm…

 Papperstuggar. Häftapparater. Kopieringsmaskiner. Kulspetspennor. Arkivmappar. Filofaxer. Plastfickor. Tidskriftssamlare. Säg vad ni vill, jag går igång. Andningen blir djupare och min stämma myndigare. Jag är hemma. Det är inte för inte som mitt unika och älskade skrivbord har en mycket central plats i mitt hem.

 Naturligtvis håller jag med om att det här inte är riktigt friskt. Även om jag aldrig har gått så långt att jag gökat med pärmarna.

 Jag har funderat på orsaken till att det blivit så här. Är det bokföringstraumat, när jag som fjortonåring valdes till kassör i en förening, långt innan jag var emotionellt mogen för det? Är det astrologiskt? Är det genetiskt? Vad skulle Freud ha sagt? Jag kommer nog aldrig att få veta.

 Jag har ändå börjat få en viss kontroll på det nu. De flesta block och pärmar jag köper behöver jag verkligen. Det är flera år sedan jag sist försatte mig i självdestruktiva situationer, bara för att få omsluta mig med en trygg kontorsmiljö. Men jag kommer nog aldrig att komma ifrån själva dragningen, lusten, njutningen.

 Jag känner mig udda, utanför. Jag undrar om queer-rörelsen har plats för en sådan som mig.}

 Offofili är en mycket sällsynt sexuell preferens, och det finns inga kända föreningar varken i Sverige eller internationellt. Den hittills enda offofil som gått ut offentligt är Jennifer Love Hewitt, känd från bland annat Ghost Whisperer. [http://www.people.com/people/article/0,,20354240,00.html]

 Klassiskt omslag från den bland offofiler mycket populära tidningen Steamin' Stationery:

 File:Steamin 11 06.jpg \textsc{(s.~\pageref{818c820a3e93cd3dceb6fd0331eeada0})}

}

\small{
\textbf{Offrim}
\label{55240c8ee02604df7142b083119c1cec}
 är rim som inte riktigt rimmar, men nästan. Typ bil och syl eller lingon och bengan. Användbart för den avancerade poeten eller lindrigt anarkistiska grötrimmare. Riktiga anarkister rimmar cykel med burk, oavsett vad svenska akademin säger om saken.

}

\small{
\textbf{Ohemul}
\label{91b8873590abd15ec344c2ba93d015cd}
 är ett adjektiv synonymt med exempelvis \quotetext{orimlig} och \quotetext{orättvis}. Vanligtvis talar man om ohemula priser men ordet fungerar lika bra till mycket annat, varför inte ett ohemult utseende exempelvis? Skickligt brukat kan det också användas för att förstärka något positivt, ungefär som när man säger att något är \quotetext{djävulskt gott}. Svenska Akademien betalar ut fem kronor varje gång man använder ordet, i ett försök att vända dess tynande tillvaro. Skicka bara in bevis.
 HEAD2: Ohemul i populärkulturen
 De lyckliga kompisarna sjunger i låten Egons Fest från skivan \textit{Tomat}:
 \quotetext{Egon han var ohemult förmögen
 han hade spekulerat i en sjöbod på smögen}

}

\small{
\textbf{Oi!Per}
\label{7e0128ca1fb41d4d681972e4b13aafda}
 borde egentligen inte få ta plats på Nissepedia. Han svek sitt forsterland Ljusdal för en herrans massa år sedan. Men oi!Per är så löjligt tuff så han klarar sig. Dessutom ligger inte Ljusdal i det egentliga norrland så någon landsförrädare i ordets rätta bemärkelse är det alltså inte tal om.

 oi!Per är väldigt cool. Han är anarkoleninist och gillar bilar. Han lider av dålig självdisciplin och är för det mesta arbetslös. Dock så har oi!Per en herrans massa bilar, de flesta väldigt rostiga. Eftersom han har så många brukar det dock alltid vara någon som för tillfället kommit igenom Svensk Bilprovnings nålsöga. oi!Per har även en svets, samt dålig belysning i sin enorma verkstad. Höjdpunkten i oi!Pers musikliv måste ha varit när han körde crustpunkbandet Scumbrigade på turné i Europa. Så sent som häromdagen åt han päronhalva \textsc{(s.~\pageref{cc9c1bfa2ec4eaed89ca86a1b63e3a45})}.

 oi!Per har, tillsammans med artikelförfattaren, blivit hotad med grovt fysiskt våld på internetforumet \quotetext{sävarturbo}

 Klassiska oi!Per-citat
 \quotetext{jag är liksom det normala navet i ett hjul med krokiga ekrar}

 \quotetext{Fan, varmvattnet fungerar inte... [tystnad] ...nej vänta det är lugnt, jag drog åt fel håll}

}

\small{
\textbf{Oidipuskomplex}
\label{58fae9174c458ab30624d6c4f38da4f8}
 Det finns två typer av oidipuskomplex, den osunda där man vill mörda sin far för att lägra sin mor var populär i antikens Grekland. Den sundare varianten är den där man utåt sett inte vill bli som sin far men sakta i lönndom anammar dennes gubbighet och vanor.

}

\small{
\textbf{Ojoj-fågel}
\label{59d85af040670e462418d666ca38f277}
 n (\textit{ Albatrossi Skrotumus }) är en albatrossliknande fågel som gjorde det evolutionära snesteget att kombinera långa vingar med en stor hängpung. Den är idag utdöd av förklarliga skäl.

}

\small{
\textbf{Old Black}
\label{4442e79a16d1a2ab536eb40394a53dbb}
 är det första spåret på drone-bandet Earths fullängdare \quotetext{Angels of darkness, demons of light} och är en cover på Cacka Israelssons gamla dänga \quotetext{Gamle Svarten}. Eftersom man varit fräck nog att spela in låten instrumentalt följer här länkar till texten [http://www.saw.se/Vistext/visaresult.aspx?id=133] och Earth-covern [http://youtu.be/Lb-3eBlv_qE] så att den som så vill kan stämma upp i sång.

}

\small{
\textbf{Old ox}
\label{954ddcc10ad941a7ee93e0584ee6a78b}
 Ölet Old ox (eng. 'gammal oxe' Uttal. /gamoksä/) bryggs av Spendrups.

 HEAD3: Copytext från lanseringen 1957:

 \quotetext{6.9 \% starkt. Bryggt för gubbar som sätter sig i fåtöljen efter en lång dag på kneget och tänker att de är förjävla slitna. Massiva. Tunga i kroppen. Långsamma, men stadiga som berget. Ingen slank sötnos direkt. Men vem fan vill vara det? Ingen smäcker kycklingfilé om man säger. Snarare en oxe. Fett kramar om massiv muskulatur. Det skaver i klövarna. Mulen är narig. Men man trampar på. Inget gnäll. Old fucking ox.}

 HEAD3: Old ox och maskulinitet

 Den maskulinitet som uttrycks i old ox beskriver till synes den hos någon som jobbat på fabrik i 40+ år av sitt liv. Föga överraskande säljer ölen bäst hos aspirerande akademiker i 20-års åldern med pinnsmala armar och flaskbottnade brillor.

}

\small{
\textbf{Olika betydelser}
\label{e1ad68f47d309447a41e211a5b65e10b}
 {{Olika betydelser\textbarRäva}}

}

\small{
\textbf{Oliver/Dawson Saxon}
\label{83862ce9a45daa581e109209003ec023}
 är ett brittiskt metalband som består av tidigare medlemmar i metalbandet Saxon. Med tanke på att Saxon spelade i Smedjebacken och på finlandsfärja härom året kan man tänka sig att Oliver/Dawson Saxons karriär ligger på ungefär samma nivå som Skinned Alives \textsc{(se Skinned Alive s.~\pageref{88b01b69ece92f30d83711e8a65fd542})}, men utan den senare gruppens obestridliga status i undergroundkretsar.

}

\small{
\textbf{Olja}
\label{59505758215c74f673dd94e519ad459c}
 är en mörk trögflytande vätska som består av förmultnade dinosaurier (skräcködla) \textsc{(se skräcködla s.~\pageref{60dfc16d3c521fea596aa4c65bc1e3f5})} och gamla plankton. Den fanns länge som ett naturligt inslag i naturen och bidrog till skapande av social hållbarhet och kulturell utveckling. Numera utsätts oljan för en starkt kritisk lobbying, främst från arbetslösa nyutexaminerade arkeologer.

}

\small{
\textbf{Oljegark}
\label{d78fbbc214d52206f58476f02f66f0b6}
 En oljegark är en person i någon av de forna Sovjetstaterna som skor sig (eller har gjort det tillräckligt mycket för att numera kunna ägna sig åt att t.ex. köpa kända fotbollslag om dagarna) i gangsterstil på olje och energiaffärer. Det vanligaste är att oljegarken genom personliga kontakter, mutor eller mord och hot fått ett stort innehav av aktier eller en viktig post i något statligt eller privat energibolag strax efter att kapitalismen införts efter Sovjets kollaps. De nämnda taktikerna har många ojegarker lärt sig genom att tidigare i livet haft poster inom KGB eller FSB, eller så är de tillräckligt tjenis med någon inom FSB för att denne skall beordra grovjobbet åt dem.  Liksom deras kollegor i maffian så håller sig oljegarkerna från att ha politiska poster och nöjer sig med att låta politrukerna vara knähundar åt dem. Även andra med informell makt förväntas att medvetet rätta sig in i ledet.[http://www.dn.se/nyheter/varlden/fakta-anna-politkovskaja-1.898373]

 Ibland kommer oljegarkerna inte överens sinsemellan, vilket kan sluta med att någon får en lägerplats i Sibirien istället för att fortsätta sitt liv i materiellt överflöd.[http://www.svd.se/nyheter/utrikes/ny-rattegang-mot-rysk-oligark_3082301].

 De flesta oljegarker har vett nog att hålla folket de skor sig på strax ovanför existensminimum samt via statliga medier ge ett sken av att det är politrukerna som styr så att missnöjet riktas åt andra syndabockar än de själva. Vissa oljegarker har ett självdestruktivt ego och vill själva ratta landet eller regionen på en s.k. legitim post, säg som president. Om oljegarken till synes inte sköter den posten \textsc{(s.~\pageref{cd13d688571681e426231485b732444b})} så väl så kan denne bli osams med folket och förlora sin post till en rivaliserande oljegark (antingen Pentagon eller Kreml-sponsrad) som ger ett sken av att vara demokrat, alternativt till någon som till en början kommer från en folklig opposition. Detta kan ske på en fredlig eller ibland direktdemokratisk väg. [http://www.telegraph.co.uk/news/worldnews/asia/kyrgyzstan/7563492/Kyrgyzstan-protesters-kill-government-minister-as-violence-escalates.html]

 Medier i väst gillar att skriva om vissa oligarker mer än andra. Några som figurerar i olika överklassrepotage brukar vara Roman Abramovitj, Viktor Zubkov, Michail Chodorkovskij mfl.[http://www.vk.se/Article.jsp?article=339826]

 För den som inte sitter på så mycket öststatskunskap kan det vara lätt att blanda ihop en oljegark med en \textit{oligark} i rapporteringen. Man särskiljer de två genom vilket område de i huvudsak skor sig på. Många i toppskiktet i öststaternas överklass pendlar i gråzonen mellan att vara en oljegark och en oligark. Vanliga öststatsnäringar som oligarker till stor del är inblandade i är vapenindustri, narkotika, människohandel och skattesmitning - ofta har de fingrarna i flera syltburkar simultant.

}

\small{
\textbf{Olle Findahl}
\label{433e1dc6d01073b9b2b4a5a6294d0597}
 är professor och tillsammans med Birgitta Höijer \textsc{(s.~\pageref{9c402d608d4c87384133bd5f8b522574})} författare till boken \textit{Text och innehållsanalys – En översikt av några analystraditioner}.
 Se även: Kvantitativ innehållsanalys \textsc{(s.~\pageref{8cbd40215a0453bdd47cd6ef47c53ec2})}.

}

\small{
\textbf{Olm}
\label{417fdd822daffa6a09872ff90a8ae4e2}
 en, \textit{Proteus anguinus}, är ett stjärtgroddjur som lever i grottsjöar. Vad dom fyller för syfte är oklart men dom ser roliga ut. Enligt Steve Irwin ingår de i humordelen av naturens näringskedja.

}

\small{
\textbf{Olof Palme}
\label{702b78623785546fb9c9890222376178}
 är död.
 Polis, media och säpo gick tidigt ut med att han blev skjuten och lurade därmed en hel nation att så var fallet, i själva verket dog han i en hjärtinfarkt efter att ha blivit jagad genom skogen av två fiskevårdare. Tjuvfiskare var han den jäveln, han hade näsa för sånt.

}

\small{
\textbf{Om kriget kommer}
\label{86325b0844aed9a3678fc492c795ba16}
 Inte Om utan när Ryssen kommer finns det några saker man kan göra.
 För det första kan man leta efter en telefonkatalog \textsc{(s.~\pageref{a1c3d8187f7afc13f933d7d93b27f536})} från förr, längst bak finns instruktioner. Eller om du är postmodern surfa in här med din Ipad \textsc{(s.~\pageref{09401fded433c34709fd1f1872728162})}.
 Annars så fly söderut, återsamling vid skolan i Blattnicksele. Och glöm inte: Varje meddelande att motståndet skall uppges är falskt!

 http://www.scribd.com/doc/4069002/Om-kriget-kommer-ur-Telefonkatalogen-1984-85 \textsc{(s.~\pageref{41554a668d4346d34799867a40b2dea7})}

}

\small{
\textbf{Oman}
\label{ad437fd2f44c7b3f8208a162604d81d0}
 -Oman är ett land i Mellanöstern.
 -Oh man!

 Se även: Jemen \textsc{(se Jemen s.~\pageref{908706852c5107d727d8d0eeffe8782d})}

}

\small{
\textbf{Omdirigering}
\label{8b0eadf48add253e5e22ec5680410d9c}
 Räva \textsc{(s.~\pageref{c13f687883c1eb0be3be218fff63e6b8})}

}

\small{
\textbf{On the floor}
\label{71521aaa8b27f4d00d5b020276b3b0e4}
 Här följer en översättning av j-lo feat pittbulls kioskvältare \textit{On the floor} som förövrigt i detta nu ligger på 6 plats på billboard listan. Fast för att förenkla detta lyriska mästerverk till en greppbar nivå kör vi den på, just det, modersmålet svenska.
 här kan mästerverket även avnjutas bildsatt: http://www.youtube.com/watch?v=t4H_Zoh7G5A

 J-LO \textsc{(se Hur man ritar ett snyggt lodjurshuvud s.~\pageref{85f12831da9d7403326be028c34be8a9})}

 det är en ny generation(Mr. Worldwide)
 av festande människor
 Nu, gå till golvet
 Dale, gå till golvet (Red One)

 Låt mig presentera dig för mina festkompisar
 på klubben...

 [Pitbull]
 ja är lös
 och alla vet att jag kliver av tåget
 älskling det är sant
 jag är som filmen inception, jag leker med dig hjärna
 jag sover inte jag snoozar
 jag spelar inga spel så bli inte förvirrad nej
 för du kommer att förlora yeah
 nu pumpa upp det
 och backa upp som en Tonka truck
 Dale!

 [Jennifer Lopez]
 om du går ut hårt så kommer du bli tagen på golvet
 om du är en partygalning gå till golvet
 om du är et djur slit sönder golvet
 bli svettig på golvet
 Yeah vi jobbar på golvet
 sluta inte röra på dig på golvet
 håll up din dryck
 lyft upp din kropp och släpp den på golvet
 låt rythmen ändra din värld på golvet
 du vet att det är vi som styr skiten på golvet i kväll
 Brazilien Marocco
 London till Ibiza
 rakt till LA, New York
 Vegas till Afrika (Dale!)

 [Chorus]
 dansa bort natten
 lev ditt liv, och håll dig ung på golvet
 dansa bort natten
 ta tag i någon och drick lite mer
 Aaa la la la la, lalalalalala la laaaa
 i kväll ska vi vara DET på golvet
 Aaa la la la la, lalalalalala la laaaa
 i kväll ska vi vara DET på golvet

 [Verse 2]
 jag vet att du har det klappa dina händer på golvet
 och fortsätt rocka, rocka på på golvet
 om du är kriminell döda det på golvet
 sno det snabbt på golvet,på golvet
 sluta inte fortsätt att röra på det
 håll upp din dryck
 det blir illamående det blir sjukt på golvet
 vi slutar aldrig, vi vilar aldrig på golvet
 om jag inte har fel så kommer vi förmodligen att där på golvet
 Brazilien Marocco
 London till Ibiza
 rakt till LA, New York
 Vegas till Africa (Dale!)

 [Chorus]
 dansa bort natten
 lev ditt liv ,och håll dig ung på golvet
 dansa bort natten
 ta tag i någon och drick lite mer
 Aaa la la la la, lalalalalala la laaaa
 i kväll ska vi vara DET på golvet
 Aaa la la la la, lalalalalala la laaaa
 i kväll ska vi vara DET på golvet


 [Pibull]
 det där badonka donk är som en bagage full av bas på en gammal cheva
 allt jag behöver är lite vodka och lite ... kokain
 och titta… du blir donkeykongad
 … om du är redo för att saker kommer att bli tunga
 jag går till golvet och beter mig som en tok om du tillåter mig
 Don't believe me just vet me
 mitt namn är inte Keath men jag förstår varför du är jobbig mot mig.
 L.A. Miami New York
 sej inget mer gå till golvet
 Aaa la la la la, lalalalalala la laaaa
 i kväll ska vi vara DET på golvet
 Aaa la la la la, lalalalalala la laaaa
 i kväll ska vi vara DET på golvet

}

\small{
\textbf{Onorrland}
\label{a262149ea1d67f4e950bea957df3ff19}
 Allting söder om Hakkas, Pitälven, Ö-vik eller Gävle, beroende på vem man frågar. För andra syner på saken se Lapplands Väsby.

}

\small{
\textbf{Onsdag}
\label{7c5c8bad63ebb03416f8b6e9ccc4f0f6}
 en, eller lill-lördag som skämtare brukar säga, är en dag mitt i arbetsveckan. På onsdagen tycker vissa att det är okej att dricka alkohol \textsc{(s.~\pageref{11c589cba1a208e0359048a78e6b88b8})} (om man inte blir för full).

}

\small{
\textbf{Opel kadett}
\label{482dbfb056cb7085f9dac53c80264305}
 Vägarnas skräck, trafikpolisernas nemesis. Detta tyska fartvidunder kan skryta med 58 hästkrafter under huven (gäller högkompressionsmodellen 1.2S) och en prestanda du tidigare endast vågat drömma om. Med en modern utformning och snabba linjer blir du snabbt bygdens samtalsämne om du kör kadett.

 Kör rätt! Kör kadett!

}

\small{
\textbf{Ordvitsar}
\label{2c323f88d04e1e9a99ed4bb36b601827}
 Ibland kan ett och samma ord eller uttryck tolkas på flera sätt. Ofta blir det bara otydligt vad som menas, men stundtals uppstår avancerad och fantastisk komik.

 Det händer också att resultatet uppfattas som fyndigt, men ändå tråkigt.

 HEAD2:  En Svensk Tiger
 Bertil Almqvist hittade på 'En Svensk Tiger' på uppdrag av Statens Informationsstyrelse under andra världskriget. Tvetydigheten gjorde det lätt att skapa en gullig logga med en blågulrandig tiger, när budskapet egentligen var att folk skulle hålla käften.

 Detta alster var resultatet av många års hårt tankearbete och ratade förslag som 'En Svensk Stör', 'En Svensk Åker' och 'En Svensk Lever'.

}

\small{
\textbf{Organiserat grytlock}
\label{821eb16f170fb1374bfbc293842cd6e1}
 Hi-hat:en på ett modernt trumset.

 Källa: Auktionsutropare, 60+ från Rökå \textsc{(s.~\pageref{b06106b8f786098f1ff569a4f75dc3c8})}

}

\small{
\textbf{Orgasmatron Andersson}
\label{992f857a2415202c7eb4b9f973ea11a0}
 är en gitarrist \textsc{(se gitarr s.~\pageref{a08bf8420208934bc59c7ed7385d4308})}, nervklen ensamstående trebarnsfar och en av förgrundsgestalterna i det svenska Oi!-bandet Skinned Alive \textsc{(s.~\pageref{88b01b69ece92f30d83711e8a65fd542})}. Orgasmatron utvecklade redan tidigt ett öra \textsc{(s.~\pageref{c4774ec92abe06f5664e18f44446d7e7})} för enkla gitarrslingor, en problematisk maskulinitet och en läggning för skummande öl och graviterade således mot Oi!-punken. Tyvärr utvecklade han ganska snart också sjukdomsliknande reaktioner på alkohol och kött. Han kan därför till exempel inte bära oxblods-Dr. Martens, som ju är gjorda av ett slags formpressat kött, utan tvingas istället ha jumpaskor \textsc{(se sneakers s.~\pageref{a1743c0d39461290efc551490aafc1e2})}. Men lika glad för det är han, för han kompenserar detta på många andra vis. På senare år har han ofta setts hugga ved på ett enligt grannarna alarmerande vis.

}

\small{
\textbf{Orientdressing}
\label{e445b99d659acec84b6ab9f56fdf091c}
 sätter man på pizza.
 Obs! Dock inte i södra Sverige!

 Se även gudarnas nektar, Hawaii-pizza \textsc{(s.~\pageref{742e4954c36e42931521b0a417511c7c})} och ambrosia.

}

\small{
\textbf{Orkidé}
\label{af2a249e617dd0bb299ba8e533777222}
 -Ork must kill hobbits. Then ork will feast on mud and worms.

}

\small{
\textbf{Orm}
\label{51b0b5a943ae2b04076f7a6cb037afd6}
 Det var ormen som ställde till en massa skit för Adam och Eva. Den lurade i dem att de skulle käka frukt och då blev gud \textsc{(s.~\pageref{91e49146121c992aab11a19c77e26cf0})} arg och sen dess måste man arbeta och stå i, gå till affären i snöslask, se till att man har med sin astmamedicin som man ändå glömmer, visar det sig, samt tömma luddfiltret. Vissa hårdrockare gillar ormar för att de tyckar att ormar är balla, men ormar har inga armar eller ben så så jävla balla kan de inte vara.

}

\small{
\textbf{Ornässtugans dass}
\label{17d2effff6c1590dbff6a7ac39f46a19}
 är lokalen där Sverige slutgiltigt kastade av sig det danska oket (Skåne ej inräknat) och började resan mot modern nationalstat. Det var här Gustav Eriksson genom en listig fint förde danska knektar bakom ljuset för att i ensam majestät ta sig till Stockholm \textsc{(s.~\pageref{edcd259e0a03c7ab70feb186bae19f13})} och tillträda tronen och avskaffa katolicismen. Christian Tyrann hade inte en sportmössa \textsc{(s.~\pageref{a6a5f90825518f69ef5df7d814de68d8})}.

 Dasset står idag tillsammans med Haile Selassie I:s \textsc{(se Haile Selassie s.~\pageref{9accb6bc1893f934737ecb7710da49a4})} gravplats som enda avträden på FN:s världsarvslista.



 försök att återinföra katolicismen genom att påvisa att Jesus \textsc{(s.~\pageref{110d46fcd978c24f306cd7fa23464d73})} minsann var en flitig träckare han också.]]

}

\small{
\textbf{Orolighetskeps}
\label{f151477c1567a3a7ee990758a8d46a8d}
 en är en röd keps med företagslogotyp \textsc{(se kepsar med olika företagslogotyper s.~\pageref{6a414633590fd4cd6d6ac64798d14c14})}, närmare bestämt Gustavssons Åkeri och El AB, som finns på kontoret på Malå Sågverk \textsc{(s.~\pageref{39a99a78876fd85985cc06fa0baa3c1a})}. När någon av de anställda bär denna keps vill de signalera till sina medarbetare att de är oroliga över något. Om de vill visa en medarbetare som inte är närvarande på kontoret skickas ett MMS innehållande ett porträtt där man är iklädd kepsen.

}

\small{
\textbf{Orrkammens isolering och gokart}
\label{15f0f078bbdd9632fd4921ae58abfd5a}
 är en sorts smutt \textsc{(s.~\pageref{d9114ffee4f2dcee302ae2b19ce5eea9})} kombinationsaffär \textsc{(s.~\pageref{54328b839527f9917e5d057845b4fc5c})}, eller inte affär men företag i alla fall, som ligger i Orrkammen, Norrbotten \textsc{(s.~\pageref{0e8c003b75982032cde152609ee94154})}. Ägaren driver under vinterhalvåret en firma som sysslar med isolering, medan sommarhalvåret ägnas åt dennes stora passion, gokart. Banan \textsc{(s.~\pageref{aec7bd708ed2ad3435b9a9883ac7f45c})} är belägen där riksväg 95 korsar tvärbanan från Jörn, inte så långt ifrån Arvidsjaur. Här finns även ett café med våfflor till barnen och kaffe \textsc{(s.~\pageref{a51a0cac0ce374a853d2359417debc28})} för de vuxna. Vuxna FÅR såklart äta våfflor också, men barn får fanimig inte dricka kaffe, hur skulle det se ut?

}

\small{
\textbf{Oscar Dronjak}
\label{64fa9a25a7c25348f50ee9f70554a7f1}
 är en jättelång göteborgare som spelar gitarr i bandet Hammerfall. Han lyssnar inte på Mob 47 \textsc{(s.~\pageref{9955900bded21660e7f4e15ae8d23e3a})}, för hade han gjort det hade han vetat att rustning är ett brott.

}

\small{
\textbf{Oscar Wilde}
\label{379811323e22c1397246d12721dc9fc8}
 var en homosexuell irländare som skrev noveller och pjäser samt levererade snärtiga aforismer åt höger och vänster. Hör du ett lite ironiskt och underfundigt citat och avkrävs att gissa vem som sagt det är Oscar Wilde en mycket bra gissning. Är det inte han så är det Mark Twain som sagt det.

}

\small{
\textbf{Ostbricka}
\label{bf06c995c523e159eb93017810ee8f44}
 är en samling ostar på en bricka \textsc{(s.~\pageref{0072a4ca9825dbec7dfa6e6cb9b23022})} eller tallrik. Samlingen måste innehålla minst två sorters ostar för att vara en samling och kan serveras med en liten frukt \textsc{(s.~\pageref{7b0faed51fc6c55d2431ed677d0989ad})}, tex päron \textsc{(se päronhalva s.~\pageref{cc9c1bfa2ec4eaed89ca86a1b63e3a45})} eller druva.
 Ordet ostbricka kan också användas som en klassmarkör \textsc{(s.~\pageref{6a9c0c6836a0777442468f821837e795})} då lite finare folk inte säger ostbricka utan namnet på osten de ämnar äta tex \quotetext{brie} eller bara \quotetext{ost} rätt och slätt \textsc{(s.~\pageref{a9cde01124ca41f23d6044b3ba27b979})}.

}

\small{
\textbf{Osthyvel}
\label{c09f8306965e1344e1102a46d084cab9}
 En slags kortare hyvel \textsc{(s.~\pageref{1668e2298ba60f14922e2cca6aa96538})} gjord för ost, och inte alls \textit{av} ost, som man annars kunnat tro.
 Osthyveln uppfanns 1925 av den norske möbelsnickaren Thor Bjørklund som förövrigt också är upphovsmannen bakom den beryktade smörhyvel som dock inte fått samma genomslag. Världens största osthyvel står att finna i Ånäset \textsc{(s.~\pageref{60c60004be04d826c3dd64de3b90cd05})} alldeles vid E4.

}

\small{
\textbf{Ostkaka}
\label{ec7f9709ebcba225373e556544e0c6d7}
 Resultatet av att en snubbe som hette Örjan stod i köket, duktigt bakfull och i y-front, och tänkte för sig själv: \quotetext{... Jag gillar ost... Jag gillar kakor... Hmmm...}.

}

\small{
\textbf{Ostmacka}
\label{2e2a02f9cf463d37a5ab2cae4e0bed2a}
 är en av våra mest vanliga och populära smörgåsar. Mackan består till 70\% av bröd, 20\% av ost och 10\% av smör (procentsatser approximativa). Överstiger brödhalten 75\% är ostmackan bröstark \textsc{(se bröstarkt s.~\pageref{9d9073d6a3c73a175f38e959d256cbff})}. Ostmackan är vanlig som inslag i frukostmåltiden samt som tillbehör till lättare måltider så som soppa och sallad.
 HEAD2: Tillverkningsprocessen
 Bästa sättet för den som önskar lära sig själv tillverka en macka är att memorera följande steg:
 \begin{itemize}
 \item Anskaffa brödskiva. Här måste man välja mellan hårt och mjukt bröd. Dessa olika sorter har olika fördelar. Väljs hårt bröd anskaffas detta genom att bryta en bit av en knäckebrödsskiva eller genom att fiska upp en skiva ur brödkartongen. Väljs mjukt bröd är det vanligt att det krävs ett visst mått av sågande medelst brödkniv i brödlimpa.
 \item Smöra skivan. Skrapa upp en lagom mängd smör ut smörpaketet med en trubbig kniv. Tags för mycket smör i detta läge kan smöret återföras till paketet, men med lite övning och tålamod kommer detta så småningom kunna undvikas av den skicklige frukostätaren. För med jämna, lätta gester kniven över brödets ena yta så att smöret fördelas jämt.
 \item Placera ost på mackan. Hyvla två (i vissa extrema fall tre) skivor ost från ostpjäsen med hjälp av osthyvel. Vissa rutinerade frukostätare använder också hyveln för att på ett mondänt vis placera ostskivan på brödet, men detta är knappast praktiskt nödvändigt - en mer vardaglig metod är att med den fria handen plocka upp ostskivorna från hyveln och placera dem bredvid varandra längsmed brödskivan.
 \item Ej nödvändig garnityr. Den som så önskar kan placera en handfull gurkskivor, paprikabitar eller en späd basilikakvist på ostskivan för att på så vis addera lite färg till den. Tillskyndare av denna metod menar ofta att man njuter av en ostmacka inte bara med smaklökarna utan också med synapparaten, vilket antyder att det visuella inte är oviktigt när det gäller en mästerligt tillverkad ostmacka.
 \end{itemize}

}

\small{
\textbf{Ououou}
\label{b60aaad82eae375b78945ccdf16817c7}
 är det lite \quotetext{bluesiga} ljud som John \quotetext{The fog \textsc{(s.~\pageref{576875ef0042ff21c04f5f1b9377d4e7})}} Fogerty diggar fram efter första raden i Creedence Clearwater Revivals låt \textit{Long as I can see the light}:
 Put a candle in the window Ououou.

}

\small{
\textbf{Overallstudenter}
\label{09a5062cf884d746996bf5a9f3669d1b}
 finns på universitet, stundtals lämnar dom dock detta område varför det kan vara bra att känna till lite om arten.
 HEAD2:  Kännetecken

 Utan att bli Anti-speciecist \textsc{(se Anti-speciesism s.~\pageref{f520aed28c98e2d9f59ce26d3c8bc523})} vilket anses mycket fult i vissa kretsar, kan man lätt jämföra overallstudenter med djur \textsc{(s.~\pageref{c5acdb95a7feeab7475c8df4cd4c20c5})}. Dom kan inte tala begripligt, rör sig i stora flockar, är påfallande ofta kälkborgare \textsc{(s.~\pageref{0f34b469a48952e93688861083ace75a})} och så vidare. Overallstudenter envisas med skojiga stavningar på saker och ting och svänger sig med en viss jargong \textsc{(se rå, men hjärtlig jargong s.~\pageref{d6d2daa05753e48d8250a9b74bb0f163})} pattar och brännvin \textsc{(s.~\pageref{ff49ececa32cff978496a39635496f46})} blir det trots att ingen av dom nånsin haft en tjej och inte kan supa som en riktig karl.Precis som namnet anger bär dom overall, denna är färglad och har skojiga texter här och där.

 HEAD2:  Overallernas semiotik
 Den uppmärksamme har redan märkt att overallerna har olika färg, det här beror på att studenten i fråga läser ett visst program. Detta varierar från stad till stad även om mönster finns. Kemistudenter har ofta gröna overaller, fysiker blå. Alla lappar som sitter på overallerna kommer från olika fester, så kallade sittningar, som studenterna besökt, eller från kurser de läst. En del lappar är bara rena budskap till eventuell betraktare som t.ex. \quotetext{Jag kommer ha högre lön än dig}. Det finns även overaller som har lite olika färger, t.ex. ett grönt ben på en blå overall, det betyder att två studenter skojat till det och bytt med varandra. Har krage eller slutet på ärmar bytts, säg en krön krage på en blå overall, så berättar detta för oss att två studenter legat, eller på annat sätt varit intima med varandra.
 HEAD2:  Historik

 Innan Overallstudenten började på universitetet bodde han, för det är så gott som alltid en han, hemma hos mamma och pappa i Bromölla eller Mariestad. Det är förklaringen till att han nu måste förpesta tillvaron för invånarna i den stad som drabbats av ett universitet. Under sin tid på naturvetenskapliga programmet var han precis som alla andra, dvs. en tönt. Nu ska han kompensera. Och det med råge.

 HEAD2:  Bekämpning

 I väntan på kommunismens genomförande är järnrör \textsc{(se metallstång s.~\pageref{6b45527e41ce216a150d4ac5950322bd})} och en säck mjältbrand de säkraste metoderna.Undvik fysisk kontakt, försök inte \quotetext{diskutera} utan slå bara hårt och länge. Mjältbrand kan lämpligen placeras i någon av overallstudendernas svartkrogar eller spridas vid \textit{Kårtegen},\textit{LURF}, eller andra event där dom flockas.
 HEAD2:  Se även:
 Den oambitiösa studenten \textsc{(s.~\pageref{773fb9013bfd8af98ed84fe0abc8748e})}
 Den ambitiösa studenten \textsc{(s.~\pageref{02257ef6d6da8e0f0721e2758eec3c71})}
 Utbytesstudenter \textsc{(s.~\pageref{397699f3732b0c22f3c532a111697539})}

}

\small{
\textbf{Pada}
\label{867f3a3cceaf3cfbcbcc47a5adc6cb54}
 är född på den senare delen av 1900-talet och syns ofast norr om polcirkeln.
 Han kan vara en aning skygg men lockas lättast fram ur gömmorna ifall du är iförd reflexväst, men han lyssnar inte till orden 'Kom' och 'Snälla'. En gång bodde han i Umeå men det gav honom grava alkoholproblem så han fick vända hem till pojkrummet.
 Nu mer spenderar han tiden på att lämna in sin gubbil på rekond. Han blöder automatiskt näsblod om temperaturen överstiger +17 grader, all vätska i Padas kropp traspoteras ut genom armhålorna.

 \textbf{Sport}:
 Pada gjorde en bra match mot Masugnsbyn i en fotbollstunering i Junosuando sommaren 2011.
 På vägen hem från segerbanketten senare den kvällen plockade han sveriges längsta rallarros.
 När han var mindre och spelade landhockey ute på gatan så for han ofta hem och kokade ägg.

}

\small{
\textbf{Paddan i pannrummet}
\label{f3e250508788285c8d4b2ea74db0c6e7}
 är ett uttryck som används av den lite mindre bemedlade men mer livsnjutande delen av Stockholmsområdets \textsc{(se Stockholm s.~\pageref{edcd259e0a03c7ab70feb186bae19f13})} manliga befolkning. Paddan syftar på pedalen och i detta fall gaspedalen i en bil. Pannrummet syftar i sin tur på ett rum som ligger i källaren. Uttrycket kan alltså ungefärligt översättas till \quotetext{gasen i botten \textsc{(se botte s.~\pageref{59d505d6448b03a0331e6fc09a69d3b9})}} men används inte bara om hög fart utan även om fall då någon, till exempel Matt Pike \textsc{(s.~\pageref{f1e5b05112d62b84340f4d287585d83d})}, levererar till 110\%.

}

\small{
\textbf{Paint}
\label{c1940aeeb9693a02e28c52eb85ce261c}
 är ett populärt grafikprogram som används av många PC-ägare världen över för att framställa olika bilder och för att manipulera fotografier. Den stora spridningen av programmet har skapat en debatt om de etiska problem som föreligger vid publiceringen av foton och bilder på nätet liksom i andra media, eftersom många bilder kan manipuleras utan att detta tydligt framgår. Medan vissa oroar sig för det pressetiska i det hela menar andra på att paint är en tydlig signal att utvecklingen sprungit ifrån förlegade regelsystem och synen på digital information.

}

\small{
\textbf{Paj}
\label{0b438dd454bc6a17de239ebf0a46b91b}
 är ett ord med, åtminstonde, fem \textsc{(se femma s.~\pageref{d974e0811fe7a4d49a9062d33b66a88d})} betydelser.

 HEAD2: Inom modet
 Paj åsyftar här en jacka eller väst i skinn eller jeanstyg, så kallad skinnpaj eller jeanspaj. På de snyggaste plaggen står det \quotetext{Hawkwind} skrivet för hand på ryggen.
 HEAD2: Inom nikotinismen
 Paj är här en bit bakat lössnus.
 HEAD2: Paj inom mekaniken
 Inom mekaniken betyder paj att något är ställt ut bruksskick.
 HEAD2: Inom gastronomin
 Inom gastronomin betyder paj ett slags rund macka av pajdeg och äggstanning.
 HEAD2: Norrtäljeprofil
 Pajen är en snubbe från Norrtälje/[[Finland]] \textsc{(se Norrtälje s.~\pageref{7527f7dad9445013a559dc7e2a91f3b3})} som ibland bär hatt.

}

\small{
\textbf{Pajen på kajen}
\label{813e80b092f456fc81da1b8e0e83a273}
 kallas i folkmun det etablissemang i hamnen i Norrtälje \textsc{(s.~\pageref{7527f7dad9445013a559dc7e2a91f3b3})} stad där den törstige flanören under sommaren 2010 kunde köpa en öl av, och byta några ord med, den omtyckta och smutta \textsc{(se smutt s.~\pageref{d9114ffee4f2dcee302ae2b19ce5eea9})} Norrtäljeprofilen Pajen \textsc{(se Paj s.~\pageref{0b438dd454bc6a17de239ebf0a46b91b})}. Pajen på kajen blev aldrig någon braksuccé \textsc{(s.~\pageref{678371d35369d3d29afceb1445630833})}, men inbringade ändå lite välbehövda fickpengar åt Pajen. Etablissemanget var arrangerat så att besökaren kunde välja mellan fyra olika bord med tillhörande bänkar på vilka Pajen hade placerat ut dynor, och där slå sig ner. När detta hade skett kom Pajen framskyndande och tog upp beställning, som han skrev ner på ett litet block för att komma ihåg. Ville man betala med kort fick Pajen gå ombord på den intilliggande resturangbåten S/S Norrtälje och greja med en kortläsare som fanns där någonstans. Detta var till stort förtret för Pajen, som gärna slapp gå så mycket fram och tillbaka, fram och tillbaka.

}

\small{
\textbf{Paleontologi}
\label{7dfd5e638a8efdca671f24f6bfef45fb}
 är läran om roliga stenar. Genom att studera en gammal sten kan forskaren (paleontologen) få fram information om vad som finns i stenen och om den ser rolig ut. Om forskaren bedömer att stenen ser tillräckligt rolig ut tar han med den till sitt laboratorium där han tittar ännu noggrannare på den innan den åker upp på en hylla i ett museum för att glädja andra. Paleontolog blir personer som saknar det estetiska handlaget för att bygga egna stentroll \textsc{(se stentrollsaffär s.~\pageref{f832a905e0a0a857d0d7eae1520c14b1})} och istället måste ge sig ut i skogen för att hitta stenar som ser naturligt roliga ut.

}

\small{
\textbf{Palle Kuling}
\label{5929d8f94522386aefeb70ec4f139090}
 , eller kort och gott \quotetext{Palle,} är en påhittad figur som är ansiktet utåt för Fazers godisproduktserie med samma namn. Palle är påfallande lik Rasmus Nalle \textsc{(se Rasmus Klump s.~\pageref{eac88a6def9b9f47888e7e3b62719cf1})} polare Pelle, som likt Palle är en pelikan \textsc{(s.~\pageref{ecf1b944439a171dfe1163001feeed19})}. Detta kan få den misstänksamme att tro att Fazer utnyttjar Rasmus Nalles omåttliga populäritet hos den yngre publiken för att sälja sina produkter.

}

\small{
\textbf{Palltruck}
\label{9417f2a2e1478e6e63cba47cf2d1a505}
 , eller pallbjörn som den även kallas (är den av fabrikat BT kan den även tituleras \quotetext{bolsjevik} då den är både röd och stark som tusan) är en nödvändighet på en normal arbetsplats. I efterblivna länder, typ Spanien \textsc{(s.~\pageref{84c63835ca2fcac8636cf7d36aa48fa4})}, finns inte pallbjörnar, så dom får ingenting gjort.
 Att kunna bruka en palltruck är en tydlig klassmarkör \textsc{(s.~\pageref{6a9c0c6836a0777442468f821837e795})} men kan även ses som ett intelligenstest för att se om någon är slug nog att få gå lös.
 Pallbjörnen uppfanns 1947 av den sjukt underskattade Ivar Bryntse \textsc{(s.~\pageref{a5e922bb2ad7c32a2419b5ba3afcdc99})}

}

\small{
\textbf{Palltrucken}
\label{1315ee23693c2c382b0e6b878be74cbc}
 Palltruck \textsc{(s.~\pageref{9417f2a2e1478e6e63cba47cf2d1a505})}

}

\small{
\textbf{Panflöjt}
\label{ce107b52f922e556b394fa5303dc6b0f}
 är ett förtrollande \textsc{(se mani s.~\pageref{07cd55c7b42715ec44c133a6a165e8d2})} blåsinstrument med både anor och attityd. Med sina sälla toner har panflöjten både upprört och berört allt sedan den första uttråkade sydamerikanen bröt av ett vasstrå och provade att tuta i. Dess hypnotiska egenskaper är omvittnade och det var ingen slump att gammelgubben Mozart lär Papageno samla frimurarna i \textit{Trollflöjten} med hjälp av en panflöjt istället för en australiensisk \textsc{(se Australien s.~\pageref{e727d8d1b3162a732c7f706d55de64f3})} didgeridoo, en tysk \textsc{(se Tyskland s.~\pageref{b1b58da783b6d5fa090f3015f1889869})} dragbasun \textsc{(s.~\pageref{0315aaaabb57a67312aa3316fd2006e1})} eller en dansk \textsc{(se Danmark s.~\pageref{5331d7fd27772396f412a5b6d19bad44})} kazoo. Panflöjten återfinns över hela världen och tycks ha uppstått ungefär samtidigt på så olika platser som Kina, Rumänien och Peru. Materialet varierar efter naturresurserna men låter alltid lika vackert \textsc{(se skensnygg s.~\pageref{b8c4b0c5f26691aa5a96a144f2276349})}. Vissa lyssnare blir så förhäxade av instrumentet att de kan hamna i en rökrock \textsc{(s.~\pageref{0eaceca559fb616779dbe004972eba07})}.

}

\small{
\textbf{Pangkaka}
\label{92c460311a38ca00425bf4cb3651baba}
 är en kaka som man äter till middag eller lunch, gärna tillsammans med ärtsoppa. Pangkaka är jävligt fint men är inte så nyttigt. Man äter ett gäng pangkakor tillsammans med sylt \textsc{(s.~\pageref{be6fa0cf731bdae2e13ad52084c90fbc})} och grädde. Pangkakan finns i två former: ungs- och vanlig.

}

\small{
\textbf{Pangsionärerna}
\label{190103720ce3cafeb2aaaf1c5ed5cd69}
 är PROs väpnade gren.

}

\small{
\textbf{Panik}
\label{bca410441b88e24768f3f385548edfbe}
 är en känsla som uppstår när stressen \textsc{(se stress s.~\pageref{e10a36f1a5231e597daf8f42dc1ab55a})} har gått helt över styr. Individen som drabbas slutar ofta helt att tänka rationellt och logiskt och handlar antingen i blindo eller helt styrt av ryggmärgen. Detta medför sällan att individen löser problemet som lett till paniken utan bara försätter sig djupare in i Pans grepp. Här behöver man därför en medmänniska som kan säga \quotetext{Lugn!} och sedan hjälpa individen att åter ta kontroll över situationen. När man sedan är tillbaka i sitt lugna, metodiska jag förvånas man över hur konstigt man agerat när man hade panik, men så är det bara. En del människor är helt inkapabla att få panik, medan andra har det mest hela tiden.

}

\small{
\textbf{Pannband}
\label{bf8b10df2ba586607f80258b6c1c8dca}
 Huvudbonad för skinheads \textsc{(se skinhead s.~\pageref{a54bc1b5d472b5afed8e84004b6441c4})} med känsliga öron \textsc{(se öra s.~\pageref{c4774ec92abe06f5664e18f44446d7e7})}.

}

\small{
\textbf{Papist}
\label{9e0a37391a1827c035ec21cb07a39853}
 är ett skällsord \textsc{(s.~\pageref{e0fc85fd2f5249557257965783ac136e})} som endast fungerar på italienare, polacker och irländare.

}

\small{
\textbf{Pappersform}
\label{37edcb2e533bd9c3e51f475c598b8671}
 kan syfta på

 \begin{itemize}
 \item A4 (210×297 mm) definierad i ISO 216 \textsc{(s.~\pageref{2f6bbcf45fd8afaa6eed567fcfe4c722})},
 \item Bakformar
 \begin{itemize}
 \item Muffinsform
 \item Knäckform
 \end{itemize}
 \end{itemize}

}

\small{
\textbf{Paris}
\label{ccbee73cd81c7f42405e1920409247ec}
 Större, men inte bättre, än Ånge. Men mera ostar har dom.

}

\small{
\textbf{Parisare}
\label{5aca28013b9a7e4088e7fb228f3e4827}
 Är en parisare en person från Paris? I Sverige säger alla söder om Härnösand bestämt ja. I det egentliga Norrland (allt norr om Härnösand) blir svaret nej. Om man då frågar den egentlige norrlänningen vad en parisare är kommer den berätta om en korv. Parisarkorven.
 Parisarkorven är ungefär som en falukorv, men inte alls lika krökt och smal och med mindre kötthalt och större mängd mjöl. Helst ska korven komma från Bastuträsk. För att parisarkorven ska bli en riktig parisare, måste den serveras på rätt sätt. En minst en centimeter tjock skiva korv placeras, efter att ha grillats lagom mycket, mellan två hamburgerbröd (puritaner kräver ofta att bröden ska ha värmts i våffeljärn, andra nöjer sig med att de ska ha grillats lätt, tills det blir ränder på dem från grillgallret). På parisaren har man ketchup, senap och bostongurka. Bostongurkan kan ersättas med gurksallad (eller gurkmajonnäs, om man tror att man är nåt).
 Parisaren för en allt mer tynande tillvaro, även i det egentliga Norrland, då den kulturimperialistiska produkten hamburgare oupphörligen tvingas ned i halsen på de korvälskande norrlänningarna. Men de som är rädda att parisaren ska försvinna har fortfarande en korvens bastion att vända blicken och be mot. Skellefteå, guldstaden. I Skellefteå har enligt tidningen Norran [http://norran.se/nyheter/norrochvasterbotten/article312926.ece en akademi för parisarens bevarande upprättats.]

 HEAD3: Kritik mot Svenska Parisareakademin (SP)
 Man kan dock ifrågasätta akademins nobla syften. Dels iakttar man snart i artikeln hur det mest är förmögna direktörer, advokater och reklamgurus som utgör akademin. Man märker även att flera av dem inte ens bor i Skellefteå, vilket man tycker borde påverka deras förmåga att skapa en gräsrotsrörelse och organisera kamp kring vad de själva kallar \quotetext{allmogeburgaren}. En sak som står klar att det inte är allmogen i sig som akademin vurmar för. Välkommen i akademin är den i alla fall inte. Den mest relevanta kritiken förs fram av den kommentator som går under (den förmodade) pseudonymen \quotetext{korven} i Norrans artikel: \textlessi\textgreater\quotetext{När Korv Ivars \textsc{(se korv-Ivars s.~\pageref{e42e3fd4f6b398bd3bb69c234431269d})} var Korv Ivars gick det att få världshistoriens godaste parisare där. Jag minns att han grillade två bröd i våffel järn, gurksallad , ketchup, senap och lök.. Mmmmmm. Det var tider det.
 När jag ser Ungdommarna i pariserakedemin hålla i en frasse 'parisare' slutar mitt hjärta slå. SKÄRPNING KILLAR, tänk på vad ni gör..}\textless/i\textgreater
 Kritiken består alltså i att gubbarna (enligt korven ungdommar) på bilden i artikeln svullar parisare från hamburgerkedjan Frasses. Det många verkar känna är en oro över hur välplanerat allt är. Advokater, reklamgurus, corporate big wigs \textsc{(se storfräsare s.~\pageref{4db17005692cd83e3e946a1311b81ed0})}, fat cats och big business hyllas i media, vilket lämnar en fadd smak i munnen hos den genomsnittlige parisarekonsumenten. Hur saker och ting utvecklas är fortfarande oklart. Är SP en mörkrets eller ljusets makt i frågan om parisare? En sak är säker. De är under vaksamma skelleftebor, som korvens, lupp.










 HEAD3: Sörlänningens förståelse av parisaren
 Inom citationstecken kan besökaren läsa hur någon som inte kommer från det egentliga Norrland ser på parisaren. Lägg märke till den enorma skillnaden i engagemang när det kommer till korvsort, brödsort och vad man ska ha på.

 \textit{\quotetext{Parisare är en bit falukorv som stoppas mellan två bröd. Ofta kan man även ha lite ketchup och senap på. Vilken typ av bröd spelar inte så stor roll men man brukar ha handburgerbröd [ sic ]. Parisaren är nära besläktad med grabbimacka.}} - Potmo Kobulev

}

\small{
\textbf{Passa tider}
\label{82979bc0f5b86ddadbf8b836f3aabc9f}
 Att passa tider är nära besläktat med att göra rätt för sig \textsc{(s.~\pageref{c8c01e0e8b4ad8e5ff6011b8af6405a5})} och går ut på att inte vara sen. Säger man att man ska vara på plats 07.00 så är man där 06.55 så att man hinner finljuga \textsc{(s.~\pageref{4eee5e7eab6f049c4084d3a5161016f9})} i fem minuter. Frågar man ärkeidioten Jimmie Åkesson så svarar han att det svenskaste \textsc{(se Sverige s.~\pageref{b1999637949ed135b2ca03f3a38460cc})} som finns är att passa tider och att stå i kö.

}

\small{
\textbf{Patentkork}
\label{1e39785f5bab52f931dac485727645b6}
 en är en tysk uppfinning framtagen i slutet av 1800-talet. Vid denna tid hade så kallat \quotetext{sodavatten} börjar bli populärt inom Europa och tillhörande kolonier och denna dryck krävde en kork som kunde tåla tryck. Patentkorken löste med bravur detta problem med hjälp av en gummiring och återförslutningsmekanism. Den enkla konstruktionen gjorde patentkorken omåttligt populär och med större kärl kunde även sill konserveras på samma sätt. Carl von Linné \textsc{(s.~\pageref{5e8380bf6b7ce99678e6752b6d9e709e})} sägs ha blivit så fascinerad av konstruktionen att han gav order om att all mat i hushållet skulle konserveras på detta sätt. Det hela lär ha gått över styr när han åt ett tråg \textsc{(s.~\pageref{1e0e0470206e0f2baad8e628ba8f770c})} konserverad julgröt från året innan och fick spendera resten av långhelgen på avträdet.

}

\small{
\textbf{Patrik Sjöberg}
\label{77703d875078935741a7e0904cd69fa4}
 är en lång kille med tribaltatueringar som gillar Sydamerika och allt som kommer därur.


 HEAD2:  Se även

 Myspedofiler \textsc{(s.~\pageref{2cb4b0ffec59535759020365fd8c7cd1})}

}

\small{
\textbf{PATSY award}
\label{7bc393214fc2140e26e634476bc1866f}
 var en slags djurens motsvarighet till Oscar i filmvärlden. PATSY står för \textbf{P}icture \textbf{A}nimal \textbf{T}op \textbf{S}tar of the \textbf{Y}ear och delades ut till särskilt framstående djur inom filmbranschen. Priset delades ut första gången 1951 \textsc{(se etta  s.~\pageref{ba48f6c4097b7fc25ca11f1e544842d7})} och gick till den talande åsnan Francis. Bakgrunden till priset var att många kände ett behov av att höja djurens status i filmvärlden, som dittills varit på ungefär samma slit och släng-nivå som en papptallrik. Under inspelningen av \textit{Ben Hur} dog närmare 150 hästar \textsc{(se häst s.~\pageref{b4c608370b339da095c5f8db7fab0945})}, och Ronald Reagan slog ihjäl tre schimpanser mellan tagningarna i \textit{Bedtime for Bonzo}. PATSY awards lades ned 1986 på grund av bristande finansiering. Men det gjorde inte så mycket för då hade nästan alla roliga djur hunnit bli  utrotningshotade \textsc{(se utrotningshotade djur  s.~\pageref{24a427a5537c2c8918cfa213ae099a74})} ändå.

}

\small{
\textbf{Paul du Chaillu}
\label{74af3e95173d78ca788cf1cd2ed42808}
 blev 1859 den förste vite mannen att se en levande gorilla, och kort därefter den förste vite mannen att skjuta en.

}

\small{
\textbf{Paxa}
\label{0e00979a45d6f4083485e9c9fb01f590}
 Att paxa är att rättmätigt (eller orättmätigt) tilldela sig något, t.ex. den sista varma mackan eller framsätet i bilen.

 HEAD2: Historiska paxningar
 \begin{itemize}
 \item England, Frankrike och Holland paxar Nordamerika till många indianers förtret
 \item De flesta länder i Europa kapp-paxar Afrika, förutom Liberia och Etiopien.
 \item Kapitalisterna paxar alla produktionsmedel och naturresurser i hela världen
 \item Sionisterna paxar Palestina
 \item Kyrkan paxar tolkningsföreträdet vad gäller universum och allt däri
 \item Den tyska mustigheten \textsc{(s.~\pageref{682ccd5fdc3aff0c97e8845c3d6b6ca8})} paxar Polen
 \item Män paxar merparten av alla resurser och maktpositioner
 \item Uvarna \textsc{(se Uv s.~\pageref{45210da832f9626829457a65e9e7c4d0})} paxar platsen som det mäktigaste djuret, långt före människorna kommer på villfarelsen att de är det.
 \item Stockholmskapitalet paxar utdelningen från gruv- och skogsindustrin annorstädes i landet.
 \item I linje med \textit{Pax Americana} paxar the United States of America \textsc{(s.~\pageref{ade6b3bd5e720abb20ed8a9a4c6b9ae8})} allt mellan himmel och jord, samt månen.
 \item Nyliberalismen \textsc{(se Nyliberalism s.~\pageref{a562ace16486d966be4513ea22aee287})} paxar plats som chefsideologi i de flesta media
 \end{itemize}

}

\small{
\textbf{Payin Dues}
\label{f7718d3dc310948047d45738b3aab4c4}
 Payin' Dues - Disc 2 är Van Morrisons bästa platta.

 Låtlista:
\begin{enumerate}
\item Twist and Shake
\item Shake and Roll
\item Stomp and Scream
\item Scream and Holler
\item Jump and Thump
\item Drivin' Wheel
\item Just Ball
\item Shake It Mable
\item Hold on George
\item The Big Royalty Check
\item Ring Worm
\item Savoy Hollywood
\item Freaky If You Got This \textsc{(se gottis s.~\pageref{90555c8edd726ba0b0a03d0676a4ae48})} Far
\item Up Your Mind
\item Thirty Two
\item All the Bits
\item You Say France and I Whistle
\item Blow in Your Nose
\item Nose in Your Blow
\item La Mambo
\item Go for Yourself
\item Want a Danish \textsc{(se Danmark s.~\pageref{5331d7fd27772396f412a5b6d19bad44})}
\item Here Comes Dumb George
\item Chickee Coo
\item Do It	Listen
\item Hang on Groovy
\item Goodbye George
\item Dum Dum George
\item Walk and Talk
\item The Wobble
\item Wobble and Ball
\end{enumerate}

}

\small{
\textbf{Payin' the Dues}
\label{85d33eb38a1a94e8d3fc50d80fc894cb}
 är inte the Hellacopters bästa skiva. Den är dock helt okej och singeln \quotetext{Hey!} är riktigt bra. Skivans titel är också den engelska motsvarigheten till att göra rätt för sig \textsc{(s.~\pageref{c8c01e0e8b4ad8e5ff6011b8af6405a5})}
 HEAD2: Låtlista
\begin{enumerate}
\item \quotetext{You Are Nothin'} - 2:39
\item \quotetext{Like No Other Man} - 3:15
\item \quotetext{Looking at Me} - 2:04
\item \quotetext{Riot on the Rocks} - 1:23
\item \quotetext{Hey!} - 3:21
\item \quotetext{Soulseller} - 3:13
\item \quotetext{Where the Action Is} - 2:41
\item \quotetext{Twist Action} - 2:04
\item \quotetext{Colapso Nervioso} - 4:04
\item \quotetext{Psyched Out \& Furious} - 4:14
\end{enumerate}

}

\small{
\textbf{Pekas lastbil}
\label{d3c56b26743b36bf145abd6e17513e66}
 (Finskt uttal) är en helig lastbilsfirma i Pukavik.
 Peka själv är en finsk alkoholist med 50+ år på nacken och fler hjärtinfarkter än det finns sandkorn på Hälleviks alla stränder.

}

\small{
\textbf{Pelikan}
\label{ecf1b944439a171dfe1163001feeed19}
 en är ett slags fågel som har ett slags pung under näbben.

}

\small{
\textbf{Pelle Fosshaug}
\label{b6caa53a9a50eb546517552a5503e323}
 (född 1965) är en föredetta svensk bandyspelare som enligt egen utsago lider av \quotetext{klockeren} ADHD [http://www.expressen.se/sport/1.786093/jag-maste-ligga-pa-gransen]. Fosshaugs ADHD har lett till elitseriens snabbaste mål någonsin (fem sekunder från avslag) samt ett och annat övertramp, som när han bröt sig in på Zinken och lärde ett iranskt barn åka griller, eller när han sopade till en finländare i magen med klubban och sedan krosscheckade nästa lirare som kom till finnens undsättning. Fosshaug lyssnar på Slayer och är bandyns stor grabb \textsc{(se Stora Grabbars och Tjejers Märke s.~\pageref{3b527f8b13885eb277c77de4b1f51658})} nummer 197.

 Den som inte kan få nog med information om Fosshaug gör rätt i att läsa biografin \quotetext{Bandygalen} (Anders Lif, 2007).

}

\small{
\textbf{Pelle Karlsson}
\label{1a8c873ff230698396c324f14c02b7fa}
 Alla ställen som säljer begagnad vinyl har minst en lp av Pelle Karlsson.
 Vem är denne mytiske man och varför hamnade i princip hela upplagan på loppis?

}

\small{
\textbf{Pelle svensson}
\label{26d88b383fd38f349c7741ca7051904e}


}

\small{
\textbf{Pelle Svensson}
\label{26d88b383fd38f349c7741ca7051904e}
 , född 6 februari 1943 i Nylandsån, är en svensk brottare \textsc{(se blomkålsöra s.~\pageref{abc81463a2d11b31c192a0fce03510a8})} och rättskämpe.

 Pelle Svensson har två VM-guld i brottning, grekisk-romersk stil, och ett OS-silver. För dessa framgångar gavs han smeknamnet \textit{Pelle Sving}. Framgångarna till trots är det ändå rättvisa som alltid varit Pelles huvudsakliga intresse. Under 80-talet försvarade Pelle offentligt bland annat den man som planlagt en kidnappning av Peter Wallenberg. Genom åren har Pelle kommit att hjälpa många andra människor som anklagats för sådant som av staten ansetts vara \quotetext{brott}.


 Som den renässansman han är ägnar sig Pelle ibland också åt poesi. Så här skalldade han 16 maj 2010 på sin blogg: \textit{\quotetext{Att twittra är som att kvittra. Vad rör sig inom mig just nu? Gör som fåglarna sjung ut, för annars tar livet plötsligt slut}}.


 Pelle har nu mera statlig sjukersättning efter att ha lyckats få förföljelse av massmedia klassat som arbetsskada.


 HEAD2:  Externa länkar

 [http://www.pellesvensson.net/ Pelles hemsida]

}

\small{
\textbf{Per Thunell}
\label{8455763330baa255e654b44fc37a5c92}
 (aka Per Banner) spökskrev alla låttexter till Mob 47 \textsc{(s.~\pageref{9955900bded21660e7f4e15ae8d23e3a})} för att Mob 47-Åke \textsc{(s.~\pageref{486ee67ac39debabed3d92a7555dcebd})} hellre skrev låtar om ägg än om cellbyggnader.

}

\small{
\textbf{Perspektiv}
\label{1606dd19366985367d677f7b6de46e52}
 beror på vad man jämför med, eller vem man frågar. Vanligtvis kontrolleras perspektiv av institut och tankesmedjor \textsc{(s.~\pageref{c276b5997d5af80504f79b30d121cf62})}.

}

\small{
\textbf{Peru}
\label{32e8419a7ecb8f918c70fdadf783e3d8}
 Enligt uppgift sydamerikas Finland \textsc{(s.~\pageref{631d44eaa1254ff71a1e11ba021d1266})}.

}

\small{
\textbf{Perversa elektriker}
\label{43ce6090ae4836b80a55f23e15b5e835}
 är som elektriker är mest, fast mycket, mycket sämre. De är moderater \textsc{(se moderat s.~\pageref{c4564b188cb670841733a3ff923c2fb0})} och tycker att alla som vill betala skatt är dumma i huvudet \textsc{(se huvud s.~\pageref{e906cd95a540df9b16d0460fb4cf0adc})}. De pratar inte om nåt annat än jobb på fikat, och då bara om hur dumma i huvudet f.d. arbetskamrater är, eller nåt annat än jakt när man arbetar. Varför är dessa elektriker perversa, undrar kanske ni? Jo det är för att de tycker att det är \quotetext{kul} att MMS:a porr \textsc{(se pörr s.~\pageref{5faa435e2f0af7617816f0cade262581})} till sina arbetskamrater. Dessa bilder är påfallande ofta bögporr och ofta så pass extrem karaktär att mottagaren får fundera på om den begår ett brott genom att bara titta på dom. Det enda sympatiska draget hos den perverse elektrikern är att han (för det är alltid, alltid, alltid en man) föder upp strävhåriga taxar, och det är inte så jävla sympatiskt egentligen. Perversa elektriker är inte heller med i med i Elektrikerförbundet, LO:s stridbaraste fack.

}

\small{
\textbf{Peter Forsberg}
\label{fb05035aa25d5f4c26465d41e1543e60}
 är grymt besviken på Börje.


 \quotetext{Du det skiter jag i nu ska jag hem och duscha} [http://www.youtube.com/watch?v=Td6kAggdIao\&feature=related]

}

\small{
\textbf{Petrus de dacia}
\label{e07b2cf719fa237191665c127c7080c2}
 (1230-tal - 1289) omnämns ofta som Sveriges \textsc{(se Sverige s.~\pageref{b1999637949ed135b2ca03f3a38460cc})} första författare, det faktum att han endast skrev på latin, ej svenska, till trots. Hans mest kända verk är nedtecknandet av Kristina av Stommelns uppenbarelser.


 HEAD2:  Tidigt liv


 Petrus de Dacia föddes på 1230-talets Gotland. En på den tiden hemsk, karg plats där måsar \textsc{(se mås s.~\pageref{04f599c35052d2060c70cb99b09f94dd})} stora som dagens albatrosser härskade i skyarna, svävandes ovan de gråa lavafälten som utgör Gotlands blasfemiska grund, spanandes efter deras huvudsakliga föda; människokött. Raukarna hade inte hunnit bli eroderade i lustiga skepnader ännu, utan var än så länge bara tråkiga stenar. I denna miljö växte Petrus de Dacia upp. Han kom sedemera att gå med i dominikanerordern, i och med ett dominikanerkonvent i Visby, där hans passion för det skrivna ordet och hans lust att berätta \textsc{(s.~\pageref{4f84e02a70b3bbb57fa83da31bf7a16f})} kom att växa sig än starkare än innan. Han drevs av sin kunskapstörst ut ur Sverige \textsc{(s.~\pageref{b1999637949ed135b2ca03f3a38460cc})}, till kontinenten, Europa. Där kom han att stöta på flera av dåtidens stora tänkare, däribland Thomas av Aquino, som var hans lärare i Paris.

 HEAD2:  Paris


 Petrus de Dacia skrev själv i sina memoarer om skoltiden i Paris: \quotetext{\textit{Paris är så bedårande. Ack det skälver uti min lekamen när jag vandrar längst med Notre Dame och känner Guds kärlek skaka min existens. Om det inte hade varit för skolan hade allt varit perfekt. Min magister Thomas av Aquino är en sådan rese. Han tror på fullaste allvar att det går att kombinera Aristoteles idé om en Primo movens och universums Evighet med kristen teologi. Löjeväckande är bara förnamnet!}}

 Petrus de Dacia och Thomas av Aquino hamnade ofta i väldiga dispyter med varandra, vilka allt som oftast slutade med att Thomas av Aquino, i egenskap av magister, beordrade Petrus de Dacia att ta på sig en Chapeau de paysan \textsc{(s.~\pageref{27aa75146d9ab723d1423168a2539d5d})} inför hela klassen, varpå all diskussion omedelbart upphörde.


 HEAD2:  Konflikt


 Petrus de Dacia reste efter en längre tids studier i Paris runt på fastlandet. Hans resväg är tämligen lätt att följa då han ofta dyker upp i diverse straffregister. Lösdriveri och offentlig berusning var de domar han oftast ådrog sig. I Bologna sägs han ha träffat på Halbera Snorresson, dotter till Snorre Sturlasson, på en nedgången taverna. De två ska ha hamnat i bråk med varandra i och med att Petrus de Dacia anklagade hennes far för att vara en ooriginell klåpare som stal stora delar av sina historier från den höviska litteraturtraditionens mest kända verk. Halbera Snorresson tillbakavisade detta å det grövsta och kontrade med att kalla Petrus de Dacia för ett skitsvin \textsc{(s.~\pageref{62911ad86d6181442022683afb480067})}. Hur konflikten löste sig förblir oklart.


 HEAD2:  Kristina av Stommeln


 Biografin över Kristina av Stommeln kom till i och med att Petrus de Dacia anlände till Köln i studiesyfte. Han hade bränt alla sina broar i Paris och var nu på desperat jakt efter ett break. I Köln träffade han, på en ost och vin-kväll \textsc{(se vin s.~\pageref{62911ad86d6181442022683afb480067})} arrangerad av ett beginkloster, Kristina av Stommeln. Kristina av Stommeln var dotter till en grisfarmare och småfifflare tillika, Gerhard av Brömmeln. Sin fars skojargener hade gått i arv till dottern som under kvällen insåg att Petrus de Dacia var en kille som \quotetext{sällan var omöjlig}, som hon själv uttryckte det i ett brev till sin väninna Katryna. Petrus de Dacia och Kristina av Stommeln kokade så ihop en historia som skulle göra dem båda kända. Efter ett utdraget parti whist avgjordes det att lotten föll på Kristina av Stommeln att fejka vansinne för att de båda skulle bli kända. Resten är historia. Petrus de Dacia nedtecknade Kristina av Stommelns påhittade visioner och skrev vitt och brett om hennes stigma. Kristina av Stommelns anfall av stigma upphörde i och med Petrus de Dacias död 1289.

 HEAD2:  Skänninge och hem till byn


 Petrus de Dacia bodde under sju år i Skänninge där han, förutom att skriva brev till Kristina av Stommeln, drev ett kvinnligt dominikanerkonvent med ett flertal fromma och förmögna kvinnor. Dominikanerkonventet hette S:ta Ingrids systrakonvent. Stämningen på konventet beskrevs av en syster som \quotetext{bizarr} och \quotetext{skandalös}. Händelserna som utspelade sig i Skänninge kom sju hundra år senare att influera Olle Hellbom när han skrev manus till sin och Lasse Hallströms dunderhit Tuppen.

 På sin ålders \textsc{(se ålder s.~\pageref{d7a7467d6b0b94f50c209220eab58dd1})} höst flyttade Petrus de Dacia hem till Visby. Måsarna hade krympt i och med att inget byte fanns kvar. Gotlands avbefolkning hade inletts och raukarna hade börjat lakas ur på grund av försurningen av Östersjön. Petrus de Dacia dog utan att lämna någon avkomma efter sig.

}

\small{
\textbf{Petter}
\label{3f3b423295473405f1eda282c3531e75}
 är en kille från Stockholm \textsc{(s.~\pageref{edcd259e0a03c7ab70feb186bae19f13})} som bor på Söder, går på kändisfester och driver ett skivbolag. Han har det otroligt svårt, och hans sätt att hantera livets ofrånkomliga tragedier och orättvisor är att pratsjunga om dem i en mikrofon, liksom att åka till sommarstugan på västkusten som han pratar om i en av sina många låtar. Ibland är han dock glad och då pratar han om att gå på krogen och vifta med armarna.

}

\small{
\textbf{Philibert Humla}
\label{a8b5a215825d27317390c9519375c237}
 (1814-1891) var en svensk jurist från Karlskrona. 1862 skrev han boken \textit{Inledning till läran om stöld och snatteri}, och med facit i hand kan man se att den har influerat många människor. Framförallt småskurkar \textsc{(se småskurk s.~\pageref{c25031c5d78d9ad6fae8ab8f08d5e9dd})}. Philiberts bästa kompis hette Herman Bergman, så om man såg dom tillsammans på stan kunde man säga: \quotetext{\textit{där går Philibert Humla och Herman Bergman}}. Smaka på orden, de ligger väldigt bra i munnen.

}

\small{
\textbf{Picknickbog}
\label{696b50bd1480adf411314859b3464652}
 (från hönsens \textit{picka} ungf. \quotetext{hacka fram}, amerikanskans \textit{nickel} ungf. \quotetext{liten valör}, skånskans \textit{bog} ungf. \quotetext{grisarsle}) är formpressat kött som tryckts ner i en plåtburk. Innan det hamnar i burken har köttet kokats i saltlag så när man \textsc{(s.~\pageref{39c63ddb96a31b9610cd976b896ad4f0})} öppnar är det färdigt att äta. Massan har en konsistens i gränslandet mellan paté och blodpudding och faller lätt sönder. Det hela ser ganska vidrigt ut så det är inte så många som köper picknickbog. Såvida du inte känner en veteran från första världskriget är det troligt att du inte vet någon som käkat picknickbog mer än en gång på skoj. Enligt Wikipedia finns det typ inte picknickbog i USA, men enligt Ronny \textsc{(se Användare: Ronny s.~\pageref{c7fc87f27db026e1c60a6ac2cb1fd820})} finns det visst. Pick your king.

}

\small{
\textbf{Ping}
\label{df911f0151f9ef021d410b4be5060972}
 arbetar på ICA och är bland annat mejeriansvarig, men han gör också en hel del kassaarbete och frontar varor när det är lugnt i butiken. Ping är också lite av en mentor för den efterblivna killen som också arbetar där, mest med att fronta varor.

 Ping är upphovsman till den uppskattade drycken pingvin \textsc{(s.~\pageref{a5c3190fd8fc0a6cbf0cb645b8add9d0})}.

}

\small{
\textbf{Pingisrummet}
\label{57aece0bcbea0008c9d72282cbea198b}
 är en mytologisk plats som sägs ha legat i källaren på Humhuset på Umeå Universitet före dess att någon nyliberal \textsc{(se tokliberal s.~\pageref{531cb70b602e3f3c32d40bac64400830})} kom på att det skulle vara bra om staten bildade ett företag \textsc{(se entreprenad s.~\pageref{2d3b60492ed3cebe0a3cf341bc5b20b5})} (Akademiska hus) som statens universitet kunde hyra sina egna lokaler av. Ägandet av universitetsbyggnaderna flyttades således från staten till statens företag Akademiska hus. Detta ledde till att universitetens institutioner förlorade alla sina ekonomiska resurser eftersom de i praktiken var tvugna att betala för att kunna ha föreläsningar och seminarier inomhus, vilket i sin tur ledde till att Akademiska hus fick så mycket pengar att de var tvugna att hitta på de mest bisarra idéer för att bli av med dem, samt att pingisrummet försvann. Pingisrummet sägs ha varit en glädjens och lyckans lokal, där sorglösa studenter och akademiker spelade pingis för allt vad tygen höll. På den plats som grånande språkforskare och historiker pekar ut som pingisrummets läge finns idag en ekande lokal. I den finns ett antal pappkartonger och nermonterade lagerhyllor, och, kanske, en avlägsen doft av svett. Eller? Sensmoral: Allt fast förflyktigas.

 HEAD2: Andra betydelser
 Pingisrummet kallas även kylavdelningen på ICA Berghem, där Ping \textsc{(s.~\pageref{df911f0151f9ef021d410b4be5060972})} är mejeriansvarig.

}

\small{
\textbf{Pingvin}
\label{a5c3190fd8fc0a6cbf0cb645b8add9d0}
 är en alkoholhaltig dryck som tagits fram av den kände Ica-handlaren Ping \textsc{(s.~\pageref{df911f0151f9ef021d410b4be5060972})}. Receptet är inte hemligt, men ändå okänt då det består av alla gamla jästa bär som Ping inte lyckats kränga i tid i sin butik.

 Funny fact: Isbjörnsvin \textsc{(s.~\pageref{2879df543437c30c0a2d0dfaf8649ac7})} är egentligen vanligt pingvin men med annan etikett.

}

\small{
\textbf{Pirk}
\label{83b12f6e4a572176927d27c9bc9db930}
 är ett onomatopoetiskt-kinestetiskt-synestetiskt nonsensuttryck, lämpligt att utbrista i t.ex. när ens käresta oväntat kysser en i nacken när man står och diskar.

}

\small{
\textbf{Pissbutik}
\label{3be611a8fcd1a49d8aa59b77092c1bbe}
 En Pissbutik innehåller inte något särskilt sortiment. Det är mer en samling av små detaljer som skapar en pissbutik. Rent praktiskt kan en butik vara en pissbutik (saker är inte prismärkta, hyllorna är för höga/låga, de säljer schampo men inte dusch-tvål o.s.v.) men det är även möjligt att butiken kvalificerar som Pissbutik genom den enskilde besökarens subjektiva politiska/etiska/estetiska synsätt. Ex. kanske en person från Söder skulle tycka att en butik som bara säljer skoter-kepsar \textsc{(se butiker som bara säljer skoter-kepsar s.~\pageref{1104d57d523c5abf0a8273fff6b5fdd7})} är en Pissbutik. Det är även möjligt att ovan nämnda person tycker att ovan nämnda butik är en smutt \textsc{(s.~\pageref{d9114ffee4f2dcee302ae2b19ce5eea9})} butik. Om så är fallet är personen ev. en hipster eller försöker i alla fall bli. ICA Berghem är, rent objektivt, en pissbutik

}

\small{
\textbf{Piteå}
\label{db694f60fd74ffa986e086d8e29f73dd}
 är en stad i Norrbotten \textsc{(s.~\pageref{0e8c003b75982032cde152609ee94154})} som inte luktar speciellt gott. I Piteå är det, och kommer alltid att vara, herrens år 1992.

}

\small{
\textbf{Pizza}
\label{7cf2db5ec261a0fa27a502d3196a6f60}
 Ett runt, platt bakverk. Målet vid bakningen är att få ett cirkelformat bakverk. Oftast misslyckas detta i och med bakfulla pizzabagare. På tal om bakfylla är pizzan just ämnad för bakfulla personer. Då toppingen på pizzan kan liknas vid sopor, och den bakfulla individen också känner sig som skräp \textsc{(s.~\pageref{75f1a5320951ea0dd9aa3c0eaba2c2c7})} bildas en harmonisk symbios mellan båda parter.
 Den bakfulla pizzabagarens resultat är i alla fall en oval-liknande klump med tunn botten. Ovanpå detta läggs något rött \textsc{(s.~\pageref{dacd03b85a85d8c8b67c702e1872c498})}, något gult, samt gårdagens matrester.

 \begin{itemize}
 \item Författaren var här tvungen att avbryta försöket till en beskrivning av pizzan för att inhämta empirisk data kring ovan nämnda objekt.
 \end{itemize}

 Se även Hawaii-pizza \textsc{(s.~\pageref{742e4954c36e42931521b0a417511c7c})}

}

\small{
\textbf{Pizzabulle}
\label{ab4b7ce7f398f16e31ae85294e457ff2}
 Bulle med skinka och eventuellt tomatsås istället för kanel och socker.
 Typiskt multikulti \textsc{(s.~\pageref{25eea9148080d30d384ce1c1277ef126})}.

}

\small{
\textbf{Pizzaracer}
\label{19c6d2a54dcb50b16c3a3b7c6c8a1a09}
 Folklig benämning på Toyota Celica,BMW M3 och liknande bilar med \quotetext{sportig} approach. Köres företrädesvis av entrepenörer i pizzabranchen.

}

\small{
\textbf{Pizzarulle}
\label{f921e45e18230869bb442d20b8b205a6}
 En rulle oskattade hundralappar. Företrädesvis förvarade i fickan på ett par säckiga kockbyxor \textsc{(se yxa s.~\pageref{bd74f429522c7c1481fbba07187efc6b})}.

}

\small{
\textbf{Pjäxfett}
\label{a297c3dde307c3d98a9433e88b02432d}
 är ett svenskt adjektiv som syftar på något starkt positivt. Vinner man 25 kronor på en trisslott är det fett, men hittar man en spelbutik som säljer en rulle fulsnus för under 200 kronor är det pjäxfett.


 Källa: Tacklistan i Totalitärs vinylsingel \textit{Dom lurar oss}.

}

\small{
\textbf{PK}
\label{1cd3c693132f4c31b5b5e5f4c5eed6bd}
 är en förkortning av Politiskt Korrekt. Avser ofta smygrasister som tycker illa om sverigedemokraterna för att man borde, i stället för av naturliga skäl. Men eftersom sverigedemokrater av princip aldrig ser sig själva i spegeln så tror de att alla som tycker illa om dem egentligen fejkar för att plocka poäng. Att som rimlig människa själv använda uttrycket PK är inte tillrådigt, eftersom ett lillfinger åt det hållet lätt leder till att man helt plötsligt har hela handen i det bruna.

}

\small{
\textbf{Place jourdan}
\label{3966a9028306fa1aa0d9dd197b0acac5}
 är ett torg i Bryssel, beläget i närheten av parlamentskvarteren. I mitten av torget finns ett slags stuga där man friterar pommes frittes i grisfett. Här kan den förbipasserande 24 timmar om dygnet se lokalbefolkningen girigt köa för att köpa sig en strut av de eftertraktade frittorna. Ivriga bävrar har öppnat barer kring denna stuga där man får sitta och äta sina avlånga bintje-stavar \textsc{(se bintje s.~\pageref{f21f4f64cb0df1775b5c2a7dc0d83c6c})} och samtidigt dricka sig redlös. Gläd dig medan du kan, tycks budskapet vara, för imorgon kan du dö av en maffig propp i aorta.

}

\small{
\textbf{Plassare}
\label{12950d5d65bc221b46c02ba5d3a89bcf}
 Nån som inte är byssare \textsc{(s.~\pageref{99317503481e8bdd90e670c6c43f6fdf})}. Personen ifråga är antagligen från Kyrkholmen, Öberget eller Norrbacka.

}

\small{
\textbf{Pling plång-taxi}
\label{e19ef88e296f918fd02ed3c62b87fd20}
 är slang för antingen ambulans eller polisbil. Ordets egentliga betydelse är ambulans, som synes i detta arkaiska exempel, bevarat från då Sverige var ett ungt land: \textit{\quotetext{Passare' så 'runte åker på en tjalablängare mellan lysmaskarna och kolar vippen på stampen, så'ru få' åka pling plång-taxi till plåsterstugan.}}
 På senare år har uttrycket kommit att även inkludera polisbil, då i sambandet att en person som varit ute och rullat hatt \textsc{(se rulla hatt s.~\pageref{7c7afc9fb7bb52962f954c0cb548c10c})} i sitt hattrullande blivit något överförfriskad och då får assistans av pling plång-taxin att ta sig till en bädd. Det är viktigt att notera att man inte kan bestämma vart pling plång-taxin ska köra en i något av exemplen. I fallet av ambulans är det enda alternativet plåsterstugan (även det uråldrig slang, med innebörden sjukhus) och i det andra fallet är ändhållplats oundvikligen en fyllecell, ofta med tillhörande lätt misshandel \textsc{(s.~\pageref{da5052972c3a081d8e951c69da453722})}. Med tanke på den dialekt uttrycket förknippas med (Stockholm med omnejd) är det inte omöjligt att fyllecellen är belägen i Norrtälje \textsc{(s.~\pageref{7527f7dad9445013a559dc7e2a91f3b3})}.

}

\small{
\textbf{Plocka päron}
\label{4db1fecfcb40624ab38021166b8aaa05}
 är en sång av Sveriges proggflaggskepp Philemon Arthur \& the Dung. Låten släpptes för första gången år 1992 på bandets skiva \textit{Musikens Historia Del 1 o 2} (Silence records), och blev en omedelbar hit bland landets batikklädda befolkning. Sången premierades framförallt för sitt allmogebudskap och sin enkla refräng som var lätt att komma ihåg och sjunga med i. För att ge extra kraft åt sången plockades alla verser bort och texten består enbart av refräng.  Hoola Bandoola Bands träpinneslagare Håkan Skytte \textsc{(se vansinnets historia s.~\pageref{18a20c9ab3852aa00d423ae3a72cfc50})} var från början kritisk och menade att päron inte är ett proletärt bär \textsc{(se proletära bär s.~\pageref{9c2b32147a2af1b7a64cf4ac37a20a94})}. Han fick dock svar på tal av Thomas Mera Gartz i Träd, Gräs och Stenar \textsc{(s.~\pageref{82a271b29bea1b3fd0073fe6668179bd})}, som alltid bjöd sina vänner på päronhalva \textsc{(s.~\pageref{cc9c1bfa2ec4eaed89ca86a1b63e3a45})}. Hösten 1992 svämmade den svenska matpressen över av delikata recept på päronsoppa, spagetti med päronsås, fyra små päronrätter och päronburgare. Succén \textsc{(se braksuccé s.~\pageref{678371d35369d3d29afceb1445630833})} blev dock för mycket, och Philemon Arthur \& the Dung har aldrig sjungit om någon annan frukt sedan dess.

 HEAD2: Text
 \textit{Jag plockar, plockar, plockar päron}
 \textit{Plockar päron, plockar päron}
 \textit{Plocka, plocka, plocka päron!}
 \textit{Plocka päron! Plocka päron!}

}

\small{
\textbf{Plugg}
\label{b72b2b4322e41e2b31a84082007ae037}
 är söderslang för potatis och används inom köksbranshen.
 \quotetext{Kirra plugg} betyder således \quotetext{gå och hämta potatis} eller \quotetext{Gå och köp potatis.}

}

\small{
\textbf{Pluta}
\label{16ef63e4111915f4c039fc33d91ddd33}
 Att pluta kan vara två saker. Den ena är att med läpparna göra en pussmun. Den andra är att nedgradera något. T.ex. om någon säger att lunchen var jättegod, för att sedan ändra sig och säga att den var \quotetext{Helt OK}, så har således lunchen blivit plutad.

 HEAD2: Etmyologi

 Verbet härstammar från sommaren 2006 då Pluto blev nedgraderad från planet till en vanlig jävla himlakropp.

}

\small{
\textbf{Pneumatiska rör}
\label{76a226cef94924fb3f03b3652deb16e7}
 är ledningar i vilka information eller saker (så som pengar) skickas i små kapslar med hjälp av lufttryck. Idag används pneumatiska rör i första hand för att skicka pengar från kassa till bankvalv i större varuhus och banker, men då det begav sig, i viktorianska England, hade man planer på att försöka blåsa folk och allt möjligt genom tunnlar under marken. Det sägs att parisarna \textsc{(se parisare s.~\pageref{5aca28013b9a7e4088e7fb228f3e4827})} var så taggade på uppfinningen att ett nätverk byggdes under jord i centrala Paris, men att detta aldrig kom till användning eftersom det troligtvis skulle ha blivit sjukt trassligt.

}

\small{
\textbf{Poetisk rättvisa}
\label{72fd78f6f034d86b7e7dc10501a38bfc}
 Drabbar ofta onda människor såsom:
 \begin{itemize}
 \item Ayn Rand
 \item Jörg Haider
 \end{itemize}

}

\small{
\textbf{Polask anti-fascism}
\label{6d9a0cbff8ecc14648d1849710c87329}
 Polsk anti-fascism \textsc{(s.~\pageref{fdd955614e489c1c62b5eb4234e1c5ec})}

}

\small{
\textbf{Polis}
\label{fa296149fa58bfd4408e407cc3fd3be5}
 ens uppgifter är varierande men till största del finns de till för att förhindra att man har roligt.
 Tycker du om att köra bil så har polisen alltid laglig rätt att stoppa dig för att sedan ge dig böter, har du inte gjort något fel så kan du få böter för det också.

 Polisen har två typer av hot som dom gärna svänger sig med, dessa är:
 Skall jag ge dig böter eller?
 Och det andra är: Ska vi ta med dig till stationen kanske?

 Är du en \quotetext{före-detta buse} så kommer du alltid att få höra spydiga kommentarer och få ett taskigt bemötande av en polis, det är polisens huvudsyfte; att minnas om du varit dum vid några tillfällen för 20 år sedan.

 Det finns poliser som tycker om att bryta av armarna på dem de griper och slå batongen mellan benen på en överförfriskad gammal man som kissar mot fel byggnad i staden och så finns det trevliga snälla poliser, ca 25 st i hela landet, 22 av dom har gått i pension. De som är kvar är generellt mer kriminella än genomsnittsmedborgaren; vilket i mångt och mycket förtar poängen med en poliskår.

 HEAD3: Polis i populärkultur
 Polisen är ett tacksamt objekt att besjunga. Företrädesvis av råpunkare.
 Därtill finns flera tv-kanaler som enbart förhärligar dessa ruttnande själars värv. Osmakligt är vad det är.

 HEAD3: Kontakta Polisen
 Poliser hittar man ofta på Cafe Bygård i Rutvik, det kallas att \quotetext{patrullera}. Annars är 7-11 ett säkert kort. Om nån tjackis försöker sno din bil så är det bäst du har en cykel på lut så du kan cykla iväg dit och meddela,ty nå dom feta svinen på telefon är stört omöjligt,och sätter du bollträet i nacken på den lilla hippieparasiten kan du se fram emot ett par år bakom galler på hotell Gripen \textsc{(s.~\pageref{b2f675b76432ab4dafb41f8b683bd35a})}.

}

\small{
\textbf{Polsk anti-fascism}
\label{fdd955614e489c1c62b5eb4234e1c5ec}
 Att förklara totalt krig mot nazisterna och att föra sådant, utan hjälp från omvärlden och med hemmagjorda vapen, tills nazisterna inte ser någon annan utväg än att jämna mer eller mindre hela Warsawa med marken.

}

\small{
\textbf{Polsk omkörning}
\label{40180c19fd3547a7c5490b9bcb5ba73d}
 Att köra om genom att tvinga ut mötande trafikanter i väggrenen.

}

\small{
\textbf{Polsk parkering}
\label{24eec1681ab8b8450fc1aeee9b0faceb}
 Enligt en källa så parkerar polacker sina bilar genom att baxa upp bilen med domkraft, plocka bort hjulen för att sedan placera dom inunder bilen. Detta gör dom för att inte få bilen stulen.
 Ingen vet om det ligger till så att nycklarna vanligen sitter kvar i bilen, så är man väldigt sugen på att prova tjuvköra en polskt parkerad bil så kan det räcka med att man har med sig ett fälgkors.

}

\small{
\textbf{Polsk riksdag}
\label{1447e47946b967e549fe3b8f67aca498}
 är ett begrepp som syftar till att beskriva en församling som präglas av en massa skrik och gormande. Det här har i själva verket inget att göra med riksdagen i Polen, som lär gå till  i lugn och sansad ton som så många andra riksdagar. Britternas parlament är däremot den polskaste som vi står att finna idag, där debatten enligt deras \textit{common law} måste föras i ett högt tonläge. Partikollegorna såväl som meningsmotståndarna förväntas också kommentera å det högljuddaste.

}

\small{
\textbf{Polska helgdagar}
\label{d583de044ed76bea1eaa00e763c8785f}
 Polen är ett land som gillar att fira saker. Och vad är väl ett bättre sätt att fira än att vara ledig från kneget? För att alla ska få vara lediga och fira samtidigt ibland har Polens riksdag \textsc{(se polsk riksdag s.~\pageref{1447e47946b967e549fe3b8f67aca498})} infört en rad nationella helgdagar. Vissa har man lånat in från andra håll i världen och vissa har man hittat på helt själv. Nissepedia kan som första webbaserade uppslagsverk presentera en redogörelse för samtliga av Polens helgdagar på svenska.




 \begin{itemize}
 \item \textbf{1 Januari} En klassisk helgdag hos alla länder som kör med den gregorianska kalendern. Eftersom nyåret inträffar mitt i natten är det klart att man vill ha lite sovmorgon dagen efter. Inget konstigt med det.
 \end{itemize}

 \begin{itemize}
 \item \textbf{6 Januari} Denna dag firar man i Polen att dom tre vise männen hittade Jesus. Redan här blir man lite misstänksam mot polackernas benägenhet att vara lediga från jobbet. Är det verkligen något att fira att tre snubbar kommer och grattar Jesus på födelsedagen flera dagar för sent?
 \end{itemize}

 \begin{itemize}
 \item \textbf{Påskdagen} Jesusrelaterat igen. För de som bryr sig var det ju himla tråkigt att Jesus blev uppspikad på ett kors. Eftersom Polen är ett katolskt land förstår man att polackerna vill vara hemma och tänka lite extra på Jesus då.
 \end{itemize}

 \begin{itemize}
 \item \textbf{Annandag påsk} Eftersom påskdagen alltid infaller på en söndag kan man förstå om polackerna känner sig lite blåsta på konfekten eftersom de flesta ändå redan är lediga på söndagar. Då är det klart man vill kompenseras med en ledig måndag också.
 \end{itemize}

 \begin{itemize}
 \item \textbf{1 Maj} Officiellt är denna dag inte arbetarrörelserelaterad i Polen. Det kan tänkas hänga ihop med att en del polacker är rätt lacka på kommunismen men fortfarande gillar att ta ledigt för att fira. Man har dock inte hittat på någon annan anledning heller så lite konstigt är det.
 \end{itemize}

 \begin{itemize}
 \item \textbf{3 Maj} Här börjar man ana ett mönster. Helgdagarna ligger ofta två och två i tät följd, vad är grejen med det? Troligtvis är polackerna lite deppiga efter den första helgdagen, det var ju så fint att vara hemma liksom. För att muntra upp dom sätter man in en helgdag till, och ångesten känns genast mindre tung.
 \end{itemize}

 \begin{itemize}
 \item \textbf{7:e söndagen efter påsk} Okej, det låter lite konstigt, men spelar inte så stor roll eftersom det ändå är på en söndag.
 \end{itemize}

 \begin{itemize}
 \item \textbf{9:e torsdagen efter påsk} Här börjar det kännas som att polackerna inte är helt seriösa när dom instiftar sina helgdagar. Är inte det här mest en ganska illa formulerad ursäkt för att kompensera för att många var osmarta nog att ta ut all semester i början av sommaren innan det hunnit bli riktigt varmt?
 \end{itemize}

 \begin{itemize}
 \item \textbf{15 Augusti} Jungfru Marias himlafärd. Firar man att Jesus dör i två dagar kan man väl fira att hans morsa dog i en dag, tänkte antagligen polackerna.
 \end{itemize}

 \begin{itemize}
 \item \textbf{1 November} Alla helgons dag. Alla andra helgdagar i Polen som har med firande av döda att göra handlar ju om folk man inte känner personligen. Någon gång kan det ju vara nice att fira dom man faktiskt träffat också.
 \end{itemize}

 \begin{itemize}
 \item \textbf{11 November} Påvens födelsedag. Enligt Polsk-katolsk tro fyller alla påvar år denna dag. Fast egentligen handlar det nog mest om att polackerna tycker det är rätt spejsat med ett datum som bara består av 1:or. Och så är det ju lätt att komma ihåg därför också, vilket säkert var viktigt förr i tiden när man inte gick så länge i skolan och samtidigt ofta var undernärd.
 \end{itemize}

 \begin{itemize}
 \item \textbf{25 December} Den stora finalen på alla andra Jesusrelaterade dagar man firat under året. Och så får man ju dricka glögg också.
 \end{itemize}

 \begin{itemize}
 \item \textbf{26 December} Då är det klart att man vill fira lite extra och lyxa på med två lediga dagar på rad.
 \end{itemize}

}

\small{
\textbf{Pompekunskap}
\label{42397593fde2dd1727b411ada4a9adbb}
 Samma sak som Nissepedia \textsc{(s.~\pageref{62400dadecd90cb5cd39062abe5a3e4a})}.



 HEAD2:  Se även

 Nisse Schwartz
 Brandklipparen \textsc{(s.~\pageref{e8aaa0dc22fb08e055f0f48b2f25e35d})}

}

\small{
\textbf{Ponera}
\label{81de0f38ad2cd422870c2e70763f3510}
 Sätta sig in i en konjektural \textsc{(s.~\pageref{97b105b94d1adc2125ccd7409f18beda})} situation.

 Exempelvis: \quotetext{Ponera  att en italiensk sportbil håller längre än en svensk stålåsna \textsc{(se Volvo 740 s.~\pageref{e262951543da05bac43c7b87235a115c})}}.

}

\small{
\textbf{Pop-rock}
\label{d559954a05239a86feae3a0d3216cf56}
 är en blandning av dels pop och dels rock. Ofta är det svårt för en artist att definiera sin personliga stil. \quotetext{Jag spelar både pop och rock,} kan man höra svenska musiker säga. Ofta går artisten sin egen väg, trött på att föras in i olika \quotetext{fack}. Musiken får tala för sig själv, utan en massa regler och förväntningar. Vi har denna otvungna attityd att tacka för odödliga låtar som Tomas Ledins \quotetext{Sommaren är kort}.

}

\small{
\textbf{Popemobil}
\label{f799fd993a87f98705d3cbb95faff6e0}
 e \textsc{(s.~\pageref{78f38a857e764c025bde36ef4196f63e})}

}

\small{
\textbf{Popemobile}
\label{78f38a857e764c025bde36ef4196f63e}
 En popemobile är en mindre kusin till läderlappens asfeta bil, som dock är betydligt mer att åka med än popemobilen. Med det sagt har popemobilen vissa klara fördelar framför vanliga personbilar då dess kub av skottsäkert glas tillåter att man står upp i den och vinkar till de fotgängare man mot förmodan kör om. Bilmodellen är populär bland fruktansvärt gamla nazianstrukna östeuropeiska män i lustiga hattar. Högst densitet av dessa bilar \textsc{(se bil s.~\pageref{b3188f47d2eac7efc3f1258dc673a9fe})} finner man i Vatikanstaten \textsc{(s.~\pageref{2a3f7cd77d26fb21d605c562c409d7e9})}, varför flertalet reklamfilmer för dylika fordon vänder sig speciellt till straffade pedofiler.

}

\small{
\textbf{Porr}
\label{9b8b59ed4eeed8840fee4480feb98f17}
 är ett fenomen som främst män utan fantasi använder sig av för att tömma sin säd (ur pung/ prostata). Porr finns tillgängligt främst på internet i form av bilder och filmer och är en stor bidragande faktor till bråk och svartsjuka inom relationer.

}

\small{
\textbf{Porsche 911}
\label{8750acbc992e40a29eeca5c71a21d4b8}
 är ett slags bil som oftast är gul och har motorn i bakluckan, konstigt nog. Att köra en sådan är likvärdigt med att erkänna sig besegrad av den allmänna omvärlden, eftersom det betyder att man lider av småhetsvansinne \textsc{(s.~\pageref{09beca787ad2414755414613ca522605})}.

}

\small{
\textbf{Porträtt av det postmoderna renässansgeniet som ung}
\label{131ac0f3032dab26eb216742734f8c6a}
 \textit{Porträtt av det postmoderna renässansgeniet som ung} är en självbiografisk roman av den svenske belletristen, smugglaren och skriftställaren Prof. Etienne \textsc{(se Användare: Prof. Etienne s.~\pageref{a9878d2280e5a39becac8f73d113df91})}. Romanen handlar om en ung man, Stefan Dahlås, med konstnärliga ambitioner och dennes spirituella utveckling. Till yttermera visso handlar den om Dahlås' intellektuella utveckling. I de första tjugo kapitlen får vi följa bokens hjälte då denna får kontakt med intellektuella musiker. Denna del av romanen baserar sig på den period i Etiennes liv som han så förtjänstfullt beskrivit i Självbiografi, del 2 - De förlösande thinneråren \textsc{(s.~\pageref{b90e1d79e5552c043da4a3fea8505e82})}. Nästa utvecklingsfas tar vid då huvudpersonen kommer i kontakt med Margit Sandemos \textsc{(se Margit Sandemo s.~\pageref{d4a62753375ff2e975534b9ca740fd28})} böcker om Isfolket, vilket utlöser en period av kvasi-religiöst grubbel hos Dahlås. detta grubbel återges som en öppen medvetandeström, en litterär teknik som Prof. Etienne enligt egen utsago ska ha uppfunnit, men också i form av dialog mellan huvudperson och bikaraktärer som representerar olika institutioner och filosofier och livsval. Boken slutar dramatiskt med att Dahlås accepterar sin lott i livet och kommer till insikten att det ska levas till fullo efter att ha sett Ivar Bryntse \textsc{(s.~\pageref{a5e922bb2ad7c32a2419b5ba3afcdc99})} beställa en korv i smörpapper \textsc{(s.~\pageref{401e9eb6cef7fa42d543ef85f5925021})} utanför Zinken, vilket för honom representerar ren livsglädje och -intensitet. Bokens epigraf är hämtad från Linnés Systema Naturae och lyder: \quotetext{Människans fysiologi, både den yttre och den inre, visar att korv och rullpizza utgör hennes naturliga föda.}

}

\small{
\textbf{Post}
\label{42b90196b487c54069097a68fe98ab6f}
 Ett populärt prefix för allehanda frifräsare. Postgult hade exempelvis hela den tyska familjen Thurn und Taxis länge förstatjing på. Men allt roligt måste ha ett slut.

}

\small{
\textbf{Post-coitus}
\label{c0b154a4b061b684da6fb1e3cbe6a843}
 Latin för \textit{efter samlag}.

 Uttrycket används för att beskriva skedet efter ett fullbordat könsumgänge mellan två eller flera parter. Inom begreppet finns flera dimensioner, då det post-coitala tillståndet kan variera från gång till gång. Om man är lyckligt lottad innebär det att röka cigaretter, småtjafsa och skoja och kanske dricka ett glas vin i sängen med den man haft sex med. Men det kan också innebära inträde i det psykologiska tillståndet \textit{post-coital tristesse}. Med det menas en lätt melankoli som sprider sig i kroppen efter könslek. Och visst är det så att det går att känna igen sig i ett milt missnöje som kittlar i en, när man ligger i en säng efter fullbordad akt, och känner ejakulat och andra könsvätskor kallna och stelna, fastklibbat mot ens bäcken.

}

\small{
\textbf{Posten}
\label{cd13d688571681e426231485b732444b}
 Den förut så stolta institutionen posten för i vår digitaliserade och privatiserade värld en tynande tillvaro. Därför är det skönt att veta att en av våra mest namnkunniga radiopersonligheter, Jonas Hallberg, är en ivrig tillskyndare när det gäller just posten och postverket. De i politiken som ägnar sig åt att privatisera \textsc{(se privatisering s.~\pageref{19d879b8459370dc917f278e51f90c2b})} och att göra nedskärningar i välfärden, till exempel folkpartiet \textsc{(s.~\pageref{b692fa6a23fd557940474dc94909d80f})}, kristdemokraterna \textsc{(s.~\pageref{18a843e4776b5003d411ce0148bab148})}, moderaterna \textsc{(se moderat s.~\pageref{c4564b188cb670841733a3ff923c2fb0})}, centerpartiet \textsc{(s.~\pageref{e331dec360e356adc1e2db36fe9a9f3f})} och socialdemokraterna, har gått så till väga att man först ersatte alla postkontor med något märkligt som kallades för Svensk Kassaservice, för att skapa förvirring. Sedan en dag togs dessa kontor bort och alla postångare \textsc{(s.~\pageref{8982ca3edf9c09ab93590d16ee7c2387})} sänktes och alla postlådor \textsc{(se brevlåda s.~\pageref{82763e379777595d3c8c8f7b38e51bbd})} flyttades samt fick en ny design. Därför krävs det idag noggrann planering och lite jävlar-anamma för att skicka iväg ett brev.

}

\small{
\textbf{Postiljon}
\label{cdf093679eb50a181e4ef701ed856b97}
 1000000000000000000

}

\small{
\textbf{Postkolonialism}
\label{c4d81020416c923fdc6b91cafd3bbf5e}
 är en vetenskaplig teoribildning som i korthet går ut på att se hur den tidigare kolonialiserade delen av världen påverkats av detta faktum. Framträdande teoretiker är Edward Said, Homi K. Bhabha \textsc{(se Homi K. Bhabhas son s.~\pageref{66ce2281df988914500cb1c269d7418f})} d.ä, Gayatri Chakravorty Spivak, Trinh T. Minh-ha och Chandra Talpade Mohanty. Det vanligaste postkoloniala inslaget i vanliga människors vardag är det berömda maximet \quotetext{Men tänk på barnen i Afrika!}

}

\small{
\textbf{Postlåda}
\label{8134b1e3ad53642c3d3373e6ea72ed35}
 En postlåda är en behållare som används av privatpersoner och företag för att ta emot information och objekt som sänts från jordens alla hörn. Det är så genialiskt att det nästan låter löjligt.
 Se även: posten \textsc{(s.~\pageref{cd13d688571681e426231485b732444b})}

}

\small{
\textbf{Postmodern morförälder}
\label{739c4c2e41c756708ce80adef26bf68b}
 Den postmoderna morföräldern är en morförälder som förhåller sig till sitt barnbarn på ett fragmentariskt, motsägelsefullt och liberalt vis. Hos morföräldern finns inga rätt och fel. Morföräldern vill bara barnbarnets väl, men samtidigt finns där alltid ouppnåliga krav som liksom hänger outtalade i luften. Men exakt vad är det som förväntas av det lilla barnbarnet? Morföräldern har lagt svångrämmen på hyllan och levererar julklappar som är könsneutrala. Detta betyder inte att barnbarnet har det särskilt mycket bättre hemma hos morfar och/eller mormor. Istället för aga används subtila ideologiska mind-games mot barnbarnet. Istället för tvång används skuldbeläggande. Istället för bibliska \textsc{(se bibeln s.~\pageref{7de7d2a7d608c9a2044f50688bc63e27})} fabler med solklara budskap inpräntas motsägelesfulla värden medelst förvirrande filmer med hysteriskt dansande djur och animerade gröna troll. Istället för det nedbrytande tickandet från marmoruret stimuleras barnbarnet med ett mummel av olika röster, åsikter, åskådningar, krav och förväntningar. Barnbarnet blir av allt detta mycket förvirrat och vet just inte vad det ska ta sig till. Barnbarnet vill bara sitta och läsa sin morsas gamla serietidningar, men vissa sidor saknas och av berättelserna återstår bara fragment. De obesvarade frågorna hopar sig. Fikat är ekologiskt, men å andra sidan kommer det från Brazilien.
 HEAD2: Se även
 Psykedelisk morförälder \textsc{(s.~\pageref{2679b01c98ac3132281e1d0c114698b8})}

}

\small{
\textbf{Postpostrock}
\label{8fff8f69c6b000bfdbd8b639f2f1a658}
 (jmf. postpostmetal, postposthardcore) är en benämning för rock producerad efter att postrocken slutade kännas OK, det vill säga någon gång  tidigt 00-tal. Termen är inte applicerbar i vissa Europeiska länder, så som Island, där man ännu inte tagit steget in i postpostrockens era utan även i fortsättningen sitter och sjunger i falsett och plingar i nån jävla klocka man hittat i nån gammal låda.

}

\small{
\textbf{Postseminarium}
\label{3dcf7466504a8591f86ba7e472606ef6}
 syftar på det som händer efter ett akademiskt så kallat högre seminarium (post \textsc{(s.~\pageref{42b90196b487c54069097a68fe98ab6f})} betyder \quotetext{efter} på hebreiska). Medan det högre seminariet består i att forskare och doktorander diskuterar och ventilerar manus till vetenskapliga artiklar och avhandlingar går postseminariet ut på att samma doktorander och forskare går till en krog och äter pubmat, dricker sig berusade och antingen idkar älskog eller hamnar i luven på varandra.

}

\small{
\textbf{Postångare}
\label{8982ca3edf9c09ab93590d16ee7c2387}
 är båtar som främst transporterar post och helt eller delvis drivs av ånga från ångmaskiner. Tidigare hade den som varit sugen på att skicka ett vykort från en kontinent till en annan varit begränsad till att be en salt sjöbuse eller en välvillig men ack så naiv ballongfarare transportera försändelsen. Vilket i 99 fall av 100 resulterade att brevet försvann, och i det sista fallet blev kvarglömt hemma. Världens postiljoner \textsc{(se postiljon s.~\pageref{cdf093679eb50a181e4ef701ed856b97})} följde med rödgråtna ögon denna veritabla misshandel av skråets själva grundsten. Men eftersom postens \textsc{(se posten s.~\pageref{cd13d688571681e426231485b732444b})} medarbetare alltid varit det offentliga samhällets stoltaste yrkesgrupp tog man saker i egna händer och började skeppa över försändelserna för egen maskin. Från början gick det åt en hel del brev för att hålla ångan uppe i pannorna, men sedan telefonkatalogen \textsc{(se telefonkatalog s.~\pageref{a1c3d8187f7afc13f933d7d93b27f536})} uppfanns har över hälften av all skickad post till en transoceanisk adress tagit sig fram.


 HEAD2: Trivia
 Postångare är ett viktigt inslag i nästan alla Jules Vernes böcker.

}

\small{
\textbf{Potatisbar}
\label{0618111d978b96a54684e25267085a86}
 En potatisbar är ett matställe där allting har anknytning till den delikata rotfrukt dansken \textsc{(se danmark s.~\pageref{5331d7fd27772396f412a5b6d19bad44})} känner som kartoffel. Till förrätt kan man till exempel få en tallrik mos garnerat med pommes och till huvudrätt King Edward-rakor \textsc{(se King Edward s.~\pageref{a081b9ae5423fc12a8439e33b2af8bed})} vända i potatismjöl av finaste Blå Kongo. Detta sköljs ned med ett glas bärs \textsc{(se bärsfylla s.~\pageref{9380b60f9ee744b9acf978fe6f1a9545})} som hällts upp i en urgröpt Bintje \textsc{(s.~\pageref{f21f4f64cb0df1775b5c2a7dc0d83c6c})}. Till efterrätt blir det självklart potatisschwish med lika delar, Jansson, gräddpytt, Hasselback \textsc{(se hasselbackspotatis s.~\pageref{9354ad56ba8ac2ead1daeea852c88bec})} och en Magnum Bonum på toppen.

 På barnmenyn återfinns traditionella rätter som förgyllts med en potatistouch. Det kan exempelvis vara spagetti och potatis, fattiga potatisar \textsc{(se fattiga riddare s.~\pageref{c53ee9c63bb93f36773e3c72dcccb306})}, quatro potato eller potatisar med sylt \textsc{(s.~\pageref{be6fa0cf731bdae2e13ad52084c90fbc})} och grädde.

}

\small{
\textbf{Potatistryck}
\label{a658ef6f51769cd542118c30fddd3bf1}
 är en enkel metod för att uttrycka sig.
 En potatis delas i två delar och sedan karvas det ut ett motiv som sedan doppas i färg. Det fungerar bra med vattenfärg men som vanligt är det roligare med en typ som aldrig går bort. Den enkla grundregeln är att det som sticker upp är det som syns.
 Men spegelvänt förekommer också, eftersom en del grafiker älskar att se när motivet de skapat plötsligt förändras och blir något annat - det blir speglat och nästan främmande, som att röra sig själv med en handske. Potatistryck används ofta i terapeutiska syften för att få barn \textsc{(s.~\pageref{5dfcc0aab2f3db925b2d51ba73e48946})} och gamla att reproducera sin talang.
 En del utövare av så kallad gatukonst använder sig av de små charmiga knölarna för att trycka sina budskap på stolpar och i hörn.
 HEAD2: Se också
 Bintje \textsc{(s.~\pageref{f21f4f64cb0df1775b5c2a7dc0d83c6c})}

}

\small{
\textbf{Potin}
\label{be378416f985a5882d6d4d061d9cffbb}
 Den gänglige trumslagaren i Diagnos Damp.

}

\small{
\textbf{Praktarsle (negativ)}
\label{97a372c56edf8bade3fc4bdc4456f303}
 Praktarsle kan man kalla en person som man tycker gör dåliga saker. Om en person till exempel ringer till Ticnet samma dag som det släpps biljetter till en konsert med Bruce \quotetext{Bosse Sprängsten} Springsteen, och kommer fram, men istället bokar plåtar till ett Status Quo-coverband, ja då har man att göra med ett riktigt praktarsle.

 Se även: Praktarsle (positiv) \textsc{(s.~\pageref{ec1f5a634088019acf000718397d0b8a})}

}

\small{
\textbf{Praktarsle (positiv)}
\label{ec1f5a634088019acf000718397d0b8a}
 En på alla sätt iögonfallande ändalykt.

 Se även: Praktarsle (negativ) \textsc{(s.~\pageref{97a372c56edf8bade3fc4bdc4456f303})}.

}

\small{
\textbf{Prikosnebelurk}
\label{a74a351517d7039d68cd88c34092cc14}
 En efterfrågad burk konserverade aprikoser, eller möjligen persikor, klockan 03.27 en söndagmorgon efter löning/studielån i vilken nattöppen jourbutik som helst.

}

\small{
\textbf{Prins Charles}
\label{9b545432192db93ace6632261cc410eb}
 är en vingmutter \textsc{(s.~\pageref{3d474f53ae61d29d3b924d44c21410b5})} och brittisk tronföljare, men helst av allt vill han vara en tampong.

}

\small{
\textbf{Prinskorv}
\label{981d6501577f8e905435799959f99cb2}
 HEAD3: Användning
 Ersätter morötter i recept där detta föreskrivs.
 HEAD3: Prinskorv i kulturen
 Hundar gillar att sno långa snören prinskorv hos slaktare och springa, glada som fan, längs stadens gator.

}

\small{
\textbf{Privatisering}
\label{19d879b8459370dc917f278e51f90c2b}
 Stöld från folket.

}

\small{
\textbf{Privatspanare}
\label{b7a4113e7c457f65a55f866e146bcf69}
 En privatspanare är en privatperson som hjälper polismyndigheten att lösa brott. Många poliser tycker det är mycket roligare att slå sönder saker och arrestera istället för att lägga ihop ett och ett, så därför sköts istället mycket av det jobbet av privatspanande eldsjälar. Likt bibelns berättelse om den barnhärtige samariten handlar privatspanare helt utan egen vinning. En rättvisare värld är istället vad som driver dessa välvilliga förebilder. Generellt kan man säga att ju svårare brottet är, desto fler privatspanare behövs för att lösa det. Alla privatspanare tar fram sin egen teori och dessa ställs sedan mot varandra tills den bästa vunnit och brottet blivit löst. Poängsystemet är tyvärr för komplicerat för att redogöras för här men ibland kan det ta flera år innan man kommer fram till vilken teori som fått flest poäng. Likt seriespelet i bandyallsvenskan kan en teori toppa länge men sedan tappa mark och se sig omsprungen av andra. Det kan låta orättvist men hänger ihop med att sanningen alltid måste segra. Sveriges skickligaste privatspanare är Pelle Svensson \textsc{(s.~\pageref{26d88b383fd38f349c7741ca7051904e})} som bland annat löst palmemordet, friat Tomas Quick och tagit OS-silver i brottning. Nissepedias främste privatspanare är medarbetaren Ekis \textsc{(se Användare: Ekis s.~\pageref{f02c98447a157dd8c654b3c15ea88f29})} som vigt sitt liv åt att bringa upprättelse åt medborgarrättskämpen Lars Tingström och att avslöja finansskurken \textsc{(se storfräsare s.~\pageref{4db17005692cd83e3e946a1311b81ed0})} Jan Stenbeck.

}

\small{
\textbf{Problematiskt}
\label{6a7aeb89dad7c33d014faa5010737dbf}
 Att använda ordet \textit{problematiskt} för att beskriva en företeelse är i nio fall av tio helt menings- och ryggradslöst. Bruket av ordet ska tillskriva den som använder det en överlägsen intelligens, som att den förstod något på ett flertal nivåer, med en flugas mångfacetterade blick, och som att personen hade en ständigt rasande dialog inom sig, \quotetext{är detta rätt eller fel?}. Läsaren/lyssnaren ska ges intrycket att personens hjärna är fylld av kolliderande dialektiska åskviggar, en urladdning som skulle kunna lysa opp hela Roslagens Hallsberg (det vill säga Rimbo), hade bara blixtarna manifesterats i fysisk form, inte psykisk.

 Egentligen signalerar användandet av \textit{problematiskt} bara att en person inte törs säga om den tycker att någonting är dåligt eller bra.

}

\small{
\textbf{Processa mot länsstyrelsen}
\label{0ae3fdeda52fe82800b04c624330139c}
 Att processa mot länsstyrelsen är en gammal hederlig svensk tradition som syftar till att manifestera medborgarnas autonomi mot kronan. I ödsligare delar av landet betraktas en persons första process mot länsstyrelsen ofta som ett tydligare steg in i vuxenvärlden än konfirmationen eller det första hemmakokade brännvinet \textsc{(se brännvin s.~\pageref{ff49ececa32cff978496a39635496f46})}. De abrahamitiska religionerna snappade tidigt upp traditionens betydelse och lät föra in ett avsnitt i Gamla Testamentet som behandlade temat under rubriken \quotetext{David och Goliat}. Tanken med detta var ungefär densamma som när man lät döpa om midvinterblot till julafton.

 Se även trasmattans dag \textsc{(s.~\pageref{657be5ffddeb957c97682755edcffe6e})}

 HEAD2: Populära frågor att processa om

 \begin{itemize}
 \item Rätten att ha hur många bilvrak man behagar på gården.
 \item Att vargen redan var nedgrävd när man kom till platsen.
 \item Att man bygger vad fan man vill på sin egen tomt.
 \item Att hastighetesgränsen borde höjas på de vägar man själv använder mycket.
 \item Att det inte är någon som använder naturreservatet, så det är visst en bra idé att köra enduro där.
 \end{itemize}


 HEAD2: Legendrarer inom aktivitetens historia
 \begin{itemize}
 \item Folke Pudas \textsc{(se Pudaslåda s.~\pageref{6a56958e2057dd500650e2be8049e033})}
 \item Bengt Sändh \textsc{(s.~\pageref{14671d7872985203e69e4392e5754734})}
 \end{itemize}

}

\small{
\textbf{Produktionsknull}
\label{c89065c54da84e9bc9dc59992a4dffa8}
 Ett produktionsknull är ett samlag vars enda syfte att producera en avkomma. Det är ett kallt, själlöst arbete. Det kan utföras så väl under lysrörsbelysning som framför en sprakande brasa, utan att det gör någon skillnad. Njutning är underkastat. Allt som spelar roll är att ättenamnet fortlever. Man kan även produktionsknulla för att säkra sin pension. Barn kan vara bra att ha om man vill bli omhändertagen på ålderns höst.

}

\small{
\textbf{Prof. Etienne}
\label{56957a267e57df32753cf7f3b8a603d8}
\begin{enumerate}
\item REDIRECT Användare:Prof. Etienne \textsc{(s.~\pageref{5f97f2ebbb66f5067dd278b973466e75})}
\end{enumerate}

}

\small{
\textbf{Professor skytteanus}
\label{8b3f8bc992a03d22bfb551a9e1d0478a}
 är en yrkestitel som innehas av den manlige akademiker som erhåller Johan Skyttes \textsc{(se Johan Skytte s.~\pageref{bfb1371ca26a0f7f54616d1076e7adf1})} professur i retorik och statsvetenskap vid Uppsala universitet. Idag innehas den av Li Bennich-Björkman, men hennes titel är professor skytteana för att hon är brud. Det är världens äldsta professur, till råga på allt.

}

\small{
\textbf{Projekt Nissepediabok}
\label{184889fddfc7ebde8d9af598d77f229e}
 :-D

}

\small{
\textbf{Prokrastrinering}
\label{45b389ec37bb87511b773b3e565670d4}
 är ett slags beteendemönster som är speciellt vanligt hos unga män från övre arbetar- eller lägre medelklass och tar sig uttryck i att den unge mannen skjuter upp att göra saker genom att sysselsätta sig med en annan, mindre produktiv men minst lika tråkig aktivitet. I det protestantiska Norden ser många på prokrastrinering som något ont, vilket är en tämligen onyanserad hållning, medan det i själva verket har medfört många positiva och (för andra människor) uppbyggliga resultat under årens lopp. Nissepedia \textsc{(s.~\pageref{62400dadecd90cb5cd39062abe5a3e4a})}, med dess mer än 1200 artiklar, är ett typexempel härav.
 HEAD2: Prokrastrinering som psykologiskt fenomen
 Inom psykologin förklaras prokrastrinering som ett slags uppsättning av mekanismer genom vilka ett företagande som i sig är minst lika oattraktivt för individen \textsc{(se individ s.~\pageref{41beed76a0af9b4f550f7ebdecd3e700})} som det man förelagts att utföra framstår som mer önskvärt. Istället för att skriva uppsats samlar man med händerna damråttor från golvet under sängen. Istället för att klippa gräset ser man hur många stenar man kan pricka ett visst träd med i följd. En skicklig prokrastriatör kan i princip hitta en aktivitet att sysselsätta sig med i vilken given situation som helst och i avsaknad av extern input eller redskap.
 HEAD2: Prokrastrinering som kampmetod
 Inom arbetslivet liksom den moderna familjen används prokrastrinering som ett slags kampmetod i lågintensiva motsättningar mellan kapital och arbetare (sommarjobbare) eller familjeöverhuvud och lägre rankade familjemedlemmar. Kampmetoden går ut på att in i det sista, när arbete inte längre kan undvikas, spela \textit{Angry Birds} eller skriva nissepedia-artiklar och på så vis försinka och fördröja den utdelning som den ekonomiska eller geneaologiska makten förväntar sig av prokrastriatören.

 HEAD2: Lista på populära aktiviteter inom prokrastrineringens ädla konst
 \begin{itemize}
 \item Betapet
 \item Kasta macka
 \item Tugga tuggummi och lyssna på rock'n'roll
 \item Skriva på nissepedia \textsc{(s.~\pageref{62400dadecd90cb5cd39062abe5a3e4a})}
 \item Tetris \textsc{(s.~\pageref{f76e534251c8595a9746fde225f9289b})}
 \item Tälja
 \item Röka gräs
 \item vilskita \textsc{(s.~\pageref{d8991eedd83b1eb75ae7c2cf9daaad92})}
 \item Bygga en fågelholk som rymmer en solpanelsdriven bandspelare med en loopkassett med \textit{Surfin' bird \textsc{(s.~\pageref{a6167bedaff9931674f3f67c27f8607c})}}
 \end{itemize}

}

\small{
\textbf{Proletära bär}
\label{9c2b32147a2af1b7a64cf4ac37a20a94}
 är bär som spisas i stora mängder av arbetarklassen. Gemensamt för dessa bär är att de växer vilt i stora mängder, så vem som helst med lite tid över kan plocka dem och konsumera. Har man inte tid kan man också köpa dem relativt billigt i sin närmsta konsumbutik \textsc{(s.~\pageref{70e4875f7c2c177596305006e46b7ca9})}. De två mest klassiska proletära bären är blåbär (till arbetarens morgonfil) och lingon (till arbetarens blodpudding på lunchen). Även smultron och hjortron räknas faktiskt till de proletära bären, men hör till den speciella undergruppen \textit{proletära festbär}. En annan undergrupp är \textit{trasproletära bär} och dit räknas rönnbär och enbär.

 De proletära bärens antites är de borgerliga bären med jordgubben i spetsen. Denna måste planteras och vattnas och skyddas mot fåglar; inte så proletärs direkt. Även körsbäret hör hemma här eftersom det växer på ett träd som tar skitamånga år att växa till sig.

 Bananen \textsc{(se Banan s.~\pageref{aec7bd708ed2ad3435b9a9883ac7f45c})} står som vanligt utanför den gängse klassificeringen. Å ena sidan måste det fraktas från andra sidan jordklotet (skapligt borgerligt), men å andra är den ofta smårutten när den väl hamnar i affären (ganska proletärt).

}

\small{
\textbf{Propeller}
\label{5eba517a2887595e2fd711e32090a0a7}
 En propeller är en maskindel som underlättar maskinell drift av marina farkoster. Den sitter vanligtvis baktill på en båt och påminner till formen vagt om Storbritanniens före detta premiärminister, Tony Blair.

}

\small{
\textbf{Propellerkeps}
\label{34087753e20a67ca90f6c51bcae4528e}
 Danmarks \textsc{(se Danmark s.~\pageref{5331d7fd27772396f412a5b6d19bad44})} nationalhuvudbonad. Lanserades först som en del i opinionsbildningen för vindkraft (vid denna tidpunkt fruktade många danskar att minskad kolkraft kunde medföra bristande tillgång på kol så att de inte kunde rena sitt hemmakokade brännvin. Resultatet blev dock det omvända och det är därför alla danskar älskar vindkraft nu. En annan förklaring till danskens kärlek till vindkraft är deras huvudsakliga naturresurs - Lars Krogh \textsc{(se Prutta högljutt s.~\pageref{8fa3e05871b2747109018026471a935a})}.

}

\small{
\textbf{Propp}
\label{8bde165d41e3a0363954a27cc7164e2e}
 En propp är som en patentkork \textsc{(s.~\pageref{1e39785f5bab52f931dac485727645b6})} eller en smäll på käften.
 En propp kan också vara en typ av elektrisk säkring, de flesta människor sparar på proppar som \quotetext{gått}, (dvs proppen har utlösts och bör ersättas med en ny) i nära anslutning till proppskåpet, varför de gör så är okänt men kanhända väntar de på den dagen då ett pantsättnings system införs.

}

\small{
\textbf{Prosopagnosi}
\label{eeeb7f6a3487bed33138a7a03844ecaa}
 är den del av hjärnan som gör att det går att se hur en person mår med hjälp av den personens ansiktsuttryck. Om man skulle få för sig att lobotomera bort den så skulle man inte förstå smileys eller när någon är skitförbannad. Om man funderar på hur det känns att ha lobotomerat bort den kan man försöka bedöma drottning Silvias, eller valfri annan botox-fejja, ansiktsuttryck.

}

\small{
\textbf{Protes Bengt}
\label{921bbbbb29de88e13256319e8559ccc4}
 är ett svenskt mangelband från Stockholm. Bandet existerade under några år på 1980-talet och var ett sidoprojekt till Mob 47 \textsc{(s.~\pageref{9955900bded21660e7f4e15ae8d23e3a})}. Många av bandets låtar handlar om ägg och detta har Mob 47-Åke \textsc{(s.~\pageref{486ee67ac39debabed3d92a7555dcebd})} i intervjuer förklarat med att han tycker om ägg. Bandets mål var från början att spela in en kassett med 300 låtar men detta mål uppnåddes aldrig.


 HEAD2:  Diskografi


 Bengt e sängt kassett
 In Bengt we trust 7"
 Garanterat mangel/In Bengt we trust CD/LP (split med Mob 47)
 Manglar som ägg \textsc{(s.~\pageref{7b1e91fdfd952485ddd3bc6ef4e40b3c})} LP (samlingsskiva)


 [Lyssna\textbarhttp://youtu.be/XWFd9bbgix0]

}

\small{
\textbf{Protes Bengt-Åke}
\label{8a9d100ddd40511581a6c85b2ee84a43}
 Samma som Mob 47-Åke \textsc{(s.~\pageref{486ee67ac39debabed3d92a7555dcebd})}.

}

\small{
\textbf{Prunka}
\label{1a7e455906cae443fe4ac445b6c093e1}
 Att prutta och runka samtidigt. Prutten kan komma antingen från anus eller vara pruttliknande ljud från prunkarens könsorgan. Ibland är prunkning avsiktlig. Ibland är den inte det. Vid ofrivillig prunkning övergår ofta onanisessionen i att man antingen skrunkar \textsc{(se skrunka s.~\pageref{7e3152e0cbea2212bd02444c45ee00db})} eller grunkar \textsc{(se grunka s.~\pageref{6b79b7e074be4c86299c3ee48160b626})}.

 Prunka betyder även att vara grann och att pryda sig, på gränsen till högmod.

}

\small{
\textbf{Prutta högljutt}
\label{8fa3e05871b2747109018026471a935a}
 Att prutta högljutt är en av humor-historiens klassiker. Skämtet har idag en viss rustik atmosfär kring sig, men förekommer då och då även i de mest metropolitiska miljöer. En av de saker som gör skämtet så uppskattat är att det är förenat med en viss risk för den som utför det, eftersom det kan resultera i oönskade och obekväma olägenheter för denne. Lyckas skämtet blir det å andra sidan många gånger en riktig glädjespridare och framkallar skratt och munterhet.

 HEAD2: Högljutt pruttande i den danska kulturindustrin
 Det sägs att Lars Krogh kan prutta så våldsamt att det är som att en vindil drar genom trädtopparna. Om han bara vill, men han vill inte. Han vill bara ge ut sjutummare \textsc{(se sjua s.~\pageref{e7bf63fa6d0d29bd89c23f833b979a15})} med olika garageband.

}

\small{
\textbf{Psykedelisk morförälder}
\label{2679b01c98ac3132281e1d0c114698b8}
 En psykedelisk morförälder är en person som har en dotter som i sin tur har ett barn \textsc{(s.~\pageref{5dfcc0aab2f3db925b2d51ba73e48946})}. Till yttermera visso omger sig den psykedleiska morföräldern av en kaljdoskopisk tillvaro där tonerna av Baby Woodroses \textit{Third Eye Surgery} och Blue Cheers \textit{Vincebus Eruptum} liksom omsluter barnbarnet likt en kokong av mjukhet och föränderliga proportioner. När barnbarnet frågar sin psykedeliska morförälder hur det var förr i tiden sätter denne sin avkommas avkomma i knät, lägger på en \quotetext{lakritspizza} med Lamp of the Universe på skivspelaren och rabblar upp en svårbegriplig massa ord vars första bokstäver tillsammans bildar orden \textit{Pluteus salicinus}. Sedan cyklas det Christianiacykel \textsc{(s.~\pageref{671a1992db86e328dc9c068647d57d6b})} till sjön där den psykedeliska morföräldern gör solhälsning, letar svamp och till sist somnar in, som ett barn, under en gran, medan barnbarnet på uppdrag åt sin mors förälder pliktskyldigt bygger ett sandslott åt Sun Ra.
 HEAD2: Se även
 Postmodern morförälder \textsc{(s.~\pageref{739c4c2e41c756708ce80adef26bf68b})}

}

\small{
\textbf{Pudaslåda}
\label{6a56958e2057dd500650e2be8049e033}
 En Pudaslåda är en låda, stor nog att rymma en fullvuxen människa, som används som ett medel att protestera mot något. Den ursprungliga Pudaslådan användes av den norrbottniske \textsc{(se Norrbotten s.~\pageref{0e8c003b75982032cde152609ee94154})} taxichauffören Folke Pudas när denne skulle processa mot länsstyrelsen \textsc{(s.~\pageref{0ae3fdeda52fe82800b04c624330139c})} (och i förlängningen regeringen) genom att ligga i sin låda på Sergelstorg och vara sur.

 När ni läser detta framstår kanske Pudas som en gammeldags rättshaverist, men i slutändan stod det Pudas 1 - Svenska staten 0. Denna process resulterade i lagen lex Pudas. Pudaslådan är således ett underutnyttjat medel i kampen mot byråkratin. Att kalla Pudas för en folkhjälte \textsc{(s.~\pageref{a7f0eb3e0e8ca199c745f5c9ea550404})} är inte en överdrift.
 HEAD2: Pudaslådan i kulturen
 Pudaslådan besjungs i en sonett av en viss välkänd tidningsredaktör och poet som pratar lite som om han vore särske [http://2.bp.blogspot.com/_lIXH9yYlYvo/Swz5srod8VI/AAAAAAAAEW0/9CldaofE8ko/s1600/700px-Goran-greider-2009.jpg]:

 \quotetext{Sonett \textsc{(se Sonett (engelsk) s.~\pageref{fdb1c2b57d6b6fba90e1f70229e105dd})} till Folke Pudas}
 Högt i skyn flyger en vit pelikan \textsc{(s.~\pageref{ecf1b944439a171dfe1163001feeed19})}.
 Säg, var är vår pungprydda vän på väg?
 Han sjunker nu ner mot mitten av stan
 för att på en liten kub slå sig ned.
 Vad är detta för gåtfullt litet skåp,
 med sina många och arga plakat?
 Inifrån hörs dunsar och ylande gråt
 och snyftningar som är fulla av hat.
 I lådan ligger en norrbottnisk man
 som  processar mot länstyrelsens \textsc{(se processa mot länsstyrelsen  s.~\pageref{0ae3fdeda52fe82800b04c624330139c})} dom.
 Vi får troligtvis veta vem som vann
 då denna märkliga låda står tom.
 För kampen mot trams och byråkrati
 fortgår så länge nån ligger däri.

}

\small{
\textbf{Punk}
\label{f72e9105795af04cd4da64414d9968ad}
 är en musikstil som kom till i USA kring mitten av 70-talet men fick sitt stora genomslag i och med att britterna i Sex Pistols släppte en riktigt, riktigt kass skiva. Förr i tiden var det provocerande och lite farligt att vara punkare. Idag är det lika farligt och provocerande att vara punkare som jazzpjatt.

 Nedan följer en länk till Sveriges enda riktiga punklåt:
 http://www.youtube.com/watch?v=qM6_6mMjxUg

}

\small{
\textbf{Punkgryta}
\label{bf12834e2173c29a65420ed558b53c6b}
 Här följer ett recept på alla malätna crustares favoritmat, punkgrytan.

 1. Bli lite full, gärna bärsfull \textsc{(se Bärsfylla s.~\pageref{9380b60f9ee744b9acf978fe6f1a9545})}.
 2. Kolla om det finns lite mat i diverse containrar. \textsc{(se Sopletare s.~\pageref{a9729f0b2c1c1ec17cec4dc9fdb10007})}
 3. Köp det som fattas, nä jag skojade bara.
 4. Gör valfri mat bestående av olika grönsaker, tillsätt krossade tomater.
 5. Koka ris tills det ser ut som mannagrynsgröt.
 6. Upptäck att grytan bränt fast, rör om, låt bränna fast igen.
 7. Servera, maten får under som inga som helst omständigheter mätta sällskapet.

}

\small{
\textbf{Punkscenshumor}
\label{1a33c499f83c6b860ac9712507a990a1}
 är ett slags humor som är besläktad med PK-humor och liknande grenar på humorns vidsträckta släktträd. Punkscenshumorn består i skämt som på något vis har att göra med punkscenen och som delvis har som uppgift att cementera den gemenskap som punkaren känner då han eller hon umgås med sina punkkompiser. Rolighetsnivån på skämten är i genomsnitt strax ovanför att förlora en närstående till kräfta \textsc{(s.~\pageref{31d4f9ec82e212d1a52dc283f7335710})}. Ett ofta förekommande skämt är att mellan låtar ropa \quotetext{Spela snabbare!} eller \quotetext{Mangla!} Detta skämt har förekommit på samtliga svenska punkspelningar åtminstone sedan mitten av nittiotalet och har alltså traderats mellan generationer. Andra punkskämt har anekdotform och är uppdiktade historier om inom scenen namnkunniga manodepressiva narkomaner som skämtaren i lönndom ser upp till, så som Jonsson i Anti-Cimex. Ännu en gren av skämt går ut på att antingen häckla äldre band som inte spelar punk eller att skämtaren påstår sig lyssna på ett band som anses vara dåligt eller töntigt, som exempelvis Rövsett eller Moderat \textsc{(s.~\pageref{c4564b188cb670841733a3ff923c2fb0})} Likvidation. Det som gör denna humor speciell är att ingen faktiskt tycker att den är rolig. Liksom ett pidgin-språk är den heller inte någon individs faktiska \quotetext{modersmålshumor,} utan används enbart inom \quotetext{scenen.}

}

\small{
\textbf{Punschklubb}
\label{e69eff4dce860329d1222368c45e9332}
 En punschklubb är ett forum där en samling stadgade figurer möts för att tillsammans under mer eller mindre burdus humor ponera \textsc{(s.~\pageref{81de0f38ad2cd422870c2e70763f3510})} över sin världsbild. Ett exempel på en punschklubb är nissepedia \textsc{(s.~\pageref{62400dadecd90cb5cd39062abe5a3e4a})}.

}

\small{
\textbf{Putsbilar}
\label{cb242351ab9f9d6b8a5afe8bed7b2dbd}
 En putsbil är ett äldre motorfordon som ägs av en svårt neurotisk man i medelåldern, vilken ägnar all sin lediga tid åt att polera eller tänka på att polera sin bil. Bilarna står oftast still, trots renovering till nyskick. Dock får de en aning motion på sommarhalvåret då de ska visas upp på diverse helylletillställningar. Det närmsta en putsbil kommer hederligt arbete är när ägaren (då hans egna företag går lite svajigt) hyr ut sig själv och sin bil till ordnade tillställnigar, exempelvis bröllop.



 Antalet putsbilsägare är ungefär 10\% av antalet klassikerbilssjälvmord \textsc{(s.~\pageref{8d866f91bbb57d360447acf0fadaec45})}

}

\small{
\textbf{Putshyvel}
\label{82aace730b3087db7cfc8b4ed5d7dae0}
 En något kortare hyvel \textsc{(s.~\pageref{1668e2298ba60f14922e2cca6aa96538})}.

}

\small{
\textbf{Puttefnask}
\label{a36265e28ca6f48da82e7a6e9f75565c}
 En prostituerad dvärg

}

\small{
\textbf{Pysrunka}
\label{c2248f706e60e6958687bcf1c83e3772}
 Att pysrunka är att ejakulera så många gånger att det till slut bara kommer varm luft ur könsorganet. Medelsnittssvensken uppnår detta efter 10 ggr under ett dygn.

}

\small{
\textbf{Pysselbyxa}
\label{9b2a5a638b7d3b6e0d47bd4c082bc76f}
 n är den mer vanliga och konventionella arbetsbyxans syskon och är avsedd att användas i samband med hobbyn pyssel. Brukaren bär denna förtjänstfulla byxa \textsc{(s.~\pageref{bd74f429522c7c1481fbba07187efc6b})} när hen dekorerar ljusstakar med mossa, målar ägg, limmar samman flirtkulor eller tillverkar en ivrig bäver \textsc{(se Ivriga små bävrar s.~\pageref{6d10ab1ba7bd378ba7cc1629ddf2bbde})} av \textit{papier maché}. Periodvis har pysselbyxan också används som mode av medvetna stockholmsprofiler \textsc{(se stockholmsprofil s.~\pageref{daaee4666c210c7a40537c2399f01556})}, men eftersom pysselbyxan inte har något speciellt utseende utan är ett par byxor som användaren valt att avsätta för pysseländamål är det svårt för gemene man att förstå detta fashion-statement.
 HEAD2: Historia
 Redan Leonardo da Vinci ska ha ägt ett par pysselbyxor, vilka han använde då han målade Mona Lisa. Vissa konsthistoriker har spekulerat i att detta kan vara anledningen till Mona Lisas lite hånfulla leende som Da Vinci undermedvetet skulle ha målat in i sin berömda tavla på grund av den osäkerhet som oundvikligen smyger sig på en fullvuxen skäggig man som iklädd pysselstrumpbyxor arbetar inför en okänd kvinna i ett tiotal timmar.

}

\small{
\textbf{Päronhalva}
\label{cc9c1bfa2ec4eaed89ca86a1b63e3a45}
 är en i arbetarklassen omåttligt populär efterrätt \textsc{(s.~\pageref{5fff6e8d7fdf5598341319db050f14c3})} och är något av det finaste man kan bli bjuden på när man är på besök hos någon. Päronhalvan köps alltid i konservburk och serveras med fördel tillsammans med vispgrädde eller After Eight, vilket är den förnämaste formen av päronhalvsservering. Den som bjuds på päronhalva med After Eight står traditionellt i oupplöslig tacksamhetsskuld till värden. När någon har ätit så många päronhalvor att den snart nog inte kan få ner en enda liten päronhalvstugga till ska värden säga \quotetext{ta den sista du - annars står det bara och blir gammalt} och då måste man kliva fram och verkligen göra sitt bästa.

}

\small{
\textbf{På fat}
\label{068c450db48e3bfe2c97cd3ea4c0e083}
 Kaffe, denna svarta och vederkvickande dryck, inmundigas med fördel på fat.

 Detta kan göras på en hel del olika sätt, men enligt den vindpinade, skäggige och LO-anslutna elektrikern Anders \quotetext{Skäggu-Anders} Johansson vid Malå \textsc{(s.~\pageref{41da4620e87888eaaeafcb3004a8d177})} Sågverk \textsc{(s.~\pageref{39a99a78876fd85985cc06fa0baa3c1a})} görs det enligt följande. Kaffet hälls upp på fatet genom att låta det rinna längs koppen på fatet. Sedan lyfts fatet medelst tre fingrar (nybörjare tillåts använda fyra för att undvika söl) och en sockerbit placeras mellan läpparna, framför tänderna. Sedan sörplas kaffet graciöst genom sockerbiten och får således en perfekt sötma. Ljuvligt.

 Om det däremot är helg rekommenderas detta att följas upp med en liten bit choklad, kanske en after-eight, en stadig konjak (gärna Grönstedts tre-stjärniga) och en liten cigarill. Här började kamrat Johansson drömma sig bort från det grådassiga fikarummet till Greklands kritvita stränder och fantastiska bistros, varpå skifteslaget fick påminna om att det var flera månader kvar till semester.

}

\small{
\textbf{På spaning efter den bov som flytt}
\label{b0c7545c68966ced2a217a2e575fb207}
 är Prof. Etienne \textsc{(se Användare: Prof. Etienne s.~\pageref{a9878d2280e5a39becac8f73d113df91})} första försök att skriva en kriminalroman. Liksom alla stora författare tycker han om att koppla av ibland genom att skriva något lite lättsammare. Intrigen är dock på intet sätt lättviktig utan tvärt om en hårdkokt politisk thriller.

 HEAD2: Synopsis
 Göte Swettberg, socialdemokrat och kommunstyrelsens ordförande i Prof. Etiennes hemkommun Ödeshög, hittas mördad i bostadsrättsföreningens gemensamma bastu. Rykten florerar om att en journalist på den lokala dagstidningen \textit{Ödebladet} ska ha komprometterande bilder på Swettberg där han dubbelsovlar \textsc{(se dubbelsovla s.~\pageref{4a58428516d8ba930242406ad6073922})} en fralla på en konferens finansierad med kommunala medel. Den italienske playboysnuten Forp Enneite kopplas in på fallet och han lyckas genast lösa en massa brott och ligga med en massa brudar. Mysteriet med Swettberg verkar dock vara en omöjlig nöt att knäcka. Forp Enneite tänker så det knakar; vad är det som stinker i den här soppan? Varför vill ingen hjälpa honom med det här fallet? Varför tycks inte Göte Swettbergs folkpartistiske \textsc{(se folkpartiet s.~\pageref{b692fa6a23fd557940474dc94909d80f})} enäggstvilling \textsc{(se ägg s.~\pageref{128a5feb8e12d0aa622e0298a8332980})} Hjalmar sörja sin bror? Vad handlar det här fallet om? Handlar det om...... Forp Enneite själv?

 HEAD2: Utmärkelser
 Boken har inte fått några fina priser över huvud taget. Litteraturkritiker har över lag ansett den vara ganska värdelös och dålig.

}

\small{
\textbf{På Spåret}
\label{1151b0ad0009ea36c9a1a95736a37a8e}
 är Sveriges \textsc{(se Sverige s.~\pageref{b1999637949ed135b2ca03f3a38460cc})} bästa TV-program, tätt följt av Kobra.

 När På Spåret uppdaterade sin besättning och tog in Kristian Luuk och Fredrik Lindström, istället för det gamla radarparens radarpar Ingvar Oldsberg och Björn Hellberg, bytte de även jingel. Detta upprörde gastronomen Carl-Jan Granqvist något så jävulskt.

}

\small{
\textbf{Påsförslutare}
\label{fb286f8fda3c574913f25fa4ae87b391}
 En påsförslutare är en lite mackapär som används för att försluta en plastpåse och på så viss förhindra att dess innehåll går förlorat vid transport. Påsförslutaren kan vara gjord av plast och likna en hästsko till en mycket liten ponny. Den kan också bestå av en plastingjuten ståltråd som liksom viras runt påsen. Dessa sorters påsförslutare får man på köpet då man inhandlar fabriksbröd, men det finns också andra, bättre och mer beständiga påsförslutare som kan köpas enkom och som då levereras i storpack om minst ett tjog. Dessa är avsedda för storfräsare \textsc{(s.~\pageref{4db17005692cd83e3e946a1311b81ed0})} som inte ids att varsamt samla på sig påsförslutare, en i taget, i försänkningen bredvid den som är avsedd för teskedar i besticklådan. Som med så mycket annat måste storfräsaren ha omedelbar utdelning och belöning, och värderar således inte sina påsförslutare lika högt som gemene man, som med all rätt prioriterar två Tingsryd 2,8or \textsc{(se Tingsryd 2,8 s.~\pageref{1248f188e41a1f64d2dac526a8b6704c})} att avnjuta frampå fredagskvällen \textsc{(se fredag s.~\pageref{80d41f1e0b14eacb229eea9618632e88})}.

}

\small{
\textbf{Påsk}
\label{f8f0dd13b69a5c8ce56498e750551d3e}
 .]]
 Påsk är en högtid som firas varje år för att uppmärksamma att Jesus \textsc{(s.~\pageref{110d46fcd978c24f306cd7fa23464d73})} fortfarande är död. Den är väldigt populär att fira för man får äta hur många ägg \textsc{(s.~\pageref{128a5feb8e12d0aa622e0298a8332980})} man vill utan att behöva tänka på kolesterolen. Enligt en seglivad myt kommer på långfredagen en Belgisk jättekanin \textsc{(s.~\pageref{4683df87b389ab00a45e8287521a73f6})} hoppande så det dånar i marken och alla barn blir rädda och springer in. Men som alla bildade människor vet så är det påskpingvinen \textsc{(s.~\pageref{833627ff46116bfc06ed8cb876f82ee8})} som kommer. Dock upprätthåller den amerikanska godisindustrin myten om påskharen då man lätt kan sälja gammal playboy-merch till intet ont anande.

}

\small{
\textbf{Påskpingvinen}
\label{833627ff46116bfc06ed8cb876f82ee8}
 kommer varje påsk och lägger godisägg. En del förvirrade själar har gått på den sionist-implenterade myten om påskharen, men allvarligt talat: har någon nånsin sett en hare lägga ägg? Eller äta sill?

}

\small{
\textbf{Påskris}
\label{a9d744074ec3fda67c8e7b52801e5178}
 Ord för den förargelseväckande situation då ens plast- eller papperspåse går sönder och alla saker man har i den faller ut på marken.

}

\small{
\textbf{Påsmygande själv-alienation}
\label{43e35393ec85ece2dd17a96e49256372}
 Tänk dig att du medan skymningen faller varit på väg hem från en avlägsen tätort. Din Saab 900 har kokat och du har tvingats söka husrum, och glatt välkomnats, hos en pensionerad man som tidigare drivit lanthandel med sin nu bortgångna hustru, Hagar. Det bjuds på portvin och ni spelar canasta. Sent på kvällen befinner du dig smygandes på tå, med långa, försikta steg längs gårdstomten, med regelbundna blickar över axeln och famnen full av uppstoppade fåglar som du för ditt liv inte kan förstå att du snott från din givmilde värd, som nu möter dig med ett artigt leende på väg tillbaka från utedasset.

}

\small{
\textbf{Pörr}
\label{5faa435e2f0af7617816f0cade262581}
 Riktigt äckligt snusk. Sånt din mor skulle göra dig arvlös för om hon fick reda på att du tittade på't. Sånt som gör fullvuxna karlar helt likbleka av skam och självförakt. Sånt småkillar hittar under en mossig sten i skogen när de letar patroner. Sånt riktiga vidron stoppar in mellan Fievel i vilda västern och Lilla sjöjungfrun VHS-erna på videoline. Sånt kyrkor startas för att protestera mot. Sånt innebandykillar och militärer kollar på i grupp. Sånt hemslöjdsläraren Doris och gympamagistern Hr. Schulz spelar in på Super-8 tillsammans. Sånt som har titlar man skrattar åt när man hör, men blir mörkrädd och lite ledsen av när man tänker efter.
 \textit{Såntdär pörr}.

}

\small{
\textbf{Pølsenautologi}
\label{765f084d8da82e07e5e9acbbadd0d3f2}
 Läran om att bada i korv.

}

\small{
\textbf{Queequeg}
\label{12be8b7fb94819e82b875a2cb5fb35a2}
 är både harpunisten i Moby Dick och agent Dana Scullys hund i Arkiv X. Den senare dog när den blev uppäten av en krokodil eller en dinosaurie, det hela är något oklart.

}

\small{
\textbf{Quicktänkt}
\label{27b95c92dee2e106157ff07529c6f059}
 Lika delar fantasi och gelenskap.

 Se även: kvicktänkt \textsc{(s.~\pageref{f06ed437f6ad7eeafae17b1a824bf4ee})}

}

\small{
\textbf{RAC}
\label{04a33dd13425a941a5739880a7526687}
 , förkortning för \quotetext{rock against communism}, är en musikstil smalare än tantsång \textsc{(s.~\pageref{25b8200d011a4766f4b3a64a2e17f374})}. Likt black metal karraktäriseras den snarare av en inställning till livet och teman i texterna än av hur musiken låter. Det finns förvisso flera genomgående musikaliska drag, såsom att det låter sämre än både belgisk \textsc{(se Belgien s.~\pageref{f79ffe9e826a19f9f6a446c90e21c4e3})} grishardcore och irländsk stadiumrock, och att sångaren stönar fram texterna på grund av sin övervikt. Men det är mer märkliga sammanträffanden \textsc{(s.~\pageref{f46282d99158f351a81b9deaff157b4e})}. Något annat som kan te sig lite märkligt i sammanhanget är att det mest är politiskt vänsterorianterade människor som lyssnar på musiken. Att på detta sätt trotta \textsc{(s.~\pageref{918b2980ffb5f16acf768fa89f71021b})} en musikstil som man egentligen inte gillar har inte skådats sedan kristna började lira rock \textsc{(se lista på kristna rockband s.~\pageref{10661f12937e040980e5afdb417a3ba7})}.

 Det vanligaste att sjunga om inom RAC är så klart att alla rödingar ska dö. \quotetext{Hellre död än röd} är en vanlig paroll som den överviktiga sångaren i mjukisbyxor \textsc{(se mjukisklädsel s.~\pageref{57a78bf29e9f6fb6a4dba89fc21bc897})} grymtar fram. För att konceptet inte ska bli allt för förutsägbart brukar man stoppa in en eller två låtar som handlar om något annat på varje skiva. Det kan till exempel vara att Loket är pedofil eller att pitbulls är ena riktigt gulliga hundar.

 HEAD2: Framträdande grupper inom RAC
 \begin{itemize}
 \item Pluton Svea
 \item Ace of Base
 \end{itemize}

}

\small{
\textbf{Race}
\label{2e2a7a2e57d1578bf33a559b20c463f6}
 Är ett bra spel som man spelar med bilar på en bana. Det finns ett annat nästan likadant spel som heter Jägersro

}

\small{
\textbf{Radioreklam}
\label{3b1986a57ed78fc91a6b0c8a8edbd9d2}
 är ett fenomen som skapats för att vokalgrupper som The Real Group och Viba Femba ska ha något att göra nu när resten av världen äntligen börjat avfärda acapellamusik som ren skit. I radioreklamens värld har acapellan nämligen fått en fristad eftersom hippiesen \textsc{(se hippies s.~\pageref{4dc77d6258fd18e7c0dd5eece5c7c81c})} i Stockholms innerstad fått för sig att det bästa sättet att skapa uppståndelse hos pöbeln är medelst hjälp av en rejäl skopa skit. På den yttersta dagen ska reklamarna få betala för detta genom att Aleksandr Karelin \textsc{(s.~\pageref{7db555630a4ad78feb3477db9b1ee464})} och Pelle Svensson \textsc{(s.~\pageref{26d88b383fd38f349c7741ca7051904e})} bryter av deras ben från varsitt håll samtidigt. På den yttersta dagen har alla regler nämligen upphört och det är okej att bruka kampsport utanför sin dojo \textsc{(s.~\pageref{9bea15890f18ef35a12767fef5d234b8})}.

}

\small{
\textbf{Rafael}
\label{9135d8523ad3da99d8a4eb83afac13d1}
 är en sammansättning av det latinska slangordet \textit{Rafa} som betyder \quotetext{stjäla} eller \quotetext{låna på obestämd tid} och det likaså latinska ordet \textit{elius} som betyder \quotetext{köksassistent} eller \quotetext{bakmaskin.} Berömda personer som heter Rafael är teenage mutant ninja turteln \textsc{(se Teenage Mutant Ninja Turtles s.~\pageref{fd9ccf7b23fd53b8c3bb91065ab585ee})} Rafael och han som teenage mutant ninja turteln är döpt efter.

}

\small{
\textbf{Raggarbil}
\label{6e08a40fb188c3340f9820e4317f4e31}
 En raggarbil är den högsta drömmen. Med en normal raggarbil förlängs penis till någonstans mellan 130-170\% av ursprunglig längd. Har man däremot satt i en dieselmotor i en raggarbil kan man komma upp i uppemot 250\% förstoring.



 En raggarbil bör ej förväxlas med klassikerbilar och putsbilar \textsc{(s.~\pageref{cb242351ab9f9d6b8a5afe8bed7b2dbd})}. Raggarbilen används gärna och väl och det inte för inte som de kallas för pilsnerbilar.

}

\small{
\textbf{Rainbow Riders}
\label{54b5b4739e6bc150148c5019e1793413}
 är en fruktad MC-klubb hemmavarande \quotetext{på} (som det heter här) stadsdelen Haga \quotetext{\textbf{på}} Umeå. Klubben har två medlemmar och tre prospects - en som har en elcykel, en som har mustasch samt ett treårigt barn som påstår sig ha en traja-chopper. Mellan sig har de fasta medlemmarna fyra motorcyklar och ca 4150 kubik rå vridkraft. Klubben försörjer sig i huvudsak på träslöjd och humanistisk forskning och klubbaktiviteterna cirkulerar kring byggande och meckande, samt korta turer till Tavelsjö där festis och kubb \textsc{(s.~\pageref{de7f6954ec8c6e346b8ba18ae018d334})} vanligtvis inhandlas. Klubben är också stökiga supportrar av Berghem HC \textsc{(s.~\pageref{72c5e1ef562098496277726ca12aa149})}.

 Se också Rainbow riders-stuket \textsc{(s.~\pageref{e31ec5cf9cada9eaab86c175a39aa3e6})} och Fjäriln vingad syns på Haga \textsc{(s.~\pageref{9d56dfc3badeeecc2cfdab9095057706})}.

}

\small{
\textbf{Rainbow riders-stuket}
\label{e31ec5cf9cada9eaab86c175a39aa3e6}
 är ett slags stuk som man kan köra för sitt höga nöjes skull, men framförallt för att imponera på andra och för att passa in. Rainbow rider-stuket karaktäriseras till det yttre av skäggväxt, om man är man, och i beteendemönstret av en viss ovilja till att gå den långa vägen om den korta ändå ligger öppen och finns till för att användas.

 Saker man kan göra för att lägga sig till med Rainbow Riders-stuket: \textsc{(se Rainbow Riders s.~\pageref{54b5b4739e6bc150148c5019e1793413})}


 \begin{itemize}
 \item Dricka en fyrahundrakronors Black Label ur avklippta ölburkar medan folk väntar på en någon annanstans
 \item Äta kubb \textsc{(s.~\pageref{de7f6954ec8c6e346b8ba18ae018d334})}
 \item Göra kortslutning på sin hoj
 \item Föredra alla typer av fordon som lätt går sönder
 \item Ha rosa solhatt
 \item Dricka isbjörnsvin \textsc{(s.~\pageref{2879df543437c30c0a2d0dfaf8649ac7})} på balkongen och skriva dikter
 \item Sluddra om Marx på fyllan
 \item Skita på sig av skräck när man spelar zombie-TVspel
 \item Meka tillsammans med andra medan man oavbrutet talar med sig själv om det man gör
 \item Ta choppern till Willys \quotetext{för att handla fetaost}
 \end{itemize}

}

\small{
\textbf{Raka rör}
\label{4c0054aebdec556983a95f4350ca0ab6}
 är avgasrör utan katalysator eller annat tjafs som sitter i vägen och gör motstånd mot dina försök att göra dig fri från vardagens betungande fängsel. Detta gör att ditt fordon får lite fler hästkrafter och är naturligtvis just därför olagligt, enligt principen att allt som äger \textsc{(se Ägmästare s.~\pageref{8324518500d7e7ccd22ae364887d4476})} är olagligt. Därför har du ett \quotetext{besiktningsrör} som du hänger på när det ödesdigra datumet infaller och som du byter ut 22 minuter efter att du fått klartecken från besiktningsman.
 HEAD2: Geopolitik och demografi
 Ovanstående berör endast människor som inte har de ekonomiska medel som tillåter en mer sofistikerad tillvaro (se storfräsare) \textsc{(se storfräsare s.~\pageref{4db17005692cd83e3e946a1311b81ed0})} och som inte valt att bo i stadsmiljö. De som valt att bo i stadsmiljö eftersom utbudet där i alla former vida överträffar landsbygdens ofta lite knappa tillvaro associerar ofta ordet \textit{rör} med alkohol, åker kommunalt eller cyklar räsercykel. Och framförallt har de annat att göra än att klicka omkring på nissepedia \textsc{(s.~\pageref{62400dadecd90cb5cd39062abe5a3e4a})}.

}

\small{
\textbf{Rakkniv}
\label{e4d480f741e2857104a3553126a8bf44}
 Att raka sig med kniv är det manligaste som finns. Många påstår dessutom att rakningen är den mest hudnära man kan åstadkomma. De flesta av oss som rakar oss med kniv tycker nog mest det är jävligt tufft och hade säkert blivit lenare med en begagnad engångs-bic (orange modell).

 På www.straightrazorplace.com finns det många trevliga män. de utbyter tips och trix för att få den vassaste kniven och den lenaste huden. Emellanåt hettar det till mellan medlemmarna, vilket kan leda till verbala rallarsvingar, men oftast hålls en mycket god och kamratlig ton. Dessutom är \quotetext{ställa upp}-faktorn hög på forumet. Man ska dock vara på sin vakt. Om någon med ordet \quotetext{vendor} under sitt avatarnamn rackar ner på eller höjer en produkt till skyarna kan det mycket väl vara så att de handlar i egenintresse. Till exempel så behandlades ett inlägg om den slovakiska naturstenen \quotetext{rozsutec} med en nästintill iskall tystnad, ett förfarande som påminde om biblioteksbranschens medvetna ignorerande av \quotetext{äldreomsorgen i övre kågedalen}.

}

\small{
\textbf{Rakt}
\label{92be9c2f6a2fa0abd7fbcbebc76531ea}
 Utan omvägar eller böjar. Kan även beskriva hur ett trumkomp eller en basgång ska spelas. Mark Levengood hade rakt hår innan han blev rakad.

}

\small{
\textbf{Randall finefield}
\label{42cdb7e2ced62d34c4f447c47eae332a}


}

\small{
\textbf{Randall Finefield}
\label{42cdb7e2ced62d34c4f447c47eae332a}
 är Sveriges statsminister Fredrik Reinfeldts \textsc{(se Fredrik Reinfeldt s.~\pageref{0c16c01849fc86b54e9e0e815490f747})} täcknamn när han vill röra sig mer fritt. Iförd en sportig blazer, en jordgubbsblond tupé och ett par kolsvarta Ray bans reser Randall Finefield runt jorden för att förlusta sig. Han introducerar sig som miljardär, filantrop och älskare av kvinnor. Från kamelpolo i Dubai till exklusiva konstauktioner i New York; Randall Finefield är överallt, hela tiden. Eftersom det egentligen kan vara ett rätt uppslukande jobb att vara stadsminister har Fredrik Reinfeldt sett till att klona sig själv, så att han kan tillbringa mer tid i rollen som sitt supersexuella jetsetande alter-ego, Randall Finefield. Samtidigt sitter hans klon hemma och snyter den egentlige Reinfeldts barn och slaskar upp en alkoläskfylla \textsc{(s.~\pageref{8234165b965f2b1378f10acd340dc126})} med Anders Borgs alter-ego, Robert Wells.

 Kända personer som ingått i Randall Finefields entourage:
 - Professor Etienne \textsc{(se Användare: Prof. Etienne s.~\pageref{a9878d2280e5a39becac8f73d113df91})}
 - Patrik Sjöberg
 - Joe Labero
 - Prins Albert av Monaco
 - Boutros Boutros Ghali
 - Kylie Minogue
 - Henry Kissinger
 - Hugh Hefner
 - Kung Carl XVI Gustav av Sverige
 - Bashar al-Assad
 - Dick Cheney
 - Chippen
 - Kim Kardashian
 - Omar al-Bashir
 - Lasse Åberg
 - Lars Vilks
 - Anthony Giddens

}

\small{
\textbf{Randolfo}
\label{b8f0a32f840f1db27a2c12e17b640fb2}
 är en krullhårig man (superskurk) på 40+ med flaskbottenglasögon som driver en kombinerad \textsc{(se kombinationsaffär s.~\pageref{54328b839527f9917e5d057845b4fc5c})} akvariefisk- och skivaffär. Han har ungefär tio akvarier med guldfiskar och fyra backar hårdrocksvinyler. Det är allt han har. Affären ligger i en betongkällare med små fönster och knarrande dörr, där jobbar Randolfo åtta timmar om dagen. Tack vara den dåliga ventilationen i lokalen luktar Randolfo alltid fisk men det gör inte så mycket för det har gett honom det råbarkade (tycker han) smeknamnet \quotetext{aqua man}. Kunderna är ganska få så det är nästan aldrig någon som klagar. Han har en låtsaskompis som han fantiserar om ibland. Kompisen har 40.000 skivor och han vill att Randolfo ska sälja ALLA i sin affär. När kunder kommer in i affären händer det att Randolfo berättar om sin kompis men alla tror inte på att kompisen finns. Då tar Randolfo fram en skiva med Tygers of Pan Tang och säger att: \quotetext{jo, den här till exempel. Den har jag fått av honom}.

 När arbetsdagen är slut kryper Randolfo in bakom ett skynke i affären, där har han sin bostad som består av en madrass+sovsäck, en klockradio utan antenn och en ficklampa. Där kurar Randolfo ihop sig och drömmer om bättre tider där kunderna köar till hans affär som har flyttat upp en våning och han har kompisar som kallar honom häftig utan att behöva lukta fisk. När han har svårt att sova händer det att han går ut till akvarierna och jagar en liten fisk med sin håv. Han tar upp fisken som tror att sina dagar är räknade men sen släpper han ner den igen. Andra nätter ligger han bara i sin sovsäck och blinkar med ficklampan och tänker på om det är han eller lokalen som luktar fisk.

}

\small{
\textbf{Ranta Runtiringen}
\label{44eb270a4a94167c17971b9d1bb80843}
 Ranta Jarmo Runtiringen (1928-1958) är Finlands \textsc{(se Finland s.~\pageref{631d44eaa1254ff71a1e11ba021d1266})} sämsta boxare någonsin. Han dog på sin 30-årsdag i en alkohol-, hundspann-, jojk-, kniv- och lädervästrelaterad olycka. I hemstaden Oulu i norra Finland finns än idag ett monument över stadens största kändis kvar. Utanför det magnifika paradisbadet står en boxarhandske greppandes en morakniv, uthuggen i massiv granit, med inskriptionen \textit{\quotetext{Yksi, Kaksi, Kuolema}} (Ett, Två, Död). Inskriptionen är ett citat av Runtiringen, från innan han 1935 gick in i ringen för att möta sin främste konkurrent om bottenplaceringen, Jagger \quotetext{Ballista} Keffiringen.

}

\small{
\textbf{Raplex}
\label{1006339386e0c89262cd24fc0dc6e113}
 är ett ord som beskriver att tiden går undan och verkligheten är knepig i det senmoderna samhället \textsc{(s.~\pageref{1a4c3b1112bd2b510a8c47eff69397b8})}. Ordet är sammanslaget av de två andra engelska orden \quotetext{rapid} och \quotetext{complex}. Begreppet myntades av den amerikanske sociologen Peter Berger som också myntade uttrycket \quotetext{butherface}.

 Om man tycker att det är jobbigt att hantera samhällets komplexa natur och höga tempo kan man vända sig till Harry Sarve, mannen bakom [http://www.raplex.se www.raplex.se].

}

\small{
\textbf{Rasmus Klump}
\label{eac88a6def9b9f47888e7e3b62719cf1}
 är det danska originalnamnet på den tecknade brunbjörn i toppluva och prickiga hängselbyxor svenskar känner som Rasmus Nalle. Med tanke på seriernas bristfälliga manus får man för en gångs skull faktiskt ge danskarna rätt – Klump passar mycket bättre.
 Helt uppenbart är Rasmus Klump egentligen bara en taskig rip off av den sedelärande Bamse, men utan den senares känsla för solidaritet och civil olydnad. Som sidekicks har Rasmus Klump den nervösa pingvinen Pingo (Lille skutt), pelikanen \textsc{(se pelikan s.~\pageref{ecf1b944439a171dfe1163001feeed19})} Pelle som alltid har något användbart i sin näbbpung (Skalman \textsc{(se Diskussion:Cashewn\%C3\%B6t s.~\pageref{1e5a3d8941679600dad3f8675264b44c})} och den ständigt piprökande sälen Sälle (Burres \textsc{(se Burre s.~\pageref{6e54c504971bbe1f8d46e006550af1ca})} pappa).

 Oklart varför men hela 37 album om Rasmus Klump finns faktiskt publicerade. Det är i och för sig lätt att dra till med en hög siffra eftersom den som får för sig att kontrolläsa ändå skulle tröttna efter 3-4 album.
 Varje år delas \textit{Rasmus Klump-priset \textsc{(se danska hedersbetygelser s.~\pageref{799941a6e98a1446da72ff5483c6503d})}} ut i Danmark \textsc{(s.~\pageref{5331d7fd27772396f412a5b6d19bad44})} till en dansk som ”uppvisat de egenskaper som serien förespråkar”. Vad fan nu det är. Pristagare är i alla fall bland andra kronprins Fredrik och Michael Laudrup.

}

\small{
\textbf{Rataxes Mörtberg}
\label{ac781684510a71972370f130779b7f18}
 , född 22 december 1955 i Klorköping i Härjebotten, är en svensk äventyrare och simlärare. Hans största bedrift hittills var att under det glada 70-talet lyckas få krita på samtliga av Sveriges korvmojar.

 Rataxes Mörtberg har varit med om det mesta. Han föddes hemma i biljardrummet på familjens torp och redan fem år gammal besteg han den ostaplade vedhögen på gården. I skolan utmärkte han sig för sin till synes outtröttliga förmåga att lösa matematiska tal och alltid ange svaren i antal silltunnor (för frågor rörande omkrets så väl som volym). Efter avslutad karriär som äventyrare har Rataxes valt att ägna den mesta av sin tid åt familjen, som enligt honom är det största äventyret man kan ge sig ut på.

 HEAD2:  Det första äventyret
 Rataxes första äventyrsresa utgick från föräldrahemmet och tog sikte på grannarnas äppelträd. Genom en noga uttänkt plan som involverade en stege, en konsumkasse och hans lådbil lyckades Rataxes bli den förste svensk att framgångsrikt palla ett helt tjog äpplen från Johanssons baksida. Vissa menade att flera av äpplena egentligen bara var kart och var så sura att dom inte gick att äta men dom flesta barnen på gatan var jätteimponerade.

 HEAD2:  Riksberömmelse och erkännande
 Bragden med de pallade äpplena spred sig fort över skolgården på Klorköpings mellanstadieskola och Rataxes blev snabbt något av en lokalkändis. Det stora genombrottet kom dock när han hittade en öppnad kondom och skrev till KP för att fråga vad det var för något. Över en natt visste hela Sverige vem Rataxes var och flera klassdiscoarrangörer började genast höra av sig för att boka ett framträdande med den unge celebriteten.


 HEAD2:  \quotetext{Affären Britta}
 \quotetext{Affären Britta} var namnet massmedia gav den expedition som allmänt skulle komma att betraktas som Rataxes största bragd. Året var 1978 och efter att ha blivit nekad av Britta på grillen till att krita en Pucko till mosmenyn lovade Rataxes att inte återvända förens han fått krita på alla Sveriges andra korvmojar. Den lokale storfräsaren \textsc{(se storfräsare s.~\pageref{4db17005692cd83e3e946a1311b81ed0})} Ove Äggberg fick via anslagstavlan på torget nys om rekordförsöket och ställde upp som huvudsponsor. Med dennes hjälp kunde Rataxes köpa busskort och skaffa nödvändiga kontakter för att få tillgång till lokala telefonkatalogers gula sidor med förteckningar över korvgrillar. De första månaderna förlöpte utan större förhinder men problemen började hopa sig när resan kom till Skåne och i Klippan tog han enligt kritikerna egentligen bara en vanlig springnota av rädsla för att blandas in i konflikten mellan punkare och raggare. Väl hemma igen firades dock bragden stort och lokaltrafiken var gratis inom hela kommunen över veckoslutet.


 HEAD2:  Senare liv
 På ålderns höst trappade Rataxes ner på sina offentliga framträdanden. Årtionden i stormens öga hade slitit på honom mer än många ville tro och han gör numera bara sporadiska framträdanden på byamarknader för att demonstrera dammsugarmunstycken. I ett halvhjärtat försök att skaka liv i karriären skrev Rataxes en insändare och klagade på att Sandviks sågblad var så jävla slöa. Det uppdagades dock att en av hans ungar försökt kapa en plåt med sågen och att felet troligtvis inte låg hos tillverkaren. Gjord till åtlöje inför öppen ridå drog sig Rataxes tillbaka och undanbad all uppvaktning på sin 60-års dag. I en längre intervju med tidningen Kattliv förklarade han sin flykt från offentligheten med att han var less som fan på att alla var på honom och att dieselskatten borde sänkas.

}

\small{
\textbf{Raw justice}
\label{abf3e3f234373f67929405b3754cb097}
 var ett svenskt punkband från Fagersta. Själva skulle de nog subkategorisera sig som råpunk, men de flesta lyssnare brukar anse att det snarare rör sig om jävligt dålig punk. Från början kallades bandet \textbf{Cruel Maniax}, och under denna etikett släppte man en splitsjua \textsc{(se sjua s.~\pageref{e7bf63fa6d0d29bd89c23f833b979a15})} med eskilstunaorkestern \textbf{No Security} och var med på samlingsskivan \textit{The vikings are coming}. Sen bytte man namn, oklart varför, och gick tillbaka till att släppa kassettdemon. I ett nummer av fanzinet \textit{Chaos is king} utsågs Raw Justice till världens sämsta band. I ett samtal mellan en nissepediamedarbetare och en medlem i bandet, deklarerade dock bandmedlemen (sittandes pissfull på en träbänk i Virsbo med könet hängandes ut genom gylfen) att Raw Justice var världens bästa band.


 Källa: Prof. Etienne \textsc{(se Användare: Prof. Etienne s.~\pageref{a9878d2280e5a39becac8f73d113df91})} - \textit{Självbiografi, del 2 - De förlösande thinneråren}. Timbro förlag, Finland \textsc{(s.~\pageref{631d44eaa1254ff71a1e11ba021d1266})} 2002.

}

\small{
\textbf{Ray Jones IV}
\label{4e526c431120b692a2dc2fc9aa8612be}
 (född 7 december 1967 som \textit{Roy Ray Jones}) är en svenska actionskådis som spelat in milstolpar såsom \textit{Sökarna}, \textit{30:e november} och \textit{Sökarna 2 - Rebelz}. Dessutom har Ray Jones IV gjort flera bejublade inhopp i TV-serien \textit{Rederiet}. Att Ray Jones IV har romerska siffror i sitt namn gör att man lätt misstänker att han inte bara är kung på bioduken utan också i verkliga livet. Nissepedias medarbetare har ännu inte hittat några konkreta bevis för detta men sökandet fortsätter. Helt klart är man speciell om man lyckas byta ut \textit{ Roy}  i förnamn mot \textit{ IV}  i efternamn.

}

\small{
\textbf{realister}
\label{3ab43c7f3424fc9915776529066a2840}
 \quotetext{Realister} är svenska män som, ofta utan byxor \textsc{(se sans pants s.~\pageref{e690d08a3200d783d98b198f0354bc85})}, kommenterar nyheter i dags-, kvälls- så väl som lokaltidningar. Där uttrycker de sin oro inför mångkultur och genusdebatt, vänsterpolitik och bensinpriser. \quotetext{Realisten} inser att progressiv debatt, en modern invandrings- och asylpolitik och andra fenomen som karaktäriserar eller borde karaktärisera samtidsklimatet är mycket skadliga och hotar att leda till samhällets kollaps. Med \quotetext{samhällets kollaps} menar realisten skapandet av ett jämlikt samhälle och den desperata situation där han och andra män utan byxor inte längre utgör en speciallt priviligerad grupp inom väljarkåren och får lika mycket eller litet att säga till om som alla andra människor i det demokratiska Sverige \textsc{(s.~\pageref{b1999637949ed135b2ca03f3a38460cc})}. Realisten är därför en varm tillskyndare av etnisk rensning, anti-intellektuell häxjakt och högerextremism, som realisten tycker är realistiska politiska krafttag i kampen mot invandrare, kvinnor, homosexuella, vänsterintellektuella (sk. \quotetext{kulturmarxister}), journalister, damfotboll, muslimer, konstnärer och författare (dock inte Lars Wilks), miljörörelsen, fredsrörelsen och pride-paradens deltagare.

}

\small{
\textbf{Red Bull shanti roney}
\label{6fa12a43a71775c8f0947dbf8e0217b9}
 Korv. Innan han dog har man rivit bort hans skinn och haft det i korven. Det är Red Bull i den också.

}

\small{
\textbf{Referenshumor}
\label{71696a7ff27047c7ac019d10fa72f452}
 Att skriva \quotetext{Källa: internet}.

}

\small{
\textbf{Refused}
\label{723634aa8cde73188d4661bb3fe81ce4}
 HEAD3: Refused
 Refused är ett band från Umeå. I början av 90-talet, då de började spela, lät deras musik som det senaste i HC-väg från USA. Hårt, långsamt och med texter som handlade om veganism, drogfrihet och radikal vänsterpolitik. De blev väldigt poppis i sin hemstad Umeå och resten av Sverige. Och i Tyskland, så klart. När de spelade i USA gick det ok, men i en intervju från den tiden klagade sångaren i Madball på att han tyckte killarna i Refused var nördar. Och det var de nog också. Men det är å andra sidan de flesta i jämförelse med Freddie Madball.

 HEAD3: Nya skivan
 Efter ett tag blev Refused less på att spela samma gamla tuggtugg-riff och bestämde sig för att inkorporera frijazz, dansmusik (elektronisk) och klassisk musik i sin hardcore. Det gick självklart helt åt helvete, och det var inte så många som gillade skivan som lät så. När den kom ut.

 HEAD3: Splittring och eftermäle
 Efter det splittrades bandet och alla började göra sin egen grej. Sångaren blev sexikon, men har sen dess bara spelat i undermåliga band som mest får uppmärksamhet för att han en gång spelade i Refused. De andra i bandet gjorde först en asbra experimentell skiva under namnet TEXT, men splittrades snart efter släppet för att det var alldeles för weird. Sen började trummisen röka asmycket weed och spela i ett soloprojekt som till en början också var asbra. Seriöst, kolla in första och andra plattan (kanske till och med tredje). Mysigt, vemodigt, knepigt och poppigt och rockigt all in one. Men trummisen är alldeles för dålig på att exponera sig själv i media på ett attraktivt sätt, så han är och förblir någon sorts märklig haschprofet (som man inte vet om han är väldigt smart eller helt bakom flötet) med gjutna pop-sensibilities. De andra två i bandet tog lite time out. Den ena utbildade sig till operadirigent, den andra satte sig enligt ryktet i en källare och började studsa en boll mot en vägg. Sen satt han så i ca 20 år.

 HEAD3: De hårda åren
 När Refused la ner skickade de ut en kommuniké där de sa att de absolut aldrig mer skulle spela igen. Föga anade de då att deras sista skiva efter ett tag skulle bli en enorm succé. Metalfans världen över älskade de progressiva elementen i musiken och band som Korn och Limp Bizkit har båda uppgett Refused som stora influenser. Sedan dess har erbjudandena om återförening haglat över bandet. Summorna som bandet utlovats har bara ökat och ökat. Till en början kändes det nog pinsamt för bandet. Vem fan vill influera Fred Durst, liksom? Och man ville ju inte böja sig och sälja ut heller. Några hade musikprojekt på gång och tänkte nog:
 - Det här ska nog lyfta snart, visst kan jag livnära mig på min musik utan att luta mig på gamla meriter än! Jag kan fortfarande göra något innovativt och spännande igen, och jag är ju för fan en människa med integritet.

 Men så blev det aldrig, och det var de inte. Operaproducenten hade antagligen enorma CSN-skulder att betala av och han som satt i källaren lurades till sist upp därifrån genom att en metrev med en tusenlapp på sänktes ner i hans håla.

 HEAD3: Duduuuh - We're back!
 Så kom återföreningen. Men hur skulle man motivera den? De hade ju sagt att de aldrig skulle. Att erkänna att det var för att de skulle bli miljonärer skulle antagligen inte falla i god jord hos bandets fanbas. De gnuggade sitt hår och funderade utav bara attan! Sen kom snilleblixten - det var ju för fansen! De hade ju aldrig fått höra sista skivan live, eftersom de inte var födda när den gavs ut. Klart ungarna ska gå på konsert! Detta klassiska kryphål, som använts om och om igen av trötta rockers på dekis, fungerade otroligt nog ganska bra. Pressen satt med hakan i golvet och dräglade. \quotetext{Äntligen kan jag bli femton igen!} skrek de unisont. \quotetext{Radikala budskap, och jag får dricka bärs på debaser när jag kikar på't!}. Succén var ett faktum och Refused, plus en inhyrd basist som inte får vara med på affischer eller så mycket pengar och misstänks vara en robot, började flyga världen runt. Först skulle det bara var några spelningar. Men sen ville fler och fler betala grova mängder stålars för att Refused skulle lira på deras festival. Så det blev några fler datum. Vem skulle banga på att åka till typ Australien, spela inför tusentals människor och efter giget få en tjock jävla sedelbunt klämd i näven på en, liksom? Och antagligen få klappa en koala eller två också. Nä, just det.

 HEAD3: Slutet?
 Vad ska ske efter detta? Sångaren håller enligt rykten på med att skriva sin självbiografi som kanske kommer att heta \quotetext{Jag var bara en vanlig arbetarkille från Vännäs som ville spela lite rock, typ. Det är ju great.} Men inget uppges vara spikat än. Sannolikt kommer han använda alla Refusedpengar till att köpa obskyra skivor och sen börja hetsa om en återförening igen om femton år. Då finns det ju nya barn som aldrig heller upplevt sista skivan live. Trummisen käkar mest fasan, sätter upp dansföreställningar, dricker vin och läser krångliga franska böcker. Och det får man göra. Om han gör egen musik igen skulle väl det vara kul, men det är ok om han inte gör det också. Operadirigenten kommer kanske sätta upp sista skivan som episk opera, typ som Metallica gjorde. Fast med opera. Robotbasisten kommer antagligen plockas isär och hans delar kommer förhoppningsvis sitta i en Volvo 740 \textsc{(s.~\pageref{e262951543da05bac43c7b87235a115c})} tills det är dags att dra ut och robotlira igen.

 Och han med bollen då? Efter att ha suttit i sin källare ett tag, kom han upp igen i samband med att Refused fick ett pris av den borgerliga regeringen. Sångaren och trummisen åkte för att ta emot priset. I samband med det sa de något om att borgarna suger och att socialdemokraterna äger (radikalt!), vilket gjorde källargitarristen skitsur; då han tyckte det sög att ställa upp på ett mörkblått reklamjippo, även om det innebar att man fick säga en dryg grej eller två. I slutändan fick gitarristen rätt, eftersom det enda man minns från prisutdelningen är att Refused fick ett diplom och lite blomster av handelsminister Ewa Björling, inte att de sa något negativt om regeringen i samband med det.

}

\small{
\textbf{Regn}
\label{03456beeae643b4c33b17500a17d1d1e}
 är ett slags väder som man känner igen på att det märkligt nog kommer vatten från himlen. Hur detta kan komma sig har länge förbryllat vetenskapen, men på senare tid står det mer och mer klart att det är molnen \textsc{(se moln s.~\pageref{9da1014bea9aa67f9cae12e619d34aae})} som gör det.

}

\small{
\textbf{Rektumkrypare}
\label{cb229e8a6fd6c84c7c530cb6717386f4}
 n är en insekt som finnes endast i mellersta norrland, närmare bestämt Hjoggsjö med omnejd. Rektumkryparen har fått sitt namn av sitt levnadssätt. Den kryper in i rektum på djur eller människor när de badar och lägger sina ägg.

}

\small{
\textbf{Renläpp}
\label{8e08e4a1d5dfa892602312864a76ad87}
 Enligt Johan dahlberg \textsc{(s.~\pageref{11023feb5a10d8d6fc311c732ca7b077})} världens godaste maträtt. Enligt friska människor ett skällsord.

}

\small{
\textbf{Rensadel}
\label{bdf7a0afd203d6384c36cf3669ad7b95}
 En rensadel är en sadel framtagen speciellt för renar. Sadeln är ofta gjord av härdat läder med ett lite mjukare material längst upp för komfort. Den placeras på renens rygg och ryttaren tar sedan plats i sadeln och greppar tag om renens horn eller kring dess hals och så bär det av.

}

\small{
\textbf{Repet}
\label{0714379932aa997070168553fe416a96}
 är det tredje scoutmärket och föregås av Nyingen och Scouten. Märket sys fast väl synligt på scoutskjortan. För att få repet måste man kunna:

 \begin{itemize}
 \item Knopar; Råband, Skotstek, Överhandsknop, Pålstek
 \item Raksurrning, vinkelsurrning, trefot
 \item Miniorknoparna samt timmerstek,tältlineknop, dubbelt halvslag
 \end{itemize}

 För att få utmärkelsen är det inte nödvändigt att tillverka ett vapen av en socka med en tvål i och spöa upp en tjock unge med det. Att göra så kan till och med försvåra processen med att få utmärkelsen avsevärt eftersom det anses gå emot scoutrörelsens värderingar.

}

\small{
\textbf{Revyscenen}
\label{c49ee096fcc5a9e626c3e1da73205d6d}
 Som punkscenen, fast folk jobbar oftast vid sidan av sitt engagemang. Där punkscenen har punkscenshumor \textsc{(s.~\pageref{1a33c499f83c6b860ac9712507a990a1})} har revyscenen självklart revyscenshumor.
 Revyscenen är 75 cm hög.

}

\small{
\textbf{Richard Dybeck}
\label{05b14afb019c5d198ca235fd03a178d0}
 Född Mutapuro (Rickardo), eventuellt Mutapiki. Stamm' aus Finnland, echt Schwede.

}

\small{
\textbf{Rick ta Life on a horse}
\label{ab69c5daca1a48059444dc96de28537e}
 är en hemsida helt tillägnad sammanhang där Rick ta Life sitter på hästar. Hittills har man dokumenterat ett tillfälle. Rick ta Life är sångare i hardcoregruppen 25 ta Life från New York, New York, USA. Ricks faktiska intresse för hästar är omstritt men faktum kvarstår att han åtminstone en gång suttit på en. Har du tips om Rick ta Life i hästsammanhang bör du ta kontakt med sidans ansvarige.


 Till hemsidan: http://www.ricktalifeonahorse.com

}

\small{
\textbf{Riddare}
\label{4e6de79c29df2feb36a9f265b6b662e1}
 är ett yrke som funnits i hundratals år och går ut på att försvara heder och ära. Ofta verkar det mest vara sin egen heder riddaren försvarar, så det är lite oklart var finansieringen kommer ifrån. Börsanalytiker har på grund av detta länge fruktat en riddarbubbla, men då höjer riddaren bara sin mäktiga lans och alla blir genast lugna. Vanliga arbetsredskap för att klara av ridderiet är, förutom lans, sköld och häst, vilka riddaren tar med sig till ett tornerspel. På tornerspelet använder riddaren all sin styrka och list till att besegra andra riddare, vilket ger den omåttlig ära och berömelse. Tack vare sin mystiska aura har riddare gett upphov till det engelska ordet för gåta, riddle, mejeriprodukten riddargrädde, och maträtten fattiga riddare \textsc{(s.~\pageref{c53ee9c63bb93f36773e3c72dcccb306})}.

}

\small{
\textbf{Rikemanssidan}
\label{9cf20261c419c136cb95565585481f65}
 Som de flesta spörsmål har även livsmedlet bröd två sidor. En del brödsorter som bakas på häll, exempelvis tunnbröd och polarkaka, har en bullig sida och en slät sida. Dessa sidor är som de mesta annat här i världen präglat av klassamhället och utgörs av en fattigmanssida och en rikemanssida. Rikemanssidan är den bulliga sidan av mackan och är det för att den rymmer helt kopiösa mängder smör. Då fattiga bönder och industriarbetare vänder på sitt tunnbröd för att inte slösa på smöret så öser storfräsaren \textsc{(se storfräsare s.~\pageref{4db17005692cd83e3e946a1311b81ed0})} på som om det inte fanns någon morgondag. Vill du använda rikemanssidan men saknar resurserna kan du sitta och pilla ut smör ur hålen men det har du fanimig inte tid med!

}

\small{
\textbf{Rikskuponger}
\label{2e3acaa8f24b5db948a51e402a6f2349}
 är en smart uppfinning som används som valuta \textsc{(s.~\pageref{cf1e2a0af4955aa7539b6e12e9d282e6})} men som man bara kan köpa ett fåtal sorters varor med, oftast pizza \textsc{(s.~\pageref{7cf2db5ec261a0fa27a502d3196a6f60})}. Rikskupongerna får man i ett behändigt litet häfte som man kan ha i bakfickan \textsc{(se bakficka s.~\pageref{d259b5ebe8541b74129f0c78a82335b7})} och bara slänga fram när man ska betala för sin Hawaii-pizza \textsc{(s.~\pageref{742e4954c36e42931521b0a417511c7c})} eller vad det nu kan vara. Häftet innehåller kuponger med valörerna 2, 5, 10, 20, 40, 50, 60, 70. Den som har rikskuponger som intresse och hobby kan ta för vana att besöka [http://www.ticketrikskuponger.se/  Rikskupongers hemsida] med jämna mellanrum för att där lära sig mer om denna populära kupong.

}

\small{
\textbf{Riksregalier}
\label{6b724c12e247a754571012ad4661ae6b}
 är föremål som bärs av ett lands statschef som en symbol för makt. Generellt brukar riksregalier för det mesta ligga inlåsta i skattkammare och bara åka fram vid viktiga ceremonier. Ordet kommer från latinets \textit{regalis}, kunglig.

 De mest kända riksregalierna är de brittiska där man lyxat på med att trycka in världens största slipade diamant i spiran och världens näst största slipade diamant i kronan. Sverige har inte lika pråliga riksregalier, men väl två ganska tuffa i form av Gustav Vasas gamla svärd. Danmark \textsc{(s.~\pageref{5331d7fd27772396f412a5b6d19bad44})} går som vanligt mot strömmen och har bland annat gett riksregalestatus åt en gammal slokhatt som tillhört Kim Larsen, en oöppnad Tuborg \textsc{(s.~\pageref{49bb0f04b9993881c9d9c5b115cc35f0})} Grøn med feltryckt etikett och en slarvigt sydd dannebroge \textsc{(se dannebrogen s.~\pageref{d8c97891c74597fa443ed507c4191fe0})} som Christian Tyrann gjorde i slöjden.

}

\small{
\textbf{Rikssamtal}
\label{b7af024f1adeddd237d706ebece98b89}
 Att ringa någon som bor så långt bort att det tar minst två växeltelefonister att koppla samtalet. Desto längre bort man vill ringa, desto kortare bör man fatta sig. Dels eftersom det blir dyrare och dyrare för varje involverad station, och dels eftersom det kan finnas andra som också behöver ringa som man inte vill störa genom att blockera ledningarna. Den sparsamme använder sig istället av det stolta postverket \textsc{(se posten s.~\pageref{cd13d688571681e426231485b732444b})} som tar lika mycket betalt oavsett vart i riket du än ämnar skicka ditt vykort. Ett vykort erbjuder också möjligheten att inkludera ett vackert lokalmotiv i fyrfärg alldeles gratis; en egenskap som inte ens telefaxen \textsc{(se fax s.~\pageref{236c3b7f761221f195b428aca2f06c4b})} kan mäta sig med.

 Det enda som är dyrare att ringa än rikssamtal är utomlands eftersom det även involverar en gränspolis, och till mobiltelefon eftersom det involverar strålskyddsinstitutet.

}

\small{
\textbf{Rimbo}
\label{d93bb016d5aea9c6faa22c5b544a4fdb}
 är en tätort i Norrtälje kommun och har cirka 4600 invånader, give or take.  Här finns ett äventyrsbad \textsc{(s.~\pageref{8e36481b72c8061bb9ff74c1df3b0b66})} som dessvärre är nerlagt efter att ha tömts på pengar av yuppien som köpte det, ett vattentorn, Britts mode \textsc{(s.~\pageref{4222116edbe095681ea4a4513b21bd44})}, två bensinmackar, kanske sju snabbmatsställen, nynazister så att det räcker och blir över samt en idrottsanläggning. På skylten vid den östra infarten till tätorten har någon skämtsamt sprayat ett a av i:et i \quotetext{Rimbo} så att det står \quotetext{Rambo}.
 \textbf{Lista över vad Rimboborna gör en vanlig onsdagkväll:}
 1. Spelar bilbingo
 2. Hyr Congo \textsc{(s.~\pageref{443fd8c93d17446bad49472af0e22dc3})} på VHS.
 3. Går ut och röker
 4. Går ut med hunden
 5. 2, 3 och 4 samtidigt
 6. 1 och 3 samtidigt.
 7. Spelar i Oi!-band.

}

\small{
\textbf{Rimbo-kvinnor}
\label{fee636b63bc1a08e245dc0aaf820b974}
 är kvinnor som av en eller annan anledning har kommit att bosätta sig i Rimbo \textsc{(s.~\pageref{d93bb016d5aea9c6faa22c5b544a4fdb})} i Norrtälje \textsc{(s.~\pageref{7527f7dad9445013a559dc7e2a91f3b3})} kommun. Kanske arbetar de på Rimbotvätten, platspåsefabriken därstädes eller så är det kanske den riksomtalade rimbobullen som för dem till detta hörn av vårt avlånga land. Möjligen kan den höga frekvensen av nynazister ha lockat eller så är Rimbo-kvinnan rent av född i byn och har bara inte fått tummen ur och flyttat.

}

\small{
\textbf{Ring P1}
\label{4db71b60775c55748348514df36a155d}
 är kanske det bästa Sveriges \textsc{(se Sverige s.~\pageref{b1999637949ed135b2ca03f3a38460cc})} Radio har att erbjuda. Där stöts blött med smått, stort med torrt. Många programledare är sympatiska, och försöker verkligen förstå och diskutera med den som ringer Ring P1. Det finns också de programledare som är osympatiska. Som Täppas Fågelberg. De som sitter i slussen har ett hästjobb varje morgon. De ska sortera bort de som är dårar, och sortera in de som vill säga något annat än det vanliga dravlet om invandrare, etc. Ring P1 är en orgie i sekundärskam, glädje, illvilja, hatkärlek och kanske framförallt skadeglädje.

 HEAD2: Exempel
 Så här kan det gå till när någon ringer till Ring P1.

 Pia: Hej, vad har du på hjärtat?
 Inringare: Hej Pia jag heter X X och bor i Y, och jag skulle vilja prata om problemen med yttrandefriheten i det här landet.
 Pia: Vad tänkte du då?
 Inringare: Jag tänker främst på att polisen inte ingriper när det är oroligheter på SD:s torgmöten.
 Pia: Men gör de verkligen inte det?
 Inringare: Nä de gör de inte, och jag vill bara klargöra att jag inte är någon anhängare av partiet, utan bara en oroad medborgare.
 Pia: Så du menar alltså att man inte får säga vad man tycker i det här landet?
 Inringare: Ja! Till exempel får man inte ifrågasätta att 3 miljoner judar gasades ihjäl, för då blir man kallad förintelseförnekare!
 Pia: Eh ja då har vi nog gjort vårt för yttrandefriheten denna morgon, tack.

}

\small{
\textbf{Ringhals}
\label{ddd0de1cfca8045dfb1ef0eccdfdfeba}
 Sveriges enda kända arbetsplats där alla anställda går kalle anka \textsc{(s.~\pageref{64db68f686a0ca4d9d641061cb3fdf13})}.

 Alla svenska punkband men någon form av självaktning bör i bild eller text referera till Ringhals.

}

\small{
\textbf{Robotgräsklippare}
\label{108731cb2300809a0968a111a229c3af}
 Ett sociokulturellt skapad varelse som tycks härstamma från stadsdelen Rosenlund i Jönköping. Här lever Robotgräsklipparen sida vid sida med sin skapare. Den livnär sig på gräs, växelström och lite oljigt kelande då och då. Förutom sina nyttoeffekter så har Robotgräsklipparen tagit över gårdskaninens roll som socialt umgänge för barnen, när de vuxna är less på dem. Populära lekar som \quotetext{ride-the-mower} och \quotetext{mow-race} återfinns hos Rosenlundsungdomarna.

 Robotgräsklipparen delar många attribut med tamfåret, Ovis orientalis aries, men tycks besitta en något lägre intelligens. Av Robotgräsklipparens ull tillverkas de populära foliehattarna som hindrar oss från att kontrolleras av extraterrestiella krafter. Forskningsunderlag för möjligheterna till tidelag har uteblivit än så länge, även om inofficiella rapporter hävdar att det faktiskt går, om än med visst obehag.

 Husqvarna \textsc{(s.~\pageref{6671b561d336f97592b06a183ea47d3e})} hävdar bestämt att de har uppfunnit Robotgräsklipparen genom biodesign, men den kritiskt lagde kan givetvis se orimligheten i detta. Detta är förstås bara ett sätt för Husqvarna att tillskansa sig ära. Robotgräsklipparen har förstås skapats av miljoner år av gräsmattetävlande grannar emellan.

}

\small{
\textbf{Rocket science}
\label{cd83ae1072d0de2dfd8d3cf288511443}
 är engelska och betyder raketvetenskap, men är i förlängningen också ett ord för något som är intellektuellt utmanande. Man kan utan att överdriva säga att det vi gör här på Nissepedia \textsc{(s.~\pageref{62400dadecd90cb5cd39062abe5a3e4a})} är lite som rocket science.

}

\small{
\textbf{Rockin dopsie}
\label{b24cc3b0204ab50cff26fe318704a271}
 är en kreolfransk man i mantel \textsc{(se cape s.~\pageref{8b04f4091aa625f56b3f7da315a1e231})} och kungakrona som öser dragspel som det inte fanns nån morgondag.

}

\small{
\textbf{Rockpad}
\label{e714d651acb78d6c2d39110f84a4ef8f}
 En rockpad är en iPad \textsc{(se ipad s.~\pageref{09401fded433c34709fd1f1872728162})} helt fullmatad med skön rock såsom AC/DC, 'tallica, SLAYER!!, Ozzy, Accept och andra guldkorn.

 Rockpad är också benämningen på en lägenhet där den enda inredningen är en svart akustisk gitarr på väggen, ett kylskåp innehållandes 3 bärs \textsc{(se ha bärs s.~\pageref{a74b297c15834437ac2e49095492133c})}, en stereoanläggning laddad med en Creedence Clearwater Revival-CD och en flaska Jack Daniels stående på stereon.

}

\small{
\textbf{Rodriguesflyghund}
\label{19cc9824f1fa5834a866261fe69352ea}
 är egentligen inget djur utan en kille som heter Rodrigues som tror han är en flygande hund. Det började i skolan när det var temadag om droger och en föreläsare drog den gamla valsen om snubben som sitter i madrasserad cell och tror han är en apelsin och skriker \quotetext{Skala mig inte!} så fort någon kommer in. Rodrigues tyckte det lät jätteläbbigt och lovade sig själv att aldrig mer äta apelsiner eller knarka. Som ny hobby skaffade han pilgiftsgrodor men ibland glömde han tvätta händerna efter han lekt med dem så allt gick åt pipan ändå. Ingen har sett Rodrigues på länge, så det är därför han är listad som utrotningshotad \textsc{(se utrotningshotade djur s.~\pageref{24a427a5537c2c8918cfa213ae099a74})}.

}

\small{
\textbf{Roland Alkemyr}
\label{cccb313eb02b1fe9a4d236786de0c69e}
 HEAD2: CV Roland Alkemyr (Old Papa)


 1946 – 1953 Småskola, Skövde
 
 1953 – 1956 Realskola, Skövde
 
 1956 – 1958 Jungman på tankfartyget m/s Orient. Seglade primärt Hongkong – Singapore – Manilla
 
 1959 Rekryten som gruppchef på P4 i Skövde
 
 1960 Anställd som lagerarbetare på ICAs centrallager i Skövde
 
 1960 Korrespondenskurs i mekanik
 
 1961 Tjänstgör i 9månader som FN soldat i Kongo
 
 1962 – 1964 Anställs i Atlas Copco, arbetar som underingenjör i Södertälje
 
 1964 – 1965 Arbetar som administratör inom Atlas Copcos gruv och anläggnings division
 
 1966 Jobbar för AC i dåvarande Syd Rohdesia
 
 1967 – 1970 AC i Nacka, kompressortekniks divisionen
 
 1971 – 1974 Mellanchef för AC Nacka
 
 1975 – 1976 Projektanställd av Sida för brunnbornings projekt i Mozanbique
 
 1976 – 1978 AC svenske chef i Bangkok, Thailand
 
 1979 Kompleteringsutbildning av AC i Nacka värvas med 50\% administrativtjänst
 
 1980 – 1983 Arbetar med Bai Bang projektet i Vietnam
 
 1983 – 1984 Chef AC kontoret Taipei
 
 1985 Projektanställd av SIDA för arbete på Filipinerna
 
 1986 – 1989 Chef AC kontoret Filipinerna (Får 1987 guldklocka)
 
 1990 – 1992 Administarativ tjänst Nacka (trivs dåligt, Sverige är inte som det en gång var, dricker lite för mycket)
 
 1993 Tar 6 månader tjänsteledigt och reser runt i Borneos djungler
 
 1993 – 1996 Chef Malaysia kontoret
 
 1996 Avslutar sin anställning på AC efter 34 år
 
 1996 – 2004 Egen dykskola i http://angelescity.com/
 
 2004 -  Pensionerad, bor i Angeles men tillbringar 3 veckor runt jul och minst en månad på sommaren i Sverige på sitt lantställe på Västgötaslätten.

}

\small{
\textbf{Roland-hållningen}
\label{a432efc5e6070322761aee4f9e5748b2}
 är ett begrepp som utvecklats på bloggen Katastrofala skivomslag och som anspelar på ett sätt att stå: Man står helt enkelt lite kutryggig, men armarna hängande längs sidorna och ler tomt ut i luften. Perfekt när man är på krogen och är för snål för att köpa en öl, men är där för att titta på fruntimmer/mantimmer.

}

\small{
\textbf{Rolf}
\label{6c36a051b929758aec2d67fd9ee09f4e}
 är från början ett tyskt namn och betyder \quotetext{någon som skrattandes rullar omkring på golvet.}
 Rold Lassgård, bland många andra svenskar, bär detta anrika förnamn.

}

\small{
\textbf{Roliga timmen}
\label{6b677b75ecaeb31784a0b506dfac733d}
 är den sista timmen på framförallt låg- och mellanstadielevers skolvecka. Den fungerar som ett slags smidig övergång mellan realskolans alla bördor och krav och helgens alla förnöjelser med 24-karat på Sveriges \textsc{(se Sverige s.~\pageref{b1999637949ed135b2ca03f3a38460cc})} television, äventyrsbad \textsc{(s.~\pageref{8e36481b72c8061bb9ff74c1df3b0b66})} och glad lek med rullande tunnband. Det äts bullar. Det dricks festis. Storfräsarens ungar äter munkar och snickers och dricker coca-cola. Några spelar upp en pjäs som man repeterat i förväg. Den är inledningsvis i bästa fall diffus och snart blir skådespelarna för ivriga och nervösa och allt faller samman i skrik och spring. Men det är OK. Någon tagit med sin minikanin och visar upp den för klassen och den tjocka killen med jobbiga hemförhållanden skrämmer den och åker ut ur klassrummet. \textit{Roliga timmen}.

 HEAD2: Se även
 För information om hus roliga timmen firas i Storbrittanien, se happy hour \textsc{(se bärsfylla s.~\pageref{9380b60f9ee744b9acf978fe6f1a9545})}.

}

\small{
\textbf{Romantik}
\label{25c9060d553026737cc13fb8bd8474b4}
 kan vara att sitta på en pittoresk restaurang och stirra djupt in i ögonen på den du älskar. Det kan också vara att sätta sig i skogen med en flaska sprit, börja supa den och bara titta på svampar och mossor, ståtliga tallar och burriga granar. Om man i det tillståndet väljer att måla av det man ser, mynnar det hela ofta ut i tysk mustighet \textsc{(se den tyska mustigheten s.~\pageref{682ccd5fdc3aff0c97e8845c3d6b6ca8})}. Romantik kan också vara att titta på sengekantsfilm \textsc{(s.~\pageref{36f1eab94ebc00e11292cfaa67acafa0})}, själv eller tillsammans med nån schysst polare. Den formen av romantik är den som är minst ok.

}

\small{
\textbf{Roy Andersson}
\label{6191e0c6f2d7a26c73faa766913de0ef}
 Svensk vädergud. Roy Anderssons kamrater var inte riktigt snälla vid honom då han var pojk, och därför tar han nu ut en gruvlig hämnd på oss svenskar. Tack och lov så kommer Kenta Gustafsson \textsc{(s.~\pageref{b2e62eef29bb3f50253d932b26d4de76})} en gång om året och brottar ner honom. Då kommer sommaren till oss. Efter några månader har dock Kenta lyckats bli så full på all sommardricka att Roy vågar sig på ett tjuvnyp. Oftast lyckas han med det. Hade han inte gjort det hade vi fått EVIG SOMMAR, och alla skoterägare hade tagit livet av sig. En potentiell lösning på klimatproblemen vore således att lägga in Kenta \textsc{(se kennet s.~\pageref{eb251e3745d960e2100c5435a32764c4})} på alkoholklinik.
 Se även: Roy Andersson-väder \textsc{(s.~\pageref{a6da83b01b464521e7ce7ea04e61d314})}

}

\small{
\textbf{Roy Andersson-väder}
\label{a6da83b01b464521e7ce7ea04e61d314}
 är ett meteorologiskt begrepp som syftar på sådant trist, grådaskigt, mulet väder som får allt att se smutsigt och fult ut (inklusive människor).

}

\small{
\textbf{Rubank}
\label{b1c373a9ae319af9e1bf15a62fdf85cf}
 En något längre hyvel \textsc{(s.~\pageref{1668e2298ba60f14922e2cca6aa96538})}. MEN DET FÅR MAN VÄL INTE SÄGA I DET HÄR JÄVLA LANDET!

}

\small{
\textbf{Rudolf}
\label{c1ed4c1369f6af7c64dce16701d5383e}
 \textbf{Rudolf } är ett mansnamn med forntysk \textsc{(se tyskland s.~\pageref{b1b58da783b6d5fa090f3015f1889869})} ursprung, \textit{Hruodulf} bildat ur ord med betydelsen \textit{ära}, \textit{beröm} och \textit{varg}. Namnet har använts i Sverige sedan slutet av 1300-talet.

 Rudolf är ett vanligt namn bland de manliga pensionärerna i Sverige, men bland de yngsta är namnet sällsynt. Endast någon enstaka pojke i varje årskull får namnet som tilltalsnamn/förstanamn.
 Den 31 december 2005 fanns det totalt 5 102 personer i Sverige med namnet,
 varav 793 med det som tilltalsnamn/förstanamn.
 År 2003 fick 21 pojkar namnet, varav 1 fick det som tilltalsnamn/förstanamn.

 Namnsdag: 27 mars \textsc{(s.~\pageref{0dd330ae542925a3cf67035ae5a05bb7})}, (1993-2000 27 augusti).

 HEAD2:  Personer vid namn Rudolf
 \begin{itemize}
 \item Rudolf Meidner, löntagarfondernas fader
 \item Rudolf von Roth, tysk indolog
 \item Rudolf Schadow, tysk konstnär
 \item Rudolf Steiner
 \item Rudolf Hitler \textsc{(s.~\pageref{c9f9ecff9e5071300a593974776e5085})}, Adolf Hitlers okände tvillingbror
 \end{itemize}

}

\small{
\textbf{Rudolf Hitler}
\label{c9f9ecff9e5071300a593974776e5085}
 , född 20 april 1889 i Braunau am Inn i Österrike-Ungern, nuvarande Österrike, vid den lilla gränsfloden Inn till Tyskland. Rudolf är Adolf Hitlers okände tvillingbror. Historiker är övertygade om att Rudolf skulle antagligen nått större framgångar på världshärraväldesfältet än Adolf, om det inte vore för hans missprydande läppspalt.

}

\small{
\textbf{Rugga}
\label{e8c63efc29ae8ab4ec71c90efd7d7866}
 En rugga, på norrtäljeslang, är något som är mycket generöst tilltaget i proportion eller motsvarande. I andra delar av landet används ordet \textit{rackabajsare} för att tala om samma sak. Exempel på en ruggor är Tolstojs \textit{Krig och Fred} och Gerard Depardieus gurka \textsc{(s.~\pageref{1cf02b8eacd57c92e9df0a1a3eaa8946})} (dvs näsa).

}

\small{
\textbf{Rugguggla}
\label{6af1b4e2e210d1aef03643fb57c86bc2}
 En rugguggla är man om ens utseende talar emot en eller om man för en tid har misslyckats med att vårda sitt yttre, även om man i normala tillstånd icke är särskilt ful \textit{per se}. Ruggugglan är således icke en fågel i släktet uvar \textsc{(se uv s.~\pageref{45210da832f9626829457a65e9e7c4d0})}.

}

\small{
\textbf{Rulla hatt}
\label{7c7afc9fb7bb52962f954c0cb548c10c}
 betyder samma sak som att gå på lokal, det vill säga att gå ut i senkvällen och dricka alkohol. För att man ska kunna sägas rulla hatt måste man dricka en viss mängd - man kan inte bara ta en öl och sedan ursäkta sig med att man har tvättid. Uttrycket kommer sig av att det det förr var vanligt att överförfriskade herrar tog av sig hatten och som ett litet spratt rullade den nedför gatan.

}

\small{
\textbf{Rundpingis}
\label{921d37e7abc34a9b324440904981aabf}
 Uppehållsrummets skiljelinje.

}

\small{
\textbf{Running man}
\label{400d526ce58fd38a219bc4d896ad313a}
 är inte bara en film med Arnold Schwarzenegger i rollen som den springande mannen utan också ett populärt dansmove som odödliggjordes av MC Hammer i videon till U Can't Touch This [http://www.youtube.com/watch?v=otCpCn0l4Wo]. Movet går i korthet ut på att man springer på stället samtidigt som man viftar med armarna i takt till musiken som spelas på det disco eller party man är på. Det som gör Running Man unikt är att det är ett move som inte riktigt är en dans, vilket gör att det är okej att gå upp till dansgolvet, lägga in en halv minut kvalitets-runningmannande och sedan oberört gå därifrån igen utan att någon kan reta en för att man dansar.

}

\small{
\textbf{Rusta}
\label{2217dae26ef18f6a32d5aa0d7a032a16}
 är ett annat och bättre ord för att laga (käk). \quotetext{Nej nu kan jag inte sitta \textsc{(s.~\pageref{123c3e95c62201513a344526a2fec502})} här längre. Nu är det dags att rusta käk,} kan man säga, till exempel.

}

\small{
\textbf{Rut}
\label{915a6a53e87cde98efa46d557a4625b9}
 Alla barnen tittade i mikrovågsugnen utom Rut för hon tittade ut.

 RUT är även en typ av socialbidrag som rika människor som inte gillar att städa kan söka.

}

\small{
\textbf{Ryck-i-pung-vägen}
\label{52661fd211a8c2f2a99fb0a099501cac}
 , eller som det heter på japanska, \textit{Heng Dang Hao}, är en kampsport som uppfanns av shintoistiska munkar under Ashikagaperioden (1338 - 1573 e.kr). Kampsporten går i mångt och mycket ut på att förolämpa sin manliga motståndare genom att röra vid dennes skrev.
 För andra betydelser, se Ryckepungvägen \textsc{(s.~\pageref{5de8d116216d8cccf96cef9171a0eb69})}.

}

\small{
\textbf{Ryck-i-punk-vägen}
\label{7bbc284b11fe1e2cc0d28eb1cf7d8dc8}
 Ryck-i-pung-vägen \textsc{(s.~\pageref{52661fd211a8c2f2a99fb0a099501cac})}

}

\small{
\textbf{Ryckepungvägen}
\label{5de8d116216d8cccf96cef9171a0eb69}
 är en gata i Falun där Mekonomen ligger.

}

\small{
\textbf{Rygg}
\label{7c74337e8958a60864119ecbd907e85d}
 en är den kroppsdel som finns bakpå kroppen, ovanför stjärten, men under halsen. Ryggen har i den västerländska kulturhistorien åtnjutit en aura av mystik, vilken tros ha uppkommit genom det faktum att ryggen endast med stor möda kan ses av den som äger den. Alla människor och djur har en rygg, men på vissa djur, till exempel maneter \textsc{(se manet s.~\pageref{f95133906d1eba86b61fc05be6aecd9c})}, är det svårt att avgöra exakt var på djurets kropp ryggen finns. På ryggen placerar man de fyrkantiga tygstycken som det står \quotetext{Saxon} på.

}

\small{
\textbf{Rygga}
\label{af4187177b54a8f9bf0a4e0ee8b1f8e9}
 Att rygga något är att snatta genom att placera varan, t.ex. ett paket kaffe, mellan ryggsäcken och ryggen och puta ut med stjärten för att hålla den tillskansade varan på plats.

}

\small{
\textbf{Ryggtryck}
\label{a5caecaf8bd113e5ecd7c924db801c44}
 Sverige har Carl von Linné \textsc{(s.~\pageref{5e8380bf6b7ce99678e6752b6d9e709e})}, Norge har sin olja, Danmark \textsc{(s.~\pageref{5331d7fd27772396f412a5b6d19bad44})} har avsagt sig alla sociala normer samt har lego \textsc{(s.~\pageref{3a22c9ea9a3039d180e0a514a5a3b619})} - och Finland \textsc{(s.~\pageref{631d44eaa1254ff71a1e11ba021d1266})} har ryggtryck.
 Ryggtryck är en modeterm som benämner bilder som föreställer antingen exotiska djur, turnéscheman eller en klatchig anti-social slogan som tryckts på ryggen av en t-shirt eller en munkis. Denna oväntade vändning i modehistorien gör det möjligt för en finländare att signalera för en person som står framför henneom att hen lyssnar på thrash metal samtidigt som alla som råkar passera bakom uppmanas att fara dit pepparn växer. Finlands framstående position i rockens universum har gjort att ryggtrycket spritt sig till andra delar av den industrialiserade världen, men utanför Finland har det inte riktigt tagit sig in i finrummen än.

}

\small{
\textbf{Rymden}
\label{6d5ad1e8996d7ec9d8ac6058649290c0}
 är ett riktigt jävla gammalt ställe, äldre än både dinosaurierna och Centerpartiets \textsc{(se Centerpartiet s.~\pageref{e331dec360e356adc1e2db36fe9a9f3f})} senaste bra idé. Flera har försökt räkna ut exakt hur gammal rymden är men ingen har kommit längre än till en postiljon \textsc{(s.~\pageref{cdf093679eb50a181e4ef701ed856b97})} utan att tröttna. Fastän den är så gammal är det jättemånga saker som ingen vet om den, till exempel var den kommer ifrån och vad som finns i den. Detta har dock fört det positiva med sig att man kan spekulera ganska fritt, vilket många också tagit chansen till. De bästa teorierna kommer från musiker som antigen spelar så kallad \quotetext{rymdrock} eller \quotetext{kosmisk doom}, och bäst spekulerar gruppen Deathrays from Space. Enligt dessa idéer är rymden ett jäkligt häftigt ställe som man blir helt överväldigad av. Precis så som man vill ha det.

 HEAD2: Geografi
 Förutom att vara gammal är rymden också sjukt stor. Någon komplett karta finns inte så den som besöker rymden har chansen att känna sig lite som Christopher Columbus när han hittade USA \textsc{(se United States of America s.~\pageref{ade6b3bd5e720abb20ed8a9a4c6b9ae8})} eller som England när dom hittade Skottland \textsc{(se skottar s.~\pageref{c2e5f84c76d823ea9482387bfb950791})}. Det mesta är bara tomrum men här och där ligger några gamla planeter och stjärnor. Avstånden är så långa att man varit tvungen att hitta på helt egna måttenheter.

 HEAD2: Politik
 Rymden var under kalla kriget en stor källa för konflikt då USA och Sovjet tävlade om vem som kom upp i den först. Sovjet vann. Jänkarna började sedan att kolonialisera rymden genom att sätta stjärnbanéret på dess grå och skrovliga yta, men huruvida detta räcker som paxning \textsc{(se paxa s.~\pageref{0e00979a45d6f4083485e9c9fb01f590})} är omtvistat. Sveriges förehavanden i rymden överses av Rymdminister Jan Björklund \textsc{(s.~\pageref{0b9b757044804b9be0e218acdad358cc})} som efterträdde den tidigare rymdministern Maud Olofsson \textsc{(s.~\pageref{eb913a2e9be929654908a05017401bd6})}, ett uppdrag båda får antas ha skött ganska bra.

 HEAD2: Ekonomi
 Rymden är full av värdefulla metaller men avstånden gör det tyvärr svårt att hämta dom till en fabrik. Den största inkomstkällan för rymden har därför blivit turism. Ryssarna var först ut med att investera i en stugby som hette MIR, men den fungerade inte så bra och är nu skrotad. Ett antal länder gick därefter ihop och startade kollektivboendet ISS som har fungerat lite bättre. Fast än så länge är det inte jättemånga som åkt dit för det finns en piratkopia av rymden i Stockholm \textsc{(s.~\pageref{edcd259e0a03c7ab70feb186bae19f13})} som heter Cosmonova som är mycket billigare att besöka. Kända individer som turistat i riktiga rymden är bland annar Albert II \textsc{(s.~\pageref{8a80daf328e56d2b30df9fb6c782146d})} och Tsygan \textsc{(s.~\pageref{4da32eca7858fd215a250d871bf7fa7b})}.

 HEAD2: Kultur
 I rymden finns nästan ingen kultur. NASA har ställt en amerikansk flagga på månen och skickat ut en vinylskiva med Beethoven och Rolling Stones. Sovjet slängde upp en satelit som sände ut en midiversion av \textit{Internationalen}, men annars har det typ inte hänt så mycket spännande i rymden.
 HEAD3: Kulturell rymdresa
 Desto mer spännande är hur fok reser ut i rymden med hjälp av musik. Ända sen sitaren uppfanns har människan tagit sig ut i rymden med hjälp av musiken. Genom att lägga en vinylbit Pink Floyd på spelaren, halsa tre deciliter sprit och sluta ögonen för att fokusera sina tankar på orions bälte, har många sjukpensionärer iförda mjukisklädsel \textsc{(s.~\pageref{57a78bf29e9f6fb6a4dba89fc21bc897})} kommit närmre en totalupplevelse av rymden än Christer Fuglesang. Att resa ut i rymden med musik är idag vanligast i Danmark \textsc{(s.~\pageref{5331d7fd27772396f412a5b6d19bad44})}. Kanske för att de flesta hellre befinner sig i ett svart vakum än i norra Europas främsta fascisthôl.
 HEAD3: Kemisk rymdresa
 Man kan också genom att konsumera olika sorters kemiska sammansättningar resa ut i rymden för ett slag. Även detta är vanligt i Danmark \textsc{(s.~\pageref{5331d7fd27772396f412a5b6d19bad44})} av ovanstående anledning, men förekommer också i Sverige \textsc{(s.~\pageref{b1999637949ed135b2ca03f3a38460cc})} och andra i-länder där folk har tid och pengar. Det är viktigt att tänka på att göra detta vid ett tillfälle då man inte befinner sig på jobbet eller har främmande på besök.

 HEAD2: Sammanfattning
 Rymden är bäst hemma i soffan, så där lite på avstånd, där man kan överväldigas av den på ett bekvämt och avslappnat sätt.

}

\small{
\textbf{Ryska rövare}
\label{196e0458db510192146b2f885a9a3fee}
 har skägg och ledsna ögon. De förvarar  gurkor \textsc{(se gurka  s.~\pageref{1cf02b8eacd57c92e9df0a1a3eaa8946})} och korvar under sadeljorden.
 Deras främsta fiender är stäppvargen och kapitalismen.

}

\small{
\textbf{Räka}
\label{2e1cdd6fa81f4968c8c527854e0c629b}
 Trygga räkan
 ingen vara
 i Guds \textsc{(se Gud s.~\pageref{91e49146121c992aab11a19c77e26cf0})} lilla
 barnaskara


 Inom gitarrbranschen är räka det samma som engelskans \quotetext{lick} (ung. \quotetext{slick}).

}

\small{
\textbf{Räkmacka}
\label{2d749ddfe869664c96fe0a0b572c5f0b}
 .]]
 Räkmacka är en typ av underlag med väldigt bra glid. Dess glatta yta har visat sig ytterst lämpligt som smörjmedel vid de flesta tillfällen men används främst av bekväma personer.

 Den nära besläktade Räksmörgåsen \textsc{(se Räksmörgås s.~\pageref{e462805dcf84413d5eddca45a4b88a5e})} är en av få maträtter som innehåller både å, ä och ö.

}

\small{
\textbf{Räksallad}
\label{b7e642c04871e612342088c068b7bd65}
 en var ursprungligen ett postmodernt provokationsprojekt utfört av den franske tankeakrobaten Jean-François Lyotard. Lyotard inkluderade i sitt magnum opus \textit{Det postmoderna tillståndet} ett recept på räksallad. Fransosen menade däri att tre deciliter majonnäs med lite skaldjursskulor i, hade lika mycket rätt kallas för en sallad som den mer traditionella grekiska eller caesarianska modellen. Detta var en uppenbar provokation, riktad mot det totalitära tolkningsföreträde som dittills rått i de gastronomiska finrummen. Till mångas förvåning (säkerligen också Lyotards) blev snart räksalladen en folkkär favorit, särskilt på Sveriges västkust, med viss spridning även i norra Småland och Värmland. Till dags dato har räksalladen krävt lika många liv som Jonestownmassakern, Francesco Schettino och Zodiacmördaren sammanslaget.

 Se även gurkmajonnäs \textsc{(s.~\pageref{67accedc0d647557edea0dc0f54fe3be})}.

}

\small{
\textbf{Räksmörgås}
\label{e462805dcf84413d5eddca45a4b88a5e}
 s gamla programledare Ingvar Oldsberg, här hållandes en räksmörgås.]]
 Räksmörgås är en typisk rätt som serveras fredagsmysande \textsc{(se fredagsmys  s.~\pageref{bba247fdca80768603373605cbe7a934})} vuxna och består av en macka med räkor på. Räksmörgåsen blev populär i Sverige \textsc{(s.~\pageref{b1999637949ed135b2ca03f3a38460cc})} på sjuttiotalet och har egentligen inte försvunnit från topplistan över smutta \textsc{(se smutt s.~\pageref{d9114ffee4f2dcee302ae2b19ce5eea9})} saker att äta framför På Spåret \textsc{(s.~\pageref{1151b0ad0009ea36c9a1a95736a37a8e})} eller Melodifestivalen sedan dess, trots hård konkurrens från bland annat fondue-grytan, popcornmaskinen och ostbrickan \textsc{(se ostbricka s.~\pageref{bf06c995c523e159eb93017810ee8f44})}. Ja, och så innehåller ordet \textit{räksmörgås} - till skillnad från kusinen \textsc{(se kusin s.~\pageref{f7f20d5744925e2e72e5524035a162be})}, räkmacka \textsc{(s.~\pageref{2d749ddfe869664c96fe0a0b572c5f0b})} - de svenska diakritiska bokstäverna \textit{å, ä} och \textit{ö}.

}

\small{
\textbf{Rätt stuk}
\label{82092b3633fbc6a8adb6b6e274db11f0}
 Pentagram, Death Row \& Bedemon

 Se även Rått stuk \textsc{(s.~\pageref{f96d9458c52af76daad996ec3a7502a4})}, Rött stuk \textsc{(s.~\pageref{542346924f395703bc97026934d6d645})}

}

\small{
\textbf{Rättshaveri}
\label{5934ac72115b22d16348e6c6e5eb2fd9}
 Att ha rätt, ett rättsmangel \textsc{(se mangel s.~\pageref{ecc5b41821ed829b0c3fb48d4d5389ed})} så att säga.

}

\small{
\textbf{Räva}
\label{c13f687883c1eb0be3be218fff63e6b8}
 Rävning är ett annat ord för Ragga/Raggning: eller som en van jägare brukar säga, en benämning på en mycket seriös form av raggning! Lite uttryck ni behöver veta om rävning:

 För att kunna räva så måste du först inleda en jaktsäsong: detta innebär att du har bestämt dig för att försöka fälla en räv och att du därför är på 100\% rävningshumör.

 När väl jaktsäsongen är inledd är det dags att välja ut rätt jaktmarker för den kommande rävningen. Vanliga typer av jaktmarker är krogar, pubar och diskotek (heter det så?). Jaktmarker bör väljas ut efter vart man tror det finns mest rävar och störst sannolikhet att fälla en fin guldräv.

 När sedan både jaktsäsongen och jaktmarken är inledd och utvald brukar en jägare svida om till flanellskjorta och bössa. Egentligen är ”flanellskjortan” ett kodord för sexiga snygga festkläder och bössan är i sig ett symboliskt uttryck som står för att man ska ”pricka” en räv på krogen och dra med han hem.

 En jaktsäsong kan man inleda och avsluta precis när som helst. När man har avslutat en så brukar det vara vanligt att man säger att man \quotetext{lagt rävningen på hyllan}. Det innebär bara att man helt enkelt inte är sugen på att ragga just den helgen. Enas man inte om beslutet om att lägga rävningen på hyllan så kan din jaktkamrat välja att ta den och sparka, skjuta, kasta eller slå bort rävningen från hyllan. Detta sker genom att ens jaktkamrat gestikulerar en låtsas sparkning, kastning osv i luften. Haha kan se mycket lustigt ut för utomstående som inte förstår vad din polare står och sparkar i.

 Avslutningsvis så finns det flera olika rävar att jaga på. Rävar är ett annat ord för killar om ni inte redan har fattat det. Det finns: Ungrävar \& Rävungar – unga killar i åldrarna 16 till 19år. Vanliga rävar – killar som är som det antyder: vanliga. Guldrävar – riktigt snygga killar i åldrarna 20 till 30 som har allt man letar efter. Silverrävar – äldre herrar, vanligtvis i sena 40års åldern och uppåt. Skabbrävar – killar som antigen är fula, kladdiga eller på andra sätt inte lockande för jägaren.

 KOM IHÅG! Var ansvarsfulla när ni jagar! När väl en räv är skjuten så är det bara att släpa hem han! Jaktlicens finns att hämta hos Moe \& Andersson och får användas både för målskjutning och jakt. Bor ni i Umeå borde ni veta vart ni kan finna dem.

}

\small{
\textbf{Rävioli}
\label{42bf16daa2dc99e533f20521b3f4cd15}
 (ibland stavat raevioli) är raviolikuddar fyllda med räv. Det är även något av ett artistnamn på den begåvade konstnärska från Luleå som uppfann maträtten. [http://raevioli.blogspot.se/2009/03/ravioli.html] De smakar ungefär som uvioli, men låter annorlunda när man sticker gaffeln i dem. Om det röda som rinner är tomatsås eller något annat får var och en avgöra.

}

\small{
\textbf{Rävir}
\label{d91594f5ca5978c2d85119bb629cf10f}
 Den areal en räv lever inom.

}

\small{
\textbf{Rå, men hjärtlig jargong}
\label{d6d2daa05753e48d8250a9b74bb0f163}
 Utan detta hade alla jämställdhetskonsulter varit utan jobb.

}

\small{
\textbf{Rått stuk}
\label{f96d9458c52af76daad996ec3a7502a4}
 Bastard Priest, Repugnant \& Morbus Chron

 Se även Rätt stuk \textsc{(s.~\pageref{82092b3633fbc6a8adb6b6e274db11f0})}, Rött stuk \textsc{(s.~\pageref{542346924f395703bc97026934d6d645})}

}

\small{
\textbf{Råååål}
\label{0fe5024fd518671cbc30c762e942eed4}
 är den ål som har flest ”å” i namnet. Det är den självklart jävligt kaxig över. Den lever i Rååån som rinner genom Råå.

}

\small{
\textbf{Röda Rummet}
\label{aafa97eeb30ae5e18db5b03260afaa5a}
 var fram sin alldeles för tidiga död 2011 en läsesal vid Umeå Universitet \textsc{(s.~\pageref{11dfc744fa396b961a6cc9cf89c4eaea})}. Här blev man hyschad om man så mycket som tappade en penna, vilket ägde. Strindberg lär ha blivit så betagen av detta läsrum då han var i trakten att han valde att förlägga händelseförloppet i sin mest lästa roman i denna lokal.

 Rummet ovanför detta som nu ärvt Röda Rummets funktion och kallas \quotetext{Tysta läsesalen} officiellt, kallas i folkmun fortfarande \quotetext{Röda Rummet}, eller kort och gott \quotetext{Röda}.

}

\small{
\textbf{Rödhalaffel}
\label{20d52b6aa68cf52d47bb5025ba05cb68}
 En Rödhalaffel är som en falafel \textsc{(s.~\pageref{b2d6ec45472467c836f253bd170182c7})} i pitabröd, men istället för traditionell falafel har kocken lagt i ett gäng friterade rödhakar.

}

\small{
\textbf{Rökrock}
\label{0eaceca559fb616779dbe004972eba07}
 Att köra bil och bränna med full sula rakt in i en vassrugge.

}

\small{
\textbf{Röksignaler}
\label{d609b8ec548594516ebe26a488c622ca}
 är en interaktiv teknik för spridande och inhämtande av information över stora avstånd. Tekniken bygger på samma princip som transistorn \textsc{(se transistor s.~\pageref{aaa0d78af49a2fbc2f7ad8fbb11de1aa})} (ström/inte ström, rökpuff/inte rökpuff) och uppfanns av en gammal indian, Hövding Gamnacke, som tröttnat på att sitta \textsc{(s.~\pageref{123c3e95c62201513a344526a2fec502})} och svettas i sin malätna poncho. Istället bytte han om till höftskynke och slängde den gamla trasan på lägerelden. En annan indian, Lilla Coyotedräparn', ryckte dock ut och räddade plagget eftersom han insåg hur festligt och bizarrt det skulle kännas att vara naken under den. Båda insåg genast vilket revolutionerande fenomen de snubblat över, så dom rökte en fredspipa och resten är historia.

}

\small{
\textbf{Rökå}
\label{b06106b8f786098f1ff569a4f75dc3c8}
 är en ort i Malå \textsc{(s.~\pageref{41da4620e87888eaaeafcb3004a8d177})} kommun. Byns slogan är \quotetext{Hä ska du vetta och hä väit du} som på rikssvenska betyder \quotetext{Det ska du veta och det vet du}.

 Rökåbygden såg till att hålla Umedalens mentalsjukhus igång. I Rökå bor 50 personer uppdelat på tre efternamn, Hedström, Bjuhr och Hultmar. Här bor också det äkta paret Atmar och Lucia.

}

\small{
\textbf{Rör}
\label{ce0e999c89048236e26f8fed04c0b338}
 Ett rör kan vara tillverkat av diverse olika material som tex plast, stål, betong osv.
 Det mest utmärkande med ett rör är att det har ett hål som löper oftast i mitten inuti materialet längs hela dess längd.

 Hålet kan vara i olika storlek men det är viktigt att hålet är inuti röret, annars glappar det på utsidan.

 Ett rör kan vara kort eller långt, men det finns också extra långa rör.

 Ett väldigt mjukt och böjligt rör kallas för slang.

 Rör som liknar skor kallas för bandyrör.

 \quotetext{Far, får rör rör?}
 \quotetext{Nej, rör får rör}

}

\small{
\textbf{Rötmånad}
\label{e494bf49e0b4698a75ec1c3ff6397e5e}
 Den svenska motsvarigheten till Ramadan.

}

\small{
\textbf{Rött}
\label{dacd03b85a85d8c8b67c702e1872c498}
 är ett etablissemang i stadsdelen Berghem \textsc{(s.~\pageref{a6b1df39fa9b1b94dc92200594a8ccd6})} i Umeå. Här säljs olika alkoholhaltiga drycker till den törstige \textit{flanûren} och enklare mat till den hungrige. På grund av krogens läge invid Umeå Universitets campus går affärerna relativt bra eftersom postseminarium \textsc{(s.~\pageref{3dcf7466504a8591f86ba7e472606ef6})} och arbetsrelaterade middagar ofta går av stapeln just på Rött och på grund av den utbredda alkoholism som många akademiker har utvecklat som ett sätt att hantera år av förödmjukande nederlag, aggressiva studenter och föraktfulla medarbetare. Man serverar också gladeligen fotbollsspelare som tagit ut sig på någon av de närbelägna fotbollsplanerna \textsc{(se fotboll s.~\pageref{961bd74d34872ff94a4df3a16119096e})}. Den andra stora inkomstkällan, förutom försäljning av mat och dryck, är människosmuggling. Så här kan det se ut en vanlig fredageftermiddag: \textsc{(se fredag s.~\pageref{80d41f1e0b14eacb229eea9618632e88})} [http://www.youtube.com/watch?v=9Zi8BQ7hncg\&feature=player_embedded\#at=42]

}

\small{
\textbf{Rött stuk}
\label{542346924f395703bc97026934d6d645}
 Knutna nävar, Röda Bönor \& Nynningen

 Se även Rätt stuk \textsc{(s.~\pageref{82092b3633fbc6a8adb6b6e274db11f0})}, Rått stuk \textsc{(s.~\pageref{f96d9458c52af76daad996ec3a7502a4})}

}

\small{
\textbf{Röv}
\label{08e887710123ae191d6b777e3e65170c}
 Substantiv (slang, vulgärt) \textit{bakdel} (på människa). Böjes röv, röven, rövar, rövarna. Mest kända är Finnforsrövarna.
 Förekommer lokalt även som interjektion och preposition.
 Se också: Ryska rövare \textsc{(s.~\pageref{196e0458db510192146b2f885a9a3fee})}

}

\small{
\textbf{Rövgitarr}
\label{bbe5289fcebfb381fdf0056e5963d97d}
 En rövgitarr är ett strängförsett musikinstrument. Den uppfanns i Danmark \textsc{(s.~\pageref{5331d7fd27772396f412a5b6d19bad44})} och skiljer sig mot andra  gitarrer i det att den \textit{alltid} låter rent förjävligt (på danska: røvigt.), därav namnet. Att spela rövgitarr är inte särskilt svårt, det räcker i princip att slå på strängarna med vad som helst för det låter ändå lika dant. Länge trodde konspirationsteoretiker att Lemmy \textsc{(s.~\pageref{6cc2f8758343439728f308f08a4a8fad})} spelade på en kamouflerad rövgitarr. Men det visade sig att det var en helt vanlig Lemmy-bas \textsc{(s.~\pageref{8788bd84a131e0d292f8d966b03745d4})}. Den kändaste brukaren av rövgitarr är istället Kim Larsen, som drar ett långt solo på singelversionen av sin hit \textit{ Væd gør vi nü lillæ dü?}. Danskättlingen, tillika Metallicatrummisen, Lærs Uølrich försökte rida på hajpen genom att införa rövtrumsettet. Men det är han fortfarande ganska ensam om.

}

\small{
\textbf{Rövioli}
\label{79e177b40d15248934717a0eb5d7dbe6}
 är små pastakuddar med älgbajs inuti. De enda som köper dem (och äter dem!!!) är tyska turister med minst 0,2 promille i blodet eller levercancer. Trots sin unga status är röviolin snabbt på väg att införlivas i traditionell tysk medicin.

}

\small{
\textbf{Rövmust}
\label{1691a5d2bb9213e11f5cce859518326b}
 är den skummiga, något mustiga vätska som utsöndras i samband med avföring, oftast efter en vild kväll av alldeles för mycket folköl, mäsk och salta pinnar.
 Rövmust kan också förekomma under perioder av allvarlig magsjuka, såsom influensa och biverkningarna av att vidröra en Träskpunkare \textsc{(s.~\pageref{484838b3db1adb135ea74d6fc61e44c0})}.
 Lexe Crustare \textsc{(s.~\pageref{4dc09ec7fd33842218230329beb42691})} är också en känd smittokälla.

}

\small{
\textbf{Rød pølse}
\label{dd9f0cd7c204300945924c7de9eb5649}
 är korv som görs på överblivna delar från andra maträtter inom det danska köket. Alla rester (exempelvis överdelen på Ballerinakex, sallad, fläsksvålar med kort datum) samlas först i en kompostliknande behållare som har ett grovmaskigt nät över sig för att förhindra allt för stora objekt såsom pilsnerflaskor och klövar följer med. Därefter lyfts nätet bort och man går loss på det som trillat igenom med en högaffel eller golfklubba för att få det mera finkornigt. Därefter späds massan med tapetklister och tappas upp i formar innan den hinner stelna. Den karaktäristiska röda färgen uppstår i slutfasen av framställningen genom att korvarna kokas i falu rödfärg.


 Källa: Prof. Etienne \textsc{(se Användare: Prof. Etienne s.~\pageref{a9878d2280e5a39becac8f73d113df91})} - \textit{101 saker man kan lösa med våld}. Bonnier Fakta, Bälinge 2001.

}

\small{
\textbf{S:t Olof}
\label{7559216f1b9708d5a2bc8ff1b1a4f066}
 Alla vägar bär till S:t Olof. Alla vägar bär även från S:t Olof. Enligt ZOG är S:t Olof resultatet av den judiska skökans framfarter. Men alla visa människor vet att S:t Olof är världens mittpunkt.

}

\small{
\textbf{Saida Andersson}
\label{b415cf75bbb474ceaed1e38d2d637939}
 , född 8 oktober 1923 i Älvsbyn, död 14 november 1998 i Boden, var en svensk \textsc{(se Sverige s.~\pageref{b1999637949ed135b2ca03f3a38460cc})} sierska som påstås ha haft övernaturlig förmåga.

 Sierskan Saida Andersson levde under sina första år i en vinterladugård med stampat jordgolv i byn Petbergsliden i Älvsbyn. Redan som 15-åring blev Saida gravid. Hon lämnade föräldrahemmet och födde dottern Siv-Gun. Som 19-åring gifte hon sig och fick sedan ytterligare fem barn: Tony, Eilert, Dennis, Sol-Britt och Morgan.  Hon skilde sig i slutet av 1960-talet och träffade långtradarchauffören Hilding Nilsson. Saida tog hand om barnen. Hon födde upp grisar, städade bussar, skottade snö och plockade bär för att kunna försörja dem. 1960 flyttade familjen till byn Bredåker i Boden.

 Den gåva som Saida påstås ha haft handlar om att förutsäga framtiden.  Saida hävdade att hon tidigt kände till sin förmodade gåva, men att hon länge avstod från att skylta med den av rädsla för att uppfattas som \quotetext{konstig}.  Saida ska ha förutspått när och hur tre av hennes söner skulle dö, liksom även sin livskamrats död.

 1988 belönades hon med en guldmedalj av \textit{Svenska jägareförbundet} som tack för all hjälp. Samma år spådde hon bl.a. Ingvar Carlsson i Ulf Elfvings radioprogram \textit{Efter tre}. Senare sade hon att USA skulle drabbas av en stor jordbävningskatastrof. Den kom i San Francisco i oktober 1989, men det är oklart om förutsägelsen var så specifik och oväntad att den var meningsfull – jordbävningar i USA är inte ovanliga.  Samma sak gäller när hon i Nordnytt, SVT:s lokala nyhetsprogram för Norrbotten \textsc{(s.~\pageref{0e8c003b75982032cde152609ee94154})} och Västerbotten \textsc{(s.~\pageref{d4b008c5143dcffb6b8c35f3876c2a19})}, siade att Ian Wachtmeister och Ny Demokrati skulle komma in i riksdagen – det hade gått att gissa utan övernaturlig hjälp.

 I början av 1990-talet vann hon ett test med synska personer i norsk tv med vad som sades vara 80\% \quotetext{rätt}. 1991 utsågs hon till Årets Bodensare.
 Saida Andersson blev riksskänd när hon i \textit{Café Luleå} på TV under 1990-talet i direktsändning gav människor \quotetext{hjälp} att finna sådant de tappat bort.

 Hon hade under många år en spalt i veckotidningen \textit{Hemmets Journal}, där hon gav råd till människor som tappat bort föremål, husdjur eller personer. Hon kunde få 1000 brev i veckan och ytterligare 80-100 brev till bostaden. Bl.a. hjälpte hon artister, tjänstemän, en landshövding, riksdagspolitiker, flera kommunalråd, poliser och flera av landets större företagare. Hon svarade självsäkert och kortfattat på frågorna, men det saknas belägg för hur träffsäker hjälpen var.

 Förutom att hon ska ha hjälpt människor att hitta smycken, djur och försvunna personer ända in i det sista, så var hon skrockfull.  Den 14 november 1998 klockan 11 avled Saida Andersson på Bodens sjukhus.

 HEAD2: Släktbandet
 \begin{itemize}
 \item Saida sades ha ärvt gåvan från sin mor \textbf{Hanna Lidman} som föddes år 1900 och ansågs vara en sentida häxa. Det sägs att Hanna fick två barn med gåvan.
 \begin{itemize}
 \item \textbf{Erry Lidman}, Saidas bror, avliden.
 \item \textbf{Saida Andersson} fick sex barn. 1998 var tre av dem vid livet. De tre påstås ha ärvt sin mors synskhet.
 \begin{itemize}
 \item Saidas dotter \textbf{Sol-Britt} är aktiv sierska och figurerar i \textit{Aftonbladet \textsc{(s.~\pageref{e9ebf180c01d806db2fefd7f53b7a146})}}. Sol-Britt har tre döttrar som alla sägs ha gåvan.
 \item Saidas son \textbf{Dennis} arbetar/arbetade inte praktiskt med sin påstådda förmåga.
 \item Saidas äldsta dotter \textbf{Siv-Gun} sägs använda/ha använt förmågan ibland.
 \item Saidas son \textbf{Morgan}, avled 1987
 \item Saidas son \textbf{Tony}, avled 1979
 \item Saidas son \textbf{Eilert}, avled 1978
 \end{itemize}
 \end{itemize}
 \end{itemize}

}

\small{
\textbf{Saippuakauppias}
\label{f5119f497d56eedfa098dd2df699369b}
 är ett finskt ord som betyder tvålförsäljare. Det är världens längsta palindrom.

 Källa: Forum för levande historia.

}

\small{
\textbf{Saltbas}
\label{f3110a9d4fef2f2adad9020c8b59249a}
 är en speciell teknik att spela bas där man typ duttar och slår på strängarna i stället för att spela som vanligt. Namnet kommer sig av att det lättaste sättet att få rätt feeling när man spelar är att tänka sig att basen är en stor hummer som man måste salta ordentligt innan förtäring. Med detta i minnet är det bara att låta fingrarna börja studsa runt över instrumentkroppen.


 HEAD2:  Källa

 Överhört samtal på ett dassigt ölhak.

}

\small{
\textbf{Saltsjöbadsavtalet}
\label{8fb9ced0a7fc25125895e5496f9e95b8}
 tecknades i överklassområdet Saltsjöbaden i Stockholm mellan facket och kapitalisterna.

 Kortfattat gick det ut på att facket sa: \quotetext{Vi lovar att inte strejka och ställa till med jävelskap \textsc{(s.~\pageref{46845591177f16920cd586a5baf5a625})} för er.}
 Kapitalisterna sa \quotetext{Tack} med munnen full av oxfilé och vaktelpaté.

 Ett av många nederlag i mänsklighetens historia \textsc{(s.~\pageref{5d87ba4132f8bdfa8c6294514c570c3f})}.

}

\small{
\textbf{Samer och Læstadianer}
\label{5124a9a5187b57c0108a65fe4f0a2de9}
 är en genre inom svensk populärfilm som i mycket påminner om amerikansk såkallad \textit{cowboys and indians}-film. Många samer och læstadianer-filmer utspelar sig i Västerbotten \textsc{(s.~\pageref{d4b008c5143dcffb6b8c35f3876c2a19})} och skildrar hur læstadianer systematiskt förföljer samer för att kristna och slå ihjäl dem.
 Se även: Lars Levi Laestadius \textsc{(s.~\pageref{c91fcd34b5328c4a87e4ae93efa97bfc})}

}

\small{
\textbf{Sametjej fetisch}
\label{f1f750854c80fafbff4558ec03b1cfa3}
 Beskriver enkelt någons förkärlek för sametjejer \textsc{(s.~\pageref{a7aa534c82ea6388af6dc3e25e3fd01b})}. Ofta förekommande bland nordliga veganer/vegetarianer \textsc{(se veganer s.~\pageref{2a12d5d6ae91d2f4f7d9af3cef58e75c})} som är rätt så otekniska av sig.

}

\small{
\textbf{Sametjejer}
\label{a7aa534c82ea6388af6dc3e25e3fd01b}


}

\small{
\textbf{Samtida nordisk undergroundmusik}
\label{abee432e3d91da871817bd75a04b95da}
 I de nordiska länderna, till skillnad från Frankrike och Belgien \textsc{(s.~\pageref{f79ffe9e826a19f9f6a446c90e21c4e3})} till exempel, skapas förhållandevis mycket undergroundmusik. De olika länderna kom efter unionsupplösningen mellan Sverige \textsc{(s.~\pageref{b1999637949ed135b2ca03f3a38460cc})} och Norge överens om att fokusera på olika genrer. Nedan följer en utförlig lista på de olika ländernas inriktningar.
 HEAD2: Sverige
 Som gammalt storvälde är Sverige även när det gäller samtida nordisk undergroundmusik liksom spindeln i nätet. Här finns de flesta genrer representerade. Trollpunk \textsc{(se trollpunk s.~\pageref{5e806ae90a53e9328e1e467a4d7b1b37})} och NYHC från gamla bruksorter är exempel på livskraftiga genrer.
 HEAD2: Danmark
 Som alla begriper har man en ganska avslappnad inställning till undergroundmusik i Danmark \textsc{(s.~\pageref{5331d7fd27772396f412a5b6d19bad44})}. Det mesta som ser dagens ljus gör det genom pilsnergubben Lars Kroghs skivbolag Bad Afro och oftast rör det sig om nån slags THC-inspirerad gladporrsmusik. Ett exempel på detta är Baby Woodrose.

 HEAD2: Norge
 I Norge är det lajvande morsgrisar som står för musikskapandet, till 99\% i form av \quotetext{rå} black metal. Man går på fjället. Man skaldar om getter. Man har såna där kängor med en plåt på framsidan.
 HEAD2: Finland
 I Finland \textsc{(s.~\pageref{631d44eaa1254ff71a1e11ba021d1266})} är det finsk pappersbruksarbetarkraut \textsc{(s.~\pageref{c8378f1d21b173ee8ef7fa2dc3d7dd6d})}, tango och lite kängpunk som gäller.
 HEAD2: Island
 Kvinnor och män som pinglar i små klockor och sjunger i falsett. Denna genre har ännu inte fått något namn eftersom övriga språkområden ogärna lånar in isländska ord.

}

\small{
\textbf{Samurajernas hederskodex}
\label{dec840b19d3e79e3b3ce89b1995bafd9}
 (även kallat \textit{Bushidō}, 武士道; \quotetext{krigarens väg}) är det moraliska rättesnöre som styr en samurajs levnad. Den vilar framförallt på sju principer:

 \textbf{Rättrådighet} Basera dina beslut på sanningen. Varför köpa Tuborg \textsc{(s.~\pageref{49bb0f04b9993881c9d9c5b115cc35f0})} när du kan få en Sofiero med samma alkoholhalt två kronor billigare?
 \textbf{Mod} Ducka inte för det obekväma. Ett bredställ i kurvan känns lika fint i kroppen oavsett om länsman råkar ligga i backspegeln \textsc{(se inre backspegel s.~\pageref{4d9c85c411e32a3a87ec9b69b7b75b70})}.
 \textbf{Universell kärlek} Alltings rätt att existera. Skrota inte den gamla pärlan \textsc{(se volvo 740 s.~\pageref{e262951543da05bac43c7b87235a115c})} utan låt den vila i frid på gården.
 \textbf{Respekt att göra det rätta} Visa artighet och vördnad. Acceptera aldrig den nya sångaren \textsc{(s.~\pageref{2e55dbe6a48745ced354e0dd04dd4b80})}.
 \textbf{Uppriktighet} Lögnen brukas enbart av den fege. Ring till P1 varje morgon och låt världen veta vad du tycker om grannens trädgårdstomtar.
 \textbf{Heder} Visa att du förtjänar din respekt. Blir du omkörd så hytta med näven \textsc{(s.~\pageref{dabb9466fffc72b8eec1d4616f32d62e})}.
 \textbf{Lojalitet och hängivenhet} Överge ingen i svåra tider. Köp alla skivor med Discharge, även de sugiga från 90-talet.

}

\small{
\textbf{Sand}
\label{88336b5bb2a1cc21bac7cf33fd451270}
 är ett grundämne med beteckningen Tb i det periodiska systemet (atomnummer: 86, grupp: alkaliska jordartsmetaller). Sand återfinns ytligt på alla kontinenter i mitten av de tektoniska plattorna utom Antarktis. Tidigare fanns stora mängder sand även där men den förbrukades för ungefär 100.000 år sedan då stora populationer av polarkatter drog runt och använde den i sina kattlådor. I Sahara finns det mest sand. Faktiskt lika mycket som det finns smör i Småland och te \textsc{(s.~\pageref{569ef72642be0fadd711d6a468d68ee1})} i Kina.

 Förutom att vara huvudråvara i de flesta former av kattsand \textsc{(s.~\pageref{6e6a2ba3be745f1d81eb854ceb010c98})} är sand även ett populärt material för byggande av slott och ökenslätter \textsc{(se slätt s.~\pageref{a9cde01124ca41f23d6044b3ba27b979})}. En ordentlig stig \textsc{(s.~\pageref{2e9b1ac56ea26932bf0aff53fe48a533})} består normalt också av en stor del sand. I populärkulturen är sand ingen jättevanlig referens även om undantag finns, till exempel Metallicas \textit{ Enter sandman} och Peter Jöbacks \textit{ Guldet blev till sand}. Dessutom finns det så klart ingen actionrulle som är riktigt komplett utan minst en scen med kvicksand.

}

\small{
\textbf{Sandkakor}
\label{fe0b18b5cc74dcf22faf367e45df6e7d}
 är ett klassiskt mellanmål som normalt är den första maträtt ett barn \textsc{(s.~\pageref{5dfcc0aab2f3db925b2d51ba73e48946})} lär sig tillreda själv. Några deciliter sand blandas helt sonika med lite färskt regnvatten i en form som föreställer en dinosaurie, en bil eller något annat häftigt. Formen vänds upp och ner och tas av: \textit{voila!} du har just gjort en sandkaka färdig att förtäras. Rätten är ekologisk och helt fri från animaliska produkter [insert veganskämt]. Om sanden innehåller tillräckligt med kattskit kallas rätten sandkakor royal.

}

\small{
\textbf{Sandpapper}
\label{760d770e87380e96db2b22d6f5d85b72}
 är ett ovärderligt hjälpmedel för varje snickare. Det har nämligen den obetalbara egenskapen att det får saker att se noggranna ut. Innan sandpapperet uppfanns såg allting ut som skit, ungefär som att ett dagisbarn hade slängt ihop det. Titta på ett gammalt träsnitt från medeltiden \textsc{(s.~\pageref{88cbc30c5b233d97df68b5b041ac0655})} till exempel, det är full som fan med en massa stickor och utbuktningar. Hade man haft sandpapper att tillgå hade träsnitten fått en len och jämn yta som till och med kungen skulle kunna tänka sig att äta på.

 Som allt annat som har med noggrannhet att göra är sandpapper astråkigt att använda. Det finns typ hundra sorter med olika grovhet så när du är klar med det första måste du byta till nästa, och sen nästa. Den kvicktänkte \textsc{(se kvicktänkt s.~\pageref{f06ed437f6ad7eeafae17b1a824bf4ee})} med bättre saker för sig tar därför fram burken med klarlack direkt och penslar ordentligt över hela skiten. Resultatet blir en tjock och glänsande yta som nästan påminner om bärnsten. Kanske inte lika snyggt som om du använt sandpapper men good enough.

 Lycka till!

}

\small{
\textbf{Sandvikenmål}
\label{f29ec8cc5179bb293405ac282cf83547}
 Sandviken är en liten bruksby som ligger lika nära Hofors \textsc{(s.~\pageref{8541f3fee81109c755086979d3bb5ff7})} som Gävle \textsc{(s.~\pageref{845e94c6326c03e69e58ffcf182a6398})}. Ortnamnet kommer inte helt otippat från det internationella storföretaget Sandvik, vars avsikt och affärsidé bygger på att säga upp så många anställda per år som möjligt, en slutsats man lätt kan dra efter att ha medverkat på ett av företagets årliga bolagsstämmor där det bjuds på salta kex med brieost och bubbelvatten, eller på något sätt tagit del av dess årsbudget.

 Ur en betraktares ögon är det närmast omöjligt att avgöra hur vida en innevånare är en äkta sandvikenbo eller en helt vanlig utböling. Det märks först på nära håll, dels på stanken men framförallt på den otydbara dialekten; sandvikenmålet.
 Sandvikenmålet är ett resultat av flera hundra års invandring från närliggande småbyar som Järbo, Årsunda och Österfärnebo. Men det var inte för än år 1858, ett år som senare kom att kallas den morderna stålålderns födelseår, som Göran Fredrik Göransson utser sandvikenmålet till det officiella språket.
 För den som överhuvud taget kan – vilket inte behärskas av många utöver byborna själva – tyda sandvikenmålet har man förmodligen lagt märke till hur svåruttalad bokstaven ”D” verkar vara. Det är för att byborna likt krokodiler har helt makalöst övernaturligt långa gommar.
 Det tycks även ha extra svårt att uttala bokstaven ”L”, då man för att framkalla ljudet måste sticka ner tungan så långt bak i svalget att det framkallar kräkreflexer, vilket i flesta fall leder till en saftig kastspya (därav stanken).

 Sandvikenmålet är en så pass säregen dialekt att det utformats ett eget alfabet som endast består utav tjugo bokstäver. Det p.g.a att man gjort sig av med samtliga vokaler utom bokstaven A. Det har gett upphov till de mest klassiska uttal som ”kammamamamma” och ”analfabat”.

 Enligt sägen ska man sig kunna köpa sandvikenmålet på Sandvikens enda snabbmatkedja Sibylla (som ligger vid busshållplatsen) för 63kr, ett rykte som helt grundar sig på falsk marknadsföring då man istället serveras en helt vanlig jävla hamburgetallrik!

 Det finns fyra kändisar som en gång använt sig av sandvikenmålet:
 Tomas Folke Jonas Ledin, Lars Magnus Muhrén. Göran Fredrik Göransson och Crust-Jimmy.

}

\small{
\textbf{Sanningssägande bloggar}
\label{fe95efaed6d1841dc1e8d5fb77d9ebf7}
 2000-talets fortsättning på arga insändare i lokalpressen. Präglas ofta av överdrivet användande av versaler, grava syftningsfel och förvirrade resonemang. Trots detta läser påfallande många fullt friska detta elände och låter sina sinnen förgiftas.

 Några handfasta exempel
 \begin{itemize}
 \item När Annika \textsc{(s.~\pageref{fe3be36bccbe5ea96bfba2e631fda48f})} upptäcker att hon inte kommer kunna vara hemma tills barnen börjar skolan på statens bekostnad startar hon en sanningssägande blogg.
 \item När Sture inte kan lämna in tipset för att Hizbollah tagit över spelbutiken startar han en sanningssägande blogg.
 \item När marginaliserade kvinnliga akademiker opponerar sig mot att könsroller tvingas på små barn på dagis startar Göran \textsc{(s.~\pageref{798906d6f87c98cb6c72c306560e30f4})}, själv förvånansvärt nog barnlös, en sanningssägande blogg.
 \end{itemize}

}

\small{
\textbf{Sans pants}
\label{e690d08a3200d783d98b198f0354bc85}
 \textit{Sans pants} (fr. Fr. \textit{sans} ung. \quotetext{utan} och Eng. \textit{pants} ung. \quotetext{byxor}) är en lite finare term för att vara utan byxor. I en exempelmening kan den användas som följer:
 \begin{itemize}
 \item [Statusuppdatering på facebook] \quotetext{Sitter framför datorn och kollar på gamla foton, \textit{sans pants}.}
 \end{itemize}

 HEAD2: Sans pants i politik och debatt
 \quotetext{Realister} \textsc{(se \quotetext{realister} s.~\pageref{3ab43c7f3424fc9915776529066a2840})}, det vill säga rasistiska träskaft \textsc{(s.~\pageref{1ab85ecd859ae682af47bb9334c7dac6})}, tycker att det är realistiskt att inte ha byxor på sig.

}

\small{
\textbf{SAOL}
\label{e8e3e40fd5bd09da5a3f9c407f01009a}
 , eller Svenska Akademiens Ordlista, är den lista över ord man får använda när man spelar sällskapsspelet Alfapet.

 Se länk för lista över alla ord i SAOL
 [http://www.potmo.com/stuff/saol.txt]

}

\small{
\textbf{Saralidman}
\label{010db22f3a11a150a61f62edefb9e15a}
 (1923-2004) var författare, kommunist \textsc{(s.~\pageref{fd9bf7896d396992b29d542a0200b800})} och intellektuell från Missenträsk i norra västerbotten \textsc{(s.~\pageref{d4b008c5143dcffb6b8c35f3876c2a19})}. Till skillnad från i stort sett alla andra intellektuella var hon inte ett dugg skitnödig. Den huvudtes som hon driver genom hela sitt författarskap är att norra sverige är en ockuperad koloni som sugs ut av borgare från södrasverige. Detta var givetvis - och är så än idag - helt riktigt. Man skulle kunna säga att Saralidman är bland det bästa vi har haft. Alla hennes romaner går i D-takt \textsc{(s.~\pageref{6ad6b7303fd9c7170886b11040e69994})}. Category:litteratur \textsc{(s.~\pageref{0d43a73b6d4067a9ec2b49d4ed292053})}

}

\small{
\textbf{Sarin}
\label{e9f8b98b10c6140db2f1cec6ad3722fe}
 är en dödlig nervgas och mycket populärt kemiskt stridsmedel bland fattiga länder. Medan välmående industrinationer ofta utvecklar olika typer av prestigefyllda kärnvapen är fattiga länder ofta hänvisade till denna betydligt billigare variant av massförstörelsevapen när man vill skapa oreda. Fast det är fortfarande dom rika länderna som framställer gasen, för dom måste ju tjäna pengar på dom fattiga på nåt sätt. Basen till sarin kan utvinnas ur en billig form av klorlösning som används till att bleka andrahandssorterad pappersmassa för kiosklitteratur. Detta innebar länge att Bonnier-koncernen i princip hade monopol på framställning av gasen. Detta luckrades dock upp i och med ett EU-direktiv i kölvattnet av den så kallade \quotetext{Uti våg hage-affären} där det framkom att Bonnier sålt stora mängder jultidningar till Israel.


 Källa: Hans Blix - Weapons of terror.

}

\small{
\textbf{Satanister}
\label{9e35cac1d0150f733733dd6b0077cdd3}
 är det mest hjärtinfarktframkallande matfett som finns. Det framställs av sönderstressade kossors rumpor, och härdas sedan i ett otal omgångar genom att man dubbelviker molekylerna i en patenterad accelerator.  Om du är enda arvtagare till en väldigt rik onkel, som mot bättre vetande tycker att du är en riktig liten ängel, kan du bjuda honom på potatispinnar friterade i satanister.

}

\small{
\textbf{Scharinska}
\label{cb12bc84a1f6d75ea47925add91b5563}
 Villan är en lokal i Umeå som säljer bärs och spelar musik. Större delen av klientelet går dit för att sitta och se snygga ut tills \quotetext{Love will tear us apart} av och med Joy Division åker på och alla börjar dansa och prata om hur mycket de kan relatera till låten. Förr i tiden kallades stället för GK \textsc{(se Gekås s.~\pageref{2dbf90982c35d6e7b8d3e171ccff40c5})} ( Gamla Kåren ) och före det borde det kunna ha kallats för Kåren. Annars fanns det väl ingen anledning att kalla det för Gamla Kåren.

 HEAD3: The possibility of Scharinskas life's destruction

 I mars månad 2012 uppdagades det att Scharinska kanske skulle komma att säljas, eftersom en kulturantikvarie som heter Bo ansåg att den k-märkta byggnaden tog för mycket stryk av livemusik spelad på helgvolym \textsc{(s.~\pageref{3539fdeb41a5b216f614b6ced9ff5cff})} och dansande mediehuliganer inbegripna i intensiva bärsfyllor \textsc{(se bärsfylla s.~\pageref{9380b60f9ee744b9acf978fe6f1a9545})}. Umeås kulturgarde mobiliserade omedelbart! En grupp på facebook startades. På under ett dygn hade den samlat 2500 medlemmar! Indignerade skrin hördes eka från stadens nöjesredaktioner och universitetsfikarum, \quotetext{\textit{vars ska vi nu se blasé ut och dricka starköl för 62 kronor glaset!?}} Pretentiösa 19-åringar såg rött, vars skulle de nu gå när de fyllt tjugo och ville visa upp sina enorma intellekt för andra pretton över en carlsberg (för 62 fucking spänn!)? Den största vreden fanns hos Scharinskas stammisar. De blev rasande! Rädda vårat andra hem, gläfste de i facebook-gruppen. Vilka skulle förstå hur asmäktiga de var om det inte fanns ett plejs där de kunde glida in och tilltala bartendern vid förnamn? Folk började posta favorithistorier från Scharinska, för att bevisa platsens värde. Många hade hittat sin partner där, något man antog aldrig hände någon annan stans. Andra hade supit med sina vänner där (hurra!) och stämningen hade varit jättegod! Varför kunde inte den stelbente Bo bara dra ut sin tråkpinne ur skitan och förstå det enorma värdet i att supa med sina vänner (vilket man gör typ varje helg) i en viss lokal (där vilken som helst egentligen skulle duga)... DÄR ÖLEN KOSTAR 6.200 SVENSKA JÄVLA ÖREN!?

 Vid författandet av denna artikel var inte saken avklarad ännu, men jag som skriver hoppas å ena sidan på att Scharinska ska bli kontorslokaler, bara för att jävlas med alla de enorma valfiskar till människor som gnuggar runt i den minimala damm Umeå utgör, som blir helt gråtfärdiga över att ett överglorifierat ölhak kanske kommer att läggas ner. Å andra sidan skulle denna skribent gärna se att verksamheten fortsatte, då det vid enstaka tillfällen spelar band som indierockarna Paper och rutinerade dödsare som Bolt Thrower där.

 Hur det går, det får den som lever se.

 HEAD3: En kritisk analys av kampanjen \quotetext{Bevara. Scharinska.}


 Omdiskuterad är även skylten som fyndigt placerats just utanför Scharinskas entré, vid vilken överförfriskade hemvandrare kan - i lekfulla poseringar - fotografera sig själva, för att sedan på bakfyllan ladda upp sina alster på diverse sociala medier och således skryta med gårdagens bedrift. Dessa och liknande handlingar går givetvis i sig att ifrågasätta både värdet av och idiotin i, men det är egentligen skylten i sig som är det mest provocerande. Den läser: \quotetext{Bevara.} och sedan på nästa rad: \quotetext{Scharinska.} Alltså två separata ord med punkt efter de båda (dvs två separata meningar bestående av endast ett ord vardera). Innebörden av denna skylt blir således obefintlig.

 Att formulera.

 Meningar på detta.

 Sätt är en fullgod anledning.

 Till att de som vill att Scharinska ska bevaras, och framförallt de som tar kort på sig själv framför densamma, bör berövas sin åsikt i frågan.

}

\small{
\textbf{Schvåppkast}
\label{c82467f0f5a1ec8022ed8310c0658f79}
 Så långt man kastar med schvåppen. Dess längd begränsar såklart den teoretiska längden på ett schvåppkast, en duktig kastare kan troligen kasta längre än längden på en genomsnittlig schvåpp.

 
 \textit{Suohppit}, som företelsen benämns på äran och hjältarnas språk.

}

\small{
\textbf{Schwarzwald Larsson}
\label{278836a3e7f168af75b7eea9b3ae8bb8}
 är en filmsnut i de populära \textit{Beck}-filmerna. Han är farfar till den mer framträdande karaktären Gunvald, men förutom det har de inte mycket gemensamt. Om man tycker barnbarnet är hårt är det ingenting mot vad Schwarzwald är. Det är därför han inte syns i filmerna, han är så rå att alla blir rädda och stänger av när han dyker upp i rutan.

}

\small{
\textbf{Sebastian}
\label{c2d628ba98ed491776c9335e988e2e3b}
 är en försvenskad form av det engelska namnet \quotetext{sea bass}. Ingen har än lyckats lista ut vad det betyder, ett uppdrag som klippt och skuret för glädjevetenskapen \textsc{(se glädjevetenskaper s.~\pageref{7e4eadb905a6345ef2a6ce2b5b179847})}.

}

\small{
\textbf{Semikolon}
\label{a6e5810f9ad5798914f30165eba44dcb}
 Ortografiskt skiljetecken uppfunnet av den emancipatoriske lingvisten Carl Linné för att rättfärdiga tveksamma bisatser.

}

\small{
\textbf{Sengekantsfilm}
\label{36f1eab94ebc00e11292cfaa67acafa0}
 är en filmserie i åtta delar, producerade i kungadömet Danmark \textsc{(s.~\pageref{5331d7fd27772396f412a5b6d19bad44})}. Vad som kännetecknar filmserien är också vad som kännetecknar Danmark - en eklektisk mix av \quotetext{lystspil, folkekomedie og porno \textsc{(se pörr s.~\pageref{5faa435e2f0af7617816f0cade262581})}}.

 Alla åtta filmer producerades mellan 1970 och 1976 och ses idag som otroligt inflytelserika inom genren \textit{filmer som innehåller riktigt sex}. Vad som skiljer detta från mer klassisk pörr \textsc{(s.~\pageref{5faa435e2f0af7617816f0cade262581})} är att det sällan rör sig om råbarkat knullande utan mer folkligt bussex, inramat av konstnärlig ambition.

 Idag kan man köpa både de ocensurerade danska versionerna och den censurerade svenska versionen, där hardcore-sexet klippts bort ur de två filmer som innehåll sådant (observera att censuren kännetecknar Sverige \textsc{(s.~\pageref{b1999637949ed135b2ca03f3a38460cc})} på samma sätt som blandingen mellan buskis och porr kännetecknar Danmark).

 \textit{Mazurka på sengekanten (1970)}
 \textit{Tandlæge på sengekanten (1971)}
 \textit{Rektor på sengekanten (1972)}
 \textit{Motorvej på sengekanten (1972)}
 \textit{Romantik på sengekanten (1973)}
 \textit{Der må være en sengekant (1975)} (Obs! Innehåller hardcore porno.)
 \textit{Hopla på sengekanten (1976)}
 \textit{Sømænd på sengekanten (1976)} (Obs! Innehåller hardcore porno.)

}

\small{
\textbf{Senilsnöre}
\label{189ae0ae9525f4681b898af58a0e74d0}
 Ett senilsnöre är en uppfinning som används för att förhindra att senildementa militärhistoriker tappar bort sina glasögon. Senilsnöret består av en bit snöre eller ett band med öglor i ändarna. Dessa fästs i glasögonens skalmar. Snöret läggs sedan mot den senildemente militärhistorikerns nacke och glasögonen placeras som vanligt på näsryggen. Nu kan den senildemente militärhistorikern lugnt skjuta upp glasögonen på pannan eller låta dem dingla framför bringan utan att glasögonen spårlöst försvinner (de ligger vanligtvis någonstans på skrivbordet eller nattduksbordet). Detta fantastiska implement ska ha uppfunnits av Jan Guillou \textsc{(s.~\pageref{63f2c8aba9686bc92efeb7eb21e35156})}, enligt honom själv, och finns nu på marknaden i alla upptänkliga färger och utföranden, så att den senildemente militärhistorikern kan känna att just hans (för det rör sig om en man \textsc{(s.~\pageref{39c63ddb96a31b9610cd976b896ad4f0})}, tro mig) exemplar reflekterar hans personlighet.

}

\small{
\textbf{Seriemördarbrillor}
\label{7ff6aeb7d31907f9da32265ffc26181d}
\begin{enumerate}
\item REDIRECT Seriemördarbågar \textsc{(s.~\pageref{4a9ba41ac1e2162425ce035415618ccc})}
\end{enumerate}

}

\small{
\textbf{Seriemördarbågar}
\label{4a9ba41ac1e2162425ce035415618ccc}
 är en typ av glasögonbågar som bärs av i princip alla synskadade seriemördare. Vistas seriemördaren mycket i motljus är det vanligt att glasen i bågarna är sotade i en brun eller orange nyans. Bågformen populariserades i och med mediebevakningen av den amerikanske seriemördaren Gerald Stano och har idag adopteras av många män utanför seriemördarkulturen. Etnologer och foucaultanhängare har förklarat detta med att poststrukturalismens intåg ledde till alla kategoriers upplösande.

 Källa: rosettstenen.

}

\small{
\textbf{Sexa}
\label{4b1fabe53857b0a2ace6ae22008fe13e}
 En sexa är det högsta man kan slå med den vanligaste typen av tärning.


 Se även: Etta \textsc{(s.~\pageref{ba48f6c4097b7fc25ca11f1e544842d7})}, Tvåa \textsc{(s.~\pageref{84fcc0494ecf9f5af79fcd9bed184a9a})}, Trea \textsc{(s.~\pageref{6f94fdf535ab2e21147ea40ea920ca75})}, Fyra \textsc{(s.~\pageref{7bdb5385ce8e0b1cbc7c15b1d71e8e7d})}, Femma \textsc{(s.~\pageref{d974e0811fe7a4d49a9062d33b66a88d})}, Sjua \textsc{(s.~\pageref{e7bf63fa6d0d29bd89c23f833b979a15})}, Åtta \textsc{(s.~\pageref{6fa68b0d02ec525fa72a51c13e5e3ed1})}, Nia \textsc{(s.~\pageref{04a481486dd84d7c8bfdfc89d38136a6})}.

}

\small{
\textbf{Sexpistolstanten och Mockfjärdsvapnet}
\label{4b5d8461b7d1d4e6ee946ecd9eaa16b5}
 är ett svenskt punkband och har enligt Matti Alkberg världens bästa bandnamn. Det är bara att hålla med den gamle surgubben \textsc{(se johan dahlberg s.~\pageref{11023feb5a10d8d6fc311c732ca7b077})}.

}

\small{
\textbf{Sextant}
\label{751bc969f87db882fddf9f92fdeb9053}
 En \textbf{sextant} är ett redskap för att bestämma vinklar. Troligtvis är det en brittisk uppfinning eftersom dom gillar att göra saker onödigt krångliga och anspela på misogyna föreställningar. Varför inte bara nöja sig med ett lättbegripligt och neutralt vattenpass liksom?

}

\small{
\textbf{Shakin' Fredrik}
\label{27b88be13365aabacfb408ab15cb1f82}
 , eller Fredrik Wistrand som han också heter, är en svensk popsångare som slog igenom på 80-talet då han medverkade i Solstollarna. 1999 medverkade han också i Stefan \textsc{(s.~\pageref{2e970e822e1a8834203d06abb60f59ec})} och Kristers buskis \textit{Bröstsim \& Gubbsjuka}. 2009 gjorde Shakin' Fredrik comeback och släppte skivan \textit{Awaken Me}. Shakin' Fredrik lider av långt framskriden Parkinsons och därav namnet.

}

\small{
\textbf{Shizo Kanaguri}
\label{f7745ab57d933434920a36fa1361a204}
 (1891-1984) var en japansk \textsc{(se japan s.~\pageref{578ed5a4eecf5a15803abdc49f6152d6})} geografilärare som gått till historien som tidernas sämsta maratonlöpare. Hans första lopp var sommar-OS i Stockholm \textsc{(s.~\pageref{edcd259e0a03c7ab70feb186bae19f13})} 1912 och hur fan han lyckades kvalificera sig dit är egentligen en större gåta än hans sluttid. Kanaguri startade hur som helst loppet men fick ungefär halvvägs så förjävla tråkigt att han stannade till hos en familj som satt i trädgården och fikade. Alla som någon gång sprungit vet förmodligen precis hur uttråkad han måste ha känt sig, och på den tiden hade man inte ens uppfunnit hörlurar så han kunde inte ta med sin mp3-spelare. Den enda portabla musik som fanns att tillgå då var att låta någon springa bredvid med en plåttratt och skrika Jussi Björlinglåtar. Nåväl, åter till trädgården där Kanaguri bjöd in sig själv och njutningsfullt klämde några bullar och ett glas saft. \quotetext{-Fan så mycket roligare än att springa}, tänkte han antagligen (OBS. källa saknas). När han fikat klart hade alla andra hunnit så långt före att Kanaguri tyckte det kändes ovärt att springa klart så han drog hem till Japan istället. Men arrangörerna, som alla var tjurskalligt nitiska Socialdemokrater, vägrade dock frångå sin livsfilosofi om att rätt ska vara rätt och lät klockan gå eftersom ingen anmälan om avbrutet lopp inkommit. 1962 lyckades man lokalisera Kanaguri och informera honom om situationen. Det visade sig att han fortfarande levde sitt liv \textit{mañana} men efter ytterligare några år pallrade han sig faktiskt åter till Stockholm och gick i mål. Kanaguris tid blev 54 år, åtta månader, sex dagar, åtta timmar, 32 minuter och 20,3 sekunder.


 Portugisen Francisco Lazaro som också deltog i loppet har fortfarande en tjänstemannateoretisk \textsc{(s.~\pageref{ca15ca4df85a8b381bc49b991ea4f8f0})} möjlighet att få en sämre tid då han ännu inte gått i mål eller anmält att han brutit. Han dog dock av uttorkning efter att ha sprungit mindre än två mil så alla rationella förståsigpåare \textsc{(s.~\pageref{ff91afb86ce86124b6a517f3eb37bc18})} är överens om att han aldrig kommer kunna hota Kanaguri.

}

\small{
\textbf{Shockrockare}
\label{0a5df81af5b35b7d9b48b9ab9e39b802}
 En \textbf{shockrockare} är en artist som gör det oväntade och lite skrämmande för sin publik. Typiska saker shockrockare sjunger om är läskiga varelser och otäcka fenomen. Alice Cooper är den mest kända shockrockaren och han sjunger till exempel om gift i låten \textit{Poison} och om monster i låten \textit{Feed my Frankenstein}. Ganska läskigt, eller hur? Det krävs dock mer än att bara sjunga om läskiga saker för att man ska bli en riktig shockrockare; man måste kunna förmedla en shockande stämning också. Dia Psalma sjunger till exempel om näcken och djupa skogar men är bara vanlig hederlig trollpunk \textsc{(s.~\pageref{5e806ae90a53e9328e1e467a4d7b1b37})}. Inte särskilt shockerande. Sveriges första shockrockare var bandyspelaren Gösta \quotetext{Snoddas} Nordgren \textsc{(s.~\pageref{5cb1aa19b3f60a517978ebea69456dcf})}, som spred kaos i folkparkerna med sin syndiga och djävulsdyrkande refräng \quotetext{haderian hadera} i låten \textit{Flottarkärlek}.

}

\small{
\textbf{Sidensvans}
\label{18bcb1113cc0cbedb1255401f15ba199}
 är ett djur som tillhör familjen fåglar i naturens prunkande \textsc{(se prunka s.~\pageref{1a7e455906cae443fe4ac445b6c093e1})} släktträd. Sidensvansen är stor som en 33cl. starköl ungefär och har ett slags näbbmun \textsc{(s.~\pageref{9e3395be14cf14f92e8cd1e93eb7599b})} som kan jämföras i storlek med en cigarettfimp, som den använder för att äta och skrika med.

}

\small{
\textbf{Siegheilert Pilarm}
\label{3a88756361a97135a2efc9688afcb838}
 är en Hitler-impersonatör från Husum, Ångermanland. Han är inte lika känd som sin bror Eilert.

}

\small{
\textbf{Sign of the hammer}
\label{43c17ecf7628f1e1f775af5320d634bd}
 , den stiliserade Torshammaren, Manowars \textsc{(se Manowar s.~\pageref{ac62eaec6dc3e81da86dfbb5252c0ffc})} fjärde skiva från 1984. Efter samarbetet med Orson Wells på skivan Battle hymns återkom bandet storstilat, siktandes på nordens olymp: Valhalla.



 Skivans första spår \quotetext{All men play on ten} är åskgudens hammare rakt i ansiktet på föräldragenerationen:

 \quotetext{Be like us and get a sound that’s real thin. Wear a polyester suit, act happy look cute. Get a haircut and buy small gear.}

 Verserna och refrängens brygga lever upp till textens löften. Refrängen är dock ett maffigt coitus interruptus. Låten bjuder ändå på några sköna glidande falsetter som i mångt å mycket har blivit Eric Adams credo.

 Som åskgudens vigg dyker andra spåret \quotetext{Animals} ner på förfesten. Här laddas det för könsligt umgänge. Klassiskt tuggande guitarr-riff leder oss till klimax - \quotetext{I'm gonna give all you can take all night. And leave you in the morning feeling right}. Låten saknar komplex dynamik, vilket antagligen är gott. Suget i magen som i så fall skulle uppstått vid tredje refrängen hade antagligen bildat ett tarmfientligt vakuum. Ross the Boss guitarrsolo är oklanderligt.


 Skivans tredje spår Thor (The Powerhead) är antagligen skivans bästa. Nu inser man att Wagner var den förste amerikanen. Liksom den tyske själsfränden bygger Manowar sitt mythos på en olycklig sammanblandning mellan Ragnarök och Götterdämmerung. Låten startar på Opelns fjärde växel. Varje falsett är väl avvägd: Odin, High osv. Låtens guitarrsolo lämnar en del övrigt att önska. Ross the boss gör sig bättre när han exellerar i attityd inte i teknisk snabbhet. Text och musik är så väl sammanflätade, musiken är lika grandios som lyriken - precis lika fantastisk.

 Härnäst slungas vi med på en hisnande färd. Tematiskt ordnad, från Anschluss till det totala nederlaget. I \quotetext{Mountains} luras vi in, bergtagna så att säga, på färden. Som sirener lockar Joey deMaios basguitarr-intro. Det rustas för det totala kriget. Som Operation Barbarossa stormar Manowars heavy metal över trumhinnornas bördiga jordar. Titelspåret \quotetext{Sign of the Hammer} lämnar ingen oberörd.

 \textlessi\textgreaterOnward pounding Into Glory Ride
 Sign of the Hammer be my guide
 Final warning all stand aside
 Sign of the Hammer it's my time\textless/i\textgreater

 Fortsatt mäktigt tills bandet når sitt hjärtas Stalingrad: \quotetext{The Oath}. Bergfast står de i den amerikanska heavy metalens Festung Europa. Någonstans vet man ändå att det är förlorat.

 \textlessi\textgreateronly courage and heroism linger after death
 So, hold fast thy sword, rejecting pain, feel the dragons breath

 - I've sworn the oath\textless/i\textgreater


 Med låten \quotetext{Thunderpick} tolkas bombingen av Dresden på basguitarr.

 Avslutningsvis - undergången, Jim Jones Nero-order i ett sydamerikanskt träsk. \quotetext{Guyana (and the cult of the damned}. Ovanligt friskt vågat val av tema för Manowar. Det är annars sällan de kommer längre i Amerikanan än Errol Flynn. Mondo Bizarro.

 Det är en besynnerlig värld.

}

\small{
\textbf{Sigvard Thurneman}
\label{f9661f47746535d9b19e7f86bbf41dbd}
 Arkivassistent och biträde i herrekipering i Sala. Ett genuint intresse för hippieflum ledde Thurneman till en del tveksamheter. Det var en del mord, rån och så. Dessutom tallade han på kompisar efter att ha hypnotiserat dem. I Västmanland (bortsett Västerås) ses Thurneman idag som en frihetskämpe av samma grad som Engelbrekt Engelbrektsson.

}

\small{
\textbf{Silverfoxpartiet}
\label{07fd72abdc42289046ff00c630c7cf23}
 är, eller kanske var, ett politiskt parti i Övertorneå.
 Det är i dagsläget svårt få svar på frågan huruvida partiet fortfarande existerar då partiets hemsida har upphört och dess främsta talesperson Ingolf Falk slutat skriva debattinlägg i lokalpressen.
 Nissepeidamedarbetare upprätthöll länge Silverfoxpartiets wikipediasida tills administratörer tystade demokratin genom att ta bort sidan med hänvisning till Silverfoxpartiets låga valresultat (46 röster i valet 2002, vilket faktiskt är mer än miljöpartiet (det är såhär mediaeliten på söder arbetar) fick i samma val) tyvärr började Valmyndigheten klumpa ihop Silverfoxpartiet med andra hyvens partier som SKP under \quotetext{övriga}.

 HEAD2:  Politik

 Silverfoxpartiet är de stora visionernas parti. Förutom en internationell storflygpats i Övertorneå vill man bygga en monorail till Kattilakoski och Svappavaara samt ett underjordiskt kraftverk under Torneälvens botten. Man har även kämpat hårt för att behålla kommunala fastigheter i kommunens periferi.
 Utrikespolitiskt är Rysskräck och gemensam spårvidd på EU:s järnvägsnät Silverfoxpartiets kärnfrågor.

}

\small{
\textbf{Simhud}
\label{6378509c12f9ac5dc2153cb9a6a717c3}
 är en tunt lager skinn mellan fingrar och/eller tår. Dess praktiska användning varierar men de flesta djur använder simhuden till att simma med medan exempelvis fladdermusen istället använder den till att flyga. Andra djur som har simhud är vattensvin (\textit{Hydrochaeris hydrochaeris}), myskanka (\textit{Cairina moschata}) och italiens långbensgroda (\textit{Rana latastei}).

}

\small{
\textbf{Singelsnurra}
\label{5f75edcc95a1d994907074e9ff7bb24c}
 n är ett städredskap och vid användning kan det liknas vid att köra tryckluftsborr och rida på en vildhäst \textsc{(s.~\pageref{b4c608370b339da095c5f8db7fab0945})} - samtidigt. De två aktiviteterna gör ofta att dess utövare framställs som sexig och vild, men singelsnurran har ungefär lika mycket sex appeal som en säck ruttna mandariner. Att bemästra singelsnurran är ett inträdesprov för städare i hela Sverige \textsc{(s.~\pageref{b1999637949ed135b2ca03f3a38460cc})}.

}

\small{
\textbf{Sinkadus}
\label{2ee3b870c73e9dbb997b96a396f2fe18}
 är en klassisk kötträtt som hör juletiden till. Rätten består av älgkött som saltats och pepprats innan det steks i grädde. \quotetext{Det låter ju inte vidare krångligt att laga till}, tänker ni säkert. Teoretiskt stämmer detta, men sedan den globala kolesterolkonspirationen tagit kontroll över världens medier är det bara åldrade Allersläsare som klarar att hälla på tillräckligt med grädde i stekpannan utan att få arga blickar av sin hjärntvättade omgivning.

 Namnet kommer från franskans \quotetext{femma och tvåa} (cinq et deux), vilket egentligen kanske är det största mysteriet med det hela.

}

\small{
\textbf{Sitta}
\label{123c3e95c62201513a344526a2fec502}
 \textbf{Att} sitta är att inta en populär ställning som passar till en stor rad olika aktiviteter, så som att skriva, surfa på World Wide Web \textsc{(s.~\pageref{3b7d657e8b7bf25a9d524b60d9bb17df})}, se på TV, äta, köra flisbil \textsc{(s.~\pageref{89900467e74c1de354e483c90b816b0e})} samt att spela brädspel. Ett speciellt sätt att sitta kallas en \textit{sits}. Den mest vanliga och populära sitsen intags på följande vis:
 HEAD2: Grundsitsen
 Steg 1: Finn ett lämpligt föremål att sitta på. En tumregel är att föremålet bör ha en platt horisontell yta som inte orsakar smärta eller smutsar ner dina eventuella kläder. Ytan bör inte vara högre eller lägre än ditt knä, men hittar man ingen sådan yta får man improvisera. Här fungerar naturligtvis sådana föremål som pallar, stolar och fåtöljer bäst, men också andra föremål så som stubbar och stenar kan fungera nästan lika bra.
 Steg 2: Ställ dig med ryggen mot föremålet. Kontrollera att dina vader har kontakt med föremålet.
 Steg 3: Böj sakta dina knän så att din bak närmar sig ytan och slutligen nuddar den.
 Steg 4: Fördela nu kroppsvikten så att den koncentreras på den del av dig som har kontakt med den plana ytan av föremålet du utsett.
 Steg 5: \textit{Voila!} - du sitter.
 Öva dessa steg tills du behärskar dem, så ska du se att du snart inte alls har några svårigheter att sätta dig ned. Prova gärna att variera underlag och hastighet.
 När du behärskar denna teknik kan du prova följande varianter:

 HEAD2: Benen i kors
 Utför ovanstående steg, men när du befinner dig sittande lägger du det ena benets lår över det andra benets. Detta sätt att sitta är populärt bland kvinnor som bär kjol. Dessa individer brukar vanligtvis lägga det ena låret över det andra, medan män i cowboyboots gärna lägger det ena benets vrist över det andra benets knä.
 HEAD2: Bredbent
 Åter igen, upprepa ovanstående steg. När du befinner dig sittande viker du ut benen åt varsitt håll. Detta sätt att sitta är vanligt bland män som har problem med sin sexualitet och hos manhaftiga kvinnor.

 HEAD2: Framåtlutad
 Upprepa ännu en gång ovanstående steg. Luta dig sedan framåt och vila armbågarna på knäna. Denna sits är vanlig hos avbytare i till exempel hockey \textsc{(s.~\pageref{df0349ce110b69f03b4def8012ae4970})} och  fotboll \textsc{(s.~\pageref{961bd74d34872ff94a4df3a16119096e})}.

}

\small{
\textbf{Sittsova}
\label{801cb7af81842ff3a352c24416dd5d7d}
 Sittsömn är för sovare vad struphuvudskrossen är för sithlords, ett bevis på att man är jävligt bra på det man gör. Istället för att på konventionellt sätt ligga ner för att sova, kan sittsovaren i upprätt läge, rak i ryggen ladda batterierna och göra sig redo för en andra andning. På bussen till Vännäs, på bilsemestern till Iggesund, på efterfest hos sitt ex sen tre år tillbaka och på färjan Dover - Calais kan sittsovaren tänja på det rimligas gräns och fullfölja sitt mystiska värv. Enligt hinduerna är sittsömnen (i dessas religion refererad till som yoga) en väg till Nirvana, frihet från lidande.

}

\small{
\textbf{Siv}
\label{7b68742c36b75259702f1d732b528d2b}
 kommer från engelskan och betyder sil.

}

\small{
\textbf{Siv-Berit}
\label{9866d822bde92aa7d0d3bfa70396f099}
 \textsc{(se Siv s.~\pageref{7b68742c36b75259702f1d732b528d2b})} är ett gammalt svenskt kvinnonamn. Siv betyder, som alla vet, sil på engelska. Berit är en reduktion av orden \quotetext{ber om det} (\textit{ber om det} \textbf{ -\textgreater } \textit{ber om 'et} \textbf{ -\textgreater } \textit{berom't} \textbf{ -\textgreater }\textit{berit}). Siv-Berit betyder således \quotetext{den som vill ha knark}. Den maskulina formen av Siv-Berit är Syd Barrett.

}

\small{
\textbf{Sjua}
\label{e7bf63fa6d0d29bd89c23f833b979a15}
 En sjua, även kallad '7"', är skivsamlofilers beteckning för en vinylsingel. Det är också en populär siffra.

 Se även första sjuan \textsc{(s.~\pageref{7b7c558fc3f8d8557ba30b082e644ea1})}

 Se även: Etta \textsc{(s.~\pageref{ba48f6c4097b7fc25ca11f1e544842d7})}, Tvåa \textsc{(s.~\pageref{84fcc0494ecf9f5af79fcd9bed184a9a})}, Trea \textsc{(s.~\pageref{6f94fdf535ab2e21147ea40ea920ca75})}, Fyra \textsc{(s.~\pageref{7bdb5385ce8e0b1cbc7c15b1d71e8e7d})}, Femma \textsc{(s.~\pageref{d974e0811fe7a4d49a9062d33b66a88d})}, Sexa \textsc{(s.~\pageref{4b1fabe53857b0a2ace6ae22008fe13e})}, Åtta \textsc{(s.~\pageref{6fa68b0d02ec525fa72a51c13e5e3ed1})}, Nia \textsc{(s.~\pageref{04a481486dd84d7c8bfdfc89d38136a6})}.

}

\small{
\textbf{Sjundedagsadventistisk skola}
\label{e80411442bbb22f9ae7ed44d5780cfc2}
 På den sjundedagsadventistiska skolan får eleverna lära sig att förhålla sig till vuxenvärldens regler. Här är det rektor, tillförordnad av den helige ande, som sätter agendan och inte popmusik och ungdomsfilm och dataspel och smart drugs och allt vad det är. Här får eleverna lära sig att klä sig som små tanter och farbröder redan i tonåren och därmed kan man redan där rationalisera bort alla livets faser mellan oskuldsfull barndom och fruktlös pensionsålder. Puberteten är en styggelse som man i nuläget tvingas ha överseende med, och man har lärt sig rutiner för hur man bäst skapar förvirring och osäkerhet kring kropp och samliv och på så vis kraftigt reducerar riskerna för att eleven lägger sig till med en fallenhet åt hedonistiskt leverne. Alla föräldrar kan alltså vara lugna. Här är flickor, flickor och pojkar är pojkar och där med basta!

 HEAD2: Glädjes åt skapelsen!
 Lärarlaget består av ariska män med mustasch \textsc{(s.~\pageref{78fe8e02985abb5090cb3f33ac2842d4})} och kvinnor med långkjol som har det gemensamt att de icke tvivla på herrens ord och att de kräver militärisk disciplin från sina lärjungar. Men även dessa personer har väl varit unga en gång, och ibland - inte för ofta, men ibland - går man ungdomen halvvägs till mötes och tar fram gitarren \textsc{(se gitarr s.~\pageref{a08bf8420208934bc59c7ed7385d4308})} och går ut till en solig plats och leder hela klassen i allsång, bara för att det är viktigt att glädjas ibland också och inte bara plugga tyska böjningsformer (undantaget homosexualitet). Och visst är det lätt att se all det där vackra när en tjugohövdad kör låter Ted Gärdestads \quotetext{Sol, Vind och Vatten} eka mellan kapellet och gymnastiksalen, där en grupp flickor tränar uppvisningsgymnastik med fladdrande band och små färglada bollar..

}

\small{
\textbf{Sjungande trummis}
\label{86761451ff4f807963858d8a2afece37}
 ar är nästan lika ovanligt som trummisar som skriver sina egna låtar, det vill säga väldigt ovanligt. Det finns två trummisar som sjunger som är värda att nämna och det är Phil Collins i Genesis som sen gick solo, och Reemu Altonen i Hurrgianes som märkligt nog lever fortfarande. Den ene är alltså ett brittiskt pretto som gjort en bra låt och den andre en alkoholiserad och kriminell finne \textsc{(se finländare s.~\pageref{fc472090d678bd6f029cd80792f4a36d})} som stal sitt första trumset och skapade en ganska solid bakkatalog. Genesis är som alla vet tio gånger mer svårlyssnat än en urspårad Zappa-skiva. Pick your king.

}

\small{
\textbf{Självbiografi, del 1 - Snälla mamma, mata mig som vore du en fågel}
\label{601d26ca5273a18aecefd397445478c4}
 Den första delen i Prof. Etiennes \textsc{(se Prof. Etienne s.~\pageref{56957a267e57df32753cf7f3b8a603d8})} självbiografiska verk avhandlar författarens första stapplande steg, från födsel till det att han fyller åtta.
 HEAD2:  Synopsis

 Författaren föds i Ödeshög. Tidigt i sitt liv märker den unge Etienne att han är mer benägen att identifiera sig med fåglar än människor. Han vill flyga långt bort, obunden av sina fysiska begränsningar (han var ett korpulent barn). För att bli en fågel vill den unge Etienne bli behandlad som en sådan och får sin mamma att mellan 4-7 års ålder mata honom som fågelmammor matar sina ungar, det vill säga genom att modern först tuggar maten för att sedan vomera den i sin avkommas mun \textsc{(s.~\pageref{6585f290ce92c3de5ff339920330e26f})}. Om Etienne inte fick som han ville drabbades han av extrema raseriutbrott som kunde pågå i flera timmar. Hans fixering vid fåglar släpper när han blir attackerad av en kråka som han slår ihjäl i självförsvar. Efter att ha dödat sin nemesis öppnar Etienne bröstkorgen på fågeln med sin schweiziska armékniv, tar ut kråkans hjärta och äter det på plats, rått. Efter denna ritual anser Etienne att han tillförskansat sig fågelns makt och kan återgå till att bara vara en plufsig människopojke.

}

\small{
\textbf{Självbiografi, del 2 - De förlösande thinneråren}
\label{b90e1d79e5552c043da4a3fea8505e82}
 Den andra delen i Prof. Etienne \textsc{(se Användare: Prof. Etienne s.~\pageref{a9878d2280e5a39becac8f73d113df91})} självbiografiska verk avhandlar författarens tonårsperiod där han börjar utforska sin egen kropp och på allvar forma sin identitet.

 HEAD2: Synopsis

 Författaren känner sig desillusionerad i skolan och längtar bort till något annat. Han börjar skolka och ligger mest hemma i sängen och räknar sina kroppsöppningar. Han experimenterar med hur dessa kan stimuleras av kemiska substanser och tror sig en dag av en händelse lösa livets gåta. Ödeshög kan inte längre erbjuda allt det författaren vill uppnå så han bestämmer sig för att ge sig ut på luffen. I ett dike träffar han på några medlemmar i rockbandet Rövsvett, som däckat där kvällen innan. Han tar anställning som bandets chauför (det anses fördelaktigt att han ännu inte är straffmyndig) och får ibland gästsjunga på scenen. Efter en konsert i Köpenhamn blir han frånåkt och flyttar in i ett rivningshus tillsammans med en före detta läkare som påstår sig vara den som introduserade meskalinet i Norden. Tillsammans grundar de en sexkult och det mesta blir dimmigt under några år.

}

\small{
\textbf{Självbiografi, del 3 - En bärs, en bärs, min järndanksamling för en bärs}
\label{324597faa7f8fbab47ec612d49ddc2de}
 Del tre \textsc{(se trea s.~\pageref{6f94fdf535ab2e21147ea40ea920ca75})} i Prof. Etiennes \textsc{(se Prof. Etienne s.~\pageref{56957a267e57df32753cf7f3b8a603d8})} svit självbiografiska romaner.

 HEAD2:  Synopsis

 Alla goda ting har ett slut, något Prof. Etienne blir medveten om när ledaren i den sexkult han gått med i blir arresterad på Bälinge torg efter att beväpnad med en luger ha delat ut antisemitisk propaganda. Vid arresteringen av ledaren upplöses sexkulten med omedelbar verkan. Vilsen och förvirrad irrar Prof. Etienne nu runt i Bälinge, enbart utrustad med en påse järndankar \textsc{(se dank s.~\pageref{eee1edb16ac8987af66023852db6c513})}, den enda gångbara valutan i sexkulten, samt ett par harembyxor han bytt till sig av Claes Malmberg i den ormgrop där Gottfrid Svartholm Warg blev till. Som så många andra sökande exkultister i Bälinge fann sig Prof. Etienne inom kort levande som busstationsalkoholist. Under en redig bläcka med några skräniga kärringar får professorn slut på bärs \textsc{(se ha bärs s.~\pageref{a74b297c15834437ac2e49095492133c})} och ropar ut i högan sky de bevingade orden \quotetext{En bärs, en bärs, min järndanksamling för en bärs}. En av damerna i sällskapet fick då nog och kastade en Arboga 10.2\% och ett slitet exemplar av Sven Hedins första reseskildring \textit{\quotetext{Genom Persien, Mesopotamien och Kaukasien}} på antagonisten. Det var efter att ha inmundigat den goda brygden och sträckläst den vise Hedins ord som Prof. Etienne beslöt sig för att ta tag i sitt liv och själv bli upptäcktsresande!

}

\small{
\textbf{Självbiografi, del 4 - När jag sköt Elefantmannen}
\label{760ef86f3e992d21fb0379fe35c73d98}
 Den fjärde delen i Prof. Etienne \textsc{(se användare: Prof. Etienne s.~\pageref{a9878d2280e5a39becac8f73d113df91})} självbiografiska verk avhandlar författarens intåg i den gyldene medelåldern på resande fot genom Asien. Vid utgivningen anklagades boken av flera recensenter för att innehålla stereotypa beskrivningar av folkgrupper och tveksamma sensmoraler om makt. Dessutom ifrågasattes sanningshalten i flera av de påståenden som förekom. Prof. Etienne \textsc{(se Användare: Prof. Etienne s.~\pageref{a9878d2280e5a39becac8f73d113df91})} bemötte kritiken med att \quotetext{be alla förståsigpåare \textsc{(s.~\pageref{ff91afb86ce86124b6a517f3eb37bc18})} fara åt helvete}.

 HEAD2: Synopsis
 Någonstans i gränstrakterna mellan Kina och Vietnam. Författaren är vilse i djungeln tillsammans med en elefant och fyra \textsc{(s.~\pageref{7bdb5385ce8e0b1cbc7c15b1d71e8e7d})} tvångsrekryterade bärare. Vietnamkriget har nyligen avslutats och militären är nu istället på jakt efter den människosmugglarliga författaren lett och försörjt sig på de senaste åren. Till råga på allt börjar hans lager av thailändska bantningspiller som han de senaste månaderna utvecklat ett beroende för att ta slut. Dagarna går och när han i paranoia slutligen skjuter den sista bäraren verkar allt hopp ute. Han vänder geväret mot sin egen panna och ska just trycka av när han hör människorop. Ett indiskt arméförband på tigerjakt har fått syn på den utmärglade mannen \textsc{(se man s.~\pageref{39c63ddb96a31b9610cd976b896ad4f0})} och för honom till sitt läger. Väl där får han genomgå läkarvård och hans hälsa förbättras snabbt. Livet har fått en ny mening för den modige äventyraren och allting \textsc{(s.~\pageref{2ea7603b8880ffdf729128008f5d252d})} är underbart, tills han en natt drömmer hur högste befälet för \textsc{(s.~\pageref{5a98c81c7b5b60a5777a92b943f53a41})} lägret ger order om att koka vit djävulssoppa av hans nu fullt friska kropp. Författaren springer upp, griper sitt gevär och springer raka vägen in i officerstältet och skjuter befälet. Snabbt omringas han av soldater som verkar ovilliga att ge honom en rättvis rättegång. Han skriker allt vad han förmår att han handlat på uppdrag av gud \textsc{(s.~\pageref{91e49146121c992aab11a19c77e26cf0})} och att han är deras nye frälsare. För att bevisa sina magiska krafter lär han upp hela lägret i den ädla konsten om tantraonani, och soldaterna låter sig glatt frälsas. Han blir lägrets nye ledare och tar sig titeln \quotetext{professor}. Han har stirrat döden i vitögat men har istället återfötts med ett nytt kall i livet.

}

\small{
\textbf{Självbögare}
\label{398dd2a3bea77cfe04df91ba1a8b8c65}
 Luthersk beteckning för vad som idag kallas onanister.

}

\small{
\textbf{Självförtroendeplagg}
\label{0b79066fe3ed7b0c8eaf689c8a72a285}
 Ett plagg man bara kan ha när man känner sig riktigt ball. En t-shirt som är snygg som fan, men på gränsen till för liten. När man vaknar på morgonen och känner sig som Steve McQueen, då kan man ha den. De morgnar man vaknar och känner sig som Sven-Otto Littorin är det helt omöjligt. Eller kanske ett par jeans som man inte riktigt vet om de är snygga eller ej egentligen, men ibland bara förstår att man kan bära upp, precis som Wendy O Williams skulle göra.

 HEAD3: Falskt situationsbundet självförtroende

 Vid vissa situationer kan en person få för sig att den kan bära upp ett plagg på grund av situationen den befinner sig i. Ett exempel på detta är när en person är full. Då är det lätt att gå bananas \textsc{(s.~\pageref{ec121ff80513ae58ed478d5c5787075b})} och dra på sig en skinnpaj med fransar på ärmarna, eller skoja upp en chapeau de paysan \textsc{(s.~\pageref{27aa75146d9ab723d1423168a2539d5d})} på hjässan och tro att man kommer undan med det. Det gör man sällan, vilket gör skammen än större när man vaknar dagen efter och inser vilken \textit{schmuck} man var.

 Ett annat klassiskt tillfälle då allt omdöme fallerar är utlandssemester, särskilt till sydostasien. På plats kan det vara ok att ha på sig flip flops, snäckhalsband - kanske till och med kläder av märket billabong. Men när personen kommer hem till sitt västerländska, senmoderna samhälle och försöker dra runt på stan med samma billabongshorts som var helt ok i Phuket, kommer den oundvikligen att drabbas av enormt stigma \textsc{(s.~\pageref{98410ec61c6964eac5c923a594841696})}. Det inneboende självförtroende som kommer med en utlandssemester visar sig vara en chimär. Väl tillbaka i Uppsala \textsc{(s.~\pageref{1db4e388df1df7057b8f3d984c65ee88})} är man inte lika cool som Heath Ledger i Point Break, utan bara ännu en tönt som ska falla in i det chinosbeiga ledet.

}

\small{
\textbf{Självmordsspåret}
\label{63ccd596f5759eb2acb9922e27eba790}
 är förmodligen den teori privatspanare \textsc{(s.~\pageref{b7a4113e7c457f65a55f866e146bcf69})} ägnat minst tid åt kring mordet på Olof Palme \textsc{(s.~\pageref{702b78623785546fb9c9890222376178})}. Enligt förespråkare för detta spår var mördaren ingen minder är Palme själv, i ett försök att lansera sig som den nye Che Guevara. Palme var omvittnat förtjust i standar och flaggor med politiska ledare och när Ches porträtt började spridas på T-shirts och nyckelringar blev han grön av avund. Spåret belystes första gången i förordet till Prof. Etiennes \textsc{(se Användare: Prof. Etienne s.~\pageref{a9878d2280e5a39becac8f73d113df91})} bok \textit{På spaning efter den bov som flytt \textsc{(s.~\pageref{b0c7545c68966ced2a217a2e575fb207})}}. I vanligt ordning presenterade Prof. Etienne inga konkreta bevis för sitt påstående utan litade helt till sin gubbsäkerhet \textsc{(se gubbsäker s.~\pageref{e6cb916b91ceed5550ee4204e7b6c902})}.

}

\small{
\textbf{Skagen}
\label{d88f07528fa07f7be9318ece7656fd0b}
 kallas det område i norra Danmark \textsc{(s.~\pageref{5331d7fd27772396f412a5b6d19bad44})} där ljuset är så fantastiskt att konstnärer från hela norra Danmark vallfärdar dit på dagarna för att måla av det. På kvällarna vallfärdar dom hem igen för det är som bekant ganska kallt vid havet. I ärlighetens namn är det sällan någon som orkar måla så mycket eftersom fantastiskt ljus även inbjuder till att ha picnic på en filt och lyssna på \textit{Dark side of the moon}. Många konstnärer började ta med sig burkar med majonäs i stället för målarfärg, som de på förvirrat konstnärsmanér glömde skruva tillbaka locket på efter avslutad picnic. Insekter kröp ned i burkarna och fastnade i majonäsen och det var så skagenröra uppfanns.

}

\small{
\textbf{Skam}
\label{e7d275bbd2f3522805002be76a53ccd8}
 Skäms gör man rätt ofta. Ibland för sin egen skull och ibland för andras. Skam har av en psykiater på umeå universitetssjukhus beskrivits för denna artikels författare som en av de hemskaste känslor en människa kan känna. Ett gott exempel på när man skäms är när man suttit på en uteservering i en sydsvensk stad, limmat på en soft tjej bara för att se henne dra med din polare. Besvikelse \textsc{(s.~\pageref{a3cdc7d1b33db6959c1d3a78b1f47012})} är den första känslan man känner i anknytning till en sådan händelse. Men sen kryper skammen på, när man börjar fundera över hur uppenbart allt var hela tiden och vilken total sopa man måste ha framstått som. Om man dricker alkohol är det lätt att förskjuta skammen, då människan efter ett litet intag av tidigare nämnda drog förlorar precis all skam i hela kroppen. När skammen kryper sig på morgonen efter slår den ofta till med dubbel styrka. Det är inte ovanligt att man ligger och spelar upp sekvenser från kvällen innan i sitt huvud och ba neeeeeiiiin!!!! Något man kan göra för att bota skam i en sådan situation är att hänga ett tag med sin vän Emanuel, äta lite fika och sen dra över till Köpenhamn och bli full igen.

}

\small{
\textbf{Sked}
\label{81883ec4ef5ec0b545b6d293b5793d2e}
 är ett populärt redskap när man äter rinnig mat.

 HEAD2: Historia

 Skedens historia börjar i ett kök i Marocko. Ibrahim Al-Hazir hade bjudit över sin goda vän Adiba Bin-Hassan på ett mål linssoppa. Ibrahim ställde fram skålar och karotter och även ett nybakat bröd. Ibrahim börjar äta sin soppa genom att liksom badda upp soppan med brödet, men Adiba hade ett ess i kaftanärmen. Hon halade fram ett stycke trä med en grop i ena änden, doppade den i soppan och sörplade sedan i sig linssoppan högljutt.

 Marxistiska historiker brukar hävda att bordsskicket inte existerat före skedens och de andra bestickens uppfinnande.

}

\small{
\textbf{Skedstork}
\label{76648d90910c2fd6fcd81b3f3f28d9ea}
 en är en fågel som dragit en nitlott i evolutionens lotteri. Istället för en näbb fick den salladsbestick i bakelit i ansiktet. Den är antagligen utrotningshotad \textsc{(se utrotningshotade djur s.~\pageref{24a427a5537c2c8918cfa213ae099a74})} då den enda mat den kan äta är caesarsallad som det är sparsmakat med i dess habitat och på grund av att den används vid tillverkningen av storkskedar \textsc{(se storksked s.~\pageref{68dfcd370c776cd068ad3b00f9cecd7b})}.

}

\small{
\textbf{Skellefteå}
\label{6f7057df8c42e3cc22d2ad46de5c6597}
 är en stad som ligger norr om Umeå och söder om Piteå. Det är svårt att sätta fingret på vad grejen är med Skellefteå. Det är väl egentligen en vanlig håla, som ett halverat Västerås. Men det ligger nåt i luften. Är det guldet som förfinas av Boliden AB, strax utanför staden, som lägger ett ömsom euforiskt ömsom ångestladdat Midasskimmer över skylinen? Eller är det de mystiska influenserna som glider in i på ångare från orienten till Skelleftehamn som sprider ett fördunklande dis över möjligheternas torg? Är det Torgny Lindgrens mytologiska sagor som tunnat ut gränsen mellan verklighetens bas och fantasins överbyggnad mellan Kåge i norr och Bureå i söder, mellan Boliden i väst och Ursviken i öst? Eller är det inaveln, den primitiva blodsreligion som förbjuder att blanda ut den starka Marklundska genpoolen med undermåligt Anderssonskt piss, som driver på den pulserande rödlila nerv som slår an takten för livet vid Skellefteälvens mynning?

}

\small{
\textbf{Skensnygg}
\label{b8c4b0c5f26691aa5a96a144f2276349}
 Förgängligt utseende som avslöjas först på nära håll. Beskrevs träffsäkert av David Lynch i Twin Peaks med repliken \quotetext{the  owls \textsc{(se uv  s.~\pageref{45210da832f9626829457a65e9e7c4d0})} are not what they seem}.

}

\small{
\textbf{Skiftnyckel}
\label{4eeef21c5cf8813ae82d2882f54c8e28}
 ns (också känd som muttersvarvare) \textsc{(se muttersvarvare s.~\pageref{78ccc210543eb2a008f7999e1f1f905b})} främsta användningsområde är att knacka, slå och sparka på saker när en hammare inte finns, eller helt enkelt inte är placerad inom bekvämt avstånd. De längre varianterna kan användas som brytspett. Ska man dra muttrar ska man ha fasta nycklar \textsc{(s.~\pageref{ad577d76d7747bfd314d442197fc8587})}, hör ni det!?

}

\small{
\textbf{Skillnad på ånga och dimma}
\label{737b9a7e6831860a9dcf7ed318e47ffc}
 Ibland när man befinner sig i ett vitt moln \textsc{(s.~\pageref{9da1014bea9aa67f9cae12e619d34aae})} kan det vara svårt att fastställa om fenomenet är dimma \textsc{(s.~\pageref{b63ad17940b78107e72e63b7d637d91b})} eller bara vanlig ånga. Som tur är finns några enkla tummregler att tillgå:

 Ånga uppstår ur en enskild källa. Till exempel från bastun som lokförarens fackförbund förhandlat fram ska finnas i alla lok, eller från en het källa som stenåldersmänniskor använde för att koka ägg innan de lärt sig bemästra elden.

 Dimma däremot uppstår överallt där det är vatten i luften som är för lite för att regna och för mycket för att dunsta. Överallt där det är så blir det dimma. Överallt!

 Om du befinner dig i ett vitt moln och snabbt måste veta om det är dimma eller ånga räcker det alltså med att du ser dig omkring och tittar om det finns en enskild källa eller inte. Ta det lite försiktigt bara, bastuaggregat och varma källor är väldigt varma.

}

\small{
\textbf{Skinhead}
\label{a54bc1b5d472b5afed8e84004b6441c4}
 Klä sig som en pensionär och supa med nazister, jättekul om du har \quotetext{kraftig benstomme} och bor i nån jordbruksort \textsc{(s.~\pageref{3257bf804d763afce5a153f73ce80f7c})}.

}

\small{
\textbf{Skinned Alive}
\label{88b01b69ece92f30d83711e8a65fd542}
 är ett råbarkat Oi!-band från Berghem och Tunnelbacken, Umeå. Bandets musik karaktäriseras av drivig gitarr och HC-inspirerad sång. Låttexterna tar ofta upp de mindre glamorösa sidorna av arbetarklassens tillvaro i stadslandskapet samt lite grumligt uttryckta politiska åsikter. Bandet har också, vänligt nog, stöttat äldre band genom att göra covers på deras låtar. Exempel på sådana band är The Last Resort och Blitz. Bandets merch sköts av Hawaii-kråkan \textsc{(se Användare: Hawaii-kråkan s.~\pageref{a777da05d6e59c7961af7b56578d657a})} och utförs i broderi. Grabbarna är också stökiga supporters till fotbollslaget Berghem HC \textsc{(s.~\pageref{72c5e1ef562098496277726ca12aa149})}

 HEAD3: Medlemmar

 Orgasmatron Andersson \textsc{(s.~\pageref{992f857a2415202c7eb4b9f973ea11a0})} - Gitarr \textsc{(s.~\pageref{a08bf8420208934bc59c7ed7385d4308})}
 Iron Horse Hållén - Trummor
 Love me Like a Reptile Sjöhund - Bas
 Fat \& Loose Rolandsson - Sång

}

\small{
\textbf{Skinnsbergslucia}
\label{1195cf539556d5d28ac8418c613f2676}
 Nedsättande uttryck om person av kvinnligt kön. Härstammar från perioden då tidningen Fagersta-Posten \textsc{(s.~\pageref{e879bdcb386d850bb5606058db7464d4})} hade luciaomröstningar även för Skinnskattebergs \textsc{(se Skinnskatteberg s.~\pageref{f0666ee995da080da55f5f6892fe3dcc})} kommun. Tävlingen lades sedermera ned. Redan den store filosofen, naturvetaren och rabulisten Carl von Linné \textsc{(s.~\pageref{5e8380bf6b7ce99678e6752b6d9e709e})} sade att \quotetext{Då jag liggat hos en ful, ängslas jag, men vacker tycks  gjort mig väl.} Hans skarpa ögon observerade även att: \quotetext{En kåt flicka är gemenligen vacker, jag vågar ej säga vice versa.}

}

\small{
\textbf{Skinnskatteberg}
\label{f0666ee995da080da55f5f6892fe3dcc}
 (ursprungligen kallat Skinnsäckeberg) är en kommun i norra Västmanland. Skinnskatteberg är födelseort för välkända figurer såsom Fakta \textsc{(s.~\pageref{fce663ae73dc87a727148bc3b94d1ffa})}, Andreas Kleerup, Slisken \textsc{(se Användare: Slisken s.~\pageref{0434b6e7c92786761d7fb5b1e5e0dd3d})} och Sober-Jimmy \textsc{(s.~\pageref{62629d44a92716a33e051e9a6c04d0d4})}. Även Jan Myrdal har fattat tycke för ortens blomstrande stillhet och bosatt sig där på äldre dagar.

 Skinskatteberg har i likhet med Malå \textsc{(s.~\pageref{41da4620e87888eaaeafcb3004a8d177})} ett sågverk.

}

\small{
\textbf{Skinnslips}
\label{19eeaa0136ea45fc9f8d1f129da32d63}
 en kan ses som näverslipsens \textsc{(se näverslips s.~\pageref{b8fca0a91648db5415350e52dc2e9c94})} efterträdare. En individ i skinnslips kommer med största sannolikhet att sno din partner på dans, eller din bättre hälft på ert bröllop. Skulle handgemäng utbryta klarar sig personen i skinnslips alltid undan med hjälp av sitt garvade munläder.

}

\small{
\textbf{Skita i den korvbröda kökssoffan}
\label{531e4daf8d52be2fda4233f977daa20c}
 Den proletära varianten av att skita i det blå skåpet \textsc{(s.~\pageref{97d35803d0f77bf90f90cd3c83dc323d})}.

}

\small{
\textbf{Skita i den krovbröda kökssoffan}
\label{b1c27594af52efd3e3811a4da3b4e081}
 Skita i den korvbröda kökssoffan \textsc{(s.~\pageref{531e4daf8d52be2fda4233f977daa20c})}

}

\small{
\textbf{Skita i det blå skåpet}
\label{97d35803d0f77bf90f90cd3c83dc323d}
 Att \textbf{skita i det blå skåpet} är ett kraftuttryck för att markera att någonting gått riktigt snett. Ursprungligen tros uttrycket härstamma från 1800-talet, när färgämnet \quotetext{berlinerblått} började massframställas så att också vanliga knegare kunde börja blåmåla sina möbler. Tidigare var blått främst förunnat borgerskapet, och allmogen fick istället nöja sig med gammal hederlig korvbröd \textsc{(se \quotetext{korvbröd} s.~\pageref{6898888a74f0d42574012debf1a6d8f3})} och ockra. Associationen mellan blått och fisförnämhet levde dock vidare, vilket vi ju kan se än idag, så man blåmålade först och främst de finare möblerna i hushållet. Färgen är således egentligen inte det primära i talesättet, utan det faktum att någon varit dum nog att tömma sin tarm i finskåpet istället för exempelvis den mer vardagliga kökssoffan \textsc{(s.~\pageref{d1c2d6488fde9b41b5c6b2a03c5fd79c})}. Man kan jämföra det med att spy i handfatet, där det vanligtvis finns en för ändamålet betydligt lämpligare klosett bara någon meter därifrån.


 Uttrycket populariserades i modern tid av Janne \quotetext{Loffe} Carlsson i filmen \textit{Göta kanal} \textsc{(se Sveriges sju underverk s.~\pageref{f4f71e4db3f279d42d840c805d75820c})} från 1981. I ett trängt läge, där någon avlägsnat en propeller \textsc{(s.~\pageref{5eba517a2887595e2fd711e32090a0a7})} från Carlssons båt, utbrister denne: \quotetext{Nu är det krig! Nu har dom skitit i det blå skåpet!}. I citatboken \textit{Bevingat} säger Carlsson att det var hans far som lärde honom uttrycket, och det finns väl ingen anledning att ifrågasätta det.

}

\small{
\textbf{Skittermos}
\label{d0e4d29a5e801adb167671f20e680aef}
 En skittermos är en termos vars varmhållningsförmåga motsvarar en vanlig petflaska. Skittermosar är i det kapitalistiska samhället väldigt vanligt förekommande.

 En skittermos är termosarnas motsvarighet till skrotsked \textsc{(s.~\pageref{ccff3870ccabf38bede8618d3fffa289})}

}

\small{
\textbf{Skjortponcho}
\label{2bfcf7325f5d6bb20721be641facf4ad}
 En skjortponcho är ett klädesplagg som rent hypotetiskt används av en färgstark individ i underhållningssyfte.

 Det har varit svårt att finna några konkreta bevis på dess existens, då den endast skymtats på skivomslaget till Charles Bradleys singel Heart of Gold. Detta är dock ej tillförlitlig information i dagens tider med Photoshop och dylikt. Experterna hävdar att omslaget är manipulerat, men entusiaster världen över lever på hoppet.

 Det sägs att plagget består av peruanskt vikunjaull. Något som låter rimligt eftersom att belgisk polis enligt inofficiella uppgifter fann peruansk valuta på Bradley i samband med en rutinprocedur som uppbar vissa likheter med konventionell kroppsvisitation.

 Om uppgifterna stämmer beräknar statistiker en kraftig tillbakagång för lamabeståndet i världen, då den ökade efterfrågan uppmuntrar till tjuvjakt och annat ofog.

}

\small{
\textbf{Sko}
\label{b984ba016cb3ae0ba861dbc7c3dcb361}
 Någonting man gör sig på andra, eller möjligen en häst \textsc{(s.~\pageref{b4c608370b339da095c5f8db7fab0945})}.

 Detta är grundtanken i kapitalism.

}

\small{
\textbf{Sko sig}
\label{8c224f3894ba620a00beb97f21148656}
 Att sko sig är att \quotetext{byta} sina gamla skor mot en ny och bättre modell. Gör detta genom att gå in hos en vanlig skonasare/sportaffär och ta på ett par skor i rätt storlek, lägg dina gamla i kartongen och gå sedan ut. Lätt som en plätt.

}

\small{
\textbf{Skock}
\label{ccd7a5943edd5633df5bb3054df1395d}
 Tre tjog \textsc{(s.~\pageref{93e1254c6b6d02b89439cbea1926a4de})}.
 Eller en Nàvkka på äran och hjältarnas språk.Innebär då något mer än 4 men mindre än 15.

}

\small{
\textbf{Skogsrave}
\label{2180f77028a02c8fd94f622505937a53}
 Ett rave som tar plats i skogen. Skogsravet är en relativt ny företeelse som går ut på att ett gäng käcka ungdomar drar med sig ljudutrustning som drivs av ett dieselaggregat eller nåt ut i skogen och spelar högljudd musik, dansar samt (förhoppningsvis) tar kemiskt framställda droger. Vildlivs- och musikvetaren Mulva Mossrock har yttrat sig om fenomenet i följande ordalag: \textit{\quotetext{I teorin är det en rätt juste idé att dra ut i skogen, lyssna på vild musik och kanske kåtligga lite i en murrig backe, men just fenomenet skogsrave omgärdas ofta av den högtravande idén att det ska vara så jävla häftigt och speciellt att vara i skogen, vilket pajar softheten med hela grejen. Inte sällan kommenterar nån av deltagarna att det ändå är rätt black metal att vara ute i skogen, vilket inte är helt osant. Problemet är bara att deltagarna själva inte ens gillar black metal eller friluftsliv utan har något sorts post-ironiskt förhållningssätt till hela den hedniska naturdyrkargrejen och vill egentligen avnjuta sin blip blop på lokal. Således förlorar inte bara deltagarna i ravet på dess existens, då de blir blöta, kalla och inte sällan ådrar sig urinvägsinfektioner, utan även vi genuina naturfestdiggare förlorar på att skogsravet finns, då popkidsens dieseldrivna dunka dunka-maskiner dränker ut det sköna rocknojset från våra dassiga bandare.}}

}

\small{
\textbf{Skolbespisningsmat}
\label{0620e2bdb64059d3b73e2215e741d052}
 Gemensamt för all skolbespisningsmat är att den serveras i tråg \textsc{(s.~\pageref{1e0e0470206e0f2baad8e628ba8f770c})}.

 \begin{itemize}
 \item Fläskpannkaka (finns utan fläsk)
 \item Frikadellsoppa
 \item Grillburgare (max tre (3) per elev)
 \item Mexicanasoppa
 \item Korvslantssoppa
 \item Leverbiff
 \item Potatisplättar
 \item Kokt potatis
 \item Fiskbullar
 \item Torsk
 \item Ryssröra
 \item Vad huset förmår
 \item Raggmunk (oätlig)
 \item Lappskojs \textsc{(s.~\pageref{0d0eb99c8a08ae96acd7226a3cfec257})}
 \end{itemize}

}

\small{
\textbf{Skoter}
\label{b1120baa83f380cd42a805a4e823cb1b}
 Är du skåning \textsc{(se skåne s.~\pageref{a01d1167b9dcd72e212d876d672db261})} eller bara allmänt kockobello \textsc{(s.~\pageref{87464dbe1f4053eaf434b95c4e6b38ab})} tror du att skoter stavas \quotetext{scooter} och åsyftar en undermålig italiensk moped. Är du normalbegåvad vet du att det åsyftar ett fordon för färdsel på snö. Man bör ha en Lynx 5900 eller Forrest fox, om man nu inte sitter på en klassiker som Ockelbo 6000, Aktiv Grizzly, Aktiv Kariboo, AMV eller rentav en Larven, har du den sistnämnda är du en solklar ägmästare \textsc{(s.~\pageref{8324518500d7e7ccd22ae364887d4476})}.
 Skoter är bra att ha när man ska dra hem ved \textsc{(s.~\pageref{29e0461b02c078c89c7b2ac0b29fbfaf})} eller en undanröjd järv.
 Att ge sig ut på sin skoter, eller någon annans för den delen; utan yxa \textsc{(s.~\pageref{bd74f429522c7c1481fbba07187efc6b})}, spade \textsc{(s.~\pageref{ab1991b4286f7e79720fe0d4011789c8})} och tegsnäsare \textsc{(s.~\pageref{f2d36877ad79d9fe50d1415d462f9e8b})} är inte så smart.

}

\small{
\textbf{Skottar}
\label{c2e5f84c76d823ea9482387bfb950791}
 Hämndlystna och giriga på onödiga saker. Stora fans av dronemusik.

 Källa: Prof. Etienne \textsc{(se Användare: Prof. Etienne s.~\pageref{a9878d2280e5a39becac8f73d113df91})} - \textit{Om arternas uppkomst}. Bonnier Fakta, Svinesund 1993.

 Se även: skotte \textsc{(s.~\pageref{6b6f15aba3e3e53800e792ce5f1707c8})}

}

\small{
\textbf{Skotte}
\label{6b6f15aba3e3e53800e792ce5f1707c8}
 är en stycksak av choklad som produceras av Marabou. Levarantör är Kraft Food Sverige. Skotte finns att tillgå i dubbelutförande med två bitar på 30g vardera. Skotte består av nougat och, för ovanlighetens skull i konfektyrvärlden, russin. Detta gör naturligtvis att skotte är avskyvärt äcklig \textsc{(se äckligt godis s.~\pageref{7a949bcbd13153b7e40bc8bf8dbb481a})}, men någon, oklart vem, köper den ändå och medan vi ser självklara klassiker försvinna från godishyllorna levererar godisbilen oförtröttligt skottar \textsc{(s.~\pageref{c2e5f84c76d823ea9482387bfb950791})} till landets alla gotteaffärer. Namnet har den fått för att den ger ett liknande helhetsintryck som skotsk mat.

}

\small{
\textbf{Skottfint}
\label{9aa58c20026cda87b7ce73731fde524a}
 En skottfint är en manöver som framförallt används i fotboll \textsc{(s.~\pageref{961bd74d34872ff94a4df3a16119096e})} och som är speciellt vanlig i matcher mot det italienska landslaget. Finten går ut på att bollinnehavaren låtsas ladda för att skjuta ett hårt skott. Den mötande italienaren kastar sig då till marken med händerna för ansiktet, föranledd till detta av den italienska folksjälens narcissistiska svaghet. Skrikandes från marken anmodar italienaren bollinnehavaren att inte skjuta. Den senare kan nu obehindrat runda den liggande italienaren, vars sikt är avsevärt begränsad på grund av häftig gråt.

}

\small{
\textbf{Skrapade skraplotter med vinst}
\label{d92a1dab805e77864765073c2a276c4c}
 är små papperslappar som kan växlas in mot en viss summa pengar, vilken anges på själva lappen. Summan varierar mellan ungefär 25:- och åtskilliga miljoner. Till skillnad från sin kusin, den skrapade skraplotten utan vinst, kan denna mer förnäma lott användas som betalningsmedel, det vill säga som valuta \textsc{(s.~\pageref{cf1e2a0af4955aa7539b6e12e9d282e6})}. Detta är dock ovanligt, vilket följer av att skrapade skraplotter med vinst är mycket mindre vanliga om man jämför med vanliga pengar eller skraplotter utan vinst. Bellman \textsc{(s.~\pageref{b6b9660f4f754e67face0b2633b39aa6})}, skald och för en tid förståndare för det kungliga lotteriet, sedemera Bellmanlotteriet, ska enligt trovärdiga källor ha sagt om den skrapade skraplotten med vinst att: \quotetext{denna avklädda dam äro förvisso mer ärhevördighet än sin fruktlösa syster,} i ett för honom karaktäristiskt anfall av sexism.

}

\small{
\textbf{Skrattfnatt}
\label{e6493d2d7abe72301d48eaf5854d14f7}
 är ett stående inslag i Kalle Anka \textsc{(se kalle anka s.~\pageref{64db68f686a0ca4d9d641061cb3fdf13})} \& Co. Det består i roliga historier som läsaren hittar längst ned på varje sida i tidskriften. Dessa historier kan memoreras för att sedan hivas fram när man står kring kontorets water cooler eller är på disco med sin mellanstadieklass.

}

\small{
\textbf{Skrotsked}
\label{ccff3870ccabf38bede8618d3fffa289}
 Om en sked på något vis uppfattas som undermålig och oförmögen att uppfylla sitt syfte kan men titta lite snett \textsc{(s.~\pageref{64b2eafa388ee50a226adc9013644f08})} på den och säga \quotetext{skrotsked!}
 Se även: skedstork \textsc{(s.~\pageref{76648d90910c2fd6fcd81b3f3f28d9ea})}, storksked \textsc{(s.~\pageref{68dfcd370c776cd068ad3b00f9cecd7b})}.

}

\small{
\textbf{Skrunka}
\label{7e3152e0cbea2212bd02444c45ee00db}
 (verb, obestämd form singular; bestämd form \textit{skrunk}) är en multipel handling där utövaren skrattar högljutt samtidigt som den onanerar. Fenomenet har blivit mindre vanligt på senare tid i och med att man sällan har något riktigt roligt att skratta åt. Skrattet måste vara ärligt.

}

\small{
\textbf{Skruvkapsylöl}
\label{6b1e7b5dfefe355c219b8bd7ff4db28c}
 Vissa, främst utländska, ölmärken på bolaget har skruvkapsyl som standard (t.ex. Gösser, Miller) och andra har det periodvis (t.ex. San Miguel, Victoria Bitter).
 Det råder delade meningar bland öldrickare huruvida en skruvkapsyl är något att hänga i julgranen eller inte. Vissa förordar att skruvkapsylerna är av godo då de gör kapsylöppnare överflödiga och det på så vis går marginellt snabbare för den törstige att komma åt sin dryck. Andra menar att skruvkapsylen förtar det roliga i att smidigast öppna en flaska på det sätt man själv föredrar, såsom med en snusdosa, tändare eller helt enkelt en kapsylöppnare som var och varannan bär i sin nyckelring. Den vanliga kapsylen har som bekant skänkt oss ett av de vanligaste partytricken - att öppna sin öl med något ovanligare objekt man inte tänkt på att använda i nyktert tillstånd (såsom ögat eller kanske en bordskant där det garanterat uppstår ett eller flera permanenta märken efteråt).

 Skruvkapsylsöl kan ge upphov till olika lite mer sällsynta skador. Ifall konsumenten grundat en helkväll rejält på ett märke med skruvkapsyl och sedan byter till ett märke utan så finns risken att konsumenten under rus tror att även det andra märket går att öppna med händerna och skärsår uppstår tills motsatsen bevisats. Ölkonsumenter med bristande erfarenhet av att öppna ölflaskor med ena ögat ger sig gärna i kast med tricket då skruvkapsylen inte sitter lika hårt som en vanlig kapsyl, vilket långt från alla gånger faller väl ut. Detta skulle nog den argentinske poeten och essäisten Jorge Luis Borges vara den förste att hålla med om, ifall han hade levt idag.

 Trots att skruvkapsylen började användas redan år 1875 i England[http://www.british-history.ac.uk/report.aspx?compid=58865] så har de svenska bryggarna inte tagit den till sig än. Endast några få - däribland norrländska Werde från Zeunerts - har gett den en chans.

 Perlenbacher, som saluförs av den tämligen suspekta livsmedelskedjan Lidl, är sannolikt den enda folkölen i Sverige som har skruvkapsyl.

 Om man vill fördjupa sitt intresse för skruvkapsylens historia kan man besöka Flaskmuseumet i Sonkajärvi, Finland \textsc{(s.~\pageref{631d44eaa1254ff71a1e11ba021d1266})} (inträde endast 2€) [http://www.museot.fi/sokmuseer/index.php?museo_id=21780]

}

\small{
\textbf{Skruvspark}
\label{5b499f57aa4875e9b2a458e82badfc57}
 En \textbf{skruvspark} är att sparka en fotboll \textsc{(s.~\pageref{961bd74d34872ff94a4df3a16119096e})} i en vågrätt svängande bana i luften. Det är en mycket användbar metod om man till exempel skjuter bollen runt en motspelare, istället för rakt i magen. Det är ungefär lika lätt att slå en skruvboll som att hoppa runt ett hörn. Den som gör det är en ägmästare \textsc{(s.~\pageref{8324518500d7e7ccd22ae364887d4476})}.

}

\small{
\textbf{Skrymmande}
\label{b86f606a85a816255be1ab99895ff4e7}
 är postens definition av brev eller paket med mått som överstiger vissa gränser och därför måste ha fler frimärken för att kvala in i en högre kategori. Den mest klassiska typen av skrymmande försändelser är vinylskivor, men likt Einsteins relativitetsteori kan allt större en ett vykort i fel sammanhang vara det. Vill du till exempel skicka en palltruck \textsc{(s.~\pageref{9417f2a2e1478e6e63cba47cf2d1a505})} i födelsedagspresent till din bästa kompis på åttaårsdagen är det troligt att den kommer klassas som skrymmande även om du köper frimärken för alla pengar du har i bössan. Men även något så lätt och litet som en vanlig skedstork \textsc{(s.~\pageref{76648d90910c2fd6fcd81b3f3f28d9ea})} kan visa sig vara skrymmande om du vill skicka den med vanlig A-post. Ska man skicka något och känner sig osäker på om det är skrymmande kan man alltid höra med någon av postens \textsc{(se posten s.~\pageref{cd13d688571681e426231485b732444b})} rekorderliga medarbetare, för de vet allt om post. Var dock noga med att kontrollera att det inte är en vanlig ICA-personal i förklädnad \textsc{(s.~\pageref{f5a6964fb398df4c2da0d3bac3d8ed7a})} för de kan ingenting om post och paket.

}

\small{
\textbf{Skräckväldet}
\label{e615f6ed8b5b62ee69c6a48a4068a682}
 När historiker brukar prata om Skräckväldet handlar det påfallande ofta om den period då den spritt språngande galne Robespierre hade makten i Frankrike i slutet på 1700-talet. Mer sällan handlar det om den period i Malå \textsc{(s.~\pageref{41da4620e87888eaaeafcb3004a8d177})} kommuns historia mellan 1974 och 83 när den var hopslagen med Norsjö \textsc{(se Norsjöblicken s.~\pageref{4444470133e0178de88aa4daa4d63769})} kommun. Detta drevs igenom av Socialdemokraterna, men det ska sägas att Malås kommunalråd Karl-Ymer Berglund (s) var mot en sammanslagning, men höll sig partiet trogen. Inget ont om n'Karl-Ymer. Malåborna protesterade under parollen \quotetext{Nej till diktatur} i 9 år tills Malå kunde bryta sig ut och bli det municipalsamhälle vi alla känner och älskar idag.

}

\small{
\textbf{Skräcködla}
\label{60dfc16d3c521fea596aa4c65bc1e3f5}
 n är ett djur i familjen ödlor, naturligtvis, men har till skillnad från sina kusiner \textsc{(se kusin s.~\pageref{f7f20d5744925e2e72e5524035a162be})} i livets frodiga släktträd förmågan att sätta skräck i de mest råbarkade typer. Det finns inte längre några skräcködlor kvar så vitt man kan förstå men enligt Sarah Palin var det inte länge sedan band av skräcködlor satte skräck i Klippiga Bergen (som för övrigt är ett konstigt namn, eftersom det väl säger sig själv att berg är klippiga).

}

\small{
\textbf{Skräp}
\label{75f1a5320951ea0dd9aa3c0eaba2c2c7}
 kan vara lite vad som helst och förvaras vanligtvis i uthus och fäbodar. På vinden går också bra. Många samlar på skräp för att det kan vara bra att ha.


 HEAD2:  Inom bilindustrin

 Franska bilar är oftast skräp.
 HEAD2: Inom musiken
 Fransk musik är oftast nån slags fusion-skräp.

}

\small{
\textbf{Skröna}
\label{c51cd220359f9f2755e98dcce2251e5c}
 En skröna är en historia som hittas på i stundens hetta när två personer sitter och finljuger \textsc{(se finljuga s.~\pageref{4eee5e7eab6f049c4084d3a5161016f9})}. Ofta har historierna till en början en viss sanningshalt, men med tiden brukar denna tänjas på allt mer. Skrönor är inte samma sak som andra berättelser, som t.ex. tv-serier, filmer och böcker (med Bibeln \textsc{(s.~\pageref{7de7d2a7d608c9a2044f50688bc63e27})} och allt av Milton Friedman och hans gelikar, som saknar sanningshalt från första början) utan dessa är ofta historier antingen helt och hållet påhittade eller sanna. Gamla människor är bäst på att dra skrönor då senilitet och demensanlag gör det svårt att berätta någonting överhuvudtaget utan att rucka på vad som faktiskt hänt och inte.

 HEAD2: Exempel från litteraturens värld
 \textit{ - Hooonom träffa jag en gång. Han sa till mig: Karl Andersson, du är den styvaste på hela Metallverken! Utan dig klarar vi oss inte! Så sa han. Vet du vad jag svara? Jo, jag sa, hörru Marcus Wallenberg, du är den styvaste av alla kapitalister, men utan dig klarar vi oss alldeles utmärkt. Så sa jag t'en. Tror du mig inte, jänta? }

 - Åsa Linderborg, \textit{ Mig äger ingen }

 HEAD2: Exempel från verkligheten
 Min \textsc{(se User: HratvinnFlygur s.~\pageref{31c19e82288ba7034064ee9b096bd7cf})} morfar är som alla andra gamla människor en jävel på skrönor. Senast idag berättade han om hur han under en tågresa lessnat på att personen i slafen under honom snarkade så förtvivlat. Morfar sträcker då ner handen och luggar den snarkande karln' i skägget, som slutar snarka. På morgonen visar det sig att den snarkande mannen är två meter lång och bitig som fan. Än senare visar det sig att samme snarkande monster har en biroll i filmen \quotetext{Vi hade i alla fall tur med vädret!}. [http://www.youtube.com/watch?v=sceXSC_lCYw] Och vid 2:09 har vi då detta vidunder till man i bild, han med polisonger och osedvanligt oljig t-shirt. Han har morfar alltså luggat i skägget. Inget gäck.

}

\small{
\textbf{Skuggan}
\label{689105de7de72e11f2e4105a436c8542}
 \textbf{Inom filosofin}
 Skuggan är inom CG Jungs psykoanalaytiska teori ett begrepp som representerar det onda och irrationella inom människan.
 \textbf{Inom plantagenäringen}
 Det är också något man kan sitta och dricka Southern comfort \textsc{(s.~\pageref{b5be4e79c92c85d6b964b26652ec81a2})} i.
 \textbf{Inom idrotten}
 Hockeyspelare i Lule Hockey,flest elitseriassist!
 \textbf{På gatan}
 En civilare vid namn Roger som följer efter en väldigt odiskret.

}

\small{
\textbf{Skumbjörk}
\label{54b61914b20887bd1705be742947c0e3}
 I Lycksele halvarktiska botaniska trädgård finns ett exemplar av den mycket sällsynta lappländska skumbjörken (betula pendula lyckseliensis). Förutom Lycksele är det bara beijings och torontos botaniska trädgårdar som har ett exemplar. Mindre än tvåhundra beräknas finnas i naturligt tillstånd, samtliga i trakterna av Lycksele och Norsjö.

 Carl von Linné \textsc{(s.~\pageref{5e8380bf6b7ce99678e6752b6d9e709e})} utpptäckte den mest av en slump, då den i friskt tillstånd ser ut som vilken björk som helst. [[File:skumbjork-lycksele.jpg\textbarthumb\textbarright\textbar200px]]

 Skumbjörken försvarar sig mot angrepp av den vinterlevande skalbaggen isbarkborre genom att i det angripna området utsöndra en särskild vätska, som vid kontakten med luft bildar ett tjockt vitt skum. Skummet kväver de barkborrar som inte hinner undan, och påskyndar också återbildningen av näver. Nedan en bild av hur det kan se ut vid kraftiga angrepp.

}

\small{
\textbf{Skurt Baronsson}
\label{7e0d2e4420f81352b7e7405821687394}
 är en spambot på Facebook som kännetecknas av att \quotetext{gilla} allt som alla dess vänner gör och skriver.
 Av någon outgrundlig anledning fortsätter många vara vän med boten.
 Vem som är botens skapare är omdiskuterat.

}

\small{
\textbf{Skäggpedagogik}
\label{093f086fe1ef425d98860eb4d65c362a}
 en uppfanns i Sverige i slutet av vårt förra decennium av Oliver Pasche. Han är producent av barnprogrammen \textit{Räkna med Skägg} och \textit{Stava med skägg} som har gått på SVT de senaste åren[[http://sv.wikipedia.org/wiki/R\%C3\%A4kna_med_sk\%C3\%A4gg]. Varje program inleds med skäggpedagogik, t.ex. skall tittarna räkna hur många som bär skägg på en buss \textsc{(s.~\pageref{e57167c19ed4b7c62a6527f85687cfab})}. Punchline för den senare serien är \quotetext{Kan du räkna till tio? Då kan du räkna med skägg!}. \textit{Räkna med skägg} fick sitt internationella erkännande när barnprogrammet vann det prestigefyllda priset Prix de Jeunesse 2008[http://www.prixjeunesse.de/], priset är en utmärkelse på världens största festival för barn- och ungdomsprogram. Männen med skägg spelas av Jonas Leksell, kanske mest känd som programledare för \textlessi \textgreaterMyror i brallan\textless/i \textgreater, en annan barnprogramsuccé [http://www.rabensjogren.se/Alfabetiskt/L/Jonas-Leksell/].
 Skäggpedagogiken blandas med sång och enklare matlagningsinslag, såsom hur man gör afrikamat från Afrika.


 \textit{Räkna med Skägg} och \textit{Stava med skägg} finns allt som oftast på SVT play:

 http://svtplay.se/v/2074615/rakna_med_skagg

}

\small{
\textbf{Skällsord}
\label{e0fc85fd2f5249557257965783ac136e}
 är uttryck för fientlighet eller förakt, typ \quotetext{Svin!}; ömsinthet, typ \quotetext{Din rackare!}; eller försök till maskering av totalt nederlag i fullkomligt sakliga diskussioner, typ \quotetext{Jävla idiot! Dig går det ju inte att prata med!}

 Bland skällsordens anrika överklass står sig fortfarande bland annat flottsäck, avskrapsfnas, smädetroll och talglymmel starka.

 Somliga anser att \quotetext{Vov! Vov!} borde räknas som skällsord, men de brukar vara rätt småvuxna och blöta i blöjan, och bör därför inte tas på allvar.

 [http://forum.skalman.nu/viewtopic.php?t=1754]

}

\small{
\textbf{Skäpparkrans}
\label{c49317fa7e72dceb7f445ab65c7ebcec}
 (\textit{Podiceps cristatus}) är ett slags ansiktsbehåring som skeppare och vissa andra har valt att gå, eller segla, omkring med. Skäpparkransen är som en krage men går längs skäpparens käke - från det ena örat till det andra. Den liknar för det otränade ögat förvillande mycket ett klassiskt skägg, men kombineras inte som det klassiska skägget med en mustasch \textsc{(s.~\pageref{78fe8e02985abb5090cb3f33ac2842d4})}, för det tycker skäpparen är onödigt och opraktiskt. Skäpparkransens funktion, förutom att vara ett identitetsstatement, är att hindra mat och vätska att komma ut ur skepparens mun \textsc{(s.~\pageref{6585f290ce92c3de5ff339920330e26f})} och drälla ner över hela bröstet. Istället fastnar överskottsmaten och -vätskan i skäpparkransen från vilken den kan avlägsnas så snart skäpparen kommit iland eller fått lite tid över för rekreation och nöjen.

 Alla som har skepparkrans är inte pederaster men alla pederaster har skepparkrans.

}

\small{
\textbf{Skåne}
\label{a01d1167b9dcd72e212d876d672db261}
 är Sveriges \textsc{(se Sverige s.~\pageref{b1999637949ed135b2ca03f3a38460cc})} sydligaste landskap. En vacker plats med böljande åkrar, monumentala kustvyer, wundersköna lövskogar och turning torso. Folk som bor i Skåne tycker om att se sitt landskap som motsvarande den amerikanska södern, fast i Sverige \textsc{(s.~\pageref{b1999637949ed135b2ca03f3a38460cc})}, och skriver gärna att de kommer från \quotetext{The dirty south} på sina facebook-profiler. Liknelsen är inte helt orimlig då Skåne, likt sydstaterna i USA, utgör en bastion för grisfarmande och främlingsfientlighet. Antagligen vill skåningarna själva mena att likheten består i den fräcka laglösheten, rebellandan och den generella go against the rules-attityd som är förhärskande i den romantiserade bilden av den amerikanska djupa södern. I Sverige \textsc{(s.~\pageref{b1999637949ed135b2ca03f3a38460cc})} står dock den andan främst att finna i landets norra gräns mot Finland \textsc{(s.~\pageref{631d44eaa1254ff71a1e11ba021d1266})}, där laglöshet råder i ordets sanna bemärkelse då snutar inte existerar norr om Boden \textsc{(s.~\pageref{7c5dfb91b1d55bff98ec6d4faf83976b})}.

 HEAD2: Folk
 Det finns tre sorters skåningar.
 HEAD3: Alfaskåningen
 Alfaskåningen, är en fryntlig filur som gärna skrattar och pratar (eller \quotetext{blurrar} som infödingarna kallar det) på sitt underfundiga, diftongerade skorrande vis med vemhelst de stöter på. De är gästvänliga och väl belästa inom alla artes liberales, med särskild fokus på quadrivium - Alfaskåningarna är inte sällan duktiga på musik! Hos Alfaskåningarna är det inte heller ovanligt att studera många olika språk för att kunna vara utåt och sociala över så många nationella barriärer som möjligt (en del kan till och med latin, utifall de skulle stöta på nån gammal romare).
 HEAD3: Betaskåningen
 Den andra sortens skåning kallas Betaskåningen. Den sortens skåning är fet, lat, ful, snål, okunnig, vräkig, högljudd och dum. Till skillnad från Alfaskåningen, vars dialekt är extremt gullig och tilltalande, låter det som om nån bronkithostar en i örat när man hör Betaskåningen babbla på om hur nästa års svinskörd hotas av den ökade invandringen och Sveriges \textsc{(se Sverige s.~\pageref{b1999637949ed135b2ca03f3a38460cc})} alldeles för låga straffsatser.
 HEAD3: Cykelhippies
 
 En tredje kategori man kan finna i framförallt Malmö är inte skåningar i ordets egentliga bemärkelse då de sällan eller aldrig är födda och uppvuxna i Skåne, de talar i regel ej heller skånska.
 HEAD2: Fä
 HEAD3: Grisar
 I Tomelilla kommun finns det fler grisar än invånare, det tål att tänkas på.

 HEAD2: Sevärdheter
 \begin{itemize}
 \item Turning Torso
 \item Örkelljunga
 \item Carl P Herslow
 \end{itemize}

}

\small{
\textbf{Slagg}
\label{99084d72b557047a46c3b9ba2142afb1}
 är ursprungligen de ämnen som inte är metaller som blir över i en metallurgisk process. Förutom att bygga husgrunder av är det tämligen värdelöst. Begreppet används idag för att beskriva restprodukter i alla former av processer och Vegemite är exempelvis en slaggprodukt som bildas vid framställning av ölet Foster’s \textsc{(se Australien s.~\pageref{e727d8d1b3162a732c7f706d55de64f3})}. De mest kända slaggprodukterna är förmodligen brugden \textsc{(se brugd s.~\pageref{d6b6b68506b8f1daad3a2ddbfaf8d863})}, som uppstod ur de överblivna delar evolutionen inte behövde ha för att skapa alla råa hajar, och Belgien \textsc{(s.~\pageref{f79ffe9e826a19f9f6a446c90e21c4e3})}.

}

\small{
\textbf{Slan}
\label{caaad522de864ab45ed679c4a16edd8d}
 kan betyda en massa olika saker, men är ungefär synonymt med dåligt eller slarvigt.

 Som adjektiv eller adverb (slanigt) kan det känneteckna ett ting eller företeelse som inte är så jävla smutt \textsc{(s.~\pageref{d9114ffee4f2dcee302ae2b19ce5eea9})}.

 Exempel:

 \quotetext{Det var ganska slanigt att inte städa på tre år}

 \quotetext{Den här danska lättölen \textsc{(se Dansk lättöl s.~\pageref{3981afb990a974a3b4613a470e51e747})} smakar så jävla slanigt}

 Det kan också vara ett substantiv, men måste beteckna en levande varelse.

 Exempel:

 \quotetext{Anderssons grabb har gått och blivit ett riktigt slan}

 Slan kan också vara ett verb.

 Exempel:

 \quotetext{Nu har jag slanat i soffan hela dagen. Skitliv.}

 Rent geografiskt är användandet av ordet centrerat till Närke och Bergslagen \textsc{(s.~\pageref{2844a3a8251745f1f093b6f86f909183})}, men sprids med diasporan.

 En kulturell referens till ordet är Krigshot - Slanig snut, och innan de spelade låten på Augustibuller 2006 sa Jallo: \quotetext{Anders åkte på en fortkörningsbot på vägen hit, två  tunkor \textsc{(se brakare  s.~\pageref{da8590943fa645cfceaa235a83d1d797})}, så nu kör vi Slanig snut}.

}

\small{
\textbf{Slangopedia}
\label{13c033c5151b27aeb3c561b52a5163e8}
 En annan mycket fulare wiki än Nissepedia.

 [http://www.slangopedia.se/ Slangopedia]

}

\small{
\textbf{Slanguttryck för att gå iväg}
\label{c9ce559706f53a986be8946777e2b798}
 I vårt svenska språk finns ett stort antal valbara slanguttryck som alla handlar om att ge sig av. Det oftast använda ordet är att \quotetext{dra,} och detta ord har en geografisk spridning som är lika stor som sveriges totala yta - större, till och med, om man räknar med det svenskspråkiga Finland \textsc{(s.~\pageref{631d44eaa1254ff71a1e11ba021d1266})}, men det gör vi inte här på Nissepedia. Nåväl! Nedan följer en lista på slanguttryck som är synonyma med \quotetext{att dra}:

 \begin{itemize}
 \item Sticka
 \item Gitta
 \item Blada
 \item Glida
 \item Rulla
 \item Pulla (som i eng \quotetext{pull out})
 \item Pipa
 \item Pysa
 \item Kila
 \item Knalla
 \item Knata
 \item Dunsta
 \item Slajda
 \item Sjappa
 \item Avvika
 \item Glisha
 \end{itemize}

}

\small{
\textbf{Slash}
\label{9fbbaa4cc515bc46e0c12e82a31df736}
 Rockikon

}

\small{
\textbf{Slayerklass}
\label{81c5c862682cf8f586d4d8fa28b4607d}
 är ett slags kvalitetscertifikat på musik som kan utfärdas av vem som helst. Utfärdaren garantrar därmed att musiken i fråga är så bra att den lika gärna skulle kunna varit skriven av Slayer.

}

\small{
\textbf{Slentrian}
\label{6ce6f6f66ddb59ccb7a56a46bd9f590d}
 Lianer som växer i nerförsbackar. Ofta i klasar och ganska slarvigt.

}

\small{
\textbf{Slips}
\label{61c7bd51d579af09c10142f4b55c848c}
 En slips (ibland också kallad ”tjänsteman”) är en person som får betalt för att utföra sysslor som en vanlig pantad knegare \textsc{(se vanliga pantade knegare s.~\pageref{98d0a7dac261debb934a16b7041ef22f})} inte klarar av. Det kan till exempel vara att fylla i ett papper så att knegaren vet hur många spikar som ska slås ned i en bräda, eller hur mycket tunnare en redan tunn ståltråd ska göras. För att slipsen ska klara av detta är det viktigt att han får incitament \textsc{(s.~\pageref{f9896a922c4b9345ceebc37009eaf545})} med jämna mellanrum. Annars kanske han fyller i papperet fel så att knegaren slår i för många spikar eller gör tråden för tunn. När det händer tvingas slipsen sparka tio av knegarens arbetskamrater för att effektivisera produktionen, och det tycker den stackars slipsen naturligtvis inte är roligt.

 Vill man titta på en slips kan man gå till den matsalen som har mjuka dynor på stolarna eller till solariet. På julfesten kan man skymta dem strax nedanför scenen där chefen sitter under fördrinken \textsc{(se fyra s.~\pageref{7bdb5385ce8e0b1cbc7c15b1d71e8e7d})}. En halvtimme senare brukar de flesta ligga strypta och ihopbuntade på toaletten  och då är de inte längre så mycket att se.

}

\small{
\textbf{Sludge}
\label{2ccd23d1cd0f95dc6984215a1f1b31ca}
 (eng. \quotetext{sörja,} \quotetext{rötslam,} \quotetext{gyttja} sv. uttal /sludsh/) är en musikstil, av och för män, som karaktäriseras av tunga gitarriff, lågt tempo och ofta brutal sång då sådan finns. Sludgens klassiska hemorter är södra USA samt England. Från England kommer band som Iron Monkey och Moss. Från Södra USA kommer Eyehategod, Acid Bath, Buzzov•en, Weedeater och Crowbar.
 Sludge består till lika delar av Black Sabbath och Black Flag med lite gammal ful-crust på toppen. Sludgens favoritfärg är svart. Och lite grön.
 Se här, ett typiskt exempel på Engelsk ful-sludge [http://www.youtube.com/watch?v=sx3qpGd_4V4].

}

\small{
\textbf{Släktträffsberusning}
\label{d2f215cc09e9d611046202162781f972}
 Ett stadie där du ligger med vem som helst utan hänsyn till fysiska/psykiska defekter, konsekvenser och eventuella släktskap.
 I vissa kretsar är tillståndet permanent.

}

\small{
\textbf{Slätt}
\label{a9cde01124ca41f23d6044b3ba27b979}
 En slätt är ett platt landskap utan större urbana inslag. Riktiga slätter bör domineras av endast en typ av vegetation, med fördel gräs, sand \textsc{(s.~\pageref{88336b5bb2a1cc21bac7cf33fd451270})} eller grus. Vid utformande av sin egen slätt bör man välja material utefter vilka aktiviteter man planerar på den. För cricketspelande rekomenderas grässlätter medan ökenrallyn är bäst lämpade på sandunderlag.



 Exempel på personer som ridit på slätter är ryska rövare \textsc{(s.~\pageref{196e0458db510192146b2f885a9a3fee})} och Evert Taube.

}

\small{
\textbf{Smegma}
\label{01d70ea15e687416cbced7ff781ee170}
 är en vital ingrediens i ostiga såser, och kommer mestadels från Italien då det är högre salthalt i luften där än någon annanstans i världen. Smegma produceras genom bakteriekultur i mörk och fuktig omgivning en längre period.

 Man kan även framställa Smegma kemiskt och i livsmedel förkortas det med E1520.

}

\small{
\textbf{Smides städet}
\label{e1ce96e72e9876fbf7e9083ebde5e920}
 Smidesstädet \textsc{(s.~\pageref{baaef6be35a07ab14d7c3e01ef3d4806})}

}

\small{
\textbf{Smidesstäd}
\label{c3fed5991476f5dfb387dac8e88be084}
 et är förbannat tungt och är en typ av arbetsbänk för smidesarbeten.
 Precis som järnspettet \textsc{(se järnspett s.~\pageref{6cbe55f18d91c10e3307681ab810fd74})} så är städet homogent, dvs tillverkat av 100\% solitt stål.

 HEAD3: Tillverkning
 Man tillverkar egentligen inte smidesstäd utan letar upp ett järnvägsspår, kapar ur en lämplig bit (ca 80Cm) med en vinkelslip och monterar sedan rälsbiten på en stubbe i lämplig höjd.

 Det finns 1000-tals kilometer järnvägsräls i Sverige \textsc{(s.~\pageref{b1999637949ed135b2ca03f3a38460cc})} så ingen kommer att märka om det fattas en liten bit.

}

\small{
\textbf{Smidesstädet}
\label{baaef6be35a07ab14d7c3e01ef3d4806}
 Smidesstäd \textsc{(s.~\pageref{c3fed5991476f5dfb387dac8e88be084})}

}

\small{
\textbf{Smuldegspappor}
\label{3e905cc533f4dafa7ddeae86b74a40e4}
 är den lägre klassen i samhällsgruppen \quotetext{Moderna pappor}. En smuldegspappa tvingas leva på smulorna av surdegspappornas frodigare liv.

}

\small{
\textbf{Smulpajspappor}
\label{140638447e8844d08f7f556f89d1c0ae}
 Den bästa pappan.

}

\small{
\textbf{Smutt}
\label{d9114ffee4f2dcee302ae2b19ce5eea9}
 kan betyda nästan allt mellan himmel och jord och är ett av de mest välanvända orden i Norrtälje \textsc{(s.~\pageref{7527f7dad9445013a559dc7e2a91f3b3})} Kommun. Smutt kan betyda, förutom en liten klunk, en värdering synonym med praktiskt eller bra, trevligt, härligt och så vidare. Att vara en smutt snubbe eller lirare kan alltså innebära att man är en trevlig, härlig och bra person, men att vara smutt kan också betyda att man är salongsberusad eller lite stenad efter att ha rökt en marijuana-cigarett (också kallad hövding, gås, fuling, joppe, jolle \textsc{(s.~\pageref{4fe195f73917395e8a5851dc036ef8bc})}, Yoda osv). Vidare innebär att småsmutt att man intar horisontalläge men varken sover eller är särskilt aktiv. Man kan dock samtidigt småsmutta och deltaga i samtal av lite enklare natur om dessa inte blir allt för långa. En småsmutt är dock inte någon som småsmuttar, utan en liten grej som används istället för en stor grej, så som en liten smidig sax som används istället för en stor sådan, eller en liten båt som drags efter en större båt och som används för kortare turer, till exempel från den stora båten och in till land om man där bara ska köpa lite kex och en snusdosa var till ens polare som hänger kvar på den stora båten och kanske småsmuttar medan man själv får allt ansvar för att hålla på och jäkta in till land för att proviantera.

}

\small{
\textbf{Smyfascist}
\label{5e08489eae16ff5dd56e66885db1c6d0}
 Smygfascist \textsc{(s.~\pageref{2f5d6c7dca93e29166fa4db38ca374e6})}

}

\small{
\textbf{Smygfascist}
\label{2f5d6c7dca93e29166fa4db38ca374e6}
 är den bästa låten på KSMBs debutplatta Aktion.
 Ett tag såg den ut att kunna bli Sverigedemokraternas \textsc{(se Sverigedemokraterna s.~\pageref{44e1558e17b71c9b066fe6d5e1f2cc63})} officiella val-låt, men så kom någon på att det ju var det här man ville dölja. Smygfascismen alltså. Den fick således se sig besägrad av låten \quotetext{Jimmie Åkesson - tjalalalala} som vi alla känner till så väl idag.
 Category:musik \textsc{(s.~\pageref{38cce583d2d3675d645425cb435aa2bb})} Category:fascism \textsc{(s.~\pageref{4027d26318feac6be4b1f23fbb14d46b})}

}

\small{
\textbf{Smygsexist}
\label{8e43fca8e37e789cc3ec948dcc04a5a2}
 En smygsexist är en person som gömmer sig bland buskar och ropar \quotetext{VICKA BRÖN HÖRRÖ} till förbipasserande kvinnor.


 Category:Folk och personlighetstyper \textsc{(s.~\pageref{0992ea186560ae08b691fc79cca9cded})}

}

\small{
\textbf{Småbyxa}
\label{13b74e04e94c0c93e01e835f54a271c2}
 är ett plagg som är populärt bland små barn \textsc{(s.~\pageref{5dfcc0aab2f3db925b2d51ba73e48946})} (som ju som bekant inte har nåt som helst folkvett alls) och kvinnliga skådespelare och sångerskor från the United States of America \textsc{(s.~\pageref{ade6b3bd5e720abb20ed8a9a4c6b9ae8})}. Småbyxan är ett plagg i kategorin hygienplagg som avdelar sig från kategorin värme- och komfortplagg på så vis att småbyxan är där mer eller mindre enbart för syns skull.

}

\small{
\textbf{Småhetsvansinne}
\label{09beca787ad2414755414613ca522605}
 är likt sin storasyster, storhetsvansinnet \textsc{(se storhetsvansinne s.~\pageref{2f9c0ea6231e1de87c97eab41410c795})}, ett sjukdomsliknande fenomen och infinner sig hos personer som är ofantligt tjocka, har för korta armar \textsc{(se t-rexarmar s.~\pageref{0b2dbf0eb2888d887370538902e974d4})} eller på annat sätt utmärker sig som olik normen för hur en karl eller kvinna \textsc{(s.~\pageref{9a7760b2521c3471c47cd5d789a2d324})} ska se ut. Vidare är småhetsvansinne vanligt hos sådana som försökt sig på en viss karriär bara för att misslyckats med detsamma, genast eller efter en utdragen process kantad av penibla ögonblick och många motgångar .
 Den som vill lära sig att upptäcka symptom på småhetsvansinne bör hålla följande lista färskt i minnet:
 HEAD2: Symptom på småhetsvansinne
 Den insjuknade:
 \begin{itemize}
 \item Har för små kläder för att se stor ut eller för stora för att se mindre ut
 \item Lyssnar på hög musik i bilen
 \item Har en sänghimmel ovanför sängen (detta representerar för den sjuke livmodern och ett enklare liv inuti mammans mage)
 \item Undersöker sin stol efter avklarad session, likt Stig Helmer i \textit{Riget}.
 \end{itemize}

}

\small{
\textbf{Småskurk}
\label{c25031c5d78d9ad6fae8ab8f08d5e9dd}
 En småskurk är en brottsling som håller sina lagöverträdelser inom acceptabla ramar. Tar man exempelvis påtår på kaffet fast det egentligen inte ingår så är det okej. Lika så är det lugnt att man snattar billigare saker i mataffären så länge det inte är en konsumbutik \textsc{(s.~\pageref{70e4875f7c2c177596305006e46b7ca9})}.


 En småskurk kan även fiffla med deklarationen så länge den inte är rik. Rika personer kan aldrig vara annat än storskurkar. Det är allmänt accepterat att vara småskurk, vissa ser det till och med som lite charmigt.
 Jmfr filur \textsc{(s.~\pageref{e308f4e2553faf188385f17ebda05242})}

 HEAD2: Exempel på kända småskurkar

 George Best \textsc{(s.~\pageref{f4288789b1401dc1595a0cb6f22d5b93})}
 Torbjörn \textsc{(s.~\pageref{c3e6fb6fb2b655457597f063bd9392e8})}
 Lemmy \textsc{(s.~\pageref{6cc2f8758343439728f308f08a4a8fad})}
 Petrus de Dacia \textsc{(se petrus de dacia s.~\pageref{e07b2cf719fa237191665c127c7080c2})}
 Leila K
 Cockney Rejects

}

\small{
\textbf{Småstadsalternativ}
\label{9de395de5ead3b4d90078ac47b1205a9}
 Ett lapptäcke av influenser utmärker den småstadsalternativa stilen. En bombarjacka med Marilyn Manson-backpatch, kängor med neonlila skosnören och ravebrallor kombineras utan förbehåll. Åt fanders med de strikta regler som vanligtvis styr subkulturer. I småstäder som Bollnäs, Härnösand och Mariestad finns ingen respekt för stilmässigga regelryttare. På ett sätt kan man bara berömma den totala skit-i-allt attityd som präglar småstadens rebeller. Samtidigt ser det ju för taskigt ut.

}

\small{
\textbf{Smörgåstårta}
\label{b81fe66f0f43489f730fd6baa91a12f7}
 Höjden av lyx.

 HEAD2: När man äter smörgåstårta
 Börje har fyllt 50 och vid 9-fikat på fredag tar han fram en smörgåstårta ur kylen.

 HEAD2: Hur man tillagar smörgåstårta
 Bred ett gäng mackor. Lägg dem i en trave. Sänk försiktigt ned traven i ett bad av majonäs. Garnera med en dillkvist och undervattensinsekter.

}

\small{
\textbf{Smörskrove}
\label{c3ec1fc646dfd34ddd483f8031d649c9}
 En smörskrove har samma grundrecept som calskroven \textsc{(se calskrove s.~\pageref{84ff54e779ee49fdad21e17c20f14453})}, men istället för ett skrovmål används en smörgåstårta \textsc{(s.~\pageref{b81fe66f0f43489f730fd6baa91a12f7})}. Det här kräver lite tålamod vid tillverkning och är därför strikt begränsad till högtidliga tillfällen.

}

\small{
\textbf{Snabbdoppa}
\label{017d4cf4a8dc7d8d4801b949df3e3f6e}
 Att snabbdoppa är att snabbt doppa något eller någon i vätska. Snabbdoppning har institutionaliserats genom det kristna dopet och är en del av processen när man tillverkar en kaffefisk \textsc{(s.~\pageref{af1258c212f378e0d974ac807a91ab79})}.

}

\small{
\textbf{Sneakers}
\label{a1743c0d39461290efc551490aafc1e2}
 är ett hipsterord lånat från engelskan och betyder jumpaskor.

}

\small{
\textbf{Snefotad ultrapelikan}
\label{0def09852ec31cb5af0c38180b411782}
 (\textit{Pelecanus conspiratoris}) är en art inom familjen pelikaner \textsc{(se pelikan s.~\pageref{ecf1b944439a171dfe1163001feeed19})}. Den lever ett rätt stillsamt liv i Mogadonien och några småöar runtikring på fisk och vatten, ungefär som fattigpensionärer. I jämförelse med andra pelikanarter är den snedfotade ultrapelikanen ganska stor och kan väga upp till 15 kilo. Dess maffiga vikt i kombination med dålig flygförmåga och att den har så sneda fötter att den inte kan gå så fort har medfört att den inte klarar sig speciellt bra i trafiken. Så nu är den rätt utrotningshotad \textsc{(se utrotningshotade djur s.~\pageref{24a427a5537c2c8918cfa213ae099a74})}. Men självklart har den en stor pungliknande säck under näbben. Den har fullt med ohyra i fjäderdräkten så man bör inte klappa den, men den är väldigt social så att sällskapa går bra.

 till vänster är betydligt mer avslappnad i sin situation.]]

}

\small{
\textbf{Snesegla}
\label{fcef03191199b0ff1f582a2971955b25}
 Att snesegla är det samma som att hamna på glid och är något som  strulputtar \textsc{(se strulputte  s.~\pageref{21651c95306d1b1e281443f8620910da})} ofta gör. Början på en sneseglats är ofta att den unga människan börjar skolka från skolan och dricker folköl i någon park hela dagarna.

 Ett exempel på en sneseglare är Christer Petterson. Andra vanliga namn på sneseglare är Ronny, Conny och Johnny.

}

\small{
\textbf{Snett}
\label{64b2eafa388ee50a226adc9013644f08}
 Beskriver hur något förhåller sig till något annat, olinjärt, typ. Tavlan hänger snett ( i förhållande till taket/väggen ) Du har kommit snett i livet = följer inte de vanliga sociala normerna (se också strulputte \textsc{(s.~\pageref{21651c95306d1b1e281443f8620910da})} och sneseglare) \textsc{(se snesegla s.~\pageref{fcef03191199b0ff1f582a2971955b25})}. Peter Jihdes lugg hänger snett. Klockan är snett efter tio.

}

\small{
\textbf{Snkrbrlr}
\label{c400f66def68999be18b4ab1ee70c4ff}
 är en förkortning för \quotetext{snickarbraller.}

}

\small{
\textbf{Snowjoggers}
\label{ee340dd9a61f36aaa0f7581db6e3d374}
 kallas de vinterskor som användes utbrett bland de lägre samhällsklasserna i stora delar av västvärlden under slutet av 1900-talet. Förespråkarna menade att de var varma, bekväma och lätta att ta på medan motståndarna menade att de var bedrövligt fula. Utbredd mobbning mot dess användare ledde slutligen till att skon försvann från marknaden.

 Snowjoggers är numera förbjudna inom EU \textsc{(s.~\pageref{4829322d03d1606fb09ae9af59a271d3})}.

}

\small{
\textbf{Snus}
\label{2deb44bfdec3ef2d22b93cdba81ac183}
 Ett slags kräm gjort av malda löv som används för att fingera tillhörighet till arbetarklassen. Snus anses felaktigt vara ett prestationshöjande preparat hos isolerade grupper så som människor som känner kopplingar till Piteå, skinheadscenen \textsc{(se skinhead s.~\pageref{a54bc1b5d472b5afed8e84004b6441c4})}, innebandykultur och polisutbildningen. Men det är det inte.
 HEAD2: Logistik
 Snus levereras till konsumenten i form av puck-formade dosor som i sin tur förpackas av mottagaren i ett par jeans.

}

\small{
\textbf{Snusbrist}
\label{8c01b9e115387e4c9e5d26839746819b}
 är ett livs- och själhotande tillstånd som infinner sig hos s.k. \quotetext{etniska} svenskar och orsakar obeskrivligt lidande. I vissa fall kan det indirekt orsaka dödsfall genom att individen \textsc{(se individ s.~\pageref{41beed76a0af9b4f550f7ebdecd3e700})} drivs till vansinne och tar sitt liv genom helt sonika fylla \textsc{(se bärsfylla s.~\pageref{9380b60f9ee744b9acf978fe6f1a9545})} snusgropen med en kärve hagel.

}

\small{
\textbf{Snutkaffe}
\label{f00c99581a24572c894a356b429dc005}
 Kaffe \textsc{(s.~\pageref{a51a0cac0ce374a853d2359417debc28})} utan socker, mjölk, grädde, kask \textsc{(se kaffe kask s.~\pageref{f017294802d98446f6b5e9c0cc37d6a1})}, smör, honung, nånting. Bara kaffe. Ibland med en ryssfemma \textsc{(s.~\pageref{d974e0811fe7a4d49a9062d33b66a88d})} i. Helst i pappmugg.

}

\small{
\textbf{Snutnamn}
\label{808c06b0d4ef077deac4ed602f570fa9}
 Ett Snutnamn är ett ”nytaget” efternamn som gärna låter just nytaget. De kallas Snutnamn för att de är vanligt förekommande bland just poliser, men återfinns även i många andra sammanhang där efternamnet tillmäts stor betydelse. [http://snutnamn.blogspot.com]

 HEAD2: Exempel på snutnamn
 Carnestedt
 Fuglesang
 Gärdestad
 Orustfjord
 Tigercrona
 Ärlestål

}

\small{
\textbf{Snutröv}
\label{f9a35b35e7ef367d07be9bd1e9357f83}
 En snutröv är lika bred som en hockeyröv \textsc{(s.~\pageref{c904e78c8991794c8d598d44f0494f9c})} men kännetecknas av att den är extremt ihopknipen.

}

\small{
\textbf{Snälla killar som aldrig får ligga}
\label{630d0607c17e587ef244461bbafe9b4b}
 borde egentligen heta \quotetext{Killar som själva tycker att de är snälla men aldrig får ligga}, men det är i längsta laget, även för Nissepedia \textsc{(s.~\pageref{62400dadecd90cb5cd39062abe5a3e4a})}. Dessa killar har lite olika utgångspunkt, en del är bittra och säger att tjejer bara vill ha snubbar som behandlar dem som skit. Andra är mer oförstående och förvirrade. En del av dem skriver krönikor som heter \quotetext{Vi som aldrig sa hora}, typ Ronnie Sandahl, medan andra använder denna position för att starta akademiska karriärer, t.ex. Hans Andersson, doktor i sociologi. Snälla killar som aldrig får ligga borde egentligen bara hålla käften.


 HEAD2:  Se även
 \begin{itemize}
 \item Feministknepet \textsc{(s.~\pageref{b5a2deaae58d913dc69ee852f19bcb17})}
 \item Kvinnlig författare-knepet \textsc{(s.~\pageref{2df7cf3cc32dd55b7c833e6220d42c4a})}
 \item Lurmus \textsc{(s.~\pageref{23f18296e8df765844117b713fb4613f})}
 \end{itemize}

}

\small{
\textbf{Snälla tjejer}
\label{0aa3c8d228095f6fc73eccb8c92b8c81}
 I 9 av sju fall  lurmöss \textsc{(se lurmus s.~\pageref{23f18296e8df765844117b713fb4613f})}

}

\small{
\textbf{Snärt}
\label{6fb9ccfbd5699d12ff8d04b2a27852fb}
 En snärt är en måttenhet som betecknar det som i vardagligt tal uttrycks som \quotetext{lite grann}. En centimetertjock skiva falukorv är en korvsnärt.

 Se även Parisare \textsc{(s.~\pageref{5aca28013b9a7e4088e7fb228f3e4827})}

}

\small{
\textbf{Snåltarmen}
\label{ae50b82f824beabd2246d5aa9c7ac61e}
 är det organ i människokroppen som ser till att pensionärer alltid vill ha kvitto i affären trots att de har för dålig syn för att se vad som står. Snåltarmen är också skälet till att man blir grisfull \textsc{(s.~\pageref{80fc21ba5a45f2d0cd24855d78fa7246})} så fort det är bjudsprit \textsc{(s.~\pageref{3a68804bcc7740bc3fd426c893757a06})}.

}

\small{
\textbf{Snöre}
\label{30b7be64f820e5ec00397848f6f8d1c8}
 Jo-jo med snöre-15 öre

}

\small{
\textbf{Snöruta}
\label{af2d455c197464bfa4dad3c1675aef5d}
 Förr i tiden var det fattigt, mycket fattigt i Norrlands inland. Ibland så fattigt att man skickade barnen \quotetext{till fjälls}, som det hette. Det vill säga satte ut dem i snön i 35 minusgrader, och övergav dem där. Bättre för dem att frysa ihjäl fort än att svälta ihjäl långsamt var den allmänna åsikten. När det var lite mindre fattigt firade man ibland på söndagen med snöruta. Snörutan var ett bakverk som bestod av en utskuren bit snö med några torkade blåbär eller lingon på. Kulturhistoriska djupanalyser har konstaterat att snörutan är den ursprungliga inspirationskällan till dagens isglassar.

}

\small{
\textbf{Snöskor}
\label{abaa4c229e523e7c888d3e00ca0d6986}
 är ett fenomen som nästan alla känner till men som få har sett i verkligheten. Har man någon gång besökt ett badrum \textsc{(se Historiska händelser i badrum s.~\pageref{883e86693d6804a30ae0d22311449058})} i ett hushåll där det vistas barn har man med all säkerhet även läst \textit{Kalle Anka \& Co}. Om tidningsexemplaret är utgivet under vinterhalvåret är sannolikheten att ett par snöskor figurerar i någon av serierna ungefär lika stor som att Björnligan äter sviskonpaj \textsc{(s.~\pageref{ce8e792cda4c878050cf537f151667ff})} till middag. Snöskorna utgörs av två stycken tennisracketar med varsin läderrem genom nätet \textsc{(se World Wide Web s.~\pageref{3b7d657e8b7bf25a9d524b60d9bb17df})} så att de kan fästas utanpå ett par vanliga skor. Varför Kalle Anka \textsc{(se Kalle anka s.~\pageref{64db68f686a0ca4d9d641061cb3fdf13})} måste ha snöskor när han redan har simhud mellan tårna kan man verkligen fråga sig, men eftersom denna artikel mest handlar om sannolikhetslära och inte biologi lämnar vi den frågan därhän. Sannolikheten att dina föräldrar kommer berömma ditt initiativ att tillverka ett par egna snöskor av deras tennisracketar är närmare nalta \textsc{(s.~\pageref{fff42153008c054c6224eed0b8a1374b})}. Så bry inte din lilla hjärna med sådana projekt utan sitt istället kvar på muggen \textsc{(se tysk toalett s.~\pageref{da70822255031d2f882278fd6080bb5f})} och läs en sida skrattfnatt \textsc{(s.~\pageref{e6493d2d7abe72301d48eaf5854d14f7})}.

}

\small{
\textbf{So ends our night}
\label{61a7155d8ddb3927798751e606ad6491}
 är en låt av bandet Last Rights som berättar en modern saga från Bostons gator och torg. De handlar om hur ett gäng med en man kallad \quotetext{Choke} i spetsen är lediga men väldigt uttråkade, så de går och kollar på ett band. Bandet är väldigt bra men killarna är egentligen sugna på att slåss. De slåss lite, men måste gå hem ganska fort. Sen är det dags för nästa helg och killarna har inget att göra, de hänger ut och dricker Coca-Cola, för de gillar inte sprit - Nej tack! Sen går killarna till Kenmoretorget och hamnar i slagsmål, som de såklart vinner. Och så slutar killarnas kväll.

}

\small{
\textbf{Sober-Jimmy}
\label{62629d44a92716a33e051e9a6c04d0d4}
 är en välkänd musikprofil från Skinnskatteberg \textsc{(s.~\pageref{f0666ee995da080da55f5f6892fe3dcc})}. Han grundade No Fun At All tillsammans med Fakta \textsc{(s.~\pageref{fce663ae73dc87a727148bc3b94d1ffa})} och spelar på bandets största studiostund \textit{Vision}. Därefter kände Sober-Jimmy att No Fun tog upp för mycket tid och hoppade av för att helt fokusera på sitt andra skötebarn Sober. Där gick Sober-Jimmy en lysande framtid till mötes och släppte bland annat legendariska EP:s så som \textit{Blowjob} och \textit{Snowbored}. I nuläget ligger bandet tyvärr på is, och Sober-Jimmy har skaffat sig ett riktigt jobb.
 på gästsång. Sober-Jimmy med gitarr \textsc{(s.~\pageref{a08bf8420208934bc59c7ed7385d4308})} i mitten.]]

}

\small{
\textbf{Sockerdricke}
\label{246cc3f48a4465f8b322e8b9f4d85525}
 När man inte har någre penger, då \textit{kan} man inte köpe sockerdricke, och när man har penger, då \textit{får} man inte köpe sockerdricke.

 Källa: Ohängd unge i Lönneberga.

}

\small{
\textbf{Solidaritet}
\label{0731d63872b993832c974f802fd1bcd6}
 en är det enda som kan rädda oss från varandra och oss själva.

}

\small{
\textbf{Somalia}
\label{391fbc5b5081b1ae00b4faaae91d3a58}
 finns egentligen inte då det inte existerar någon stadig hand som håller ihop skiten.
 Engelsmännen försökte och gav upp. Konstigt nog vill aldrig rastafarianer \textsc{(s.~\pageref{76ad9ecb5de402390ed9c33a2b845594})} utvandra till detta pundarnas alldeles egna land trots att det är fritt, löst och ledigt och dessutom ligger i Afrika.

 Ett annat sätt att se på saken är som följer: Etiopien ockuperar den del av Somalia som kallas Ogaden och där finns dyrbara oljetillgångar som exploateras av utländska företag, däribland Lundin Oil. Man tog Ogaden med våld, vilket ledde till att människor flydde till Mogadishu och andra delar av Somalia, vilket i sin tur har skapat en ohållbar ekonomisk situation i dessa regioner. Denna historiska process har skildrats på ett briljant vis i \textit{Maps} och andra romaner av den Somaliske författaren Nurrudhin Farah som tillhör världslitteraturens tungviktare. Ockupationen av Ogaden är intrikat sammanvävd med andra komplexa skeenden i det postkoloniala Afrika. Det finns en stor Somalisk diaspora, som den person som skrivit ovanstående lätt efterblivna beskrivning av Somalia antyder, och en av de grupper som befinner sig i exil är somaliska intellektuella, en grupp som till exempel den redan nämnda Farah tillhör.

}

\small{
\textbf{Sombrerosamlare}
\label{daddc0e3cac86a8c2d8984272c9cab5b}
 är människor som tycker att det absolut mest givande som finns är att köpa en ny sombrero och lägga på sin gigantiska hatthylla tillsammans med dom andra i sin samling. Eftersom sombreros har ett sjukt omfång måste hatthyllan vara typ en meter djup, men det rör inte sombrerosamlaren i ryggen. På Internet kan sombrerosamlaren kommunicera med likasinnade och diskutera frågor relaterade till ämnet. Ibland kanske någon väcker frågan om det inte vore bättre att samlarna använde sina sombreros istället för att låta dem ligga och damma inomhus. Debatten böljar hit och dit men slutar allt som oftast i konsensus om att man är rädda att de kan gå sönder.

}

\small{
\textbf{Sommar}
\label{4365ed528b682a00b02d5daf05a03b0d}
 är den årstid som är mest Bruce Springsteen. Våren och hösten är inte alls lika mycket Bruce Springsteen och vintern är inte särskilt Bruce Springsteen över huvudet \textsc{(se huvud s.~\pageref{e906cd95a540df9b16d0460fb4cf0adc})} taget. På sommaren har tjejerna sommarkläder, och det tycker Bruce Springsteen om. Då brukar han åka ner till floden och ta sig ett dopp, rejsa längs gatorna i sin bil och springa efter tjejer från New York. Det är som att Springsteen liksom brinner när det är sommar.
 HEAD2: Kännetecken
 Sommaren kännetecknas av att det är betydligt varmare då än under resten av året. Många har likt Bruce Springsteen jeans-[[kortbyxor\textbarshorts]] \textsc{(se jeans s.~\pageref{a0f2589b1ced4decbf8878d0c3b7986f})} på sommaren.
 HEAD2: Sommar i Australien
 Som med mycket annat har Australien \textsc{(s.~\pageref{e727d8d1b3162a732c7f706d55de64f3})} valt att göra tvärt om så de har sommar på vintern istället.

}

\small{
\textbf{Sommarplågsmusiker}
\label{489149ebf843d8db469397e5b5841e64}
 Personer som livnär sig på att förstöra människors liv genom att komponera, framföra eller sjunga sommarplågor, medvetet. Låtar handlar t.ex. om:

 \begin{itemize}
 \item Att få ligga
 \item Att befinna sig på Stureplan
 \item Att lyssna på Reggae
 \item Att dansa Macarena och ryta
 \end{itemize}

 Vi på Nissepedia \textsc{(s.~\pageref{62400dadecd90cb5cd39062abe5a3e4a})} förespråkar såklart alla människors lika värde, men om denna samhällsgrupp skulle försvinna under mystiska omständigheter skulle vi inte ställa några knepiga frågor utan bara gå vidare med våra liv i en lite, lite bättre värld.

}

\small{
\textbf{Sonett (engelsk)}
\label{fdb1c2b57d6b6fba90e1f70229e105dd}
 Den engelska sonetten är en diktform med ursprung i medeltiden \textsc{(s.~\pageref{88cbc30c5b233d97df68b5b041ac0655})}, men som användes utbrett först under renäsanssen eftersom den inte räknas som en sakral diktform. Den behöll sin populäritet genom den engelska litteraturhistorien enda fram till modernismens formexperiment och krav på en mindre bunden poesi. Sonetten har ett fjortonradigt versmått och består av tre strofer om fyra rader plus den omisskänneliga tvåradiga coupletten. Varje rad har oftast tio stavelser, och metern är ofta iambisk pentameter. Raderna är slutrimmande och rimschemat är som fäljer: a-b-a-b-c-d-c-d-e-f-e-f-g-g. För att förklara vad detta schema innebär följer här en sonett som exemplifierar hur raderna rimmar:

 Hade det inte varit så att far (a)
 i vredesmod och med vidgad anal (b)
 alltid vrålade \quotetext{Se till att bli klar!!} (a)
 hade jag nog kissat som en normal (b)
 Hade det inte varit för farmor (c)
 som alltid fick mig att känna mig ful (d)
 där jag stod i mina nya steppskor (c)
 hade jag kanske ibland haft det kul. (d)
 Hade jag bara gått min egen väg (e)
 när bror, som skulle gadda kroppen (f)
 övertalade mig att följa med (e)
 hade jag inget hakkors på snoppen (f)
 Men för mig blev det inte riktigt rätt (g)
 och så slutar min lilla ful-sonett (g)

}

\small{
\textbf{Sonny Listons son}
\label{0301b98c123a425f687afdfebf2f8f5d}
 Sonny Liston var en ball boxare från USA som var stor och snäll men som också tyckte om att supa och knarka ganska mycket. Han dog ung och det finns misstankar om att hans död var \textit{foul play}. Det har det pratats ganska mycket om sen hans död. Vad som inte pratats om lika mycket är att Sonny Liston och hans fru adopterade en son när de var på besök i Sverige i början av 1960-talet. Pojken hette Daniel. Det finns inga bilder på Daniel, eller någon information om vad han gör idag. Om Daniel Liston läser det här kan du väl skicka oss på Nissepedia ett e-mail? Vi är nämligen alla jättenyfikna på hur i helvete ditt liv kan ha sett ut. Ha det bra

}

\small{
\textbf{Sopletare}
\label{a9729f0b2c1c1ec17cec4dc9fdb10007}
 I fattiga delar av världen finns det för många inget annat sätt att finna föda än att rota i andra människors sopor. I rikare delar av världen visar därför medelklassungar sin sympati med dessa desperata och hungrande människor genom att själva gå igenom soporna utanför matvarubutiker så som ICA och konsum \textsc{(se konsumbutik s.~\pageref{70e4875f7c2c177596305006e46b7ca9})}, men dessa människor kallas vanligtvis inte sopletare, utan går under mer snajdiga namn på utrikiska, så som \quotetext{freegans} eller \quotetext{dumpster divers.} Den som vill veta mer om dessa personer kan kontakta HratvinnFlygur \textsc{(se User: HratvinnFlygur s.~\pageref{31c19e82288ba7034064ee9b096bd7cf})}, som inte sällan brukar bjuda på bananer \textsc{(se banan s.~\pageref{aec7bd708ed2ad3435b9a9883ac7f45c})} och godissnören som någon konstaterat är oätbara och därför kasserat i nått dike.

 HEAD2: Olika sorters sopletare
 \begin{itemize}
 \item Aktivister \textsc{(se Den ambitiösa studenten s.~\pageref{02257ef6d6da8e0f0721e2758eec3c71})}
 \item Träskpunkare \textsc{(s.~\pageref{484838b3db1adb135ea74d6fc61e44c0})}
 \item Hippies \textsc{(s.~\pageref{4dc77d6258fd18e7c0dd5eece5c7c81c})}
 \item Genuint snåla människor \textsc{(se Genuint snåla människor s.~\pageref{b48797ecad31c4b98b780e115e70dcc1})}
 \end{itemize}

}

\small{
\textbf{Sorbet}
\label{2c2e2f8e1a9332edac1b587a10aab384}
 är ett annat ord för mosad calippo. En sorts förklädd isglass, alltså.

 Om man glömmer en öl i frysen lite för länge får man en smaklig sorbet späckad med B-vitaminer.

}

\small{
\textbf{Southern comfort}
\label{b5be4e79c92c85d6b964b26652ec81a2}
 \quotetext{Vill general McKenna slå sig ner i skuggan \textsc{(s.~\pageref{689105de7de72e11f2e4105a436c8542})} och vila sina ben en stund?} Mrs Kemble log sitt oemotståndliga leende. Generalen lutade sin bössa mot stammen av ett frodigt träd. South Carolina: Detta var hans hem. Mrs Kemble skickade iväg en yppig negress att hämta ett välförtjänt glas åt gentlemannen under trädet. I fjärran hördes mannarnas muntra visslande av \textit{I Wish I was in Dixie} och ekot av kanonkrevader.

 \textit{Southern Comfort - Senusalitet, Enkelhet, Rasism}

}

\small{
\textbf{Spade}
\label{ab1991b4286f7e79720fe0d4011789c8}
 n används för att gräva med, det är vanligtvis ett redskap bestående av ett träskaft \textsc{(s.~\pageref{1ab85ecd859ae682af47bb9334c7dac6})} med en plåtbit monterat i den nedre änden (förutsatt att man inte håller spaden upp och ner, då blir det tvärtom).

 En spade kan också användas att skyffla med, tex så kan man skyffla skit, en sådan spade kallas för skyffel \textsc{(s.~\pageref{22960a0dd0c60880f082b1be3f565925})}.
 Man skulle kunna säga att en spade är en glass... men det är fortfarande en spade. Är man postmodernist så kan dock en spade vara en glass, för vem bestämmer det egentligen?

 En riktigt bra spade bär sigillet \quotetext{ace of spade}.

 Källa: Någon röst i radion.

}

\small{
\textbf{Spanien}
\label{84c63835ca2fcac8636cf7d36aa48fa4}
 är ett ganska fyrkantigt land på den iberiska halvön längst ned i Europa. Till och med oseriösa länder som Italien och Frankrike \textsc{(s.~\pageref{8a28b520a53cd68763ebf19b5599412b})} ligger högre upp. Närheten till det varma Medelhavet gör att spanjacken gärna vill svalka sig och det gör hen helst med en petflaska sangria eller ett glas spanskt lättvin \textsc{(s.~\pageref{00415b996c8c85901b16f9c8c687342b})}. Framåt kvällen grillas en hel spädgris och för underhållningen står en trubadur som spelar spansk gitarr och sjunger om hur vacker hens mamma är.
 HEAD2: Spansk kultur
 Vara otrogen och kasta getter från kyrktorn. Förtrycka nationella minoriteter.
 HEAD2: Spaniens historia
 Morerna försökte styra upp detta kaos men elefanterna trivdes inte så dom åkte hem och det var ju synd.
 Hur man lyckades kolonisera typ halva världen är helt obegripligt.
 HEAD2: Kända personer från Spanien
 Manuel \textsc{(s.~\pageref{96917805fd060e3766a9a1b834639d35})}
 Cato Fong
 Don Quixote
 Enrique Iglesias
 Alberto Contador
 HEAD2: Spaniens framtid
 En sten som faller i vakum.

}

\small{
\textbf{Spansk haloumi}
\label{fd84061de7d9f7025e922f6ab9f819ec}
 Riven hushållsost.

}

\small{
\textbf{Spanskt lättvin}
\label{00415b996c8c85901b16f9c8c687342b}
 tillverkas genom att slå lite vatten i ett glas vin \textsc{(s.~\pageref{62911ad86d6181442022683afb480067})}. Det har många likheter med danskt lättöl \textsc{(se Dansk lättöl s.~\pageref{3981afb990a974a3b4613a470e51e747})}, men en stor skillnad och det är att när dansken bara dricker ett glas så drar spanjacken i sig en hel pava. Det här är fortfarande inget hinder för att dra ut och köra bil \textsc{(s.~\pageref{b3188f47d2eac7efc3f1258dc673a9fe})} eller backa med släp \textsc{(s.~\pageref{2e9a3db5d49f5a13e74bbfa2a3105c3d})} för den delen.

}

\small{
\textbf{SPAP}
\label{d9aa9e10e2020b9b48241e678285e345}
 är en förkortning för Supa På Annan Plats, ett klassiskt knep som kan förgylla vilken unken lördag som helst. När man konsumerar alkohol görs det ofta i en begränsad fysisk sfär (typ i bersån på Hallonvägen 2, Scharinska villan eller Lottas krog), vilket efter ett tag kan börja kännas rätt gammalt. Tricket består i att komma på en ny miljö att inmundiga alkohol i och således ge längre livslängd till sysslan att supa. Nedan följer några förslag på andra platser att supa på:

 \textit{-En sommarstuga}
 \textit{-Costas}
 \textit{-En idyllisk klipphäll}
 \textit{-Valhall, de fallna kämparnas sal}
 \textit{-Återvinningsstationen på Berghem}
 \textit{-Skellefteå \textsc{(s.~\pageref{6f7057df8c42e3cc22d2ad46de5c6597})}}
 \textit{-En biosalong}
 \textit{-Ensam vid Umeälven med Sabbath i lurarna}
 \textit{-Ensam på sitt rum, lyssnandes till Floyd}
 \textit{-Morsan \& farsans garagekällare}
 \textit{-Libanon \textsc{(s.~\pageref{73a3ab1ca04609e0540e692a4cc5f286})}}
 \textit{-Runtomkring Axtorpet \textsc{(s.~\pageref{11f14a5854300c512ed501986f38a609})}}
 \textit{-Domarstolen på en tennisbana}
 ''-En övergiven borgruin
 ''-Berlin Alexanderplatz.

}

\small{
\textbf{Speed-ove}
\label{0d400e24ef4d0c28e4b004d871eb63e1}
 är inte som man lätt skulle kunna tro en man förtjust i vitt pulver eller en maträtt bestående av en påse amfetamin uppkörd i en man vid namn Ove. Istället kretsar Oves liv kring den idag så grymt förbisedda genren speed-metal. Speed-Ove har många fräna patchar på sin jeansjacka där han glider fram lyssnandes på Accept i sin bärbara kassettbandspelare. Speed-Ove är ofta på Universitetsbibblan, inte för att som man lätt skulle kunna tro - ragga brudar - utan han har läst helt sjukt många fristående kurser (sant!) och besitter en bildning som är gedignare och genuinare än de flesta andra på universitetet \textsc{(se universitet s.~\pageref{11dfc744fa396b961a6cc9cf89c4eaea})}. När Ove kopplar av från de hårda studierna gillar ha att dricka Aurora \textsc{(s.~\pageref{99c8ef576f385bc322564d5694df6fc2})} hemma på Tuna \textsc{(s.~\pageref{2bf93a8a979420ff77b32fab0751cad2})}, antagligen till tonerna av sina favoritband.
 Som kuriosa kan nämnas att det i forna Östtyskland finns en herre som lever för speed metal som heter Uwe, Speed-Uwe.

}

\small{
\textbf{Speedos}
\label{22286b2c61cbd4c567b0999a958db3eb}
 är ett slags småbyxa \textsc{(s.~\pageref{13b74e04e94c0c93e01e835f54a271c2})} för  män \textsc{(se man  s.~\pageref{39c63ddb96a31b9610cd976b896ad4f0})} och används i första hand på stranden och andra vattenrelaterade platser. Ibland kallas småbyxan därför också badkalsong. Speedon inger enligt vissa en känsla av frihet \textsc{(s.~\pageref{0ee37cea60c9b45e40dbc83c0c665085})} och ledighet, medan andra emotsätter sig dess ärliga framhävande av kroppens eventuella skavanker. I Europa är speedon mest populär i Tyskland och Grekland, men har också haft en storhetsperiod här i Norden under 70- och 80-talen, innan den blev utkonkurrerad av dess storebror, badshortsen.
 HEAD2: Tillvägagångssätt vid påklädning
 För att ta på dig ett par speedos lägger du dessa framför dig på marken eller golvet, sedan kliver du med fötterna ner i de för detta ändamål ägnade hålen i plagget med vardera foten. OBS! Akta så att ingen står bakom dig och ser på! Därefter tar du tag i kalsongens överkant och drar den över knäna och så småningom upp i skrevet. Nu är det dags att justera dragskon \textsc{(se dragsko s.~\pageref{0d3beb9223700e39e09040e9bbd3644b})} och knyta en rosett av snörena samt kontrollera så att allt som ska vara inne i speedon är inne i speedon. Om allt är som det ska har du lyckats med att ta på dig badplagget och kan nu sakta jogga fram och tillbaka längs stranden, skutta omkring i vattnet eller vad du nu vill göra.
 HEAD2: Tillvägagångssätt vid avklädning
 När du är nöjd med din användning av speedon för den här gången uppsöker du med fördel en skyddad plats såsom ett ombytningsbås eller några lågt växande björkar. Här knyter du upp dragskon \textsc{(se dragsko s.~\pageref{0d3beb9223700e39e09040e9bbd3644b})} och rullar sedan försiktigt ner speedon för låren. Plagget har nämligen blivit blött sedan du tog på det, antingen av vatten eller svett, och kan inte åka av på samma vis som ett par kalsonger gör. Plagget hamnar nu på marken, medan du står hukad med benen tätt ihop. Sätt nu skyndsamt på dig ett par kalsonger och kanske ett par shorts och skynda ner till vattenbrynet för att vaska \textsc{(se vaskning s.~\pageref{12eda3c33bbe1844cc47bd51e16c6d81})} dina speedos och på så vis avlägsna sand och växtdelar som har fastnat på dem. Häng dem på en kvist för att torka.

}

\small{
\textbf{Spegel}
\label{4cbbd32137ad5680ad1f333a4702c2fd}
 är ett slags gest som används för att värja sig när någon pekar finger åt en. Gesten består i att handflatan riktas i den riktning som den obscena gesten kommer från. På så vis signalerar man att den som pekar finger åt en misslyckas med sitt tilltag och i själva verket pekar finger åt sig själv. Då får de något att tänka på.

 Flitigt använt på skolgården på mellanstadieskolor i Stockholms \textsc{(se Stockholm s.~\pageref{edcd259e0a03c7ab70feb186bae19f13})} län på 90-talet.

}

\small{
\textbf{Spegelmöte}
\label{a3aacc4fb2c14f1d9a3d4aa3ace2490f}
 t är ett fenomen som uppstår när två människor möts och håller på att krocka, och försöker kliva åt sidan fast åt samma håll. De krockar igen och en humoristisk situation uppstår. Detta spekuleras av glädjevetenskapliga \textsc{(se glädjevetenskaper s.~\pageref{7e4eadb905a6345ef2a6ce2b5b179847})} arkeologer vara världens äldsta skämt.

 HEAD2: Skämtet på stenåldern
 Det första registrerade fallet av detta är en grottmålning från en grotta i Frankrike som har daterats till att vara skitgammal. En grottmänniska var på väg hem till grottan med en sabeltandad tiger på ryggen, varpå en annan grottmänniska ser det och tänker \quotetext{Fan va pjäxfett \textsc{(s.~\pageref{a297c3dde307c3d98a9433e88b02432d})}, undrar om det finns mer sabeltandad tiger?} och ger sig av mot grottmänniska A. De möter varandra på stigen och håller på att krocka men viker av åt samma håll och båda grottmänniskorna håller på att flina läppen av sig.

 HEAD2: Skämtet på medeltiden
 Två arméer är ute och krigar och möter varandra ute på slätten \textsc{(se slätt s.~\pageref{a9cde01124ca41f23d6044b3ba27b979})}. Just de här arméerna hade inga konkreta planer på att döda och lemlästa varandra så de tar alla ett kliv åt sidan MEN ÅT SAMMA HÅLL! Ett masspsykotiskt skratt bryter ut i de båda härarna och sen slår de ihjäl varandra i alla fall, för medeltiden \textsc{(s.~\pageref{88cbc30c5b233d97df68b5b041ac0655})} var en jävlig tid.

}

\small{
\textbf{Spegelmötet}
\label{fbb0ff23274b1c9fbf127b8fcd72deb5}
 Spegelmöte \textsc{(s.~\pageref{a3aacc4fb2c14f1d9a3d4aa3ace2490f})}

}

\small{
\textbf{Spicken sill}
\label{146aeee683e9b22b4492de9e70a8eef8}
 kan ätas på hårt bröd eller som den är.

}

\small{
\textbf{Sportmössa}
\label{a6a5f90825518f69ef5df7d814de68d8}
 är en matematisk term som används vid beräkningar av sannolikhet. Om något helt saknar sportmössa är det närmast otroligt att det ska inträffa, och då skriver man i sin matematiska beräkning (eller rejält tilltagna höftning) att: \quotetext{det finns inte en sportmössa att det händer}. Etymologiskt härstammar termen från den amerikanske brottaren Rulon Gardner som lyckades med det osannolika att vinna en match mot Aleksandr Karelin \textsc{(s.~\pageref{7db555630a4ad78feb3477db9b1ee464})} och blev så glad att han åt upp sin hatt.
 Relaterade matematiska begrepp: räkmacka \textsc{(s.~\pageref{2d749ddfe869664c96fe0a0b572c5f0b})}

}

\small{
\textbf{Spritfylla}
\label{0668c687b51995118ec27cbf25061118}
 En brinnande dimma, ett skenande lok, en röst som skriker NEJNEJNEJNEJ!!!!!!!
 Vakna med ett halvbrunnet stearinljus i handen och en avsliten handklov kring handleden, en trasa kloroform i fickan, tjugo meddelanden i röstbrevlådan som alla ber för ditt liv, bakgrunden på mobilen bytt, från din hund (golden retriever) \textsc{(se golden retriever s.~\pageref{2b3d7c01f0a8a57d8fa2a18b54993a6b})} med sina valpar, till ditt uppkåtade kön. Karatesparka en jävel på käften, försök smälta en kundvagn med telekinesi, grilla korv med tändsticka, bär underkläder som hatt, skrik allt vad du orkar och kanske 110 \% till. Ge allt, lite till maxa maxa mer.

 \textit{Spritfylla}
 {{Yrsel}}

}

\small{
\textbf{Spritvev}
\label{982050892345f509daf436946af24dda}
 Handgemäng under påverkan av rusdrycker. Vanligt i byarna runt Avesta.

}

\small{
\textbf{Sprunka}
\label{70d63132ca4992c9f7f20b064c060709}
 kallas det när en man liksom \quotetext{spränger} ut ur ett par startblock, belastar sin muskelmassa och syresomsättning maximalt genom att sprinta mot en mållinje som är exakt 100 meter bort, samtidigt som han onanerar. Fenomenet har blivit ovanligare på senare tid i och med att det anses vara lite konstigt. Sprinten måste vara ärlig.

}

\small{
\textbf{Sprängmedelstekniker}
\label{f33159ddfc53e77b393d2b937b441f59}
 ns slogan lyder: Om ni ser mig springa, SPRING ni också!

}

\small{
\textbf{Språkliga lustifikationer}
\label{db9524e3dc3677984c9ec447b70ba79f}
 Så här beskriver språkvetaren John Kjederqvist hur en språklig lustifikation är uppbyggd i boken \quotetext{När ord blir roliga - språket och skämtlynnet}

 \textlessi\textgreaterTycker man, att det låter komiskt, så är det ännu inte orden som sådana som äro orsaken. Poängen, orsaken till löjet, ligger inte i själva orden, snarare skulle jag vilja säga i tankegången. Det är inget språkskämt i egentlig mening.

 Om däremot en annan, som vill berätta samma sak, säger: \quotetext{Han kolporterade nedför trapporna}, så ha vi ett fall av språkkomik, som tillhör den stora gruppen \quotetext{komisk användning av främmande ord}: kolportera istället för kullbyttera. Det förstås härav, att orden, som användas, skola på ett särskilt sätt deltaga i effekten.\textless/i\textgreater

}

\small{
\textbf{Spunka}
\label{8c6c113bf5b61e630d9f19b7820d0c6c}
 (verb, obestämd form singular; bestämd form spunk) är en multipel handling där utövaren spyr över sitt egen könsorgan i kombination med ejackulering. Fenomenet har blivit mindre vanligt på senare tid i och med utedassens dalande popularitet och Onkel Kånkels tragiska bortgång.

}

\small{
\textbf{Sputnik}
\label{4eb127800207d26651c1ff186bfe692c}
 är inte bara Sovjetunionens triumf över den fascistoida västvärlden.
 Det är även en grusåkare och countrysångare från Telemark. Han är mest känd för att ha spelat på Röda Torget strax innan imperiets sammanbrott, och för att ha släppt över 20 kassetter där han på de flesta har en fimp i mungipan. Rökning kan inte vara skadligt då Sputnik är sådär 70 år.
 Sputnik säger själv \quotetext{\textit{Imorrn är det kanske ingen som gillar Sputnik, men grus och sten - det ska dom ha!}}.

}

\small{
\textbf{Späckhuggare}
\label{8cb9605d553e2ecca26af024d2fcc220}
 Person som attraheras av överviktiga.

}

\small{
\textbf{Spärrballong}
\label{6be5aa404a270879d826e417222c7857}
 En spärrballong är ett gammalt beprövat försvar mot ovälkomna fiender som attackerar från luften på låg höjd. En stor ballong i slitstarkt material fylls med gas och förankras i ett kraftigt rep \textsc{(se repet s.~\pageref{0714379932aa997070168553fe416a96})}. Den lågflygande fienden vill förhoppningsvis inte riskera att skada sin maskin på repen, och lyfter därför mot högre höjd där den får svårare att sikta och samtidigt blir ett lättare mål för luftvärn. Ponera \textsc{(s.~\pageref{81de0f38ad2cd422870c2e70763f3510})} att du strövar omkring på ditt gods och tittar till att bokskogen växer som den ska, med saltbössan i hand. Plötsligt hör du ett surrande ljud \textsc{(s.~\pageref{461b63e21c531b2c2102d07af5ee13cf})}, och upptäcker att en av dina edsvurna fiender från grevskapet bredvid är på väg rakt mot ditt slott i en fullrustad messerschmitt. För hundra år sedan hade du varit rökt, och antagligen samma sak \textit{om} hundra år. Men är du rätt rustad \textsc{(se rusta s.~\pageref{2217dae26ef18f6a32d5aa0d7a032a16})} kan du nu tack och lov möta det stundande hotet genom att hissa dina i förväg gasfyllda, och för en förmögenhet inköpta, spärrballonger. Därefter har du bara att bege dig till ditt luftvärn och skjuta ner den edsvurna fiendens flygmaskin.

 I Sverige fanns länge bara en enda reparationsverkstad för spärrballonger, en sektion på Degerfors järnverk. Idag finns ingen alls.


 \textbf{OBS:} \textit{Nissepedia uppmuntrar på intet sätt till användning av attack från luften på låg höjd som ett sätt att vedergälla edsvurna fiender. Den som ämnar genomföra en sådan aktion får absolut inte använda sig av artikeln för att undkomma eventuella spärrballonger. Överträdelser kommer anmälas till luftfartsverket}.

}

\small{
\textbf{Staffetpinne}
\label{0fba2c28c18a36b1dcf8449e2394deb4}
 \textbf{Staffetpinnen} är den korv som på macken Statoil Öbacka i Umeå tillåts ligga kvar på grillen mellan personalens skiftesbyten. När en anställd går av sitt pass pekar denne helt sonika på 'pinnen' till sin kollega och flinar pillemariskt. På detta sätt kan den nya, fräscha kollegan lätt hålla koll på var 'pinnen' håller hus på grillen. Först när någon ovanligt otrevlig kund, eller någon som på annat sätt agerat utanför de ramar som kan anses vara lämpligt beteende på en bensinmack, beställer en korv med bröd \textsc{(se Korv med bröd s.~\pageref{8ed6a229bd465c6f2a0a73f65534056b})} eller french använder sig personalen av 'pinnen', varpå de knyter handen i fickan och räknar in en modest vinst över sina plågoandar.

 Tack vare det tveksamma klientelet på just Statoil Öbacka får 'pinnen' sällan möjlighet att bli liggande längre än några få arbetspass i sträck. Enligt lösryckta men tillförlitliga rykten ska dock en 'staffetpinne' överlevt i över en vecka vintern 2002 då det var ovanligt kallt och de riktiga fyllesvinen inte klarade av resan förbi Statoil Öbacka på väg hem från Scharinska för att fixa sig lite folle \textsc{(se Folle s.~\pageref{fe938fafef93bca3ba46995c6d409807})} och en varmkorv \textsc{(se Korv med bröd s.~\pageref{8ed6a229bd465c6f2a0a73f65534056b})} till efterfestandet.

}

\small{
\textbf{Stalin}
\label{c24539f26c60a8509747f30c3d4a761c}
 70 procent bra, 30 procent dålig. Därom tvistar de lärde.

}

\small{
\textbf{Stanley star}
\label{9525bcd71b000da8a4912a9997716365}
 är en tjeckisk bandbokare och musikpromotor. Stanley är chef för \textsc{(s.~\pageref{5a98c81c7b5b60a5777a92b943f53a41})} företaget Stanley Star Promotion som drivs med honom själv som enda anställd hemifrån hans föräldrars övervåning där han också bor. Stanley nås enklast på icq: 332781309.


 HEAD2:  Vanilj

 Stanleys farsa är en jävel på att koka brännvin.

}

\small{
\textbf{Stark mat}
\label{8f43507e39b50520de82de94d1d88d94}
 är motsatsen till svag mat \textsc{(s.~\pageref{0526761422547962084b0c0c3701cf91})}.
 Stark mat kan för en del  människor förknippas med resor till exotiska länder och spännande upplevelser. Händer som möts och oväntade möten med trummor och sång.
 Det finns undersökningar som visar att stark mat har en välgörande effekt på hälsan och själen. Den store naturfilosofen och rabulisten Linné anmärkte varnande: \quotetext{Vid val av födoämnen skall du eftersträva sådana som i och för sig hava smak eller angenäm lukt; de starkt luktande och de retande skall du undvika}.

 Exempel på stark mat är \quotetext{Chili con carne} .

}

\small{
\textbf{Stavningsregler}
\label{34f2bbdd0e56794999d257fb8c5eb8d3}
 HEAD2:  Stavning vid komparationsböjning


 Ä eller E

 En enkel minnesregel för att avgöra om ett adjektiv \textsc{(s.~\pageref{67d02147cd8595eaf13c1a90aba99dcc})} stavas med Ä eller E är att gå till ordets grundform.

 Om grundformen stavas med Ä eller Å ska ordet stavas med A, annars ska det stavas med E.

 Ex:

 * BÄST stavas med Ä eftersom grundformen BRA stavas med A.

 * ÄLDRE stavas med Ä eftersom grundformen GAMMAL stavas med A.

 * SÄMST stavas med Ä eftersom grundformen DÅLIG\quotetext{ stavas med Å.

 Således skall det böjda adjektivet stavas med E om grundformen inte stavas med A eller Å.

 Tyvärr finns inget exempel att bevisa tesen med, men det har hittills alltid fungerat för nisse


 O eller Å

 På samma sätt kan man \textsc{(s.~\pageref{39c63ddb96a31b9610cd976b896ad4f0})} lätt avgöra om ett ord stavas med O eller Å.

 Om grundformen stavas med Ö ska ordet stavas med O, annars ska det stavas med Å.

 Ex:

 * DOLD stavas med O eftersom grundformen DÖLJA stavas med Ö.




 HEAD2:  Hårda och mjuka vokaler

 Ett enkelt sätt att ta reda på om en vokal är hård eller mjuk är att lyssna hur den låter när den uttalas.
 A, O, U och Å har alla ett mjukt, nästan lent, ljud när de uttalas, det betyder att de är hårda vokaler.
 I , Y, Ä och Ö får ett betydligt hårdare och spetsigare ljud när de uttalas. Typ sånt man \textsc{(s.~\pageref{39c63ddb96a31b9610cd976b896ad4f0})} hör när man \textsc{(s.~\pageref{39c63ddb96a31b9610cd976b896ad4f0})} har tinnitus. Det betyder att de är mjuka vokaler.
 Den enda svåra att lära sig är E då den låter mjukt och också är en mjuk vokal.



 HEAD2:  Skilja på vokaler och konsonanter

 Vokaler stavas bara med en bokstav medan konsonanter stavas med två eller flera. Exempelvis stavas I bara med ett }i\quotetext{ medan P stavas }pe". Denna regel fungerar oavsett vilket språk man \textsc{(s.~\pageref{39c63ddb96a31b9610cd976b896ad4f0})} skriver på.

}

\small{
\textbf{Stefan}
\label{2e970e822e1a8834203d06abb60f59ec}
 är en numerologisk siffra som är betydligt komplexare än π (pi), 666, 1488 och andra populära tal. Ingen vet nämligen vart i den traditionella sifferföljden stefan kommer in, fastän det till och med är ett heltal. Efter ett \textsc{(se etta s.~\pageref{ba48f6c4097b7fc25ca11f1e544842d7})} kommer två \textsc{(se tvåa s.~\pageref{84fcc0494ecf9f5af79fcd9bed184a9a})} som följs av tre \textsc{(se trea s.~\pageref{6f94fdf535ab2e21147ea40ea920ca75})}, och så där håller det på enda upp till nio \textsc{(se nia s.~\pageref{04a481486dd84d7c8bfdfc89d38136a6})} utan att det passar in någon siffra som heter stefan. Om man visste vilket värde stefan hade i förhållande till andra siffror skulle det antagligen gå lättare, men det vet man inte.


 Källa: Dröm av Nissepediaanvändaren Potmo \textsc{(se Användare: Potmo s.~\pageref{3590efd14394b046b501f10cecba454c})}

}

\small{
\textbf{Stege}
\label{a835049be81e64997f972f57558a93ae}
 Väldigt brant trappa.
 Det finns en avancerad typ av stege som kallas för lönestege, den är det väldigt svårt att klättra på.

}

\small{
\textbf{Steglibab}
\label{1031ea42e093eb6ec5a2c30b2430e4eb}
 En steglibab är en maträtt som i sitt upplägg är lik en kebab, men istället för getkött används Steglitser \textsc{(se Steglits s.~\pageref{fc23cf62ee6739354604d50d807936a2})}.

}

\small{
\textbf{Steglits}
\label{fc23cf62ee6739354604d50d807936a2}
 (Carduelis carduelis) är en fågel i familjen finkar i naturens stora släkt. Den är färgrann och fin och stor som två snusdosor ungefär. Den har också en näbb som den använder för att äta med, vilket den måste för annars dör den av svält. När den misslyckats med det blir den steglibab \textsc{(s.~\pageref{1031ea42e093eb6ec5a2c30b2430e4eb})}.

}

\small{
\textbf{Stenad}
\label{dec4a3a91f0f2bf8dcf033a8cfeaa554}
 På en klippa vid en svensk insjö. Lite ont i lungorna just nu men trettio sekunder senare har det förbytts mot en torr mun \textsc{(s.~\pageref{6585f290ce92c3de5ff339920330e26f})}. En klunk folköl och allt är bra. Ett samtal om en björn i manchesterbyxor \textsc{(se yxa s.~\pageref{bd74f429522c7c1481fbba07187efc6b})}.  .... Vad är det du lyssnat på de sista minuterna? Spillkråka? Kanske.
 Om rymden är oändlig och dess innehåll är en finit massa så borde det betyda att det inte finns något utanför de himlakroppar som är längst ut i universum. Och om resten är tom är det bara potentiellt rum och då är väl universum, för fan, inte oändligt. Eller...? Hur det än är med det så borde man inte bry sig så mycket om sånt där, för det viktigaste är väl hur man har det här och nu. Fan en snus hade ju inte vart fel. Jävlar vad jag vill lyssna på Hawkwind just nu. Fast trummorna i den här låten är ju helt sjukt feta. Fast det hajar ju inte det här bandet för de är ju helt upptagna med nåt jävligt världsligt budskap som de själva inte förstår. Tänk om jag skulle kunna välja att inte finnas i typ ett år och sen komma tillbaka och bara vara som vanligt.: undrar hur folk skulle förhålla sig till mig då? Antingen skulle de bara vara som vanligt eller så skulle de på något vis vara mer distanserade - även om vi bara talade om världslig skit som tvättider och vem som spelade gitarr i vilket band och gud \textsc{(s.~\pageref{91e49146121c992aab11a19c77e26cf0})} vad jobbigt det är med relationer!!!! Vad fan är det han heter nu igen? Fan jag har det fan på tungan. Eller i bakhuvet \textsc{(se huvud s.~\pageref{e906cd95a540df9b16d0460fb4cf0adc})}...eller vad man nu brukar säga. Säger man så: \quotetext{i bakhuvet}? Varför det? Typ som att man hade ett lagerskåp med olika tankar som var ordnade i mappar i bakhuvudet. Just det jävlar! Rasmus Nalle \textsc{(se Rasmus Klump s.~\pageref{eac88a6def9b9f47888e7e3b62719cf1})} heter den lilla jäveln! Björnen i manchesterbyxor \textsc{(se yxa s.~\pageref{bd74f429522c7c1481fbba07187efc6b})}. {{Yrsel}}
 Vad skulle hända om månen plötsligt träffades av en meteorit och exploderade? Skulle jorden då sugas in i sin närmsta grannplanet och gå under som i Motörheads \textit{Metropolis}? Vad fan är det han snackar om nu igen? Just det, väderkvarnar! \textit{\quotetext{Väd-er-kvar-nar}}. Vilket jävla sjukt ord egentligen. Hahaha! \textit{\quotetext{Egen-kli-en}}. Hahaha!
 Kan man bada? Ja man kan bada. Ska jag bada? Jag borde inte bada.

 \textit{Stenad}

}

\small{
\textbf{Stenkast}
\label{7d5d8530b0f7b390d5b564073e22110b}
 Ett stenkast är så klart längre i Demokratiska Folkrepubliken Norrbotten än annorstädes. Detta eftersom allt av naturen är mycket större, bättre och vackrare än resten.
 Stenkast kan även vara ett sätt att påverka samhällsutvecklingen i rätt riktning. För denna betydelse, se Irländsk konfetti \textsc{(s.~\pageref{149459a4b475b90c8513551228efc472})}.

}

\small{
\textbf{Stenlapp}
\label{ace94f32e8c3e782e67a0bb330a8aa0b}
 En stenlapp är en bit sten som man karvat in något i för att minnas det. Idag är det inte supervanligt att folk använder denna fiffiga minnesteknik då vi har filofaxer, men i det gamla Mesopotamien var det ohyggligt vanligt förekommande. Där ristade glatt alla hemmafruar in mjölk och ägg \textsc{(s.~\pageref{128a5feb8e12d0aa622e0298a8332980})} på en bit kvarts och Babylonska borgmästarns talksrivare log ofta i godan ro när han hamrade in ca 40 000 miljarder \textsc{(se Fyrtiotusen miljarder s.~\pageref{c2160bffc9c5ca88e77204672e62e489})} tecken i stenlappar gjorda av utsökt granit. De mest kända stenlapparna är de tio budorden, vilka Gud \textsc{(s.~\pageref{91e49146121c992aab11a19c77e26cf0})} gav till människan för att de skulle komma ihåg några grejer han funderat på ett tag. Mindre kända stenlappar med mer obskyrt innehåll, typ böner tillägnade några halvluriga babylonska gudar, finns för allmän beskådan på Pergamon Museum i Berlin.

 \textbar\textbar

}

\small{
\textbf{Stensallad}
\label{9dfd3bda63c7274d1f21c75fce818b3a}
 En stensallad påminner om en fruktsallad men består, till skillnad från den senare salladen, av sten och grus. Den är inte i närheten så populär som sin kusin fruktsalladen \textsc{(se fruktsallad s.~\pageref{d7b2bae7bf161cb28b81db03874d3ecb})} men är betydligt billigare att göra.

}

\small{
\textbf{Stentrollsaffär}
\label{f832a905e0a0a857d0d7eae1520c14b1}
 En stentrollsaffär är en affär där man kan köpa stentroll, trädgårdstomtar \textsc{(se tomten s.~\pageref{3a3c1522c7155a18293fb1388055c13e})}, vargar av keramik, en och annan träbjörn \textsc{(s.~\pageref{3fa1e4f2d866814bf69e29479762b85a})} och andra sådana grejer. Stentrollsaffärer både drivs av och riktar sig till kvinnor i medelåldern. Stentrollsaffärer är ofta kombinationsaffärer \textsc{(s.~\pageref{0a2777bf1366a8a9a5b8eab9ca1496a1})}. Trots att stentrollsaffärer sällan är väldigt lukrativa finns det ett mycket stort antal sådana, speciellt i Norrtälje \textsc{(s.~\pageref{7527f7dad9445013a559dc7e2a91f3b3})}. Anledningen till detta är att det är många tanters dröm att driva en stentrollsaffär och få göra precis som de vill där inne. Vill de ta in en speciellt keramiktomte som skjuter en liten skottkärra som man kan placera en hysaint i, till exempel, så tar de helt enkelt in ett gäng tomtar som skjuter små skottkärror. Ingen kan säga emot. Tanten är chef och det är hon som bestämmer. Likaså, om hon inte vill ta in tomten så tar hon inte en den. Så enkelt är det.

}

\small{
\textbf{Stia}
\label{77dce45c04f3872ccee18b728ca4a30b}
 Stior är utrymmen som brukas av två typer av varelser; präster och grisar. Prästens stia kallas sakristia och fungerar som det man i dagligt tal benämner omklädningsrum. Grisens stia kallas svinstia och fungerar som det man i dagligt tal benämner crust as fuck existence \textsc{(s.~\pageref{bd0b07abcc2f4c2a4e1aafdfed1f0e73})}. Hur detta hänger ihop är ett riktigt trepipsproblem \textsc{(s.~\pageref{ddfa7edb7b4169a1dc8a32b1a8ad9611})}.

}

\small{
\textbf{Stig}
\label{2e9b1ac56ea26932bf0aff53fe48a533}
 En stig är den klart mest demokratiska formen av väg. Ingen övermäktig stat som bestämmer vars man får gå och inte, inte heller något företag som tar ockerpriser för anläggningen av underlag som asfalt och grus. Nej, en stig är summan av olika människors vilja att ta sig fram på egen hand, precis där de vill gå. Stigar är sällan till problem, om man inte är stighatare \textsc{(s.~\pageref{2a0e911b72b5555cedc4dcd9094c6b86})}.

}

\small{
\textbf{Stighatare}
\label{2a0e911b72b5555cedc4dcd9094c6b86}
 tycker inte att folk ska hålla på och gå lite varstans. Det stighataren då gör är att skrika \quotetext{MEN GÅ INTE DÄR FÖR FAN! DET BLIR EN STIG \textsc{(se stig s.~\pageref{2e9b1ac56ea26932bf0aff53fe48a533})} FÖR I HELVETE!}

}

\small{
\textbf{Stigma}
\label{98410ec61c6964eac5c923a594841696}
 är ett begrepp som uppfanns av Jesus \textsc{(s.~\pageref{110d46fcd978c24f306cd7fa23464d73})} men sedan återuppfanns av sociologen och teaterentusiasten Erving Goffman. Ett stigma är något som försvårar det sociala livet avsevärt.

 HEAD2: Exempel på stigman
 \begin{itemize}
 \item Armsvett
 \item Synlig stjärtskåra
 \item Finnar
 \item Högt uppdragna byxor
 \item Veganism \textsc{(se Vegan s.~\pageref{792fec82e3a0dcea1817fd9ebfaf1533})}
 \item Tjockisflås
 \end{itemize}

}

\small{
\textbf{Stjärtlapp}
\label{54ac9a50ce61792c76a284906648308f}
 är den sämsta av alla utförsåkningsredskap. Medan storfräsarens \textsc{(se storfräsare s.~\pageref{4db17005692cd83e3e946a1311b81ed0})} unge bränner ner i hissnande hastighet på sin sprillans snowracer med fotbroms och tuta sparkar du frenetiskt \textsc{(se mani s.~\pageref{07cd55c7b42715ec44c133a6a165e8d2})} med fötterna för att ta dig över första guppet, men det går segare än Sober-Jimmys \textsc{(se Sober-Jimmy s.~\pageref{62629d44a92716a33e051e9a6c04d0d4})} gitarrsolon. Du ägde en gång en klassisk orange pulka men plasten i den var så torr att farsan klev rakt igenom den i somras när han var lite ouppmärksam. Du tänkte inte så mycket på det då, eftersom det var strax efter midsommar och mest tyckte det var drygt att få skäll för att du inte tagit undan den än. Men nu sitter du där, med pissblöta tummvantar i ylle och Galne Gunnars kopia av en bävernylonoverall, och förbannar den värdelösa jävel som uppfann stjärtlappen. Det faktum att den också låg ute på grusgången hela sommaren \textsc{(se sommar s.~\pageref{4365ed528b682a00b02d5daf05a03b0d})} har inte direkt sänkt friktionsgraden. Möjligtvis fick den ett tunt vallaliknande lager när farsan använde den för att skotta undan hundskit, men det förutsätter att du vänt den sidan nedåt vilket du naturligtvis glömt. Så där sitter du nu, med hundskit i röven och svett och snor i ansiktet, och funderar på om stjärtlappsjäveln åtminstone är hård nog för att du ska kunna halshugga stekarungen med den.

}

\small{
\textbf{Stockholm}
\label{edcd259e0a03c7ab70feb186bae19f13}
 är ett ställe som ligger söder om Norrtälje \textsc{(s.~\pageref{7527f7dad9445013a559dc7e2a91f3b3})}. Här finns \quotetext{allt}. Det vill säga moderater \textsc{(se moderat s.~\pageref{c4564b188cb670841733a3ff923c2fb0})} och kaknästornet \textsc{(s.~\pageref{ffe3ac06a304714dcee7cbbdfeb20d84})}.

}

\small{
\textbf{Stockholmare}
\label{f093b936211e500e974debdfa9b6c21a}
 \begin{itemize}
 \item Kör obegagnat \textsc{(se nästan obegagnad s.~\pageref{bcc061cb9bf17532daff34259a1b2e36})} och lämnar in på märkesverkstad.
 \item Kremerar katten hos veterinär.
 \item Anammar yngre generationers mode.
 \item Känner bara till en måttenhet och det är kvadratmetern.
 \item Tänker inte på sig själv, utan hundralappar, när det är val
 \end{itemize}

}

\small{
\textbf{Stockholmsprofil}
\label{daaee4666c210c7a40537c2399f01556}
 en har kanske stora glasögon och vattenfestivaltröja.
 Stockholmsprofilen behöver inte nödvändigtvis komma från Stockholm men har ändå hängt med sen dag ett.
 Stockholmsprofilen tycker att det är nice.
 Stockholmsprofilen bär sneakers \textsc{(s.~\pageref{a1743c0d39461290efc551490aafc1e2})}.
 Stockholmsprofilen har varit med i tidningen \quotetext{Vice \textsc{(s.~\pageref{03f753c08ba5ff2bd7d2ee230b4683b1})}} minst en gång och varit på minst en fest sponsrad av Adidas.

}

\small{
\textbf{Stonerskin}
\label{b94c65dba2990b3146c2bedf663e9989}
 Ett stonerskin är en person som deltar i en subkultur som är en blandning mellan skinhead- \textsc{(se skinhead s.~\pageref{a54bc1b5d472b5afed8e84004b6441c4})} och stonerrockkulturen. Personen har kort hår och polisonger, bär ofta cammoshorts eller uppvikta jeans, Dr Martens, och en Fu Manchu-luvtröja och under denna en Goatsnake eller Cockney Rejects-tshirt. De flesta stonerskins läser en hel del böcker, målar tavlor, doktorerar, åker hoj, dricker öl och röker gräs, lyssnar på stoner och Oi! och har djur, växter och psykadeliska torn tatuerade på sina armar.

 HEAD2: Musik
 Stonerskin lyssnar på:
 \begin{itemize}
 \item The Oppressed
 \item Kyuss
 \item Cock Sparrer
 \item Dozer
 \item Last Resort
 \item Farflung
 \item 4skins
 \item Bongzilla
 \item Blitz
 \item Sleep
 \item The Templars
 \item God Grows his Own
 \item Oxblood
 \item Super Joint Ritual
 \item Bonecrusher
 \item Jajayra
 \item Gundog
 \item 500 ft. of Pipe
 \item Sham 69
 \item Weedeater
 \item Angelic Upstarts
 \item Black Pyramid
 \end{itemize}

}

\small{
\textbf{Stor-Anders}
\label{777d0562284d1dfba75c6f1b6297100d}
 har fått sitt namn inte bara för att han är stor utan också på grund av hans kollega Lill-Anders som för övrigt är gift med Anki. Stor-Anders är vaktmästare på Leksands Folkhögskola där alla ser upp till honom för den karlakarl han är. Han kan lyfta stora saker som ingen annan på skolan skulle kunna rubba. Dessutom fixar han allt som behövs. T.ex, om skolan skulle behöva bygga ut så skulle han bygga ut skolan på egen hand, utan att de skulle behöva anlita ett helt byggföretag. Stor-Anders håller så hårt på sitt hockeylag \textsc{(se hockey s.~\pageref{df0349ce110b69f03b4def8012ae4970})}, Leksands IF, att han tar ledigt från jobbet för att se dem spela kvalserien oavsett var i landet de spelar. Det sägs att han skall vara över 190 cm lång.

 HEAD2: Övriga Fakta
 Övriga fakta om Stor-Anders som inte många vet är:
 \begin{itemize}
 \item Att han äger en stridsvagn av sovjetiska modellen T34 i pepparkaka.
 \item Han kan backa med lastbil.
 \end{itemize}

}

\small{
\textbf{Stor-Stina}
\label{7f22dae34dafed9e1a1b3f2689f3793a}
 (född Christina Larsdotter) föddes i Brännäs, Malå socken, till en lappfamilj. På sin artonde födelsedag mätte hon 210 cm och hade, bokstavligt som bildligt, växt ur uppväxtens kåta och gav sig av på en resa ut i världen som skulle ta henne till de märkligaste platser.

 HEAD2: Söröver
 Stina, som var lika stark och förslagen som hon var lång, försörjde sig till en början på allsköns kroppsarbete som t.ex. att kröka järnvägsräls för hand och personlig assistans åt Peter Harrysson under en period när denne under en period satt i rullstol. Efter en vanlig arbertsdag av att baxa Harrysson upp och ner för trapporna i SVT-huset fick Stina nog och sa upp sig. Hon hade under denna Stockholmsvistelse en affär med blivande statsminister Olof Palme \textsc{(s.~\pageref{702b78623785546fb9c9890222376178})} som sägs ha uppfört Kaknästornet \textsc{(s.~\pageref{ffe3ac06a304714dcee7cbbdfeb20d84})} i hennes ära. Stina började tröttna på Stockholm och längtade efter en tillvaro med lite mer lyx och flärd. Hon satte sig på Djurgårdsfärjan och blev förvånad av att fortfarande vara i Stockholm när hon klev i land, men tjurig som hon var gav hon inte upp utan liftade till Göteborg.

 HEAD2: Västeröver
 Efter bara några månader i rikets andra stad blev Stina framgångsrik brugdförare \textsc{(se brugd s.~\pageref{d6b6b68506b8f1daad3a2ddbfaf8d863})} och vann den förnämsta brugdrace-tävlingen Tjörn Runt tre år i rad. Under en träningsrunda innan vad som skulle ha varit hennes fjärde lopp segnade hennes favoritbrugd Abmut ihop vid Doggers bank, med Stina trampandes vatten. Som den lapplänning hon var frös hon inte nämnvärt av att vistas i det kalla vattnet, men hon började få väldans långtråkigt efter flera dygn med suckande och vattentramp som enda aktiveter. Efter ändå några fler dygn, tack vare att hon som lapplänning också var osedvanligt sävlig, lärde hon sig att simma som hon sett renarna göra hemma och började simma in mot vad hon trodde var Sveriges västkust.

 HEAD2: Ändå mer västeröver
 Efter ett års simmande och meditativt jojkande klev hon i land på Ellis Island, New York. När Stina tog sin första promenad på Manhattan var det mången amerikan som höll på att trilla omkull då de trodde att självaste frihetsgudinnan var ute och spankulerade. Detta upphörde när Stina bytte sin turkosa kolt till fördel för en mer proper mörkblå. Utfattig i \quotetext{Möjligheternas land} är det ingen som vill vara, allra minst en vidunderlig lappflicka från Malåträsk, så hon tog första bästa pisskneg och började köra cykeltaxi. Stina trivdes på sitt nya jobb men blev en dag påkörd av en limousine på Manhattans Upper East Side och blev vredare än en nyväckt björn. Stina reste på sin över sex fot långa kropp, kavlade upp koltärmarna och innan bilisten hunnit be om ursäkt låg limousinen på taket. Stina lugnade genast ner sig och hjälpte först föraren ut för att sedan dra ut passageraren. Och vem var det Stina tagit ett fast grepp om kavajslaget på om inte självaste Vince McMahon, chefen för wrestlingfederationen WWE.

 HEAD2: Lite varstans
 Herr McMahon var förstås chockad över denna händelse, men tänkte direkt när han såg denna jättinna att han kunde skära guld. Han presenterade idén för Stina som direkt tyckte att det lät bättre än att köra cykeltaxi, och vem kan klandra henne efter allt hon varit med om? McMahon ordnande snabbt in i Stina i en match mot Hulk Hogan. Trots att Hulk var dåvarande stormästare så hade han inte en chans mot Stor-Stina och hennes fantastiska \quotetext{moves} såsom: \textit{Schvåppkast \textsc{(s.~\pageref{c82467f0f5a1ec8022ed8310c0658f79})} från Häcklefjäll \textsc{(s.~\pageref{40a7322a2ef5adb9efd69969d8f28f1e})}}, \textit{Stalos vrede} och inte minst det avslutande \textit{Lavinen på Ahkkávaara}. Det lite tragikomiska i det hela var att ingen brytt sig om att berätta för Stina att fribrottning inte var på riktigt, utan hon hade på riktigt skadat Hogan. Uppståndelsen lät inte vänta på sig. Polisen ville gripa henne, McMahon ville avskeda henne och Hogan låg och grät i ringen. Den ack så förslagna lappflickan greppade situationen på en gång och bestämde sig för att fly landet. Var ska man åka när det känns som att hela världen är ute efter en? I Stinas fall var svaret på frågan mycket enkel. Hem. Och det hemmet var och förblev Lappland.

 HEAD2: Österöver, sen nolöver
 Någon simtur var inte Stina intresserad av, det hade hon redan gjort och ville istället testa något nytt. Hon tog därför första bästa atlantångare tillbaka till Lappland, men bristen på turer på sträckan New York - Skelleftehamn tvingade henne till en annan resväg. Hon klev i land på Irland och började promenera omkring. Inte en av Irländarna visste var varken Brännäs, Malåträsk eller ens Lappland låg.
 Frustrerad över detta lämnade hon ön och började på nytt simma. Hon gick över Storbrittanien, en ö hon nu förbannade sig själv över att ha missat på sin första simtur över atlanten, och simmade på nytt till Göteborg. Hon minns här sina gamla glansdagar som traktens bäste brugdförare och övervägde att stanna en stund, men Göteborg är många saker; fiskdoftande, blåsigt och fullt av ordvitsare, men något Lappland var det sannerligen inte. Kosan styrdes norrut medelst en lånad orange Crescent Världsmästarcykel längs E45 och när Stina kom till Sorsele kände hon på nytt den välbekanta lappländska luften fylla lungorna. Hon cyklade till Blattnicksele och besökte Violas \textsc{(s.~\pageref{bc0b8c20b7ac9de2bb42b4c7285e93ba})} där hon ekiperade sig med ett par nya trosor, för hur konstigt det än låtar hade hon glömt att byta trosor under hela hennes långa, långa resa. Så tokigt. Hon traskade från Violas ner till Vindelälven där hon satte sig på en sten och lyssnade på forsarnas mäktiga brus. Men så plötsligt, från ingenstans, svepte en gigantisk fjälluggla ner och lyfte den väldiga kvinnan upp i skyn.

 HEAD2: Okänd plats, någonstans i Lapplandsfjällen
 Det var en upprörd och något förvirrad Stina som sattes ner på en sällsynt blåsig klipphylla. Fjälluggla, som presenterade sig som Lars-Mikael satte sig i den ring av uvar \textsc{(se uv s.~\pageref{45210da832f9626829457a65e9e7c4d0})} som satt på klipphyllan och konfererade. Den största av uvarna, berguven Buborgnifar, tog till orda med sin mäktiga stämma:

 \textit{Stina! Vi uvar har hört mycket om dina resor! Du har korsat oceaner, besegrat väldiga kämpar och fått den mänskliga civilisationens mäktigaste byggnadsvek uppkallat efter dig! Vi inser att du är av stort värde för vår plan för världsherravälde! Var dag ser vi våra kamrater uppstoppade \textsc{(se uppstoppad uv s.~\pageref{a562653cfd13c16d7f4d85967242ccdd})}, bortstoppade \textsc{(se bortstoppad uv s.~\pageref{86574b11bb49a6f8e32d9f716676236a})} och förstoppade \textsc{(se förstoppad uv s.~\pageref{06630b162e869a376076dda808c05e5f})} Något måste göras för alla uvars välbefinnande! Där kommer du in, Stina! Du ska hjälpa oss!}

 Frågorna i Stinas huvud var nu många: \quotetext{Hur orkade fjälluggla lyfta mig?}, \quotetext{Hjälpa med vadå?} och framför allt \quotetext{Kan uvar prata?}. Uvarna samlades nu i en ring och lade upp vingarna på varandras axlar och började språkas, nu lät det dock inte som de röster hon tidigare hört, utan bara en massa a-hootin' and a-hollerin' \textsc{(s.~\pageref{1928c39ea0f58992a3e5f53d143a23ff})}. När uvarna pratat klart flög den tidigare nämnda fjällugglan Lars-Mikael tillbaka Stina till samma sten som hon satt på innan det bryska bortförandet. Stina var minst sagt förbryllad. Var det hela en dröm? Hon såg på den hoande fjällugglan som satt i en gran ovanför henne och var verkligen osäker. Hon gnuggade ögonen, men den satt kvar. Hon kastade en sten på den, men den rörde inte en min. Hon blickade ut över Vindelälven och funderade länge och hårt på om mötet med uvarna var på riktigt. När hon bestämt sig för att fortsätta cykelturen hem mot Brännäs och just skulle resa sig hörde hon fjällugglan säga \quotetext{Du vet vad du ska göra!} och sedan flög det storslagna fjäderfäet sin väg. Stina visste inte alls. Hon tog en chansning och kastade sig ut i Vindelälvens vilda forsande vatten.

 HEAD2: Västerut, sen österut
 Vindelälven visade sig inte alls leda förbi Lycksele, dit hon var på väg, utan förde henne till Umeå via Ume älv. Men hon gav som bekant inte upp i första taget utan Stina gick i land vid Tegsbrons sydliga fäste, gick genom Björkarnas Stad till järnvägsstationen och tog rälsbussen till Lycksele. Stina var nu något så inni Norden trött på att färdas runt i världen och bestämde sig att nu jävlar lämnar jag aldrig mer Lappland. Hon behövde någonstans att bo och byggde därför en gigantisk kåta vid Hotell Lappland, för som ni minns var vanliga lappkåtor på tok för små för denna Lapplands Jättinna. Här behövde hon något att göra på dagarna och startade därför Lycksele Djurpark. Hon fångade in alla de djur hon känt i sin uppväxts skogar, burade in dom och lät allmänheten beskåda dessa.

 HEAD2: Stinas öde och arv till eftervärlden
 Stor-Stina hade genomlevt ett helt liv av fantastiska äventyr och klarat sig helskinnad genom dessa, men slutet på Stor-Stinas saga är inte på något sätt lika storslaget som dessa strapatser. En dag när hon som bäst utfodrade myskoxarna så råkade hon snubbla på en rot och slog ihjäl sig. Till eftervärlden lämnade hon denna fantastiska djurpark. En djurpark som, märkligt \textsc{(se märkliga sammanträffanden s.~\pageref{f46282d99158f351a81b9deaff157b4e})} nog, inte innehåller en enda uv!

}

\small{
\textbf{Stora Grabbars och Tjejers Märke}
\label{3b527f8b13885eb277c77de4b1f51658}
 är en utmärkelse som delas ut till särskilt framstående idrottsutövare i Sverige. Utmärkelsen har funnits sedan 1928, och för att bli stor grabb/tjej krävs att man samlar på sig ett antal poäng under sin aktiva karriär. Poängräkningen ser lite olika ut inom olika sporter men vanligt är att landskamper och internationella medaljer ger poäng. Varje ny medlem får också ett unikt nummer, och Pelle Fosshaug \textsc{(s.~\pageref{b6caa53a9a50eb546517552a5503e323})} är exempelvis \quotetext{stor grabb nummer 197} i bandy. Nummer ett innehas av Sven \quotetext{Sleven} Säfwenberg \textsc{(s.~\pageref{3db9bb625a0cf732a4d63171f8cc9db1})}, och nummer noll av Sune Almkvist \textsc{(s.~\pageref{8b26ee9eb5634176df1f49f6cbb71708})}. Det är oklart varför ett så anrikt pris är så pass okänt som det är bland de breda folklagren. Kanske är medlemmarna rädda att någon ska stjäla deras nummer.

}

\small{
\textbf{Stora mobba kaka dagen}
\label{807126db3a4e2769948e34980d5a8096}
 Stora Mobba Kakadagen infaller den 19:e November varje år och firas på sina håll med parader, marchmusik och kaskelotter.

}

\small{
\textbf{Storbossnörd}
\label{456018ad01124baca4c32b6567fca7b8}
 En storbossnörd är en nörd som lagt ned alla anspråk på att passa in i samhället i stort. Vanliga människor är så jobbiga och fattar inget så han (det finns kanske tre hon i hela världen också) har lagt ned det där och kör sitt eget race fullt ut istället med mjukisbyxor \textsc{(se mjukisklädsel s.~\pageref{57a78bf29e9f6fb6a4dba89fc21bc897})} i skolan och cykelhjälm. En vanlig nörd kan normalt förstå varför vissa människor till exempel föredrar att köra bil framför att åka tåg. Storbossnörden har noll förståelse för detta eftersom tåget erbjuder så mycket större möjligheter att använda restiden till att spela Gameboy eller läsa regelböcker. Ett sådant kompromisslöst beteende borde rimligtvis försätta storbossnörden i en rad onödigt besvärliga \textsc{(se kineseri s.~\pageref{1f4cf2e5ffaa23a61ff5edd509c8c10f})} situationer. Men storbossnörden har för länge sedan funnit bot på detta besvär och rör sig helt enkelt bara i sammanhang där han inte ifrågasätts. Den enda risk han löper är i princip att stöta ihop med en annan storbossnörd och då kan det gå desto hetare till. Om två storbossnördar råkar i muntlig dispyt är det högst troligt att de blir ärkefiender och att oenigheten kommer följa dem i graven. En storbossnörd har nämligen aldrig fel.

}

\small{
\textbf{Storfräsare}
\label{4db17005692cd83e3e946a1311b81ed0}
 En storfräsare är en person som inte tvekar att ta på sig spenderarbyxorna när chansen finns att göra sig lite märkvärdigare än alla andra. Det är Levis 501or istället för Konsum-jeans, och riktiga sommardäck istället för att rycka dubben ur dom slitna vinterdäcken som gäller. På barnkalas ska man inte bli förvånad om storfräsarens unge dyker upp med en magnumflaska Pommac, medan knegarungarna får hålla sig till varsin 33 cl Champis \textsc{(s.~\pageref{ce7011d454a0f4377acffb4751e18a88})}. Självklart har ungen ett sånt där sugrör som är format som ett par glasögon också. Men egentligen är det inget att vara avundsjuk på, för innerst inne mår storfräsaren dåligt över att aldrig kunna uppnå den obetalbara njutning en arbetare kan känna när han unnar sig en kubb \textsc{(s.~\pageref{de7f6954ec8c6e346b8ba18ae018d334})} till kaffet på en tisdag eller bränner fram på en solbelyst landsväg med Bruce Springsteen i kassettdäcket.

 HEAD2: Exempel på storfräsare
 Thomas Jisander
 Jan Guillou \textsc{(s.~\pageref{63f2c8aba9686bc92efeb7eb21e35156})}
 Modebloggare
 Drottningen av Storbritannien
 Ove Äggberg
 Alexandra Pascalidou
 Värdens rikaste människa

}

\small{
\textbf{Storhetsvansinne}
\label{2f9c0ea6231e1de87c97eab41410c795}
 ska man inte skoja om. Det är en åkomma som ofta drabbar skäggiga män i Tyskland \textsc{(s.~\pageref{b1b58da783b6d5fa090f3015f1889869})} (på grund av den tyska mustigheten) \textsc{(se den tyska mustigheten s.~\pageref{682ccd5fdc3aff0c97e8845c3d6b6ca8})} men har också påträffats annorstädes. Storhetsvansinnet utbryter hos individen i en rad olika steg, som redogörs för nedan:

 1. Om han är man skaffar individen \textsc{(se individ s.~\pageref{41beed76a0af9b4f550f7ebdecd3e700})} skägg (och kanske fax) \textsc{(se fax s.~\pageref{236c3b7f761221f195b428aca2f06c4b})}
 2. Ut med det gamla skrället och in med en SUV \textsc{(se uv-ljus s.~\pageref{2239e04c73609ab9e8cc9b359552fa81})}
 3. Individen köper en Chapeau de paysan \textsc{(s.~\pageref{27aa75146d9ab723d1423168a2539d5d})}, med fjäder längst opp
 4. Nu vill individen lära sig spela ett blåsinstrument, kanske basklarinett \textsc{(s.~\pageref{34c56a45635881b3d3ad006192dd39ce})}
 5. Individen livnär sig till 87\% på mörkt bröd, typ pumpernickel
 6. Olika slags medaljer och märken dyker upp på individens rock- eller kappslag och byxorna har konstant cirkuspung \textsc{(s.~\pageref{c2c41b1921dcdcab22f7d32b62d2d17a})}
 7. Individens hus har fått ett torn. Han/hon bär särk \textsc{(s.~\pageref{7a522dc7e11bd1136642b3452855c1d6})} och talar till \quotetext{Davids folk}
 8. Individen är inte längre individen, utan \quotetext{Ezekiel}
 9. \quotetext{Ezekiel} står länge på höga kullar och ser ner på världen
 10. Så, plötsligt en dag, självantänder han/hon och återföds som en uv \textsc{(s.~\pageref{45210da832f9626829457a65e9e7c4d0})}






 HEAD2: Vanliga utlösande faktorer
 \begin{itemize}
 \item Att vakna varje morgon till tonerna av Strauss' \textit{Also spracht Zarathustra}
 \item Att livnära sig på en kost som till stor del består av viltkött man jagat själv
 \item Att skaffa fler än tre barn, som är brukligt
 \item Att i sin fågelskåderi-gärning börja närma sig slutet av listan på Sveriges \textsc{(se Sverige s.~\pageref{b1999637949ed135b2ca03f3a38460cc})} alla fågelarter
 \item Att ha en båt som man kan framföra själv, utan hjälp från andra
 \item Att typsätta en bok eller tidskrift själv under tidspress
 \end{itemize}

}

\small{
\textbf{Storksked}
\label{68dfcd370c776cd068ad3b00f9cecd7b}
 en är ett föremål som är populärt bland inredningsmaniker, framförallt i Stockholms innerstad, och används till att lägga upp en liten hög väl färgkomponerad och med obskyr apelsinbalsamvinäger smaksatt \quotetext{salla\textbf{t}} på gästernas fat vid middagsbjudningar. Storksalladsskeden tillverkas på så sätt att en skedstork \textsc{(s.~\pageref{76648d90910c2fd6fcd81b3f3f28d9ea})} infångas. Sedan sågas näbben av och fågeln släpps fri eller blir en av ingredienserna i en helt vanlig uvstorke \textsc{(s.~\pageref{6458dce510a0f0cfb9a720ee5d3e62be})}.

}

\small{
\textbf{Storsien}
\label{83f914b5fcd131d3ed802b838cce4aaf}
 är en mindre ort i Kalix kommun i Norrbotten \textsc{(s.~\pageref{0e8c003b75982032cde152609ee94154})}. Där fanns ett arbetsläger dit staten antagligen hade skickat samtliga Nissepediamedarbetare \textsc{(se Nissepedia s.~\pageref{62400dadecd90cb5cd39062abe5a3e4a})} om det velat sig riktigt illa.

}

\small{
\textbf{Storspov}
\label{9a741bd370edc42a6ff0daff656e4267}
 (Fornsv. \textit{stor} ung. \quotetext{ansenlig}, Forndan. \textit{spov} ung. \quotetext{mjödhorn}) är en vadarfågel (världens största!!!) i familjen snäppor. Storspoven är brungråspräcklig med en smal vit triangel från stjärtens bas upp på ryggen, stor som en bärsback ungefär och har en lång, nedåtböjd näbb som den använder för att äta med. Näbben ser ungefär ut som en sån där sked man äter sniglar med.
 Storspovens anseende upprätthålls av FFSSB \textsc{(s.~\pageref{c708e7f1f6d118408fc77e3517417d69})}.

}

\small{
\textbf{Storspren}
\label{182c2eb7a873a312be37f8f70d1a9b12}
 en är ett djur som återfinns sporadiskt i Sveriges \textsc{(se Sverige s.~\pageref{b1999637949ed135b2ca03f3a38460cc})} två nordligaste län. Den är nära släkt med fabeldjuret grip, men istället för en korsning mellan ett lejon och en örn är det en korsning mellan en storspov och en ren.

 ]]



 HEAD2: Föda
 Storspren äter mycket sällan, och då den väl gör det är den väldigt kräsen och äter bara lappskojs \textsc{(s.~\pageref{0d0eb99c8a08ae96acd7226a3cfec257})}. Resten av tiden livnär den sig på upplevelser, som till exempel att flyga omkring och titta på saker.

 HEAD2: Fortplantning
 Storsprenen är könlös, eller tvekönad, beroende på perspektiv \textsc{(s.~\pageref{1606dd19366985367d677f7b6de46e52})} och lägger ägg. Äggen är stora som ensilage, men skickligt kamoflerade så att de liknar gråsten. Efter en veckas ruvning kommer Storsprenkalven till världen, dock utan vingar. Ungarna föds upp genom att få höra skrönor \textsc{(se skröna s.~\pageref{c51cd220359f9f2755e98dcce2251e5c})} av de äldre storsprenarna. Sen när vingarna kommit skickas de ut i världen för att själva uppleva saker. T.ex. att bli dumpade, besöka Sagrada Familia i Barcelona, ligga med klasskompisar på en folkhögskola, se solen gå upp över Sarek och allt annat man kan uppleva här i världen.

 HEAD2: Storsprenen i kulturen
 I den förkristna samiska kulturen fyllde storsprenen jultomtens funktion, fast istället för presenter i materiell form så gav den bort, ni gissade det, upplevelser. Dessa kan te sig svåra att slå in, så de samiska barnens jular var extremt tråkiga.

}

\small{
\textbf{Storswänsk}
\label{716f41dcabef6599bcf08334a8a6ae27}
 en är en obehaglig typ! Ett riktigt as faktiskt. De flesta Storswänskar bor i storstäder och den högsta populationen har uppmätts i Stockholm. En storswänsk är en som anser att det finns en svensk nationalidentitet, givetvis utformad av akademiker i Uppsala \textsc{(s.~\pageref{1db4e388df1df7057b8f3d984c65ee88})}, Stockholm \textsc{(s.~\pageref{edcd259e0a03c7ab70feb186bae19f13})} men till viss del även i Göteborg \textsc{(s.~\pageref{0e9b11e435dd9f73e87e868667e1d6f0})}. Enligt Storswänsken är Sverige \textsc{(s.~\pageref{b1999637949ed135b2ca03f3a38460cc})} ett land och inte som det egentligen är en samling av stater för de har vapenmakten att hävda det. De förespråkar en ekonomisk politik som handlar om att suga ut de delar av landet som de ockuperar. Lögner är även frekvent i deras propaganda, som att Norrbotten \textsc{(s.~\pageref{0e8c003b75982032cde152609ee94154})} kostar mer än det generar eller när och hur Svea har befolkats.

}

\small{
\textbf{Stort tack}
\label{395d7f397fe11de8c07c923900fd0576}
 Dessa personer förtjänar ett stort tack:

 \begin{itemize}
 \item Karl Marx
 \item Clas Ohlsson
 \item Hjalmar Branting
 \item Mary Wollstonecraft
 \item Les Paul
 \item John Fogerty \textsc{(se The fog s.~\pageref{576875ef0042ff21c04f5f1b9377d4e7})}
 \item Homi K. Bhabha
 \item Antonio Gramsci \textsc{(s.~\pageref{d4d0da57d321555b3550f1d7cffa3249})}
 \item Jan Wilsgaard \textsc{(s.~\pageref{213a0ab775d17e92dfd78748a2a1bc3b})}
 \item Raewyn Connell
 \item Tony Iommi
 \item Bobby Liebling
 \item Jon Ashbourne
 \item Charlie Mopps
 \item Tomas Tranströmer
 \item Jonas Claesson \textsc{(s.~\pageref{65b9252dc2a0fa610d59d72854440ae7})}
 \item Nicklas Hållén
 \item Artur Hazelius \textsc{(s.~\pageref{cfe3ab83bbf192ab78a5b06cdd7cbf9f})}
 \item Matt Pike
 \item Johan Granberg
 \item Judith Butler
 \item Sven-Eric Liedman
 \item Åsa Linderborg
 \end{itemize}

}

\small{
\textbf{Streiff}
\label{c93ace220b0ebd1edd48bc4a7344fada}
 var den häst \textsc{(s.~\pageref{b4c608370b339da095c5f8db7fab0945})} Gustav II Adolf red åt helvete på i Lützen. Pollen klarade slaget men dog några dagar senare på väg tillbaka till Sverige \textsc{(s.~\pageref{b1999637949ed135b2ca03f3a38460cc})}. Precis som med kadavret efter kungen bestämde sig soldaterna för att släpa hem den döda hästen till slottet. Kungen var så illa åtgången så honom kunde man inte göra något åt, men hästen var ännu så pass fräsch att det gick att stoppa upp den.

 Streiff var en betydligt större och snabbare häst än vad som var vanligt i den svenska armén, så kanske (förmodligen) var det dennes fel att kungen vart kanonmat eftersom den sprang före alla andra. Hade det inte varit för detta hästskrälle hade kanske krigarkonungen levat än idag. Det får vi aldrig veta.

 Den som vill visa sitt hån mot denna landsförrädare kan gå till Livrustkammaren där den står än idag och skäms.

}

\small{
\textbf{Strelka}
\label{1cf458ff9883dea9dd2083d22789aa0d}
 var den tredje hunden i omloppsbana i rymden.

}

\small{
\textbf{Streptokocker}
\label{7236e485323b49f430558d3ed6d20472}
 är en bakteriefamilj som växer i par eller i kedjor. Streptokocker kan exempelvis leda till kvarka \textsc{(s.~\pageref{845b5a3b4cbfd68185b5bc6877f01a42})}. Fast egentligen är dom inte så farliga bara man ser upp.

}

\small{
\textbf{Stress}
\label{e10a36f1a5231e597daf8f42dc1ab55a}
 är en känsla och ett tillstånd som infinner sig till exempel när man är i slutskedet av sitt avhandlingsprojekt eller om man sitter instängd i en 1x1m stor bur och någon står och tutar mot en med en sån där hockeytuta. Stressen tar sig olika uttryck. Beroende på om det ses som ett tillstånd eller en känsla leder detta tillstånd antingen till att vissa känslor infinner sig eller så leder stresskänslan till ett pärlband av andra emotionella hemskheter. Man blir arg. Man blir aggressiv. Man blir förvirrad och disträ. Man blir känslig. Kombinationen av att vara stressad, arg, aggressiv och förvirrad kan leda till många olika tragedier och dråpligheter, som normalt i sin tur leder till skam \textsc{(s.~\pageref{e7d275bbd2f3522805002be76a53ccd8})}. Men frukta icke! Det finns vissa mått man kan tilltaga för att gardera sig då stressens mörka moln samlas över en:

 Kräma på lite sjysst rymdig doom \textsc{(s.~\pageref{b4f945433ea4c369c12741f62a23ccc0})} och rulla en jolle \textsc{(s.~\pageref{4fe195f73917395e8a5851dc036ef8bc})}. Rök jollen och känn hur stressen sjunker undan samtidigt som vissa delar av ljudbilden liksom träder fram. Tänk på att det faktiskt är något utfattigt jävla geni som skapat denna fantastiska effekt som tidigare helt undgått dig. Byt så småningom till nån mer mellow rymdrock eller kraut. Drick té. Läs en bok.

}

\small{
\textbf{Strigiformes au Riesling}
\label{f79211a2d52de6abaa480e60938e98fe}
 (Fr. uv \textsc{(s.~\pageref{45210da832f9626829457a65e9e7c4d0})} med Riesling) är en populär fransk maträtt som stammar från det regionala Alsace-köket. Rätten är en smakrik gryta som består av uv kokad i vin \textsc{(s.~\pageref{62911ad86d6181442022683afb480067})} från Alsace-provinsen (s.k Rieslingvin). Uvköttet kokas tillsammans med kryddor och grönsaker, först i hög temperatur och sedan under ett antal timmar på mycket låg temperatur. På så vis skapas ett brett, mustigt smakregister som många förknippar med höstkryddor och en rund vinsmak. I andra provinser finns varianter av Strigiformes au Riesling, så som Strigiformes au Champagne.
 HEAD2: Historia
 Rätten påstås ofta härstamma från gallisk matkultur och sägs ha varit vanlig hos det romerska galliska provinsens övre samhällsskikt, men detta har inte historiskt belagts. Säkert är dock att rätten i olika varianter varit vanlig i och omkring Alsace-provinsen sedan medeltiden. I sin mer rustika version används uggla istället för uv, då uggla varit vad den medellösa landsbygdsbefolkningen \textsc{(se bönder s.~\pageref{30a6fc00c9102680b8196b1b79935ec4})} kunnat tillgå, medan uv alltså har varit förunnat adels-, borgar- och prästerskapet.

}

\small{
\textbf{Strulputte}
\label{21651c95306d1b1e281443f8620910da}
 En strulputte är någon som har strulat till det för sig eller är i färd med att göra det. Kanske har man varit hemma hos en polare och rökt gräs ur bong och lyssnat på \textit{Uprising!} Senare, på vägen hem, har man kanske fulkopplat TV8-profilen Lars Adaktussons Golf GTi 2004, blivit stoppad av länsman \textsc{(se polis s.~\pageref{fa296149fa58bfd4408e407cc3fd3be5})} och oförhappandes hamnat i handgemäng. Exempel på kända strulputtar är:
 \begin{itemize}
 \item Burre \textsc{(s.~\pageref{6e54c504971bbe1f8d46e006550af1ca})}
 \item La Camilla
 \item Kungen
 \item Christer Pettersson
 \item The Hof
 \item Mona Sahlin
 \item Kurt Cobain
 \item Tomas de Quincey
 \item Hunter S. Thompson
 \item Mel Gibson
 \end{itemize}

}

\small{
\textbf{Strumpor}
\label{90d8dd7ef7d3af061761267fda87699f}
 är ett slags cylindrar av tyg som sytts samman längst ned och används för att beklä fötter. Många människor i västvärlden byter strumpor dagligen. Bland de som inte gör det finns en viss överrepresentation av unga killar utan karriärsplaner. Storfräsaren \textsc{(se Storfräsare s.~\pageref{4db17005692cd83e3e946a1311b81ed0})} bär golfstrumpor med diskret mönster och betalar 150 spänn paret. För vanligt hederligt folk duger det bra med tubsockor med logga från nåt lokalt åkeri som man fått som tröstpris på pimplingstävling eller jaktstig.

}

\small{
\textbf{Stryknin}
\label{0a7f21d52b763dee57322b47f12d3fd2}
 är ett gift som heter som det gör för att man \quotetext{stryker med} av det.

}

\small{
\textbf{Strykrädd}
\label{75bdfdb38f443fea6318f325667d096c}
 Folk som gärna torgför avvikande åsikter i tryggt sällskap \textsc{(se folkkök s.~\pageref{15983d1934522d4d08e766108357201b})} har en tendens att backa i den hårda verkligheten \textsc{(se tung industri s.~\pageref{454e5e8cb27bed118f0a6a1a01a6e6a9})}.

 Eller; folk som gillar avvikande kläder \textsc{(se hippies s.~\pageref{4dc77d6258fd18e7c0dd5eece5c7c81c})} eller \textbar bland normala människor.

}

\small{
\textbf{Stråtrövare}
\label{49bcf5791fb20ffce4a0150d38dee0ac}
 Ett anrikt yrke för den som inte har något att förlora, och går ut på att ta den ekonomiska omfördelning som staten misslyckas med i egna händer. Stråtröveriets ädla konst är lika gammalt som äganderätten \textsc{(se äganderätt s.~\pageref{2d92d92f7fa233484ba06555728bef2a})} och går ut på att med våld avtvinga någon pengar eller annat som man behöver. Till skillnad mot det närliggande yrket tjyv så går stråtrövaren inte in i folks hem för att tillskanska sig ägodelar, utan låter istället berget komma till Mohammed. Stråtrövaren arbetar istället längs rikets landsvägar, beväpnad med förslagsvis ett armborst eller en knölpåk, och väntar på att en vagn ska passera. I bästa fall är det självaste kungen och stråtrövaren kommer därifrån med en skattkista full med guld, diamanter och smaragader, i sämsta fall är det en bonde på väg till marknaden och bytet en tunna sättpotatis. I dagsläget ägnar sig inte särskilt många åt stråtröveri utan de flesta, påverkade av förbränningsmotorns upptäckt i seklets början, tog steget att bli vägpirater.

}

\small{
\textbf{Strövtåg}
\label{5b4ab89daf17b6cd8f620f2eef5dee40}
 SJs skyddshelgon.

}

\small{
\textbf{Styckmordet i Sydafrika}
\label{a3da0ae59fb96227727302044bcae0cb}
 Vid det hemska styckmordet i Sydafrika var det Nelson man-dela.

}

\small{
\textbf{Städa}
\label{793f88411898643a984c343fa86deb5e}
 Aktivitet man gör istället för att plugga.

 Category:Psykologi och beteenden \textsc{(s.~\pageref{a15cb4e60394f5be56df613817a9efb2})}

}

\small{
\textbf{Ståuppkomiker}
\label{3d6d423564dc06ac53646ac45691566f}
 bör vara antingen tjocka, Göteborgare, turk eller helst alla tre.
 Man skämtar bara lamt om att man är tjock, har svart hår och kommer från Göteborg \textsc{(s.~\pageref{0e9b11e435dd9f73e87e868667e1d6f0})}, allt annat straffas med landsförvisning till Gotland. Den som vill bli ståuppkomiker kan med fördel studera Adde Malmberg \textsc{(s.~\pageref{1390facdddaee5ed00a964fbe93b30b9})}.

}

\small{
\textbf{Stöcksjö sunny resorts}
\label{8939dfc7e5cb4ccee97180310df3b7e4}
 , SSR, är ett turistmål i en by belägen en mil söder om Umeå centrum. Bland utbudet som erbjuds så utlovas något slags safari med bl.a. myskoxar. Det sorgliga med denna fattiga turistattraktion är dock att det inte alls är myskoxar utan två nordsvenska hästar som ägaren kastat två bruna trasmattor över ryggen på. För att ge hästarna horn har slanet \textsc{(se slan s.~\pageref{caaad522de864ab45ed679c4a16edd8d})} försett deras huvuden med cykelstyren och färgat dessa en färg snarlik hornfärg.

 Alla som besöker gården erbjuds att ta med sig sin egen vikt i metallskrot på vägen därifrån. De har helgöppet.

 HEAD3: Se även:
 Holmsunds tropikhus \textsc{(s.~\pageref{5b087d935637ad4d1823cf48036e9be6})}

 Klä ut sig till ett djur \textsc{(s.~\pageref{78663eff2fe898143e822b7f9d4851f7})}

}

\small{
\textbf{Stödkorv}
\label{06deac330885957aed93b0dfc63d32bd}
 När man beställer ett skrovmål på sin lokala grill och inte vill lida hungerns kval ända tills det är klart kan beställningen med fördel kompletteras med en korv, en s.k. stödkorv. Korvarna tar inte lika lång tid att tillaga som gängse hamburgertallrik och tillfredsställer därför kunden mer eller mindre omedelbart.

}

\small{
\textbf{Störande utekatter}
\label{ef25cdd880a2e4862a2eccc8ca6aa2ce}
 Lösgående katter \textsc{(se katt s.~\pageref{0fd9accd1d8c95e86a96f681b6805948})} kan orsaka problem i tätortsmiljö. Katter som förorenar i sandlådor, rabatter mm och som revirmarkerar på byggnader och fordon kan vara mycket irriterande för närboende. Katterna kan också orsaka lackskador på bilar, klösa sönder dynor och andra textilier och orsaka allergibesvär och trafikolyckor. Sådant kan ställa till svår grannosämja.

 HEAD2: Hur förhindrar jag att min katt stör andra?
 \begin{itemize}
 \item Prata med dina grannar om du har utekatt. Fråga om de har något emot att katten går fritt.
 \item Håll katten under uppsikt eller inomhus om det behövs för att förebygga skador eller avsevärda olägenheter för grannarna.
 \item Märk katten med halsband eller öronmärkning och håll den i vårdat skick. Den kan annars tas för en vildkatt och riskera att infångas och avlivas.
 \item Kastrera katter som inte ska användas i avel.
 \item Tänk på att du som kattägare har ansvar för katten även när den går lös!
 \end{itemize}

 HEAD2: Råd till dig som störs av katter
 \begin{itemize}
 \item Rör det sig om vildkatter? En katt som med skäl kan antas vara övergiven eller förvildad får avlivas av jakträttsinnehavaren, men inom tätbebyggt område krävs tillstånd från Polisen.
 \item Rör det sig om tamkatter ska man försöka identifiera ägaren och till denne framföra sitt ärende. Prata med grannarna och gör dem uppmärksamma på problemet.
 \item Försök avskräcka katter från att vistas på din tomt, det finns t ex luktande medel som håller katter borta att köpa i butiker som säljer djurtillbehör.
 \end{itemize}

 [http://www.nordmaling.se/default.aspx?id=6631]

}

\small{
\textbf{Stövare}
\label{53be299bc9a8935b8740369c0bc69fd2}
 Fjantiga hundar med oproportionerligt stor nos. Om ägaren är en äldre man är denne troligen gammpojk \textsc{(se gammpojkar s.~\pageref{4dcf505f68cb2f0708155b78f56ad632})}, är ägaren en yngre man är denne troligen harmynt, om ägaren är en kvinna \textsc{(s.~\pageref{9a7760b2521c3471c47cd5d789a2d324})} är hon troligen änka.
 Stövare låter som en hes överklassgubbes skratt och springer runt, runt i timmar medan dess ägare sitter och grillar korv. Detta anses vara en väldigt fin jaktform då det inte kräver någon som helst ansträngning eller kunskap. Stövarjakt är ett typiskt  storsvenskt \textsc{(se storswänsk s.~\pageref{716f41dcabef6599bcf08334a8a6ae27})} fenomen och därmed förkastligt för en renrasig homo sapiens.

 Enligt nyligen inkomna uppgifter är Stövare ytterst populära i byar runt Malå \textsc{(s.~\pageref{41da4620e87888eaaeafcb3004a8d177})}.

}

\small{
\textbf{Sugmästare}
\label{1a01ad3847daa7eabaa6496d5765be89}
 En sugmästare är en person som förstör för andra, medvetet eller  omedvetet \textsc{(se det omedvetna  s.~\pageref{d653b6e53612e79237853e2e4dfaf4a4})}. Det tydligaste exemplet är Metallicas nye basist Robert Trujillo som är jätteduktig på sitt instrument men ändå gjort att man vill spy när man hör gruppen och fått Cliff Burtons lik att ruttna lite snabbare.
 Sugmästare förhåller sig antonymt till ägmästare \textsc{(s.~\pageref{8324518500d7e7ccd22ae364887d4476})}.

}

\small{
\textbf{Sultan}
\label{9af82031d374b97c9e73132a413cbdf5}
 En Sultan är någon som kan mäta sin makt i kroksablar. Han (för de är påfallande ofta män) har ofta ett harem trots att han är helt impotent. Vill han handla så nyttjar han en bulvan \textsc{(s.~\pageref{a83ace112700bb9792e6f984abbc19b6})}.

}

\small{
\textbf{Sune Almkvist}
\label{8b26ee9eb5634176df1f49f6cbb71708}
 (1886-1975) var sveriges första stora bandyspelare och är bandyns stor grabb \textsc{(se Stora Grabbars och Tjejers Märke s.~\pageref{3b527f8b13885eb277c77de4b1f51658})} nummer noll. Han vann 11 SM-guld (samtliga finaler han var med i) och satte målrekord med nitton strutar i en och samma match. Han spelade då med IFK Uppsala och knäckte Krigsskolan (!!!) med 27-1.

}

\small{
\textbf{Supa ensam}
\label{0b1b1b2f48122e4dbb3d6fe13a1ed94f}
 Att sätta sig ner med några starköl eller en flaska sprit och bara låta det ske. Finns det något finare? Många ser ner på folk som super ensamma, men belackarna är i nio fall av tio tråkiga töntar som är för fega för att sätta sig på en pinnstol, vrida på pink floyd på helgvolym \textsc{(s.~\pageref{3539fdeb41a5b216f614b6ced9ff5cff})} och med några stadiga järn som färdmedel företa sig en resa till sitt eget inre.

 Men allt ensamsupande innebär inte spirituella resor. Ibland kan det vara schysst att bara kolla på larviga djurklipp på youtube, höhöhö:a lite och bli så full att man glider av stolen.

 Kändisar som är bra på att supa ensamma:
 - Ulf Brunnberg
 - Burres \textsc{(se Burre s.~\pageref{6e54c504971bbe1f8d46e006550af1ca})} farsa i Bamse
 - Ika i rutan (även om hon småfuskar, då hon krökar tillsammans med sin skelettpolare Åke)

}

\small{
\textbf{Superhjälteserier}
\label{c684b269a6e2112d19fe6ea6b203689d}
 Om man är en tönt kan man önska man hade superkrafter, eller åtminstonde tajta kläder med cirkuspung \textsc{(s.~\pageref{c2c41b1921dcdcab22f7d32b62d2d17a})}.

}

\small{
\textbf{Surdegspappor}
\label{617533958bf226f7259a890bb6c15822}
 En rent utav livsfarlig sammansvärjning

}

\small{
\textbf{Surfin' bird}
\label{a6167bedaff9931674f3f67c27f8607c}
 är ett fågeldjur inom familjen gäddoppingar. Den förväxlas lätt med sin nära släkting hackspetten i Kalle Ankas julafton men känns igen på sitt karaktäristiska läte \quotetext{papa-oom-mow-mow} i stället för \quotetext{ara-pa-pa-pa-pa-pa-pa-pa-pa-pia}, som den sjunger högt och ofta. Arten återfinns vanligtvis på nedlagda soptippar i Danmark där den lever på cigarrstumpar och tuggummin. På power meet-helgen flyger eller liftar den till Västerås för häckning och flipperspelande.



 Andra arter inom familjen gäddoppingar:

 Hjulben

 Palle Kuling \textsc{(s.~\pageref{5929d8f94522386aefeb70ec4f139090})}

 Duffy duck

}

\small{
\textbf{Surrande ljud}
\label{461b63e21c531b2c2102d07af5ee13cf}
 är ett ofta förekommande miljöproblem i vår allt mer maskinella omvärld. Surrande ljud är inte bara irriterande utan kan också vara skadliga då de kan orsaka hörselskador och skapa stressyndrom så som huvudvärk, sömnproblem och fetor ex ore \textsc{(s.~\pageref{d3b96d618fb972d12fb0cdfdeaf13a98})}. Därför är det viktigt att kunna lokalisera orsaken till surrande ljud så att de om möjligt kan avlägsnas ut den direkta ljudmiljö i vilken man framlever sina dagar.
 HEAD2: Vanliga surrljudshärdar
 Hör du ett surrande ljud bör du kontrollera om det kommer från de ljudkällor som listas nedan. Om ingen av följande källor verkar vara orsaken till ljudet bör du inkalla en ljudexpert \textsc{(se högtalartips s.~\pageref{67d1cdf9ebf847fa5430e998da2b7085})}.
 \begin{itemize}
 \item Grannen har startat sitt Messerschmitt och har satt kurs mot ditt gods \textsc{(se spärrballong s.~\pageref{6be5aa404a270879d826e417222c7857})}
 \item Alexander Bard har dykt upp i TV-rutan och talar om sin liberalism
 \item Gammalt kylskåp
 \item Dålig datafläkt
 \item CD-skiva som sakta slipar ner laserhuvudet \textsc{(se huvud s.~\pageref{e906cd95a540df9b16d0460fb4cf0adc})} i din CD-spelare
 \item Fluga mellan fönsterglasen
 \item Någon står bakom dig och talar till dig med väldigt låg röst
 \item En liten robot har tagit sig in i byggnaden och åker omkring och rekognoserar terrängen.
 \end{itemize}

}

\small{
\textbf{Suveränitet}
\label{c832bdabb5cbd330f791c6cce129a250}
 Högre lärosäte som endast ger undervisning inom smalare och roligare ämnen såsom bästa tum på vinylskivor och fördelar med att bära magväska, om man får fråga Orgasmatron Andersson \textsc{(s.~\pageref{992f857a2415202c7eb4b9f973ea11a0})}, eller nerv- och psykologiska skador på föräldrar till trillingar - i all hast döpta efter sociologiska nyckelfigurer - om man frågar någon annan.

}

\small{
\textbf{Svag mat}
\label{0526761422547962084b0c0c3701cf91}
 är motsatsen till stark mat \textsc{(s.~\pageref{8f43507e39b50520de82de94d1d88d94})}.
 Svag mat kan för en del  människor förknippas med pensionärsmiddagar med ett stilla tickande från moraklockan som soundtrack. Blickar som möts och väntade möten med byte av stomipåse och tantsång \textsc{(s.~\pageref{25b8200d011a4766f4b3a64a2e17f374})}.
 Det finns inga undersökningar som visar att svag mat har en välgörande effekt på hälsan och själen.

 Exempel på svag mat är \quotetext{Ängamat} och \quotetext{Krösamos} .

}

\small{
\textbf{Svan}
\label{f80f1875ab3ebccf935723ba83b6da63}
 är ett djur som under evolutionens snirkliga resa försetts med vingar och näbb, vilket kvalificerar den till gruppen fåglar. Den har vit kropp och svart huvud, och påminner därför lite om sin nära släkting späckhuggaren. En fullvuxen svankropp är stor som en bag-in-box \textsc{(se Bag-in-box s.~\pageref{1fdd5e1bb07154385669cd70e53bd354})} ungefär och halsen är lång som en Uzi. Den är en flitig utlandsresenär under vinterhalvåret. Som många vet bildar svanar par och håller sedan ihop resten av livet. Vad många inte vet är att dom hatar varandra.
 HEAD2: Svan som råvara
 Svan är en av ingredienserna i uvsvane \textsc{(s.~\pageref{c5081b14cdeb1ff42b655213e80c9d51})} och svanskrove \textsc{(s.~\pageref{e543ead268a283bfdb5ea638d6cca4a2})} och huvudråvara vid tillverkningen av svanväskor \textsc{(se svanväska s.~\pageref{f5cd47fc9fb6544d2d9e10009334bece})}.
 HEAD2: Trivia
 Svanen är det djur som har förärats störst antal fjädrar och är därmed jobbigast att plocka. Detta leder ofta till att man (i likhet med potatis) stuntar i att \quotetext{skala} den innan förtäring.

}

\small{
\textbf{Svanskrove}
\label{e543ead268a283bfdb5ea638d6cca4a2}
 är en maträtt som påminner om calskrove \textsc{(s.~\pageref{84ff54e779ee49fdad21e17c20f14453})}. Calskroven utgörs av ett skrovmål inbakat i en calzone \textsc{(se calzona  s.~\pageref{42bf60ead842afe1df27d41324e41a02})} och serveras på Pizzeria Tre kronor i Skellefteå. En svanskrove skiljer sig från denna rätt på så vis att svanskroven är ett skrovmål nedfört i en svan \textsc{(s.~\pageref{f80f1875ab3ebccf935723ba83b6da63})}.
 HEAD2: Se också
 \begin{itemize}
 \item Calskrove \textsc{(s.~\pageref{84ff54e779ee49fdad21e17c20f14453})}
 \item Transkrove \textsc{(s.~\pageref{1188281a09fb681b922e45663e5ffc4b})}
 \item Smörskrove \textsc{(s.~\pageref{c3ec1fc646dfd34ddd483f8031d649c9})}
 \item Uvsvane \textsc{(s.~\pageref{c5081b14cdeb1ff42b655213e80c9d51})}
 \item Uvstorke \textsc{(s.~\pageref{6458dce510a0f0cfb9a720ee5d3e62be})}
 \item Johnskrove \textsc{(s.~\pageref{92a6f4a71ab0087f48ba4aab7db89bdb})}
 \item Svanväska \textsc{(s.~\pageref{f5cd47fc9fb6544d2d9e10009334bece})}
 \end{itemize}

}

\small{
\textbf{Svanväska}
\label{f5cd47fc9fb6544d2d9e10009334bece}
 En svanväska är en väska gjord av en svan \textsc{(se Svan s.~\pageref{f80f1875ab3ebccf935723ba83b6da63})}.

}

\small{
\textbf{Svart alibi}
\label{168a6ccab282409b534cf3a9fcdc7029}
 Fordom tida personifierades det svarta alibit av Onkel Tom.
 Numera av Nyamko Sabuni som dessutom dubblar som folkpartiets kvinnliga alibi \textsc{(se kvinnligt alibi s.~\pageref{60da199ecfe5b75a702ff11156c333df})}.
 Ett svart alibi \textsc{(se adjektiv s.~\pageref{67d02147cd8595eaf13c1a90aba99dcc})} behöver varje organisation som vill framstå som fri från fördomar trots att man driver en politik i klass med Sydafrika före 1994.

}

\small{
\textbf{Svarta tavlan}
\label{a8820aea1395d3fb35ac982fdf5d6378}
 är ett uråldrigt pedagogiskt hjälpmedel, avsedd att skriva mattetal, tyska verbböjningar och lärarens för- och efternamn på. Man använde en vit bit krita för att skriva på den svarta tavlan och i enlighet med 1900-talets fäbless för barbari och blodsutgjutelse användes avhuggna harlemmar för att sudda på den. Redan under dess storhetstid var alla, både lärare och elever, införstådda med att det måste finnas ett bättre system för lärande än svarta tavlans dammiga gnisslande och den skolkritslunga som många ådrog sig av inandning av kritdamm. Under 90-talet kom svarta tavlan att ersättas med den glassiga whiteboarden och sedemera den interaktiva tavlan.

 Roland Barthes, strukturalismens fader, kan på denna bild inte hålla sig från att fnissa till under en föreläsning och bara peka på svarta tavlan, för att dra uppmärksamhet till det absurda föremålet. Den uppmärksamme kan för övrigt lägga märke till att den här bilden är tagen efter att Barthes under en fest tappat sin klocka i en  fonduegryta \textsc{(se fondue s.~\pageref{98254ae1bc17c73df8f3d6a47beb333f})}, stoppat ned handen i grytan och ådragit sig tredje gradens brännskador.

}

\small{
\textbf{Sven "Sleven" Säfwenberg}
\label{3db9bb625a0cf732a4d63171f8cc9db1}
 Sven \quotetext{Sleven} Säfwenberg (1899-1950) var en svensk bandymålis av grov kaliber. Han spelade i IFK Uppsala och i Sirius, samtidigt, men blev tvingad av Sune Almkvist \textsc{(s.~\pageref{8b26ee9eb5634176df1f49f6cbb71708})} att välja sida och valde då den förra klubben. När IFK Uppsalas förre målis, Seth Howander, klev upp i fältet och blev back fick Sleven ta över Howanders plats i buren, samt hans benskydd och skolmössa. Som 16-åring höll Sleven nollan mot AIK i SM-finalen 1915 och arton år senare släppte han in en strut mot 11 för det egna laget i sin sista SM-final - den gången mot Karlstad-Göta. På det fick han bandyns Stora Grabbars och Tjejers Märke \textsc{(s.~\pageref{3b527f8b13885eb277c77de4b1f51658})} nummer ett. Målskillnaden på tio baljor är den största någonsin i en SM-bandyfinal. Sleven försörjde sig som fabrikör av bandybollar och dog av en hjärtattack under en bandymatch mellan Forbacka och Sandviken 1950. Det sägs att han alltid hade en bandyboll i fickan och det låter ju rimligt eftersom han verkar ha gillat bandy jättemycket.

}

\small{
\textbf{Sven tuba}
\label{80f86418d24b9ea4077749d98546bdbc}
 är en viktig hörnsten inom finlandssvensk punk.

 Ej att förväxla med Arvid Tuba.

}

\small{
\textbf{Svensk bilsemester}
\label{55fc2a78ed3ee9c2d5a223c3e515d577}
 är en riktig klassiker i de breda folklagren, men hos storfräsare \textsc{(s.~\pageref{4db17005692cd83e3e946a1311b81ed0})} duger det naturligtvis inte att tvinga in familjen i 740 \textsc{(se Volvo 740 s.~\pageref{e262951543da05bac43c7b87235a115c})}, bränna ner till Kolmården och sedan slå upp ett tält och supa sig full. Hos storfräsaren är det Plaja del sol och komplicerade transaktioner på utrikiska som gäller. Svensk bilsemester är trots allt precis det storfräsaren skyggar för: enkelhet och frihet och en fantastisk chans att samtala med andra bilister om det alltid för höga bensinpriset.

 HEAD2: Tips inför bilsemestern
 \begin{itemize}
 \item Föreställ dig att du tagits som fånge av en framtida postapokalyptisk civilisation vars enda kvarvarande nöje består i att anordna death races. Vägen är en tävlingsbana och alla andra bilister är dina fiender. Arbeta upp en alarmerande hög stressnivå och ge dig sedan på att komma först i alla lägen som kan tänkas dyka upp.
 \end{itemize}

 \begin{itemize}
 \item Packa tills bilen är full. Finns det plats kvar är det bara att fylla på med grillkol.
 \end{itemize}

 \begin{itemize}
 \item Låt påskina att du menar allt du säger på största allvar genom att vända huvudet lite åt sidan och skrika allt du orkar åt barnen i baksätet. Redan vid första fikapausen på en skräpig parkeringsficka vid Sveriges \textsc{(se Sverige s.~\pageref{b1999637949ed135b2ca03f3a38460cc})} stolta del av Europaväg 4 ska barnen helst vara rejält rädda för dig.
 \end{itemize}

 \begin{itemize}
 \item Peka ut betonganläggningar som skymtar från motorvägen och påstå att du varit där när du låg i lumpen. På så vis ger du de vettskrämda barnen ett stycke familjehistoria samtidigt som de lär sig något om vikten av rikets försvar.
 \end{itemize}

 \begin{itemize}
 \item Efter att nogsamt ha studerat GBs priskarta, inför utan förklaring regeln att man bara får välja en glass för en summa som gör det omöjligt att ta en som är god. På så vis blir semestern en pedagogisk lektion som illustrerar att även om man kan ha det bra och roa sig ibland så är det för det mesta jävligt torftigt.
 \end{itemize}

}

\small{
\textbf{Svensk fluortant}
\label{2b583e2c23a0890a1595d2f933b710a0}
 En svensk fluortant är en tant från Sverige \textsc{(s.~\pageref{b1999637949ed135b2ca03f3a38460cc})} som fluorear saker. Ibland fluorerar hon grundvattnet och då kan det gå riktigt illa.

}

\small{
\textbf{Svenska jägareförbundet}
\label{e56c5b0ef648ae8e763d292c96f7894a}
 Tidigare namn på Sverigedemokraternas seniorförbund. Ändrades på anonym gruppbegäran när en pedofilring tog samma namn och man ville undvika risken att medlemmar glömde betala medlemsavgift till båda.

}

\small{
\textbf{Svenska Kennelklubben}
\label{da81059980796160e0efdfb7e26db8d9}
 Intresseförening för män och kvinnor som heter \quotetext{Kennel}. Föreningen har idag (2011) inga medlemmar, men märkligt nog finns det 5 personer som heter Kennel.

}

\small{
\textbf{Svenska Kennetklubben}
\label{13ceb6e883dd0d39e9246cfe89bd17bd}
 Intresseförening för män vilka bär namnet Kennet \textsc{(s.~\pageref{eb251e3745d960e2100c5435a32764c4})}.


 HEAD2:  Externa länkar

 [http://www.kenneth.se/ Föreningens hemsida]

}

\small{
\textbf{Svenskt näringsliv}
\label{cacfb64dea9ea7e46b014ee8b81d1818}
 är en nationalsocialistisk organisation som tidigare gick under det mer tydliga namnet Svenska Motståndsrörelsen.

}

\small{
\textbf{Sverige}
\label{b1999637949ed135b2ca03f3a38460cc}
 Avlångt land i närheten av Östersjön. Lever på ryktet om sin fantastiska välfärd, eftersom nyheten om att den nyliberala regimen har avskaffat densamma ännu inte nått ut till utländska nyhetsbyråer.

 Man kan vara nazist utan att älska Sverige, men man kan inte älska Sverige utan att vara nazist.


 HEAD2: Lagar och förordningar
 I Sverige kör man bil på höger sida av vägen och det har svenskarna gjort sedan 1967. Det råder också tyst trafik sedan 1935, det vill säga att man inte ska tuta i onödan, utan använda tutan till dess ursprungliga funktion; hälsa på folk man känner.
 HEAD2: Ekonomi
 Sverige har sålt alla sina stora industrier till Polen, Kina osv. och de som inte blev sålda sattes i konkurs eller flyttades till Pajala.

}

\small{
\textbf{Sverigedemokraterna}
\label{44e1558e17b71c9b066fe6d5e1f2cc63}
 (skämts. sammansättn. a. platsnamnet \quotetext{Sverige \textsc{(s.~\pageref{b1999637949ed135b2ca03f3a38460cc})}} och gr. \quotetext{demokrati}) är en partibildning som i huvudsak samlar sydsvenska grisbönder, innebandykillar och samtliga anställda inom norrortspolisen. Ideologiskt står man nära Sveriges andra högerpartier och röstar i nio fall av tio med den borgliga alliansen. SD har under sin första mandatperiod haft två prioriterade politiska arbetsområden.

 \begin{itemize}
 \item Att arbeta oförtröttligt med att genom parlamentariska beslut nedrusta välfärden och försämra arbetstagarens ställning på arbetsmarknaden.
 \item Att kasta ut alla invandrare för att det är deras fel att välfärden nedrustats och arbetstagarens ställning på arbetsmarknaden försämrats.
 \end{itemize}


 HEAD2: Kulturpolitik
 Som symbol har man en \textit{chic} blå-gul blomma, för det finaste som finns är ju Sverige, och det näst finaste är ju påhittade blommor.
 Partiets officiella sång heter \quotetext{tjalalalala}.
 HEAD2: Kritik
 SD har sedan urminnes tider av ingen anledning motarbetats av ett osynligt imperium bestående av vanvettiga ockultister som arbetar som/förkläder sig till frilansande journalister. Dessa hackar inte sällan sverigedemokraters facebook- \textsc{(se facebook s.~\pageref{26cae7718c32180a7a0f8e19d6d40a59})} och twitterkonton och skriver korkade rasistiska och homofoba saker.

}

\small{
\textbf{Sveriges sju underverk}
\label{f4f71e4db3f279d42d840c805d75820c}
 utsågs i två omröstningar som, oberoende av varandra, hölls av Aftonbladet och P1s Vetenskapsradion Historia under nådens år 2007 \textsc{(se sjua  s.~\pageref{e7bf63fa6d0d29bd89c23f833b979a15})}.

 HEAD2: Underverk utvalda av Aftonbladet
 I Aftonbladet blev följande byggnadsverk utsedda till underverk:
 \begin{itemize}
 \item Göta Kanal (ej att beblanda med filmen \quotetext{Göta Kanal} med bla. Janne \quotetext{Loffe} Carlsson)
 \item Visby Ringmur
 \item Regalskeppet Vasa
 \item Ishotellet i Jukkasjärvi
 \item Turning Torso
 \item Öresundsbron
 \item Globen \textsc{(s.~\pageref{c520b11670b9cef944588fe3849ce491})}
 \end{itemize}

 HEAD2: Underverk utvalda av Vetenskapsradion
 I P1s Vetenskapsradion Historia:
 \begin{itemize}
 \item Göta Kanal (ej att beblanda med filmen \quotetext{Göta Kanal} med bla. Janne \quotetext{Loffe} Carlsson)
 \item Ales Stenar \textsc{(s.~\pageref{2b28507979e217cfe15c9d6455eabd18})}
 \item Malmö Moské
 \item Lunds Domkyrka
 \item Karlskrona (?)
 \item Visby Ringmur
 \item Hällristningsområdet i Tanumshede
 \end{itemize}

 HEAD2: Glömda underverk
 Den kritiske historiekännaren kan påpeka att följande imponerande bedrifter skamlöst glömdes bort vid omröstningen:
 \begin{itemize}
 \item Pudaslådan \textsc{(se Pudaslåda s.~\pageref{6a56958e2057dd500650e2be8049e033})}
 \item Kaknästornet \textsc{(s.~\pageref{ffe3ac06a304714dcee7cbbdfeb20d84})}
 \item Ornässtugans dass \textsc{(s.~\pageref{17d2effff6c1590dbff6a7ac39f46a19})}
 \item Polhemshjulet
 \item Kebabnekaise
 \item Göta Kanal (filmen med bla. Janne \quotetext{Loffe} Carlsson)
 \end{itemize}

}

\small{
\textbf{Svinpäls}
\label{e071e2af0043b8da98f8f2d4dff28852}
 Negativt laddad synonym för extra gemena \textsc{(se jävelskap s.~\pageref{46845591177f16920cd586a5baf5a625})} storfräsare \textsc{(s.~\pageref{4db17005692cd83e3e946a1311b81ed0})}. Tänk dig en ulv i fårakläder, men som i brist på äkta fårskinn istället tagit en gammal dammig grishud och limmat på lite svinto med billigt epoxylim. Det är skitigt, det är oanständigt, det är riktigt jävla lågt. Bara en äkta svinpäls skulle göra något sådant.

 HEAD2: Kända svinpälsar i populärkulturen
 \begin{itemize}
 \item Karaktären Svinpäls i Disneys \textit{Räddningspatrullen}
 \item Saruman i J.R.R Tolkiens \textsc{(se J.R.R Tolkien s.~\pageref{3f0b7fcbd9fa7369ca314a46c280b67e})} böcker
 \item Gandalf i Åke Ohlmarks \textsc{(s.~\pageref{dc4829f902543aa5b8349fa82bafacb7})} \textit{Tolkien och den svarta magin \textsc{(s.~\pageref{b2cc087ddfcc973d83fe146ca31fe88e})}}
 \item Björnen i sången \textit{Mors lilla Olle}
 \item Tjorven i \textit{Vi på Saltkråkan}
 \end{itemize}


 HEAD2: Kända svinpälsar i finkulturen
 \begin{itemize}
 \item Hela familjen Bonnier
 \end{itemize}


 HEAD2: Kända svinpälsar i fulkulturen
 \begin{itemize}
 \item Charlie Sheen
 \item Johnny Takter \textsc{(s.~\pageref{7cceecf64f19c092b60fd77b28ee337d})}
 \item 50 million piece of shit \textsc{(s.~\pageref{20dea3d3f625b865ae7fd554c02c6936})}
 \end{itemize}

}

\small{
\textbf{Sviskonpaj}
\label{ce8e792cda4c878050cf537f151667ff}
 Björnligans favoritpaj.

}

\small{
\textbf{Svotto}
\label{a54b74d16960ccfdc5c60c57fb0fe954}
 Fornnordiskt namn som var vanligt åtminstone järnåldern ut. Namnet har under modern tid återigen blivit populärt, främst genom den episka diktsviten \quotetext{Badgirl eller par}, skriven av den länge anonyme författaren, politikern, och välrenommerade akademikern Svotto Littorin.


 Här: erfaren välutrustad
 bestämd social och seriös
 Man i Sthlm.

 Där: söt nyfiken tjej
 eller par som behöver
 och vågar.

 Är du nyfiken på bdsm,
 ageplay, hårda tag
 mm så hör av dig.

 (Utdrag ur \quotetext{Badgirl eller par})


 Den ursprungliga betydelsen av namnet tros ha varit \quotetext{farlig plats} eller kanske \quotetext{han som känner bra könsumgänge}.

}

\small{
\textbf{Svälta räv}
\label{f5ba1e0ca45e2d553c6282cb290878dd}
 är världens i särklass tråkigaste kortspel. Till och med att förvara leken i asken och gissa vilket kort som ligger överst är roligare. Svälta räv går ut på att \quotetext{svälta} motståndaren genom att komma över alla dennes kort. I början av spelet delas leken upp jämt mellan spelarna som alla förvarar sin del i en hög på bordet med ryggen \textsc{(se rygg s.~\pageref{7c74337e8958a60864119ecbd907e85d})} uppåt. Alla spelare drar det översta kortet från sin hög. Om korten har samma färg vinner spelaren med den högsta siffran korten. Om färgerna är olika drar man nya kort tills de överensstämmer och så där håller det på. Det enda positiva med svälta räv är att det faktiskt inte tar så lång tid att avsluta ett parti som man minns från sin barndom, och att det är ett bra sätt att lösa konflikter på.


 91:an spelar mycket svälta räv vilket speglar seriens kvalité väl.

}

\small{
\textbf{Svåra saker}
\label{8e549368c30d854e38e09fa0c6b1fa50}
 blir lätt fel.

}

\small{
\textbf{Swans}
\label{13681d689ba4e411e07bd10cb53c913f}
 är ett post-punk/industriband från USA som startades 1982. Det leddes då och leds nu av frontfiguren Michael Gira, tveklöst en av världens mest obehagliga musiker. Han kan få det mest luttrade Burzum-fanet att göra ifrån sig i byxan \textsc{(se wctbyxa s.~\pageref{dc78f9615e53d0ddb525d3975197a781})}. Kan man bör man därför undvika att lyssna på något han har lagt sin hand vid, vilket även gäller alla Swans-skivor och i allra högsta grad [http://www.youtube.com/watch?v=97t4QXPvKGo\textbar de tidiga].

}

\small{
\textbf{Syfilis}
\label{6ef63d9f4b2e7a8686a4900dbb206a54}
 Upptäcktes till skillnad från Amerika inte av Columbus utan av hans matroser. Vanligtvis har sjukdomen fått sitt namn efter staden, orten eller landet varifrån folk kom hem med \quotetext{morbus gallicus}. En uppsjö av namn således: franska sjukan eller franzosen, dock ej att förväxla med en parisare \textsc{(s.~\pageref{5aca28013b9a7e4088e7fb228f3e4827})}. Även Neapel, Kina och Tyskland \textsc{(s.~\pageref{b1b58da783b6d5fa090f3015f1889869})} var länge populärt. Till Sverige och Norden kom syfilis med det saxiska gardet under Junker Schlentz. Roskilde begåvades ett nådens år 1497 med den \quotetext{frandzoske siuge oc kranchedt}. 1508 kom sjukdomen till Finland \textsc{(s.~\pageref{631d44eaa1254ff71a1e11ba021d1266})}, hädanefter alltsomoftast kallad \quotetext{syffe}. Den olycklige libertinen, riddaren Åke Jöransson, såg på sin dödsbädd ljuset, grundade franciskanerorden.

}

\small{
\textbf{Sylt}
\label{be6fa0cf731bdae2e13ad52084c90fbc}
 är en ö i Nordsjön som tillhör förbundslandet Schleswig-Holstein inom Förbundsrepubliken Tyskland \textsc{(s.~\pageref{b1b58da783b6d5fa090f3015f1889869})}. Geografiskt ligger ön på samma latitud som södra Alaska.

 Den största kommunen på Sylt heter Sylt-Ost. Kommunen består av flera mindre byar och har sammanlagt ungefär 5.500 invånare. På hela Sylt bor det ca. 21.000 människor, dubbelt så många som i hela Härjedalen. Det är mycket.

 [
 http://www.sylt.de]

}

\small{
\textbf{Syndikalism}
\label{e42329fd29f60508785768e3e45d95f8}
 bygger på idén, hopkokad av hålögda, frihetshatande gangstrar, att man ska skänka värdet av det man skapar varken till storkapitalet eller partiet. Istället har man fräckheten att tänka sig att man arbetar för varandra, utan chef, men med en tydlig demokratisk och kooperativ struktur där besluten tas gemensamt, arbetet görs gemensamt och det värde man skapat delas rättvist. Denna fascistoida idé har som tur är gjorts till gemensam måltavla för Svenskt näringsliv, Socialdemokraterna, Alliansen, Timbro, större delen av den svenska journalistkåren, diverse motorcykelgäng och nynazister och andra frihet- \textsc{(se frihet s.~\pageref{0ee37cea60c9b45e40dbc83c0c665085})} och rättviseivrande aktörer. I morgondagens fräscha och avreglerade samhälle kan vi därför förhoppningsvis framleva våra liv utan denna styggelse till ideologi.
 HEAD2: Brott mot mänskligheten
 Umeå LS av SAC tryckte vid ett tillfälle T-shirtar med texten, \quotetext{vi tar fajten} som ryggtryck \textsc{(s.~\pageref{a5caecaf8bd113e5ecd7c924db801c44})}.

}

\small{
\textbf{Synonymer för anus}
\label{67a558303975f05c5e4cf1cec6c641c9}
 \begin{itemize}
 \item Knuten
 \item Pruppen
 \item Skitkiken
 \item Ana
 \item Dajmkrysset
 \item Pussmunnen
 \item Grötbössan
 \item Rossen
 \item Dajman
 \item Fisgluggen
 \item Brunan
 \item Hästögat
 \item Tvåan
 \end{itemize}

}

\small{
\textbf{Syo}
\label{e6ece7e1836dfe745a2b015fb2da8fc0}
 står för Studie och Yrkesvägledning och syftar till att leda ungdomar rakt ner i fördärvet. Tjänsten innehas ofta av en människofientlig och kraftigt närsynt tant. Det kan rentav vara så att detta är en en del av kravspecifikationen \textsc{(se kravspecifikation s.~\pageref{b6873dbaad6d1ae16eb34efac4218c11})} för att få förleda.

}

\small{
\textbf{System Requirements}
\label{c3c0e3736c092f9f194505694468ca8b}
 Tjugo år gammal, pengar och inte alltför full.

 Dock finns enstaka undantag. Det finns dokumenterade fall av att det räckt med fjorton i klockfrekvens, lång för sin ålder, långt stripigt hår och militärjacka för att köra programmet \quotetext{Köpa en hela Grants \textsc{(s.~\pageref{74dc2f36b83c605847a3519729a18d11})}}. Detta firades med ett skrovmål på Frasses \textsc{(s.~\pageref{971e198d8fef127906319ec98ff657ce})}, som sedan var verksam ingrediens i en framgångsrik behandling för emetofobi \textsc{(s.~\pageref{afcfb287e0c9a3f9fa7a3e6e748afdcf})}.

}

\small{
\textbf{Säcklöpning}
\label{b455d309a790e837f6b436258869ff50}
 är en idrottsgren som går ut på att förflytta sig framåt i en jutesäck \textsc{(se särk s.~\pageref{7a522dc7e11bd1136642b3452855c1d6})} på kortast möjliga tid. Sporten uppfanns av den senile indianen Trötta Sommarkatten som råkade ta på sig sin poncho upp och ned. Säcklöpningstävlingar är mycket publikvänliga tack vare att deltagarna ofta snubblar på sig själva; \textsc{(se semikolon s.~\pageref{a6e5810f9ad5798914f30165eba44dcb})} försök komma på något som är roligare att titta på liksom.

 Grovt räknat finns två olika taktiker att välja mellan. Antingen kan man hoppa jämfota eller så tar man snabba myrsteg inne i säcken. Det första är jobbigare och det andra är vingligare. Deltagaren bör därför börja med att fråga sig vad som känns värst: att bli svettig eller dratta på arslet. Båda alternativen är ju ganska tråkiga så kanske gör du istället bäst i att sitta kvar på gräsmattan och ta en bärs till.

}

\small{
\textbf{Sälar}
\label{ff14e847bd014e34af62f5c5855a1bdc}
 (Pinnipedia)  finns i tre familjer; öronsälar, öronlösa sälar och valrossar. Valrossen är den häftigaste av dessa med sina balla sabeltänder. Sälen spenderar större delen av sin vakna tid med att ligga och jäsa eller ta det lugnt. Den lever främst på fisk men vissa arter har även utvecklat en febläss för krill.



 I modern tid är sälen kanske mest känd för att ha gett upphov till det smygborgerliga Miljöpartiet \textsc{(s.~\pageref{3e11b29518eeea19128b64869699f363})} som utnyttjade sälens efterblivna utseende till att fiska röster. Sälen är dock av naturen på intet sätt knuten till besvikna proggare. Punkbandet Atomångest \textsc{(s.~\pageref{16df915e34a40562db7fab59c10ff5d9})} tog exempelvis ställning i sälfrågan i sin låt \textit{Blod} med textraden: \quotetext{Vill du döda en säl? Nej, tack!}.


 Att kalla någon för en säl betyder att man anser personen i fråga ha en aningen trind kroppshydda \textsc{(s.~\pageref{032eed30d2aad3425b9139aafdd6740f})}.

}

\small{
\textbf{Sällskapsresan}
\label{1023ca20cc8ad5b3f0233d023ad01bf5}
 \textit{Sällskapsresan, eller Finns det svenskt kaffe på grisfesten?} (1980) är den mest framgångsrika filmen i riket Sveriges \textsc{(se Sverige s.~\pageref{b1999637949ed135b2ca03f3a38460cc})} historia om man räknar till antalet gånger den spelats ombord på bussar som går till skidorter, äventyrsbad \textsc{(s.~\pageref{8e36481b72c8061bb9ff74c1df3b0b66})} och scoutläger. \textit{Sällskapsresan} handlar om en lång, smal, så kallad tönt (spelad av Musse Pigg-fetishisten Lasse Åberg) och en norrman (spelad av småskurken \textsc{(se småskurk s.~\pageref{c25031c5d78d9ad6fae8ab8f08d5e9dd})} Jon \textsc{(s.~\pageref{006cb570acdab0e0bfc8e3dcb7bb4edf})} Skolmen) som blir polare under en resa till medelhavet. Väl framme i semesterorten råkar de ut för en massa motgångar, men får till sist ligga med två lapplisor \textsc{(s.~\pageref{0d3e4c1085e1d029818497b4c7e624f9})}. Filmen gick direkt till den svenska publikens hjärtan, delvis på grund av det inte finns så mycket annat att se på, förutom Colin Nutleys \textsc{(se Colin Nutley s.~\pageref{b7e4eb146052f2edb273b55e35f4f078})} \quotetext{filmer}.

}

\small{
\textbf{Sämskskinn}
\label{ac4304bcc0fc9e737026f871b462fbe2}
 är en speciell form av mjukbehandlat skinn med väldigt hög uppsugningsförmåga och användbart till väldigt mycket. Till exempel att putsa dragbasuner \textsc{(se dragbasun s.~\pageref{0315aaaabb57a67312aa3316fd2006e1})} och bilar \textsc{(se bil s.~\pageref{b3188f47d2eac7efc3f1258dc673a9fe})}. Det kan också användas till att filtrera bort vatten i bensin om det skulle behövas. Skinnet framställs ifrån djurhudar genom fettgarvning och kommer idag oftast ifrån get, men ibland även ifrån ko eller älg. Nuförtiden tillverkas det, enligt Flashback, även på syntetisk väg. Huden skrapas ren ifrån hinnor på köttsidan och hår på narvsidan. Sedan smörjer man in hudens köttsida i fett, ofta använder man djurets hjärna. Hjärnan måste dock kokas tills den blir vit och sedan svalna om denna skall användas. Karl XIIs drabanter hade sämskat älgskinn under sina bröstharnesk. Det bästa är finskt sämskskinn \textsc{(s.~\pageref{ecdf6b5129df6ebb83a9b381b4b33553})}.

}

\small{
\textbf{Särk}
\label{7a522dc7e11bd1136642b3452855c1d6}
 var en av två sorters klädesplagg som existerade på medeltiden \textsc{(s.~\pageref{88cbc30c5b233d97df68b5b041ac0655})} (det andra var byxor men var inte lika vanligt). Särken är vanligtvis brun eller grå och ser mycket eländig ut. Den slutar en bit ovanför knäna och var man så lyckligt lottad att man hade att par byxor kunde man ha den instoppad så såg man istället mer fånig än bedrövlig ut. I princip vad som helst kan bli en särk men vanliga material är linnetyg och potatissäckar. Särken var ett sätt för staten att se till att medborgare föraktade sig själva och andra och stöptes i samma form, ungefär som skolan fungerar i Jan Björklunds \textsc{(se Jan Björklund s.~\pageref{0b9b757044804b9be0e218acdad358cc})} Sverige.


 På senare tid har särken fått en renässans och ingår som klädkod hos bland annat Lidl, Tåg i Bergslagen, Ryan air, Ica och Överskottsbolaget \textsc{(se te s.~\pageref{569ef72642be0fadd711d6a468d68ee1})}.


 Turistbyrån i Norberg har som tradition att påtvinga den minst gillade arbetskamraten detta plagg.

}

\small{
\textbf{Särske}
\label{552a5aad891937bf760fb193900ea140}
 är ett adjektivt \textsc{(se adjektiv s.~\pageref{67d02147cd8595eaf13c1a90aba99dcc})} som betecknar att någonting är undermåligt, dumt eller konstigt på ett dåligt sätt.

 Ordet tros ha sitt ursprung vid Per-Ols skolan i Fagersta \textsc{(s.~\pageref{008e08fd02751800f729d6fa6f75a857})} där stadens särskola låg i samma lokaler som ett helt vanligt lågstadium. Eleverna ska där ha studerat särskoleelevernas bristande ät-teknik, och använt ordet \quotetext{särskola} som ett nedsättande ord.

 Moralen i detta ordbruk kan tyckas tveksam, men om sanningen ska fram var särskoleeleverna vid Per-Ols inte speciellt skickliga på att stoppa mat i munnen.

 Ordet har sedan förkortats till just \quotetext{särske}

 HEAD2: Exempel

 \quotetext{Vad fan gör du, är du särske eller?}

 \quotetext{Jag är så trött på den här jävla särske-cykeln \textsc{(se christianiacykel s.~\pageref{671a1992db86e328dc9c068647d57d6b})}.}

}

\small{
\textbf{Sågverk}
\label{39a99a78876fd85985cc06fa0baa3c1a}
 Har du någonsin hållit i en två tum fyra och funderat \quotetext{Hur är det möjligt?}. Då kan du mycket snart sluta undra. Här följer nämligen en redogörelse för hur ett modernt sågverk fungerar. Empirin bygger på 6 års erfarenheter av sågverket i Malå \textsc{(s.~\pageref{41da4620e87888eaaeafcb3004a8d177})}.

 HEAD2: Från skog till sågverk
 En tall tornar upp sig majestätiskt över en myrhed. Den svajar lätt i vinden och omgivningen är väldigt fridfull. Men vad nu? Ljudet av motorbuller skär genom landskapet och vid horisonten uppenbarar sig en skördare. Denna skogsmaskin har bara en uppgift: att göra om den frodiga skogen till karga kalhyggen. Tallen kapas just ovanför roten och avkvistas på en gång, mycket fiffigt. Tallen, som nu blivit en timmerstock, läggs åt sidan och skördaren fortsätter sitt korståg genom skogen. Sen kommer skotaren, vars uppgift är att plocka upp stockarna och köra dem till närmsta väg. Där väntar en lastbil som kör stockarna i ilfart till närmsta (i idealfallet, på grund av kapitalismens irrationella verkningar blir det dock sällan så) sågverk.

 HEAD2: Mätning och avbarkning
 Tallen ligger nu på ett lastbilsflak tillsammans med sina artfränder. Lastbilen parkerar på sågverket efter vägning och en stor traktor med en ännu större grip lyfter av ett knippe stockar och placerar dessa på mätstationens timmerbord. Här skakas stockarna en och en upp på ett rullband och mäts med hjälp av laser eller nåt annat högteknologiskt och sorteras sedan i olika fack. Samma traktor lyfter sedan upp stockarna och placerar dessa på ännu ett timmerbord, det som leder in i barkmaskinen. Samma princip som i mätstationen, men istället för att mätas åker de genom en så kallad barkmaskin som, ni gissade det, skalar av barken med hjälp av en osthyvelsliknande \textsc{(se osthyvel s.~\pageref{c09f8306965e1344e1102a46d084cab9})} grej. Stocken fortsätter sedan sin resa mot själva sågen.

 HEAD2: Sågning och paketering
 Stocken åker med en våldsam fart genom en maskin uttrustad med två lodräta klingor, sen en maskin med två horisontella. Vips är stocken inte längre karaktäristiskt rund, utan fyrkantig. Sedan klyvs stocken av en tredje maskin till brädor eller plankor - skillnaden här emellan är så komplex att den kommer avhandlas för sig senare. Plankan och brädan sorteras sedan var för sig och åker genom ytterligare en station där de mätas och sorteras i olika fack för att bli till virkespaket när ett tillräckligt antal plankor eller brädor uppnåtts. Virkespaketen blir till den så kallade läggaren som ser till att virket ligger i snygga rader med torkströn emellan, men torkströna sköts av en annan maskin som har en tendens att jävlas väldigt mycket, till allas förtret. När paketet staplats till önskvärd höjd åker den ut ur sågen och får en lapp häftad på sig där det står vilket mått som virket har, samt ett serienummer för att signalera precis hur unikt varje paket är.

 HEAD2: Torkning
 Trä, i egenskap av att vara ett levande material, innehåller vatten. Det här vattnet ska inte vara kvar i den slutliga produkten för då blir den inte särskilt rejäl, som de flesta som köper trä vill att det ska va. Därför torkas virket i de fantasilöst döpta torkarna. Hade jag bestämt hade det hetat virkesbastu. I alla fall, virket bärs till torken av en truck. Det finns två sorters torkar, vanliga och kanaltorkar. Vanliga torkar ser ut som jättelika garderober där paketet ställs in för att sedan tas ut från samma håll. Kanaltorkarna har portar i två ändar och virkespaketen åker på en räls genom torken. Det spekuleras i att virkespaketen tycker att det här är jätteroligt, men detta har inte kunnat styrkas empiriskt. I torkarna blir det väldigt varmt, säkert 100 grader, och allt vatten i trät försvinner. Vill man att det ska vara lite vatten kvar går det säkert också att ordna. Man tar inte reda på vattnet utan låter det försvinna ut i atmosfären så att det blir regn \textsc{(s.~\pageref{03456beeae643b4c33b17500a17d1d1e})}.

 HEAD2: Justering och/eller hyvling
 Virket har när det kommer ut från sågen samma mått på höjden och bredden, men inte på längden. En del som köper virke vill att det ska vara samma längd på alla brädor eller plankor och det här sköts på justerverket. Processen liknar den på sågen, efter det att stocken kluvits. En truck kör virkespaketen hit och virket blir sedan mätt, sorterat, kvalitetsbedömt, kapat och paketerat så att kundens alla önskemål uppfylls och tillråga på allt får paketen en lapp med specifikationer och serienummer.
 Om kunden inte önskar få en massa stickor av sitt virke måste det dock hyvlas. Det här sköts på det återigen fantasilöst döpta hyvleriet. Här skrapas det yttre lagret trä (där stickorna sitter) bort med hjälp av en jättelik elektrisk rubank \textsc{(s.~\pageref{b1c373a9ae319af9e1bf15a62fdf85cf})}. Virket blir slätt och fint och lämpligt att använda till en uppsjö saker, t.ex. sommarstugor, dansbanor och Trojanska hästar. Ska träet vara utomhus så måste nästa steg genomföras.

 HEAD2: Impregnering
 Virket doppas i en bassäng fylld med en massa mer och mindre farliga kemikalier så att inget vatten kan tränga in mellan träets fibrer. Sen åker det ut och är redo att möta alla moder jords påfrestningar, utom eldsvådor. Det är något av träets akilleshäl.

 HEAD2: Den eviga debatten: planka eller bräda?
 Vår \textsc{(se Nissepedia s.~\pageref{62400dadecd90cb5cd39062abe5a3e4a})} konkurrent har besvarat denna fråga: en bräda blir en planka när den överstiger måttet 38 X 100 mm. Den minsta brädan som skickas ut från Malå Sågverk är 19 X 100. Den största plankan är 63 X 200. När det sågas 19 X 100 så är tempot på sågverket väldigt lågt och alla är glada och hälsar glatt. När det sågas 63 X 200 är det vansinnigt hektiskt och dålig stämning är kutym. Så nästa gång ni ska använda detta monster till planka tycker jag att ni ska tänka på hur mycket slit som ligger bakom.

 HEAD2: Sågverkets fauna
 Det dominerande inslaget i faunan är knäppskorven, eller snytbaggen som Carl von Linné \textsc{(s.~\pageref{5e8380bf6b7ce99678e6752b6d9e709e})} döpte den till. Den älskar trä och på sågverk finns denna vara som bekant i överflöd. De flesta knäppskorvar kommer till den sista vilan genom att åka med virkespaket i torkarna. Här samlas döda knäppskorvar i stora högar och sopas på somrarna ut av ortens ungdomar. Timmermannen är en annan insekt som trivs på sågverk. Dennas främsta egenskap är dess långa horn, varav hanen har längst. Ibland landar hanarna i intet ont anandes sommarjobbares skägg och panik \textsc{(s.~\pageref{bca410441b88e24768f3f385548edfbe})} utbryter, men de gör inte så mycket mer än att bara sitta och idla. Honorna däremot bits ganska hårt men gör det bara i självförsvar. Vidare kan i alla fall Malå Sågverk stoltsera med ett visst bestånd av ren. Dessa söker sig till området i jakt efter skugga och ligger ibland och softar i spånhögar. Tack vare trucktrafiken lever dessa renar farligt och blir därför ibland avhysta, men det är för deras eget bästa. Det har påträffats benrester av ren på området, så det är ingen säker plats för våra beklövade vänner. Inne i sågen brukar svalor bygga bo under semesteruppehållet, men när maskinerna startar upp igen antar jag att de drar någonstans där bullernivån är betydligt lägre. Det finns också ett rikt bestånd av människor. Dessa spenderar dagarna med olika former av hårt arbete och livnär sig på kaffe \textsc{(s.~\pageref{a51a0cac0ce374a853d2359417debc28})} och matlådor. Dessa för också in ytterligare en art i biotopen, nämligen hunden. Dessa sitter ofta i arbetarnas bilar och gnyr, men ibland sitter de fast med koppel i bilens kula och skäller på förbipasserande truckar. Traktorförarna som har små hundar brukar skjutsa runt dem i traktorerna.

 HEAD2: Klasskamp på sågverk
 Den överlägset vanligaste formen av klasskamp som bedrivs är att \quotetext{Livsfarlig Ledning}-skyltar sätts upp på chefers dörrar. En anekdot berättar om hur en arbetare en gång gått in på kontoret på Malå Sågverk och sagt \quotetext{Om vi int få en krona till i timmen, då stann vi av!}. Chefen hade inget annat val än att höja lönen, och den kronan, den finns kvar än idag! Även en bild på Setras (som äger sågverket) VD Bengt Börjesson har hittas på en anslagstavla på Malå Sågverks justerverks anslagstavla, med ett häftstift rätt i pannan! Obehagligt.

}

\small{
\textbf{Sårrengöringsvätska}
\label{9871d881c6a63c113f66b202767334ca}
 Vätska för rengöring av öppna sår, ögon, näs- och munhåla. Kiss är den renaste som finns. Kejsaren Vespasianus blev till exempel hårt åtgången av sina samtida då han lyckades med att både strypa katten å ha den kvar. Ergo - han lät ta skatt på urinoarerna samtidigt som urinet samlades in för de romerska legionernas behov av sårtvätt. Han lär då ha sagt \quotetext{Non olet}, det luktar inte. I historieböcker beskrivs det ofta som om att det är pengarna som inte luktar, hans skamliga profit. Vad vi nu vet är att han syftade på kisset - det är det renaste som finns.

}

\small{
\textbf{Söndag}
\label{85b2e5c3758394a24221d1abac79191a}
 en är den sista, och enligt många den keffaste \textsc{(se keff s.~\pageref{890a42bbf6c2e6888fb851dd76e1e980})}, dagen i veckan. Andra, däribland Micke Alonzo \textsc{(s.~\pageref{4bd27d3cf2641236f956496c779a0dc2})}, menar att måndagen \textsc{(se måndag s.~\pageref{1086ddd192a30419d01e5c28b74cab2f})} är sämst. Söndagen är i kristendomen vilodagen och kallas i Bibeln \textsc{(s.~\pageref{7de7d2a7d608c9a2044f50688bc63e27})} den sjunde dagen. På söndagen är nästan inget öppet, ingen vill göra något och det finns inget att göra.
 HEAD2: Regionala seder
 I Västerbotten \textsc{(s.~\pageref{d4b008c5143dcffb6b8c35f3876c2a19})} är det vanligt att man ber och skäms på söndagen.

}

\small{
\textbf{T-rexarmar}
\label{0b2dbf0eb2888d887370538902e974d4}
 har man om man har svårt att nå och därför till exempel tvingas använda krattan när man spelar biljard eller måste stå på en stol för att kunna ta ner glas eller tallrikar från skåpet ovanför diskbänken.

}

\small{
\textbf{Ta för sig}
\label{da38d3921d90c6551623165ebb693bb3}
 Att ta för sig kan antingen vara något väldigt fult eller något väldigt fint, beroende på perspektiv \textsc{(s.~\pageref{1606dd19366985367d677f7b6de46e52})}. Att ta för sig, oberoende av hur man ser på det, går ut på att man inte väntar på sin tur eller att man inte är beredd att dela på kakan med andra, utan att man helt sonika kliver fram och tar det man vill ha. Är man tokliberal \textsc{(s.~\pageref{531cb70b602e3f3c32d40bac64400830})} är det väldigt fint att ta för sig: så fint, faktiskt, att tokliberalerna ofta får något sentimentalt över sig och de genomgår en känslostorm som inte sällan gör att en tår sakta rullar nedför deras rödlätta kinder. Att ta för sig är nämligen det som nyliberalism går ut på. Nyliberaler ser det som ett stort problem att de inte får ta för sig av allt omkring dem och tycker att detta är det värsta av det offentliga samhällets många övergrepp på individen \textsc{(se individ s.~\pageref{41beed76a0af9b4f550f7ebdecd3e700})}. Man har till och med utvecklat en egen, mycket framgångsrik, form av feminism som går ut på att kvinnor ska lära sig att ta för sig. Om det i något sammanhang uttrycks missnöje med att någon tagit för sig eller att människor uppmanats att ta för sig brukar tokliberalerna himla med ögonen om tala om jantelagen och att det är \quotetext{typsikt svenskt} att inte ta för sig. Med \quotetext{typsikt svenskt} menar de att det är ett arv från arbetarrörelsens formativa årtionden att det bland gemene man anses ofint att stjäla andras eller kollektiv egendom för eget bruk och nyttjande.

}

\small{
\textbf{Taggen}
\label{3a8f627d2405431a21b8c08b01408423}
 är en fiktiv karaktär i den svenska dramaserien Tre Kronor. Han har utländskt påbrå och vänstersympatier. Han står med ena foten utanför samhällets normer och regler men följer ett slags inre moral, så han är ganska snäll trots allt. Han har en syrra också, men jag kommer inte ihåg vad hon heter.

}

\small{
\textbf{Taikonaut}
\label{5bcdf744568ee326dba5514b27f8f7c6}
 Rymdfarare som inte är längre än 1.55m, vanligtvis av asiatiskt ursprung. I Taiwan och Hong Kong används ofta beteckningen \quotetext{tàikōng rén} (太空人, \quotetext{rymdman}). Det finns ingen särskild taikonautlegitimation, så det står en och envar fritt att titulera sig detta, så länge man uppfyller ovanstående krav.

}

\small{
\textbf{Talgoxe}
\label{4e9d46e4dca35138132c2977b1fcab12}
 n (\textit{Parus major}) tillhör liksom kopparormen \textsc{(se kopparorm s.~\pageref{b8f4fa38453856ba979bc2898e116e5a})} och uven \textsc{(se uv s.~\pageref{45210da832f9626829457a65e9e7c4d0})} de djur som heter en sak men i själva verket är något helt annat. Den är nämligen  paradoxalt nog en fågel. Den är väldigt vanlig i Sverige, där den bor och lever av att äta frön och talgbollar. Den är gul, vit lite här och var och svart längst upp. Talgoxen tycker om att flyga, sitta i träd och buskage och att äta talgbollar. Enligt ett visst konkurrerande internetlexikon låter fågeln \quotetext{tjitt, tjitt,} \quotetext{ping, ping \textsc{(s.~\pageref{df911f0151f9ef021d410b4be5060972})}} eller, helt enkelt, \quotetext{pitt, spick}.

}

\small{
\textbf{Tantkläder}
\label{876b87b4d70dd7225b5b665038841361}
 Krymplèn
 Allvädersstövlar
 Stora halsband

}

\small{
\textbf{Tantnöjd}
\label{e62d653970e3f2b0d9482683e8b38142}
 Att vara tantnöjd är ett sätt att stävja ångesten inför åldrande och ett ljummet parförhållande genom att finna glädje i de små tingen runt oss. Att ta en promenad en vacker söndag i mars eller få besök av ett litet barn \textsc{(s.~\pageref{5dfcc0aab2f3db925b2d51ba73e48946})} som säger så många klokheter som vi vuxna borde ha vett att ta till oss. Kanske finner vi glädje i ett klokt ord med tillhörande naturbild som postats på facebook \textsc{(s.~\pageref{26cae7718c32180a7a0f8e19d6d40a59})} eller i en TV-dokumentär om någon som gått igenom en svår sjukdomsperiod utan att för den delen tappa livsgnistan. Det fina med tantnöjdheten är att den i sig genererar mer material för tantnöjdhet i form av tänkvärda ord som sprids medelst broderier eller hemgjorda vykort, facebook-uppdateringar och kylskåpsmagneter. En dag kommer kanske tantnöjdheten att sprida sig över vår blåa planet och göra krig och elände till ett avlägset minne. Man kan bara hoppas. Och tro.

}

\small{
\textbf{Tantsång}
\label{25b8200d011a4766f4b3a64a2e17f374}
 är en sub-genre inom musiken som utmärker sig på så vis att den troligtvis är den enda genren som har gått från att vara mainstream till underground. Idag är fans av genren tvugna att leta med ljus och lykta efter skivor med tantsång medan man nästan bara kunde köpa skivor med tantsång före det att Sverige blev sekuläriserat. När piratradiostationerna Radio Syd och Nord tvingade fram etablerandet av radiokanalen P3 var slaget om skivlistorna förlorat för tantsångsgenren. Idag vet inga tantsångsfans hur man spelar CD-skivor och mp3-filer, vilket omöjliggör tantsångens återintåg på de kommersiella arenorna.
 HEAD2: Tantsång idag
 Idag hålls tantsångspelningar på mindre spelställen som ofta är anslutna till kyrkor, missionshus och äldreboenden. I likhet med den centripetala rörelsen inom Norges black metal-scen i början av nittiotalet, då sub-genren och -kulturen vände sig inåt och medvetet gjorde sig otillgänglig för den breda massan, har tantsången under senare år sett till att det är hart när omöjligt för den genomsnittlige medborgaren att veta var och när tantsångsspelningar äger rum. Detta ska dock inte ses som att tantsången är på utdöende - tvärt om, tantsången är mer vital än någonsin, om man tillåts uttrycka det så. I och med att 40-talisterna går i pension och börjar komma till åren förväntas tantsången explodera i aktivitet, om ej försäljningsmässigt. En annan stor förändring som kan skönjas är att influenser idag tas inte bara från den standardiserade psalmboken och ett och annat skillingtryck utan även från sådana genrer som arbetarrörelsens musikarv, dragspelande smilfinkar från \textit{Allsång på Skansen} och ett och annat örhänge från svensktoppens tidiga dagar.

}

\small{
\textbf{Tax}
\label{06565e5611f23fdf8cc43e5077b92b54}
 är typ som en hund fast väldigt lång och med extremt korta ben.

}

\small{
\textbf{Taxichaufför}
\label{cfb4f19433bacb1ec7b89a00ac04a68c}
 En taxichaufför är oftast en person som, utan möjligheten att få betalt för att skjutsa runt folk som har en redig bärsfylla \textsc{(s.~\pageref{9380b60f9ee744b9acf978fe6f1a9545})}, skulle gå under i dagens samhälle. Taxichauffören är mer ofta än inte en blekfet antisocial 20-nånting som gillar nihilistisk black metal, en gubbjävel som misslyckats med allt han företagit sig och nu hatar världen, eller en aggressiv stigmatiserad bärshagga som tycker om att rulla runt i en bil mellan (alt. under) fyllorna. Att dessa förtappade själar skulle krossas av det cyniska marknadssamhället om det inte vore för det sociala skyddsnät som taxiyrket utgör, förhindrar dem inte från att vara tokliberaler \textsc{(se tokliberal s.~\pageref{531cb70b602e3f3c32d40bac64400830})} och hylla övermänniskoidealet.

 Sen finns även de taxichaufförer som utgör den altruistiska motvikten. Tanter, gossar, tjejer och gubbar som gärna säger att de är en \quotetext{glad skit} och lyssnar på ZZ Top på helgvolym \textsc{(s.~\pageref{3539fdeb41a5b216f614b6ced9ff5cff})} i bilen. Men de är få, allt för få.

}

\small{
\textbf{Te}
\label{569ef72642be0fadd711d6a468d68ee1}
 Man blir skitpigg av att dricka te, och så är det med de(t).
 HEAD2: Tillvägagångssätt för att brygga te
 För att brygga en kopp te gör du följande. Koka upp vatten på valfritt vis (kastrull, tekokare, blåslampa, kittel över eld etc). När vattnet kokat upp lägger du en tepåse i en kopp med en volym mellan en och tre deciliter. Kontrollera noggrant att den lilla lappen som är förtöjd i påsen med ett litet snöre dinglar på utsidan av kruset \textsc{(se krus s.~\pageref{2a95ddf371e46d685f45c0f173f8b7e2})}. Nu häller du sakta på önskad mängd vatten i koppen. På tepaketet kan det stå att tepåsen ska ligga i vattnet i ca 3 minuter. Detta är lögn. Istället tar du genast tag i den redan nämnda lappen och snabbdoppar \textsc{(se snabbdoppa s.~\pageref{017d4cf4a8dc7d8d4801b949df3e3f6e})} påsen i vattnet några gånger, sedan slänger du den över gärdesgården. När temperaturen har sjunkit och är som du önskar den kan du dricka ditt té.

 HEAD2: Te och samhället
 Arbetarklassen föredrar kaffe \textsc{(s.~\pageref{a51a0cac0ce374a853d2359417debc28})}, medelklassen dricker te med förtjusning och gärna specialare som \quotetext{lapsang} eller \quotetext{darjeeling} och överklassen är som vanligt helt verklighetsfrånvänd.

 HEAD2: Te i världen
 I Storbrittanien dricks det mycket te (det är, förutom simhallsdoftande kranvatten, den enda icke-alkoholhaltiga dryck som förtärs), och det här anses av de flesta etnologer bero på vita Anglosaxernas \textsc{(se Anglosax s.~\pageref{75591674b0deca83291ccfef6f4f557c})} kollektiv-psykologiska svårigheter att handskas med sitt koloniala arv.

 I Kina lär det finnas hur mycket te som helst, samma mängd som det finns smör i Småland faktiskt.

}

\small{
\textbf{Techno}
\label{457cb7ec5ed461a6d7ccb025d67bef32}
 är en sorts musik man dansar till. Den uppfanns i Detroit på 80-talet sen spred den sig över hela världen. I Berlin finns mycket techno. Det finns massa subgenres men dom låter typ likadant så det är ingen idé att bry sig riktigt. Man kan med fördel ta olika sorters knark när man ska dansa till techno.

}

\small{
\textbf{Teenage Mutant Ninja Turtles}
\label{fd9ccf7b23fd53b8c3bb91065ab585ee}
 är en USA-berättelse om fyra sköldpaddor som medelst retomutagen och fysisk kontakt med ungdomar (?) blir transformerade till humanoida ninjasköldpaddor som utnyttjas av en likaledes humanoid råtta vid namn Splinter till att slåss mot noshörningar och hjärnor från andra dimensioner.


 HEAD2: Teenage Mutant Ninja Turtles vs. Skalman
 Teenage Mutant Ninja Turtles kan inte, som vår egna svenska mostsvarighet till dem, Skalman, dra in huvudet i skalet eller ens gömma saker i det. Däremot kan de röra sig mycket snabbare än Skalman någonsin skulle vilja göra. Skalman kan uppfinna saker, men det kan en av Teenage Mutant Ninja Turtles också göra, om än inte lika finurliga grejer. Skalman är smart och pacifist, medan Teenage mutant Ninja Turtles inte är lika smarta och inte skyr våld i kampen för rättvisa. Skalman bor i ett fett hus, medan Teenage mutant Ninja Turtles bor i kloakerna i New York. Skalman har tagit sig ur ett omfattande drogmissbruk, vilket skildras i en serie från 1988 [http://www.aftonbladet.se/nyheter/article13055.ab], medan Teenage mutant Ninja Turtles uteslutande äter pizza och inte har några betänkligheter med det. Så till \textit{syvende og sisdt} får man nog säga: Skalman - Teenage mutant Ninja Turtles, 1-0.

 Externa källor: [http://www.bamse.se/bamsefakta/figurerna-i-bamse/skalman.aspx].

}

\small{
\textbf{Tegare}
\label{61a9e94d20a0e011579891609fa7d765}
 En tegare är i Umeå en korv i bröd med allt på. Tegaren är döpt efter den rika stadsdelen Teg i Umeå. Således passar tegaren in i kategorin fet och grisig mat döpt efter lyxiga ställen/personer \textsc{(s.~\pageref{f4e4d985528ce8d2da975e2a5cca4146})}.

}

\small{
\textbf{Tegsnäsare}
\label{f2d36877ad79d9fe50d1415d462f9e8b}
 är ett begrepp för de träskidor som produceras i Granö, Västerbotten \textsc{(s.~\pageref{d4b008c5143dcffb6b8c35f3876c2a19})}. De är det överlägset bästa transportmedlet på snö. Jämfört med en skoter \textsc{(s.~\pageref{b1120baa83f380cd42a805a4e823cb1b})} så kan de inte köras fast, skära, få slut bensin och annat otrevligt. De har inte heller någon variatorrem som måste bytas ute i kylan. Jämfört med snöskor \textsc{(s.~\pageref{abaa4c229e523e7c888d3e00ca0d6986})} får man skjuts i nedförsbackar och de långa skidorna har ofta bättre bärighet i lössnö. Skulle det vara så att de ändå sjunker igenom så hittar laxstjärten längst fram på skidan alltid upp i det fria igen. Drivmedlet består av mat som t.ex. palt \textsc{(se lista över anständig mat:palt s.~\pageref{4f4a723510a797072acc59e652235be6})}, och den i många sammanhang fantastiska människokroppen omsätter detta i kraft framåt. Framåt mot nya upptäckter.

 HEAD2: Förslag på aktiviteter med Tegsnäsare
 Åk till Sveriges \textsc{(se Sverige s.~\pageref{b1999637949ed135b2ca03f3a38460cc})} nordligare delar om där skulle finnas snö. Åk rakt ut i skogen tills du är halvtrött \textsc{(s.~\pageref{07a5c50b32349c286c73a9ef44eec914})}, gör ett lappkast \textsc{(s.~\pageref{9dd6698c53a9d42abffb80092f739ae2})} och åk tillbaka till civilisationen. Väl tillbaka kanske du bastar eller myser vid öppna spisen, trevligt för alla!

}

\small{
\textbf{Telefon}
\label{15a957eec81cff8df3172257b813e2d3}
 En telefon hade från början sitt syfte att låta folk kommunicera med varandra på avstånd.

 Den första telefonen bestod av två stycken konservburkar med ett spänt snöre emellan.

 Numer använder man telefonen för att skicka textmeddelanden (Sms) till varandra och för att spela spel (Tex Angry birds).

 Det händer att man försöker att ringa upp någon med sin telefon men då är det nästan alltid slut på ström i batteriet.

}

\small{
\textbf{Telefonkatalog}
\label{a1c3d8187f7afc13f933d7d93b27f536}
 En telefonkatalog är en katalog med telefoner i SKULLE MAN KUNNA TRO.
 På den gamla goda tiden fanns det stora ljuvliga böcker med flera hundratusentals sidor i med namn och telefonnummer \textsc{(s.~\pageref{0978f3303660fc9c74d08f85b89ba974})} till de personer som hade en telefon \textsc{(s.~\pageref{15a957eec81cff8df3172257b813e2d3})}.

 En normalstor telefonkatalog som tex Gävle-delen vägde ca 7 Kilo och Televerket \textsc{(s.~\pageref{4cf398db49e1da53e4a4d3f34dce77e3})} som gav ut katalogen gjorde busiga försök att få katalogen att se hipp ut genom att från år till år ha mörkgröna och ångestframkallande blå färger på pärmarna.

 HEAD3: Andra användningsområden
 Telefonkataloger för områden som större städer t ex Stockholm \textsc{(s.~\pageref{edcd259e0a03c7ab70feb186bae19f13})} var av sådan vikt att de lätt hade kunnat fungera som ankare till en mellanstor oljerigg.
 Ville man ha en evighets klabb till vedbrasan så var telefonkatalogen oumbärlig, det finns berättelser om sådana som brunnit i månader utan att elden falnat, dock blir den tråkigare att läsa (om möjligt) i detta skick.
 Telefonkataloger fungerade även som trumset hemma hos folk innan \quotetext{Rock band} lanserades till tv-spelet.
 Får man slut på skithuspapper är telefonkatalogen en välsignelse.
 När man befinner sig i uppbyggnadsskedet av sitt herbarium kan en telefonkatalog fungera som växtpress.

 HEAD3: Utveckling/Konsekvens
 Precis som allt annat som fungerar bra, är billigt och nöjsamt för telefonkatalogen idag en tynande tillvaro. Styggelser såsom Ipad och betaltjänster såsom 118118 tar snabbt mark. I framtiden kommer ingen kunna spela trummor eller tända i spisen.

}

\small{
\textbf{Telefonnummer}
\label{0978f3303660fc9c74d08f85b89ba974}
 Ett telefonnummer är ett nummer som ofta går till mobilsvar eller som upphört.
 Många personer byter telefonnummer väldigt ofta, detta för att reta gallfeber på den som vill få tag på vederbörande eller för att fylla upp telefonboken hos sina vänner.

 Om det dyker upp ett telefonnummer i din mobil och detta nummer inte är länkat till en kontakt i din telefonbok så svara för tusan inte! Utan du bara väntar tills du kommer hem och söker på Eniro alt Hitta.se för att få reda på vem som faktiskt ringt och sökt dig, enklast hade varit att svara men det kan vara förenat med fara av okänt slag.

 HEAD2: Kända telefonnummer
 Bert Karlssons telefonnummer \textsc{(s.~\pageref{958ccc32173aa9f4086ac4a314f4909e})}

}

\small{
\textbf{Televerket}
\label{4cf398db49e1da53e4a4d3f34dce77e3}
 På den gamla goda tiden när det fanns nån att ställa till svars för saker och ting fanns ett verk som såg till att medborgarna kunde kommunicera medelst telefoni.
 Detta fungerade år efter år utan fåniga reklamkampanjer eller tonåringar som \quotetext{brinner för att sälja \textsc{(s.~\pageref{4397dcfa1c80db06a775fb49f5171806})}} i köpcentrum. Televerket skapade en helt ny färg, Televerksorange \textsc{(se Färgskala:Televerksorange s.~\pageref{c3a50833aa9223a066ecd2a535dd928f})} som det Holländska \textsc{(se Holland s.~\pageref{b95f379f6ae245614d2f949801524317})} fotbollslanslaget sedan plankade. Verkets personal hade uniformer och kunde om kriget kommer \textsc{(s.~\pageref{86325b0844aed9a3678fc492c795ba16})} beväpnas för att försvara medborgarens rätt att ringa hem. Den medborgare som visat sig pålitlig och dygdig kunde få tillåtelse att ha telefon - notera att medborgaren inte själv ägde telefonen, den var Televerkets egendom och bara till låns.

 \quotetext{Det här är ju för bra för att vara sant} tänkte Carl Bildt när nazisterna kom till makten 1991 och så krossade han detta hedervärda verk.

}

\small{
\textbf{Television}
\label{79464212afb7fd6c38699d0617eaedeb}
 är en form av informationsteknik som tillhör etermedia på informationsteknikens vittförgrenade släktträd. Den kommer konsumenten till godo via en digital-teknologisk apparat som till det yttre påminner lite om en mikrovågsugn vari Lars Adaktussons huvud talar om något på ett ganska borgligt och sövande vis. I likhet med facebook \textsc{(s.~\pageref{26cae7718c32180a7a0f8e19d6d40a59})} utgör televisionen något som medlemmar av den lite finare delen av det svenska folket påstår att de vägrar befatta sig med, eftersom den genomsnittliga människan har för vana att ibland titta på TV för att inhämta information om dagsaktuella händelser, Frida Kahlos liv, hur man talar Serbokratiska, vilken av fyra olika djurliknande pappfigurer som ska bort eller helt enkelt för avkoppling efter en tung arbetsdag på sågverket eller Institutionen för språk och litteratur.

}

\small{
\textbf{Telverksorange}
\label{ca5043d4e81ab5163ab467d10a37cc95}
 Borgarna kan ta vårt televerk \textsc{(se Televerket s.~\pageref{4cf398db49e1da53e4a4d3f34dce77e3})} men de kan aldrig ta vår färgnyans. När revolutionen kommer ska de doppas i tjära färgad R253 G139 B52 och rullas i fjädrar från Hedemorahöns uppfödda på Blåvitts \textsc{(se Blåvitt s.~\pageref{0f52aa49c4f8ad1ffc94f831701fc119})} allfoder.

 Se även: färgskala \textsc{(se Färgskala:ASEA-grönt s.~\pageref{bfd16917b4c5964997f496f424382446})}.

}

\small{
\textbf{Tengah}
\label{60645771c976ff0a124f1de0c1122015}
 är den ö i Stilla havet som förmodligen har störst plats av alla Stilla havets öar i svenska folkets hjärtan. Det var nämligen här den första upplagan av Expedition Robinson spelades in. Miljontals svenskar har förälskat sig i den lilla korallön med soliga sandstränder och svalkande djungel. Vem har inte någon gång kommit på sig själv med att dagdrömma om hur det vore att byta ut sin urbana tillvaro mot Tengahs inbjudande och okomplicerade. Mumsandes på ett palmhjälta vandrar man på samma stigar där en redan då åldrad Harad Treutiger kanske smet ifrån TV-teamet för att hitta en lämplig stock att slå sig ned på med neddragna byxor och förrätta sina behov. Var det bakom den här stenen Jochem prövade lyckan med Dr. Åsa? Var Zübeyde en hemsk människa redan innan hon kom hit? Livet, vad är det egentligen?

 Tengah har svaren.

}

\small{
\textbf{Tenzing Norgay}
\label{5064d0d0c513aec890ffa0ef5d7577ac}
 var en Nepalesisk bergsbestigare och en av de första människorna att sätta sina fötter på Mount Everests \textsc{(se Mount Everest s.~\pageref{b5b5d890ef4ff008c7821da350799545})} topp tillsammans med Edmund Hillary \textsc{(s.~\pageref{8c30ada4e29fa9820d3f6850dc843b0c})}. Han säger själv att han var tvåa men påpekar att \quotetext{om det är en skam att vara den andre mannen på Mount Everest så är det en skam jag ska bära}. Detta sa han efter att ständigt ha blivit ställd frågan om vem som var först \textit{egentligen} av journalister som inte tyckte att svaret \quotetext{Vi gjorde det tillsammans.} dög. Detta är en viktig minnesbeta i våran åt helvete för individualistiska kultur: Mount Everest \textsc{(s.~\pageref{b5b5d890ef4ff008c7821da350799545})} bestegs inte av en enskild individ, utan \textit{tillsammans}. Glöm aldrig det.

}

\small{
\textbf{Teodicéproblemet}
\label{b48e9b1bbe84760c44d04c37f2a2ab52}
 syftar på det trepipsproblem \textsc{(s.~\pageref{ddfa7edb7b4169a1dc8a32b1a8ad9611})} som kristna as står inför. De måste nämligen på något vis förklara hur detta går ihop:
\begin{enumerate}
\item Gud \textsc{(s.~\pageref{91e49146121c992aab11a19c77e26cf0})} är god
\item Gud är allsmäktig
\item Gud gör inget åt att merparten av världens befolkning lever en väldigt crust as fuck existence \textsc{(s.~\pageref{bd0b07abcc2f4c2a4e1aafdfed1f0e73})}.
\end{enumerate}

 Dessa olika steg går inte att logiskt kombinera, så någonstans däremellan har det blivit lite bananas \textsc{(s.~\pageref{ec121ff80513ae58ed478d5c5787075b})}, för:
\begin{enumerate}
\item Antingen är gud inte god...
\item ..eller inte allsmäktig...
\item ...för annars skulle han ha tagit tjuren vid hornen och gjort något åt saken.
\end{enumerate}

}

\small{
\textbf{Tetris}
\label{f76e534251c8595a9746fde225f9289b}
 är ett spel som går ut på att man ska samla poäng och har uppfunnits av en ryss. Vad han hette går att hitta på Google \textsc{(s.~\pageref{c822c1b63853ed273b89687ac505f9fa})}.

}

\small{
\textbf{Text-TV}
\label{358c0f26078d2f24ba1ea75068a7fc0f}
 Man ska inte lita på vad som står på internet, däremot är text-tv en pålitlig källa med korrekt och advekat information .
 Fordom tida; under text-TV:s guldålder fanns även en chatfunktion som dock krävde att man var betrodd hos Televerket \textsc{(s.~\pageref{4cf398db49e1da53e4a4d3f34dce77e3})}.
 De pålitilga samhällstjänarna på Sveriges Televison har fattat galoppen och erbjuder därför text-TV även på internet: [http://www.svt.se/text]

}

\small{
\textbf{Thailändsk jordnötssås}
\label{dd870c763347e6c2fa699a378106f4a8}
 Denna smakfulla och söta sås är perfekt till grillspett av grönsaker och marinerade sojabitar som serveras till ris.
 \begin{itemize}
 \item En burk kokosmjölk (EJ light)
 \item En tredjedels burk skippy jordnötssmör
 \item Lite vatten
 \item Ca två matskedar röd kurrypasta
 \item Ca två teskedar vitvinsvinäger
 \item Salt
 \end{itemize}

 Blanda alla ingredienser i en kastrull. Värm upp och småsmutta \textsc{(se smutt s.~\pageref{d9114ffee4f2dcee302ae2b19ce5eea9})} på låg värme en stund - gärna så länge grillspett \textsc{(se järnspett s.~\pageref{6cbe55f18d91c10e3307681ab810fd74})} grillas så att såsen grisar till sig riktigt fint. Servera med folköl.

}

\small{
\textbf{The Ben Richards}
\label{5e543c52e56b4f7f48ba0d8d41a9c43d}
 var ett punkrockband från Fagersta/Norberg som var aktiva mellan ungefär 2000-2005. Bandet startade som ett skolprojekt på estetlinjen vid Domarhagens gymnasium i Avesta men kom med tiden att uppträda över stora delar av Sverige.

 Bandet startade under namnet Sedo och frontades ursprungligen av Fredrik Hellman, men efter endast några få liveframträdanden hoppade denne av för att flytta till Mora och satsa på en karriär som hockeyspelare \textsc{(se hockey s.~\pageref{df0349ce110b69f03b4def8012ae4970})} (väl där blev han också medgrundare av bandet Fan De Roj). Jimmy Koivisto plockades in som ny sångare och i och med detta bytte man också namn på bandet till The Ben Richards, ett namn man lånade från en Arnold Schwarzenegger-rulle. Under de aktiva åren var bandet mycket uppskattat i hemtrakten och spelade flitigt inför rockälskande ungdomar. Populariteten växte och bandet kom att spela så långt hemifrån som Kiruna. Alla var dock inte lika glada och i Kopparberg hånades bandet under en hel spelning av fulla raggare (samma människor som bokat bandet) som bara tyckte om Turbonegro. När det inte var roligt att spela längre la man ner. Medlemmarna bor dock ihop i ett husvagnsläger utanför Skinnskatteberg \textsc{(s.~\pageref{f0666ee995da080da55f5f6892fe3dcc})} som dom kallar nudistkoloni. Men det är bara dom som går nakna.

 HEAD2: Medlemamr
 \textit{Mest frekventa medlemmar:}


 Jimmy Koivisto - Sång
 Erik Wisell - Elgitarr
 Love Lundkvist - Elgitarr
 Erik Olsen - Elbas
 Christian Nordlander - Trummor
 Johannes Risberg - Klaviatur

 \textit{Tidigare medlemmar:}

 Fredrik Hellman - Sång

 HEAD2: Diskografi
 \textit{Demo 1} kassett
 \textit{Demo 2} CD-r
 \textit{Playin' High} CD-r

}

\small{
\textbf{The fat Spanish waiter}
\label{f7c48b9fefc4a468445da238409fee27}
 Supportrarnas smeknamn på Rafael Benitez efter att denne blivit klar som tränare för Chelsea FC. Vid första matcherna hälsades han välkommen med burop och fyndiga, men inte särskilt välvilliga, banderoller. The fat Spanish waiter varade i klubben en halv säsong.
 .]]

}

\small{
\textbf{The Fog}
\label{576875ef0042ff21c04f5f1b9377d4e7}
 \textbf{John \quotetext{The Fog} Fogerty}, född 28 maj 1945 i Berkeley, Kalifornien, är en amerikansk sångare, gitarrist och levnadsförebild. Hans karaktäristiska sång- och spelstil har lett till att hans musik betecknas som swamp rock. Tillsammans med Bruce Springsteen tävlar han om titeln \quotetext{Världens mest heterosexuella man}.

 The Fog började sin musikaliska karriär i Creedence Clearwater Revival där han skrev de flesta låtarna. Gruppen spelade bara in hits och vid utgivning av samlingsskivor är det kutym att man slumpar låtar från back-katalogen. Karriären varade dock bara i fem år och splittringen föregicks av vilda debatter i media om huruvida medlemmarna verkligen sett ett riktigt träsk någon gång. The Fog fortsatte därefter som soloartist och skrev bland annat \textit{Rocking all over the world} som senare blev en hit med Status Quo.

}

\small{
\textbf{The fog}
\label{576875ef0042ff21c04f5f1b9377d4e7}


}

\small{
\textbf{Thomas Wassberg}
\label{0d360191af8bff0ae30fa23ff0af3a11}
 Thomas \quotetext{Säcken} Wassberg åkte ofta skidor, men hade framförallt skägg.

}

\small{
\textbf{THOR}
\label{575e22bc356137a41abdef379b776dba}
 var den 3:e raketen som skulle skickas upp men den startade inte på grund av att stubin hålet va för grunt men efter att det rättats till så stod den och rök, inte heller den var ett så lyckat försök.

 Tidigare raketer:

 Tsygan II \textsc{(s.~\pageref{da919dfb81083059022a634b495dac7d})}

 LANCELOT II \textsc{(s.~\pageref{386a45bf415cd217bf0eb4ab02876db8})}

}

\small{
\textbf{Thorstenkram}
\label{49bea2c07a6583a3375e1d5704a3eee0}
 En thorstenkram är en kram under vilken en långtidssalongsberusad man i sina övre medelår för sin ena hand ogenerat ned mot sin betydligt yngre, kvinnliga motparts rumpa. En thorstenkram är ofta, från givarens sida, ett resultat av behagsjuka eller lojalitetskänsla.

}

\small{
\textbf{Thrashzan}
\label{eba46a1d37758daf585fb60b69c7991c}
 är en fiktiv superhjälte som beskyddar de hårda bandens universum. Hans superkraft är att kunna röja sanslöst på spelningar längre än någon annan. Han kan alla klassiska moves och blandar friskt mellan genrer så man ska inte bli förvånad om man ser honom köra kängnäven \textsc{(se Veva med kängnäven s.~\pageref{0b5f330433e1fc19a412718dba802627})} och djävulstecknet samtidigt med varsin hand. Tyvärr är Thrashzan ibland lite väl hård på spriten, vilket gör att han tillfälligt tappar sina krafter. Klassiska strider han förlorat är mot Scary Guy på Load och Reload, och mot Vic Rattlehead på Risk.

}

\small{
\textbf{Tia}
\label{e7292d5ba58672ce7f6fc3c0b646ab63}
 Mässingsmynt som berättigar till två stycken fika a 5 kronor. En tia KAN vara en vinylskiva av sagda tumsmått, men är sällsynt och ofta värdelös.


 Se även: Etta \textsc{(s.~\pageref{ba48f6c4097b7fc25ca11f1e544842d7})}, Tvåa \textsc{(s.~\pageref{84fcc0494ecf9f5af79fcd9bed184a9a})}, Trea \textsc{(s.~\pageref{6f94fdf535ab2e21147ea40ea920ca75})}, Fyra \textsc{(s.~\pageref{7bdb5385ce8e0b1cbc7c15b1d71e8e7d})}, Femma \textsc{(s.~\pageref{d974e0811fe7a4d49a9062d33b66a88d})}, Sexa \textsc{(s.~\pageref{4b1fabe53857b0a2ace6ae22008fe13e})}, Sjua \textsc{(s.~\pageref{e7bf63fa6d0d29bd89c23f833b979a15})}, Åtta \textsc{(s.~\pageref{6fa68b0d02ec525fa72a51c13e5e3ed1})}, Nia \textsc{(s.~\pageref{04a481486dd84d7c8bfdfc89d38136a6})}, Elva \textsc{(s.~\pageref{788bd84addbcf8f1767869759d4a2ad9})}.

}

\small{
\textbf{Tingsryd 2,8}
\label{1248f188e41a1f64d2dac526a8b6704c}
 (ord kan inte beskriva hur utsökt denna dryck är)

}

\small{
\textbf{Tisdag}
\label{47ee958d272b159c6a0dfb024c6f9155}
 är den dag då tvådagars-bakfyllan efter helgen släppt och det är dags att bygga upp kroppen igen. Med fördel spelas fotboll \textsc{(s.~\pageref{961bd74d34872ff94a4df3a16119096e})} under de varmare månaderna, om inte ordningsmakten har tänkt annat, och under de kallare månaderna längdskidor eller TV-spel.

}

\small{
\textbf{Titta på ord}
\label{0ba1b1de8f15a42c24012ab201be4485}
 En person som tittar på ord sitter med en bok (oftast kurslitteratur) i ett bibliotek (oftast universitetsbibliotek), hukad under en lampa och tittar på orden i den tidigare nämnda boken. Personen kan lätt misstas för en ambitiös student \textsc{(se Den ambitiösa studenten s.~\pageref{02257ef6d6da8e0f0721e2758eec3c71})}, då den ofta sitter länge med boken och bläddrar ganska ofta. I verkligheten har personens blodsocker sedan länge sjunkit till en nivå som omöjliggör intagande av information. Således tittar bara personen på ord, snarare än att läsa. Distinktionen mellan dessa två (läsa/titta på ord) är ytterst viktig.

}

\small{
\textbf{Tivoli}
\label{5926d4ffe73f106a0bc0929068981515}
 Dansken besitter som bekant en no-nonsens-attityd och är som lika bekant en hedonist av rang. Därför är det inte förvånande att ett av världens äldsta tivolin, med det träffande namnet Tivoli, finns just i Danmarks huvudstad Köbenhavn. Hit går dansken för att ta öldrickandet till en högre nivå genom att dricka Tuborg ombord på sinnrikt konstruerade åkattraktioner utan säkerhetsbälte. Mellan åken mosar dansken i sig rød pølse \textsc{(s.~\pageref{dd9f0cd7c204300945924c7de9eb5649})} och wienerbröd som funnes det ingen morgondag samtidigt som hen står och skjuter lite halvhjärtat på plåtburkar i hopp om att vinna en stor nalle att gå runt och göra sig lustig med.

 HEAD2: Historik
 Tivoli byggdes redan på 1400-talet, alltså på den tiden då man låg i fejd med de gamla grekerna \textsc{(s.~\pageref{4a5fb3d6ce79b5ff43b33f8f7d843672})} och hunnerna. Man behövde en fästning som alla kunde springa in i om det kom nån armé och ville hålla på, så man påbörjade arbetet med vad som planerades att bli en stor borg och som skulle kallas Borg. Arbetet gick dock sakta och det var inte så många som kom dit och hjälpte till. När Christian Tyrann sa \quotetext{amen nu fo di faen skaerp er! Kom igen nå. Hvem kan hva med och bygge denne bork?} så sparkades det mest i gruset och mumlades bortförklaringar om inbokade hundpromenader och besök från släktingar. Kungen dog sedermera av vällevnad och i glappet mellan honom och hans arvinge, Hamlet, kom nån på idén att bygga ett tivoli istället. Då blev det ett fasligt liv och allt stod klart innan någon hunnit säga flasklock. På den vägen är det.

}

\small{
\textbf{Tjabo \& the Fuck Offs}
\label{6bbe1ec86075ced3b130de5f73b3ab4a}


}

\small{
\textbf{Tjack}
\label{d7a6f2fe218c9a3acc82501f02a89587}
 Vit substans i pulverform med ursprungsland i Östeuropa. Orsakar vid ymning intagande att barndomens spring i benen för en stund återvänder. Perfekt för den som behöver städa ur garaget för att få fram klotgrillen till valborgsfirandet eller har punktering på cykeln och således måste gå till dansbanen i grannsocknen.

 HEAD2: Tjack i populärkulturen
 \textit{\quotetext{Sunrise, wrong side of another day},
 \textit{Sky high and six thousand miles away},
 \textit{Don't know how long I've been awake},
 \textit{Wound up in an amazing state,}
 \textit{Can't get enough,}
 \textit{And you know it's righteous stuff,}
 \textit{Goes up like prices at Christmas,}
 \textit{Motorhead, remember me now Motorhead, alright}

 \textit{Brain dead, total amnesia,}
 \textit{Get some mental anasthesia,}
 \textit{Don't move, I'll shut the door and kill the lights},
 \textit{I guess I'll see you all on the ice,}
 \textit{All good clean fun,}
 \textit{Have another stick of gum,}
 \textit{Man, you look better already,}
 \textit{Motorhead, remember me now Motorhead, alright}

 \textit{Four days, five day marathon,}
 \textit{We're moving like a parallelogram,}
 \textit{Don't move, I'll shut the door and kill the lights,}
 \textit{I guess ill see you out on the ice}
 
 \textit{I should be tired,}
 \textit{And all I am is wired,}
 \textit{Ain't felt this good for an hour,}
 \textit{Motorhead, remember me now, Motorhead alright}}

 - L. Kilmister

}

\small{
\textbf{Tjackad}
\label{b12ceb5f265e6ab9afcd2c662715e0b5}
 Du vill gå ut och promenera i fem \textsc{(se femma s.~\pageref{d974e0811fe7a4d49a9062d33b66a88d})} timmar samtidigt som du deltar i fem \textsc{(se femma s.~\pageref{d974e0811fe7a4d49a9062d33b66a88d})} olika konversationer, och om du är man är ditt könsorgan -1 cm långt. Å andra sidan är det relativt billigt och om det blir jidder har du överhanden, eftersom du har:
 \begin{itemize}
 \item energi
 \item ingen känsel
 \item ett icke obetydligt mindrevärdeskomplex.
 \end{itemize}

 \textit{Tjackad}

}

\small{
\textbf{Tjamstan}
\label{76f026797a3d868f6a32a26b28f76f8e}
 Ett berg som med sina 400 m ö h tornar upp över Malå \textsc{(s.~\pageref{41da4620e87888eaaeafcb3004a8d177})} samhälle. På dess västra och norra sida finns skidbackar, på dessa södra ett stup. Detta stup inhyser i sin tur Ättestupan \textsc{(se Ättestupa s.~\pageref{0724235523055ba38202b1a661a68722})}. Här skickades gamla och sjuka mot sin död för att inte vara sina släktingar till last. Detta kommer antagligen bli aktuellt igen med tanke på de nedskärningar landstinget försöker få igenom.

}

\small{
\textbf{Tjejanarkism}
\label{3f42833f95ddc62fd6a1b2f609847586}
 Synonymt med minarkism; man försöker inte maniskt \textsc{(se mani s.~\pageref{07cd55c7b42715ec44c133a6a165e8d2})} avskaffa staten, utan nöjer sig med nattväktarstat.

 HEAD2:  Se även
 Tjejsamla \textsc{(s.~\pageref{9244b603c303d1d48ceeb79a3a0e4c71})}

}

\small{
\textbf{Tjejrent}
\label{904a1c27d10e1b9d641f57c45953ec3f}
 En för personer av honkön \textsc{(s.~\pageref{204e209b96ab0d93124f83ebe1dd4b03})} upplevt nödvändig förutsättning för t.ex. tjejkvällar eller parmiddagar. Tar fyra timmar i en etta. Doftar gott och ser ut som en ikeamonter, förutom de nyplockade ängsblommorna i en liten vas på bordet. Dubbelt så rent som en grabbren \textsc{(se grabbrent s.~\pageref{b9a8d4c49a300de05ce98dfa59b80ff8})} flyttstädning.

}

\small{
\textbf{Tjejsamla}
\label{9244b603c303d1d48ceeb79a3a0e4c71}
 är ett ord som Språkrådet menade var ett nyord 2011. Det betyder att man samlar på något, men inte lika maniskt \textsc{(se mani s.~\pageref{07cd55c7b42715ec44c133a6a165e8d2})} som killar. För att ytterligare förklara så kan man säga att om man tjejsamlar på Discharge-plattor köper man bara \textit{Why?} och \textit{Hear nothing, say nothing, see nothing}. Killsamlaren köper däremot första sjuan \textsc{(s.~\pageref{7b7c558fc3f8d8557ba30b082e644ea1})} \textit{Realities of war} utan att tveka, och går till och med så långt att han köper \textit{Grave New World} och \textit{Massacre Divine}. Finns det ingen hejd på hans samlanade? Nä det gör det inte, för han köpte också \textit{Beginning of the end}. Ni hör ju.

 Det är säkert fullt möjligt för killar att tjejsamla på något, och för att tjejer att maniskt samla på något, men så fungerar tydligen inte svenska språket.

}

\small{
\textbf{Tjena Roger!}
\label{ab6879ac5fc56c1ee4071bfc88456bcb}
 är en till användningen mycket bred och mångsidig interjektion som är Sveriges svar på spanskans \textit{Ay Caramba}. Ordet är av okänd härkomst men används i huvudsak Norrtälje \textsc{(s.~\pageref{7527f7dad9445013a559dc7e2a91f3b3})}, Skandinaviens Mexico, och inom den alltid växande norrteljska diasporan. Tjena Roger! kan användas i situationer som uppfattas av talaren och dennes vänner som positiva såväl som i sådana som uppfattas som helt igenom negativa (keffa). En person som hittar en brakare \textsc{(s.~\pageref{da8590943fa645cfceaa235a83d1d797})} på gatan och avser att använda den för att köpa förfriskningar till sig själv och sina vänner utan att behöva göra rätt för sig \textsc{(s.~\pageref{c8c01e0e8b4ad8e5ff6011b8af6405a5})} kan av förtjusning däröver utropa Tjena Roger! Lika så kan någon som plötsligt får se att den egna huden har attackerats av psoriasis eller någon annan oönskad hudåkomma i bestörtning utropa detta användbara uttryck.

}

\small{
\textbf{Tjhosan-mysteriet}
\label{f518bd125a313dc5278790a9e4d4c8e9}
 Begreppet tjhosan-mysteriet syftar till den offentliga debatt som blossade upp under hösten 2008 kring den i populärkulturen uppmärksammade skivomslagsbild som ses här till höger. Bilden blev först offentlig i ett avsnitt av SVTs populära och prisbelönta kulturprogram \textit{Kobra} och blev snabbt föremål för seriösa så väl som skämsamt menade frågeställningar i blogosfären, tv-program, dags- och kvällspress och till och med i P1s \quotetext{Spanarna}. Enligt \textit{Kobra} hade skivomslaget hittats, utan tillhörande skiva, av arkivarier i Sveriges Radios skivarkiv. I de mer seriösa diskussioner som föranleddes av bilden ställdes följande frågor:

 \begin{itemize}
 \item Är skivan autentisk?
 \item Vad är egentligen skivans titel - \quotetext{Tjhosan!,} \quotetext{Dragspelslåtar av Ålands Walter Eriksson} eller \quotetext{Åland i mitt hjärta}?
 \item Är Walter Eriksson artisten som spelat in skivan eller endast låtarnas upphovsman?
 \item Är artisten svensk eller finsk - dvs, vilken är anknytningen till Åland?
 \item Hur uttalas \quotetext{tjhosan}?
 \item Var sitter dragspelsmuskeln? \textsc{(se dragspelsmuskeln s.~\pageref{4265ffc7068c10706460aa133c2918bf})}
 \end{itemize}

}

\small{
\textbf{Tjillevippenspik}
\label{7482c9f865623139c98a482a37422586}
 Tänk va jobbigt om du hade en tjillevippenspik hemma. Du har bestämt dig för att för en gångs skull vara lite duktig och dammsuga ordentligt hemma. Du har baxat bort byrån och står på knä och så – pang! – har du kommit åt tjillevippenspiken och blivit asliten. Du kan inte skrika på hjälp för ingen hör dig, du kan inte gömma dig i någon trygg vrå för där bor det garanterat en spindel, och du kan inte slå på någon kul film \textsc{(se sällskapsresan s.~\pageref{1023ca20cc8ad5b3f0233d023ad01bf5})} för det är för tungt. Det enda du kan göra är att ha skittråkigt och sitta och vänta på att bli stor igen. När du så äntligen vuxit till dig igen går du raka spåret till verktygslådan för att hämta hammaren och dra ut det där jävla spikhelvetet en gång för alla. Men så fort du sätter hammarens baksida i den så – pang! – har du blivit liten igen. Hel-vet-e.

}

\small{
\textbf{Tjock}
\label{638fc266ae8a51a3eeb87a4cab84e057}
 Om man går över 21 när man spelar Black Jack är man tjock.

}

\small{
\textbf{Tjock-TV}
\label{2cc5866a7a0b7164945aadbf9b5d13ce}
 n är den historiskt sett mest populära formen av TV-apparat, men har på sistone förlorat en hel del terräng till förmån för platt-TVn framför vilken allt fler familjer samlas i de svenska hemmen. Tjock-TV utmärks av att vara ungefär lika tjock som den är bred, det vill säga kubformad. Den är också vanligtvis extremt tung och ger efter några månaders användning ifrån sig ett högfrekvent pipande ljud. Tjock-TV kan, liksom den konkurrerande TV-formen, levereras tillsammans med en fjärrkontroll, eller som det tidigare kallades, en dosa. Med denna kan den som manövrerar tjock-TVn byta kanal och finjustera ljudnivå och bildkvalité utan att behöva resa sig och gå fram till apparaten. En stor bonus som köparen av en tjock-tv får vid införskaffandet av en apparat är text-TV \textsc{(s.~\pageref{358c0f26078d2f24ba1ea75068a7fc0f})}, där man kan läsa om sportresultat, väder och aktuella världshändelser. Text-TV kan ses som en alternativ, och föregående, form av internet \textsc{(s.~\pageref{c3581516868fb3b71746931cac66390e})}.

 HEAD2: Platt- eller tjock-TV?
 Många är de som har svårt att välja mellan att köpa en tjock-tv eller en platt-TV. Här kan nissepedia \textsc{(s.~\pageref{62400dadecd90cb5cd39062abe5a3e4a})} hjälpa konsumenten att göra ett informerat val, men valet måste konsumenten själv göra. Det viktigaste att tänka på är hur apparaten gör sig i rummet, det vill säga inredningsnivån av frågan. Har du ett smalt rum och saknar plats för alla de föremål du vill ha i det? Tja, då kan kanske en platt-TV vara din grej. Har du istället för mycket plats och för få ägodelar? Då är tjock-TVn det val som det mesta talar för. Den som vill avgöra om en viss TV-apparat är en tjock- eller platt-TV går så till väga att hen ställer sig vid sidan av TVn och kontrollerar hur bred apparatens sida är. Är den cirka en decimeter tjock rör det sig om en platt-TV, men om den snarare är närmare en och en halv meter tjock har man att göra med en gammal hederlig tjock-TV. Ett ytterligare sätt att kontrollera detsamma är att försöka lyfta TVn. Kan man göra detta med begränsad ansträngning är det man har i sin famn antagligen en platt-TV, men om ens rygg hotar att brytas av och svett tränger fram i armhålor och på tinningar är det nog en tjock-TV.


 HEAD2:  Klass och TV
 Vanliga pantade knegare nöjer sig med tjock-tv medan glidarjobbare och andra kälkborgare \textsc{(s.~\pageref{0f34b469a48952e93688861083ace75a})} suktar efter senaste modellen av platt-tv trots att man inte kan göra ett skvatt mer än med en tjock-tv. Således en pålitlig klassmarkör \textsc{(s.~\pageref{6a9c0c6836a0777442468f821837e795})}, men även en skiljelinje såväl etniskt som socio-ekonomiskt.

}

\small{
\textbf{Tjog}
\label{93e1254c6b6d02b89439cbea1926a4de}
 Tjugo stycken av något, i normalfallet ägg.

}

\small{
\textbf{Tjuv}
\label{8b51a2758b01437981669228f13ee224}
 En tjuv är en uv med en benägenhet att ta saker som de inte äger. Uvarna lever dock efter Proudhons maxim \quotetext{Egendom är stöld!} och tar därför allt de kan komma över.

}

\small{
\textbf{Tjänstemannateoretisk}
\label{ca15ca4df85a8b381bc49b991ea4f8f0}
 Föreställningar av typen: Tvingas söka jobb som VD på Volvo för att erhålla försörjningsstöd, när din högsta arbetsmerit är att ha sommarjobbat åt kommunen. Det \textit{kan} ju faktiskt gå.

}

\small{
\textbf{Toalettpapper}
\label{c26c6794f88231a7f4c6c295a900906b}
 är ett oftast mjukt papper upprullat på en hylsa av papp. Vanligtvis ligger det och dräller eller är prydligt packat i ett plastemballage i badrum samt under eller bredvid sängar och datorer där grabbar i puberteten lever och sover.

 När man som gäst nyttjar ett badrum kan man pga toapapprets kvalité bedöma vilken ekonomi ägaren till badrummet har.
 Låginkomsttagare som skiter i att läsa på förpackningen och bara kollar på kilopriset tenderar att köpa budegetmärken såsom Euroshopper, Eldorado (som för övrigt förvånansvärt vanligt på allehanda ovanliga bensinkedjor såsom Q-Star, St1 och Uno X) och Coop X-tra. Det tragiska för den omedvetne konsumenten i sammanhanget är att de märken som ligger några kronor över det billigaste ofta är mer ekonomiska sett till hur många meter du får per rulle. T.ex. är Änglamark kompakt fyrtiotvå meter per rulle i lager om tre, i kontrast till det billigare Coop X-tra som är strax under de trettio i lager om två.

 I samband med att folk vill tro att man kan konsumera sig till ett gott miljösamvete har toalettpappersidustrin börjat lansera toapapper för den miljömedvetne konsumenten som är gjort av returpapper. Detta är oftast mycket dyrare än de andra men också mer kompakt då hylsorna är mindre för att minska transportvolymen.

 Toalettpapper som är kritvita och har någon form av gulligt mönster är i regel dyrare än papper som ser ut att vara gjorda av innehållet i din lokala papperscontainer. Ingen vet ännu varför det förekommer hjärtformade mönster på toalettpapper. Stjärtformade mönster känns ack så mycket mer logiskt.

 HEAD2: Toapapper i levd så väl som medierad folkkultur
 En dekorativ funktion som toalettpapper kan fylla är på barnkalas och Halloween-fester där någon lustigkurre klär ut sig till mumie genom att linda toalettpapper kring större delen av sin kropp. Ett klassiskt busstreck i västvärlden där toalettpapper utgör ammunitionen är när barn eller ungdomar går till ett hus under en sen kväll eller natt och kastar rullarna över taket på målet för att jävlas \textsc{(se jävelskap s.~\pageref{46845591177f16920cd586a5baf5a625})} med de inneboende. Vill man krydda till det ytterligare kan man satsa på att även kasta rullar över ev. träd i anslutning till huset.

 Då och då fyller toalettpapper en funktion i olika former av media. Mest känt är kanske den animerade tv-serien South Parks avsnitt \quotetext{Toilet Paper} (avsnitt tre, säsong sju).[http://www.southparkstudios.se/episodes/703/]

 Även vid en så folklig företeelse som folkölsfyllor \textsc{(se folkölsfylla s.~\pageref{3cebf480c2ec39df0180688bf7c727ea})} fyller toalettpappret tydligen en funktion.

}

\small{
\textbf{Tobaksalternativ}
\label{ce7311323773c585240e411b26fa7551}
 Nästan-cigg \textsc{(s.~\pageref{2bcc66e1261fa5199a4f4decf2720ef5})}.
 Att nyttja örtsnus eller låtsascigg är jämställt med att äta vegankorv eller läsplatta.

}

\small{
\textbf{Tofu}
\label{5df7f1701b778d03d57456afea567922}
 är en vit massa med oklart ursprung, antagligen nån form av könsvätska från delfinen \textsc{(se delfin s.~\pageref{a62b1fca6b53d6670a84aa2c7b373b27})}. Men det är gott.

}

\small{
\textbf{Tofuve}
\label{e444d832130cf5b2b82fc96dca21e68a}
 En tofuve är en uv-imitation \textsc{(se uvbulvan s.~\pageref{2ebfd82fbe9976dbc2dc2c45ae58bde9})} gjord i Tofu \textsc{(s.~\pageref{5df7f1701b778d03d57456afea567922})}. Den sägs smaka ganska dåligt så som all veganmat \textsc{(se veganer s.~\pageref{2a12d5d6ae91d2f4f7d9af3cef58e75c})}.

}

\small{
\textbf{Tokfeminin}
\label{034c7c61e1d190c6d0f5f41b239801f2}
 En Pussy Galore's comin' down and we like it. En ung Britt Ekland. Deras idol är Satan. Postmodernt så det förslår. En mogen tysk kvinnas röv, \quotetext{Reife und Dicke} så att säga.

}

\small{
\textbf{Tokliberal}
\label{531cb70b602e3f3c32d40bac64400830}
 er kan en massa fina ord, såsom incitament \textsc{(s.~\pageref{f9896a922c4b9345ceebc37009eaf545})}, frihet \textsc{(s.~\pageref{0ee37cea60c9b45e40dbc83c0c665085})}, laissez-faire \textsc{(s.~\pageref{68f6c515dec88c0215c4ce05a1166c85})} och individ \textsc{(s.~\pageref{41beed76a0af9b4f550f7ebdecd3e700})}. Ord som solidaritet har den aldrig hört talas om.

}

\small{
\textbf{Tolkien och den svarta magin}
\label{b2cc087ddfcc973d83fe146ca31fe88e}
 är en reportagebok från 1982 av den svenske akademikern Åke Ohlmarks \textsc{(s.~\pageref{dc4829f902543aa5b8349fa82bafacb7})} (1911-1984). Bokens syfte är att avslöja vad vi alla länge misstänkt; familjen Tolkien och de tolkiensällskap som bildats över hela världen är i själva verket en maffialiknande sammanslutning som ägnar sig åt sexorgier, nazistiska ritualmord och organiserad brottslighet. Allt detta sker i hemlighet och kontrolleras av Tolkiens efterlevande från residenset i Oxford. Försäljningen av alvöron och gandalfhattar är i själva verket bara en täckmanter för att finansiera betydligt grövre saker. För att bevisa sin tes använder sig Ohlmarks av den, inom humaniora, mycket beprövade metoden \textit{guilt by association}. Exempelvis belyser Ohlmarks det klockrena sambandet mellan pappan till Svenska Tolkiensällskapets ordförande, som är tandläkare, och grundaren av Ku Klux Klan, som också är tandläkare. Han drar också paralleller från tolkiensällskap till \quotetext{\textit{... satansprästen Allister Prowley}} [sic!] och drogkarteller \quotetext{\textit{}... där röks troligen hasch; antagligen har man också börjat eller kan när som helst börja också med tyngre droger} (sid. 71). Bland de mer allvarliga anklagelserna finns också mordbrand. Det framgår till slut att mordbranden handlade om att Ohlmarks fru sängrökte, och att Åke anser att det inte hade lett till så allvarliga konsekvenser (en brinnande madrass) om Åke inte varit upptagen med att bevisa att J.R.R. Tolkien själv var ond trollkarl.


 HEAD2: För den som vill veta mer
 Förlag: Sjöstrands förlag AB

 ISBN: 91-7574-053-2

}

\small{
\textbf{Tolv}
\label{e9975628f89a1ce55ee39f34b04396ab}
 4 + 4 = 8
 8 x 2 = 16
 16 - 4 = Tolv!

 Tolv dagar kvar till Julafton, den 12/12 är en högtid som firas enligt gamla traditioner mestadels för att det är tolv månader kvar tills 12/12 faller igen.

 Tolv är också en mycket bra fotbollstid \textsc{(s.~\pageref{3d900fac062e1ecff848efad8875cb66})}.

}

\small{
\textbf{Tolva}
\label{75e2490604087d3d303b09a98803a16b}
 är ett annat ord fär LP- eller grammofonskiva. Tolvor kan aldrig bli bättre än första sjuan \textsc{(s.~\pageref{7b7c558fc3f8d8557ba30b082e644ea1})}

 Se även: etta \textsc{(s.~\pageref{ba48f6c4097b7fc25ca11f1e544842d7})}, tvåa \textsc{(s.~\pageref{84fcc0494ecf9f5af79fcd9bed184a9a})}, trea \textsc{(s.~\pageref{6f94fdf535ab2e21147ea40ea920ca75})}, fyra \textsc{(s.~\pageref{7bdb5385ce8e0b1cbc7c15b1d71e8e7d})}, femma \textsc{(s.~\pageref{d974e0811fe7a4d49a9062d33b66a88d})}, sexa \textsc{(s.~\pageref{4b1fabe53857b0a2ace6ae22008fe13e})}, sjua \textsc{(s.~\pageref{e7bf63fa6d0d29bd89c23f833b979a15})}, åtta \textsc{(s.~\pageref{6fa68b0d02ec525fa72a51c13e5e3ed1})}, nia \textsc{(s.~\pageref{04a481486dd84d7c8bfdfc89d38136a6})}, tia \textsc{(s.~\pageref{e7292d5ba58672ce7f6fc3c0b646ab63})}, elva \textsc{(s.~\pageref{788bd84addbcf8f1767869759d4a2ad9})}

}

\small{
\textbf{Tomgå}
\label{736a242ce397e36da48e96a1015406cd}
 Att gå utan mål. Mycket vanligt bland urban befolkning. Somliga tomgår till och med inomhus på transportband. Tomgång är riktigt okynnes.

}

\small{
\textbf{Tomte}
\label{09e6beb1584d480dbcc991fa8050b139}
 är ett suffix som läggs till ett adjektiv eller verb för att ge ordet motsatt betydelse. \textit{Tomtebra} betyder således \quotetext{dåligt} och \textit{tomtesnyggt} betyder \quotetext{fult}. Suffixet uppfanns i slutet av nittiotalet av en norrtäljekille som heter Gustav och som vid tiden för uppfinnandet var ca 14 år gammal. Det används framförallt i konversation som nedan:
 HEAD2: Exempel
 -Var filmen bra?
 -Tomtebra i alla fall. Tack för tipset!

}

\small{
\textbf{Tomten}
\label{3a3c1522c7155a18293fb1388055c13e}
 är ett spretigt fenomen vars minsta gemensamma nämnare är att alla har toppluva. Förutom denna uppmuntrande huvudbonad är de olika typerna av tomtar dock väldigt olika.

 HEAD2: Gårdstomten
 Två äpplen och en brakskit lång smyger denna lilla gynnare runt din bostad och ser till att skogsrået, bäckahästen och annat oknytt håller sig borta från ladugården. Tyvärr kan gårdstomten vara lite lynnig på samma sätt som en träskpunkare \textsc{(s.~\pageref{484838b3db1adb135ea74d6fc61e44c0})} och plötsligt få för sig att göra något rådumt som att låta brunnen sina eller höet mögla. Varför detta sker är något oklart och de enda råd som finns för hur man får in gårdstomten på den fromma vägen igen är tyvärr muntligt traderade skrönor fulla av absurda inslag.

 HEAD2: Trädgårdstomten
 Gjuten i betong eller porslin. I vissa undantagsfall av brons. I industrialialismens tidevarv kom fler och fler människor att överge den agrara ekonomin och därmed minskade behovet av att hålla sig väl med den stundtals väldigt slaniga \textsc{(se slan s.~\pageref{caaad522de864ab45ed679c4a16edd8d})} gårdstomten. Trädsgårdstomten har på många sätt kommit att bli den stora symbolen för kapitalismen då den låter sig köpas för pengar och därefter aldrig sätter sig upp mot sin herre.

 HEAD2: Jultomten
 Nej vänta det kanske är den här som är den stora symbolen för kapitalismen. Skit samma, den är inge rolig att skriva om i alla fall. Lyssna på Onkel Kånkels \textit{En gammaldags jul} istället.

 HEAD2: Haschtomten
 Lika oberäknelig som gårdstomten fast på ett annat sätt. Medan gårdstomten stjäl mjölken från korna drar haschtomten på mackshopping \textsc{(s.~\pageref{f6cbaa37785643222fb462b32a199d29})} och försnillar sina egna pengar på strandleksaker, en finsk \textsc{(se finland s.~\pageref{631d44eaa1254ff71a1e11ba021d1266})} fasadflagga och gasol till husvagnen som hen inte äger. Många av de skrönor som berättar hur man kommer till rätta med gårdstomten tros härröra från haschtomtens expanderade fantasivärld när denna lyssnat på Pink Floyd eller \textit{Dopesmoker \textsc{(s.~\pageref{cebc8a343bbfefbfac0078fcd926a0e0})}}. Haschtomtens bidrag till samhället är att besitta utbredd kunskap om djuphavsfiskar, botanik och rymden \textsc{(s.~\pageref{6d5ad1e8996d7ec9d8ac6058649290c0})} (kanske inte bråddjupa kunskaper om det sistnämnda utan mer en förmåga att beskriva hur ball den är).

}

\small{
\textbf{Torbjörn}
\label{c3e6fb6fb2b655457597f063bd9392e8}
 är en punkare från Umeå som älskar rollspel, Bathory, kir \textsc{(s.~\pageref{002e1a6e54da86cabc77fbb474c2df49})} och grisig mat. Alla andra som heter Torbjörn är döpta efter honom, som en eloge.

}

\small{
\textbf{Torbjörn rolandssång}
\label{17cab01b9df781d28784558eb964a971}
 är namnet på en av norra Sveriges \textsc{(se Sverige s.~\pageref{b1999637949ed135b2ca03f3a38460cc})} mest obskyra skalder. Namnet antas vara en pseudonym. Skaldens egentliga identitet är okänd, men det spekuleras i att han ska vara en pensionerad arkivarie med svår gikt. Rolandssång har publicerats i tidskrifter som Sydöstra Västerbottens Skaldjursblad med sin dikt \textit{Ode till langusten} såväl som i den Kramforsbaserade tidningen Räknytt där hans poem \textit{Äta kräfta på rim} orsakade en månadslång session av aggressivt formulerade insändare.
 Poetens senaste publicering var en inspelning av Ainbusk Singers klassiska svensktoppsplåga \textit{Jag mötte Lassie}, men med ny text, där refrängen går \textit{\quotetext{Jag åt en langust, jag åt en langust, å nu mår jag som en jävla skit}}. Rättsprocessen som inleddes så fort låten offentliggjordes pågår fortfarande.

}

\small{
\textbf{Torgny Mogren}
\label{b1054c39d0e166d40537c9d15cbe612e}
 är en svensk skidåkare från Hällefors \textsc{(s.~\pageref{e144fd5ba5ee4d7c395f18c9b1a4cd1f})}. Han har bland annat ett OS-guld och fyra VM-guld hemma i prisskåpet, varav ett från Oberstdorf. Under ett träningsåk 7 februari 1995 bröt han ryggen och sitter sedan dess i rullstol. Idag är han verksam inom VVS-branschen.

}

\small{
\textbf{Torsdagar}
\label{42daf19d5e9b792612b2038788e7ded1}
 Torsdag är den fjärde dagen i veckan. På grund av sin mittplacering är torsdagen den första dag som slumpas vid undantagstillstånd.


 Historiska händelser på torsdagar:

 Robinson Kruse och Fredag kör en flotte på grund och döper händelsen till Skärtorsdag.

 Benny Bus \textsc{(s.~\pageref{a8289efd495ef49dbe0225de89f7f019})} skolvecka börjar.

 Sommartiden infaller år 1972.

 Sagoman har namnsdag.

}

\small{
\textbf{Torsten Bengtsson}
\label{2a474e10dd4038864483b75c61e191ce}
 Torsten Pippi Bengtsson (1914-1998) var en centerpartistisk politiker. De två politiska frågor han var kände mest passion för, var den osannolika kombinationen att alla burfåglar skulle släppas fria (varför han kallades \quotetext{Pippi}) och strikt nykterhet. Bengtsson propagerade bland annat hårt för den drakoniska alkohollagsskärpning som i och med sitt instiftande 1977 förbjöd matvarubutiker att sälja mellanöl i Sverige. Trots att hela grejen med fåglarna är helt vansinnig, minns eftervärlden honom mest tack vare hans hårda avståndstagande från alkohol. Detta i och med att den dionysiske skalden Eddie Meduza besjungit Bengtsson i flera av sina mest älskade låtar, däribland \textit{Torsten hällde brännvin i ett glas åt Karin Söder}, \textit{Ursäkter} och \textit{Mera brännvin}, som ska vara tillägnad Torsten Bengtsson.

}

\small{
\textbf{Traditionell finsk medicin}
\label{26ffcef3a4e2b5a57ca68b21d65ccab9}
 Läran om att dricka läkarsprit, bada bastu och amputera med morakniv.

 \quotetext{Där varken bastu, brännvin \textsc{(s.~\pageref{ff49ececa32cff978496a39635496f46})} eller tjära hjälper finns ingen bot}


 HEAD2:  Se även:
 
 Traditionell kinesisk medicin \textsc{(s.~\pageref{b910794bdd09f8d29cf4b7e9a3fe966e})}.

}

\small{
\textbf{Traditionell kinesisk medicin}
\label{b910794bdd09f8d29cf4b7e9a3fe966e}
 Läran om att göra potensmedel av udda delar på utrotningshotade djur \textsc{(s.~\pageref{24a427a5537c2c8918cfa213ae099a74})}. Till exempel myrkottsfjäll, noshörningshorn, tigergalla och hajfena.


 HEAD2:  Se även:
 Traditionell finsk medicin \textsc{(s.~\pageref{26ffcef3a4e2b5a57ca68b21d65ccab9})}

}

\small{
\textbf{Traditionell litteraturvetenskap}
\label{6a338db129a9884dd517bc1d860e10dd}
 är en vetenskap som både flitigt praktiseras och sörjs av borgarbrackor på universitet runtom i vårt avlånga land. Traditionell litteraturvetenskap går i mångt och mycket ut på att skriva biografier om, i huvudsak manliga men nu även kvinnliga, författare med överklassbakgrund. Dessa författare, tycker den klassiske litteraturvetaren, skiljer sig mycket från andra författare genom att vara överlägset bättre. Eftersom överklassen nu för tiden i huvudsak istället för att skriva böcker ägnar sig åt att dricka så mycket alkohol att de i ett svagt ögonblick uttalar sina sympatier för nazistorganisationer för att i nästa stund kissa på sig tycker den traditionelle litteraturvetaren innerst inne att det var bättre förr. Till skillnad från många samtida kollegor skrev dessa författare till den traditionelle litteraturvetarens stora glädje långa böcker om diffusa ämnen så som den uppfyllande känsla av gemenskap som huvudpersonen finner i nationens historia, lite mindre upplyftande grejer så som nostalgi och religiöst tvivel, samt olika former av hyllningar till kung, fosterland, prästerskap och adel. \quotetext{Ååå va bra såna böcker är,} skriver den traditionelle litteraturvetaren, fast på ett mer invecklat vis och inte i en mening utan i två-tre band om ca 300 sidor var. Den traditionelle litteraturvetaren bara \textit{måste} få veta mer om författarens barndom, var författaren gick på lokal och vem han/(numera)hon umgicks med. Den traditionella litteraturvetenskapens möte med liberal feminism har gjort att det sedan ett par decennier är helt okej att skriva biografier om överklasskärringar och kalla dessa böcker för avhandlingar, varpå författaren till dem blir doktor och anställd på ett universitet.

}

\small{
\textbf{Trans}
\label{4738019ef434f24099319565cd5185e5}
 \textit{Trans} (1982) är en skiva av den kanadensisk-amerikanske rockartisten Neil Young. Skivan har ett väldigt digitalt, cyborg-aktigt ljud och på fem av de nio låtarna används en vocoder, eller röstlåda som vi säger på Nissepedia \textsc{(s.~\pageref{62400dadecd90cb5cd39062abe5a3e4a})}, som transformerar Youngs röst till ett slags robotröst. Detta ska ha föranletts av två saker:

\begin{enumerate}
\item Youngs son led av en CP-skada och var oförmögen att tala, varför Young valde att experimentera med förvrängt tal.
\item Youngs tyckte att det var ball.
\end{enumerate}

 Skivomslaget föreställer naturligtvis ett slags science-fictionartat urbant landskap, komplett med framtidsbilar \textsc{(se bil s.~\pageref{b3188f47d2eac7efc3f1258dc673a9fe})} och flygande farkoster.
 Det har spekulerats i varför skivan är så hemskt dålig. Skivbolaget Geffen påstod i en rättstvist att Young medvetet skulle ha producerat osäljbar musik, medan Youngs tillskyndare har påstått att \textit{Trans} innebar ett slags ironisk kommentar till samtida musik och alla dess många brister.

}

\small{
\textbf{Transistor}
\label{aaa0d78af49a2fbc2f7ad8fbb11de1aa}
 Den felande länken mellan proletariatets bojor och Jan Stenbecks byxmått.

}

\small{
\textbf{Transkrove}
\label{1188281a09fb681b922e45663e5ffc4b}
 En transkrove är en svanskrove \textsc{(s.~\pageref{e543ead268a283bfdb5ea638d6cca4a2})}, minus svanen, plus en trana.

}

\small{
\textbf{Trasmattans dag}
\label{657be5ffddeb957c97682755edcffe6e}
 är en tillställning som firas varje år i samband med Kristi himmelsfärd. I Fagersta \textsc{(s.~\pageref{008e08fd02751800f729d6fa6f75a857})} firas högtiden som mest på hembygdsgården med utställning av trasmattor och inte så mycket annat [http://fagersta-posten.se/mingel/beratta/fagersta/1.1238233-trasmattans-dag-i-fagersta]. Det är oklart om det finns något historiskt samband mellan trasmattor och Kristi himmelsfärd, vilket man lätt kan tro eftersom de firas samtidigt. Kanske är det bara så att kristendomen i vanlig ordning förlagt sitt firande samtidigt för att parasitera på en redan etablerad högtid.

}

\small{
\textbf{Trea}
\label{6f94fdf535ab2e21147ea40ea920ca75}
 En trea är en bostad med tre rum och kök.


 Se även: Nolla \textsc{(s.~\pageref{21cdce774e105d593f6ea43014412b28})}, Etta \textsc{(s.~\pageref{ba48f6c4097b7fc25ca11f1e544842d7})}, Tvåa \textsc{(s.~\pageref{84fcc0494ecf9f5af79fcd9bed184a9a})}, Fyra \textsc{(s.~\pageref{7bdb5385ce8e0b1cbc7c15b1d71e8e7d})}, Femma \textsc{(s.~\pageref{d974e0811fe7a4d49a9062d33b66a88d})}, Sexa \textsc{(s.~\pageref{4b1fabe53857b0a2ace6ae22008fe13e})}, Sjua \textsc{(s.~\pageref{e7bf63fa6d0d29bd89c23f833b979a15})}, Åtta \textsc{(s.~\pageref{6fa68b0d02ec525fa72a51c13e5e3ed1})}, Nia \textsc{(s.~\pageref{04a481486dd84d7c8bfdfc89d38136a6})}, Tia \textsc{(s.~\pageref{e7292d5ba58672ce7f6fc3c0b646ab63})}, Elva \textsc{(s.~\pageref{788bd84addbcf8f1767869759d4a2ad9})}, Tolva \textsc{(s.~\pageref{75e2490604087d3d303b09a98803a16b})}.

}

\small{
\textbf{Trepipsproblem}
\label{ddfa7edb7b4169a1dc8a32b1a8ad9611}
 är problem som kräver sådan djupgående och vidsträckt tankeverksamhet att problemlösaren hinner röka hela tre välstoppade pipor tobak innan en lämplig åtgärd kan skönjas.

}

\small{
\textbf{Trevlig}
\label{e793711eaa3d54d216213f059218f834}
 Urban medelklasslang för halvtråkig. Nedan följer exempel på bruk av ordet i vardagssamanhang.

 - ''Trevlig grillfest Lothar! \textsc{(se Lothar s.~\pageref{a2c85ab64a0c7a0197c17fd3eefe47d5})} Att servera ljummen bärs var ju en riktig home run.
 - \textit{Du, Gunborg \textsc{(s.~\pageref{9e29dc34382963ae7d76a742e98637a4})}, din nya karl, Petter \textsc{(s.~\pageref{3f3b423295473405f1eda282c3531e75})}, han verkade ju riktigt trevlig! Intresserad av bildäcksproduktion? Hade jag aldrig anat.}
 - \textit{Jörgen \textsc{(s.~\pageref{92592dfe96a3cac8ff3eae81584f9b42})}, det här amerikanska rockbandet du hade på kassett var ju väldigt trevligt. Groovy, liksom.}
 - \textit{Din konstsmak är ju genial, Hillevi \textsc{(s.~\pageref{758961a2bf96af710999ea02164fb582})}. Se på den här väldigt dramatiska porslinsdalmatinen till exempel. Jättetrevlig!}

}

\small{
\textbf{Trilobit}
\label{e27534bc9e236c7d40371dc51cac67be}
 Allmän benämning på pusselbitar som ramlat ner från bordet eller tappats bort på annat sätt.

}

\small{
\textbf{Trivselskrot}
\label{6235563333e8dc26c9fc54e9e70c85ed}
 är skrot som man samlat på gården för tivselns skull i första hand och för att det kan komma till användning i andra hand. I tredje hand har man samlat det där för att öka möjligheterna för att någon förbipasserande tvillingskäl ska knacka på och fråga om ens rostiga bromsok eller kofångare till ens skrotade Saab 99 är till salu. Då går man ut på gården, står framför skroten och försöker komma fram till om man har någon gemensam bekant. Detta är en viktig komponent i affären så det är viktigt att inte ha bråttom.

 Exempel på saker som kan ingå i ens samling av trivselskrot:
 \begin{itemize}
 \item Nedslitna bil- och traktordäck
 \item Pusch Dakota utan hjul
 \item Snöplog
 \item Brädstumpar och lastpallar att palla upp saker med
 \item Cykelkedjor och kättingar
 \item Defekt utombordare
 \item Spillolja i plåttunna
 \item Rått styre till en motorcykel
 \end{itemize}

}

\small{
\textbf{Trocadero}
\label{a2dc54a81d5e68d69e177470f802c0c1}
 Denna gyllene brygd är de nordligaste länens livsblod. Dess smak är svår att beskriva i ord, men enligt innehållsförteckningen borde den smaka både äpple och apelsin,~samtidigt~. Whoa. Den görs av en uppsjö av tillverkare och de som är bäst att köpa av är:
 \begin{itemize}
 \item Nyckelbryggerier
 \item Vasa bryggeri
 \item Zeunerts
 \end{itemize}
 Dessa tre har i alla fall sina bryggerier norr om Dalälven, men Nyckel är bäst då den kommer från den förnämliga staden Luleå \textsc{(s.~\pageref{3cefb5ac35187749592f1ebb25472b99})}.

}

\small{
\textbf{Trollkull}
\label{900bbb023078fe24707bd4c6f8f46f95}
 är en idrott som ännu inte fått OS-status, men förhoppningsvis blir det ändring på det till OS 2018 (mer information kommer). Idrotten går ut på att ett tjog \textsc{(s.~\pageref{93e1254c6b6d02b89439cbea1926a4de})} förpubertala 10 \textsc{(se tia s.~\pageref{e7292d5ba58672ce7f6fc3c0b646ab63})} rusar omkring och skriker i en jumpasal. Alla har ett tygband nerstoppat i bak på byxorna. En av 10-åringarna försöker desperat jaga klassens tjockis in i ett hörn för att där rycka åt sig tjockisens band och på så vis kulla honom/henne. Nu är det två \textsc{(se tvåa s.~\pageref{84fcc0494ecf9f5af79fcd9bed184a9a})} som kullar och ju fler troll \textsc{(se lakritstroll s.~\pageref{7d95faffde1363bedb69dce2da3947b5})} de kullar desto mer spännande blir det, för sisten kvar med bandet bestämt nerkört, eller i extrema fall fastknutet, i underkläderna vinner nämligen.

}

\small{
\textbf{Trollprutt}
\label{5f804331935073a494b581a9c7ba35db}
 är ett ord som hädanefter kommer att ersätta det uttjänta och lite rustika begreppet \quotetext{brakskit}. Även om orden syftar på samma sak, för det tidigare till skillnad från det senare tankarna till folktro och mytologi. Det tillåter gamla människor att berätta \textsc{(s.~\pageref{4f84e02a70b3bbb57fa83da31bf7a16f})} för yngre släktingar att man förr i världen brukade säga att \quotetext{det går troll i området}, då det karaktäristiska ljudet hörts och en vägg av rötstank slår emot ynglingen. Likaså kan man roa utlänningar som kommer på besök med att redogöra för betydelsen av detta begrepp och på så vis föra in samtalet på lokala seder och bruk och på nordbornas traditionella föreställningsvärld.

}

\small{
\textbf{Trollpunk}
\label{5e806ae90a53e9328e1e467a4d7b1b37}
 På det gamla hederliga 90-talet visste inte så många i Sverige \textsc{(s.~\pageref{b1999637949ed135b2ca03f3a38460cc})} att det fanns undergroundmusik. Därför kallades allt som inte var Nordman för antingen punk eller synth. När så pop-rockbandet \textsc{(se pop-rock s.~\pageref{d559954a05239a86feae3a0d3216cf56})} Dia Psalma kapade åt sig marknadsandelar genom att sjunga om troll och näcken som rövar bort jäntor eller att trilla och drunkna i en bäck, lades grunden för musikgenren trollpunk. Reaktionerna lät inte vänta på sig. Dia Psalmas debutskiva \textit{I Gryningstid} såldes i 10.000 \textsc{(se Fyrtiotusen miljarder s.~\pageref{c2160bffc9c5ca88e77204672e62e489})} exemplar, och blev en ekonomisk succé som direkt ledde till att Per på Birdnest tog skivbolagets kassaskrin och satte iväg över Östersjön med den erfarne Prof. Etienne \textsc{(se Användare: Prof. Etienne s.~\pageref{a9878d2280e5a39becac8f73d113df91})} som guide och mentor. De två vännerna tog sig ner till skatteparadiset San Marino via Baltikum och Östeuropa och har sällan setts till sedan dess. Detta slag mot trollpunkenscenen gjorde den dock bara starkare, eftersom attityden hos kidsen alltid varit att man inte deltar i den för pengarnas skull, utan för att uttrycka sig i sång och musik om troll och andra oknytt.

}

\small{
\textbf{Trotskij}
\label{71204085b9cdbdd914506625c6169e15}
 \textbf{Lev Trotskij}, ursprung okänt, är en av de 31 gestaltfunktionerna i den ryske formalisten Vladimir Propps \textsc{(se Propp s.~\pageref{8bde165d41e3a0363954a27cc7164e2e})} dekonstruktion av ryska folksagor. Enligt Propp återfinns 31 gemensamma drag, gestaltfunktioner, kring vilka alla ryska folksagor är uppbyggda. Propp menar att en Lev Trotskij kan återfinnas mot slutet av alla berättelser, strax efter att hjälten vunnit sitt erkännande.


 HEAD2:  Propps 31 gestaltfunktioner
 (Variationer i ordningsföljden kan förekomma men detta är den vanligaste)
 \textbf{Ursprungssituation - familjens medlemmar presenteras; hjälten presenteras}
 1. \textbf{Bortavaro} - en i familjen håller sig borta
 2. \textbf{Förbud} - förbudet riktas mot hjälten (kan upphävas)
 3. \textbf{Kränkning} - förbudet kränks
 4. \textbf{Spaning} - skurken söker skaffa sig information
 5. \textbf{Utlämning} - skurken får information om offret
 6. \textbf{Bedrägeri} - skurken söker lura offret
 7. \textbf{Delaktighet} - offret luras
 8. \textbf{Skurkaktighet} - skurken vållar en familjemedlem skada; familjemedlem saknar/önskar något
 9. \textbf{Förmedling} - olyckan blir känd; hjälten sänds efter
 10. \textbf{Motstånd} - hjälten går med på, bestämmer sig för, att göra motstånd
 11. \textbf{Avresa} - hjälten lämnar hemmet
 12. \textbf{1:a givarfunktionen} - hjälten testas, får en magisk agent eller hjälpare
 13. \textbf{Hjältens reaktion} - hjälten reagerar på agenten eller givaren
 14. \textbf{Mottagande av agenten} - hjälten får bruk för agenten
 15. \textbf{Förändring i rummet} - hjälten börjar sökandet
 16. \textbf{Kamp} - hjälten och skurken i direkt strid
 17. \textbf{Stämpling} - hjälten stämplas
 18. \textbf{Seger} - skurken besegras
 19. \textbf{Undanröjning} - ursprungliga olyckan eller bristen undanröjs
 20. \textbf{Återvändo} - hjälten återvänder
 21. \textbf{Förföljande, jakt} - hjälten förföljs
 22. \textbf{Räddning} - hjälten räddas från förföljarna
 23. \textbf{Icke igenkänd ankomst} - hjälten anländer till hemmet eller annan plats utan att kännas igen
 24. \textbf{Ogrundade krav} - falsk hjälte framför ogrundade krav
 25. \textbf{Svår uppgift} - hjälten föreläggs en svår uppgift
 26. \textbf{Lösning} - uppgiften löses
 27. \textbf{Igenkänning} - hjälten känns igen
 28. \textbf{Trotskij} - den falske hjälten eller skurken avslöjas
 29. \textbf{Omgestaltning} - hjälten får ett nytt utseende
 30. \textbf{Straff} - skurken bestraffas
 31. \textbf{Giftermål} - hjälten gifter sig, bestiger tronen

}

\small{
\textbf{Trotta}
\label{918b2980ffb5f16acf768fa89f71021b}
 Att trotta är ett annat ord för att infiltrera en organisation med en ganska slarvigt dold agenda.
 Trotte = trotskist (skällsord i vissa kretsar sent 60- och tidigt 70-tal)

 HEAD2: Exempel
 Trotten söker medlemskap i en hemslöjdsförening, schweiziskt institut, arbetarkommun eller varför inte ett politiskt ungdomsförbund. Trotten är till en början en helt vanlig medlem och är inte så högljudd utan bara flyter med. Sen plötsligt en dag utbrister denne något i stil med \quotetext{Hur kan vi få den här slöjdföreningen att jobba för införandet av socialismen i hela världen?} och går vidare med att svamla om massmöten och namnlistor och frontorganisationer i Kamerun och fan och hans moster. Efterspelet kan resultera i:
 \begin{itemize}
 \item Trotten lyckas med sin kupp och får organisationen att bli liiite mer vänster
 \item Någon handlingskraftig individ greppar trotten i svångremmen och skickar denne med huvudet före över tröskeln.
 \item De övriga lämnar organisationen med trotten kvar som trots detta tjuras med att det är en massorganisation.
 \end{itemize}


 HEAD2:  Att motverka trotteri
 Enligt säkra uppgifter är en gammal hederlig ishacka det bästa redskapet. Den tidigare något aggresiva killen Adolf H (verksam under 2:a världskriget) hade egna sätt att hantera detta. Senare amerikanska presidenter har på olika sätt försökt minimera förekomster av trottar, ibland har man kanske gått lite långt eller över gränsen till andra länder både bokstavligt och fysiskt. Oftast med en större mängd vapen och diverse svepskäl för att få vistas i dessa land och skjuta på alla med T-shirt och trotteglasögon \textsc{(se trotteglasögon s.~\pageref{bb66f62765f1c06d956997e461cc383c})}

}

\small{
\textbf{Trottebrillor}
\label{5809521f1d31d1c1dc114866da846bb4}
 Trotteglasögon \textsc{(s.~\pageref{bb66f62765f1c06d956997e461cc383c})}
\begin{enumerate}
\item OMDIRIGERING Trottebrillor
\end{enumerate}

}

\small{
\textbf{Trotteglasögon}
\label{bb66f62765f1c06d956997e461cc383c}
 är en sorts glasögon som bärs av trottar \textsc{(se Trotta s.~\pageref{918b2980ffb5f16acf768fa89f71021b})}. Trotteglasögonen saknar ofta bågar på undre halvan av glasen, eftersom det är småborgarsossigt. Se även kommunistglasögon \textsc{(se Kommunistglasögon s.~\pageref{1bc58f6f6a934c05a63add653dbeadf0})}.

}

\small{
\textbf{Trumman}
\label{beb878ae34283709d572eb4222482d9c}
 är en märklig mojäng dit saker oväntat försvinner. Till utseendet har den formen av en bastrumma (även kallad \quotetext{kaggee}). Dess ursprung är okänt men den första kända dokumentationen av trumman i modern tid börjar med att den faller i Stigs, son av Karin och Folke, ägo. Till en början uppförde sig trumman som vilken trumma som helst men plötsligt började märkliga saker hända. Saker som länge varit spårlöst borta började oförklarligt dyka upp i trummans innandöme. Rimliga förklaringar saknas men faktum kvarstår att trumman nästan alltid innehåller det man tappat bort. Ett tag var trumman själv borta och det spekulerades i att den försvunnit till sitt eget innandöme. Efter en tid återfanns den oförklarligt i ett uthus bland en massa skräp \textsc{(s.~\pageref{75f1a5320951ea0dd9aa3c0eaba2c2c7})}. Trummans inverkan på universums balans kommer aldrig helt kunna förklaras.


 Beroende tester har visat att man kan hitta saker i trumman betydligt oftare än mediumet Solveig i Aftonbladet söndag kan.

}

\small{
\textbf{Tryggves tröja}
\label{cd8caa11ba57f4513273095b8bd4c622}
 är ett plagg som just nu upplever något slags renässans på det norra halvklotet.

 Under oklara former fann den en ny ägare under senvåren 2013 och har på förhållandevis kort tid upplevt ungefär fyrtiotusen miljarder \textsc{(s.~\pageref{c2160bffc9c5ca88e77204672e62e489})} fyllor. Man kan därför lätt förledas att tro att den numera bärs av Boris Jeltsin.
 Herr Jeltsin är dock död och begraven sedan länge och har adressändrat till Nangijala, således kan han uteslutas ur ekvationen.

 Tryggves tröja har deltagit i de mest skilda aktiviteter, såsom svampplockning, 97-fest och filminspelning.
 Gemensam nämnare för dessa aktiviteter är en promille på 2 eller högre.
 Tryggves tröja har haft en positiv inverkan på den nye bärarens fortplantningsmöjligheter och är därför strategiskt viktig i kampen för att föra värdefulla gener vidare.
 Det har visat sig möjligt att bära denna tröja fem dagar under loppet av en vecka utan att skämmas.

 Framstående etnologer menar på att svenska folket skulle upphöra med särskrivningar om alla hade varsin sådan tröja.
 Tyvärr är endast ett exemplar av Tryggves tröja hittills känt för mänskligheten.
 Många sätter sin tilltro till att det inom 3-4 år med ny teknik ska vara möjligt att klona detta enda exemplar för att få en bättre värld men det är dumt att ha för höga förväntningar, man blir bara besviken.

}

\small{
\textbf{Träbit}
\label{d27020124419b69173fc321fe29e1d4b}
 Mindre del av ett träd, som förutom att barken/nävret avlägsnats, sönderdelats i mindre enheter som brukar benämnas sparrar, plank, bräder, läkt eller ribb. För att bli en träbit måste tidigare nämnda enheter dessutom kapas på längden till max 1 meter.

}

\small{
\textbf{Träbjörn}
\label{3fa1e4f2d866814bf69e29479762b85a}
 En träbjörn är en björn byggd utav trä. För att en björn ska klassas som en träbjörn måste dess beståndsdelar främst bestå av trä. Till yttermera visso ska dessa träbitar vara ihopförda på ett sätt som får dem att tillsammans likna en björn.

 Världens största träbjörn återfinns på torget i Sveg.

}

\small{
\textbf{Träd, Gräs och Stenar}
\label{82a271b29bea1b3fd0073fe6668179bd}
 är ett svenskt rockband bildat 1969. Bandet bestod då av Bo Anders Persson (gitarr),  Arne Ericsson (elcello), Torbjörn Abelli (bas) och Thomas Mera Gartz (trummor). Gruppen räknades till proggrörelsen, men saknade den tydliga politiska inriktning som kännetecknade delar av denna rörelse. Progressiv rock med lite politisk ton var det gruppen sysslade mest med under 1970-talet. Basisten Torbjörn Abelli avled den 11 augusti 2010. År 2012 gjorde bandet ett musikaliskt lappkast \textsc{(s.~\pageref{9dd6698c53a9d42abffb80092f739ae2})} och började införa sång i alla sina låtar. Alla true proggare gråter över detta.

}

\small{
\textbf{Träffningen vid Ratan}
\label{4c1c24984f855772452c7149d148d89d}
 var ett slag under finska kriget som rasade mellan 1808-1809. Slaget, eller snarare det fredsavtal som följde har en plats i allas våra hjärtan av två anledningar.
\begin{enumerate}
\item Sedan Träffningen vid Ratan har ingen krigshandling officiellt ägt rum på svensk mark.
\item Genom fredsavtalet efter slaget slapp vi Finland \textsc{(s.~\pageref{631d44eaa1254ff71a1e11ba021d1266})}, som ryssarna istället åtog sig att fostra och förvalta.
\end{enumerate}
 HEAD2: Se även
 Finland \textsc{(s.~\pageref{631d44eaa1254ff71a1e11ba021d1266})}
 Ålandskrisen \textsc{(s.~\pageref{967c6b3cd72e6de161ca9e911779795a})}

}

\small{
\textbf{Träskaft}
\label{1ab85ecd859ae682af47bb9334c7dac6}
 Ett träskaft är antingen den del på till exempel yxan \textsc{(se yxa s.~\pageref{bd74f429522c7c1481fbba07187efc6b})} som man håller i, förutsatt att den är gjord av trä, eller en person som är särske \textsc{(s.~\pageref{552a5aad891937bf760fb193900ea140})}.

}

\small{
\textbf{Träskpunkare}
\label{484838b3db1adb135ea74d6fc61e44c0}
 (även kallad containerpunkare, pisspunkare, sumppunkare och grispunkare. Ibland förväxlad med kängpunkare och crustare) är en subgenre till vanliga punkare. Träskpunkaren lever i möjligaste mån på bidrag och sopor och vistas gärna i flock. Den duschar sällan och sköter istället sin hygien genom att bada naken eller vandra runt Kalle anka \textsc{(s.~\pageref{64db68f686a0ca4d9d641061cb3fdf13})} på festival. Den bor i en svart skinnpaj dekorerad med musikgrupper, som alla börjar på prefixet dis-, påmålat med tipp-ex. Den karaktäriseras i övrigt av sina hastiga växlingar mellan svintrevlig och outhärdlig. Vill man lära känna en träskpunkare bör man mata den med folköl eller mäsk. De frodas som bäst i åldern 16-27 och blir sällan mycket äldre än så då den lever efter devisen: live fast - die. Vissa exemplar i fångenskap har i undantagsfall blivit över 50 år.


 HEAD2:  Fortplantning och övervintring

 Under sommarhalvåret söker sig träskpunkaren ut på öppna fält där den plankar in på festivalcampingar för att finna en partner och para sig till allmän beskådan. De brunstiga individerna lockar på varandra genom att spela Anti Cimex genom en kassettbandspelare eller sjunga Onkel Kånkel. Fortplantningen försvåras dock av träskpunkarens obenägenhet att göra skilland på kön och parar sig med lite vad som helst. När vintern kommer söker den sig till socialen eller övervintrar på ett squat i Tyskland.


 HEAD2:  Kända träskpunkare

 Benny Bus \textsc{(s.~\pageref{a8289efd495ef49dbe0225de89f7f019})}
 Nasse
 Ramen (har evolverat vidare)
 Jonsson
 GG Allin


 Länk till träskpunkare fångade på film: [http://www.youtube.com/watch?v=bIdK3jyCBA0]

}

\small{
\textbf{Tråg}
\label{1e0e0470206e0f2baad8e628ba8f770c}
 Ett tråg är ett föremål med försänkning i mitten som man har mat i. Vad som skiljer tråget från andra liknande föremål såsom badkaret, baljan, hon och den igenproppade stuprännan är just att det är avsett att äta ur. Vad som skiljer tråget från andra föremål avsedda att äta ur såsom djuptallriken, bunken, grytan och papptallriken är dess storlek. Ett tråg är nämligen så stort att flera kan äta ur det samtidigt. Vanligtvis är det konstruerat i någon form av trä och har en avlång form. Det behöver inte rengöras så noga utan det räcker att man spolar av det någon gång ibland. På medeltiden \textsc{(s.~\pageref{88cbc30c5b233d97df68b5b041ac0655})} åt alla människor ur tråg men nu för tiden är det mest grisar och kor som gör det. Om man idag ser en människa inta sin förplägnad ur ett tråg är de troligt att denna drabbats av den tyska mustigheten \textsc{(s.~\pageref{682ccd5fdc3aff0c97e8845c3d6b6ca8})}.

}

\small{
\textbf{Tsygan}
\label{4da32eca7858fd215a250d871bf7fa7b}
 (Цыган, ryska för zigenare) var den första hunden \textsc{(se Alkisschäfer  s.~\pageref{347febbc28041eae88556d2e7ced587b})} i rymden. Många uppger felaktigt att Laika var den första hunden men hennes bedrift låg istället i att vara den första hund att flyga i omloppsbana. Tsygan överlevde sin flygning och landade säkert på sovjetisk mark 29 januari 1951 efter en kortare åktur.

}

\small{
\textbf{Tsygan II}
\label{da919dfb81083059022a634b495dac7d}
 (RC I) var en SS raket som avfyrades någon gång mellan kl00:00 - 01:00 den 11/4-10 på Åkerö i Leksand. Avfyrningen var lyckad med ändå inte med tanke på att den troligt vis exploderade ca 3 m upp i luften, men den flög ändå 3 m.

}

\small{
\textbf{Tuborg}
\label{49bb0f04b9993881c9d9c5b115cc35f0}
 (uttalas  [tˢub̥ɒ:ˀ]) är danska för prosit.

}

\small{
\textbf{Tuff-frysa}
\label{4b6b102a59146e2319539b999b8c571f}
 Att tuff-frysa är ett väldigt framkomligt sätt att vinna resepekt på skolgården och busshållsplatsen och går i korthet ut på att man genom den svenska sub-arktiska vintern envist vägrar ha vantar och mössa samt att man har jackan oknäppt. Knepet är vanligt i åldrarna 7 \textsc{(se sjua s.~\pageref{e7bf63fa6d0d29bd89c23f833b979a15})} år och bland samhällsgruppen hockeykillar \textsc{(se kukenkillar s.~\pageref{3cf0284428a6f396e261986d14927a1b})} som röker \textsc{(se cigg s.~\pageref{2bcc66e1261fa5199a4f4decf2720ef5})} utanför sportbarer. Ibland kan det, om man är ett barn \textsc{(s.~\pageref{5dfcc0aab2f3db925b2d51ba73e48946})}, vara nödvändigt att använda sig av en hel del list och uttänkta strategier för att lura sina föräldrar som, som vanligt, ska hålla på och tjata om att man måste ha mössa och vantar \textit{et cetera}, men det är det ofta värt för när man väl börjat närma sig ögonblicket då Sussi i 5b tycker att man är tuff finns det egentligen ingen återvändo. Just på det viset, och också för att det suger att frysa som fan dag ut och dag in, är situationen lite som för de själar Dante möter i inferno i \textit{Den gudomliga komedin}.

}

\small{
\textbf{Tumring}
\label{443f46117e049146bc200cab02ace4ac}
 En modeacessoar som inte för något gott med sig. Visa föräldrar ger sina tonårsdöttrar råd så som: \quotetext{har han tumring-då må du spring}

}

\small{
\textbf{Tuna}
\label{2bf93a8a979420ff77b32fab0751cad2}
 En gång ett tufft ghettoliknande område, numera mysljuvlig \textsc{(s.~\pageref{7c615cb1629d4d8df30d60d2e47ea6d7})} idyll för barnfamiljer och utbytesstudenter \textsc{(s.~\pageref{397699f3732b0c22f3c532a111697539})}.
 HEAD3: Tuna i populärkulturen
 \begin{itemize}
 \item \quotetext{Grå grönska} på Rekyls första platta.
 \item \quotetext{Tunaskolan} MABD
 \end{itemize}

}

\small{
\textbf{Tung industri}
\label{454e5e8cb27bed118f0a6a1a01a6e6a9}
 Skogs-, gruv- och stålindustri. Marx favoritstudieobjekt.

}

\small{
\textbf{Tunna}
\label{00f1b109163e2c7e424e60cda2354c55}
 En tunna är en lite vag måttenhet som kan anspela på allt mellan 100 och 150 liter.

 En tunna kan också vara ett klädesplagg för människor som spelat bort alla sina pengar på hästar \textsc{(se häst s.~\pageref{b4c608370b339da095c5f8db7fab0945})}.

 Att hoppa i galen tunna är att göra något vansinnigt, och sägs härstamma från Diogenes, en av de gamla grekerna \textsc{(s.~\pageref{4a5fb3d6ce79b5ff43b33f8f7d843672})}.

}

\small{
\textbf{Tunnland}
\label{4179c4ac28cd06dcdaefbb02d6db3599}
 Ett tunnland motsvarar den landyta som kan besås med en tunna \textsc{(s.~\pageref{00f1b109163e2c7e424e60cda2354c55})} utsäde, en halv hektar.

}

\small{
\textbf{Tunntarmen}
\label{3233745e68b1703eb55058f4e1be2126}
 Enligt Nisse kommer \textbf{tunntarmen} innan  magsäcken i människans matspjälkningssystem. Sedan kommer tjocktarmen följt av ändtarmen. Någonstans bland alla dessa olika sorters tarmar finner man den övre magmunnen \textsc{(s.~\pageref{b0fbb0780611129ae5fc27c88d23d8f3})}.

}

\small{
\textbf{Tura}
\label{32f46c73bef44a331691310a259b0012}
 Att tura är att åka fram och tillbaka över Öresund mellan Helsingborg och Helsingør \textsc{(se Danmark s.~\pageref{5331d7fd27772396f412a5b6d19bad44})} i syfte att i första hand bli full, i andra hand att äta mat. Det är inte mycket att se längs denna farled, förutom Hamlets \textsc{(se Hamlet s.~\pageref{ea3596139530b2abe7089082ab57ecbd})} slott.


 HEAD2:  Se även

 Dieselbil med lastgaller \textsc{(s.~\pageref{73b1f975c67393304ff101482965163c})}

}

\small{
\textbf{Turtlestestet}
\label{c315884212135c7be43cfb8c67166372}
 är ett psykologiskt diagnosverktyg för att fastställa mustigheten i en persons karaktär.
 Testet går i korthet ut på att man monterar in en bärbar hydrograf i munhålan på patienten för
 att sedan visa denne valda utdrag ur en vanlig svensk standardpizzameny varpå man mäter flödet av snålvatten som rinner till för de olika pizzasorterna.

 Tvingas man använda salivsug redan vid margherita så kan man avluta testet och direkt utesluta all form av mustighet.
 Stannar man på Vesuvio eller en klassisk Capricciosa finns det fortfarande hopp om försöksobjektet i fråga bara är ett barn.
 Självfallet kan patienten ofrivilligt räkna in förbättrande omständigheter så som, bearnaise, vitlökssås eller andra smakförstärkare vilket kan ge s.k. metodfel.
 Högre mustighet finner man hos de som går i spinn på t.ex. en calskrove \textsc{(s.~\pageref{84ff54e779ee49fdad21e17c20f14453})}, hawaii \textsc{(se Hawaii-pizza s.~\pageref{742e4954c36e42931521b0a417511c7c})} eller någon av de lokala specialpizzorna.

 För att kvantifiera patientens mustighet använder man förutom sunt förnuft även en rad olika parametrar som sätts in i en hyperkub för att beräkna mustighetskoefficienten.
 Parametrar kan t.ex. vara
 \begin{itemize}
 \item Antal ingredienser
 \item Antal ingredienser som endast återfinns i just denna pizza på menyn.
 \item Sannolikheten att en eller flera ingredienser endast återfinns i just denna rätt.
 \item Transportsträcka för ingrediensen från skördeområde.
 \item Om pizzan innehåller ingredienser från både växt-, djur- och mineralriket.
 \end{itemize}

 \textbf{Bakgrund}
 Det är sedan länge känt att Teenage Mutant Ninja Turtles \textsc{(s.~\pageref{fd9ccf7b23fd53b8c3bb91065ab585ee})} var föregångare i kategorin postavantgardistisk gastronomi och då framförallt Michelangelo (Den orangea med nunchucks).
 Michelangelo var så framstående att han aldrig tog en pizza rakt från menyn utan komponerade själv sin pizza på plats i ett stream of consciousness (en. inre monolog).

}

\small{
\textbf{TV-cirkeln}
\label{f0b13dfb57d6720218930869f99f9793}
 Teveprogram som visades på Sveriges \textsc{(se Sverige s.~\pageref{b1999637949ed135b2ca03f3a38460cc})} Television varje torsdag efter \quotetext{Rederiet} åren 1992-2002. Programkonceptet köptes av tyska Süddeutsche Rundfunk som redan i slutet av 70-talet experimenterat med en fusion mellan pseudointellektualism, bokcirkel och television. Den tyska förlagan sändes mellan åren 1978 och 2011 med titeln \quotetext{Der magische Kreis}. Där hyllades samstämmigt allt från den senaste Ilsa-filmen till ett tidigt Doktorn i dalen-avsnitt. Den svenska \quotetext{TV-cirkeln} kretsade mycket kring cirkel-deltagarnas identifikation med karaktären Robert \quotetext{Raspen} Torstensson. Ett särskilt minnesvärt avsnitt är hur TV-cirkelprogramledaren Arne Tammer grät då Raspen åkte på både syffe och gånne under en natt i Helsingfors. TV-cirkeln lär också ha varit ytterst pådrivande till både Peter Falcks Augustpris 1996 och resandet av en staty i Mariehamn föreställande Torbjörn \quotetext{Joker} Jonasson.

}

\small{
\textbf{Tvåa}
\label{84fcc0494ecf9f5af79fcd9bed184a9a}
 Slangord för anus.


 Se även: Etta \textsc{(s.~\pageref{ba48f6c4097b7fc25ca11f1e544842d7})}, Trea \textsc{(s.~\pageref{6f94fdf535ab2e21147ea40ea920ca75})}, Fyra \textsc{(s.~\pageref{7bdb5385ce8e0b1cbc7c15b1d71e8e7d})}, Femma \textsc{(s.~\pageref{d974e0811fe7a4d49a9062d33b66a88d})}, Sexa \textsc{(s.~\pageref{4b1fabe53857b0a2ace6ae22008fe13e})}, Sjua \textsc{(s.~\pageref{e7bf63fa6d0d29bd89c23f833b979a15})}, Åtta \textsc{(s.~\pageref{6fa68b0d02ec525fa72a51c13e5e3ed1})}, Nia \textsc{(s.~\pageref{04a481486dd84d7c8bfdfc89d38136a6})}.

}

\small{
\textbf{Twitter}
\label{b73c2d22763d1ce2143a3755c1d0ad3a}
 är en så kallad mikroblogg. Den är mikro för att den, till skillnad från typ en wordpressblogg, kan publicera högst 140 tecken åt gången. Vad man ska ha twitter till är omdiskuterat. En del menar att medier som twitter, med sin otroliga hastighet, kan hjälpa till att ställa makthavare mot väggen, eller att hjälpa till att stärka det civila samhället genom ökad dialog mellan medborgare och politiker. Andra, till exempel Eran Fisher, menar att såna idéer är helt uppåt väggarna bananas \textsc{(s.~\pageref{ec121ff80513ae58ed478d5c5787075b})} och att medier som twitter endast hjälper till att legitimera en ökad implementering av nyliberalism i samhället. Utan att sälla sig till endera sida i den diskussionen, vill Nissepedias redaktion ändå göra ett inlägg i debatten i form av följande bild. Det här är poängen med twitter:

}

\small{
\textbf{Tyra}
\label{e33c5b99ea077587bec01dc69f40c565}
 är ett namn man kan ge spolformade katter som älskar att busa.

}

\small{
\textbf{Tysk toalett}
\label{da70822255031d2f882278fd6080bb5f}
 En vanlig toalett har, som bekant för de flesta på 10-talet, ett vanligt hål som ens träck ramlar ner i och ofta döljs bitvis av. I Tyskland \textsc{(s.~\pageref{b1b58da783b6d5fa090f3015f1889869})} däremot är det standard praxis att ens avfall hamnar på en liten porslinstallrik i toaletten så att man kan undersöka sin orenhet i lugn och ro. För den med en gnutta insikt i tysk kultur är det fullkomligt logiskt att man som följd av den tyska mustigheten \textsc{(s.~\pageref{682ccd5fdc3aff0c97e8845c3d6b6ca8})} finner ett enormt nöje i att examinera och dagbokföra sitt exkrements karaktär. Kant skrev exempelvis långa tirader om sitt brunas kvaliteter från dag till dag (publicerades i bokform som \textit{Ding an sich} och översattes till svenska med ett underbart själfullt förord av underhållaren, perukbäraren tillika laktosallergikern Carl von Linné) \textsc{(se Carl von Linné s.~\pageref{5e8380bf6b7ce99678e6752b6d9e709e})}.

}

\small{
\textbf{Tysk tårttant}
\label{321d7def7593d2371f3b78446b19ff13}
 En tysk tårttant är en tant från Tyskland \textsc{(s.~\pageref{b1b58da783b6d5fa090f3015f1889869})} som ofta bakar tårtor. Kanske är detta hennes leve\textlessu\textgreaterbröd\textless/u\textgreater? Jo, det är det. Den tyska tårttanten är en motsvarighet till den svenska fluortanten \textsc{(se Svensk fluortant s.~\pageref{2b583e2c23a0890a1595d2f933b710a0})} och kommer till klassrummen för att ge alla barn Schwarzwaldtårta \textsc{(se Schwarzwald Larsson s.~\pageref{278836a3e7f168af75b7eea9b3ae8bb8})} så att de blir riktigt mustiga \textsc{(se Den tyska mustigheten s.~\pageref{682ccd5fdc3aff0c97e8845c3d6b6ca8})}.

}

\small{
\textbf{Tyskland}
\label{b1b58da783b6d5fa090f3015f1889869}
 är en förbundsrepublik som oblygt breder ut sig mellan Polen och Belgien \textsc{(s.~\pageref{f79ffe9e826a19f9f6a446c90e21c4e3})}.

 HEAD2: Kultur
 Tysk kultur består i huvudsak av Freikörperkultur \textsc{(s.~\pageref{40cdc17a157b501b2c84835ce6204f9c})}, pampiga bensinmackar, olika slags festliga hattar samt Werner Herzog. Germanofilen Edward Blom brukar sammanfatta det som är bra med Tyskland med treenigheten \quotetext{Öl och korv och blåsmusik}. Tysken avnjuter dessa tre tillsammans och känner att de inte kommer till sin fulla rätt om de konsumeras var för sig. Därför finns i landet ett oräknerligt antal \textit{bierschtubes} där den långväga besökaren möts av ett varmt sken från brasan, ompa-ompamusik, klanget av ölsejdlar och en vägg av korv-andedräkt \textsc{(se fetor ex ore s.~\pageref{d3b96d618fb972d12fb0cdfdeaf13a98})}.

 HEAD2: Litteratur
 Man läser en hel del Goethe och Schiller, men oftast läser tysken Henning Mankell.

 HEAD2: Natur
 Ruhrområdet sägs vara så gastkramande vackert att många gripits av plötsligt och oåterkalleligt storhetsvansinne \textsc{(s.~\pageref{2f9c0ea6231e1de87c97eab41410c795})} redan vid avfarten vid Essen. Det finns i princip inga djur i denna del av Tyskland, vilket beror på att man malt ner allt som lever och gjort korv av det. Alperna i söder fungerar som den tyska nationalsjälens andliga skafferi. Här vandrar skäggiga män med fjädrar i hatten längs bergsryggar och bygger upp metafysiska system som de efter hemkomsten skriver ner i tjocka böcker. Här har även några väldigt traumatiserade djur lyckats gömma sig undan den tyska matlagningsglädjen, men de är få. Alltför få.

 HEAD2: Idrott
 För att kunna genomföra sina själastärkande vandringar i Ruhrområdet, och för att orka äta all korv \textsc{(se mangel s.~\pageref{ecc5b41821ed829b0c3fb48d4d5389ed})}, är det viktigt för tysken att ha en god fysisk kondition. Idrott är därför ett utbrett fenomen och tyskar återfinns i elitskiktet i alla sporter. De flesta håller på med minst en lagsport, för där kan man agera som en \quotetext{maskin} vilket skänker tysken stor tillfredställelse. Att räkna upp alla tyskar som vunnit VM och OS är helt omöjligt. Listan skulle bli så lång att Internet skulle kollapsa.

 Förut var Tyskland uppdelat i två delar, så nästan alla VM-finaler i lagsporter blev ointressanta eftersom det bara var en massa tyskar på båda sidorna. Till slut höll alla andra på att bli så sura att Tyskland fick sluta upp med det, och numera har man bara ett landslag i varje sport (förmodligen skulle det fungera med samma landslag till alla sporter om man bara ville).

}

\small{
\textbf{Tåga}
\label{01741a2bca362d3355c84528fe9267b4}
 är fackspråk bland friluftsmänniskor för att bära en kanot eller kajak på land längst med ett vattendrag. Varför man skulle vilja göra det kan rimligtvis inte bero på något annat än att det är hål i kanoten, men att man vill spara den av sentimentala skäl. Konceptet är troligtvis danskt, för vem skulle annars komma på idéen att släpa runt på en trasig kanot i skogen bara för att man tog sin första brök \textsc{(se bröka s.~\pageref{60862d3b986c7bbedc86064c842c5a6c})} i den? Man, just let go...

}

\small{
\textbf{Törley gala}
\label{896d6fc796f5eb5ed2924f0b4c6bc540}
 Promenera hem från jobbet på en tisdag. Du kan redan känna smaken av Ica basics fiskpinnar mot tungans ytterkanter. Det plaskar om dina fötter mot den novemberslaskiga asfalten. För all del inte mer suicidal än någon annan tisdag i november. Men det är inte fredag i juli, direkt.

 Plötsligt minns du. En liten rest från helgen. Kvarglömd av en träskpunkare \textsc{(s.~\pageref{484838b3db1adb135ea74d6fc61e44c0})} som firat löning. Inslagen i en systemkasse med hexadecimal färgkod \#177F75. Gömd längst in i kylen. \textit{Törley gala}, stavas kvällens frälsning.

 Tanken om fiskpinnarnas brödiga yta mot tungan byts ut mot den om en porlande, bubblande tungkyss. Den bekanta vägen hem kryddas med danssteg, som Sinatra. Du nynnar oskrivna sonetter i trapphuset. Att dra upp kylskåpsdörren är som att svepa en älskare av sina fötter. Att öppna påsen (färgkod \#177F75) är som att slita kläderna av den du mest åtrår. Att dra ur korken - \textit{közösülés \textsc{(s.~\pageref{0ebb408ff797d9cc15611b4f0c691685})}}. Varje efterföljande klunk av den klara drycken är ett post-coitalt \textsc{(se post-coitus s.~\pageref{c0b154a4b061b684da6fb1e3cbe6a843})} cigarettbloss, på den mjukast smakande cigarett du någonsin slutit dina läppar kring.

 Du kommer aldrig älska igen som när du älskat \textit{Törley gala}.

}

\small{
\textbf{Uggelbävern}
\label{e1269c785bafb5a7fa2f36766bcba3ee}
 Har tydligen haft en skrot. Och han vinkar alltid när han kör förbi.

}

\small{
\textbf{Ugglekonst}
\label{660afe2e36272ef971181caca78a43bd}
 är konst föreställande ugglor. Tekniken kan variera men grafik är överrepresenterat. En viss 70-talsanda bör finnas med i verket för att det ska få betecknas ugglekonst.
 Uttrycket Ugglekonst kan lite slarvigt användas för att beteckna föreställande konst med andra motiv än ugglor. Det kan vara andra fågelarter, djur eller barn \textsc{(s.~\pageref{5dfcc0aab2f3db925b2d51ba73e48946})}, allt med ett stänk av politisk realism.

}

\small{
\textbf{UK}
\label{c2431a9d201559f8de1dcfb6a9dd3168}
 är en förkortning för Uvarnas Konferens. För mer information, se uv \textsc{(s.~\pageref{45210da832f9626829457a65e9e7c4d0})}.

}

\small{
\textbf{Ulmerkott}
\label{60f73d6cde46ade4a704c8c68e48250b}
 En mycket sällsynt hundras, först omtalad i sketchen \quotetext{Hundägare Barbro Lindeman}, Barbro (eller var det Brabor, eller Rabarbro) Lindeman är ägare av en hund av den ovanliga rasen Ulmerkott från Japanska Schweiz.
 För att korrekt beskriva rasen saxar vi helt enkelt en del av sketchen:

 H:      Väldigt dyrbara hundar från Japanska Schweiz. Alla hundar kommer från
 Schweiz. Nå i alla fall... denna Ulmerkott som hette... vad hette den?
 T:	Eeeh..
 P:	Grogg!
 H:	Ja! Hon kan ropa på grogg, den damen, nå i alla fall... så, det hette
 han i alla fall. Då, eh, kom det plötsligt in... han var stor han
 brukade jaga ut andra hundar, fick inte vara hundar på krogen nej.
 Men en dag kom det in en liten gul hund. En liten konstig gul hund.
 Och då, kasta sig Grogg över honom, men den gula hunden bara \quotetext{Chlapp!}
 klippte utav huvet på Grogg som bara i ett som ett nafs.
 T:	Det var det värsta!
 H:	Det var det värsta ja,... som kan hända med en hund faktiskt, ja.
 Så... jag gick fram till ägaren och fråga va, va en attans klämmig
 dogg du ligger inne med sa jag, kan bita huvet av en Ulmerkott så
 där bara... rätt av.
 \quotetext{Va är det för ras?} sa jag.
 \quotetext{Ja jag vet... ja ras och ras alltså... innan jag... målade den gul, va
 ...och klippte av svansen, va... och filade ner taggarna, va...
 så var det en krokodil.}

}

\small{
\textbf{Umeå}
\label{bd1e37dc477bb704c667ed1a4606df71}
 är ett slags tätort som ligger i Västerbotten \textsc{(s.~\pageref{d4b008c5143dcffb6b8c35f3876c2a19})}, lustigt nog utmed Umeälven. Befolkningen utgörs av urbaniserade samer, studenter, akademiker, vanliga människor och folk som lyssnar på hardcore, samt Martin Emtenäs, trumslagare och pappaledig programledare för Mitt i Naturen.
 HEAD2: Politik
 I Umeå bestämmer en lite läderartad, grodliknande man som lystrar till namnet Lennart Holmlund \textsc{(s.~\pageref{26d063a59c90487b11c8f5b4fa9af348})}.
 HEAD2: Trafik
 Grodmannen har bestämt att man i Umeå ska belägga alla gator och körbanor med ett slags engångsmaterial, vilket får som resultat att man måste göra om detta varje vår. Det gör inget, tycker grodmannen, men att vägbyggen upptar mer än 227\% av stadens vägnät förhindrar effektivt invånarna att ta sig runt i stan per förbränningsmotor, så gå eller cykla om du kan.

}

\small{
\textbf{Underbet}
\label{48b856b8c4d6c42a49490007fb944f7c}
 Som ett led i Nissepedias anpassning till EU-direktivet Z:554 2012 om rätlinjig humor har artikeln om Underbet vid vite klassats som inte tillräckligt varken rak eller grön, och därför ej lämplig för konsumtion utan skyddsdräkt. Får dock säljas som djurfoder.

}

\small{
\textbf{Undra}
\label{2892279329abc5031d4bdc2c32b2a129}
 Att undra över något är detsamma som att vara vetgirig eller jävligt nyfiken.
 Man undrar mycket när man funderar över något.

 HEAD3: Exempel
 Undra om hon ser på mig att jag ljuger?

 Undrar om någon läser och tycker det är roligt det jag lagt upp på Nissepedia?

 Varför är blåbär blåa?

 Hur fan kunde dom veta att det var jag?

}

\small{
\textbf{United States of America}
\label{ade6b3bd5e720abb20ed8a9a4c6b9ae8}
 The United States of America är världens största nöjespark och det globala nyliberala \textsc{(se tokliberal s.~\pageref{531cb70b602e3f3c32d40bac64400830})} imperiets centrum. Här är i princip alla moderater \textsc{(se moderat s.~\pageref{c4564b188cb670841733a3ff923c2fb0})} och de som inte är moderater är tokkonservativa, homofoba sociopater som blir galna om man inte skickar ut den egna arbetarklassen, speciellt svarta, i anfallskrig runtom i världen. Man har också många olika sorts fantasifullt utformade jumpaskor \textsc{(se sneakers s.~\pageref{a1743c0d39461290efc551490aafc1e2})}, som görs av barn i Asien, till försäljning. På grund av detta fascineras många av denna plats. Martin \quotetext{Rocky} Kellerman brukar till exempel ofta rita serier om hur han och hans killkompisar besöker the United States of America för att köpa jumpaskor.

}

\small{
\textbf{Universitet}
\label{11dfc744fa396b961a6cc9cf89c4eaea}
 En samling hus på slät mark vilken påminner om en prärie. Där drar stora horder av overallstudenter \textsc{(s.~\pageref{09a5062cf884d746996bf5a9f3669d1b})} runt likt boskap medan förvirrade utbytesstudenter \textsc{(s.~\pageref{397699f3732b0c22f3c532a111697539})} letar efter busshållplatsen. Här huserar också andra sorters stundenter, t.ex. ambitiösa \textsc{(se Den ambitiösa studenten s.~\pageref{02257ef6d6da8e0f0721e2758eec3c71})} och oambitiösa \textsc{(se Den oambitiösa studenten s.~\pageref{773fb9013bfd8af98ed84fe0abc8748e})}.

}

\small{
\textbf{Universitetsbiblioteket}
\label{e69cf3b6f7c7f3872fe561600a7e9aa7}
 Platsen där Bayta Darell (för att rädda Stiftelsen undan oåterkallelig undergång) dödar Ebling Mis, strax innan han ska avslöja Andra Stiftelsens hemvist för Mulan. Beläget på Trantor, det tidigare vintergatsimperiets centralplanet.

}

\small{
\textbf{Uns}
\label{07525332b1617934911c9fbadb3a304e}
 En storhet som inte alls är särskilt stor. Ska man ta bort ett uns från en mindre träbit \textsc{(s.~\pageref{d27020124419b69173fc321fe29e1d4b})} så kan man t ex använda en putshyvel \textsc{(s.~\pageref{82aace730b3087db7cfc8b4ed5d7dae0})}, är träbiten större så får man ta till rubank \textsc{(s.~\pageref{b1c373a9ae319af9e1bf15a62fdf85cf})}. Uns kan även användas när man skall definiera sanning, den här artikeln innehåller mer än ett uns sanning. En filur \textsc{(s.~\pageref{e308f4e2553faf188385f17ebda05242})} däremot talar inte ens ett uns sanning.

}

\small{
\textbf{Uppkavlade ärmar}
\label{a86a08e7e2aca91a21350bc184e05367}
 signalerar: Nu är det allvar!

}

\small{
\textbf{Uppland}
\label{0f24a6eb0b60bdcd74885743cb7099d8}
 Sveriges \textsc{(se Sverige s.~\pageref{b1999637949ed135b2ca03f3a38460cc})} Washington. Här samlas den politiska makten - det vill säga tillresta fascistiska skånska \textsc{(se skåne s.~\pageref{a01d1167b9dcd72e212d876d672db261})} grisbönder och Thatcher-nostalgiska liberaler från Täby som av rördhet snyftandes ser tillbaka på fornstora dagars massavrättningar på latinamerikanska fotbollsarenor. Att Uppland är Sveriges \textsc{(se Sverige s.~\pageref{b1999637949ed135b2ca03f3a38460cc})} politiska centrum märks inte minst på alla de runstenar och historiska lämningar som vittnar om den tusenåriga historia av demokrati och fredsälskande brödraskap som förtryckta kommentarfältherrar sammanbitet beskyddar mot feministisk och mångkulturalistisk revisionism. I Uppland finns natursköna områden så som Roslagens skärgård och Sigh-tunas vacka omnejd där resliga ariska män och rejäla blonda fruntimmer vandra hand i hand genom lummiga ekskogar. I Uppsala \textsc{(s.~\pageref{1db4e388df1df7057b8f3d984c65ee88})}, Sveriges \textsc{(se Sverige s.~\pageref{b1999637949ed135b2ca03f3a38460cc})} fjärde stad, är borgerligheten som sig bör fortfarande vid makten - vilket är en anmärkningsvärd skillnad från Sveriges \textsc{(se Sverige s.~\pageref{b1999637949ed135b2ca03f3a38460cc})} mer förfallna kommuner där hjärntvättade kulturmarxister, uppfyllda av stalinistisk ondska, smider ränker för att lavinartat höja antalet dagismammor och -pappor (!) i den kommunalt drivna barnomsorgen. Nej, i Uppland råder än så länge den ordning som cementerades under femtiotalets ekonomiska storhetsperiod då husmödrar knaprade amfetamin som vore det Salta katten och husbönderna rökte pipa bakom uppslagna dagblad \textsc{(s.~\pageref{d074687e28275e17cdd4f778e9cf96c9})}, även om detaljer har förändrats och amfetaminet bytts mot cafe latte och dagbladen mot sanningssägande bloggar \textsc{(s.~\pageref{fe95efaed6d1841dc1e8d5fb77d9ebf7})}. Och tur är väl det, när den stolt vajande svenska \textsc{(se Sverige s.~\pageref{b1999637949ed135b2ca03f3a38460cc})} fanan hotas att halas av shiamuslimska HBTQ-nihilister som törstar efter den utdöende nordiska rasens blod!

}

\small{
\textbf{Uppsala}
\label{1db4e388df1df7057b8f3d984c65ee88}
 är en stad som passande nog ligger på Uppsalaslätten. Här finns Uppsala domkyrka, Gustavianum, Uppsalaslottet, Engelska parken, Karolina Rediviva, Studenternas idrottsplats och mycket annat att besöka och se på. Men det som gör Uppsala riktigt unikt är att ett antal subkulturer som inte finns kvar på annat håll fortfarande lever och frodas i just Uppsala. Där finns det till exempel forfarande gothare och folk som frivilligt lyssnat på Nu Metal. Detta gör att folk som fascineras av det lite aparta dras till staden. Carl von Linné \textsc{(s.~\pageref{5e8380bf6b7ce99678e6752b6d9e709e})} och Olof Rudbeck är endast två i den månghövdade skara människor som flyttat till Uppsala för att försöka lista ut varför man skulle vilja kombinera rapp \textsc{(se hip-hop s.~\pageref{66c22415908267e727d3fa4a63c16672})} och metal \textsc{(se hårdrock s.~\pageref{a4566a810e7ad85a57ddc84083a8139b})}. För att alla ditresta skulle ha någonstans att jobba byggde man tidigt ett universitet i staden, för alla kan faktiskt inte jobba på Slotts senapsfabrik.

}

\small{
\textbf{Uppstoppad uv}
\label{a562653cfd13c16d7f4d85967242ccdd}
 En uppstoppad uv är en uv \textsc{(s.~\pageref{45210da832f9626829457a65e9e7c4d0})} som genom en tragedi, som det alltid är, har dött och därpå påträffats av en människa som stoppat upp den för att bevara den som ett slags prydnad. Uppstoppningen går till så att munstycket till en tryckluftskompressor förs till uvens mun \textsc{(s.~\pageref{6585f290ce92c3de5ff339920330e26f})}, det vill säga näbben. Sedan blåses uvens innanmäte ut, och kanske ner i en gryta mustig Strigiformes au Riesling \textsc{(s.~\pageref{f79211a2d52de6abaa480e60938e98fe})}. Därpå pulas mossa och liknande saker ner eller upp i uven. Ståltråd används för att forma uven så att det ser ut som den lever.
 Se även: förstoppad uv \textsc{(s.~\pageref{06630b162e869a376076dda808c05e5f})}.

}

\small{
\textbf{UR}
\label{baceebebc179d3cdb726f5cbfaa81dfe}
 är en förkortning och står för Utbildningsradion, men för samtidigt tankarna till det allmänt förstärkande prefixet ur- som i urbra eller urtråkig. Utbildningsradion är märkligt nog inte en radiokanal utan en del av public services televisionen. UR lär folk att \quotetext{språka} på serbokratiska, hur man sätter samman ett blomarrangemang, hur det är för döva eller finländare \textsc{(s.~\pageref{fc472090d678bd6f029cd80792f4a36d})} att se på TV och mycket, mycket mer. Utan UR hade Sverige fortfarande på 2010-talet sällat sig till världens många u-länder och folkets utbildningsnivå hade varit att jämföra med dumma amerikaners. Kompletera gärna din inlärning via UR med nissepedia \textsc{(s.~\pageref{62400dadecd90cb5cd39062abe5a3e4a})} för att på middagar kunna konversera på ett fritt och avslappnat vis bland de mest pålästa personer, utan att känna dig tillknäppt och efterbliven.

}

\small{
\textbf{Urin}
\label{524fd7acb94f9c2d879b5c1cf8335669}
 är ett lite finare ord för kiss, en guldfärgad vätska som kommer ut ur kroppen efter att man druckit något. Detta är helt normalt och är inget att bli förskräckt över.

}

\small{
\textbf{Urinvånare}
\label{db93d18c9b0bfc6c994aaea928672399}
 Ordet \quotetext{urinvånare} uttalas \quotetext{urin-vånare} och inte på något annat sätt.

 Sammanslagning av orden urin \textsc{(s.~\pageref{524fd7acb94f9c2d879b5c1cf8335669})} (ur·in [-i'n] s. -en • vätska utsöndrad från njurarna) och vånnare (Vånna v. -de • 〈prov.〉 bry sig om; önska - Mest i pres. indikativ vånnar o. pret. konjunktiv vånne, t.ex. jag vånne jag vore urin jag ville jag vore urin.)

}

\small{
\textbf{Usama con carne}
\label{7ebbb820c50cf5686f7bdc4019552e45}
 är en  vegansk \textsc{(se veganer s.~\pageref{2a12d5d6ae91d2f4f7d9af3cef58e75c})} variant av Chili con carne som utvecklades av den sedemera världskände islamisten Usama bin Laden före det att han slog in på terrorbanan \textsc{(s.~\pageref{aec7bd708ed2ad3435b9a9883ac7f45c})}. Istället för chili använder man i en Usama daddlar och russin och andra slag av torkad frukt som man tycker speciellt mycket om. Därför borde den, kan man tycka, heta Usama sin Carne, men det gör den inte och varför får vi aldrig veta nu när jänkarna har skjutit Usama i huvudet \textsc{(se huvud s.~\pageref{e906cd95a540df9b16d0460fb4cf0adc})} och slängt hans kropp i havet. Usama con carne är typisk multikulti-mat \textsc{(se multikulti s.~\pageref{25eea9148080d30d384ce1c1277ef126})}.

}

\small{
\textbf{Utbytesstudenter}
\label{397699f3732b0c22f3c532a111697539}
 Glatt leende men inombords förbannar de syokonsulenten \textsc{(se syo s.~\pageref{e6ece7e1836dfe745a2b015fb2da8fc0})} som inte visade på kartan var de kommer att spendera de närmaste fyra åren.
 HEAD2:  Se även:
 Den oambitiösa studenten \textsc{(s.~\pageref{773fb9013bfd8af98ed84fe0abc8748e})}
 Den ambitiösa studenten \textsc{(s.~\pageref{02257ef6d6da8e0f0721e2758eec3c71})}
 Overallstudenter \textsc{(s.~\pageref{09a5062cf884d746996bf5a9f3669d1b})}

}

\small{
\textbf{Utlastarskämt}
\label{7b5944fdf38b9fe45be355a0ec7863df}
 är skämt med huvudsyfte att underhålla någon, men även påvisa att individen är en så kallad utlastare. Exempel på klassiska utlastarskämt är den upptejpade leksaksgitarren på vilken man kan läsa ”sex, droger och plasta burar”. Ett annat är att surra fast en cykel (med samma plast som skyddar varorna!!!) fyra meter ovanför marken i en stolpe. Ett mindre lyckat skämt var när Olle plastade fast Ronny i en bur full av ananas och skickade honom till ett daghem i Dorotea.

}

\small{
\textbf{Utlastningen}
\label{718011908ce5418796943076d564c741}
 är det lagerområde som drar flest utlastarskämt \textsc{(s.~\pageref{7b5944fdf38b9fe45be355a0ec7863df})} och slutar sist. Där kan man, om man använder mellanlägg, lägga varor till olika kunder i samma bur. Man kan även plasta burar dels i plastmaskin, dels genom att ihållandes en plastrulle (där ena änden sitter fast i burens överkant) springa runt buren.

}

\small{
\textbf{Utlåtande om det danska köket}
\label{cf12f25542a13a7741ba4d5e8fcd4307}
 ”I år var det ju något hopsläng av grejer där på slutet... Skulle beställa vegetarisk burgare men hon sa att den är slut och när man köat så länge ger man ju inte upp i första taget. Så jag valde lasagne i stället. Fick en degklump som skulle föreställa lasagne med tomatsås och en skopa äpplen. Gick väl ner med trevligt sällskap och öl men gott vad det inte. Kompisen fick två korvbröd på sin mat? Så allt kändes rätt ’nu slänger vi upp vad som finns av det vi har’. Då borde priset sänkas något... tack ändå för att det var öppet för vi var ju hungriga”.

 Källa: Besökare av Norbergfestivalen år 2012.

}

\small{
\textbf{Utrikiska}
\label{91a87f1436fd9d07bb94f048d2e9aeed}
 är tungo- eller bokmål som utmärkser sig genom att inte vara svenska. Många svenska musikartister, så som Europe, Stakka Bo och några andra, har gjort sig ett namn genom att sjunga på utrikiska.

 HEAD2: Grammatik
 Utrikiskan har en säregen grammatik som kräver år av övning för att bemästra. Ta till exempel följande exempel;
 Svenska: *Grisen är förvånad.
 Detta heter på franska:
 \begin{itemize}
 \item Liberté, egalitet, gerard depar dieu, et fretârneté.
 \end{itemize}

}

\small{
\textbf{Utrotningshotade djur}
\label{24a427a5537c2c8918cfa213ae099a74}
 är djurarter som det inte finns så många individer kvar av. Genomgående för sådana arter är att dom har ganska konstiga och lustiga namn. Tyvärr har ingen tittat närmare på detta samband och inga empiriska studier har gjorts på  vad som händer om man döper om arterna till något vanligare och lite tråkigare.

 Djurarter som är utrotningshotade är bland annat: Hawaiis munksäl \textsc{(s.~\pageref{18ec056305ef5577db08b7219c1dfd5c})}, Eskimåspov \textsc{(s.~\pageref{7d88e5ee3ad2479d7c1dfe662396fa1a})}, Dvärgpungsovare \textsc{(s.~\pageref{d751d87e9373bbed2540bb46dba30e5c})}, Filipinsk apörn \textsc{(s.~\pageref{33d996c9277047f9a54e50c2222031cb})}, Amazonskrake \textsc{(s.~\pageref{c8b53d55fb445965eb6afbce9865c210})}, Rodriguesflyghund \textsc{(s.~\pageref{19cc9824f1fa5834a866261fe69352ea})}, Snefotad ultrapelikan \textsc{(s.~\pageref{0def09852ec31cb5af0c38180b411782})}, Gulsvansad ullapa \textsc{(s.~\pageref{c9022fba896b3e8a41420242680d2480})}, Havsmunk \textsc{(s.~\pageref{81441ab6429e473c2a7679b4a54246e3})}, Flodkanin \textsc{(s.~\pageref{5b2fd3512fae865f843dfe95c778fa07})}.

 Nästan 56000 skojjiga namn har organisationen IUCN skrivit in på sin röda lista [http://www.iucnredlist.org/technical-documents/spatial-data]. Dess svenska motsvarighet, Artdatabanken [http://snotra.artdata.slu.se/artfakta/GetSpecies.aspx?SearchType=Advanced], är inte lika fyndiga men gör så gott de kan.

}

\small{
\textbf{Utskrapan}
\label{9765e20ddf846e5d9821c7319c05f6ae}
 var i mitten av åttiotalet en låtsaskompis till en kille som idag är stonerskin \textsc{(s.~\pageref{b94c65dba2990b3146c2bedf663e9989})} och som mot sin vilja tvingas bo i Växjö \textsc{(s.~\pageref{2fc07b846123d1c41a4c7eb55c40df40})}. Utskrapan var till det yttre en mer eller mindre precis mental kopia av en gul plastskrapa som killens far använde för att rengöra familjens akvarium. När det begav sig umgicks de två vännerna i huvudsak i samband med läggdags och vid tillfällen då det skulle badas badkar, men mot slutet av årtiondet dök Utskrapan upp allt mer sällan för att tillslut helt upphöra med besöken. Vissa spekulerar i att han nu mer fungerar som rådgivare åt Anton Abele \textsc{(s.~\pageref{0906f6e1d290c547e1fb93c6ff6a0b44})}, även om det är svårt att tro att Utskrapan skulle svika sina politiska ideal på det viset.

}

\small{
\textbf{Uv}
\label{45210da832f9626829457a65e9e7c4d0}
 ar (latin: \textit{bubo}) är ett släkte fåglar som vid en första anblick kan likna en uggla men egentligen är något helt annat.
 
 Uvar kännetecknas av sin skyhöga intelligens, totala brist på empati och allmänna mäktighet.
 Varje år i mörkaste december hålls Uvarnas Konferens någonstans i de lappländska fjällen. Här diskuterar uvarna hur de ska ta över världen [http://youtu.be/VVSTUHdWVVM], något som hittills (tack och lov) har misslyckats.

}

\small{
\textbf{Uv-ljus}
\label{2239e04c73609ab9e8cc9b359552fa81}
 är en elektromagnetisk strålning som kommer från uvar \textsc{(se uv s.~\pageref{45210da832f9626829457a65e9e7c4d0})}. Varje gång en uv öppnar sin näbb \textsc{(se näbbmun s.~\pageref{9e3395be14cf14f92e8cd1e93eb7599b})} för att hoa skjuts ljusstrålarna ut och paralyserar allt som träffas av det. Ungefär som med laserögon \textsc{(s.~\pageref{7d642f9221f16b36fa9d731166ba3416})}. Det bästa sättet att skydda sig mot uv-ljus är att vistas i en säker uv-bil, en så kallad \quotetext{SUV}, och/eller att ta för vana att alltid bära uv-säkra solglasögon.

}

\small{
\textbf{Uv-rugby}
\label{c0116c901e33015a052c0402d01607e4}
 är en form av rugby framtagen speciellt för uvar \textsc{(se uv s.~\pageref{45210da832f9626829457a65e9e7c4d0})}. Som med vanlig rugby är det oklart vad spelet faktiskt går ut på, men uv-rugbyn innebär i alla fall ett väldans massa a-hootin' and a-hollerin' \textsc{(s.~\pageref{1928c39ea0f58992a3e5f53d143a23ff})} runt en avlång boll.

 HEAD2: Andra betydelser
 Det finns en variant av vanlig rugby som också kallas uv-rugby som har samma regler som vanlig rugby, men istället för en boll används en uv \textsc{(s.~\pageref{45210da832f9626829457a65e9e7c4d0})}. De flesta matcher slutar med att \quotetext{bollen} flyger iväg.

}

\small{
\textbf{Uvbulvan}
\label{2ebfd82fbe9976dbc2dc2c45ae58bde9}
 I nådens år 1996 ledde en irländsk diskussion \textsc{(s.~\pageref{d832a3c58a177f7c838e1307dbbecbb6})}  till att några forskare i Dublin skapade världens första klonade djur; fåret Dolly. Allt sedan den dagen är inte ens döden \textsc{(s.~\pageref{6f3c270eb5b4d979c777b4ec26dd106f})} längre helig utan som allt annat föremål för människans söndrande värdsherravälde. De sista fria kämparna för en alternativ världsordning, uvarna \textsc{(se uv s.~\pageref{45210da832f9626829457a65e9e7c4d0})}, var redan försvagade till följd av uppstoppning \textsc{(se uppstoppad uv s.~\pageref{a562653cfd13c16d7f4d85967242ccdd})}, förstoppning \textsc{(se förstoppad uv s.~\pageref{06630b162e869a376076dda808c05e5f})} och bortstoppning \textsc{(se bortstoppad uv s.~\pageref{86574b11bb49a6f8e32d9f716676236a})}, men detta innebar det definitiva avbräcket. Kort efter fåret Dolly lanserades de första uvbulvanerna. Gjutna i hårdplast och naturtroget målade fyllde uvbulvanerna snabbt människans fulla behov av fågelskrämmor mot måsar \textsc{(se mås s.~\pageref{04f599c35052d2060c70cb99b09f94dd})} på balkongen, svanar \textsc{(se svan s.~\pageref{f80f1875ab3ebccf935723ba83b6da63})} på bryggan och enkelbekasiner \textsc{(se enkelbekasin s.~\pageref{92790a7d98a9c95e665e46ff0cc91f00})} på åkern. Som så mycket annan ondska i världen saluförs uvbulvaner av Svenska jägareförbundet \textsc{(s.~\pageref{e56c5b0ef648ae8e763d292c96f7894a})} [http://www.jagareforbundet.se/blekinge/default.asp?pageid=7995].

}

\small{
\textbf{Uvgodis}
\label{58de09e078ac891b067c0ec53d780b8a}
 är godis särskilt avsett för uvar \textsc{(se uv s.~\pageref{45210da832f9626829457a65e9e7c4d0})}. Det är ett godtaget ord att lägga i Alfapet.

}

\small{
\textbf{Uvmytologi}
\label{82c4190e226289700fef893b7a7a1a08}
 I den engelska översättningen av det vediska verket Srimad Bhagavatam, kapitel och vers 1.14.14 står att läsa:
 \quotetext{The shrieks of the owls and their rival crows make my heart tremble. It appears that they want to make a void of the whole universe.}
 Detta är en uppenbar referens till uvarnas \textsc{(se uv s.~\pageref{45210da832f9626829457a65e9e7c4d0})} flertusenåriga plan.

}

\small{
\textbf{Uvsmör}
\label{240a6e2f1169dc87b9533f6b9c7b0aec}
 Smör gjort på uv \textsc{(s.~\pageref{45210da832f9626829457a65e9e7c4d0})}. Utan tvivel det godaste, för att inte säga ljuvaste \textsc{(se ljuv s.~\pageref{632bf5372f37d760ebb25b34ab411f71})}, smör man kan äta.

}

\small{
\textbf{Uvstorke}
\label{6458dce510a0f0cfb9a720ee5d3e62be}
 Vi börjar kunna det här nu va?
 HEAD2: Se också
 \begin{itemize}
 \item Calskrove \textsc{(s.~\pageref{84ff54e779ee49fdad21e17c20f14453})}
 \item Transkrove \textsc{(s.~\pageref{1188281a09fb681b922e45663e5ffc4b})}
 \item Smörskrove \textsc{(s.~\pageref{c3ec1fc646dfd34ddd483f8031d649c9})}
 \item Uvsvane \textsc{(s.~\pageref{c5081b14cdeb1ff42b655213e80c9d51})}
 \item Johnskrove \textsc{(s.~\pageref{92a6f4a71ab0087f48ba4aab7db89bdb})}
 \end{itemize}

}

\small{
\textbf{Uvsvane}
\label{c5081b14cdeb1ff42b655213e80c9d51}
 En Uvsvane är en maträtt som påminner om en svanskrove \textsc{(s.~\pageref{e543ead268a283bfdb5ea638d6cca4a2})} och består av en Uv \textsc{(s.~\pageref{45210da832f9626829457a65e9e7c4d0})} inbakad i en Svan \textsc{(s.~\pageref{f80f1875ab3ebccf935723ba83b6da63})}.

 Serveras oftast vid disputationer och andra festliga tillställningar i Västerbotten \textsc{(s.~\pageref{d4b008c5143dcffb6b8c35f3876c2a19})}.

}

\small{
\textbf{Uvtårar}
\label{8f041258aba5f2fb5aca5d11b2c5f1b0}
 Potent blandning i vätskeform, där alkohol \textsc{(s.~\pageref{11c589cba1a208e0359048a78e6b88b8})} är den instabila \textsc{(se bil s.~\pageref{b3188f47d2eac7efc3f1258dc673a9fe})} ingrediensen. Gör folk, enligt utsago, lömska och hämndlystna. Blandningens sammansättning presenteras ej här av säkerhetsskäl, helt enkelt.

}

\small{
\textbf{Valentina Vladimirovna Teresjkova}
\label{9ac2066d1c8dad6bcb2236eb70fb7cc7}
 (1937 - ) var (och i någon mening är) en Sovjetisk kosmonaut och den första kvinnan \textsc{(se kvinna s.~\pageref{9a7760b2521c3471c47cd5d789a2d324})} i rymden och blev det 1963. Det skulle dröja ytterligare 20 år innan jänkarna skickat upp en kvinna, så tänk på det!

}

\small{
\textbf{Valfrihet}
\label{07440011607a3b08083f7250c34175cf}
 Är något som liberaler \textsc{(se tokliberal s.~\pageref{531cb70b602e3f3c32d40bac64400830})} ofta hänvisar till i pressade situationer och när orden inte räcker till. T.ex:
 att skjuta jonk och sälja röv till radhuspappor samt att kalla en låda från IKEA ett hem. Ja det är ju du som valt den livsstilen so stick with it. Det är helt enkelt lösningen på allt. Du valde själv.

}

\small{
\textbf{Valsvärk}
\label{8e19e3ad5b45b707007e27dd6760e393}
 Vanlig åkomma bland dansare. Symptom är bland annat bristningar i dragspelsmuskeln \textsc{(s.~\pageref{4265ffc7068c10706460aa133c2918bf})} och ömma tår.

}

\small{
\textbf{Valsång}
\label{ba7577a4f99849a9141cc8c9d9d91e62}
 är en typ av degenererande ljud som drabbar människor i batikkläder, vindsvåningsägare på Södermalm och andra riskgrupper. Efter bara några minuters exponering har valsången intagit lyssnarens hela hjärna och löst upp den sista gnuttan sans. Ett klassiskt exempel är när rockgruppen U2 spårade ur på sin Joshua Tree-turné och skrev upp en 100 kubiks vattentank på ridern där man tryckte ner en strandad tumlare. Fy fan vad det lät.

}

\small{
\textbf{Valuta}
\label{cf1e2a0af4955aa7539b6e12e9d282e6}
 är något man \textsc{(s.~\pageref{39c63ddb96a31b9610cd976b896ad4f0})} betalar med. Valutor har olika värde och man måste därför använda olika mängder valuta när man betalar. Att försöka betala med en valuta som inte är direkt giltig i sammanhanget räknas som socialt extremt.


 \textbf{Exempel på valutor:}
 Frimärken
 Oskrapade skraplotter
 Skrapade skraplotter med vinst \textsc{(s.~\pageref{d92a1dab805e77864765073c2a276c4c})}
 Pantburkar
 Olåsta cyklar
 metalmynt \textsc{(s.~\pageref{66b54fb5f4c4b119a30452f71d678055})}
 Rikskuponger \textsc{(s.~\pageref{2e3acaa8f24b5db948a51e402a6f2349})}
 Presentkort
 Lettisk smuggelcigg
 Koppar-, zink- och mässingsskrot.
 Skräp \textsc{(s.~\pageref{75f1a5320951ea0dd9aa3c0eaba2c2c7})}

}

\small{
\textbf{Vanliga fel i Engelska språket}
\label{6a72e5450492ee6465257e954287d322}
 Allteftersom Engelskan brett ut sig i Amerika har folk börjat säga fel.

 Här är en utmärk lista för den som alltid vill göra rätt [http://http://www.wsu.edu/~brians/errors/errors.html\#errors]

}

\small{
\textbf{Vanliga pantade knegare}
\label{98d0a7dac261debb934a16b7041ef22f}
 \quotetext{Vanliga pantade knegare} är inte anslutna till ett Marxist-leninistiskt parti utan går till jobbet varje dag helt ovetandes om att revolution ligger i deras valkiga händer. Till skillnad från kälkborgaren \textsc{(se kälkborgare s.~\pageref{0f34b469a48952e93688861083ace75a})} så har dessa arbetare i alla fall vett att klaga något så jävulskt.

}

\small{
\textbf{Vansinnets historia}
\label{18a20c9ab3852aa00d423ae3a72cfc50}
 är undertiteln på Håkan Skyttes självbiografi om åren som träpinneslagare i Hoola Bandoola Band. Boken början med att Håkan sitter i en sackosäck hemma i sitt kollektiv och virkar makramé när hans bror Göran \textsc{(s.~\pageref{798906d6f87c98cb6c72c306560e30f4})} plötsligt älgar in och skriker att det skett en massa nya orättvisor som dom måste protestera mot. På protestmötet träffar dom Görans dåvarande bandkompis Mikael, som hänförs av Håkans bländande progguppsyn med milt ansiktsuttryck, Gustav Vasa-frisyr \textsc{(se Gustav Vasa s.~\pageref{aeb7f10919b25762e3d031a0b583a2e8})} och helskägg. Mikael inser att en som Håkan är just vad \quotetext{Hoola} (till vardags brukar han kalla bandet så) behöver för att öka sin trovärdighet. För själv ser han ut som en korgosse, och som andre frontman har dom en kroniskt grinig halvdansk. Tyvärr är alla instrument redan upptagna i gruppen så Håkan får nöja sig med att slå på två träpinnar. Publiken betraktar en träpinneslagare i gruppen som väldigt proggigt och succén \textsc{(se braksuccé s.~\pageref{678371d35369d3d29afceb1445630833})} är ett faktum. Tillsammans åker Håkan och Hoola Bandoola Band runt i flera år och protesterar mot saker. Sen är boken slut.

}

\small{
\textbf{Var}
\label{b2145aac704ce76dbe1ac7adac535b23}
 Den där göttiga gula sörjan som man alltid hoppas på när man får sår, men som allt som oftast uteblir. Liknar vaniljkrämen (mormors hosta) på till exempel danska wienerbröd (spandauer).

}

\small{
\textbf{Varför Marx hade rätt}
\label{809ed1a081f14fb42e23fee9b229c44f}
 är främst två saker: ett onödigt påstående och en bok av Terry Eagleton. Här skall vi uppehålla oss vid boken. Boken heter Why Marx was right på engelska. Boken är dock utgiven på svenska av Tankekraft Förlag.

 I boken avfärdar Eagleton elegant tio olika myter som diverse knäppskallar har kommit på och alltid upprepar om just varför Marx hade fel och är dålig.

 \textbf{De tio osanna myterna är:}

 \textbf{1.} \textit{Marxismen är ett avslutat kapitel.} Den var möjligen relevant förr då det bara fanns fabriker, hungerkravaller, misär etc. I dagens postindustriella, klasslösa och socialt rörliga samhälle är den dock inte längre användbar.

 \textbf{2.} \textit{Marxismen är kanske bra i teorin. }Varje gång den omsätts i praktik går det åt helvete och slutat i elände, massmord och tyranni. De som fortsätter tro på marxismen är trögtänkte eller är moraliskt lågt stående.

 \textbf{3.} \textit{Marxismen är deterministisk. }Marx trodde på järnhårda historiska lagar som inga mänskliga handlingar kan påverka. Det var alltså förutbestämt att feodalismen skulle ge vika för kapitalismen som i sin tur oundvikligen kommer ge vika för socialismen.

 \textbf{4.}\textit{ Marxismen är en utopisk dröm.} Marx tror på det perfekta samhället utan fattigdom, lidande och elände. Under kommunismen kommer alla konflikter att försvinna. Ingen kommer vara överlägsen och alla kommer leva i harmoni. Marx blåögda framtidsvision speglar en absurd verklighetsuppfattning.

 \textbf{5.} \textit{Marx reducerar allt till att handla om ekonomi.} Detta är typ ekonomisk determinism. Konst, religion, politik, moral etc. betraktas enbart och förenklat som en återspegling av ekonomi och klasskamp. Marx bortser helt från komplexiteten i mänskligt beteende.

 \textbf{6.}\textit{ Marx var materialist. }Han ansåg att materian var det enda som existerade. Han var inte intresserad av våra andliga sidor utan de var bara en återspegling av den materiella världen. Marxismen tömmer mänskligheten på allt värde och reducerar oss till materia, denna andefattiga bild av människor leder sedan till förbrytelser som de bland annat Stalin genomförde.

 \textbf{7.} \textit{Det finns inget som är så förlegat som marxismens tröttsamma fixering vid klass.} Sedan Marx levde har samhället helt förändrats, det finns nästa inte kvar någon arbetarklass. Klass spelar allt mindre roll då den sociala rörligheten ökar. Den revolutionära arbetaren och kapitalisten i hög hatt är marxistiska fantasifasoner.

 \textbf{8.}\textit{ Marxisterna förespråkar våldsamma politiska aktioner.} De väljer blodiga revolutioner framför hederlig långsam utveckling. En liten grupp kommer göra revolution, ta makten och tvinga på alla sin vilja. Marxism är motsättningen till demokrati. Ändamålet helgar medlen, oavsett hur många som stryker med.

 \textbf{9.} \textit{Inom marxismen tror man på en allsmäktig stat.} När socialisterna tagit över har de avskaffat privat egendom, infört ett despotiskt styre och gjort slut på all individuell frihet. Så har det alltid varit, alla böjer sig under det despotiska styret. Den liberala demokratin är inte perfekt men mycket bättre då man åtminstone inte blir inspärrad för att man tycker olika och vågar kritisera den auktoritära regimen.

 \textbf{10.} \textit{De senaste fyrtio åren har alla riktigt intressanta radikala rörelser uppstått utanför marxismen. }Rörelser som varit aktiva inom feminismen, miljö-frågor, hptq-frågor, frågor om etnicitet och antiglobalisering har tagit över efter den föråldrade klasskampen. Man har lämnat marxismen bakom sig och den har inget att bidra med.

}

\small{
\textbf{Varför?}
\label{fc921f8d8924448e0bb92f2a28280ee9}
 Därför!

}

\small{
\textbf{Vaselino Tittifitti}
\label{a511ba294ccb35e2507cedf859f7f562}
 är Italiens främsta gynekolog. Har varit huvudnumret på ett flertal medicinska konferenser i Schweiz.

}

\small{
\textbf{Vaskning}
\label{12eda3c33bbe1844cc47bd51e16c6d81}
 är ett begrepp \textsc{(se ledingreppet s.~\pageref{528126abac2b649bb4fdf7bd2764726f})} som nästlade sig in i svenska folkets medvetande sommaren 2009. Då handlade det om champange. Sån skit får väl överklasen hålla på med om de vill, vad vanligt folk gör är att vaska tid. Det gör man genom att istället för att göra det man ska, typ städa, arbeta eller studera, göra något helt annat, gärna icke-produktivt. Nissepedia \textsc{(s.~\pageref{62400dadecd90cb5cd39062abe5a3e4a})} skulle inte existera utan omfattande tidsvaskning.

 En intressant detalj som aldrig belystes i massmedia är att alla typer av champagneköp är vaskning, oavsett om den hälls i slasken eller inte. Allt bubbel förutom Törley gala \textsc{(s.~\pageref{896d6fc796f5eb5ed2924f0b4c6bc540})} är nämligen pengarna i sjön.

}

\small{
\textbf{Vatikanstaten}
\label{2a3f7cd77d26fb21d605c562c409d7e9}
 är världens enda kvarvarande pedofilreservat för priviligerade herrar samt världens minsta stat. Reservatet är inneslutet av Italiens huvudstad Rom, som låtit uppföra en vägg runt staten för att enklare hålla koll på dess invånare. I modern tid har staten existerat sedan Lateranfördraget 1929 då Italienska byråkrater erkände Vatikanens suveränitet och dåvarande överstepedofilen Pius XI som dess rorsman.

 Då Vatikanstatens yta endast uppmäter 0,44 km² så är det med sina ca 800 invånare det pedofiltätaste området i världen. [http://www.vaticanstate.va/EN/State_and_Government/General_informations/Population.htm]

 Reservatet finansieras av \quotetext{Peterspenningen} [http://www.signum.se/signum/template.php?page=read\&id=2906] som består av gåvor från olika församlingar som drabbats av religösa pedofilers framfart och därför vill samla en betydande del av dem i reservatet i ett försök att minska fortsatta övergrepp från gärningsmännen. De senare försöker att locka till sig nya offer genom att ha entrérabatt för barn och ungdomar i reservatets olika påstådda sevärdheter, såsom museum och Peterskyrkan. För att underlätta övergrepp anser Vatikanstaten barn inom reservatet vara byxmyndiga vid tolv års ålder[http://www.avert.org/age-of-consent.htm], till skillnad från FN och Sverige som har satt gränsen för samtycke vid sexuella handlingar till femton. Varje år vid påsk håller överstepedofilen ett tal från en balkong med lite lösryckta citat från Bibeln. Hittills har han och sina föregångare i sina tal skippat att nämna de brott som hans undersåtar varje år begår. [http://www.svd.se/nyheter/utrikes/trostens-ord-till-paven-pa-paskdagen_4521961.svd]

 Den enda ordningsmakt som finns inom reservatets murar är Påvliga Schweizergardet (\textit{Guardia Svizzera Pontificia}). De har funktionen av civil polisstyrka och består av kyska män från de scweisziska alperna. På grund av sina till utseendet clownliknande uniformer misstänks deras syfte även vara att locka in barn på området.

 Av en ren tillfällighet är Vatikanstaten även katolicismens politiska och kulturella centrum. Nuvarande överstepedofil med huvuduppgift att förringa sexuella brott inom den katolska kyrkan är påven Benedictus XVI.

}

\small{
\textbf{Ved}
\label{29e0461b02c078c89c7b2ac0b29fbfaf}
 är själva syftet och ändamålet.

}

\small{
\textbf{Vedklyv}
\label{34782a935cffabcc8964e073c921a629}
 Lättjans redskap!

}

\small{
\textbf{Vegan}
\label{792fec82e3a0dcea1817fd9ebfaf1533}
 er \textsc{(s.~\pageref{2a12d5d6ae91d2f4f7d9af3cef58e75c})}

}

\small{
\textbf{Veganer}
\label{2a12d5d6ae91d2f4f7d9af3cef58e75c}
 Människor som lider av den ätstörning man i vardagligt tal kallar veganism. Dessa gråter vid åsynen av ullvantar, vägrar vara med på bild och hyser agg mot alla som inte fiser konstant. Personerna är påfallande ofta vit medelklass, och pga att de inte äter ost får de utväxter på huvudet, sk \quotetext{dreadlocks} eller haschrökarfrisyr. Bristen på ost ger även brist på självdistans.

 Säkra vegantecken:
 Vegandarr, tron på chips och läsk som \quotetext{mat} samt
 Kommunistglasögon \textsc{(s.~\pageref{1bc58f6f6a934c05a63add653dbeadf0})}

}

\small{
\textbf{Vegankokboken}
\label{8c52672f38e80fd29ac8dbc6dbc47008}
 Även känd som \quotetext{Satansverserna} uti vilken en hel generation tonåringar fann stöd för sin ätstörning.

}

\small{
\textbf{Vem vill bli känd?}
\label{9d298877e1c7a50bae356c8557a1a7da}
 ? är ett publikt jippo och affärsidé som tyvärr aldrig (ännu) realiserats. Uppslaget för Vem vill bli känd? fick två väldigt stenade \textsc{(se stenad s.~\pageref{dec4a3a91f0f2bf8dcf033a8cfeaa554})} unga män i en lägenhet i närheten av Zinken någon gång kring 2004. Jippot gick ut på att de två unga männen skulle annonsera efter personer som ville bli kända och deltaga i ett kommande TV3-baserat program. Mer information om vad programmet skulle gå ut på fick inte allmänheten i denna fas av processen. När ansökningarna började strömma in skulle en väljas ut på måfå, man skulle ringa upp personen i fråga och stämma träff med denne. Personen skulle sedan bli ombedd att tatuera ena handen så att den såg ut som en fem-bent bläckfisk. Sedan skulle den tatuerade personen få ett trettio minuter långt program på TV3, som troligen skulle ses av alla som ville veta vad det hela gick ut på, och även bli exponerad i olika sammanhang, så som Allsång på Skansen, Hultsfred, DN på stan \textsc{(s.~\pageref{5a22ad78e75653c891a4ab0a4a94df7b})}, Vice \textsc{(s.~\pageref{03f753c08ba5ff2bd7d2ee230b4683b1})} och så vidare. Under hela processen skulle en dokumentärfilm spelas in och den skulle ha premiär lagom till dess att den tatuerade människans femton minuter i ramljuset var över. Poängen med dokumentären skulle vara att folk i allmänhet är dumma i huvudet, vilket är lite ironiskt med tanke på den sinnesnärvaro som idén kläcktes ur.

}

\small{
\textbf{Vemod}
\label{e62077d5f9bc508c9800fbc147a76f19}
 För lite vemod blir ingen gladare av.

}

\small{
\textbf{Verklighetens folk}
\label{9404cebf965f11b660c6ad262d3c2433}
 är en cover med försvenskad text på låten \textit{Common people} av den brittiska popgruppen Pulp. Covern spelades in av den svenske komikern Göran Hägglund, som dock inte uppnådde några större listframgångar med låten.

}

\small{
\textbf{Vernissage}
\label{ce23171c27a2528bf43af71778ad0046}
 är något man får gå på om man har vänner som är eller utbildar sig till att bli konstnärer. Det är ett tillfälle för de aspirerande van Gogharna att visa upp sitt ofta hutlöst pretentiösa trams och ett tillfälle för dig att dricka sanslösa mängder rötjut ur bib och svulla OLW-hjärtan. Om man kommer till evenemanget tidigt på dagen är en vernissage ett utmärkt tillfälle att skaffa sig en dagsfylla \textsc{(s.~\pageref{e79459471993abd0ccde4df08bafdb22})}.

}

\small{
\textbf{Veva med kängnäven}
\label{0b5f330433e1fc19a412718dba802627}
 Att \textbf{veva med kängnäven} är ett klassiskt tecken för att visa belåtenhet. Lirar Mob 47-Åke \textsc{(s.~\pageref{486ee67ac39debabed3d92a7555dcebd})} ett grisigt solo? Veva med kängnäven! Ser du en träskpunkare \textsc{(s.~\pageref{484838b3db1adb135ea74d6fc61e44c0})} som badar i sin egen avföring? Veva med kängnäven! Lyckas du sparka av backspegeln på en sportbil och komma undan med det? Veva med kängnäven!

 Att veva med kängnäven ska inte förväxlas med att hytta med näven \textsc{(s.~\pageref{dabb9466fffc72b8eec1d4616f32d62e})}.


 Källa: Prof. Etienne \textsc{(se Användare: Prof. Etienne s.~\pageref{a9878d2280e5a39becac8f73d113df91})} - \textit{Hur man blir en vinnare utan att behöva vara nykter i onödan}. Alla barns bokklubb, Säffle 2003.

}

\small{
\textbf{Vevlira}
\label{8a9ef383edf77146d263989874102260}
 HEAD3: Gammal betydelse: Musikinstrument
 Vevlira brukade vara namnet på ett stränginstrument som spelades flitigt under medeltiden. Det speciella med instrumentet var att du vevade på en vev som var lite som en cirkelformad stråke som gneds mot instrumentets strängar för att skapa ljud. Tonerna på ljuden påverkades genom att trycka på knappar, lite som på en nyckelharpa. Instrumentet användes flitigt av drone-skalder, men efter medeltiden så svalnade intresset för instrumentet, då folk börjat tröttna på drone för att de skotska \textsc{(se skottar s.~\pageref{c2e5f84c76d823ea9482387bfb950791})} drone-pionjärerna ännu inte skickats till Australien \textsc{(s.~\pageref{e727d8d1b3162a732c7f706d55de64f3})}.

 HEAD3: Ny betydelse: Gitarrteknik
 Efter många diskussioner bland kulturhistoriker så har man nu bestämt att vevlira ska vara namnet på den spelteknik som Pete Townshend, gitarrist, sångare tillika materialförvaltare i rockgruppen The Who, använde. Han slog an strängarna på sina gitarrer (märk plural; han var känd för att byta gitarr lika ofta som han bytte underkläder, orsaken till detta är omtvistad, men ibland slog han an gitarren så hårt att den gick sönder) genom att veva med armen likt en propeller \textsc{(s.~\pageref{5eba517a2887595e2fd711e32090a0a7})}. Det gav honom precis det twang i ljudet han var ute efter.

}

\small{
\textbf{Vhs}
\label{115751c3a0f5273a9e039b96b44250ae}
 Ett på alla sätt överlägset format för film vilket naturligtvis för en tynande tillvaro, som allt annat som fungerar.

 För entusiasten finns alla filmer från 2005 och bakåt att köpa på närmsta källarloppis för 10 spänn.

}

\small{
\textbf{Vice}
\label{03f753c08ba5ff2bd7d2ee230b4683b1}
 är en dryg tidning.
 De gillar folk som har vargar på sina t-tröjor och de gillar töntar av princip men inte på riktigt.
 Tidningen Vice borde förbjudas snarast.

}

\small{
\textbf{Videotex}
\label{4dbd2386bf7ea2190fb4d03e6efb4775}
 (i Sverige även kännt som Datavision och Teledata) var från ca 1980 Televerkets stora satsning på ny informationsteknik. Användarna (både företag och privatpersoner) kunde nå centralt lagrad text och enkel grafik via telenätet, och ta del av informationen genom en bildskärmsterminal liknande en enkel dator. Tekniken var enkelriktad så att användaren bara kunde hämta hem information men inte sända någon egen.


 Informationen presenterades i en trädliknande sidstruktur och hade en grafisk form som påminner om Text-TV. Svenska massmedieföretag, bland dem TT och flera landsortstidningar, engagerade sig tidigt i den nya tekniken, bl a med nyhetstjänster. Fagersta-Posten \textsc{(s.~\pageref{e879bdcb386d850bb5606058db7464d4})} engagerade sig dock aldrig. Även Posten byggde under namnet Postel upp tjänster som gjordes tillgängliga genom publika terminaler. Tekniken begränsades av låga överföringshastigheter och att den enkla grafiken inte tillät några bilder. Bland de tjänster som överlevde längst fanns en koppling till bilregistrets databas.


 Videotex blev aldrig någon succé i Sverige och lades ner i början av 1990-talet. I  Frankrike, under namnet Minitel, hade man \textsc{(s.~\pageref{39c63ddb96a31b9610cd976b896ad4f0})} med viss framgång använt systemet, med bl.a. gratis terminaler till hela befolkningen. Det mest använda området där var dock enklare former av så kallade \quotetext{heta linjer} där användaren tog emot korta erotiskt orienterade meddelanden.


 Videotex skrala genomslag berodde antagligen på att tekniken var relativt dyr att köpa in jämfört med en vanlig telefonkatalog, som i princip kunde uppfylla samma tjänster.

 [http://www.videotex.se/]

}

\small{
\textbf{Viggen}
\label{12f3e7475287206dd63fde85d6669552}
 är alls icke ett flygplan, som många tycks tro, utan är en mustaschprydd \textsc{(se mustasch s.~\pageref{78fe8e02985abb5090cb3f33ac2842d4})} man som för närvarande är hemmahörande i Umeå \textsc{(s.~\pageref{bd1e37dc477bb704c667ed1a4606df71})}. Enligt källor närstående denna person liknar han en rakkniv, fastän han är en människa.

}

\small{
\textbf{Viggo}
\label{dc82a479996d392e01b88c6f82144f15}
 Sjukt underskattad snubbe.

 Se även: viggen \textsc{(s.~\pageref{12f3e7475287206dd63fde85d6669552})}

}

\small{
\textbf{Viktiga papper}
\label{810193ff4e7ae05223a81e960d806ddf}
 ska, om de inte förvaras i bakfickan \textsc{(se bakficka s.~\pageref{d259b5ebe8541b74129f0c78a82335b7})}, alltid sättas in i en pärm. Detta går till så här:
\begin{enumerate}
\item Avlägsna eventuella kuvert och liknande.
\item Placera försiktigt pappret i en hålslagare (som kan köpas från välsorterade varuhus eller stjälas från privata arbetsgivare).
\item Kontrollera att pappret ligger rätt.
\item Stansa hål i pappret.
\item Öppna pärmen och öppna pärmens metallöglor.
\item Lägg pappret så att två av öglorna penetrerar två av hålen i arket.
\item Stäng metallöglorna och pärmen.
\end{enumerate}

}

\small{
\textbf{Viktoria (namn)}
\label{9ed58b33ac5b5298d1f9ab5c6c364176}
 Viktoria hette typ varannan tjej under 90-talet. Således kommer namnet alltid vara förknippat med buffalodojor, jazzbyxor, smygröka bakom knuten, äcklig r \& b och misslyckade discon (misslyckade eftersom Viktoria sket i dig om du var kille och frös ut dig om du var brud. Hon hade antagligen askul på alla discon).
 I dag heter inte alls lika många Viktoria, vilket beror på att många människor byter namn hela tiden. Ett annat möjligt scenario är att folk som heter Victoria har haft en onormalt hög dödlighet men att kopplingen mellan namnet och dödstalet inte gjorts av vetenskapen.

}

\small{
\textbf{Vild-Hasse}
\label{fd620cd014060b0610e819a6b5a5b3d5}
 är en korvförsäljare från Malung, Dalarna och livnär sig på att åka runt på marknader och sälja korv och annat som görs av vilt. Vild-Hasses försäljningstaktik bygger till 90\% på att skrika, eller som han själv kallar det, munvighet. Detta har gett honom epitetet \quotetext{Korvens främste skald} och även \quotetext{Korvarnas konung}. När Vild-Hasse inte åker omkring som en annan gårdfarihandlare så kan man hälsa på honom i hans hemmarn Bengtgården där man kan besöka hans lada Vindarnas Tempel och verkligen känna korvatmosfären kring sig.

 HEAD2: Nissequotes

 \quotetext{I Guds skafferi finns det bara korv}

 \quotetext{Om faror förfära
 och nöden tränger på.
 Om hungern står nära
 och vännerna gå.

 Om vänner Dig sviker
 kan hända dom gör.
 Ät lite korv.
 Inget ska då rubba
 Ditt goda humör.}

}

\small{
\textbf{Vilskita}
\label{d8991eedd83b1eb75ae7c2cf9daaad92}
 Träcka i gemytligt tempo. Närapå sävligt.  Med fördel läsandes en \textit{Kalle Anka \& Co} eller någon form av faktabok. En äkta vilskitare lämnar inte avträdet förrän benen somnat minst en gång.

}

\small{
\textbf{Vimpel på pinne}
\label{9d4d1b6ad76984908d499dfa587d6753}
 är det tydligaste beviset på att någon är glad eller har roligt.

 Ej att beblanda med ödla på pinne \textsc{(s.~\pageref{9159305fc9033e2af7fee11a993874d9})}.




 {{Utmärkt}}

}

\small{
\textbf{Vin}
\label{62911ad86d6181442022683afb480067}
 är en blandning av kroppsvätskor och druvsafter, odlade på olika exklusiva håll i världen. Dess främsta egenskap är att det orsakar vinfylla \textsc{(s.~\pageref{aee462fab19723e71e7f1f3302309d1e})}. Vin framställs genom en komplicerad process som inbegriper både det ena och det andra. Ofta är det någon utomeuropeisk människa som får dra det tyngsta lasset i själva framställningsprocessen, men å andra sidan får en vit människa dricka själva slutprodukten, så på det viset utjämnar det sig i den nyliberala värld vi onekligen lever i. Vill en lära sig mer om vin är det enklast att fråga en vinkännare \textsc{(s.~\pageref{d5158bbf539f3ff78e895f04d24c4d86})}.

}

\small{
\textbf{Vincent Bugliosi}
\label{479f851e8a149e8df76a8f29707554e3}
 , född 18 augusti 1934 i Hibbing, Minnesota, är en jänkare som jobbade som åklagare i rättegången mot Charles Manson \textsc{(s.~\pageref{068614c686e26d839098938aa4b847d2})}.

}

\small{
\textbf{Vinfylla}
\label{aee462fab19723e71e7f1f3302309d1e}
 Liggandes på divan \textsc{(s.~\pageref{a01d1d03f174a0e65c9d2e21afccb478})}. Huvud \textsc{(s.~\pageref{e906cd95a540df9b16d0460fb4cf0adc})} böjt lätt bakåt. Vindruvor i klasar ovan din öppna mun \textsc{(s.~\pageref{6585f290ce92c3de5ff339920330e26f})}. Känner du begär? Tillfredsställ dem. Ljummen vind blåser in under vit toga. Mjuk lust. Du driver runt. Ett tillstånd av långsam extas. Röda löften om ömhet droppar fram mellan våta läppar.

 \textit{Vinfylla}
 -doftande andedräkt.]]

}

\small{
\textbf{Vingmutter}
\label{3d474f53ae61d29d3b924d44c21410b5}
 En vingmutter är en person med mycket utstående öron. Ett exempel på en sådan är Prins Charles \textsc{(s.~\pageref{9b545432192db93ace6632261cc410eb})} av England.

}

\small{
\textbf{Vinkännare}
\label{d5158bbf539f3ff78e895f04d24c4d86}
 En vinkännare är en person som inte bara kan mycket om vin \textsc{(s.~\pageref{62911ad86d6181442022683afb480067})}, utan till och med känner och är på \quotetext{du-}-basis med denna kontinentala dryck. Vinkännaren är så van med att diskutera om och fundera på vin att hen inte längre orkar med att uttala förledet vin- i ordet vindruva utan säger \quotetext{en druva}. Medan vi andra väljer vin efter en kombination av lågt pris och rolig etikett vet vinkännaren precis vad tillfället kräver. Medan vi i bästa fall kan känna skillnad på spanskt vin och spanskt lättvin \textsc{(s.~\pageref{00415b996c8c85901b16f9c8c687342b})} kan vinkännaren, om så behövs, skapa komplicerade, kodade budskap genom att rada upp vinslattar av olika slag inför en annan vinkännare. Vinkännaren \quotetext{luftar} inte sällan sitt vin och låter det \quotetext{stå och dra till sig} medan vi andra rycker den lilla foliepluggen från kranen på Foot of Africa-lådan redan på farstukvisten och innan vi fått av oss lovikavantarna. Om vi inte bara köper en kasse Kung och Sofiero som brukligt, vill säga.
 HEAD2: Se även
 Vinfylla \textsc{(s.~\pageref{aee462fab19723e71e7f1f3302309d1e})}
 kir \textsc{(s.~\pageref{002e1a6e54da86cabc77fbb474c2df49})}

}

\small{
\textbf{Violas}
\label{bc0b8c20b7ac9de2bb42b4c7285e93ba}
 Eftr., populärt kallat Violas, är en kvinnoklädesaffär \textsc{(se kvinnokläder s.~\pageref{d09ebf3842ad7452891cf646bf47b3a0})} i Blattnicksele, Sorsele socken. Här saluförs samma kläder som i de flesta andra klädesaffärer, med ett fantastiskt undantag, Maud Olofsson-trosan! \textsc{(se Maud Olofsson s.~\pageref{eb913a2e9be929654908a05017401bd6})} Denna trosa är en helt vanlig trosa, men har förstärkt gren. Historien bakom detta stycke underkläder börjar när den käre ministern besökte Violas 2002. Föreståndarinnan Lena \textsc{(s.~\pageref{8dbc672497bdc46f88e864bb1121232c})} tyckte gott att ministern kunde köpa något varpå kärringen svarar \quotetext{Men vad ska jag köpa, jag har ju en egen kläddesigner!}. Vi tar det igen \quotetext{Men vad ska jag köpa, jag har ju en egen kläddesigner!}. Här har vi alltså Maud \quotetext{De kvinnliga småföretagarnas \textsc{(se Ivriga små bävrar s.~\pageref{6d10ab1ba7bd378ba7cc1629ddf2bbde})} skyddshelgon} Olofson som tvekar att bedriva kommers med just en sådan, man blir ju äcklad! Tillslut, efter allsköns tjat, köpte hon det par trosor som sedan den dagen är känd som just, Maud Olofsson-trosan.

}

\small{
\textbf{Visitkort}
\label{07901352e2cfb4976fc023413c86711e}
 Ett på lite hårdare papper tryckt dokument som wannabees överlämnar till varandra i tid och otid.
 Där finns förutom namnet på ägaren ofta även mobiltelefonnummer, mailadress och någon konstnärlig bild.

}

\small{
\textbf{Viskositet}
\label{17328a3aa2e9e596e033ccebf7995cc1}
 är ett mått för att beskriva vätskors \quotetext{tröghet}. Desto tjockare en vätska är, desto högre viskositet har den. Eftersom viskositetsbegreppet uppfanns av en tysk (Viskosimund Fritzl år 1847) är det extremt detaljerat, och ämnet beck har exempelvis fyrtiotusen miljarder \textsc{(s.~\pageref{c2160bffc9c5ca88e77204672e62e489})} gånger högre viskositet än vatten. Måttet har kritiserats för att vara svårt att använda praktiskt och belackare tar vanligtvis upp det faktum att man inte ens kan fastställa viskositeten i en vanlig människas snor eftersom ämnets tröghet varierar mellan inner- och yttervägg i näsan \textsc{(se näsa s.~\pageref{eb02670054310d89c985dfe12c3ba7b8})}.

}

\small{
\textbf{Vit månad}
\label{0053dddaeda1834a1640905d3d61254d}
 Den finska motsvarigheten till Ramadan.

}

\small{
\textbf{Vita Bergen}
\label{1178051b36f8061216356d69d38ea503}
 Alla. I Norrland, på vintern.

}

\small{
\textbf{VK}
\label{5d44a032652974c3e53644945a95b126}
 (Västerbottens-Kurien) är en tidning i Västerbotten (såklart).

}

\small{
\textbf{Vladimir Krutov}
\label{1136787336c890e5600ba962d8a34dd2}
 är en rysk hockeyspelare.

 Under en period bodde Krutov i Östersund och har då rapporterats köra en Ford Fiesta. Därifrån slogangen \quotetext{Ford Fiesta - om det duger åt Krutov så duger det åt dig!}

}

\small{
\textbf{Vodka}
\label{234eaecc0efd4460dfeb921a50feea08}
 Ett finare ord för brännvin \textsc{(s.~\pageref{ff49ececa32cff978496a39635496f46})}. Ordet används främst av folk som vill hålla sig förmer och distansera sig från den obildade pöbeln. Dessa människor köper Smirnoff i tron att det är \quotetext{rysslands bästa vodka} men klarar inte av att uttyda \quotetext{made in UK \textsc{(s.~\pageref{c2431a9d201559f8de1dcfb6a9dd3168})}} på etiketten ens innan man satt i sig innehållet.
 Vanligt hederligt folk dricker Renat,det är gott till all mat.

}

\small{
\textbf{Volvo 240-serien}
\label{9a8db892f6b42596f2abce57f62a6399}
 Sammanlagt någonstans kring 4,5 miljarder kilo ren körglädje, gjutna i så gott som hela stycken av världens mest skötsamma stam arbetare.

}

\small{
\textbf{Volvo 740}
\label{e262951543da05bac43c7b87235a115c}
 Bil \textsc{(s.~\pageref{b3188f47d2eac7efc3f1258dc673a9fe})} som oftast går \quotetext{likt en traktor} - driftsäkert och utan krångel. Volvo 740 är även bra på så sätt att den inte behöver omvårdnad likt dess syskon 360 och 480. En annan fördel med just 740 i jämnförelse med andra svenska bilar, som saab, är dess yppeliga bakhjulsdrift som tillåter föraren att köra med \quotetext{ställ} genom kurvor. Med en 740 i ägarregistret kan du även stoltsera som volvoraggare och om inte det är ett plus i kanten så vet jag inte vad som är det.
 Volvo 740 tillverkades mellan våren 1984 och hösten 1992.

}

\small{
\textbf{Vrålonani}
\label{50836a8ff5c619c0a1ace80d4a5edc1b}
 I Malmö är det vanligt med grannar som ägnar sig åt vrålonani.

 [http://www.aftonbladet.se/nyheter/article13449933.ab]

}

\small{
\textbf{Vuxennapp}
\label{d842277d157268c94379dbd3624b8e4c}
 Något som vuxna människor i åldern 18 - 60 måste hålla på och snutta med då dom känner sig otrygga eller uttråkade. tex iphone \textsc{(s.~\pageref{0b3f45b266a97d7029dde7c2ba372093})} rökning snus.

}

\small{
\textbf{Vädret}
\label{495336a8161e7cecbcc37f2f9a7745f3}
 Molnen dominerar i större delen av landet, men lokalt är det klart och kallt. Nu till natten kan det friska i ordentligt på kalfjället, i alla fall i Jämtland och Lappland, men ett snöfall kommer in västerifrån. Det tar sig vidare österut fram till imorgon och en del av snöfallet drar sig också söderut över landet, men det blir förmodligen bara en eller ett par centimeter snö. Mildare luft följer i norr men lokalt kan det forfarande vara kallt, så stora temperaturskillnader i norr imorgon.

 Källa: Väderleksrapporten, SVTplay.

 Se även: Vädret som socialt fenomen \textsc{(s.~\pageref{b3c99cb81b6628bacd5037a4e24ae198})}

}

\small{
\textbf{Vädret i Jönköping}
\label{1a77b58e15fe502d4212703c364c5152}
 Temperaturen i Jönköping räknas ut med följande ekvation:

 (Fredrik boltes \textsc{(se user:boltes s.~\pageref{60cdffbd00ad87d0d682a7d997999ccd})} utsago) - 7° = (Faktisk temperatur i Jönköping)

 Category: Meteorologi \textsc{(s.~\pageref{4908b249d4d23a84ba5c4d25bf8c420c})}

}

\small{
\textbf{Vädret som socialt fenomen}
\label{b3c99cb81b6628bacd5037a4e24ae198}
 Vädret är, i alla fall i Sverige \textsc{(s.~\pageref{b1999637949ed135b2ca03f3a38460cc})}, ett oerhört populärt samtalsämne som sällan ter sig kontroversiellt. Många vänskaper och/eller kärleksrelationer har inletts med frasen \quotetext{Men vilket väder vi fått!}. På senare tid har dock vädret avtagit lite som det självklara samtalsämnet och istället till viss del byts ut mot det lite mer moderna \quotetext{Roliga videoklipp på internet \textsc{(se World Wide Web s.~\pageref{3b7d657e8b7bf25a9d524b60d9bb17df})}}.

}

\small{
\textbf{Vältagravstensfull}
\label{38b98456134208680c37fb10c911ee6a}
 När en person, oftast en man är sexuellt frustrerad och berusad på trettio-fyrtio starkpilsner och går lös på en kyrkogård.

}

\small{
\textbf{Vänort}
\label{862bae45805fb0236db8bf3ae4ae1a0e}
 En vänort är en kommun i ett främmande land som din kommun har bestämt ska vara kompis med dig.  Hela din klass ritar teckningar och skriver brev till en annan klass som sitter i en lika dan skola fast i en del av världen långt, långt bort. Kanske i Danmark \textsc{(s.~\pageref{5331d7fd27772396f412a5b6d19bad44})} eller varför inte Estland, ett spännande land på andra sidan Östersjön. Syftet med vänorter är framförallt att bevara freden i världen. Säg väl den människa som skulle vara så hjärtlös att hon sköt en person som hon tvingats skicka slarvigt färglagda huvudfotingar till under hela sin uppväxt?

 Vänorten har ofta någon typ av likhet med din egen kommun för att banden ska kännas starkare. Det kan vara att det bor ungefär lika många människor där, att det är lika långt avstånd mellan dem från båda hållen, eller att båda kommunerna ligger i sovjetiska satelitstater. Vänorterna Jämsä och Fagersta \textsc{(s.~\pageref{008e08fd02751800f729d6fa6f75a857})} har till exempel båda ungefär 20.000 invånare och är finskspråkiga. Stockholm nöjer sig naturligtvis inte med ett grannland utan låter sina barn brevväxla med Mordor.

}

\small{
\textbf{Vänsterhäntas dag}
\label{365f625ecf97b9adfea0fbd029c25003}
 infaller den 13 augusti varje år. Den instiftades 1992 av lobbyorganisationen \textit{Lefthanders' Club} och kommer alltså som mycket annat nytt från den engelskspråkiga världen. Målet med instiftandet av en speciell dag för vänsterhänta är att uppmärksamma de många svårigheter som denna minoritet lider av. Bland dessa utmärker sig svårigheter
 \begin{itemize}
 \item att knäppa braxorna
 \item att skjuta tävlingsluftpistol
 \item att äta ordentligt på fina middagar
 \item att klippa med, för vänsterhänta, greppovänliga saxar
 \item att Left-handa \textsc{(s.~\pageref{ccb801aab0e3e02a30e125d2c414410e})} (för då skulle det ju heta right-handa)
 \item att bli professionell armbrytare
 \end{itemize}

 Ett förslag till firare av vänsterhäntas dag är att samtidigt glädjas åt
 \begin{itemize}
 \item att man har fördel i fäktning och boxning
 \item att man har ett slags lådsashandikapp som berättigar till uppmärksamhet men inte samtidigt gör livet outhärdligt
 \end{itemize}

 Firandet av vänsterhäntas dag har aldrig spårat ur i oroligheter, än. Men vad framtiden bär i sitt sköte \textsc{(se framstjärt s.~\pageref{843a33cdc4d90488cea3030f8b941e08})} är det som bekant inte möjligt att veta.

}

\small{
\textbf{Vänstersida}
\label{341f764c1eb359a1f6120f92693342ad}
 är den som sitter till vänster om mitten, och i kroppens fall helt meningslös om man bortser från att man skulle se skitful ut med bara en högersida.

}

\small{
\textbf{Världens mest målade baseboll}
\label{276f06d42c48fbf3538409cfaf055efe}
 har över 23000 lager färg på sig. Troligtvis är den världens mest målade boll alla kategorier men på detta saknas källa. Den finns i Alexandria, Indiana, USA och började målas för lite mer än 35 år sedan av dess ägare Mike Carmichael. För bara 10 \textsc{(se tia  s.~\pageref{e7292d5ba58672ce7f6fc3c0b646ab63})} dollar kan den som vill dedikera ett lager färg till någon i sin närhet, ett sant bevis på kärlek. Tidigare användes främst sprayfärg men numera rollar man på lagren. Den som vill veta mer om denna fantastiska boll kan besöka dess hemsida: [http://ballofpaint.freehosting.net].

}

\small{
\textbf{Världens näst ondaste band}
\label{118e700810f4178191a337d7d422b007}
 är ett begrepp som tillskrivs det band som för tillfället inte räknas som världens ondaste band, utan kvalar in på en andra plats.

 Rankningen av band beror på subkultur, då olika subkulturer rankar band efter olika epitet. Fans av grindcore brukar prata om världens snabbaste band (Napalm Death), fans av kängpunk brukar prata om världens råaste band (Disclose), fans av stoner och doom \textsc{(s.~\pageref{b4f945433ea4c369c12741f62a23ccc0})} talar om världens tyngsta band (Sleep), och fans av experimentell musik brukar prata om världens konstigaste band (Locust, numera utmanade av Thrones).

 Då det kommer till Black Metal så är denna debatt ovanligt, tja, debatterad. Vid en första anblick kan det te sig helt oproblematiskt att utse denna titel. Någon kanske säger \quotetext{Mayhem är världens ondaste band} varpå någon annan säger \quotetext{Nej, Burzum är världens ondaste band}. Då skulle alltså Mayhem vara världens näst ondaste band och Burzum det ondaste. Tredje part kommer då in och påpekar att Burzum inte är ett band utan ett soloprojekt, och diskvalificerar således Burzum från platsen. Då borde alltså Mayhem vara världens ondaste band och kanske Emperor det näst ondaste. Problemet är då att Mayhem har bytt medlemmar jätteofta och det då måste fastställas vilken konstellation som var ondast. Därmed diskvalificeras Mayhem, och sådär håller det på.

 För att försöka komma fram till en slutsats här så verkar koncensus på Flashback vara Burzum, medans Close-Ups chefredaktör Robban Becirovic säger Watain \textsc{(s.~\pageref{bc7e444bdae25d4580d9478d39e7ceb3})}. Således blir världens näst ondaste band: Burzum, eller Watain \textsc{(s.~\pageref{bc7e444bdae25d4580d9478d39e7ceb3})} då det är allmänt känt att Nifelheim är världens ondaste band (även om vissa \textsc{(se Användare: Spike s.~\pageref{9ddbaf7eff59710301eb492ddab8efb0})} menar att det i själva verket är Royal Downfalls republikanska dödspop som tar hem bucklan för det ondaste i musikväg).

}

\small{
\textbf{Världens näst största byggnad}
\label{3a66962b6aa9d503a3076b68cc261d23}
 Folkets palats är med sina 12 våningar (86 meter högt), minst åtta källarplan (alla inte färdigkonstruerade) och 1100 rum världens näst största byggnad belägen mitt i Bukarest. Endast Pentagon är till ytan större. Palatset är ett uttryck för Nicolae Ceauşescus megalomani som gick upp i varv efter ett statsbesök till asien och Nordkorea 1971. För att finansiera bygget togs på 80-talet stora lån från västvärlden på folkets bekostnad; ransonering av livets alla förnödenheter infördes och inhemska varor sattes på export för att bekosta avbetalningarna. För att bygget med tillhörande boulevarden \quotetext{Socialismens seger} skulle få plats beslöt arkitekterna att riva 7000 byggnader, en femtedel av dåvarande Bukarest centrum.  1984 startade byggandet och det sägs att 200 000 arbetare knegade i skift om tre för att hinna med deadline som var satt till 89. [http://www.bok.nu/Peter_York/Hemma_hos_diktatorn]

 Ironiskt nog stod merparten av byggnaden klar, undantaget de många våningarna under jord, mitt i revolten dagarna innan Ceauşescu med fru serverades en kärve bly till lunch den 25e december -89. [http://www.youtube.com/watch?v=2yHToKLUbBg]

 Förste person att använda huvudbalkongen Ceauşescu tänkt använda för att tala till folket blev Michael Jackson. Han hälsade ett hav av fans med att skrika \quotetext{I LOVE BUDAPEST!}, varefter han utbuad fick åka hem till USA med sin helikopter från palatsets tak.[http://www.bucharest-life.com/bucharest/palace-of-parliament]

 Under turistvisningar får man se ca 5\% av byggnadens innanmäte, som idag efter revolutionen 1989 bla rymmer Rumäniens parlament. Herrelösa hundar sägs springa i korridorerna till många turisters förvåning.

}

\small{
\textbf{Världens tråkigaste skämt}
\label{9a91dfbcab537399153c24172db504e3}
 är, efter omfattande empiriska studier, frasen \quotetext{Working hard or hardly working?}.

}

\small{
\textbf{Världsmusik}
\label{9cd49f18bb03335a937129c4694d9027}
 är musik skapad av folk som inte är vita och som inte kommer från USA eller Västeuropa. Säg att ett gäng Senegaleser sjunger och spelar gitarr- \textsc{(se gitarr s.~\pageref{a08bf8420208934bc59c7ed7385d4308})} då har du just hört världsmusik. Säg till exempel att du hör en indisk man nynna på en truddelutt...världsmusik! Så många som nittio procent av alla världsmusikskivor har en strand med palmer på framsidan och på baksidan en bild på ett glatt gäng sorglösa svarta musiker som umgås förbehållslöst.

}

\small{
\textbf{Värme}
\label{95c69130bbd7d5a267734172f823f890}
 är motsatsen till kyla och är alltid ganska gött.

 HEAD2: Värmens historia
 De gamla grekerna \textsc{(s.~\pageref{4a5fb3d6ce79b5ff43b33f8f7d843672})} trodde att värme var ett element som bestod av vassa partiklar som därför liksom skar sönder ved \textsc{(s.~\pageref{29e0461b02c078c89c7b2ac0b29fbfaf})} och annat antändligt. På 1700-talet formulerade tysken \textsc{(se den tyska mustigheten s.~\pageref{682ccd5fdc3aff0c97e8845c3d6b6ca8})} Georg E. Stahl Flogistonteorin, som gick ut på att värme var ett ämne som fanns i det mesta som var antändligt. Om något avgav väldigt mycket värme vid brand som t.ex. grillkol hade det en hög flogistonhalt. Svensken C.W. Scheele upptäckte sedan grundämnet syre som fick fransosen de Lavoisier att poängtera att förbränning är när olika ämnen förenas med detta syre. Så där fick han så han teg, Stahl. Dock bör påpekas att han var död och hade tigit i vilket fall. de Lavoisier själv föll offer för giljotinen under skräckväldet \textsc{(s.~\pageref{e615f6ed8b5b62ee69c6a48a4068a682})}. Sensmoralen är att allt går åt helvete på slutet.

}

\small{
\textbf{Västerbotten}
\label{d4b008c5143dcffb6b8c35f3876c2a19}
 är en plats där man arbetar. På söndagar \textsc{(se söndag s.~\pageref{85b2e5c3758394a24221d1abac79191a})} skäms och ber man, resten av veckan arbetar \textsc{(se göra rätt för sig s.~\pageref{c8c01e0e8b4ad8e5ff6011b8af6405a5})} man. Här tågar ortordoxt marxistiska skogsarbetare, uppgödda på vegansk \textsc{(se veganer s.~\pageref{2a12d5d6ae91d2f4f7d9af3cef58e75c})} husmanskost, in i den täta granskogen för att ideellt bidraga till rikets exportnäring. I landskapets enda tätort, Umeå, regerar socialdemokratins utsände missionär, Lennart Holmgång, i samråd med entreprenörskråets representant, Krister Olsson, över bortskämda kräk med akademisk examen. Här tillåts ingen bakåtsträvande och tillväxtfientlig folkvilja stävja bygg-entusiasmen. Kommunens budget för att godtyckligt flytta omkring olika kommunala institutioner säkras bland annat genom att man årligen låter riva upp och asfaltera om stadens samtliga gator, vilket skapar arbetstillfällen och därmed skatteinkomster. Landskapates många karga bergssluttningar har av dess rättmätiga ägare, det Stockholmska \textsc{(se Stockholm s.~\pageref{edcd259e0a03c7ab70feb186bae19f13})} näringslivet, upplåtits till den tålamodsprövande men exotiska urbefolkningen som där vallar sina renar till trolltrummans sataniska rytm \textsc{(se doom s.~\pageref{b4f945433ea4c369c12741f62a23ccc0})}.

 Västerbotten är, som den uppmärksamme redan slutit sig till, en kulturell smältdegel där samiska snöskotermekaniker \textsc{(se skoter s.~\pageref{b1120baa83f380cd42a805a4e823cb1b})} lever sida vid sida med laestadianska punklegender och akademiska stjärnskott som lämnat prestigefulla utländska läroverk, lockade av universitetets \textsc{(se universitet s.~\pageref{11dfc744fa396b961a6cc9cf89c4eaea})} rektors oförtröttliga nigande inför det annorstädes förfördelade storkapitalet (jmfr. Balticgruppen) \textsc{(se Balticgruppen s.~\pageref{9cf3b30a8e655c9893ae8d71505e72ea})}. I inlandet har allianspartierna i Sveriges \textsc{(se Sverige s.~\pageref{b1999637949ed135b2ca03f3a38460cc})} regering låtit den sociala ingenjörskonstens ivriga små bävrar \textsc{(s.~\pageref{6d10ab1ba7bd378ba7cc1629ddf2bbde})} bygga ett uppsamlingsläger för varghatande Volvoägare och arbetslösa thailändska ungmöer. Det för bygden karaktäristiska svårmodet upplättas av den frihetsälskande lokalpatriotism som tar sig uttryck i skändandet av lika fridlysta som ihjälplågade lodjur.

}

\small{
\textbf{Västerbottensgård}
\label{e31e9a9f06f91f5c673e19e1739f79bc}
 Boplats byggd av träd, gräs och stenar \textsc{(se Träd, Gräs och Stenar s.~\pageref{82a271b29bea1b3fd0073fe6668179bd})}.

}

\small{
\textbf{Växjö}
\label{2fc07b846123d1c41a4c7eb55c40df40}
 är en svensk stad som ligger i Småland eller ungefär där. Trots att Växjö är en ganska gammal stad finns det faktiskt inte så mycket att berätta om den. När Växjö omtalas i populärkulturen är det oftast för att Electric Wizard konstigt nog gjorde sin första sverigespelning där, och för att staden är en av få som har en växande population av stonerskins \textsc{(se stonerskin s.~\pageref{b94c65dba2990b3146c2bedf663e9989})}.
 
 Om man tar bort X:et i namnet och istället lägger till två R så blir Växjö ett anagram för ”rävröj”. X kan vara förkortning för \quotetext{straight edge} och RR \textsc{(se Rainbow Riders s.~\pageref{54b5b4739e6bc150148c5019e1793413})} dess raka motsats \quotetext{rock n roll}. Om man tänker så blir Växjö också ungefär som en palindrom. Kanske är det sådana spaceade resonemang som gör att stonerskinsen söker sig dit.

}

\small{
\textbf{Våld}
\label{c01df500e07826fb356183119ff0d07c}
 Att skada andra, fysiskt eller mentalt. Skillnaden mot det närbesläktade ordet jävelskap \textsc{(s.~\pageref{46845591177f16920cd586a5baf5a625})} är att det på inget sätt är socialt accepterat att skratta åt någon som blir utsatt för våld.

 Vad som däremot är jävligt socialt accepterat är att ogilla våld. Det faktumet lyckades Anton Abele \textsc{(s.~\pageref{0906f6e1d290c547e1fb93c6ff6a0b44})} nyttja för att komma in i riksdagen trots sina blott tolv levnadsår. Vi på Nissepedia \textsc{(s.~\pageref{62400dadecd90cb5cd39062abe5a3e4a})} betackar oss oftast våld. Oftast, men inte alltid.

}

\small{
\textbf{Vår resa runt jorden med våra gomspalter}
\label{bbc38d6237915e0e5fdb65de33ea33e0}
 är en kokbok av en akademiker och en osnuten friluftsmänniska från Umeå \textsc{(s.~\pageref{bd1e37dc477bb704c667ed1a4606df71})}. Den riktar sig i första hand till människor med gomspalt och en gunst för världens alla spännande kök. Med hjälp av kokboken kan man nämligen resa jorden runt genom att laga mat från sådana ställen som New Orleans, Marocko och Polen. Boken säljer dock inte så bra och dess författare har efter publiceringen blivit utdragna på gatan och ihjälslagna av en arg folkmassa.

 Boken har nu också blivit blogg [http://gomspalt.blogspot.com], men hur författarna bloggar från andra sidan graven är lite av ett trepipsproblem \textsc{(s.~\pageref{ddfa7edb7b4169a1dc8a32b1a8ad9611})}.

}

\small{
\textbf{Völkerschau}
\label{d4fd8802a2b39b1f37f372acf2c1ba11}
 En idag undanskuffad företeelse som endast lever kvar i Stockholms \textsc{(se Stockholm s.~\pageref{edcd259e0a03c7ab70feb186bae19f13})} innerstad.

}

\small{
\textbf{W2}
\label{62d7d5184b7a313dc64255bdb8187847}
 var en föga framgångsrik pop/rock-duo från Norrtäljetrakten \textsc{(se Norrtälje s.~\pageref{7527f7dad9445013a559dc7e2a91f3b3})} som spelade både egna och andras låtar, som det brukar heta. De var verksamma under mitten av nittiotalet och var förmodligen inspirerade, åtminstone till namnet, av pop/rock-kollegorna i U2, vilket man antagligen ovetandes signalerade genom att bandets namn inte alls uttalas \quotetext{We two} som man antagligen trodde, utan \quotetext{Double-U two.} De enda andra sätt man kunnat tro att W2 uttalas på är \quotetext{dubbel-v två,} vilket inte betyder någonting, och \quotetext{Vee two,} vilket är namnet på de raketer som tyskarna använde under blitzen, när man jämnade större delen av London med marken.
 HEAD2: Fanclub
 W2 hade en enhövdad fanclub bestående av en mycket ensam vikarielärarinna som också fixade bandets antagligen enda större framträdande genom att tjata till dem fem minuter på TV4s nyhetsmorgon. Förutom detta bejublade framträdande har de också, enligt envisa rykten, spelat på Green Man i Fagersta.

}

\small{
\textbf{Warcollapse}
\label{d39e006de7daeb1166d6b6c7990582dd}
 är ett gäng gubbar som lirar crust. De kommer från Kalmar och flera av dem gillar att röka på jättemycket. Detta faktum avspeglas i flera av deras texter där THC, Stonerpunk och Divine Intoxication är några riktiga pärlor. De skriver ibland om seriösa saker också som att polisen är svin (Nightstick Raids) och att det är skit med krig (Mass Genocide), vilket illustreras särskilt väl på deras skiva Crust as fuck existence \textsc{(s.~\pageref{bd0b07abcc2f4c2a4e1aafdfed1f0e73})}.


 Trivia: I Göteborg \textsc{(s.~\pageref{0e9b11e435dd9f73e87e868667e1d6f0})} kan man köpa Warkallops på ett vegankök. Se bild nedan för källa.

}

\small{
\textbf{Watain}
\label{bc7e444bdae25d4580d9478d39e7ceb3}
 Enligt vissa Världens näst ondaste band \textsc{(s.~\pageref{118e700810f4178191a337d7d422b007})}.

 Watain vann 2011 en grammis för albumet \textit{Lawless Darkness}. Två av medlemmarna blev sedan utkastade från den efterföljande grammismiddagen på Café Opera, en innan och en efter huvudrätten.

}

\small{
\textbf{Wctbyxa}
\label{dc78f9615e53d0ddb525d3975197a781}
 Norsk högtidsdräkt.

}

\small{
\textbf{We are the world}
\label{fe7d2c587f7277c37d2d70168a2eedf9}
 Supergruppen USA for Africas mest kända singel. Otaliga är de gånger  barnkörer har framfört sången på skolavslutningar och ”alternativa” luciafiranden \textsc{(se dansk advent s.~\pageref{dc3610baedc341bc7fecde12589b848b})}. Bäst är den på slutet när Bob Dylan tjuter som en sån där get på youtube.

 Bono blev så avundsjuk på gruppens framgångar att han bestämde sig för att dra ihop det egna projektet Ireland for Africa där Irlands musikelit också skulle göra gemensam sak. Shane McGowan tackade ja till att medverka på fyllan men hade så klart glömt bort allt på inspelningsdagen. Enya visade också intresse till en början men hoppade av när hennes slagruta visade att Bonos hem låg rakt på ett currykors. Phil Lynott hade just dött så honom var inge idé att fråga och Bob Geldof ville självklart vara med men blockerades av The Edge som vägrade ha med fler gitarrister. Så i slutändan blev det som vanligt bara Bono, Edge och The Coors som än en gång sjöng något smäktade om kärlek. Låten saknar ännu distribution.

}

\small{
\textbf{Weiron Holmberg}
\label{77da44867b58597ce8cd0c64bbba40bf}
 En gigant!
 Till skillnad från dagens bortklemade scenskoleprodukter hade Holmberg en rad hederliga arbeten samtidigt som han var djupt engagerad i revyscenen \textsc{(s.~\pageref{c49ee096fcc5a9e626c3e1da73205d6d})}. Så småningom fick han spela nyckelroller i 80-talets kioskvältare såsom Jönssonligan, Sällskapsresan \textsc{(s.~\pageref{1023ca20cc8ad5b3f0233d023ad01bf5})}, med mera.
 Som alla äldre skådespelare från Göteborg \textsc{(s.~\pageref{0e9b11e435dd9f73e87e868667e1d6f0})} är Holmberg självklart kommunist \textsc{(s.~\pageref{fd9bf7896d396992b29d542a0200b800})}.

}

\small{
\textbf{Weiron holmberg}
\label{77da44867b58597ce8cd0c64bbba40bf}


}

\small{
\textbf{Wham, bam, thank you ma'am}
\label{cbfea99fba2a9103983536a2abd4d8a7}
 \quotetext{Wham, bam, thank you ma'am} är den fetaste raden i David Bowies låt Suffragette City.

}

\small{
\textbf{Wikipedia}
\label{12672e79b01e9ca7018105efb0ef871c}
 En samling lögner. Problemet med hemsidan är ju att vem som helst kan gå in och redigera på den. Vafalls! Nissepedias \textsc{(se Nissepedia s.~\pageref{62400dadecd90cb5cd39062abe5a3e4a})} huvudkonkurrent.

}

\small{
\textbf{Will}
\label{18218139eec55d83cf82679934e5cd75}
 Kalif är varelsen bakom \quotetext{Storm the Castle} [http://www.stormthecastle.com/]. Han är en mångsysslare av rang. Bland hans projekt kan räknas raketgevären \quotetext{Dragonslayer 1-3}, filmen \quotetext{The Magic Egg} och teckningskursen \quotetext{Fantasy Art School}. En annan av Wills oräkneliga kvalitéer är att han har skrivit två episka fantasyromaner vilka är utgivna på det respekterade förlaget Writers Club Press. Någonting man inte får glömma är det att han även har en outtömlig mängd information på runt tolv hemsidor. Hans kunskaper och informationsmängd om  medeltid \textsc{(se medeltiden  s.~\pageref{88cbc30c5b233d97df68b5b041ac0655})} och tillhörande fantasy sägs vara nära att få jorden att tippa över av klokhet. Vanliga frågor som tas upp i diskussioner om Will är: Hur många timmar har han per dygn? Var får han alla pengar till projekten ifrån och vad för någon sociopatisk störning har han?

}

\small{
\textbf{William Banting}
\label{0552b7de4c2c4fdb5f9c63bc63c925ba}
 var en engelsk dögrävare på 1800-talet som var förjävla fet. Under den viktorianska eran var det annars riktigt inne att se ut som en välsmord isterbuk men just i Bantings fall tyckte hans husläkare att det blivit lite för mycket fredagslyx \textsc{(s.~\pageref{2508e5844cd43a2aa6b385d72100ca2c})} och uppmanade honom att göra något åt saken. Banting tänkte att ett smart sätt att tappa i vikt borde vara att äta mindre, och mycket riktigt! Han skrev om sina forskningsrön i en bok som sålde i så feta upplagor att hans eget namn blev synonymt med viktminskning. Sveriges \textsc{(se Sverige s.~\pageref{b1999637949ed135b2ca03f3a38460cc})} bästa motsvarighet är Bantar-Björn.

}

\small{
\textbf{World Wide Web}
\label{3b7d657e8b7bf25a9d524b60d9bb17df}
 är en låt som Nick Borgen, Norges svar på Tom Jones, hittade på. Han tävlade med den i melodifestivalen 1997 och slutade på nionde plats. På melodifestivalen körade it-pinuppan Helen Wellton.


 HEAD2:  Text
 World wide web
 World wide web
 Här lever jag lycklig, här finns ingen stress \textsc{(s.~\pageref{e10a36f1a5231e597daf8f42dc1ab55a})}
 Här är min nya hemadress

 Jag väntat så länge så jag tog min chans
 Jag ville flytta nån annanstans
 Ikväll är det party, jag flyttar in
 Så följ med mig hem på ostron och vin \textsc{(s.~\pageref{62911ad86d6181442022683afb480067})}

 Oh uh oh uh ohh, jag har party ikväll
 Oh uh oh uh ohh, om du vill får du gärna följa med
 När staden krymper och du känner dig less
 Sök upp mig på min hemadress

 World wide web
 World wide web
 Här lever jag lycklig, här finns ingen stress \textsc{(s.~\pageref{e10a36f1a5231e597daf8f42dc1ab55a})}
 Här är min nya hemadress

 Ta med dig vänner till ett häftigt place
 Surfa hem till mig i cyberspace
 De vildaste drömmar vi hade förr
 Kanske finns här bakom min öppna dörr

 Oh uh oh uh ohh, jag har party ikväll
 Oh uh oh uh ohh, om du vill får du gärna följa med
 När staden krymper och du känner dig less
 Sök upp mig på min hemadress

 World wide web
 World wide web
 Här lever jag lycklig, här finns ingen stress \textsc{(s.~\pageref{e10a36f1a5231e597daf8f42dc1ab55a})}
 Här är min nya hemadress

 [http://www.youtube.com/watch?v=7y3_5Ca6nR4 Länk till framförande på Youtube]
 [http://www.youtube.com/watch?v=09lbI_QOqCY Nick Borgen i Blåsningen]

}

\small{
\textbf{Wunderbaum}
\label{cdcb21ec6725e89f532eff2b6504ed46}
 , eller Magic tree som den kallas på Brittiska öarna, är en doftprodukt som tyvärr inte alls är tysk utan tvärt om amerikansk som hängs upp i inre backspegeln \textsc{(se inre backspegel s.~\pageref{4d9c85c411e32a3a87ec9b69b7b75b70})} på respektabla bilar för att sprida välbefinnande. 1952 rullade den första granen ut från fabriken i Watertown, New York, och blev snabbt en braksuccé \textsc{(s.~\pageref{678371d35369d3d29afceb1445630833})} för upphovsmannen Julius Sämann som till vardags arbetade som mjölkbud och kemist. Originaldoften var den ännu populära tallbarr, som dominerade marknaden tills Sämann introducerade vanilj på 80-talet. I Sverige har granarna funnits sedan 1962 och är enligt företagets hemsida en symbol för ”friskhet och kvalitet”. Och det finns väl inte mycket att orda om där.

}

\small{
\textbf{X3m sports}
\label{e10ffd4ba064353fb2c008cf56052f95}
 (uttalas /extri:m spårtsh/) är inte alls sporter utan sådana fritidsaktiviteter som unga män som inte spelade fotboll höll på med på 90-talet, dvs snowboard (åka på en skida), klättervägg (ung. \quotetext{inte-nudda-mark}) och forsränning (gummibåt). x3m sports skapade, trots dess utövares bergfasta övertygelse om att det var en anti-social subkultur för obotliga ensamrävar, en mycket stor marknad som omsatte enorma summor pengar, speciellt genom att entreprenörer hällde ihop lite olika vätskor i en burk, tillförde karamellfärg och sålde det till x3m sportutövarna som gladeligen betalade i tron att det var ett slags magisk dryck (energidryck) som gav dem energi och kraft att klättra på sina väggar, åka på sin lilla skida och fan och hans moster. Utövarna lyssnade ofta och gärna på x3ma band så som Millencollin, Bad Religion och NOFX och hade sina byxor halvt nerdragna, samt mössa inomhus. Nuförtiden för fenomenet en tynande tillvaro, till nästan allas glädje, och utövas mest av Australiensare \textsc{(se Australien s.~\pageref{e727d8d1b3162a732c7f706d55de64f3})} som talar med för hög röst.

}

\small{
\textbf{Ympning}
\label{2f0620302a3ae66008d32f1a71712d13}
 är en odlingsmetod där en främmande gren eller liknande planteras in i en annan växt av samma taxonomiska familj. Metoden är vanlig vid förädling av fruktträd. Den vanligaste metoden för ympning är att en mindre gren skärs av med ett vertikalt snitt och att denna sedan förs in under barken på huvudstammen hos ett savande träd, t.ex en annan sorts äppelträd eller ett päronträd. En annan vanlig ympning är att en fisk ympas in på ett träd.
 HEAD2: Ympningens historia
 Ympning kan spåras så långt bort som till det antika grekland där det var vanligt att man ympade mer ädla vinrankssorter på roten av en mer livskraftig sort. Egentligen finns det tydliga spår av ympning på flera håll subsahariska Afrika, men på något vis känns det inte relevant.
 HEAD2: Professionell ympning
 I plantskolor och förädlingsgårdar ympar man för att skapa olika sorters fruktrräd som passar för olika kommersiella eller icke-kommersiella ändamål. Till exempel skapas fruktträd som passar olika odlingszoner och olika jordmåner.
 HEAD2: Vanliga fruktsorter som ympas
 Äppelsorter som ympas på andra äppelsorter:
 \begin{itemize}
 \item Allington
 \item Bleinhem
 \item Galloway
 \item James Greive
 \item Signe Tillisch
 \item Veseäpple
 \end{itemize}

 Päronsorter som ympas på andra päronsorter:
 \begin{itemize}
 \item André Desportes
 \item Flemish Beauty
 \item Kongresspäron
 \item Kongresspäron
 \item Sommarbergamott
 \end{itemize}

 Matbär som kan ympas på andra matbär:
 \begin{itemize}
 \item Svarta vinbär
 \item Röda vinbär
 \item Hallon
 \end{itemize}


 HEAD2: Fiskympning
 Man kan också ympa på fiskar på många vanliga frukt- och ädellövträd. Brax är det till exempel vanligt att man ympar på päron och äppelträd. Anledningen till detta är att braxen skimrar så vackert bland de dignande frukterna. Den svenska gäddan har man i århundranden ympat på fläder och cypress för att på så vis få tillskott till den magra fångst som insjöar utan inlopp bjuder på. I skansens rosenträdgård har den svenske professionell trädgårdsmästaren John Andersson som ende svensk lyckats ympa en löja i toppen av en ståtlig gran.

 Se även:
 Lista på korsningar av frukt och fisk \textsc{(s.~\pageref{fc226bf1567a0c95dc81da4185ca317c})}.

}

\small{
\textbf{Yngwiefiering}
\label{fa77dd97389254540590b9f23800970a}
 är när en person börjar skrika och skräna till synes utan anledning. Därefter häller personen i sig stora mängder sprit och börjar bete sig ännu värre. Yngwiefieringen har börjat. Det enda sätter att kurera en person som börjat yngwiefieras är att ge den en spruta bon joviaccin. Snacket kommer då fortsätta gå men aldrig någonsin leda till handling

}

\small{
\textbf{YouTube}
\label{ba9bf05693b9fa202d922dd43a08f281}
 är, tillsammans med bl.a. TV, en del i den världsomspännande hemliga konspirationen för att öka analfabetismen (och även genital- och oralfabetismen). Så att så småningom de få återstående oanalfabeterna ensamma ska kunna härska över världen. Nästa år kommer YouTube att ersätta textsökningen med att man ska vissla låten man vill höras refräng in i iPhonens mikrofon.

}

\small{
\textbf{Yxa}
\label{bd74f429522c7c1481fbba07187efc6b}
 n är ett implement som används för att hugga ner träd och buskage. Den uppfanns av Jan Guillou \textsc{(s.~\pageref{63f2c8aba9686bc92efeb7eb21e35156})} och är försedd med ett träskaft \textsc{(s.~\pageref{1ab85ecd859ae682af47bb9334c7dac6})}.

}

\small{
\textbf{Zeke Eriksson}
\label{e8be71518b9ce77355d354d6a488be13}
 är en bandyspelare i Sandvikens A-lag.

}

\small{
\textbf{ZiL-fil}
\label{cae9980abbabeaf6c72f0e2285abd439}
 är smeknamnet på de vägfiler i Moskva som är särskilt avsedda för toppolitiker och höga tjänstemän \textsc{(se storfräsare s.~\pageref{4db17005692cd83e3e946a1311b81ed0})}, till exempel Karelin \textsc{(se Aleksandr Karelin s.~\pageref{7db555630a4ad78feb3477db9b1ee464})}. Filerna anlades i mitten av 1960-talet på initiativ av politbyråns dåvarande ordförande Leonid Brezhnev, som tyckte att det var för jävla tråkigt att sitta i bilkö. Namnet syftar på det ryska limousinemärket ZiL som tillhandahöll nästan alla de pansarbilar som sovjetiska ledare föredrog att resa i. Filerna löper längst ut till vänster på flera av centrala Moskvas huvudleder och vållar ibland tyst irritation hos medtrafikanter som får vänta framför rödljuset en halvtimme för att kunna korsta en ZiL-fil (det finns nämligen en särskild trafikcentral på Kreml som kontrollerar stoppljusen runt omkring så att ingen viktig oljegark \textsc{(s.~\pageref{d78fbbc214d52206f58476f02f66f0b6})} blir stående). I perestrojkans efterdyningar lättade Gorbatjov en aning på restriktionerna så att även utryckningsfordon får nyttja filerna, men för övriga är det än idag fortfarande stopp. Vad påföljden blir för den som olovligen kör i en ZiL-fil har Nissepedia \textsc{(s.~\pageref{62400dadecd90cb5cd39062abe5a3e4a})} tyvärr inte lyckats utröna men vi tror definitivt inte att det är värt det.

}

\small{
\textbf{Zine El Abidine Ben Ali}
\label{e7e5994067ede27ffcb22349eed4c698}
 är ute på tunn is igen.

}

\small{
\textbf{Zoom}
\label{15913c103a5238e5a80ac2f498ee090d}
 är ett mestadels optiskt fenomen som förvränger uppfattningen av rumsliga avstånd. Det är exempelvis zoomen som gör att en myra blir stor som en jättemyrslok \textsc{(s.~\pageref{c6ca587e20a7103f6ea7f656968165bd})} om du tittar på den i en kikare \textsc{(s.~\pageref{e2b5cb2875a91cea4345226ce26ada44})}. Det låter lite skrämmande, för om myror var så stora skulle ju jättemyrsloken dö ut. Men det är ingen fara, för zoom är som sagt bara ett optiskt fenomen och inte på riktigt.

 För den som inte har vit rock på sig och jobbar i ett laboratorium med att läsa tjocka böcker hela dagarna, så som Nicke  \textsc{(se Användare: Jons polare Nicke s.~\pageref{8ce485f3f6dd381bcce24026b0512612})} gör, kan det vara lite svårt att förstå hur detta egentligen fungerar. Men tänk då att du ligger hemma i soffan med en Tuborg \textsc{(s.~\pageref{49bb0f04b9993881c9d9c5b115cc35f0})} i handen och lyssnar på \textit{Dark side of the moon}. Du stänger ögonen och lagom till \textit{Time} har du omslutits så pass av ljuden att du känner hur stjärnorna lyser klarare där ute och himlen kryper närmare. Din kropp blir lättare och när så \textit{Eclipse} äntligen kickar igång är det som att du och Gubben i månen är bästa polare. Det är för att du med hjälp av skivan zoomat in rymden \textsc{(s.~\pageref{6d5ad1e8996d7ec9d8ac6058649290c0})}.

}

\small{
\textbf{Äckligt godis}
\label{7a949bcbd13153b7e40bc8bf8dbb481a}
 är ett av de bästa bevisen på att valfrihet inte är den universella lösningen på alla världens problem. I motsats till vad den danske \textsc{(se danmark s.~\pageref{5331d7fd27772396f412a5b6d19bad44})} filosofen Lars von Trier säger får man med äckligt godis inte ta det goda med det onda utan det äckliga med det onda. Det räcker inte med att Karius och Baktius invarderar din munhåla och bildar fetor ex ore \textsc{(s.~\pageref{d3b96d618fb972d12fb0cdfdeaf13a98})}, du ska dessutom lida romrussinets alla kval på vägen. Precis som Lenin sade behövs här en stark kader som tar kontroll över frågan och leder folket i rätt riktning. Aldrig mera Plopp Lakrits!

}

\small{
\textbf{Ädelost}
\label{8aeecc3a132ce5d2e562fa7a2ca29a06}
 är en så kallad oxymoron (från grekiskans \quotetext{dum som en dum tjur}), alltså en sammansättning av två element som är varandras antites. Det finns inget ädelt med ost, särskilt inte med vad som kallas ädelost, då den är möglig och luktar konstigt. Akten att äta ost kan dock vara ädel om den utförs i rätt kontext (typ när man hjälper en kompis äta upp sista biten pizza på dennes talrik då resten av gänget är borta, så att vännen ska slippa skämmas, trots att man är bautamätt och håller på att sprängas. Alternativt då man, när ens respektive eller någons idiotiska kusin \textsc{(s.~\pageref{f7f20d5744925e2e72e5524035a162be})} välter ut fonduegrytan, kastar sig fram för att skydda de man älskar från den kokande osten genom att äta den i luften). Men det finns inget ädelt med att sitta och äta en ostmacka \textsc{(s.~\pageref{2e2a02f9cf463d37a5ab2cae4e0bed2a})} i köket, inte ens om osten är möglig och luktar skit.

}

\small{
\textbf{Äganderätt}
\label{2d92d92f7fa233484ba06555728bef2a}
 kan vara lite vad som helst, beroende på perspektiv \textsc{(s.~\pageref{1606dd19366985367d677f7b6de46e52})}.
 Den vidast spridda definitionen går dock ut på att dom rika ska kunna få vad dom vill från alla som inte är rika.

 Den inte fullt så spridda definitionen åsyftar palt.
 .]]

}

\small{
\textbf{Ägg}
\label{128a5feb8e12d0aa622e0298a8332980}
 (mental status) \textsc{(s.~\pageref{c85ef9d53215bd8b967cacb6d169a541})}
\begin{enumerate}
\item OMDIRIGERING Manglar som ägg (skiva) \textsc{(s.~\pageref{2f2a45e36274754b2afef2f2c96eeae6})}
\end{enumerate}

}

\small{
\textbf{Ägg (mental status)}
\label{c85ef9d53215bd8b967cacb6d169a541}
 Att kalla någon för ett ägg betyder att man anser personen vara något sinnesslö.
 Se även särske \textsc{(s.~\pageref{552a5aad891937bf760fb193900ea140})} och träskaft \textsc{(s.~\pageref{1ab85ecd859ae682af47bb9334c7dac6})}.

}

\small{
\textbf{Ägg (skiva)}
\label{1af3107fb4df4aa14f6a9815d47f1ed7}
 Manglar som ägg \textsc{(s.~\pageref{7b1e91fdfd952485ddd3bc6ef4e40b3c})}

}

\small{
\textbf{Ägmästare}
\label{8324518500d7e7ccd22ae364887d4476}
 En ägmästare är någon som, medvetet eller omedvetet, får det att äga för andra. En typisk ägmästare sprider äg omkring sig och behöver bara kliva in i ett rum eller kränga på sig en gitarr \textsc{(s.~\pageref{a08bf8420208934bc59c7ed7385d4308})} för att andra ska känna att allt plötsligt blev jävligt sjysst. I den här videon, till exempel, får Rory Gallagher det att äga för sina bandpolare, publiken och alla i hela världen [http://www.youtube.com/watch?v=zYYbK2sDaJ4].
 HEAD2: Externa länkar
 Här kan du studera en australiensisk \textsc{(se australien s.~\pageref{e727d8d1b3162a732c7f706d55de64f3})} ägmästare[http://youtu.be/P1mARlt7Njg]

 HEAD2: Se även
 sugmästare \textsc{(s.~\pageref{1a01ad3847daa7eabaa6496d5765be89})}

}

\small{
\textbf{Ägmästarnas ägmästare}
\label{d015b463458c8a36f5471807706ae4bd}
 Världen blev en otroligt fattig plats när Kim Peek gick ur tiden den nittonde december år 2009.

}

\small{
\textbf{Äkkli Päkkinen}
\label{454d40c04275024ef94ca8c95f5bbc5a}
 är Finlands snuskigaste medborgare. Han har alltid jackfickorna fulla med maggots och ibland tar han upp en näve och käkar. Det spelar ingen roll om han sitter på bussen hemma i Finkelranta omgiven av dagisbarn, han kan ändå vara hur äcklig som helst. Lika om hans gamla sjuka farmor ringer och frågar hur han har det kan han utan omsvep berätta att han sitter och pillar loss sårskorpor för att blanda i mjölkskorven \textsc{(se mjölkskorv s.~\pageref{0340144b99f5ede3c88664d279049104})} som står och puttrar på spisen. Han är verkligen ett riktigt vidro.

}

\small{
\textbf{Äkthetsbevis}
\label{461589348b1407656d28f6631889401a}
 är precis som äkthetsintyg \textsc{(s.~\pageref{880eb35bb5227d3a5f925ac6c41784d7})} beskriver men med den skillnaden att det heter bevis istället för intyg.

}

\small{
\textbf{Äkthetsintyg}
\label{880eb35bb5227d3a5f925ac6c41784d7}
 Ibland kan man få erbjudande om att köpa ett jubileumsmynt, nytillverkad klassisk klocka eller liknande, och medföljer gör ofta ett äkthetsintyg. Ett intyg på att kopian är äkta.

 (Se Äkthetsbevis) \textsc{(se Äkthetsbevis s.~\pageref{461589348b1407656d28f6631889401a})}

}

\small{
\textbf{Ängelholm}
\label{046460c66c50e827fb26873edbef1353}
 är en stad i nordvästra skåne, beläget i Kullabuktens innersta. Den är väl som de flesta andra städer av liknande storlek, förutom att SMR har klistermärken på mer än varannan papperskorg.

}

\small{
\textbf{Ärtpåse}
\label{2a6ad0bd7ec9c32684d784108094fdc5}
 En ärtpåse är ett idrottsredskap och består vanligtvis av två ihopsydda tygbitar. Mellan dessa finns en mängd torkade ärtor som förlänar påsen en viss tyngd. Ärtorna gör att påsen är samtidigt mjuk och hård på ett ganska konstigt vis.

 Ärtpåsens huvudsakliga användningsområde är olika övningar inom skolgymnastiken. I många av dessa ska två eller flera lag hämta och lämna ärtpåsar, eller \quotetext{skatter} som de då kallas, på olika utmarkerade ytor i gymnastiksalen, emedan konkurrerande lag ämnar hämta tillbaka dem. Ett ickeobligatoriskt försvårande moment är att en av deltagarna fungerar som \quotetext{von Ribbentrop} och tillåts stjäla ärtpåsar från alla lag. De ärtpåsar \quotetext{Ribbentrop} stulit räknas som kasserade och kommer ej tillbaka in i spelet.

 Mer stillsamma övningar går ut på att påsen balanseras på hjässan medan gymnastikdeltageran agilt balanserar på en bom. Ärtpåsen bildar tillsammans med det färgade tygbandet, ett viktigt element i trollkull \textsc{(s.~\pageref{900bbb023078fe24707bd4c6f8f46f95})}, basen i gymnastikämnets redskapspark.
 HEAD2: Obeskrivligt förtryck
 Ärtpåsen har ännu inte upptagits på olympiekommitténs officiella lista över godkända idrottsredskap, trots ihärdiga påtryckningar. Denna nonchalans kan med all rätta uppfattas som provocerande och ovärdig ett organ av nämnda kommittés dignitet.

}

\small{
\textbf{Ättestupa}
\label{0724235523055ba38202b1a661a68722}
 En ättestupa är en plats, ofta på ett berg, där gamla och sjuka togs av daga för att inte vara familjen eller samhället till last.  Åldringarna sätts i en pulka och skickas utför stupet till efterlivet, sen går de efterlevande vidare med sina sysslor. Inget mer med det.

 HEAD2: Ättestupans historia
 Den moderna vetenskapen menade länge att det här inte var något som hörde den moderna världen till och att det inte praktiserades någonstans annat än i historiens annaler, men ack så fel förståsigpåarna \textsc{(se förståsigpåare s.~\pageref{ff91afb86ce86124b6a517f3eb37bc18})} hade. Det hela började 2011 i Malå \textsc{(s.~\pageref{41da4620e87888eaaeafcb3004a8d177})} när den senmoderna kapitalismens oförmåga att skapa värde lett till att landstingspolitikerna inte hade något annat val än att dra in ambulansen där. Byborna protesterade högljutt till ingen nytta och fick på nytt börja släpa sina till åren komna släktingar till Erik Sjulsa-stenen vid Tjamstans \textsc{(se Tjamstan s.~\pageref{76f026797a3d868f6a32a26b28f76f8e})} södra bergsida. Gubbarna och tanterna sattes i ackjor, fick höra något i stil med \quotetext{Förlåt farfar, men vi orkar faktiskt inte ta hand om dig längre. Ha det bra!} och sen med en lätt men bestämd knuff skickas utför stupet. Denna braksuccé \textsc{(s.~\pageref{678371d35369d3d29afceb1445630833})} spred sig till resten av landet i takt med att åldringsvården i Sverige \textsc{(s.~\pageref{b1999637949ed135b2ca03f3a38460cc})} börjat gå ut på att väga kissblöjor istället för att faktiskt ta hand om gamlingar och typ ge dom mat och liknande av Carema bortprioriterade arbetsmoment.

}

\small{
\textbf{Äventyrsbad}
\label{8e36481b72c8061bb9ff74c1df3b0b66}
 Ett säkert tecken på att en kommun är på väg ner i fördärvet är när man anlägger ett äventyrsbad.

}

\small{
\textbf{Åka på safari}
\label{81c7409524ed7506f454be7fb17d4c38}
 Slang för att dricka öl med olika sorters djur på burken. För att det ska räknas som ett riktigt safari måste bolagskassen (även känd som Djungelkasse) innehålla minst fem olika djuröl. Nissepedia tipsar naturligtvis om dom bästa alternativen samt fallgroparna.

 \textbf{Illeröl}
 Formellt kallad \textit{5,2:an}. Billigt pris, neutral smak och ingår i standardsortimentet. Helst vill du så klart bara köpa denna, men så jobbar inte en riktig äventyrare.

 \textbf{Krokodilöl}
 Guldmedalj i \textit{World beer cup} 1991. Dessutom Benny Bus \textsc{(s.~\pageref{a8289efd495ef49dbe0225de89f7f019})} favoritöl. Ett givet val.

 \textbf{Elefantöl}
 Farfars favorit. Finns bara i liten burk så glider snabbt ner. Föredömlig APK då det är en dansk \textsc{(se danmark s.~\pageref{5331d7fd27772396f412a5b6d19bad44})} produkt.

 \textbf{Björnebryg}
 Har tappat lite av den glans som omgärdade burken för ett par år sedan. Håller dock fortfarande vad den lovar och blir avslagen efter att du druckit ungefär en tredjedel.

 \textbf{Old speckled hen}
 Gammal späckad höna, ska det va nåt? Nja, ur smaksynpunkt är den ungefär lika rolig som att suga på en gammal disktrasa. Men har du börjat med de fyra ovanstående bör smaklökarna vara bortdomnade vid det här laget så det är ingen fara.

 \textbf{Red seal ale}
 Rostig smak och ganska dyr. Köpes endast om expeditionen tar plats på en löne-/bidragshelg.

 \textbf{Sleepy Bulldog}
 Lika mäktig som en Guinness. Den här rackaren bör du beta av tidigt. Utmärkt om du inte hunnit klämma en burgare innan festen.

 \textbf{Old ox \textsc{(s.~\pageref{954ddcc10ad941a7ee93e0584ee6a78b})}}
 Samma för- och nackdelar som bulldogen.

 \textbf{Girafföl}
 Mesigare kopia av elefanten. Köpes bara för att få en större blandning.

 \textbf{Älgöl}
 Något som man ska lägga i botten på kassen men är utmärkt avslutning med sin spritiga smak. Perfekt när man börjat tappa tempo.

 \textbf{Karhu}
 En klassicker från vår östra landshalva. Bra standardbärs. Har varit \quotetext{veckans öl} på Carmen i Stockholm i tio år.

 \textbf{Cobra}
 En öl man egentligen bara dricker när man äter indisk mat men saluförs som \quotetext{Internationell märke} hos gröna skylten.
 Finns även i beställningssortimentet som brittiska \quotetext{Kobra King} om man vill hålla sig inom imperiet.

 \textbf{Falcon bayersk}
 Så självklar att den lätt glöms bort. Du vet redan hur den här smakar.

 \textbf{Saxon}
 Trots sitt balla namn och feta as-gam på etiketten är detta en riktigt fånig öl. För allergiker \textsc{(se allergi s.~\pageref{23773a17729d8e7e24da798e97533aeb})} då den är glutenfri. Mellanöl från Finland.

}

\small{
\textbf{Åka vikingaskepp}
\label{59819f07de01e025b0ca8c53b6481ac6}
 Slang för att dricka Explorer.

}

\small{
\textbf{Åkarbrasa}
\label{1bd2e1b3d4eed7b633eb2598887066f8}
 kommer att vara det enda sättet för folk att hålla sig varma på sen  miljöpartisterna \textsc{(se miljöpartiet  s.~\pageref{3e11b29518eeea19128b64869699f363})} förbjudit oljepanna och kärnkraftverk. Det går ut på att man kramar sig själv typ, så det är lite som onani \textsc{(se självbögare  s.~\pageref{398dd2a3bea77cfe04df91ba1a8b8c65})}.

}

\small{
\textbf{Åke Cato}
\label{90d6cdf0963f56ee2c21e93158c6c3cc}
 Svensk nöjesprofil i klass med Lennart Hyland och Adde Malmberg \textsc{(s.~\pageref{1390facdddaee5ed00a964fbe93b30b9})}. Återuppstår varje år kring jul med sin monsterhit \textit{Vår julskinka har rymt}. Men Åke är egentligen mycket djupare än så. Så här skrev han till exempel på sin blogg (bara det ett bevis på att han ständigt är aktuell) den 27 november år 2012:

 \quotetext{\textlessi\textgreaterDet är oundvikligt att damer och herrar i min ålder då och då kommer att tänka på döden.
 
 Sådana tankar bör man omedelbart slå bort då de kan leda till svårartade depressioner.
 
 Mitt råd till mina jämnåriga, och alla andra med för den delen, är därför att sluta tänka på döden och i stället tänka på något annat, till exempel kallops, matematik eller lippizanerhästar.
 
 Ett ganska visset råd, men jag har inget annat.}\textless/i\textgreater

}

\small{
\textbf{Åke Ohlmarks}
\label{dc4829f902543aa5b8349fa82bafacb7}
 Många svenskar känner Åke Ohlmarks i första hand som översättare av J.R.R Tolkiens \textsc{(se J.R.R Tolkien s.~\pageref{3f0b7fcbd9fa7369ca314a46c280b67e})} \textit{Sagan om ringen}-triologi. Men Ohlmarks var så mycket mer än bara översättare. Han var grosshandlarson, granne, far, make, vän och kollega. Den som vill veta mer om den Åke Ohlmarks som hans nära och kära kände, Åke utanför kändisskapets ljuskägla, kan läsa någon eller samtliga av hans sex memoarer.

}

\small{
\textbf{Åkerdisco}
\label{6248d2a43b234b98de8b2beb2fe95ffc}
 Ett åkerdisco påminner om ett skogsrave \textsc{(s.~\pageref{2180f77028a02c8fd94f622505937a53})}, men skiljer sig ändå på vissa sätt från detta, bland annat genom att åkerdiskot äger rum på en åker någonstans i sydsverige \textsc{(s.~\pageref{b1999637949ed135b2ca03f3a38460cc})}. Populära musikartister på åkerdiscoteket är Cat Stevens, ABBA och The Temptations.

}

\small{
\textbf{Åkhoj}
\label{7de318af6f1449fa22bb3f532c053e87}
 En åkhoj är en motorcykel som du har köpt därför att den till skillnad från ditt betydligt collare \quotetext{hojprojekt} går att köra nu - inte först om cirka tio år, eller efter att ett oförklarligt mirakel inträffat. När du kör din åkhoj stannar småbarn nonchalant på övergångsstället framför dig. De ignorerar dina irriterade motorvarvningar och räddar en mask eller pekar skrattandes på fjärilar som flyger förbi. Du känner inte sällan att den där lilla utskrattade killen inom dig inte alls har fått upprättelse av att du tagit hojkort, som du så länge räknat med att han ska få.

}

\small{
\textbf{Ålands demitaliseringsdag}
\label{cd54ac436a79ace81dbeccf2a343ec99}
 är en åländsk högtid som firas varje år den 30:e mars med pompa, ståt och allmän helgdag. Mer specifikt handlar dagen för åläningarna om att hylla det fredsavtal som slöts efter Krimkriget 1856. Nu ligger ju inte Krimhalvön så jättenära Åland, men Frankrike passade faktiskt på att mula en fästning som ryssarna ställt upp där när man hade vägarna förbi. Nödvändigheten i att fira att man fått en gammal fornborg pajad kan man ju fundera över. Har det verkligen aldrig hänt något mer spektakulärt på Åland, liksom? Vore det inte roligare att fira typ att ålandskrisen \textsc{(s.~\pageref{967c6b3cd72e6de161ca9e911779795a})} fick en lyckligt slut? Men är man en litet självstyrande landskap får man ibland ta ut svängarna för att klara sig, och åläningarna svarar bara att \quotetext{-Haters always gonna hate} när man försöker reda ut detta närmare.

}

\small{
\textbf{Ålandskrisen}
\label{967c6b3cd72e6de161ca9e911779795a}
 När amerikanska spionplan upptäckte ryska raketbaser på Kuba blev det inledningen på det som kom att kallas Kubakrisen. Den svenska motsvarigheten är inte lika dramatisk eller långdragen, men ändå mycket, mycket allvarlig. Åland är som alla vet en ö mellan Sverige \textsc{(s.~\pageref{b1999637949ed135b2ca03f3a38460cc})} och Finland \textsc{(s.~\pageref{631d44eaa1254ff71a1e11ba021d1266})} som officiellt hör till Finland, fast ändå inte. Vad som däremot är säkert är att Åland är en demilitariserad zon \textsc{(se ålands demitaliseringsdag s.~\pageref{cd54ac436a79ace81dbeccf2a343ec99})}, vilket innebär att inget av länderna tillåts ha militära enheter på ön. Ålandskrisen kom ur att två svenska officierare söp och svinade på en finlandsfärja på en konferensresa till Åbo för att prata om hur man bäst krigar. Dessa två i tjänst och därför uniformerade kronans män klev av för att spy på Åland. Det här i sig är såklart inget problem, hade det inte varit för att de båda var uniformerade. Detta var alltså enligt rådande förordning en krigshandling och Sverige \textsc{(s.~\pageref{b1999637949ed135b2ca03f3a38460cc})} och Finland \textsc{(s.~\pageref{631d44eaa1254ff71a1e11ba021d1266})} låg åter i krig för första gången sen träffningen vid Ratan \textsc{(s.~\pageref{4c1c24984f855772452c7149d148d89d})}. Det blev dock inte så mycket till krig, utan de svenska officerarna torkade sig runt munnen och bad Tarja Halonen om ursäkt varpå freden åter rådde i Skandinavien.

}

\small{
\textbf{Ålder}
\label{d7a7467d6b0b94f50c209220eab58dd1}
 spelar ingen roll, om man inte är en ost.

}

\small{
\textbf{Ålidhem}
\label{de74ea3d4d0195bd5ef1467ffc549590}
 Bostadsområde i Umeå \textsc{(s.~\pageref{bd1e37dc477bb704c667ed1a4606df71})} som enligt säkra källor från Flashback befolkas av studenter och hårdrockare \textsc{(se hårdrock s.~\pageref{a4566a810e7ad85a57ddc84083a8139b})} med vänsteråsikter.

}

\small{
\textbf{Ånäset}
\label{60c60004be04d826c3dd64de3b90cd05}
 är en liten bygd vid västerbottens \textsc{(se västerbotten s.~\pageref{d4b008c5143dcffb6b8c35f3876c2a19})} kustland, mest känd för att världens största osthyvel \textsc{(s.~\pageref{c09f8306965e1344e1102a46d084cab9})} ståndaktigt hälsar alla som åker förbi på E4an välkomna (alternativt välkomna åter) till ostriket. Maud Olofsson \textsc{(s.~\pageref{eb913a2e9be929654908a05017401bd6})} ska vid ett besök ha sagt att osthyveln fyllde samma funktion för Ånäset som kolossen gjorde för Rhodos och att bygden aldrig skulle bli bortglömd.

 HEAD3:  Andra saker Ånäset gjort sig känt för:

 \begin{itemize}
 \item Den utmärkt välsorterade pappershandeln
 \item Skoaffären som alltid har rea
 \item Ett brutalt familjemord i slutet på 90-talet
 \item Att det är en av de få orter i Sverige \textsc{(s.~\pageref{b1999637949ed135b2ca03f3a38460cc})} där man fortfarande kan bli jagad av raggare för att man ser ut som en kommunist \textsc{(s.~\pageref{fd9bf7896d396992b29d542a0200b800})}.
 \end{itemize}

}

\small{
\textbf{Årets göteborgare}
\label{008d19244d66bfd437982ab8cfce8a40}
 är en hederstitel som varje år sedan 1993 föräras en person som särskilt utmärkt sig och gjort något berömvärt. Bland pristagarna finns exempelvis: Leif \quotetext{Loket} Olsson, Thomas Ravelli, Lotta Engberg och Thomas von Brömssen.

 För att erhålla utmärkelsen ska man bland annat uppfylla följande kriterier:
 \begin{itemize}
 \item Gjort något gott.
 \item Vara född i Göteborg - eller någon annanstans.
 \item Vara en god ambassadör för Göteborg.
 \end{itemize}

 Som ni märker på tidigare pristagare är detta i princip bara kriterier som finns där för syns skull. Det viktigaste verkar i själva verket vara att man supit med Glenn Strömberg.

}

\small{
\textbf{Åtta}
\label{6fa68b0d02ec525fa72a51c13e5e3ed1}
 är en seglarknop som förhindrar att skot löper genom öglor. Det kan verka konstigt men är sant. Till utseendet liknar åttan Konsums \textsc{(se konsumbutik s.~\pageref{70e4875f7c2c177596305006e46b7ca9})} gamla logotyp ställd på högkant.


 Se även: Etta \textsc{(s.~\pageref{ba48f6c4097b7fc25ca11f1e544842d7})}, Tvåa \textsc{(s.~\pageref{84fcc0494ecf9f5af79fcd9bed184a9a})}, Trea \textsc{(s.~\pageref{6f94fdf535ab2e21147ea40ea920ca75})}, Fyra \textsc{(s.~\pageref{7bdb5385ce8e0b1cbc7c15b1d71e8e7d})}, Femma \textsc{(s.~\pageref{d974e0811fe7a4d49a9062d33b66a88d})}, Sexa \textsc{(s.~\pageref{4b1fabe53857b0a2ace6ae22008fe13e})}, Sjua \textsc{(s.~\pageref{e7bf63fa6d0d29bd89c23f833b979a15})}, Nia \textsc{(s.~\pageref{04a481486dd84d7c8bfdfc89d38136a6})}.

}

\small{
\textbf{Åttioåtta}
\label{accedee29196c933bdfee548d120a3c0}
 är en svårknäckt kod för \quotetext{Heil Hitler} eller en helt okej glass i GBs sortiment.

}

\small{
\textbf{Åvlaje}
\label{a744a7fe4b5d5eab82c62f8e70fafba6}
 är ett ord som tillhör den rospiggska dialekten och är synonymt med det nu, tydligen, mer populära \quotetext{kefft \textsc{(se keff s.~\pageref{890a42bbf6c2e6888fb851dd76e1e980})}}. Allting flyter. Man kan inte gå ner i samma flod två gånger. Så tänker den till naturen överseende frötunabon, men vet samtidigt att det saknas något hos den som inte druckit häxa vid Lommarsjön eller öl hos Pajen på kajen \textsc{(s.~\pageref{813e80b092f456fc81da1b8e0e83a273})}, inte fått stryk av norra Stockholms piketsnut eller sett den uråldriga klippristningen i Sika.

}

\small{
\textbf{Émile Durkheim}
\label{2cc9fe9a27928b0cefdd368350d1b07f}
 Émile \quotetext{Durken} Durkheim (1858-1917) var en fransos som uppfann durkslaget \textsc{(se durkslag s.~\pageref{c48b23fc8215397e022152f51c8933aa})}. Han lanserade även en del sociologiska teorier, vilka ledde till att han postumt förärades titeln: \quotetext{Världens minst kända kända sociolog}.

}

\small{
\textbf{Ödla på pinne}
\label{9159305fc9033e2af7fee11a993874d9}
 Kinas nationalrätt. En vanlig kinesisk risbonde äter i regel den mindre lyxiga sortens ödla på pinne, kopparorm \textsc{(s.~\pageref{b8f4fa38453856ba979bc2898e116e5a})} på pinne, samtidigt som en välbärgad grisfarmare eller kalligrafimästare har råd med ödla med vingar på pinne. Båda klasserna njuter dock åtminstone en dag i veckan av ägg kokat i urin \textsc{(s.~\pageref{524fd7acb94f9c2d879b5c1cf8335669})} [http://www.dn.se/nyheter/varlden/agg-kokade-i-urin-ska-bli-exportsucce].
 HEAD2: Ödla på pinne i kulturen
 I Fallout-serien figurerar ödla på pinne flitigt.

}

\small{
\textbf{Ögonlock}
\label{4a83c9b0ac915b5affcb5ddcb17f2823}
 är lite som håriga \quotetext{lock} av skinn som liksom \quotetext{stänger in} ögat i dess \quotetext{burk} när man ska \quotetext{sova.}

}

\small{
\textbf{Öl på polskt vis}
\label{c8864490fdcab916c70ba518f58b0183}
 Nissepedias \textsc{(se Nissepedia s.~\pageref{62400dadecd90cb5cd39062abe5a3e4a})} utsände kan meddela att åtminstone i Warsawa, Polens imponerande och intressanta huvudstad, dricker många sin öl med sugrör. Polacker som dricker öl med sugrör:

 \begin{itemize}
 \item Män
 \item Kvinnor
 \end{itemize}


 Varför är i nuläget oklart. Kanske för att man arbetat sig alldeles trött i muskulaturen och vill vila sig när man äntligen får dricka en öl i glada vänners lag. Detta är dock bara spekulationer, vilket egentligen inte anstår ett uppslagsverk av nissepedias dignitet.

}

\small{
\textbf{Ölorgel}
\label{2ffdf8ad017e8b16182c55e11ee641db}
 är ett förgängligt engångsinstrument, och spelas helst trestämmigt. Som att blåsa ton i en flaska, men flera. Fördela sådär ett par dussin ölflaskor på tre musikaliska onykterister. Sedan får var och en stämma sina flaskor, genom att konsumera lämplig mängd av flaskinnehållet, i skalor i samma tonart. Vill man ha en basist kan nån få tömma lämpliga mängder groggvirke ur enochenhalvliters petflaskor. Sedan spelar man, nästan vad som helst. Gärna improviserat. Kombinationen av ivrigt stämmande (inte ovanligt att man får i sig en liter öl på en kvart, utan att man tänker på det) och ivrigt blåsande skapar oanande yrseleffekter, och, ibland, vacker musik.

 När man är less dricker man ur resten.

}

\small{
\textbf{Öra}
\label{c4774ec92abe06f5664e18f44446d7e7}
 är en kroppsdel som man lyssnar med. Det är som en tratt av kött som sitter på sidan av huvudet. Känn efter i höjd med ögonen ungefär så ska du se att du hittar rätt.

}

\small{
\textbf{Örnäset}
\label{4d853bf44a938945e30a00d6584ebf73}
 ligger för er som inte vet,i Luleå \textsc{(s.~\pageref{3cefb5ac35187749592f1ebb25472b99})}. Invånarna i detta trivsamma område gillar sprit, knark och sport på tv.
 Har man tur, så kan man en fredagskväll stöta på en generös gubbe alternativt kvinna med finskt förnamn och tatueringar på händerna utanför kvarten; som bjuder på en slant eller två.
 Örnäsets folkdräkt består av wctbyxor \textsc{(se wctbyxa s.~\pageref{dc78f9615e53d0ddb525d3975197a781})} och huvtröja och bäres av innevånarna till både vardags och fest
 Stora horder med glopar \textsc{(se glop s.~\pageref{aae22e6d62a99e31db1de383aa15e538})} finns vanligen utanför matvarubutiken Coop \textsc{(s.~\pageref{0b5cb0ec5f538ad96aec1269bec93c9c})} Forum under varierande tider på dygnet.
 Det bästa med Örnäset är Centrum där man förutom Coop finne två friseringar, en tatueringsstudio, ett suspekt gym, en blomsteraffär, två banker, ett turklivs, två pizzerior, en spelbutik, en dataaffär och två krogar. Innan Coop byggdes fanns även Systembolag, detta medför att man än idag kan träffa på en del professionella drickare som liksom inte släppt taget än.


 Utöver Centrum finns den majestätiska Edeforsgatan, pensionärsghettot Kronogårdsringen samt en del villaområden i utkanterna där det uppenbarligen bor folk men ingen vet vem de är.

 HEAD2:  Kuriosa
 Torbjörn Säfve står staty utanför spelbutiken.

 Kvarten är troligen den enda krogen som säljer 7,2\%:ig öl.

 Den ena frisersalongen var tidigare hundfrisering, oklart om det är samma innehavare.

 \quotetext{Colamannen} träffas varje dag i Örnäset centrum.

}

\small{
\textbf{Österrike}
\label{c7c58270ca7c339e744580f8a1bc04d2}
 Vad är egentligen dealen med \textbf{Österrike}? Det kommer så jävla många knäppgökar därifrån.


 HEAD3:  Knäppgökar
 \begin{itemize}
 \item Adolf Hitler
 \item Arnold Schwarzenegger
 \item Engelbert Dollfuß
 \item Falco
 \item Ferdinand Porsche
 \item Fritz Lang
 \item Georg Luger
 \item Gustav Klimt
 \item Gustav Mahler
 \item Hapsburgarna
 \item Hermann Maier
 \item Johann Strauss
 \item Josef fritzl \textsc{(s.~\pageref{3f89c7ae4575ada3faf793ddb812a509})}
 \item Jörg Haider \textsc{(s.~\pageref{f40f229da31c801dcc9f969165e7e31d})}
 \item Ludwig van Beethoven
 \item Niki Lauda
 \item Sigmund Freud
 \item Wolfgang Amadeus Mozart
 \item Wolfgang Priklopil
 \end{itemize}

}

\small{
\textbf{Övre magmunnen}
\label{b0fbb0780611129ae5fc27c88d23d8f3}
 Den övre magmunnen är en ringmuskel som sitter mellan matstrupen och magsäcken. Dess funktion är att förhindra att mat tränger tillbaka upp i matstrupen. Det händer dock ibland att den förblir öppen och orsakar då olika olägenheter, däribland fetor ex ore \textsc{(s.~\pageref{d3b96d618fb972d12fb0cdfdeaf13a98})}, vilket i sin tur ofta leder till skilsmässa.

 Källa: Prof. Etienne \textsc{(se Användare: Prof. Etienne s.~\pageref{a9878d2280e5a39becac8f73d113df91})} \textit{Den övre magmunnen i litteratur och mytologi}. Nordiska rådets förlag, Stockholm, 1987.

}

\small{
\textbf{☭}
\label{7da15a125400bc658992457c218d7d47}
 серп и молот (latinska bokstäver: \textit{serp i molot}) är den ryska motsvarigheten till vad vi i Sverige \textsc{(s.~\pageref{b1999637949ed135b2ca03f3a38460cc})} brukar översätta till ”hammaren och skäran”. Hammaren (молот) har i det här fallet samma betydelse som slägga, som Sovjets \textit{folkkomissarie för utrikes ärenden} Vjatjeslav Molotovs efternamn anspelar på. Samme man som fått ge namn åt protestverktyget Molotov cocktail \textsc{(s.~\pageref{7d215b4a5cc923f86bb4c229af6e2d61})}. Begreppet är menat som en symbol för unionen mellan arbetarklassen och bondeklassen \textsc{(se bönder s.~\pageref{30a6fc00c9102680b8196b1b79935ec4})}. Vissa människor tar väldigt illa vid sig när de ser symbolen, men det rör sig allt som oftast om bönder \textsc{(s.~\pageref{30a6fc00c9102680b8196b1b79935ec4})} eller värre så inte så mycket att bry sig i.

}


% Back matter contains indices and stuff
\backmatter

%\printindex[saker]
%\printindex[andrasaker]
%
%\printindex[television]
%\printindex[dialekter]
%\printindex[musik]
%\printindex[litteratur]
%\printindex[djurriket]
%\printindex[politik_och_debatt]
%\printindex[dialekter]
%\printindex[musik]
%\printindex[litteratur]
%\printindex[bubologi]
%\printindex[overlevnadsknep]
%\printindex[konst_och_kultur]
%
%\printindex



\end{document}