\documentclass[a5paper, twoside]{book}


%%
% Book metadata
\title{Nissepedia ett urval}
\author{Nissepediaklubben}
%\publisher{Videnskabens Forlag}


%%
% For nicely typeset tabular material
\usepackage{booktabs}

\usepackage[swedish]{babel}
\usepackage[T1]{fontenc}
\usepackage[utf8]{inputenc}

%\usepackage{csquotes}
%\MakeOuterQuote{"}


% Inserts a blank page
\newcommand{\blankpage}{\newpage\hbox{}\thispagestyle{empty}\newpage}

\usepackage{units}

% Typesets the font size, leading, and measure in the form of 10/12x26 pc.
\newcommand{\measure}[3]{#1/#2$\times$\unit[#3]{pc}}

\providecommand\phantomsection{}


% Generates the index
%\usepackage{makeidx}
%\usepackage{multind}
%\usepackage[columns=3,font=footnotesize]{idxlayout}

%\usepackage{imakeidx}
%\usepackage[splitindex]{imakeidx}[nonewpage]

\usepackage[nonewpage,splitindex,makeindex]{imakeidx}



%\makeindex[title=Det hele,columns=1]
%\makeindex[name=saker,title=Saker,columns=1]
%\makeindex[name=andrasaker,title=Andra Saker,columns=1]
%
%%AUTOGENERATED
%\makeindex[name=television, title=television, columns=1]
%\makeindex[name=dialekter, title=dialekter, columns=1]
%\makeindex[name=musik, title=musik, columns=1]
%\makeindex[name=litteratur, title=litteratur, columns=1]
%\makeindex[name=djurriket, title=djurriket, columns=1]
%\makeindex[name=politik_och_debatt, title=politik och debatt, columns=1]
%\makeindex[name=dialekter, title=dialekter, columns=1]
%\makeindex[name=musik, title=musik, columns=1]
%\makeindex[name=litteratur, title=litteratur, columns=1]
%\makeindex[name=bubologi, title=bubologi, columns=1]
%\makeindex[name=overlevnadsknep, title=overlevnadsknep, columns=1]
%\makeindex[name=konst_och_kultur, title=konst och kultur, columns=1]



%



\begin{document}
\small
\pagenumbering{arabic}
\raggedbottom

% Front matter
\frontmatter

% r.1 blank page
\blankpage


% r.3 full title page

% v.4 copyright page
\newpage

~\vfill
\thispagestyle{empty}
\setlength{\parindent}{0pt}
\setlength{\parskip}{\baselineskip}

\par\textsc{Utegiven av Videnskabens Forlag}

\par\textsc{www.nissepedia.org}

\par Här kan man skriva in licens eller typ sånna saker

\par\textit{Första utgåvan, Januari 2014}


% r.7 dedication
\cleardoublepage

Här kan man ha en dedikation

% r.9 introduction
\cleardoublepage
\chapter*{Introduktion}
\index{Introduktion}

Här kan man ha en introduktion
\index[saker]{Här markerar jag en sak}


\chapter*{Förord}
\index{Förord}
Här kommer det ett litet förord.


%%
% Start the main matter (normal chapters)
\mainmatter

\noindent
\pagestyle{myheadings}
\markboth{Nissepedia}{Ett urval}

\chapter*{Ett urval}
\newpage

\small{
\textbf{Norrländska}
\label{e9a8473de49e4580345e0db21ff5c1df}
 är en ihopklumpande beteckning på alla dialekter som talas i Sveriges nordligaste län. Beteckningen är tillskriven av en oförstående storswänsk \textsc{(s.~\pageref{716f41dcabef6599bcf08334a8a6ae27})} kolonialmakt som inte har något sinne för nyanser överhuvudtaget. Att exempelvis inte kunna höra skillnad på Älvsbymål och Pitmål är väl en grej, men att inte bemöda sig skilja på exempelvis Lulemål och Ångermanländska är ett tecken på mental retardation och hög intellektuell viskositet \textsc{(s.~\pageref{17328a3aa2e9e596e033ccebf7995cc1})}.
}

\small{
\textbf{Adidas-Klas}
\label{a8bb38544b2c4f56034ab8536edb58e6}
 , eller Klabbarparn som han egentligen heter är Karl-Haldos storebror. Han ser väldigt farlig ut på grund av att han har en Adidas-logotyp tatuerad på hakan som löper ner till bröset, där logotypen på ett okonventionellt sätt möter ett drakskelett. Ett annat yttre attribut är Adidas-Klas knogar som bär bokstäverna F-O-R-D, i vad som tycks vara en hyllning till det amerikanska bilfabrikatet Ford (Ford har även europeiska fabriker, sedan många år tillbaka). Gamla människor kan uppfatta Adidas-Klas som varande en aningens farlig, men det beror enkom på fördomar och generationsklyftor. Adidas-Klas är en jättefin människa och dagisfröken.

 Adidas-Klas har spelat trummor i band som Zombiefied och Sista Civilisationens Död (i sistnämnda ingick även Skägg-Peter och en jazzentusiast), en uppgift han klarade med bravur. Vidare har han sett storsjöodjuret och tavelsjöodjuret.

 Mer än såhär behöver man inte veta om Adidas-Klas
}

\small{
\textbf{35007}
\label{c7d7a54da0f35a8abde8961b6897abe1}
 (utläses uppochned, ``Loose'') var ett holländskt rymdrockband som bland annat gjorde två temaskivor: En om havet och en om matematik. De gjorde också en tia \textsc{(s.~\pageref{e7292d5ba58672ce7f6fc3c0b646ab63})} om månlandningen.
 ==Discografi (urval)==
\begin{itemize}
\item Sea of Tranquility 10" (2001)
\item Liquid (2002)
\item Phase V (2005)
\end{itemize}
}

\small{
\textbf{A-hootin' and a-hollerin'}
\label{1928c39ea0f58992a3e5f53d143a23ff}
 är ett annat uttryck för ``tjo och tjim''. Uttrycket används som i följande meningar.
\begin{itemize}
\item Det var väldigt kul på krogen i fredag \textsc{(s.~\pageref{80d41f1e0b14eacb229eea9618632e88})}s. Det var en väldig massa a-hootin' and a-hollerin'.
\item Gunborg \textsc{(s.~\pageref{9e29dc34382963ae7d76a742e98637a4})} berättade om sin semester på festen, men jag kunde inte höra vad hon sa på grund av allt a-hootin' and a-hollerin'.
\end{itemize}
}

\small{
\textbf{AC4}
\label{17380c1b3ba45cc445b955cc6e133b5d}
 Kristet AC/DC-coverband från Umeå.

 Category:Musik \textsc{(s.~\pageref{38cce583d2d3675d645425cb435aa2bb})}
}

\small{
\textbf{Alienation}
\label{7c5c79d2842cce56e7e1f1d288b86f52}
 Så här är det. Människor gillar inte att göra samma sak hela tiden. De vill skapa och använda en massa delar av hjärnan dagligen. Kapitalister ogillar detta för att det är ineffektivt. Människor måste tyvärr arbeta för att få lön och kunna leva, och då måste de använda den där speciella delen av hjärnan som kapitalisterna vill att de ska använda. Genom att människor arbetar upprätthåller de också det kapitalistiska systemet genom att förmera kapitalet, och i förlängningen reproducerar de sitt eget förtryck. Tråkigt, men vad ska man göra åt det? Att gå med i ett fackförbund är en bra början, att beväpna sig en bättre.


 Funny fact: Alienation är också en jävligt bra punklåt av det engelska bandet Crisis [http://www.youtube.com/watch?v=2vlRg9uo_QI]
}

\small{
\textbf{Alkoläskfylla}
\label{8234165b965f2b1378f10acd340dc126}
 Sittandes på en stol vid ett köksbord med chipsskål. Någon spelar en Gessle-låt på gitarr. Vem är hon med flätorna? Reda ut vem som kom först, Niclas Strömstedt eller Mikael Rickfors. Ingen vet. Resa sig för att gå ut på balkongen, till synes för att röka men egentligen för att prutta lite. \textit{Vafalls!} Dörrposten gungar.

 \textit{Alkoläskfylla}.
 {{Yrsel}}
}

\small{
\textbf{Aftonbladet}
\label{e9ebf180c01d806db2fefd7f53b7a146}
 är en svensk kvällstidning som bäst kan liknas med ett hav av skit i vars mitt en enda liten ö reser sig. På den ön är allas vår Åsa Linderborg kung. Hon är världens finaste och klokaste kung.
}

\small{
\textbf{realister}
\label{3ab43c7f3424fc9915776529066a2840}
 är svenska män som, ofta utan byxor \textsc{(se sans pants s.~\pageref{e690d08a3200d783d98b198f0354bc85})}, kommenterar nyheter i dags-, kvälls- så väl som lokaltidningar. Där uttrycker de sin oro inför mångkultur och genusdebatt, vänsterpolitik och bensinpriser. ``Realisten'' inser att progressiv debatt, en modern invandrings- och asylpolitik och andra fenomen som karaktäriserar eller borde karaktärisera samtidsklimatet är mycket skadliga och hotar att leda till samhällets kollaps. Med ``samhällets kollaps'' menar realisten skapandet av ett jämlikt samhälle och den desperata situation där han och andra män utan byxor inte längre utgör en speciallt priviligerad grupp inom väljarkåren och får lika mycket eller litet att säga till om som alla andra människor i det demokratiska Sverige \textsc{(s.~\pageref{b1999637949ed135b2ca03f3a38460cc})}. Realisten är därför en varm tillskyndare av etnisk rensning, anti-intellektuell häxjakt och högerextremism, som realisten tycker är realistiska politiska krafttag i kampen mot invandrare, kvinnor, homosexuella, vänsterintellektuella (sk. ``kulturmarxister''), journalister, damfotboll, muslimer, konstnärer och författare (dock inte Lars Wilks), miljörörelsen, fredsrörelsen och pride-paradens deltagare.
}

\small{
\textbf{Acne}
\label{450e166c06161deddfc97749332c61cb}
 , en fruktad viral sjukdom och ett populärt svenskt klädesmärke. Sjukdomen yttrar sig ofta med maculopapulösa utslag, comedoner och pustler centrerad till ansikte och bål hos juveniler endast för att vidare sprida sig via de sensoriska nerverna, bl.a. trigeminus ganglierna, in till thalamus och hypothalamus för att störa de basala funktioner såsom sexualitet (ofta resulterande i överdriven sexualitet såsom ``premature ejaculation'') och aggression. Sjukdomen kallas i folkspråk för ``tonårsfrossa''. För närvarande finns det ingen behandling för Acne utan bör behandlas med ``Restriction of the social behavior and contacts'', kyshetsbälten och i de värsta fallen bör patienterna sättas i karantän på obestämd tid.

 Klädesmärket har tagit namnet då utbudet riktar sig mot de sjuka tonåringarna som har förlorat sin vett och därmed också klädessmak.
}

\small{
\textbf{Alice Tegnér}
\label{66a2a0b3aa1a42e1e5ae2d20dd1bdca6}
 är ledare för den ljusskygga organisation som i folkmun har kommit att kallas Alice Tegnér-sällskapet. Det har länge spekulerats i organisationens utbredning. Envisa urbana legender om vilda inträdesriter samt infiltration av samhällsuppehållande institutioner förekommer, men mycket är nog bara snack. Enligt avhoppare ska Tegnér kommunicera med organisationens medlemmar genom infernalisk musik, som organisationen sedan studerar i bokform likväl som genom framförandet av sånger som förts ned i leden från ledare till medlem. Gällande detta musikaliska inslag finns vissa påtagliga bevis, där boken ``Nu ska vi sjunga,'' som kan studeras bland annat på Karolina Rediviva och Kungliga Biblioteket är det mest framträdande.
}

\small{
\textbf{I'm a punk, and I like sham. Cockney rejects, are the worlds greatest band.}
\label{b975eeec75f3f25ea7adbbb906ae390d}
 ''``I'm a punk, and I like sham. Cockney rejects, are the worlds greatest band'' ær den skønaste textraden i Austin-bandet Big Boys låt ''Fun, Fun, Fun`` från deras 12'' EP med samma namn, slæppt 1982. Olyckligtvis føljs den ovanstående skøna textraden av \textit{``But I like Joy Division, and Public Image too''}. Smolk i bægaren. Den førsta raden ær dessutom stulen från ovan næmnda Cockney Rejects.
}

\small{
\textbf{krypa ihop i soffan som en katt}
\label{4d1d36fa3c68844e34049d8b7db95af2}
 Ofta kryper smala tjejer upp i soffor som katter. Där värmer de sina händer på te \textsc{(s.~\pageref{569ef72642be0fadd711d6a468d68ee1})}koppar. De talar kanske franska och säger sig älska ost men ingen har sett dem äta någon.
}

\small{
\textbf{krypa upp i soffan som en brugd}
\label{ea7efaf2898e75a3e2cd2bb76fc18568}
 Att sitta i soffan, gapa och bara ösa in olika födoämnen i mun \textsc{(s.~\pageref{6585f290ce92c3de5ff339920330e26f})}nen.
}

\small{
\textbf{...men det får man väl inte säga i det här jävla landet!}
\label{ff3b0ddb9108e1ec17e77bd33a02354e}
 ...MDFMVISIDHJL! är en bisats som uppstod samtidigt som kommentarsfält på webbtidningar. Foliehattsbärande timbroentusiaster samlade sig i dessa fält så snart de fick reda  på deras uppkomst. \textit{``En plats där jag får spy ut hat och klaga på allt jag känner mig hotad av? utmärkt!''}, lär de ha tänkt, de arslena. I Internets mörkaste avkrok, Flashback, fann våghalsiga nätarkeologer ett manifest, författat av en anonym cyberfascist. Efter en noga granskning genomförd av etnologer och historiker vid Umeå Universitet har man sedemera fastslagit att nedanstående text är den första gången MDFMVISIDHJL! nämns i text.

 ===Manifestet===
 Lyss opp, brunblåa kamrater! Den genuskulturmarxistiska kryptojudiskaochsamtidigtislamistiska maffians grepp om Sverige \textsc{(s.~\pageref{b1999637949ed135b2ca03f3a38460cc})} hårdnar. Sedan Gudrun Schyman, Kurdo Baksi och Mona Sahlin könsomskurit tre pursvenska femåriga flickor utan bedövning (för källa, se info14) har Sverige låsts i en full multikulti \textsc{(s.~\pageref{25eea9148080d30d384ce1c1277ef126})}nelson. Sedan Peter Wolodarski, Robert Aschberg och Göran Greider virat på sig turbaner och sedan gjort gemensam sak i att kväva det fria ordet genom att skita det i munnen. Sedan Liv Strömqvist, Teppas och Annika Lantz , höga på khat, dragit fram en snara och hängt åsiktsfriheten från en radiomast. Sedan de händelserna inträffat finns det en mängd saker man inte får säga i det här jävla landet. Vill du diskret påtala att familjen Bonnier är judar och att det är lite intressant hur judar alltid äger media? Får inte!
 Vill du strö ordet neger kring dig likt verbalt strössel på allas tillvarokaka? Får inte!
 Vill du skryta för dina polare om hur du tafsade på en brud bara på skoj? Får inte!
 Vill du klaga på hur turkar alltid tafsar på brudar? Får inte!
 Vill du ifrågasätta hur många judar som egentligen dog i förintelsen? Får inte!
 Vill du skämta om att tjejer är dummare än killar men mena det litegrann bara? Får inte!
 Vill du i förtroende skriva på din blogg att det är äckligt med bögar? Får inte!
 Vill du i smyg viska att han där den svartmusikga på jobbet luktar lite främmande? Får inte!

 Efter att alla dessa saker blivit bannlysta av den genuskulturmarxistiska kryptojudiskaochsamtidigtislamistiska maffian, vad i helvete ska man då tala om i det här jävla landet?!
 Får du säga att en tjej är tjock? Vem vet i det här jävla landet?
 Får du säga att en snubbe är ful? Vem vet i det här jävla landet?
 Får du be om att få mjölken skickad till dig? Vem vet i det här jävla landet?

 Res er, ni förtryckta medelålders vita män med rätt bra inkomst! Res er, ni svettiga våldtäktsbenägna innebandykillar! Och säg ifrån, alla ni Kamprads \textsc{(se Ingvar Kamprad s.~\pageref{c5f2e9ee9a39f83c39079dbcf01d8809})} därute! Våga säga vad man inte får säga i det här jävla landet!
}

\small{
\textbf{80talet}
\label{fef6822ae867bc17fd5b4761ca145293}
 var den tid under sent 1900tal som många gillade allra bäst.
 Saker var STORA och komponenter fick kosta pengar.
 Elektronik, telefoner, bilar, högtalare, gräsklippare mm tillverkades för att hålla i många många år.

 Datorer på den här tiden var däremot inte så snabba som dagens, men det gjorde ingenting för på den här tiden hade man sällan bråttom.
 Uttryck som: fasen vilken tid det tar, var ännu inte uppfunna.

 Anledningen till att saker var stora var förmodligen att man verkligen skulle känna att man ägde nåt och att det märktes väl vad man ägde och inte, sedan bodde man i större hus och lägenheter på den här tiden så man hade mer plats till allt.

 Musiken som producerades på 80talet var förebilden till all dagens musik.

 104% av Trance och Dancemusik innehåller snuttar och samplingar från kända låtar från 80talet.

 Vhs-spelaren (även kallad video) var en populär mediaspelare i var mans vardagsrum.

 Framför allt så var bensinpriset lägre under 80talet, det var därför man inte hade så bråttom att komma snabbare in på 2000talet då mycket blev sämre.

 Alla kvinnor och män såg bättre ut på 80talet, visserligen har mycket sexigare kläder kommit på senare tid men detta för att kompensera den allt fulare utvecklingen på det kvinnliga och manliga könet.

 ===Summering===

 Har du sett 80talet så har du sett allt som är värt att se.

 Men har du sett 70talet \textsc{(s.~\pageref{5f334093f378400a9b73d7c365b899a5})} så behöver du inte se 80talet.
}

\small{
\textbf{Adak}
\label{2fc7bfc5c68257bcd1717bf2898fab16}
 är en ort i Malå \textsc{(s.~\pageref{41da4620e87888eaaeafcb3004a8d177})} Kommun där det bor ungefär 200 personer. En gång i tiden fanns den tillsynes outömliga Adakgruvan med tillhörande by inte så långt från orten. Ett militärt stenkast därifrån låg den lika majestätiska gruvan Rudtjebäcken, även den med tillhörande bebyggelse. På sjuttiotalet fylldes bägge gruvorna igen och bägge byarna förvandlades till spökstäder, då Boliden AB bestämde sig för att fokusera sin produktion i just Boliden. Idag finns där bara tall, gran och en minnessten som berättar om hur kapitalismen och staten svek glesbygden.

 Idag kretsar produktionen kring kultur och institutionen Sagabiografen som varje sommar har filmfestival. På vintern kan man åka på drive-in bio medelst skoter \textsc{(s.~\pageref{b1120baa83f380cd42a805a4e823cb1b})}.
}

\small{
\textbf{A-traktor}
\label{68eb4e0240edaaca3face5a1ee84e9ac}
 Innan man skaffar dieselbil med lastgaller \textsc{(s.~\pageref{73b1f975c67393304ff101482965163c})} kan man skaffa en a-traktor.
 Alternativt skaffar man senare i livet en A-traktorreggad Scania 112 för snöröjningens skull.
 Det är inte bara jävligt coolt med a-traktor, man får numera köra den redan som 15-åring och man kan då skjutsa sina kompisar till grannbyn när det vankas åkerdisco \textsc{(s.~\pageref{6248d2a43b234b98de8b2beb2fe95ffc})}. Skatten är även förmånlig.
 Man skulle lätt kunna tro att ett så här briljant fordon vore vanligast bland mer utvecklade folkslag, men statistiken visar att Västergötland är det A-traktortätaste länet med hela 1950 stycken inregistrerade.
 Det finns några eviga frågor \textsc{(s.~\pageref{3bd505f805d94787ec0cc431648a7826})} som gäller A-traktorer:
\begin{itemize}
\item Dubbla växellådor eller AGA-spärr?
\item Flak eller lucka?
\item Porrljus eller inte?
\item Lägga pengar på stereon eller Jokkmokkstrim?
\end{itemize}

 Den som finner svar på dessa frågor och är utrustad med svets och ett tålamod i klass med Dalai Llama kan framåt vårkanten rulla ut på vägarna.
}

\small{
\textbf{Män är djur, tycker inte du?}
\label{7b3f13fdbf56a65af8ed05e40a8259bc}
 ``Män är djur, tycker inte du?'' är ett citat sagt i en dokumentär om kvinnojouren i Umeå och används ofta av anti-feminister (eller jämställdister som de själva vansinnigt nog kallar sig) för att påpeka att alla som menar att män skulle hålla på med tvivelaktigheter (eller rättare sagt, att den rådande maskulinitetskulturen i samhället innehåller problematiska element) är helt skogstokiga. För övrigt belyser reaktionerna på citatet mer än något att nittiotalet är över och med det ironins död.
}

\small{
\textbf{Alfons Åberg}
\label{3c49eba29ed964486f6392305ad63694}
 Storköpenhamns värsta hallick. Känd för att styra med järnhand
}

\small{
\textbf{ASEA}
\label{61585ca988d62029b6ee6adab6066c34}
 är ett svenskt företag med över hundra år på nacken som är verksamt inom industrisektorn. Från början riktade företaget in sig på att enbart erbjuda sina tjänster till fascistiska mönsterstater. Den besatta längtan efter mer profit gjorde dock att företaget började tumma på denna regel och man har idag verksamhet över hela världen. Den nya målsättningen är att skapa maskiner som stjäl jobben av så många arbetare som möjligt på hänsynslösast möjliga sätt. Den största framgången hittills uppnåddes med Percy Barnevik vid rodret när han lade ner den sista masugnen i Norberg och gjorde typ en fjärdedel av ortens befolkning arbetslösa. För detta belönades han med Svenska Arbetsgivareföreningens (nuvarande Svenskt Näringsliv) pris \textit{Guldsvartfoten}.

 Företaget har numera gått ihop med ett Schweiziskt institut \textsc{(se Institut och tankesmedjor s.~\pageref{c276b5997d5af80504f79b30d121cf62})} och bytt namn till ABB, som en hyllning till Anders Behring Breivik.
 ASEA har även gett upphov till ASEA-grönt \textsc{(s.~\pageref{e5ce0e93ee9c54094b4ec1c2027272ca})}.
}

\small{
\textbf{Adjektiv}
\label{67d02147cd8595eaf13c1a90aba99dcc}
 är inte okej i det postmoderna samhället.
 Allt är normalt \textsc{(s.~\pageref{5c455ca1c87070883ff0a4c13ae8937f})}. Således finns inga orättvisor, inga fattiga, inga tjocka, inga svarta, inget dyrt. Alla är individer \textsc{(se individ s.~\pageref{41beed76a0af9b4f550f7ebdecd3e700})} och allt är bara, bara bra. Kaffet är inte kallt, bussen är inte försenad, klockan går inte för sakta. Allt är bara, bara bra.
 Adjektiv är väldigt kränkande \textsc{(s.~\pageref{5311bb8220aa4c45c14a860bfaa3b0db})}.
}

\small{
\textbf{2038}
\label{2557911c1bf75c2b643afb4ecbfc8ec2}
 ==Milleniumbuggen==
 I slutet på 1900-talet började det tisslas och tasslas kring ett tekniskt problem som troligtvis skulle uppstå kring all digital teknik vi hade introducerat i våra liv. Det var om teknikprylen hade någon slags inbyggd tidsfunktion som problemet tänktes kunna uppstå. För det var när klockan slog över från 1999/12/31 - 23:59:59 till 2000/01/01 - 00:00:00 som teknik världen över skulle börja bete sig konstigt. Kanske skulle den digitala tidtavlan vid busstationen helt plötsligt visa 1900/01/01 - 00:00:00, märkliga tecken eller möjligtvis slockna helt? Vad kunde hända med känsligare utrustning? Innehåller pacemakers möjligen nån klocka som kanske slutar fungera och dräper stackars Signhild i nåt som skulle kunna likna hjärtattack orsakad av milleniumskiftets nyårsfyverkerier?

 När denna rädsla för den felande tekniken tog sitt starkaste uttryck talade man om att det USA finns en centraldator som hanterar alla missilsstationer världen över som kanske, kanske skulle råka ut för denna tekniska miss och förorsaka jordens undergång!
 Idén kring denna centraldator känns helt befängd - den har troligtvis hämtats från filmen War Games \textsc{(s.~\pageref{0b505d579ff1c7e9c1b25ccc867cfb62})}, men man vet aldrig - det finns siffror som pekar på att det spenderades 300 miljarder dollar på att förbereda sig för och reparera efter milleniumbuggen.

 ==2038-buggen==
 ''Den 19e januari 2038 klockan 03:14:07 kommer en liknande datorbugg att uppstå, men den här gången är det på riktigt.''

 I våra framtida ``smarta hem'' kan det hända att den av en centraldator styrda bakgrundsmusiken spelas i Piff och Puff-tempo, att våra Hooverboards tar oss på en lodrätt färd och sedan släpper av oss när vi är ett par meter ovanför marken. Och framförallt, alla missiler kommer att avfyras mellanm kontinenterna och SVT kommer att avbryta sina nattliga sändningar av syskonkanalens SVT24s tysta textbaserade nyhetsändningar för att visa klimaxscenen från filmen Hardware \textsc{(s.~\pageref{3ca14c518d1bf901acc339e7c9cd6d7f})}, bara som av en slump, medans bomberna börjar slå ner världen världen över. Den tekniska förklaringen för varför det här kommer uppstå följer nedan

 Det är om ungefär det här scenariot som Discharge \textsc{(s.~\pageref{7084c38f1708430f138336428e4ac7cb})} försöker förmedla i de flesta av sina låtar.

 ====Den tekniska biten====
 I väldigt många datorsystem lagras datum och tid som antalet sekunder som har passerat sedan Torsdag den 1 Januari 1970. Dagens datum (då artieln skrivs) 2012/01/06 12:24:00 representeras som 1325852640 sekunder sedan 1970/01/01. Problemet uppstår när man lagrar dom här sekunderna i datorn. Dom lagras som en integer på 32 bitar (ettor eller nollor). Det högsta talet som en Integer på 32 bitar, en serie på 32 ettor och nollor (t.ex. klassikern 00111011101010110010111110101111) kan anta är 2147483647 och det motsvarar den tidigare nämda tidpunkten 2038/01/19 03:14:07. Efter 2147483647 slår talet om till -2147483648, det vill säga 1901/12/13 20:45:52 och det är \textbf{mycket allvarligt!}
}

\small{
\textbf{Män är djur, tycker inte du det?}
\label{a064fbffd5eda4ac5320b7c565c6f549}
 ``Män är djur, tycker inte du?'' \textsc{(s.~\pageref{fb5c3cada516c1d08120bb25a3b77c53})}
}

\small{
\textbf{Abu Garcia}
\label{ebb8e709f4430ad487471fd1acdf28e2}
 är ett företag som sysslar med försäljning av fiskeartiklar. Tidigare var det två företag, Abu och Garcia. Abu var svenskt och är en förkortning för Aktiebolaget Urmakarna. I begynnelsen pysslade man enbart med urmakeri, men arbetarnas gedigna kunskaper i finmekanik visade sig på 1930-talet även passa utmärkt till att knyta fast fiskelina på en bit klarlackad bambu. Garcia var amerikanskt och är inte förkortning för någonting. Varje fiskeredskapsaffär \textsc{(s.~\pageref{1b1aa77debacc344b3c1342e51abfb55})} av rang erbjuder produkter från Abu Garcia.
 ==andra användningar==
 Abu Garcia! används ofta i Västerbotten \textsc{(s.~\pageref{d4b008c5143dcffb6b8c35f3876c2a19})} som \textit{iAy caramba!} används i Mexiko, alltså som en interjektion.
}

\small{
\textbf{Adde Malmberg}
\label{1390facdddaee5ed00a964fbe93b30b9}
 är Sveriges rolighetsminister. Tillsammans med ``Babben'' Larsson kan han få vem som helst att skratta på sig.
}

\small{
\textbf{Allergi}
\label{23773a17729d8e7e24da798e97533aeb}
 Identitetsmarkör hos stadsbarn. Tidigare kallat högfärdshosta.
}

\small{
\textbf{ADHD-fläta}
\label{20707f66b5c8077e1008cd4698c46322}
 En ADHD-fläta eller råttsvans är en tofs eller fläta som föräldrar låter växa ut från deras låg- och mellanstadie-barns runda små huvudens baksida. Flätan, eller tofsen, finns där för att signalera att barnet är ett riktigt jobbigt barn. En lös teori är att barnet får flätan bortklippt när det slutar vara jobbigt.
}

\small{
\textbf{AC/DC-gitarr}
\label{c688c3a81724e01058f2d15116b26aa9}
 En AC/DC-gitarr är en gitarr med vinröd kropp (ibland svart) och svart huvud som är extremt rå. Till formen påminner kroppen lite om eldflammor, vilket bara det är extremt rått. Dessutom är det ingen vanlig gitarr utan en elgitarr; också rått som fan. Och så figurerar den som mordvapen på omslaget till AC/DCs liveskiva \textsc{(se tolva s.~\pageref{75e2490604087d3d303b09a98803a16b})} \textit{If You Want Blood You've Got It}, så case closed. AC/DC-gitarren saluförs av den amerikanska tillverkaren Gibson, som även bygger oråa instrument såsom mandoliner och basgitarrer. Vill man höra äkta AC/DC-gitarr kan man med fördel lyssna på \textit{Manglar som ägg \textsc{(s.~\pageref{7b1e91fdfd952485ddd3bc6ef4e40b3c})}} eller \textit{Bonfire \textsc{(s.~\pageref{b0759e17c7cc70d7522a6b63a05c914e})}}.

 ==Personer som spelar på AC/DC-gitarrer==

 Angus Young
 Mob 47-Åke \textsc{(s.~\pageref{486ee67ac39debabed3d92a7555dcebd})}
 Tommy Iommi
 Pete Townshend
 Nisse Hellberg
}

\small{
\textbf{13}
\label{c51ce410c124a10e0db5e4b97fc2af39}
 är benämningen på den fasta nyckel \textsc{(se fasta nycklar s.~\pageref{ad577d76d7747bfd314d442197fc8587})} som används för att justera muttrar på alla maskiner med lite självrespekt. Kräver din maskin mindre nycklar än så har du med all säkerhet blivit lurad av en storfräsare \textsc{(s.~\pageref{4db17005692cd83e3e946a1311b81ed0})} som skrattar gott när hen tömmer sin hästhandlarplånbok \textsc{(s.~\pageref{2f8fbda5296f2f6cab04d88082ed9015})} framför en drägglande bankir. Ställ in skräpet \textsc{(se trivselskrot s.~\pageref{6235563333e8dc26c9fc54e9e70c85ed})} bland dom andra trasiga plastleksakerna i garaget och skaffa något rejält istället!

 ==Trivia==
 Troligheten att just denna fasta nyckel \textsc{(se fasta nycklar s.~\pageref{ad577d76d7747bfd314d442197fc8587})} är nednött till obrukbarhet är 100% högre än för de andra nycklarna i ditt set.
}

\small{
\textbf{Abdera}
\label{af6265e74ffc27627369e11195c4a675}
 är en ort i Grekland som i dagsläget har ungefär 4000 invånare. Redan under antiken \textsc{(se de gamla grekerna s.~\pageref{4a5fb3d6ce79b5ff43b33f8f7d843672})} beskrevs luften där som så dålig att man blev dum av att inandas den. Detta gav senare upphov till adjektivet \textit{abderitisk}, som är den etymologiska källan till det idag flitigt använda ”idiot”. Man förstår att det inte bor så många kvar där.
 (Den som vill investera i stenbaserad arkitektur med havsläge kontaktar Prof. Etienne's Aeolian Nights \textsc{(se Användare: prof. Etienne s.~\pageref{a9878d2280e5a39becac8f73d113df91})}.)
}

\small{
\textbf{50 million piece of shit}
\label{20dea3d3f625b865ae7fd554c02c6936}
 Brittiska tidningars smeknamn på fotbollsspelaren Fernando Torres efter att denne sålts till Chelsea FC och bara gjorde två betydelselösa mål under första säsongen.
}

\small{
\textbf{ASEA-grönt}
\label{e5ce0e93ee9c54094b4ec1c2027272ca}
 Har du någon gång varit inne på kontoret hos en slips \textsc{(s.~\pageref{61c7bd51d579af09c10142f4b55c848c})} har du säkert noterat att stolstyget, skrivbordsunderlägget, kassaskåpet och skrivmaskinen går i samma pacifiserande nyans. Denna stavas NCS 6020 G10Y.

 Se även: färgskala \textsc{(se Färgskala#ASEA-grönt s.~\pageref{bfd16917b4c5964997f496f424382446})}.
}

\small{
\textbf{korvbröd}
\label{6898888a74f0d42574012debf1a6d8f3}
 Korvbröd
}

\small{
\textbf{Alg-Börje}
\label{623e6649678262697f465fe2cdb41679}
 är en karl (förmodligen) från Valbo utanför Gävle som klarar uppehället genom att kränga hygienprodukter baserade på alger.
 Algerna i Alg-Börjes produkter kommer från ishavet. Alg-Börje är till och med generalagent för isländska havsalger, vilket låter ganska mäktigt.

 Exempel på produkter är:

\begin{itemize}
\item Alg-Börjes Algschampoo
\end{itemize}

\begin{itemize}
\item Alg-Börjes Algkroppstvål
\end{itemize}

\begin{itemize}
\item Alg-Börjes Liniment
\end{itemize}

\begin{itemize}
\item Alg-Börjes Tångbad
\end{itemize}

\begin{itemize}
\item Alg-Börjes Algtabletter
\end{itemize}

\begin{itemize}
\item Alg-Börjes Algmjöl (finmald)
\end{itemize}

\begin{itemize}
\item Alg-Börjes Algmjöl (grovmald)
\end{itemize}

\begin{itemize}
\item Alg-Börjes Corallkalk
\end{itemize}

 Källa: [http://www.Alg-Borje.se]
}

\small{
\textbf{Alg-Gutten}
\label{a0ec949e57eca216fde1bbd5dff07c19}
 är en man (förmodligen) från Finspång som säljer hundmat gjord på havsalger. Algbaserad hundmat sägs ge både blankare päls och ökad pigmentering. Hundmaten har flera slående likheter med Alg-Börje \textsc{(s.~\pageref{623e6649678262697f465fe2cdb41679})}s produkter för \textsc{(s.~\pageref{5a98c81c7b5b60a5777a92b943f53a41})} människor.

 Källa: [http://www.alggutten.se]
}

\small{
\textbf{Alkisschäfer}
\label{347febbc28041eae88556d2e7ced587b}
 En alkisschäfer är en hund som inte är riktigt  herrelös men heller inte har någon tydlig ägare. Vanligtvis saknar den någon av kroppens lemmar, typ ett ben eller ett öga. De flesta har lätt att vistas bland människor men vissa kan va lite folkilskna av sig och då måste man riva i. Alkisschäfern är vanligtvis kopplad, men det är inte alltid säkert att det är någon som håller i kopplet. Den är rätt skitig men glad. Typiska namn på en alkisschäfer är  Pudas \textsc{(se pudaslåda  s.~\pageref{8f0a5ec52c3822627610afa582ce2ef7})}, Kari, Sam och Torax. Som vanligt är det inte hunden det är fel på utan ägaren. Namnet till trots kan en hund från nästan vilken ras som helt bli en alkisschäfer.
}

\small{
\textbf{Albin}
\label{899dffe3a176a36256e0a00fb97698d3}
 är ett namn som nyblivna föräldrar ger till sitt barn med förhoppningen att det ska göra att barnet automatiskt blir lillgammal \textsc{(s.~\pageref{d30762f704f0731fb3b08cd0846128d4})}t. Albin betyder inte just något särskilt.
}

\small{
\textbf{krypa upp i soffan som en uv}
\label{ae8f7b4e20767fca9e50538495ec4954}
 Folk med storhetsvansinne \textsc{(s.~\pageref{2f9c0ea6231e1de87c97eab41410c795})} kryper ofta upp i soffor som uv \textsc{(s.~\pageref{45210da832f9626829457a65e9e7c4d0})}ar. Där läser de tjocka bruna böcker och hoar för sig själva av förnöjdhet. De talar kanske tyska och äter mörkt, redigt bröd som de klappat på en ostkiva på.
}

\small{
\textbf{Akademiker}
\label{05c87f09f25afaf9f129a84230ab1008}
 Att vara akademiker är att förstå det högtravande och pretentiösa språk som används på universitet och högre lärosäten. Det handlar om att till exempel säga ``esoterisk'' istället för ``intern'' eller ``kontext'' istället för ``sammanhang''.
}

\small{
\textbf{A-wop-bop-a-loo-bop-a-lop-bam-boom}
\label{fe31d426b36075592465b90605764340}
 är den fetaste raden i Little Richards låt Tutti frutti. Näst efter den kommer ``tutti frutti, oh Rudy''.
}

\small{
\textbf{Alla x i y}
\label{18d4689248d1c32d716dab95e7e57b17}
 är ett sorts versmått som används för att förolämpa någon genom att, mer eller mindre subtilt, påpeka att denne är dum i huvud \textsc{(s.~\pageref{e906cd95a540df9b16d0460fb4cf0adc})}et.
 Vanligtvis används meningen så här: ``Johanssons grabb har ju inte alla x i y''.

 Här följer en lista på förolämpningar i ``x i y''-form.

\begin{itemize}
\item Alla ägg \textsc{(s.~\pageref{128a5feb8e12d0aa622e0298a8332980})} i korgen
\item Alla får i fållan
\item Alla läsk i backen
\item Alla häst \textsc{(s.~\pageref{b4c608370b339da095c5f8db7fab0945})}ar i hagen
\item Alla indianer i kanoten
\item Alla nitar i västen
\item Alla sidor i boken
\item Alla finnar \textsc{(se finland s.~\pageref{631d44eaa1254ff71a1e11ba021d1266})} i bastun
\item Allt var i finnen
\item Alla Oi! i Oi!kören \textsc{(se skinhead s.~\pageref{a54bc1b5d472b5afed8e84004b6441c4})}
\item Alla kulor i påsen
\item Alla fän i hönsgård \textsc{(s.~\pageref{3f284439cd46e0b187d34410aa79b2fb})}en
\item Alla pennor i skrinet
\item Alla fjädrar i Chapeau de paysan \textsc{(s.~\pageref{27aa75146d9ab723d1423168a2539d5d})}en
\item Alla knappar i västen
\item Alla uv \textsc{(s.~\pageref{45210da832f9626829457a65e9e7c4d0})}ar i uvslaget
\item Alla skrovmål i calzonen \textsc{(se calskrove s.~\pageref{84ff54e779ee49fdad21e17c20f14453})}
\item Alla färger i regnbågen
\item Alla chips i skålen
\item Alla skivor \textsc{(se sjua s.~\pageref{e7bf63fa6d0d29bd89c23f833b979a15})} i fodralen
\item Alla brugd \textsc{(s.~\pageref{d6b6b68506b8f1daad3a2ddbfaf8d863})}ar i Skagerrak
\item Alla hästar i kanoten
\end{itemize}
}

\small{
\textbf{Aktersegla}
\label{fd1077f333993f7f88b3b43533db9b98}
 Lämna någon eller något bakom sig. Ett klassiskt exempel är när Dia Psalma akterseglade Birdnest records och började ge ut sina skivor på egen hand.
}

\small{
\textbf{Albert II}
\label{8a80daf328e56d2b30df9fb6c782146d}
 , född 19 april 1942 i Florida, USA, död 14 juni 1949 i rymden, var den första apan i rymden. Albert II föddes i ett av NASA:s laboratorium där han spenderade större delen av sin levnad med att klättra i träd och delta i experiment. Han döptes efter apan Albert som var den första apan att nästan besöka rymden men som olyckligtvis dog innan han hann fram. Albert II överlevde färden upp i rymden men klarade inte tillbakaresan. Inga officiella statyer eller frimärken har någonsin skapats till Albert II:s minne.
}

\small{
\textbf{"göra-allt-för-att-slippa-lumpen"-historier}
\label{86f09c4b94bc85d3c0ffc70ae67b8361}
 Dessa historier handlar om män som verkligen, verkligen, verkligen inte velat göra lumpen. Historien utspelar sig på mönstringen och kan låta så här:

 ``Min kusin \textsc{(s.~\pageref{f7f20d5744925e2e72e5524035a162be})}s polare rullade in sig i en matta och sa att han var en varmkorv för att få frisedel \textsc{(s.~\pageref{ec0e47809187866739cd1a19e3d1ed37})}.''

 Det finns också historier med en mer sedelärande tvist, t.ex. denna.

 ``Min kusin \textsc{(s.~\pageref{f7f20d5744925e2e72e5524035a162be})}s polare sa att han var från en annan planet på mönstringen, så han förlorade körkortet.''
}

\small{
\textbf{Aleksandr Karelin}
\label{7db555630a4ad78feb3477db9b1ee464}
 Aleksandr Aleksandrovitj Karelin, även kallad ``det ryska experimentet'' och ``lyftkranen från Sibirien'', född 19 september 1967 i Novosibirsk, Sovjet, är ingen kille man bråkar med. Till yrket brottare.


 ==Föda och beteenden==
 För några år sedan visade Sportspegeln ett inslag från Karelins hem i Ryssland i vilket det tydligt framgick vilken fin människa Karelin är. Där visades också hur Karelin stod i en snödriva och grillade korv på en klotgrill medan han drack en klunk vodka då och då för att hålla värmen, en typisk ägmästarsituation \textsc{(se ägmästare s.~\pageref{8324518500d7e7ccd22ae364887d4476})}.
}

\small{
\textbf{Ales Stenar}
\label{2b28507979e217cfe15c9d6455eabd18}
 är en fornlämning från Sverige \textsc{(s.~\pageref{b1999637949ed135b2ca03f3a38460cc})}s första lärosäte, långt innan Uppsala Universitet och folkskolereformen. Här fick man lära sig hur man gjorde sitt eget tråg \textsc{(s.~\pageref{1e0e0470206e0f2baad8e628ba8f770c})}, hur man gjorde för att plundra, varför det är viktigt att slänga gamlingar utför ättestupan och andra viktiga saker för att klara sig förr i tiden. Själva stenarna tros ha använts till stenkrig på rasterna, en lek som ansågs mycket pedagogisk enligt dåtidens läroplan.
}

\small{
\textbf{Alkohol}
\label{11c589cba1a208e0359048a78e6b88b8}
 är ett vanligt lösningsmedel. Det löser problem.

 Utan alkohol kan man inte ha roligt, och har man roligt utan alkohol vore det roligare med alkohol. Inte minst enligt Boris Jeltsin.
}

\small{
\textbf{Life just be that way, I guess.}
\label{11f000e9e3f96eae83688170fc343ec7}
 är ett citat från TV-serien The Wire, säsong 1 avsnitt 1. Det användes då som ett svar på påpekandet att en person fick öknamnet ``Snot'' för att han glömt sin jacka och börjat snora.

 Citatet, taget ur sitt sammanhang i varierande grad, går att tillämpa på det mesta som vi människor stöter på här i livet. T.ex. om man handlat specerier och kassen går sönder utanför affären och alla ens konserver och medelst patentkork \textsc{(s.~\pageref{1e39785f5bab52f931dac485727645b6})} förslutna kärl sprids över parkeringen kan man, istället för att svära eller nåt annat teatralt, muttra ``Life just be that way, I guess.'' och gå vidare med sitt liv.
}

\small{
\textbf{Aforismer}
\label{42ffbda16f05ff4f9e22b93031d09223}
 Många äro de munläderförsedda som ägnat tid åt att författa aforismer - Oscar Wilde \textsc{(s.~\pageref{379811323e22c1397246d12721dc9fc8})}, Nietzsche, Mark Twain och Platon är endast en handfull av dem. En aforism är en slagkraftig sentens som fångar upp och förmedlar en sanning om människan, varat eller något annat tidlöst och centralt.
 ==Exempel på aforismer==
\begin{itemize}
\item Av sina fiender lär sig den vise mycket. - \textbf{Aristoteles}
\item Nöden är geniets drivfjäder. - \textbf{Honoré de Balzac }
\item Moralister är människor som kliar sig själva där det kliar på andra. - \textbf{Samuel Beckett}
\item Mitt i den mörkaste vintern, lärde jag mig äntligen, att det inom mig, finns en oförgänglig sommar. \textbf{Albert Camus}
\item Det är som min far brukade säga; ``Vi har kanske inte så mycket att leva av, men om pojken vill ha ett smörpapper så ska han fan i mig ha ett smörpapper!'' - \textbf{Mark Frygell}
\item Den främsta dygden är att lägga band på sig och hålla tungan rätt i mun \textsc{(se näbbmun  s.~\pageref{cd958ea18e337c767df35dd9c08ec4be})}. - \textbf{Geoffrey Chaucer}
\item De flesta människor lever i ruinerna av sina vanor. - \textbf{Jean Cocteau}
\item Hä ä bar å åk! - \textbf{Ingemar Stenmark \textsc{(s.~\pageref{989f4dba6b1fb2e5920e2c251fd693a2})}}
\end{itemize}
}


% Back matter contains indices and stuff
\backmatter

%\printindex[saker]
%\printindex[andrasaker]
%
%\printindex[television]
%\printindex[dialekter]
%\printindex[musik]
%\printindex[litteratur]
%\printindex[djurriket]
%\printindex[politik_och_debatt]
%\printindex[dialekter]
%\printindex[musik]
%\printindex[litteratur]
%\printindex[bubologi]
%\printindex[overlevnadsknep]
%\printindex[konst_och_kultur]
%
%\printindex



\end{document}